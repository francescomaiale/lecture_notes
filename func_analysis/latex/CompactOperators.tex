\chapter{Compact Operators on Banach Spaces} \thispagestyle{empty}

In this chapter, we are mainly concerned with \textit{compact operators} and \textit{symmetric (compact) operators} between Banach spaces.

In the first half, we study the ideal of compact operators and the spectral theory (e.g., the Fredholm alternative theorem), which gives us an excellent estimate of the spectral radius.

In the second half, we investigate the fundamental properties of compact symmetric operators, for which the spectral theory gives us more precise pieces of information.

\section{Main Definitions and Elementary Properties}

\begin{definition}[Compact Operator]\index{compact operator} Let $X$ and $Y$ be two Banach spaces. An operator $T: X \longrightarrow Y$ is \textit{compact} if and only if the following properties hold: \mbox{}
\begin{enumerate}[label=\textbf{(\alph*)}]
\item The operator $T$ is continuous with respect to the strong topologies.
\item The image $T(E)$ of every bounded subset $E \subseteq X$ is relatively compact\footnote{Let $\mathfrak{X}$ be a topological space. A subset $\mathfrak{Y} \subset \mathfrak{X}$ is relatively compact\index{relatively compact set} if the topological closure $\overline{\mathfrak{Y}}$ is compact in $\mathfrak{X}$. } in $Y$.
\end{enumerate}
\end{definition}

\begin{remark}If the operator $T: X \longrightarrow Y$ is linear and continuous, we do not need to check it maps every bounded set in a relatively compact one. Indeed, it is enough to verify whether or not the image of the unit ball is relatively compact, that is, if $\overline{T(B_X)}$ is compact in $Y$.  \end{remark}

\begin{lemma}\label{lemmadicaratterizzazione1}Let $T \in \mathcal{L}(X, \, Y)$ be a linear continuous operator between Banach spaces. Then $T$ is compact if and only if for every bounded sequence $(x_n)_{n \in \N} \subset B_X(0, \, c)$ there exists an increasing subsequence $(n_k)_{k \in \N}$ such that
\begin{equation*} \exists \, y \in Y \: : \: \lim_{k \to + \infty} \left\| T x_{n_k} - y \right\|_Y = 0.\end{equation*} \end{lemma}

The next goal is to give two compactness criteria, easy to check, that hold in reflexive (=$X \cong X^{\ast\ast}$) Banach spaces. Before we can do that, we need a characterization of the reflexivity in terms of the compactness of the ball with respect to a suitable topology.

\begin{lemma} \label{lemma.banachepraibls} A Banach space $\mathfrak{X}$ is reflexive if and only if the closed unit ball $\overline{B_{\mathfrak{X}}(0, \, 1)}$ is weakly compact. \end{lemma}

\begin{proof}First, we notice that the closed unit ball $\overline{B_{\mathfrak{X}}(0, \, 1)}$ is weakly compact if $\mathfrak{X}$ is a reflexive Banach space, as a consequence of the \hyperref[baseq]{Banach-Alaoglu Theorem}.

Vice versa, if $B_X$ is weakly compact, then it is weakly-$\ast$ closed in $X^{\ast\ast}$. Since the weakly-$\ast$ closure of $B_X$ in $X^{\ast\ast}$ is $B_{X^{\ast\ast}}$, and we get
\begin{equation*} B_X = B_{X^{\ast\ast}} \implies \text{$X$ is reflexive.}\end{equation*}

%Vice versa, suppose that $\overline{B_{\mathfrak{X}}(0, \, 1)}$ is weakly compact and let $(x_n)_{n \in \N} \subset \mathfrak{X}$ be a Cauchy sequence. We may always assume without loss of generality that $(x_n)_{n \in \N} \subset \overline{B_{\mathfrak{X}}(0, \, 1)}$ since every Cauchy sequence is bounded. By assumption, there is a weak cluster point $x \in \mathfrak{X}$. Fix $\epsilon > 0$ and $N \in \N$ such that
%\begin{equation*} \|x_n - x_m \| \leq \epsilon \quad \text{for all $n, \, m \geq N$,} \end{equation*}
%and fix $n \geq N$. Let $f \in X^\ast$ be a linear continuous functional such that $\|f\|_\ast \leq 1$, and notice that there must be $m \in \N$ such that
%\begin{equation*} \begin{cases} |f(x_m) - f(x)| \leq \epsilon, \\ |f(x_n) - f(x_m)| \leq \| x_n - x_m\| \leq \epsilon. \end{cases} \end{equation*}
%It follows from the triangular inequality that
%\begin{equation*} |f(x_n) - f(x)| = |f(x_n) \pm f(x_m) - f(x)| \leq 2 \epsilon, \end{equation*}
%which means that
%\begin{equation*} \sup_{\|f\|_\ast \leq 1} |f(x_n) - f(x)| \leq 2 \epsilon. \end{equation*}
%If we consider the Hahn-Banach unitary extension (see \hyperref[ex:13]{Example \ref{ex:13}}), we can find a functional $f \in \overline{B_{X^\ast}(0, \, 1)}$ such that
%\begin{equation*}\|x_n - x\| = |f(x_n) - f(x)| \leq 2 \epsilon, \end{equation*}
%and this is enough to infer that the whole sequence $x_n$ converges strongly to $x$. 

\end{proof}

\begin{lemma}\label{lemma:banachseparabile} Let $X$ be a reflexive (separable) Banach space, and let $M$ be a closed linear subspace of $X$. Then $(M, \, \|\cdot\|_M)$ is also a reflexive (separable) Banach space, where
\begin{equation*} \| \cdot \|_M = \| \cdot \| \, \big|_M. \end{equation*} \end{lemma}

\begin{proof}[First Proof] Recall that $X$ is reflexive if and only if the closed unit ball is compact in the weak topology, as a consequence of \hyperref[lemma.banachepraibls]{Lemma \ref{lemma.banachepraibls}}.

Therefore, it is enough to notice that the closed unit ball of $M$ is the intersection of the unit ball of $X$ with the closed subspace $M$, that is,
\begin{equation*} \overline{B_M} = \overline{B_X} \cap M \implies \text{$\overline{B_M}$ is closed}. \end{equation*} \end{proof}

\begin{proof}[Second Proof] Let $j_M : M \hookrightarrow X$ be the inclusion map. The dual operator
\begin{equation*} j_M^\ast : X^\ast \longtwoheadrightarrow M^\ast \end{equation*}
is surjective as a consequence of the \hyperref[theorem:hb]{Hahn-Banach Theorem \ref{theorem:hb}}, since it is nothing more than the restriction map:
\begin{equation*} j_M^\ast : X^\ast \ni f \longmapsto f \, \big|_{M} \in M^\ast. \end{equation*}
The dual of this operator, which is given by the bidual of the inclusion
\begin{equation*} j_M^{\ast\ast} : M^{\ast\ast} \hookrightarrow X^{\ast\ast}, \end{equation*}
is injective, and its image is weakly-$\ast$ closed\footnote{The rank $\rank j_M^\ast$ is weakly closed because $j_M^\ast$ is surjective and $M$ is closed w.r.t. the subspace topology. It follows from the \hyperref[closedrank]{Closed Rank Theorem \ref{closedrank}} that the rank of dual operator $j_M^{\ast\ast}$ is weakly-$\ast$ closed.}, that is, the rank is $\sigma(X^{\ast\ast}, \, X^\ast)$-closed. On the other hand, it can be easily identified with the closure of $M$ by noticing that
\begin{equation*} \begin{aligned} M^\ast \cong \faktor{X^\ast}{M^\perp} \implies M^{\ast\ast} & \cong \left(\faktor{X^\ast}{M^{\perp, \, \sigma(X, \, X^\ast)}} \right)^\ast \cong \\[1em] & \cong \left( M^{\perp, \, \sigma(X, \, X^\ast)} \right)^{\perp, \, \sigma(X^\ast, \, X^{\ast\ast})} \cong \\[1em] & \cong \left(M_{\perp, \, \sigma(X^{\ast \ast}, \, X^\ast)} \right)^{\perp, \, \sigma(X^\ast, \, X^{\ast\ast})} \cong \\[1em] & \cong \overline{M}^{\sigma(X^{\ast\ast}, \, X^\ast)}. \end{aligned} \end{equation*}
If $X$ is a reflexive Banach space, it turns out that $\sigma(X^{\ast \ast}, \, X^\ast) = \sigma(X, \, X^\ast)$ is the weak topology, and therefore
\begin{equation*} M^{\ast\ast} \cong  \overline{M}^{\sigma(X^{\ast\ast}, \, X^\ast)} \cong \overline{M}^{\tau_W} = M \end{equation*}
since we assumed $M$ to be a closed subspace of $X$.
\end{proof}

\begin{lemma} \label{weakbounded} Let $X$ be a Banach space. Every weakly converging sequence in $X$ is bounded. \end{lemma}

\begin{proof}Let $(x_n)_{n \in \N} \subset X$ be a weakly convergent sequence in $X$, and let $T_n \in X^{\ast\ast}$ be defined by
\begin{equation*}T_n(\ell) := \ell(x_n) \quad \text{for all $\ell \in X^\ast$ and $n \in \N$}. \end{equation*}
If we consider $\Gamma := \{T_n \: : \: n \in \N\}$, then the weak convergence of $x_n$ implies that $\Gamma$ is a pointwise bounded family in a second-category set. It follows from the \hyperref[stein]{Banach-Steinhaus Theorem} that
\begin{equation*} \sup_{n \in \N} \|x_n\|_X = \sup_{n \in \N} \| T_n \|_{X^{\ast\ast}} < + \infty, \end{equation*}
which means that $(x_n)_{n \in \N} \subset X$ is bounded. \end{proof}

\begin{lemma} \label{weakbounded.2} Let $X$ be a Banach space, and let $(x_n)_{n \in \N} \subset X$ be a sequence. If every subsequence $(n_k)_{k \in \N}$ admits a sub-subsequence $(n_{k_\ell})_{\ell \in \N}$ such that
\begin{equation*} \lim_{\ell \to + \infty} \| x_{n_{k_\ell}} \|_X = 0, \end{equation*}
then the whole sequence converges strongly to $0$, that is,
\begin{equation*} \lim_{n \to + \infty} \| x_n \|_X = 0. \end{equation*}\end{lemma}

\begin{proof}We argue by contradiction. Fix $\epsilon > 0$. We claim that for every $k \in \N$, there exists $n_k > k$ such that
\begin{equation} \label{eq.eq.100} \| x_{n_k}  \| \geq \epsilon. \end{equation}
If the claim holds, then it suffices to notice that the subsequence $x_{n_k}$ does not admit any converging sub-subsequence, which yields to a contradiction.

To prove that the claim \eqref{eq.eq.100} holds true, notice that, if for some $k \in \N$ there is no bigger index $n_k > k$ such that $ \| x_{n_k}  \| \geq \epsilon$, then
\begin{equation*} \|x_m  \| \leq \epsilon \quad \text{for all $m \geq k$,} \end{equation*}
and therefore the sequence $x_n$ converges strongly to $0$, in contradiction with what we have assumed.\end{proof}

\begin{lemma}\label{lemmmasldsdl}Let $X$ be a reflexive Banach space. A linear continuous operator $T \in \mathcal{L}(X, \, Y)$ is compact if and only if every sequence $(x_n)_{n \in \N} \subset X$, weakly converging to $0$, has the property that the image $\left(T x_n \right)_{n \in \N} \subset Y$ converges strongly (=w.r.t the norm $\| \cdot \|_Y$) to $0$ in $Y$. \end{lemma}

\begin{proof} We divide the argument into two steps to ease the notation.

\paragraph{$"\implies"$} Suppose that $T : X \longrightarrow Y$ is compact and let $(x_n)_{n \in \N} \subset X$ be a sequence such that
\begin{equation*} x_n \rightharpoonup 0.\end{equation*}
The set $\{ x_n \: \left| \: n \in \N \right. \}$ is bounded in $X$, as a consequence of \hyperref[weakbounded]{Lemma \ref{weakbounded}}, and thus (since $T$ is compact) there exists an increasing subsequence $(n_k)_{k \in \N}$ such that
\begin{equation*} T x_{n_k} \xrightarrow{\| \cdot \|_Y} y \in Y. \end{equation*}
On the other hand, the operator $T$ is continuous; hence $T x_n \rightharpoonup 0$ since
\begin{equation*}\left\langle \varphi, \, T x_n \right\rangle = \left\langle T^\ast \varphi, \, x_n \right\rangle \xrightarrow{n \to + \infty} 0. \end{equation*}
The uniqueness of the limit immediately implies that $y = 0$, and thus the usual argument based on the behavior of the sub-subsequences (see \hyperref[weakbounded.2]{Lemma \ref{weakbounded.2}}) proves the whole sequence $\left(T \, x_n \right)_{n \in \N}$ converges strongly to $0$.


\paragraph{$"\impliedby"$} Vice versa, let $(x_n)_{n \in \N} \subset X$ be a sequence weakly converging to $0$, and let 
\begin{equation*} X_0 := \overline{\mathrm{Span}\left\langle x_n \: : \: n \in \N \right\rangle} \end{equation*}
be the closure of the linear span generated by the sequence.

By \hyperref[lemma:banachseparabile]{Lemma \ref{lemma:banachseparabile}}, it follows that $X_0$ is a separable reflexive Banach space, and thus, by \hyperref[baseq]{Banach-Alaoglu Theorem \ref{baseq}}, the unit ball of $X_0^\ast$ is sequentially weakly-$\ast$ compact. On the other hand, since $X_0$ is a reflexive space, the weak-$\ast$ topology on $X_0^\ast$ is equal to the weak topology on $X_0$, and hence the unit ball of $X_0$ is weakly compact, that is, there is a subsequence $(n_k)_{k \in \N}$ such that
\begin{equation*} x_{n_k} \rightharpoonup x_\infty \in X. \end{equation*}
The sequence $\left( x_{n_k} - x_\infty \right)_{k \in \N}$ converges weakly to $0$. Therefore, by assumption, it follows immediately that
\begin{equation*} \left\| T x_{n_k} - T  x_\infty \right\| \xrightarrow{n \to + \infty} 0 \end{equation*}
in $Y$, which is enough to infer that the set $\{ T x_n \: \left| \: n \in \N \right. \}$ is relatively compact with respect to the strong topology in $Y$.
\end{proof}

\begin{lemma} Let $X$ and $Y$ be reflexive Banach spaces. A linear and continuous operator $T : X \longrightarrow Y$ is compact if and only if for any sequence $(x_n)_{n \in N} \subset X$ weakly converging to $0$ and any sequence $(y_n^\ast)_{n \in \N} \subset Y^\ast$ weakly-$\ast$ converging to $0$, it turns out that
\begin{equation*} \left\langle y_n^\ast, \, T x_n \right\rangle \xrightarrow{n \to + \infty} 0. \end{equation*}\end{lemma}

%\begin{proof} Recall that the spaces are reflexives; thus the weak-$\ast$ topologies on the duals coincide with the weak topologies on $X$ and $Y$ respectively.

%Suppose that $T$ is compact. The sequence $\{x_n\}_{n \in \N}$ is bounded; hence $\left(T x_n\right)_{n \in \N}$ is relatively compact in $Y$, which means that there is a subsequence $(n_k)$ such that
%\begin{equation*}T x_{n_k} \xrightarrow{\| \cdot \|_Y} 0 \implies T x_{n_k} \rightharpoonup 0. \end{equation*}
%It follows that
%\begin{equation*} \left\langle y_{n_k}^\ast, \, T x_{n_k} \right\rangle \xrightarrow{n \to + \infty} 0, \end{equation*}
%and this concludes the first part of the proof (using the usual sub-subsequences argument). The vice versa is left to the reader as an exercise.
%\end{proof} 

\subsection{Ideals of Finite-Rank Operators and Compact Operators}

In this section, we prove that compact operators give rise to an ideal and we investigate some of its main properties (e.g., if it is closed), included the relation with finite-rank operators.

Let $X, \, Y$ be Banach spaces. We denote by $\mathcal{L}_c(X, \, Y)$ the subset of the linear and continuous operators which are also compact.

\begin{definition}[Totally Bounded] \index{totally bounded metric space} A metric space $\left(M, \, d \right)$ is \textit{totally bounded} if and only if for every $\epsilon > 0$ there is a finite collection of open balls $B(x_1, \, \epsilon), \, \dots, \, B(x_N, \, \epsilon)$ of radius $\epsilon$ covering the space, that is,
\begin{equation*} M \subseteq \bigcup_{i = 1}^{N} B \left(x_i, \, \epsilon \right). \end{equation*}
\end{definition}

\begin{lemma} \label{totbound} Let $(M, \, d)$ be a complete metric space. A subset $Y \subseteq M$ is totally bounded if and only if $Y$ is relatively compact.\end{lemma}

\begin{lemma}[Ideal of Compact Operators] \label{lemma:cmoascprorps}\mbox{}
\begin{enumerate}[label=\textbf{(\arabic*)}]
\item The set of linear continuous compact operators $\mathcal{L}_c(X, \, Y)$ is a closed subspace of $\mathcal{L}(X, \, Y)$.
\item Let $T \in \mathcal{L}(X, \, Y)$ and $S \in \mathcal{L}(Y, \, Z)$ be two operators. If either of them is compact, then the composition $S \circ T$ is also compact.
\item The subspace $\mathcal{L}_c(X)$ is a closed bilateral ideal in the operator algebra with respect to the composition.
\item The subspace $\mathcal{L}_f(X)$ of finite-dimensional rank operators is an ideal in the operator algebra, but it is, in general, not closed. Moreover, it is the minimal nonzero ideal.
\end{enumerate} \end{lemma}

\begin{proof}\mbox{}
\begin{enumerate}[label=\textbf{(\arabic*)}]
\item Let $T \in \overline{\mathcal{L}_c(X, \, Y)}$ and let $S \in \mathcal{L}_c(X, \, Y)$ be a compact operator. The inclusion
\begin{equation} \label{totbouns123023} T \left(B_X\right) \subseteq S \left(B_X \right) + \| T - S \| \cdot B_Y \end{equation}
holds as a consequence of the triangular inequality. For any $\epsilon > 0$ there exist $S_\epsilon \in \mathcal{L}_c(X, \, Y)$ such that $\|T - S_\epsilon\| < \epsilon/2$, and a finite\footnote{The operator $S_\epsilon$ is compact and $Y$ is a Banach (complete) space. It follows that the image of the unit ball $B_X$ (a bounded set) is relatively compact in $Y$, and hence also totally bounded.} subset $F \subseteq S \left(B_X\right)$ such that
\begin{equation*} S_\epsilon  \left(B_X\right) \subseteq F + \frac{\epsilon}{2} \cdot B_Y. \end{equation*}
Therefore, the inclusion \eqref{totbouns123023} may be rewritten as follows
\begin{equation*} T \left(B_X\right) \subseteq F + \epsilon \cdot B_Y, \end{equation*}
and this is enough to conclude that $T \left(B_X \right)$ is totally bounded. Since $Y$ is a complete space, we infer from \hyperref[totbound]{Lemma \ref{totbound}} that $T \left(B_X \right)$ is also relatively compact, i.e., $T$ is a compact operator.
\item This assertion is a straightforward consequence of the fact that any linear continuous operator sends bounded   (resp. relatively compact) sets in bounded (resp. relatively compact) sets.
\item The subspace $\mathcal{L}_c(X)$ is closed by \textbf{(1)}, and it is a bilateral ideal by \textbf{(2)}.
\item Let $\mathcal{I}$ be a nonzero ideal of $\mathcal{L}(X)$, and let $T \neq 0$ be an element of $\mathcal{I}$. Let $x_0 \in X$ be any point such that $T x_0 \neq 0$, and let $\alpha, \, \beta \in X^\ast$ be linear continuous forms.

Recall that, for every fixed $y \in X$, the operators of the form
\begin{equation*}x \longmapsto y \left\langle \alpha, \, x \right\rangle \end{equation*}
have rank equal to $1$. Therefore, if we consider the composition with $T$ (which is still an element of $\mathcal{I}$ by assumption), then for $y_0 \in X$ fixed it turns out that
\begin{equation*} \left[ y_0 \langle \beta, \, \cdot \rangle \right] \circ T \circ \left[ x_0 \langle \alpha, \, \cdot \rangle \right] \in \mathcal{I}. \end{equation*}
If we let $\langle \beta, \, T  x_0 \rangle = 1$, then the composition above is given by
\begin{equation*} x \longmapsto y_0 \langle \alpha, \, x \rangle, \end{equation*}
and thus $\mathcal{I}$ contains all the operators with rank of dimension $1$.
\end{enumerate}
\end{proof}

In particular, the minimal closed ideal in the operator algebra is $\overline{\mathcal{L}_f(X)}$. It has been for long an open problem to determine whether it is equal to $\mathcal{L}_c(X)$ or if the inclusion can be strict.

In the $'73$ the mathematician Enflo showed that the equality could be strict in a particular Banach space \cite{Enflo1973}.

\begin{lemma} \label{lemma:compactseprable} Let $X$ and $Y$ be normed spaces, and let $T \in \mathcal{L}_c(X, \, Y)$ be a compact operator. Then
\begin{equation*} \text{$T(X) \subset Y$ is a separable Banach space}.\end{equation*} \end{lemma}

\begin{proof} Note that
\begin{equation*}X = \bigcup_{n \in \N} B(0, \, n), \end{equation*}
and every ball $B(0, \, n)$ is a bounded subset of $X$, which means that its image via $T$ is relatively compact in $Y$. On the other hand, a relatively compact set is always separable and
\begin{equation*}T(X) = \bigcup_{n \in \N} T\left(B(0, \, n) \right) \end{equation*}
is a countable union of separable set, meaning that it is also separable. \end{proof}

\begin{lemma} Let $X$ be a Hilbert space. Then the closure of the finite-rank operators is the ideal of the compact operators, i.e.,
\begin{equation*} \mathcal{L}_c(X) = \overline{\mathcal{L}_f(X)}. \end{equation*} \end{lemma}

\begin{proof} First, we notice that every finite-rank operator is compact. Since the ideal $\mathcal{L}_c(X)$ is closed, the first inclusion follows:
\begin{equation*} \mathcal{L}_c(X) \supseteq \overline{\mathcal{L}_f(X)}. \end{equation*}
Let $T : X \longrightarrow X$ be a compact operator. The closure of the image $\overline{T (X)}$ is a separable Hilbert space (see \hyperref[lemma:compactseprable]{Lemma \ref{lemma:compactseprable}}). Thus, we can always consider a sequence of projections $(P_n)_{n \in \N}$ over the $n$-dimensional subspaces
\begin{equation*}P_n : X \longrightarrow \mathrm{Span} \langle e_1, \, \dots, \, e_n \rangle, \end{equation*}
in such a way that
\begin{equation*} \overline{ \bigcup_{n \in \N} P_n (X) } = \overline{T (X) } \cong \ell_2. \end{equation*}
The convergence criterion (see \hyperref[convcrite]{Lemma \ref{convcrite}} below) implies that $P_n T \to T$ strongly (in norm), which means that
\begin{equation*} \lim_{n \to + \infty} \left\| P_n T - T \right\|_{X^\ast} = 0. \end{equation*}
On the other hand, by construction $P_n T$ is a finite-rank operator for all $n \in \N$. Therefore, we just proved that $T$ is the limit of a sequence of finite-rank operators with respect to the operator norm, which means that
\begin{equation*} \mathcal{L}_c(X) \subseteq \overline{\mathcal{L}_f(X)}. \end{equation*}
\end{proof}

\begin{lemma} \label{convcrite} Let $H$ be a Hilbert space and let $(x_n)_{n \in N} \subset H$ be a sequence. Then $x_n \to x \in H$ strongly if and only if $x_n \rightharpoonup x$ weakly and $\|x_n\| \to \|x\|$. \end{lemma}

\begin{proof} Suppose that $x_n \to x$ strongly (in norm). The weak topology is coarser than the topology induced by the norm; therefore strong convergence always implies weak convergence.

Moreover, the norm $\| \cdot \|_H$ is a continuous function with respect to the norm topology as a consequence of the reversed triangular inequality:
\begin{equation*} \left| \|x_n\| - \|x\| \right| \leq \|x_n - x\| \xrightarrow{n \to + \infty} 0. \end{equation*}
Vice versa, suppose that $x_n \rightharpoonup x$ weakly and that $\|x_n\| \to \|x\|$. By definition we have that
\begin{equation*}\| x_n - x \|^2 = \left\langle x_n - x, \, x_n - x \right\rangle = \left(\|x_n\|^2 + \|x\|^2 \right) - 2 \left\langle x_n, \, x \right\rangle, \end{equation*}
and it is easy to see that the first term converges to $2 \|x\|^2$, while the second converges to $- 2  \|x\|^2$.\end{proof}

\begin{lemma}\label{closedsubspace} Let $H$ be a Hilbert space and let $T \in \mathcal{L}(H)$ be a continuous linear operator. If $T$ is not compact, then there exists an infinite-dimensional closed subspace $M \subset H$ such that
\begin{equation*}\text{$T \, \big|_{M} : M \xrightarrow{\sim} T(M)$ is an isomorphism}\end{equation*}
and $T(M) \subset H$ is a closed infinite-dimensional subset of $H$.
\end{lemma}

\begin{proposition}Let $H$ be a separable Hilbert space. The compact operators form the unique bilateral closed ideal in the operators algebra. \end{proposition}

\begin{proof}Let $\mathcal{I}$ be a nonempty bilateral closed ideal in the operator algebra $\mathcal{L}(H)$, and suppose that there exists $T \in \mathcal{I}$ that is not compact.

By \hyperref[closedsubspace]{Lemma \ref{closedsubspace}} there exists an infinite-dimensional closed subspace $M \subset H$ such that $T \, \big|_M$ is invertible onto its image. On the other hand, since $M$ and $T(M)$ are both infinite-dimensional spaces, it turns out that
\begin{equation*} \text{$H$ separable} \implies M \cong T (M) \cong H. \end{equation*}
Consequently, the operator given by the composition
\begin{equation*} S : H \xrightarrow{\sim} M \xrightarrow{T} T(M) \xrightarrow{\sim} H \end{equation*}
belongs to $\mathcal{I}$, and this concludes the proof since $S$ is invertible and $S \in \mathcal{I}$ means that $\mathcal{I}$ is equal to the whole operator algebra. \end{proof} 

\subsection{Schauder Theorem}

\begin{theorem}[Schauder] \index{Schauder Theorem} Let $X, \, Y$ be Banach spaces. An operator $T \in \mathcal{L}(X, \, Y)$ is compact if and only if $T^\ast \in \mathcal{L}(Y^\ast, \, X^\ast)$ is compact. \end{theorem}

\begin{proof} We divide the argument into three steps to ease the notation.

\paragraph{Step 1.} Suppose that $T$ is a compact operator and let $(y_n^\ast)_{n \in \N} \subset Y^\ast$ be a bounded sequence, that is, there exists $C > 0$ such that
\begin{equation*} \|y_n^\ast\| \leq C. \end{equation*}
Consider the sequence defined by the restrictions on the closure of the image of the unit ball
\begin{equation*} \left( z_n^\ast \right)_{n \in \N} := \left( y_n^\ast \, \big|_{\overline{T (B_X)}} \right)_{n \in \N}, \end{equation*}
and notice that $z_n^\ast$ is a continuous function defined on the compact metric space $\overline{T(B_X)}$ (since $B_X$ is a bounded set in $X$ and $T$ is compact).

\paragraph{Step 2.} The sequence $(z_n^\ast)_{n \in \N}$ is clearly equibounded and equi-Lipschitz continuous with constant $0 < L \leq C$. It follows from the Ascoli-Arzelà theorem that one can find an increasing subsequence $(n_k)_{k \in \N}$ such that
\begin{equation*} z_{n_k}^\ast \doublerightarrow z \in \left(\overline{T(B_X)} \right)^\ast. \end{equation*}
In particular, it is a Cauchy sequence with respect to the uniform norm (since the convergence is uniform) on the closure of the image of the unit ball, and therefore
\begin{equation*} \begin{aligned} \left\| y_{n_k}^\ast - y_{n_j}^\ast \right\|_{\infty, \, \overline{T (B_X)}} & =  \left\| y_{n_k}^\ast - y_{n_j}^\ast \right\|_{\infty, \, T (B_X)} = \\[1em] & = \left\| y_{n_k}^\ast \circ T - y_{n_j}^\ast \circ T \right\|_{\infty, \, B_X} = \\[1em] & = \left\| T^\ast y_{n_k}^\ast - T^\ast y_{n_j}^\ast \right\|_{\infty, \, B_X} = \left\| T^\ast  y_{n_k}^\ast - T^\ast  y_{n_j}^\ast \right\|_{X^\ast}. \end{aligned} \end{equation*}
In conclusion, we infer that $\left( T^\ast y_{n_k}^\ast \right)_{k \in \N}$ is a Cauchy sequence in $X^\ast$, which is complete, and thus it converges.

\paragraph{Step 3.} Vice versa, assume that the dual operator $T^\ast$ is compact. The previous steps prove that the bidual operator $T^{\ast\ast}$ is compact, that is,
\begin{equation*} \text{$T^{\ast\ast}(B_X)$ has compact closure in $Y^{\ast\ast}$.} \end{equation*}
Since $T(B_X) = T^{\ast\ast}(B_X)$ and $Y$ is closed in $Y^{\ast\ast}$, we conclude that $T(B_X)$ has compact closure in $Y$, which is exactly what we wanted to prove.
\end{proof}

\section{Riesz-Fredholm Spectral Theory}

In this section, we state and prove the so-called \textit{Fredholm alternative theorem} for compact operators, which asserts that compact perturbation of the identity behaves in a very controlled manner.

\begin{theorem}[Fredholm alternative] \index{Fredholm Alternative Theorem}\index{Fredholm Alternative Theorem!for compact operators}\label{fa}Let $T \in \mathcal{L}_c(X)$ be a compact operator and let $X$ be a Banach space. \mbox{}
\begin{enumerate}[label=\textbf{(\alph*)}]
\item The kernel $\mathrm{Ker}(\I - T)$ is a closed, $T$-invariant and finite-dimensional subspace of $X$.
\item The kernels $K_n := \left( \mathrm{Ker}\left(\I - T \right)^n \right)_{n \in \N}$ form an increasing and stable sequence of closed, $T$-invariant and finite-dimensional subspaces of $X$.
\item The rank $\mathrm{Ran}(\I - T)$ is a closed, $T$-invariant and finite-codimensional subspace of $X$.
\item The ranks $R_n := \left( \mathrm{Ran}\left(\I - T \right)^n \right)_{n \in \N}$ form a decreasing and stable sequence of closed, $T$-invariant and finite-codimensional subspaces of $X$.
\end{enumerate} \end{theorem}

\begin{proof}\mbox{}
\begin{enumerate}[label=\textbf{(\alph*)}]
\item The kernel $\mathrm{Ker}(\I - T)$ is clearly closed, while the $T$-invariance follows from a simple computation:
\begin{equation*} x \in \mathrm{Ker}(\I - T) \implies x - T x = 0 \implies T x \in \mathrm{Ker}(\I - T). \end{equation*}
Therefore, the restriction operator
\begin{equation*}T \, \big|_{K_1} : \mathrm{Ker}(\I - T) \longrightarrow \mathrm{Ker}(\I - T) \end{equation*}
is compact and equal to the identity ($x = Tx$). On the other hand, if $\mathcal{B}$ is a Banach space, then the identity $\mathrm{id}_{\mathcal{B}}$ is compact if and only if $\mathcal{B}$ is locally compact if and only if $\mathcal{B}$ is a finite-dimensional space.
\item If we expand the binomial $(\I - T)^n$ using the Newton formula, we obtain an operator of the form $\I - T^\prime$, for some $T^\prime \in \mathcal{L}_c(X)$. In particular, the sequence $(K_n)_{n \in \N}$ is increasing and made up of closed, $T$-invariant and finite-dimensional subspaces.

We only need to prove that the sequence is stable, that is, there exists $n_0 \in \N$ such that
\begin{equation*} K_{n_0} = K_m \quad \text{for all $m \geq n_0$}. \end{equation*}
We argue by contradiction. Namely, we suppose that there exists a strictly increasing subsequence $(n_j)_{j \in \N}$ such that $K_{n_j} \subset K_{n_{j + 1}}$ is a strict inclusion. First, we claim that for all $j \in \N$ there exists a point $x_{n_j} \in K_{n_j}$ such that
\begin{equation} \label{cond1} \|x_{n_j}\| \leq 2 \quad \text{and} \quad d\left(x_{n_j}, \, x_{n_{j-1}} \right) = 1. \end{equation}

\paragraph{Proof Claim.} Let $K \subset L$ be closed subspaces of a Banach space $\mathcal{B}$. Then there exists a point $x \in L \setminus K$ such that
\begin{equation*} d\left(x, \, K \right) = 1. \end{equation*}
Since the distance between a point and a set is given by the infimum of the distances, there must be a point $k \in K$ such that $d(x, \, k) \leq 2$. Hence the point $u := x - k$ belongs to $L$ and it turns out that
\begin{equation*}  \|u\| = d(x, \, k) \leq 2 \quad \text{and} \quad d\left(u, \, K \right) = d(x, \, K) = 1. \end{equation*}

Back to the proof of \textbf{(b)}, for all $i < j$ we have that
\begin{equation*} \left\| T x_{n_j} - T x_{n_i} \right\| = \| \underbrace{x_{n_j}}_{\in K_{n_j}} - ( \underbrace{x_{n_j} - T x_{n_j}}_{\in K_{n_{j-1}}} + \underbrace{T x_{n_i}}_{\in K_{n_{i}} \subset K_{n_{j-1}}}) \| \geq 1, \end{equation*}
since $\I - T : K_n \longrightarrow K_{n - 1}$ and $d(x_{n_j}, \, K_{n_{j-1}}) = 1$.

This is in contradiction with the assumption that $T$ is compact since the sequence $\left(T x_{n_j} \right)_{j \in \N}$ does not admit any converging subsequence.
\item The kernel $\mathrm{Ker}(\I - T)$ is a finite-dimensional closed subspace of a Banach space. Therefore, there exists $X_0 \subseteq X$ such that
\begin{equation*} \text{$X_0$ is closed and $X = X_0 \oplus \mathrm{Ker}(\I - T)$ topological direct sum,} \end{equation*}
that is, the application $+ : X_0 \times K_1 \longrightarrow X$ is linearly invertible.

\paragraph{N.B.} If $X$ is a Banach space and $K \subset X$ is a closed infinite-dimensional subspace of $X$, then the existence of such a decomposition is, in general, not certain.

On the other hand, if $K$ and the algebraic supplement $K^\prime$ are closed, then it holds true (e.g, as a consequence of the \hyperref[opt]{Open Mapping Theorem \ref{opt}}).

Back to the proof of \textbf{(c)}, notice that the restriction $\left( \I - T \right) \, \big|_{X_0}$ is injective and it has the same rank. Therefore, the operator
\begin{equation*} \left(\I - T \right) \,\big|_{X_0} : X_0 \longrightarrow \mathrm{Ran}(\I - T) \end{equation*}
is bijective, linear and continuous. If we can prove that $\mathrm{Ran}(\I - T)$ is complete, then we will be able to conclude that it is also closed. Clearly, it suffices to prove that
\begin{equation*} \left(\I - T \right)^{-1} : \mathrm{Ran}(\I - T) \longrightarrow X_0 \end{equation*}
is continuous. We argue by contradiction. If the restriction is not a homeomorphism, then there is a sequence $(x_k)_{k \in \N} \subset X_0$ of elements, whose norm is uniformly equal to $1$, such that
\begin{equation*} \left\|x_k - T x_k \right\| \xrightarrow{k \to + \infty} 0. \end{equation*}
Since $T$ is a compact operator, up to subsequences, it turns out that $T x_k \to \xi$. It follows from the relation above that also $x_k \to \xi \in X_0$, and the norm of $\xi$ is also equal to $1$. This yields to a contradiction since
\begin{equation*} \xi \in \partial B_X(0, \, 1) \cap X_0 \cap \mathrm{Ker}(\I - T) = \partial B_X(0, \, 1) \cap \{0\} = \varnothing. \end{equation*}
\item If we expand the binomial $(\I - T)^n$, we obtain an operator of the form $\I - T^\prime$ for a certain $T^\prime \in \mathcal{L}_c(X)$. In particular, the sequence $(R_n)_{n \in \N}$ is decreasing and made up of closed, $T$-invariant and finite-codimensional (see \hyperref[theo:Sd]{Theorem \ref{theo:Sd}}) subspaces.

We only need to prove that the sequence is stable, that is, there exists $n_0 \in \N$ such that
\begin{equation*} R_{n_0} = R_m \quad \text{for all $m \geq n_0$}. \end{equation*}
We argue by contradiction. Namely, we suppose that there exists a strictly increasing subsequence $(n_j)_{j \in \N}$ such that $R_{n_j} \supset R_{n_{j + 1}}$ is a strict inclusion. First, we claim that for every $j \in \N$ there is a point $x_{n_j} \in R_{n_j}$ such that
\begin{equation} \label{cond2} \|x_{n_j}\| \leq 2 \qquad \text{and} \qquad d\left(x_{n_j}, \, x_{n_{j+1}} \right) = 1. \end{equation}
For every $i < j$, it turns out that
\begin{equation*} \left\| T x_{n_j} - T x_{n_i} \right\| = \| \underbrace{x_{n_i}}_{\in R_{n_i}} - ( \underbrace{x_{n_i} - T  x_{n_i}}_{\in R_{n_{i+1}}} + \underbrace{T x_{n_j}}_{\in R_{n_{j}} \subset R_{n_{i + 1}}}) \| \geq 1, \end{equation*}
since $\I - T : R_n \longrightarrow R_{n + 1}$ and $d(x_{n_i}, \, R_{n_{i+1}}) = 1$.

This is in contradiction with the assumption that $T$ is compact since the sequence $\left(T x_{n_i} \right)_{i \in \N}$ does not admit any converging subsequence.
\end{enumerate}\end{proof}

\begin{lemma}\label{invertt} Let $T \in \mathcal{L}_c(X)$ be a compact operator. Then
\begin{equation*} \text{$\I - T$ is injective if and only if $\I - T$ is surjective.} \end{equation*} \end{lemma}

\begin{proof} \mbox{}

\paragraph{$"\implies"$} Suppose that $\I - T$ is injective, but not surjective, and let $y \in X$ be a point that does not belongs to the image, i.e., $y \notin \mathrm{Ran}(\I - T)$. Then it is easy to prove that
\begin{equation*} (\I - T)^n y \in R_n \setminus R_{n+1} \implies \text{the sequence $(R_n)_{n \in \N}$ is not stable}, \end{equation*}
and this is absurd, as a consequence of the \hyperref[fa]{Fredholm Alternative Theorem \ref{fa}}.

\paragraph{$"\impliedby"$}Suppose now that $\I - T$ is surjective, but not injective, and let $y \in X$ be a nonzero point of the kernel, i.e., $y \neq 0$ and $(\I - T) \, y = 0$. For every $n \in \N$ there exists $y_n \in X$ such that
\begin{equation*} \left(\I - T \right)^n y_n = y \neq 0 \quad \text{and} \quad \left(\I - T \right)^{n+1} y_n = 0. \end{equation*}
It follows that the sequence $(K_n)_{n \in \N}$ is not stable, and this is absurd for the same reason mentioned above. \end{proof}

\begin{theorem}\label{theo:Sd}Let $T \in \mathcal{L}_c(X)$ be a compact operator. Then
\begin{equation*} \mathrm{dim} \, K_n = \mathrm{codim} \, R_n = \mathrm{codim} \, R_n^\ast = \mathrm{dim} \, K_n^\ast, \end{equation*}
where
\begin{equation*}K_n^\ast = \kers (\I - T^\ast)^n \quad \text{and} \quad R_n^\ast = \rank (\I - T^\ast)^n. \end{equation*} \end{theorem}

\begin{proof}Let us consider the projection
\begin{equation*}P : X \longrightarrow K_1 \end{equation*}
onto the finite-dimensional subspace $\mathrm{Ker}(\I - T)$, corresponding to the direct sum $X = X_0 \oplus K_1$ mentioned above. Let $s : K_1 \longrightarrow Y \subset X$ be a surjective (or injective\footnote{Both the kernel $K_1$ and the algebraic supplement $Y$ of the rank are vector spaces (one of which finite-dimensional), and therefore we can always find an injective or surjective map, depending on the respective dimensions.}) mapping between $K_1$ and the algebraic supplement $Y$ of $\mathrm{Ran}(\I - T)$, that is, the subspace $Y$ such satisfying the following properties:
\begin{equation*} Y \cap \mathrm{Ran}(\I - T) = \varnothing \qquad \text{and} \qquad X = Y + \mathrm{Ran}(\I - T). \end{equation*}
If we prove that $s$ is a bijection, then the following identity will arise automatically:
\begin{equation*} \mathrm{dim} \, \mathrm{Ker} (\I - T)  =  \mathrm{dim} \, Y =  \mathrm{codim} \, \mathrm{Ran} (\I - T). \end{equation*}

\paragraph{Step 1.} Let us consider the operator defined by
\begin{equation*} \I - T^\prime := \I - (T - s\circ P). \end{equation*}
Since the composition $s \circ P \in \mathcal{L}_f(X)$ is a finite-rank operator, we can apply \hyperref[invertt]{Lemma \ref{invertt}} and infer that
\begin{equation*} \text{$s$ is surjective $\iff \I - T^\prime$ is surjective $\iff \I - T^\prime$ is injective $\iff s$ is injective}, \end{equation*}
and this concludes the first part of the proof.

\paragraph{Step 2.} Recall that, if $T$ is a compact operator, then $T^\ast$ is also a compact operator. Therefore, it is enough to prove that
\begin{equation*} \mathrm{dim} \, \mathrm{Ker} (\I - T) =  \mathrm{dim} \, \mathrm{Ker} (\I - T^\ast). \end{equation*}
Indeed, a straightforward computation shows that
\begin{equation*}\begin{aligned}\mathrm{dim} \, \mathrm{Ker}(\I - T)  & = \mathrm{codim} \, \rank(\I - T) =
\\[1em] & = \mathrm{dim} \left( \faktor{X}{\rank(\I - T)} \right) = \\[1em] & \stackrel{{\color{blue}{(\ast)}}}{=} \mathrm{dim} \, \left( \faktor{X}{\mathrm{Ker}(\I - T^\ast)_\perp} \right) = \\[1em] & \stackrel{{\color{red}{(\ast)}}}{=} \mathrm{dim} \, \left( \faktor{X}{\mathrm{Ker}(\I - T^\ast)_\perp} \right)^\ast = \\[1em] & = \mathrm{dim} \, \left(\mathrm{Ker}(\I - T^\ast)_\perp \right)^\perp = \mathrm{dim} \, \left( \mathrm{Ker}(\I - T^\ast) \right),  \end{aligned}\end{equation*}
where ${\color{blue}{(\ast)}}$ follows from the \hyperref[closedrank]{Closed Rank Theorem \ref{closedrank}}, and ${\color{red}{(\ast)}}$ follows from the fact that the quotients are finite-dimensional (and $V^\ast \cong V$ if $\mathrm{dim}(V) < + \infty$).

\paragraph{Step 3.} In conclusion, the generalization to $n > 1$ is an easy consequence of the fact that
\begin{equation*} \left( \I - T \right)^n = \I + \sum_{i=1}^{n} (-1)^i \, \binom{n}{i} \, T^i = \I - T \circ \sum_{i=1}^{n} (-1)^i \, \binom{n}{i} \, T^{i-1},  \end{equation*}
since the composition between an operator and a compact operator is still compact.
\end{proof}

\begin{corollary} \label{chorsd}Let $T \in \mathcal{L}_c(X)$ be a compact operator defined on a Banach space, and let $n_0 \in \N$ be the natural number such that, for all $n \geq n_0$, it turns out that
\begin{equation*} \mathrm{Ker}(\I - T)^n = \mathrm{Ker}(\I - T)^{n_0} \quad \text{and} \quad \mathrm{Ran}(\I - T)^n = \mathrm{Ran}(\I - T)^{n_0} .\end{equation*}
Then, for any $n \geq n_0$, the space $X$ admits the $T$-invariant decomposition given by
\begin{equation*} X = \mathrm{Ker}(\I - T)^n \oplus \mathrm{Ran}(\I - T)^n.\end{equation*}
\end{corollary}

\begin{proof}Let $x \in \mathrm{Ker}(\I - T)^n \cap \mathrm{Ran}(\I - T)^n$, and let $y \in X$ be such that $(\I - T)^n \, y = x$. The sequences of subspaces are both stable after $n_0$; thus
\begin{equation*} (\I-T)^{2n} \, y = (\I - T)^{n} \, x = 0 \implies y \in \mathrm{Ker}(\I - T)^{2n} = \mathrm{Ker}(\I - T)^{n}, \end{equation*}
and this implies that $x = 0$. Moreover, it is easy to check that
\begin{equation*} X = \mathrm{Ker}(\I - T)^n + \mathrm{Ran}(\I - T)^n,\end{equation*}
since, by \hyperref[theo:Sd]{Theorem \ref{theo:Sd}}, the dimension of the kernel is exactly equal to the codimension of the rank.
\end{proof}

\section{Spectrum of a Compact Operator}

In this section, we investigate the basic properties of the spectrum of a compact operator, and we give an estimate from above of the spectral radius.

\paragraph{Introduction.} Let $\lambda \in \C \setminus \{0\}$ be any complex number. The results obtained in the previous section can be easily generalized to operators of the form $\lambda \cdot \I - T$ since one could always consider the compact perturbation
\begin{equation*} \lambda \left( \I - \frac{1}{\lambda} \cdot T \right). \end{equation*}

\begin{definition}[Spectrum] \index{spectrum} Let $T \in \mathcal{L}(X)$ be a linear and continuous operator. The \textit{spectrum} of $T$ is defined as the space of all $\lambda$ such that $(\lambda \cdot \I - T)$ is not invertible, that is,
\begin{equation*}\sigma(T) := \left\{ \lambda \in \C \: \left| \: \lambda \cdot \I - T \notin \mathrm{G_L}(X) \right. \right\}. \end{equation*}
The \textit{eigenvalue spectrum}\index{eigenvalue spectrum} of $T$ is defined by
 \begin{equation*}\sigma_{e}(T) := \left\{ \lambda \in \C \: \left| \: \mathrm{Ker} \left( \lambda \cdot \I - T\right) \neq 0 \right. \right\}. \end{equation*}
Furthermore, the elements of the eigenvalue spectrum are called \textit{eigenvalues}, while the nonzero elements of $\mathrm{Ker} \left( \lambda \cdot \I - T\right)$ are called \textit{eigenvectors} (or eigenstates). \end{definition}

\begin{definition}[Multiplicity] \index{multiplicity!algebraic}\index{multiplicity}\index{multiplicity!geometric} Let $T \in \mathcal{L}_c(X)$ be a compact operator defined on a Banach space $X$, and let $\lambda \in \C \setminus \{0\}$. The \textit{algebraic multiplicity} of $\lambda$ is defined by
\begin{equation} \label{ma} m_a(\lambda) := m_a(\lambda, \, T) = \mathrm{dim} \, \mathrm{Ker}( \lambda \cdot \I - T)^n, \end{equation}
where $n \geq n_0$. The \textit{geometric multiplicity} of $\lambda$ is defined as
\begin{equation} \label{mg} m_g(\lambda) := m_g(\lambda, \, T) = \mathrm{dim} \, \mathrm{Ker}( \lambda \cdot \I - T). \end{equation}
\end{definition}

\begin{notation} Let $T \in \mathcal{L}_c(X)$ be a compact operator, and let $n_0 \in \N$ be a natural number big enough that the sequences $(K_n)_{n \in \N}$ and $(R_n)_{n \in \N}$ are stable at $n_0$. We denote by $V(\lambda)$ the generalized autospace associated to $\lambda$, that is,
\begin{equation*} V(\lambda) := \mathrm{Ker}( \lambda \cdot \I - T)^n, \end{equation*}
and we denote by $R(\lambda)$ the generalized rank associated to $\lambda$, that is,
\begin{equation*} R(\lambda) := \mathrm{Ran}( \lambda \cdot \I - T)^n. \end{equation*}
\end{notation}

\begin{remark} Let $\lambda \in \C \setminus \{0\}$ be a nonzero complex number. \mbox{}
\begin{enumerate}[label=\textbf{(\alph*)}]
\item The algebraic multiplicity $m_a(\lambda)$ is always bigger than or equal to the geometric multiplicity $m_g(\lambda)$, as it follows from the \hyperref[fa]{Fredholm Theorem \ref{fa}}.
\item The complex number $\lambda$ is an eigenvalue for $T$ if and only if $m_a(\lambda) > 0$ if and only if $m_g(\lambda) > 0$.
\item If $T$ is a compact operator, then $m_a(\lambda, \, T) = m_a(\lambda, \, T^\ast)$ and, similarly, $m_g(\lambda, \, T) = m_g(\lambda, \, T^\ast)$.
\end{enumerate}\end{remark}

In particular, as a consequence of \hyperref[chorsd]{Corollary \ref{chorsd}}, the Banach space $X$ admits a \textit{spectral decomposition}\index{spectral decomposition} with respect to $\lambda$, that is $X$ is the direct sum of the generalized spaces:
\begin{equation*} X = V(\lambda) \oplus R(\lambda). \end{equation*}
The decomposition is $T$-invariant, and the restriction $(\lambda \cdot \I - T) \, \big|_{V(\lambda)} : V(\lambda) \longrightarrow V(\lambda)$ is a nilpotent operator, while the restriction $(\lambda \cdot \I - T) \, \big|_{R(\lambda)} : R(\lambda) \longrightarrow R(\lambda)$ is an invertible operator.
 
\begin{remark} If $T$ is defined on an infinite-dimensional Banach space, then, in general, the spectrum $\sigma(T)$ contains the eigenvalue spectrum $\sigma_e(T)$ strictly (see, e.g., \hyperref[shiftspectrum]{Exercise \ref{shiftspectrum}}). \end{remark}

\begin{remark} If $T \in \mathcal{L}_c(X)$ is a compact operator, then
\begin{equation*} \sigma(T) \setminus \{0\} = \sigma_e(T) \setminus \{0\}. \end{equation*} \end{remark}

\begin{lemma}Let $T \in \mathcal{L}(X)$ be a linear continuous operator. The spectrum $\sigma(T)$ is a closed subset of $\C$, and it is contained in the closed ball of center $0$ and radius $\|T\|$, i.e.,
\begin{equation*} \sigma(T) \subseteq \overline{B_{\C}(0, \, \|T\|)}. \end{equation*} \end{lemma}

\begin{proof}The first assertion is an easy consequence of the fact that
\begin{equation*}A \in \mathrm{G_L}(X) \implies B\left( A, \, \|A^{-1}\|^{-1}\right) \subseteq \mathrm{G_L}(X), \end{equation*}
that is, the complement of $\sigma(T)$ is an open subset of $\C$. On the other hand, if $\lambda$ is a complex number such that $|\lambda| > \|T\|$ strictly, then
\begin{equation*}\lambda \cdot \I - T = \lambda \cdot \left( \I - \frac{T}{\lambda} \right), \end{equation*}
and it is easy to prove that the operator $\I - \frac{T}{\lambda}$ is invertible (e.g., notice that the modulus of any eigenvalue is bigger than or equal to $1 - \|T\|/\lambda > 0$.)
\end{proof}

\begin{remark}[Spectrum of Compact Operators] \mbox{}
\begin{enumerate}[label=\textbf{(\alph*)}]
\item If $X$ is an infinite-dimensional Banach space, then $0 \in \sigma(T)$ for any $T \in \mathcal{L}_c(X)$.

Indeed, a compact operator is invertible if and only if $X$ is locally compact if and only if $X$ is finite-dimensional.
\item There are compact operators such that $\sigma(T) = \{0\}$ and $T \not \equiv 0$ (see, e.g., \hyperref[zerospectrum]{Exercise \ref{zerospectrum}}).
\end{enumerate}\end{remark}

\begin{lemma}\label{lemma:Sasfasd}Let $T$ be a compact operator. The eigenvalue spectrum $\sigma_e(T)$ is made up of isolated points, i.e., for any $\lambda \in \sigma_e(T)$ there exists $\epsilon > 0$ such that
\begin{equation*} B_{\C}(\lambda, \, \epsilon) \cap \sigma_e(T) = \{ \lambda \}. \end{equation*} \end{lemma}

\begin{proof} Let $\lambda, \, \mu \in \sigma_e(T)$ be two complex numbers such that $\lambda \neq \mu$. The spectral decomposition associated to $\lambda$, i.e.
\begin{equation*} X = V(\lambda) \oplus R(\lambda), \end{equation*}
is invariant under the action of the operator $\mu \cdot \I - T$. Moreover, the identity
\begin{equation*} \mu \cdot \I - T = \left(\mu - \lambda \right) \cdot \I + \left( \lambda \cdot \I - T \right) \end{equation*}
implies that $\mu \cdot \I - T$ is equal to a nonzero multiple of the identity, minus a nilpotent operator on $V(\lambda)$. Hence, a simple algebraic lemma proves that $\mu \cdot \I - T$ is invertible on $V(\lambda)$.

In conclusion, since the operator $\mu \cdot \I - T$ is invertible for $\lambda$ sufficiently near to $\mu$ (the condition is open, as mentioned before), the unique possibility is that $\lambda$ is an isolated point. \end{proof}

If $T \in \mathcal{L}_c(X)$ is a compact operator defined on a infinite-dimensional Banach space, then $\sigma(T)$ is given by the singlet $\{0\}$ and it contains all the nonzero eigenvalues whose algebraic multiplicity is strictly greater than $0$, that is,
\begin{equation*}T \in \mathcal{L}_c(X) \implies \sigma(T) = \{0\} \cup \sigma_e(T). \end{equation*}
Moreover, as a consequence of \hyperref[lemma:Sasfasd]{Lemma \ref{lemma:Sasfasd}}, it turns out that $0$ is an cluster point for the eigenvalue spectrum of $T$, that is, the nonzero eigenvalues of $T$ form a sequence which converges to $0$.

\begin{remark}In the proof of \hyperref[lemma:Sasfasd]{Lemma \ref{lemma:Sasfasd}}, we have proved that, if $\mu$ and $\lambda$ are "far enough", then
\begin{equation*} V(\lambda) \subset R(\mu) \qquad \text{and} \qquad V(\mu) \subset R(\lambda). \end{equation*} \end{remark} 

\paragraph{Iterated Decomposition.}\index{iterated decomposition} Let $\lambda, \, \mu \in \C \setminus\{0\}$ be two complex numbers such that $\lambda \neq \mu$, and let $P_\lambda : X \longrightarrow V(\lambda)$ and $P_\mu : X \longrightarrow V(\mu)$ be the projections. It is easy to see that there is a spectral decomposition associated to both $\lambda$ and $\mu$, that is,
\begin{equation*} X = V(\lambda) \oplus V(\mu) \oplus \left( R(\lambda) \cap R(\mu) \right). \end{equation*}
Since $P_\lambda \circ P_\mu = P_\mu \circ P_\lambda = 0$, it tuns out that $(\I - P_\mu - P_\lambda)$ is an idempotent operator (a projection), and it is clearly the one associated with the decomposition above.

\begin{notation} Let $\lambda \in \sigma_e(T)$. If $T$ is a compact operator, then the projection associated with the spectral decomposition is denoted by $E(\lambda) : X \longrightarrow V(\lambda)$, and it is called \textit{spectral projection}.

In a similar fashion, the projection onto the rank is denoted by $D(\lambda) : X \longrightarrow R(\lambda)$, and it is defined as the orthogonal projection to the spectral one, i.e. $D(\lambda) := \I - E(\lambda)$. \end{notation}

In particular, for any finite subset $F \subset \sigma(T)$ (e.g., $F = \sigma(T) \setminus B(0, \, \epsilon)$) we have a spectral decomposition
\begin{equation*} X = \left( \bigoplus_{\lambda \in F} V(\lambda) \right) \bigoplus \left( \bigcap_{\lambda \in F} R(\lambda) \right). \end{equation*}

\paragraph{Complexification.} Let $X$ be a real Banach space. The complexification of $X$ is the space
\begin{equation*} X_\C := \C \otimes_{\R} X, \end{equation*}
with a complex structure induced by the action of the operator
\begin{equation*} J = \begin{pmatrix}0 & - \I \\ \I & 0 \end{pmatrix}, \end{equation*}
where $(a + \imath \, b) \cdot z := az + b \, Jz$, as $z$ ranges in $X \oplus X$.

Let $T \in \mathcal{L}^{\R}(X, \, Y)$ be an operator between real topological vector spaces. The complexification of $T$ is the operator defined by
\begin{equation*} T_\C = \begin{pmatrix}T & 0 \\ 0 & T \end{pmatrix} \in \mathcal{L}^{\C}(X_\C, \, Y_\C).\end{equation*}

\section{Symmetric Operators}

In this section, we introduce a new class of operators, called \textit{symmetric operators}, defined on a Hilbert space $H$. We investigate some of their main properties, and we prove that the kernel and the rank behave nicely with respect to the orthogonal.

\begin{definition}[Hilbert Adjoint] \index{adjoint operator} Let $A : H \longrightarrow K$ be a linear continuous operator between two Hilbert spaces $H$ and $K$. The hilbertian adjoint of $A$ is the operator $A^\ast : K \longrightarrow H$ defined by the formula
\begin{equation}\label{hiladd} \left(A  x, \, y \right)_K =  \left(x, \, A^\ast y \right)_H,\end{equation}
for all $x\in H$ and all $y \in K$.\end{definition}

\begin{remark}Let $A : H \longrightarrow K$ be a linear continuous operator between Hilbert spaces $H$ and $K$. \mbox{}
\begin{enumerate}[label=\textbf{(\arabic*)}]
\item For every complex number $\lambda \in \C \setminus \{0\}$ it turns out that
\begin{equation*} \left(\lambda \cdot A \right)^\ast = \overline{\lambda} \cdot A^\ast. \end{equation*}
Indeed, if we replace $A$ with $\lambda \cdot A$ in the left-hand side of \eqref{hiladd}, then
\begin{equation*} \begin{aligned} \left(\lambda \cdot A x, \, y \right)_K & = \lambda \left( A \, x, \, y \right)_K = \\[1em] & = \lambda \left(x, \, A^\ast y \right)_H = \left(x, \, \overline{\lambda} \cdot A^\ast y \right)_H. \end{aligned}\end{equation*}
\item The map $\mathcal{L}(H, \, K) \ni A \longmapsto A^\ast \in \mathcal{L}(K, \, H)$ is semi-linear.
\end{enumerate} \end{remark}

\begin{definition}[Symmetric Operator] \index{symmetric operator} Let $A : H \longrightarrow H$ be a linear continuous operator defined on a Hilbert space $H$. We say that $A$ is \textit{symmetric} if $A = A^\ast$, and we denote by $\Ls(H)$ the space of all symmetric operator defined on a Hilbert space $H$. \end{definition}

\begin{proposition}\label{prop:ranksym} Let $A \in \Ls(H)$ be a symmetric operator. \mbox{}
\begin{enumerate}[label=\textbf{(\alph*)}]
\item $\left(\mathrm{Ran}(A) \right)^\perp = \mathrm{Ker}(A)$.
\item $\left(\mathrm{Ker}(A) \right)^\perp = \overline{\mathrm{Ran}(A)}$.
\item If $H_0 \subseteq H$ is an $A$-invariant subset, then $\overline{H_0}$ and $H_0^\perp$ are also $A$-invariant.
\item The spectrum is contained in the open ball of radius $\|A\|$ of $\R$, that is,
\begin{equation*} \sigma(A) \subset \left[ - \|A\|, \, \|A\| \right] \subset \R. \end{equation*}
\item If $\lambda, \, \mu \in \sigma_e(A)$ are eigenvalues of $A$ such that $\lambda \neq \mu$, then the corresponding eigenvectors are orthogonal.
\item The eigenvalues are \textit{semisimple}, that is,
\begin{equation*} m_a(\lambda) = m_g(\lambda), \qquad \forall \, \lambda \in \sigma_e(A). \end{equation*}
\end{enumerate}
\end{proposition}

\begin{proof} \mbox{}
\begin{enumerate}[label=\textbf{(\alph*)}]
\item This assertion follows from \hyperref[inclsuos]{Lemma \ref{inclsuos}} since the operator is symmetric (i.e., it is equal to its dual.)
\item This assertion follows from \hyperref[inclsuos]{Lemma \ref{inclsuos}}. Indeed, the operator $A$ is symmetric and one can show that in a Hilbert space $H$, given two closed subsets $M$ and $N$, it turns out that\footnote{It suffices to notice that, for example, $\left(M^\perp\right)_\perp  = \left(M^\perp\right)^\perp = M$, as we have proved in the first chapter. }
\begin{equation*} \left(M^\perp\right)_\perp = M \quad \text{and} \quad \left(N_\perp \right)^\perp = N. \end{equation*}
\item Both the assertion follows easily from the definition; we prove the second one, and leave the first one as a simple exercise for the reader.

Let $x \in H_0^\perp$ be any point. For any $y \in H_0$, it turns out that
\begin{equation*} \left(A x, \, y \right)_H = \left(x, \, A y \right)_H = 0,  \end{equation*}
and hence $A x \in H_0^\perp$, which is what we wanted to prove.
\item First, we prove that the any element of the spectrum belongs to $\R$. Let $a \in \R$ and $b \in \R \setminus \{0\}$ be given; the idea is to show that operator $(a + \imath b) \cdot \I + A$ is always invertible.

We claim that the (composition) operator
\begin{equation*} \left[ \left( a - \imath b \right) \cdot \I + A \right] \circ \left[ \left( a + \imath b \right) \cdot \I + A \right] \end{equation*}
is invertible. Indeed, if we evaluate the product, it turns out that
\begin{equation*} \left[ \left( a - \imath b \right) \cdot \I + A \right] \circ \left[ \left( a + \imath b \right) \cdot \I + A \right] = b^2 \left( \I + \left( \frac{a + A}{b} \right)^2 \right), \end{equation*}
and thus it is enough to prove that $\I + A^2$ is invertible for every symmetric operator $A \in \Ls(H)$. The estimate
\begin{equation*} \left\| (\I + A^2) x \right\|_H^2 \geq \|x\|^2 > 0\qquad \forall \, x \neq 0, \end{equation*}
proves that $\I + A^2$ is injective with closed image (see \hyperref[corollary:skdsd]{Corollary \ref{corollary:skdsd}}). In conclusion, the assertion \textbf{(b)} implies $\I + A^2$ surjective, and thus bijective.
\item This assertion follows from a straightforward computation:
\begin{equation*} \begin{aligned} \left( \lambda x, \, y \right) & = (Ax, \, y) = (x, \, Ay) = \\[1em] & = (x, \, \mu  y) = \\[1em] & = (\mu x, \, y) \iff (\lambda - \mu) (x, \, y) = 0 \iff (x, \, y) = 0 \iff x \perp y. \end{aligned}\end{equation*}
\item It is enough to prove that $\mathrm{Ker}(\lambda \cdot \I - T)^2 = \mathrm{Ker}(\lambda \cdot \I - T)$; the assertion will follow by means of a simple induction.

Let $x \in H$ be an element such that $(\lambda \cdot \I - T)^2 x = 0$. Then
\begin{equation*}\begin{cases} (\lambda \cdot \I - A) x \in \mathrm{Ker}(\lambda \cdot \I - A) \\[0.8em] (\lambda \cdot \I - A) x \in \mathrm{Ran}(\lambda \cdot \I - A) \end{cases} \implies (\lambda \cdot \I - A) x = 0 \implies x \in \mathrm{Ker}(\lambda \cdot \I - A). \end{equation*}
\end{enumerate}\end{proof}

\paragraph{Quadratic Form.} Let $q_A : H \longrightarrow \R$ be the quadratic form\index{quadratic form} associated to an operator $A$, that is,
\begin{equation*} q_A(x) := \left( A x, \, x \right)_H. \end{equation*}
It is easy to prove that the following inequality holds (even in a more general setting than Hilbert spaces, e.g., Banach spaces):
\begin{equation*}r_A := \sup_{\|x\| \leq 1} |q_A(x)| \leq \sup_{ \substack{\|x\| =1 \\[0.1em] \|y\| = 1}}  \left| \left\langle Ax, \, y \right\rangle \right| = \sup_{\|x\| = 1} \|Ax\| = \| A \|. \end{equation*}
On the other hand, in our particular case of Hilbert spaces, the \textit{polarization equality} characterizes the scalar product in terms of the quadratic form:
\begin{equation} \label{polid} \left(A  x, \, y \right)_H =\frac{1}{4} \, \left[ q_A(x + y) - q_A(x - y) \right]. \end{equation}
It follows that
\begin{equation*}\left| \left(Ax, \, y \right)_H \right| \leq \frac{1}{4}  \left[ \|x+ y\|^2 + \|x-y\|^2 \right] \cdot r_A = \frac{1}{2} r_A  \left( \|x\|^2 + \|y\|^2 \right), \end{equation*}
and thus we infer that $r_A$ is equal to the operator norm of $A$.

\paragraph{Variational Characterization.} Let $A \in \Ls(H)$. The \textit{Rayleigh quotient}\index{Rayleigh quotient} is the function
\begin{equation} \label{Rq} f_A(x) = \frac{\left(Ax, \, x \right)_H}{(x, \, x)_H} : H \setminus \{0\} \longrightarrow \R, \end{equation}
which is clearly a $C^\infty$ function (in the sense of Fréchet derivatives). Then
\begin{equation*} \nabla  f_A(x) = \frac{2}{\|x\|^2} \left( Ax - f_A(x), \, x \right)_H, \end{equation*}
and hence a couple $(\lambda, \, x) \in \R \times (H \setminus \{0\})$ is a (eigenvalue, eigenvector) couple if and only if $x$ is a critical point for $f_A$ and $\lambda = f_A(x)$. 

\section{Compact Symmetric Operators}

In this section, we investigate the compact symmetric operators, and we prove that the standard \textit{spectral theorem} holds true even in infinite-dimensional spaces.

\begin{lemma} \label{lemmaaaa1} Let $A \in \mathcal{L}_c(H)$ be a compact operator defined on a Hilbert space $H$. If
\begin{equation*} \alpha := \sup_{\|x\| = 1} \left\langle A x, \, x \right\rangle \neq 0,\end{equation*}
then there exists $x_0 \in \partial B_H(0, \, 1)$ such that $\alpha = \left( A x_0, \, x_0 \right) $.  \end{lemma}

\begin{proof}We can assume that $H$ is a separable Hilbert space. Indeed, we can always consider the decomposition
\begin{equation*}H = \mathrm{Ker}(A) \oplus \overline{\mathrm{Ran}(A)},\end{equation*}
and restrict, without loss of generality, the quadratic form $q_A$ to $\overline{\mathrm{Ran}(A)}$, which is always a separable Hilbert space (see \hyperref[lemma:compactseprable]{Lemma \ref{lemma:compactseprable}}). Let $(x_n)_{n \in \N} \subset H$ be a maximizing sequence, that is,
\begin{equation*} \|x_n\| = 1 \qquad \text{and} \qquad \left\langle Ax_n, \, x_n \right\rangle \xrightarrow{n \to + \infty} \alpha. \end{equation*}
By \hyperref[baseq]{Banach-Alaoglu Theorem \ref{baseq}} it turns out that there exists an increasing subsequence $(n_k)_{k \in \N}$ such that $x_{n_k} \rightharpoonup x$ weakly. Notice also that the limit point $x$ has norm $\|x\| \leq 1$ since
\begin{equation*}\|x\|^2 = \left\langle x, \, x \right\rangle = \lim_{n \to + \infty} \left\langle x_n, \, x \right\rangle \leq \|x\| \|x_{n_0}\| = \|x\|. \end{equation*}
To prove that the norm of $x$ is exactly one, let us set $y = x/\|x\|$ and observe that
\begin{equation*} \alpha \geq \left\langle A y, \, y \right\rangle = \alpha \frac{1}{\|x\|^2} \iff \|x\| = 1, \end{equation*}
The operator $A$ is compact. Therefore (see \hyperref[lemmadicaratterizzazione1]{Lemma \ref{lemmadicaratterizzazione1}}) there exists an increasing subsequence such that $A  x_{n_k} \to A x$ strongly (in norm), and this implies that 
\begin{equation*} \left\langle Ax, \, x \right\rangle_H = \alpha, \end{equation*}
that is, the limit point $x$ is the sought one. \end{proof} 

\begin{lemma} \label{crit2} Let $H$ be a Hilbert space. If $(x_n)_{n \in \N} \subset H$ is a sequence weakly converging to $x$, and $(y_n)_{n \in \N} \subset H$ is a sequence strongly converging to $y$, then
\begin{equation*} \left( x_n, \, y_n \right) \xrightarrow{n \to + \infty} \left(x, \, y \right). \end{equation*}
\end{lemma}

\begin{theorem}\label{th:asdsda} Let $A \in \mathcal{L}_c^{\mathrm{sym}}(H)$ be a symmetric compact operator defined on a Hilbert space $H$, and let $q_A$ be the associated quadratic form. Then either $\|A\|$ or $- \|A\|$ is an element of $\sigma_{e}(A)$.\end{theorem}

\begin{proof} In the previous section, we have proved that
\begin{equation*}A \in \Ls(H)\implies \|A\| = \sup_{\|x\| = 1} \left| q_A(x) \right|. \end{equation*}
It follows that
\begin{equation*}\begin{aligned}\|A\| & = \sup_{\|x\| = 1} \left| \left( A x, \, x \right) \right| = \sup_{\|x\|=1} \, \max \left\{ \left( A  x, \, x \right), \, - \left( A x, \, x \right) \right\} = \\[1em] & = \max \left\{ \sup_{\|x\| = 1} \left( A  x, \, x \right), \, - \inf_{\|x\| = 1} \left( A x, \, x \right) \right\}, \end{aligned}\end{equation*}
and hence either $\| A \| = \sup_{\|x\| = 1} \left( A x, \, x \right)$ or $- \|A\| = \inf_{\|x\|=1} \left( Ax, \, x \right)$. We now assume, without loss of generality, that
\begin{equation*} \| A \| = \sup_{\|x\| = 1} \left( A x, \, x \right). \end{equation*}
The operator $A$ is compact; hence, by \hyperref[lemmaaaa1]{Lemma \ref{lemmaaaa1}}, there exists $x \in H$ such that $\|x\| \leq 1$, and there is a maximizing sequence $x_n \rightharpoonup x$ such that
\begin{equation*} \left< A x_n, \, x_n \right> \xrightarrow{n \to + \infty} \left< A x, \, x \right> = \| A \| \neq 0 \implies x \neq 0. \end{equation*}
To prove that the norm of $x$ is exactly one, let us set $y = x/\|x\|$ and observe that
\begin{equation*} \alpha \geq \left< A y, \, y \right> = \alpha \, \frac{1}{\|x\|^2} \iff \|x\| = 1, \end{equation*}
so that, by \hyperref[convcrite]{Lemma \ref{convcrite}}, the convergence $x_n \to x$ is strong (it is not needed for the proof, but it is an interesting fact worth remarking on).

The variational characterization (in terms of the Rayleigh quotient) concludes the proof, since we have proved that $x$ is a critical point of $f_A$ and $\|A\|$ is exactly equal to $f_A(x)$.
\end{proof}

\begin{corollary} \label{coorss231} Let $A \in \mathcal{L}_c^{\mathrm{sym}}(H)$ be a symmetric compact operator defined on a Hilbert space $H$. Each nonempty, closed and invariant subspace $H_0 \subseteq H$ contains the eigenvector of $A$ associated either to the eigenvalue $\|A\|_{H_0}$ or to $-\|A\|_{H_0}$. \end{corollary}

\begin{theorem}[Spectral Theorem]\index{spectral theorem} \label{sptthr}Let $A \in \mathcal{L}_c^{\mathrm{sym}}(H)$ be a symmetric compact operator defined on a Hilbert space $H$. There exists an orthogonal basis of eigenvectors, i.e., the operator $A$ is diagonalizable.\end{theorem}

\begin{proof}We can assume without loss of generality that $H$ is a separable Hilbert space, since the same argument of \hyperref[lemmaaaa1]{Lemma \ref{lemmaaaa1}} holds.

Let $\{e_k\}_{k \in \N}$ be an orthonormal system of eigenvectors for $A$, and suppose that it is maximal with respect to the inclusion (simple application of Zorn lemma). We claim that
\begin{equation*} \overline{\mathrm{Span}\left<e_k \: : \: k \in \N \right>} = H. \end{equation*}
If not, by \hyperref[coorss231]{Corollary \ref{coorss231}}, there would be an eigenvector in $H_0 := \left(\mathrm{Span}\left<e_k \: : \: k \in \N \right> \right)^\perp \neq \varnothing$, and this is absurd since the system $\{e_k\}_{k \in \N}$ is maximal.\end{proof}

In particular, any $A \in \mathcal{L}_c^{\mathrm{sym}}(H)$ is unitary equivalent to a multiplication operator defined on $\ell_2(\N)$ by an element of $c_0$, i.e., there is $\lambda \in c_0$ such that
\begin{equation*} \begin{tikzcd} H \arrow[r, "A"] \arrow[leftrightarrow]{d} & H \arrow[leftrightarrow]{d} \\ \ell_2 \arrow[r, "\cdot \lambda"] & \ell_2 \end{tikzcd} \end{equation*}

\section{Applications of the Spectral Theorem}

\paragraph{Minimax Principle.} Let $A \in \mathcal{L}_c^{\mathrm{sym}}(H)$ be a symmetric compact operator defined on a Hilbert space $H$. The \hyperref[sptthr]{Spectral Theorem \ref{sptthr}} allows us to index the positive eigenvalues with positive integers, that is,
\begin{equation*}\lambda_1 \geq \lambda_2 \geq \dots \geq \lambda_n \geq \dots,\end{equation*}
where an eigenvalue $\lambda$ is repeated as many times as its multiplicity. In a similar fashion, we can index the negative eigenvalues
\begin{equation*}\lambda_{-1} \leq \lambda_{-2} \leq \dots \leq \lambda_{-m} \leq \dots.\end{equation*}

The set of indices $I \subset \Z$ allows us to identify the Hilbert space $H$ with $\ell_2(I)$, and the operator $A$ with the multiplication operator
\begin{equation*} x \longmapsto \sum_{i \in I} x_i \lambda_i \, e_i. \end{equation*}

\begin{theorem}[Courant-Fisher] \index{Courant-Fisher Theorem} \label{c-f-t} Let $A \in \mathcal{L}_c^{\mathrm{sym}}(H)$. For every $n \in I$ it turns out that
\begin{equation*} \lambda_n = \inf_{\substack{E \subseteq H \\[0.1em] \mathrm{codim}(E) < n}}  \, \sup_{\substack{x \in E \\[0.1em] \|x\|=1}} \left( A x, \, x \right) = \sup_{\substack{E \subseteq H \\[0.1em] \mathrm{dim}(E) \geq n}}  \, \inf_{\substack{x \in E \\[0.1em] \|x\|=1}} \left( Ax, \, x \right). \end{equation*}
Similarly, for any $-n \in I$ it turns out that
\begin{equation*} \lambda_{-n} = \sup_{\substack{E \subseteq H \\[0.1em] \mathrm{codim}(E) < n}}  \, \inf_{\substack{x \in E \\[0.1em] \|x\|=1}} \left( A x, \, x \right) = \inf_{\substack{E \subseteq H \\[0.1em] \mathrm{dim}(E) \geq n}}  \, \sup_{\substack{x \in E \\[0.1em] \|x\|=1}} \left( Ax, \, x \right). \end{equation*}\end{theorem}

\begin{proof} The second assertion follows immediately from the first one since
\begin{equation*} \lambda_{-n}(A) = - \lambda_n(-A). \end{equation*}
Let us identify $H$ with $\ell_2(I)$ and $A$ with the multiplication operator $\cdot \lambda$. Let $\{e_i\}_{i \in I}$ be the set of eigenvectors of $A$, and let us consider
\begin{equation*} E_n := \mathrm{Span} \left<e_1, \, \dots, \, e_n \right>, \qquad E_{n-1}^\perp = \overline{ \mathrm{Span}\left< e_i \: \left| \: i \in I \setminus \{1, \, \dots, \, n -1\} \right. \right>}. \end{equation*}
We consider the restriction operators
\begin{equation*} \begin{aligned} &A \, \big|_{E_n} : x \longmapsto \sum_{i = 1}^{n} \lambda_i x_i \, e_i \quad \text{has eigenvalues $\lambda_1, \, \dots, \, \lambda_{n}$} \\[1em] & A \, \big|_{E_{n-1}^\perp} : x \longmapsto \sum_{i \in I \setminus \{1, \, \dots, \, n-1\} } \lambda_i x_i \, e_i \quad \text{has eigenvalues $\lambda_n, \, \dots$} \end{aligned} \end{equation*}
from which it follows that
\begin{equation*} \sup_{\substack{x \in E_{n-1}^\perp \\[0.1em] \|x\|=1}} \left( A x, \, x \right) = \lambda_n = \inf_{\substack{x \in E_n \\[0.1em] \|x\|=1}} \left( A x, \, x \right). \end{equation*}
If we take the infimum of the left-hand side and the supremum of the right-hand side, we obtain the first inequality:
\begin{equation*} \inf_{\substack{E \subseteq H \\[0.1em] \mathrm{codim}(E) < n}}  \, \sup_{\substack{x \in E \\[0.1em] \|x\|=1}} \left(A x, \, x \right) \leq  \lambda_n  \leq \sup_{\substack{E \subseteq H \\[0.1em] \mathrm{dim}(E) \geq n}}  \, \inf_{\substack{x \in E \\[0.1em] \|x\|=1}} \left( Ax, \, x \right). \end{equation*}
Let $E \subseteq H$ be a subspace of codimension strictly less than $n$, and let $F \subseteq H$ be a subspace of dimension at least $n$. It follows easily that
\begin{equation*} \begin{aligned} & E \cap E_n \cap \partial B(0, \, 1) \neq \varnothing \\[1em] & F \cap E_{n-1}^\perp \cap \partial B(0, \, 1) \neq \varnothing, \end{aligned} \end{equation*}
and thus there exist points $x_0$ in the first intersection and $y_0$ in the second intersection. We conclude the proof by noticing that
\begin{equation*} \begin{aligned} \sup_{\substack{x \in E \\[0.1em] \|x\|=1}} \left( Ax, \, x \right) & \geq \left(A x_0, \, x_0\right) \geq
\\[1em] & \geq \min_{\substack{x \in E_n \\[0.1em] \|x\|=1 }} \left( Ax, \, x \right) = \lambda_n = 
\\[1em] & = \max_{\substack{x \in E_{n-1}^\perp \\[0.1em] \|x\|=1}} \left( Ax, \, x \right) \geq
\\[1em] & \geq \left(A y_0, \, y_0 \right) \geq \inf_{\substack{x \in F \\[0.1em] \|x\|=1}} \left( Ax, \, x \right). \end{aligned} \end{equation*}
\end{proof}

\begin{theorem}[Alternating Eigenvalues Principle] Let $A \in \mathcal{L}_c^{\mathrm{sym}}(H)$, and let $H_0 \subset H$ be a closed hyperplane of $H$. Let $P_0 : H \longrightarrow H_0$ be the projection, and set $A_0 := P_0 \circ A \, \big|_{H_0} : H_0 \longrightarrow H_0$. Then $A_0 \in \mathcal{L}_c^{\mathrm{sym}}(H_0)$, and for any $n \in I$
\begin{equation*} \lambda_{n+1}(A) \leq \lambda_{n}(A_0) \leq \lambda_n(A).\end{equation*} \end{theorem} 

\begin{proof} It is a simple consequence of the \hyperref[c-f-t]{Courant-Fisher Theorem \ref{c-f-t}}. More precisely, it turns out that
\begin{equation*} \begin{aligned} \lambda_{n+1}(A) & =  \sup_{\substack{E \subseteq H \\[0.1em] \mathrm{dim}(E) \geq n + 1}}  \, \inf_{\substack{x \in E \\[0.1em] \|x\|=1}} \left( Ax, \, x \right) \leq
\\[1em] & \leq  \sup_{\substack{E \subseteq H \\[0.1em] \mathrm{dim}(E \cap H_0) \geq n}}  \, \inf_{\substack{x \in E \\[0.1em] \|x\|=1}} \left( Ax, \, x \right) \leq 
\\[1em] & \leq  \sup_{\substack{E \subseteq H \\[0.1em] \mathrm{dim}(E \cap H_0) \geq n}}  \, \inf_{\substack{x \in E \cap H_0 \\[0.1em] \|x\|=1}} \left( Ax, \, x \right) =
\\[1em] & =  \sup_{\substack{E \subseteq H_0 \\[0.1em] \mathrm{dim}(E) \geq n}}  \, \inf_{\substack{x \in E \\[0.1em] \|x\|=1}} \left( Ax, \, x \right) = 
\\[1em] & = \lambda_n(A_0) \leq  \sup_{\substack{E \subseteq H \\[0.1em] \mathrm{dim}(E) \geq n}}  \, \inf_{\substack{x \in E \\[0.1em] \|x\|=1}} \left( Ax, \, x \right) = \lambda_{n}(A).\end{aligned}\end{equation*}\end{proof}

\section{Spectral Theory of Banach Spaces}

Let $X$ be a Banach space. The space of linear continuous operators $\left(\La(X), \, +, \, \circ \right)$ is an algebra satisfying the inequality
\begin{equation} \label{ineqsdsd} \| A \circ B \| \leq \|A\| \cdot \|B\|, \end{equation}
for any $A, \, B \in \La(X)$.

\begin{definition}[Banach Algebra] \index{Banach algebra} The normed space $\left(X, \, \| \cdot \|_X \right)$ is a \textit{Banach algebra} if the following properties are satisfied: \mbox{}
\begin{enumerate}[label=\textbf{(\alph*)}]
\item $X$ is a Banach space.
\item There are $+ : X \times X \longrightarrow X$ and $\cdot : \mathbb{K} \times X \longrightarrow X$ such that $\left(X, \, +, \, \cdot\right)$ is an associative algebra over $\mathbb{K}$.
\item For any $x, \, y \in X$, it turns out that
\begin{equation*} \| x \cdot y \| \leq \|x\| \, \|y\|. \end{equation*}
\end{enumerate}\end{definition}

\begin{notation}The \textit{resolvent set}\index{resolvent set} associated to a linear operator $A \in \La(X)$ is defined as the set of complex numbers such that $A - \lambda\cdot \I$ is invertible, that is,
\begin{equation*} \rho(A) := \left\{ \lambda \in \C \: \left| \: \text{$A - \lambda \cdot \I$ is invertible} \right. \right\}. \end{equation*}
The \textit{spectrum} is the complement of the resolvent, and it is denoted by
\begin{equation*} \sigma(A) := \C \setminus \rho(A). \end{equation*}
The \textit{resolvent operator}, defined on $\rho(A)$, it denoted by $R(\lambda, \, A)$ and it is defined by setting
\begin{equation*} R(\lambda, \, A) := \left(\lambda \cdot \I - A \right)^{-1}.\end{equation*}
Finally, the \textit{spectral radius} is defined as the supremum of $\sigma(A)$, i.e.,
\begin{equation*} r_A := \sup_{\lambda \in \sigma(A)} |\lambda|. \end{equation*}
\end{notation}

\begin{lemma} Let $X$ be a Banach space, and let $A \in \La(X)$ be a linear continuous operator on $X$. \mbox{}
\begin{enumerate}[label=\textbf{(\arabic*)}]
\item The resolvent $\rho(A)$ is an open subset of $\C$.
\item The spectral radius $r_A$ is less or equal than $\|A\|$.
\item The map defined by the resolvent operator
\begin{equation*} \rho(A) \ni \lambda \mapsto R(\lambda, \, A) =: R(\lambda) \in \La(X) \end{equation*}
is analytic. If $\lambda > \|A\|$, then its power series is given by
\begin{equation*} R(\lambda) = \sum_{i = 0}^{+ \infty} \lambda^{-i-1} \, A^i. \end{equation*}
\item  The map defined by the resolvent operator admits a local representation as
\begin{equation*} R(\lambda) = \sum_{i = 0}^{+\infty} \left(\lambda - \lambda_0\right)^i \, R(\lambda_0)^{i+1}, \qquad \forall \lambda \in B \left( \lambda_0, \, \|R(\lambda_0)\|^{-1} \right). \end{equation*}
\item \textbf{Resolvent Identity.} For any $\mu, \, \lambda \in \rho(A)$, it turns out that
\begin{equation} \label{residen} R(\lambda) - R(\mu) = (\mu - \lambda) \, R(\mu) \, R(\lambda). \end{equation}
\end{enumerate}\end{lemma}

The central goal of this section is to prove that the spectral radius is equal to the limit of the sequence $\|A^n\|^{1/n}$. We will give two proofs of this fact: the first one relies on holomorphic function theory, while the second one relies on a variational argument (Palais-Smale sequences).

\begin{lemma}\label{sequencelemma} Let $(a_n)_{n \in \N} \subset \R$ be a subadditive sequence ($a_{m+n} \leq a_m + a_n$). Then the limit of the sequence $\left(1/n \cdot a_n \right)_{n \in \N}$ exists and it is equal to the infimum, that is,
\begin{equation*}\lim_{n \to + \infty} \left( \frac{1}{n} \, a_n \right) = \inf_{n \in \N} \left( \frac{1}{n} \, a_n \right). \end{equation*} \end{lemma}

\begin{proof}Let $d$ be a positive natural number. For any $n \in \N$ there are $p \in \N$ and $k \in \{0, \, \dots, \, d-1\}$ such that
\begin{equation*} n = p \cdot d + k, \qquad \text{where} \, \, p = \left\lfloor \frac{n}{d} \right\rfloor. \end{equation*}
For every $n \in \N$ it turns out that
\begin{equation*} \begin{aligned} \inf_{m \in \N}\left( \frac{1}{m} \, a_m \right) & \leq \frac{a_n}{n} \leq \frac{ a_{p\cdot d} + a_k}{n} \leq \\[1em] & \leq \frac{1}{n} \, \left( p \cdot a_d + a_k \right) = \left( \frac{1}{d} + \mathcal{O}\, (1) \right) \cdot a_d + \mathcal{O} \, (1) \, =  \\[1em] & = \frac{a_d}{d} + \mathcal{O} \, (1).\end{aligned} \end{equation*}
Thus we can take the superior limit of the left-hand side of the inequality above
\begin{equation*} \inf_{m \in \N}\left( \frac{1}{m} \, a_m \right) \leq \limsup_{n \to + \infty}\left( \frac{1}{n} \, a_n \right) \leq \frac{a_d}{d} \implies \inf_{n \in \N} \left( \frac{1}{n} \, a_n \right) = \lim_{n \to + \infty} \left( \frac{1}{n} \, a_n \right), \end{equation*}
and this concludes the proof.
\end{proof}

\begin{theorem}[Spectral Radius] \index{spectral radius} \label{srad}Let $A \in \La(X)$ be any linear continuous operator defined on a Banach space $X$. The spectral radius of $A$ is given by
\begin{equation*} r_A = \lim_{n \to + \infty} \|A^n\|^{\frac{1}{n}} = \inf_{n \in \N} \|A^n\|^{\frac{1}{n}}. \end{equation*}
\end{theorem}

\begin{proof}[Holomorphic Proof] Let us consider the sequence $a_n = \log \left( \|A^n\| \right)$.

\paragraph{Step 1.} The logarithm is subadditive; thus we can apply \hyperref[sequencelemma]{Lemma \ref{sequencelemma}} and infer that
\begin{equation*} \lim_{n \to + \infty} \frac{1}{n} \cdot \log \left( \|A^n\| \right) =  \inf_{n \to + \infty} \frac{1}{n} \cdot \log \left( \|A^n\| \right), \end{equation*}
that is, the second identity is true:
\begin{equation*}\lim_{n \to + \infty} \|A^n\|^{\frac{1}{n}} = \inf_{n \in \N} \|A^n\|^{\frac{1}{n}}. \end{equation*}
It remains to prove that this quantities coincide with the spectral radius $r_A$ of $A$.

\paragraph{Step 2.} One inequality is trivial since $\lambda \in \sigma(A)$ implies $\lambda \cdot \I - A$ not invertible; hence, for every $n \in \N$, the operator $\lambda^n \cdot \I - A^n$ is not invertible, and this easily implies that
\begin{equation*} \text{$\lambda^n \in \sigma \left(A^n \right)$ and $r_{A^n} \leq \|A^n\|$} \implies |\lambda|^n \leq \|A^n\| \implies |\lambda| \leq \|A^n\|^{\frac{1}{n}}. \end{equation*}
Vice versa, let us fix $x \in X$, $x^\ast \in X^\ast$ and let us consider the \textbf{holomorphic} function
\begin{equation*} \C \supseteq \left\{ t \in \C \: \left| \: \left| t \right| < \frac{1}{r_A} \right. \right\} \ni t \longmapsto \left< x^\ast, \, \left(\I - t \cdot A \right)^{-1} \, x \right> \in \C, \end{equation*}
where $< \cdot, \, \cdot>$ is the duality $(X, \, X^\ast)$. Consequently, the series
\begin{equation*}\Phi(t) = \sum_{k = 0}^{+ \infty} \left< x^\ast, \, A^k x \right> \, t^k \end{equation*}
converges at any point $x \in B$, where $B$ is the maximal ball contained in the domain. Therefore, when $r > r_A$, it turns out that
\begin{equation*} \sup_{k \in \N} \, \left| \left< x^\ast, \, \left( \frac{A}{r} \right)^k x \right> \right| \leq C < + \infty \end{equation*}
for every $x \in X$ and $x^\ast \in X^\ast$. By \hyperref[stein]{Banach-Steinhaus Theorem \ref{stein}} it turns out that $\| (A/r)^k \|$ is bounded, and thus
\begin{equation*} \left\|A^k\right\|^{\frac{1}{k}} \leq c \cdot r \implies \inf \left\|A^k\right\|^{\frac{1}{k}} \leq r \end{equation*}
for every $r > r_A$, which is enough to conclude that $\inf \left\|A^k\right\|^{\frac{1}{k}} \leq r_A$.
\end{proof}

\begin{remark}The same result holds true in the more general setting of Banach algebras.\end{remark}

\begin{theorem}[Spectral Map] \index{spectral map theorem} \label{theorem:sm}Let $p(z) \in \C[z]$ be a complex polynomial of degree $n$, and let $A \in \La(X)$ be a linear continuous operator defined on a Banach space $X$. Then
\begin{equation*} p \left( \sigma(A) \right) = \sigma \left( p(A) \right), \end{equation*}
where $ p \left( \sigma(A) \right) = \left\{ p(\lambda) \: \left| \: \lambda \in \sigma(A) \right. \right\}$. \end{theorem}

\begin{proof} Suppose that $p(z)$ is a monic polynomial. For any $\lambda \in \C$, we can factorize the polynomial $p(z) - \lambda$ as follows:
\begin{equation*} p(z) - \lambda = \prod_{i = 1}^{n} \left(z - \mu_j^{(\lambda)} \right), \end{equation*}
where $\mu_j^{(\lambda)}$ are the roots of the polynomial, eventually repeated. Clearly, by construction, we have that
\begin{equation*}p(\mu) = \lambda \iff \exists \, j \in \{1, \, \dots, \, n\} \: : \: \mu = \mu_j^{(\lambda)}. \end{equation*}
Therefore $\lambda \in \sigma \left(p(A) \right)$ if and only if $p(A) - \lambda \cdot \I$ is not invertible if and only if there is a factor which is not invertible, that is, there exists $j \in \{1, \, \dots, \, n \}$ such that $A - \mu_j^{(\lambda)} \cdot \I$ is not invertible. \end{proof}

\paragraph{Symmetric Operators.} Let $H$ be a Hilbert space, and let $A \in \Ls(H)$ be a symmetric continuous linear operator. We can introduce a partial ordering on $\Ls(H)$ by setting
\begin{equation*} A \geq 0 \iff \left< A x, \, x \right> \geq 0 \qquad \forall \, x \in H, \end{equation*}
from which it follows that $A \geq B$ if and only if $A - B \geq 0$.

\begin{proposition} \label{prop:symesd}Let $H$ be a Hilbert space, and let $A \in \Ls(H)$ be a symmetric operator. Then
\begin{equation*} \begin{aligned} & m_A : = \inf_{\|x\|=1} \left\langle A x, \, x \right\rangle = \min \sigma(A), \\[1em] & M_A := \sup_{\|x\|=1} \left\langle A x, \, x \right\rangle = \max  \sigma(A). \end{aligned} \end{equation*} \end{proposition}

\begin{proof} First, we observe that any positive operator $A \in \Ls(H)$ (i.e., such that $A \geq 0$) has the following property: the operator $\I + A$ is invertible or, equivalently, minus one is not an element of the spectrum $\sigma(A)$. Indeed, for every $x \in H$ it turns out that
\begin{equation*} \left\| (\I + A)(x) \right\|^2 = \|x\|^2 + \underbrace{\|A x\|^2}_{\geq 0} + 2 \, \underbrace{\left< A x, \, x \right>}_{\geq 0} \geq \|x\|^2 , \end{equation*}
and hence by \hyperref[rmk:injclos]{Remark \ref{rmk:injclos}} it is injective and with closed rank. By \hyperref[prop:ranksym]{Proposition \ref{prop:ranksym}} it implies that the operator is injective and surjective, that is, bijective. In particular, if $t < m_A$, then
\begin{equation*} -1 \notin \sigma \left( \frac{A - m_A}{m_A- t} \right) \end{equation*}
since it is a positive symmetric operator, and thus $A - m_A \cdot \I + (m_A - t) \cdot \I = A - t \cdot \I$ is invertible. If we replace $m_A$ with $M_A$, a similar argument proves that
\begin{equation*} \text{$t < m_A$ or $t > M_A$} \implies t \notin \sigma(A) \implies \sigma(A) \subseteq [m_A, \, M_A]. \end{equation*}
It remains to prove that the extremal points $m_A$ and $M_A$ are actually attained, i.e., they belong to $\sigma(A)$. Clearly, it is enough to prove it for $M_A$ (we obtain $m_A$ by replacing $A$ with $-A$). Let
\begin{equation*} a(u, \, v) := (M_A u - Au, \, v), \end{equation*}
and notice that $a(\cdot, \, -)$ is symmetric and such that
\begin{equation*} a(u, \, u) \geq 0 \quad \text{for all $u \in H$}. \end{equation*}
It follows from the Cauchy-Schwartz inequality that
\begin{equation*} |a(u, \, v)| \leq a(u, \, u)^{\frac{1}{2}}a(v, \, v)^{\frac{1}{2}} \quad \text{for all $u, \, v \in H$}, \end{equation*}
and this implies that
\begin{equation*} \left| M_A u - A u \right| \leq C \cdot a(u, \, u)^{\frac{1}{2}} \quad \text{for all $u \in H$}. \end{equation*}
Let $(u_n)_{n \in \N} \subset H$ be a sequence such that
\begin{equation*} \|u_n\| = 1 \quad \text{and} \quad (Au_n, \, u_n) \xrightarrow{n \to + \infty} M_A, \end{equation*}
and notice that the estimate above yields to
\begin{equation*} \left| M_A u_n - A u_n \right| \xrightarrow{n \to + \infty} 0. \end{equation*}
We infer that $M_A \in \sigma(A)$ since, if $M_A \in \rho(A)$, then
\begin{equation*} u_n = \left( M_A - A \right)^{-1} (M_A u_n - A u_n) \xrightarrow{n \to + \infty} 0, \end{equation*}
and this is impossible because $\| u_n \| = 1$ for all $n \in \N$. In conclusion, we want to prove that the maximum between $|M_A|$ and $|m_A|$ is equal to $\|A\|$. A simple computation shows that
\begin{equation*} \|A\|^2 = \sup_{\|x\| = 1} \left\|A x \right\|^2 = \sup_{\|x\|= 1} \left< A x, \, Ax \right> = \sup_{\|x\| = 1} \left< A^2 x, \, x \right> = \|A^2\|. \end{equation*}
In particular, we can consider the subsequence $(2^n)_{n \in \N} \subset (n)_{n \in \N}$ and observe that
\begin{equation*} \|A\| = \left\|A^{2^n} \right\|^{\frac{1}{2^n}} \implies r_A = \lim_{n \to + \infty} \|A^n\|^{\frac{1}{n}} = \lim_{n \to + \infty}  \left\|A^{2^n} \right\|^{\frac{1}{2^n}} \implies \max \{ |M_A|, \, |m_A| \} = r_A = \|A\|. \end{equation*}\end{proof}

\paragraph{Variational Approach to Spectral Radius.} Let $A \in \Ls(H)$ be a symmetric operator defined on a Hilbert space $H$. The couple $\left(x, \, \lambda \right) \in H \times \C$ is an eigenvector-eigenvalue couple if and only if $x$ is a critical point of the Rayleigh quotient $f_A$ and $\lambda = f_A(x)$.

\begin{lemma}Let $A \in \Ls(H)$ be a symmetric operator defined on a Hilbert space $H$. A complex number $\lambda$ is an eigenvalue of $A$ if and only if there exists a sequence $(x_n)_{n \in \N} \subset \partial \, B_X(0, \, 1)$ such that $(x_n)_{n \in \N} \in \left( \mathrm{PS} \right)_\lambda$, i.e., it is a Palais-Smale sequence at the level $\lambda$. \end{lemma}

\begin{proof}The details are left to the reader. The main idea is to prove that Rayleigh quotient $f_A$ is lower semi-continuous and that it satisfies the Palais-Smale condition. \end{proof}

Behind this Lemma, there is the following general variational principle (named after \textbf{Ekeland}).

\begin{theorem}[Ekeland] Let $f : X \to \R$ be a lower semi-continuous function, and let $X$ be a complete metric space. If $\inf_{x \in X}f(x) > - \infty$, then for any $x_0 \in X$ and $\delta, \, \epsilon > 0$ such that
\begin{equation*}f(x_0) < \inf_{x \in X}f(x) + \epsilon,\end{equation*}
there exists $y \in X$ satisfying the following properties: \mbox{}
\begin{enumerate}[label=\textbf{(\alph*)}]
\item $d(y, \, x_0) < \delta$;
\item $\displaystyle\sup_{ \substack{u \neq y \\ u \in X}} \frac{f(y) - f(u)}{d(y, \, u)} < \frac{\epsilon}{\delta}$.
\end{enumerate} \end{theorem}

Therefore, if $(x_n)_{n \in \N} \subset H$ is a minimizing sequence, then there exists a sequence $(y_n)_{n \in \N}$, equivalent in the sense of Cauchy, such that $(y_n)_{n \in \N} \left( \mathrm{PS} \right)_{\inf f}$.

\section{Exercises}

\begin{exercise} Let $k \in L^2 \left([0, \, 1] \times [0, \, 1] \right)$. The integral operator
\begin{equation*}T_k :  L^2 \left([0, \, 1] \times [0, \, 1] \right) \longrightarrow L^2 \left([0, \, 1] \right), \end{equation*}
defined by setting
\begin{equation*}T_k(u)(x) := \int_0^1 k(x, \, y) u(x, \, y) \, \mathrm{d}y \end{equation*}
is well-defined, linear, continuous and compact. \end{exercise}

\begin{proof}[Solution] First, we notice that the operator $T_k$ is well-defined as a consequence of both the Fubini theorem and the Hölder inequality. Namely, a simple computation shows that
\begin{equation*} \begin{aligned}\|T_k(u)\|_{L^2([0, \, 1])}^2 & = \int_0^1 \left| \int_0^1 k(x, \, y) u(x, \, y) \, \mathrm{d}y  \right|^2 \mathrm{d}x \leq
\\[1em] & \leq \int_0^1 \|k(x, \, \cdot)\|_{L^2([0, \, 1])}^2 \|u(x, \, \cdot)\|_{L^2([0, \, 1])}^2 \, \mathrm{d}x \leq
\\[1em] & \leq \left[ \left\| \|k(x, \, \cdot) \|_{L^2([0, \, 1])} \right\|_{L^2([0, \, 1])} \left\| \|u(x, \, \cdot)\|_{L^2([0, \, 1])} \right\|_{L^2([0, \, 1])} \right]^2 =
\\[1em] & = \|k\|_{L^2([0, \, 1] \times [0, \, 1])}^2 \|u\|_{L^2([0, \, 1] \times [0, \, 1])}^2 < + \infty,\end{aligned} \end{equation*}
which means that $T_k$ sends $L^2 \left([0, \, 1] \times [0, \, 1] \right)$ in $L^2 \left([0, \, 1] \right)$, and the operator norm of $T_k$ is less than or equal to the $L^2$-norm of $k$.

\paragraph{Compactness} The operator $T_k$ is clearly linear, and therefore the computation above shows that it is also continuous (since it is bounded). To prove that $T_k$ is compact, we shall show that there is a sequence $(T_k^n)_{n \in \N}$ of finite-rank operator such that
\begin{equation*} \lim_{n \to + \infty} \left\| T_k - T_k^n \right\| = 0. \end{equation*}
Let $\{ \phi_i \}_{i \in \N}$ be a orthonormal basis of $L^2 \left([0, \, 1]\right)$. Then one can easily show that
\begin{equation*} \left\{ \phi_i \phi_j \right\}_{i, \, j \in \N} \end{equation*}
is a orthonormal basis of the product space $L^2 \left([0, \, 1] \times [0, \, 1]\right)$. In particular, if we set
\begin{equation*} k_{i, \, j} := \iint_{[0, \, 1]^2} k(x, \, y) \phi_i(x) \phi_j(y) \, \mathrm{d}x \mathrm{d}y \quad \text{for all $i, \, j \in \N$},\end{equation*}
then the kernel $k$ is given by the following sum:
\begin{equation*} k(x, \, y) = \sum_{i = 1}^{+ \infty} \sum_{j = 1}^{+ \infty} k_{i, \, j} \phi_i(x) \phi_j(y).\end{equation*}
We now define the partial kernel
\begin{equation*} k_n(x, \, y) := \sum_{i = 1}^{n} \sum_{j = 1}^{+ \infty} k_{i, \, j} \phi_i(x) \phi_j(y)\end{equation*}
and the associated operator
\begin{equation*}T_k^n(u)(x) := \int_0^1 k_n(x, \, y) u(x, \, y) \, \mathrm{d}y. \end{equation*}
The operator $T_k^n$ maps $L^2([0, \, 1] \times [0, \, 1])$ into a finite-dimensional linear subspace of $L^2([0, \, 1])$, and therefore $T_k^n$ is a finite-rank operator. In conclusion, note that
\begin{equation*} \begin{aligned}\|T_k(u) - T_k^n(u) \|_{L^2([0, \, 1])}^2 & = \left\| \int_0^1 k_n(x, \, y) u(x, \, y) \, \mathrm{d}y - \int_0^1 k_n(x, \, y) u(x, \, y) \, \mathrm{d}y \right\|_{L^2([0, \, 1])}^2  \leq
\\[1em] & \leq \|k - k_n\|_{L^2([0, \, 1] \times [0, \, 1])}^2 \|u\|_{L^2([0, \, 1] \times [0, \, 1])} =
\\[1em] & = \left[ \sum_{i = n + 1}^{+ \infty} \sum_{j = 1}^{+ \infty} |k_{i, \, j}|^2 \right] \|u\|_{L^2([0, \, 1] \times [0, \, 1])},\end{aligned} \end{equation*}
and the last term goes to zero as $n \to + \infty$ since the sum is finite
\begin{equation*}\sum_{i = 1}^{+ \infty} \sum_{j = 1}^{+ \infty} |k_{i, \, j}|^2 = \|k\|_{L^2([0, \, 1] \times [0, \, 1])}^2 < + \infty.\end{equation*}
\end{proof}

\begin{exercise} Let $\lambda \in c_b$ be a bounded sequence and let us consider the operator
\begin{equation*}T_\lambda : \ell_2(\N) \longrightarrow \ell_2(\N) \end{equation*}
defined by setting
\begin{equation*} \mathbf{x} := (x_1, \, x_2, \, \dots) \longmapsto (\lambda_1 x_1, \, \lambda_2 x_2, \dots ) =: \lambda \cdot \mathbf{x}. \end{equation*}
Prove that
\begin{equation*} \|T_\lambda \| = \|\lambda\|_{\infty}, \end{equation*}
and $T_\lambda$ is compact if and only if $\lambda \in c_0$. \end{exercise}

\begin{proof}[Solution] The operator norm $\|T_\lambda \|$ can be computed by standard means. Here we only show that $T_\lambda$ is compact if and only if $\lambda$ is an infinitesimal sequence.

\paragraph{$"\implies"$} This implication is left as an easy exercise for the reader. The main idea is to use the definition of compact operator, choose an adequate sequence in $(\mathbf{z}^n)_{n \in \N} \subset \ell_2(\N)$, and show that $\lambda$ must be infinitesimal for $T(\mathbf{z}^n)$ to have a converging subsequence.

\paragraph{$"\impliedby"$} To prove that $T_\lambda$ is compact, we simply need to show that it is the limit (in the operator norm) of a sequence of finite-rank operators $T_\lambda^n$. Let us define
\begin{equation*} \lambda^n := (\lambda_1, \, \lambda_2, \, \dots, \, \lambda_n, \, 0, \, \dots) = \sum_{i = 1}^n \lambda_i e_i \in c_0, \end{equation*}
and consider the operator
\begin{equation*}T_\lambda^n (\mathbf{x}) := \lambda^n \cdot \mathbf{x}. \end{equation*}
The operator $T_\lambda^n$ maps $\ell_2(\N)$ into the finite-dimensional linear subspace of $\ell_2(\N)$ spanned by $e_1, \, \dots, \, e_n$, and therefore $T_\lambda^n$ is a finite-rank operator. In conclusion, note that
\begin{equation*} \begin{aligned}\|T_\lambda(\mathbf{x}) - T_\lambda^n(\mathbf{x}) \|_{\ell_2(\N)}^2 & = \left\| \lambda \cdot \mathbf{x} - \lambda^n \cdot \mathbf{x} \right\|_{\ell_2(\N)}^2  \leq
\\[1em] & \leq \|\lambda - \lambda^n \|_{\infty}^2 \|\mathbf{x}\|_{\ell_2(\N)}^2 =
\\[1em] & = \sup_{i > n} |\lambda_i|^2 \|\mathbf{x}\|_{\ell_2(\N)}^2,\end{aligned} \end{equation*}
and the last term goes to zero as $n \to + \infty$ since the $\lambda$ is an infinitesimal ($c_0$) sequence by assumption.
\end{proof}

\begin{exercise}\label{shiftspectrum}Let $S : \ell_2(\N) \longrightarrow \ell_2(\N)$ be the \textit{shift} operator defined by
\begin{equation*}(x_1, \, x_2, \, \dots) \longmapsto (x_2, \, x_3, \, \dots). \end{equation*}
Prove that $\sigma(S)$ is equal to $\overline{B_\C(0, \, 1)}$, but none of them is an eigenvalue. \end{exercise}

\begin{proof}[Solution] The spectrum $\sigma(S)$ is contained in the closure of the ball of center zero and radius $1$ because the spectral radius of $S$ is equal to $1$. The opposite implication is left to the reader.

\paragraph{Eigenvalues.} We argue by contradiction. Suppose that $S$ admits an eigenvalue $\lambda \in \C \setminus \{0\}$ with eigenstate $\mathbf{x} \in \ell_2(\N)$. Then
\begin{equation*}(\lambda x_1, \, \lambda x_2, \, \dots) = ( x_2, \, x_3, \, \dots), \end{equation*}
which means that
\begin{equation*} x_i = \frac{x_{i + 1}}{\lambda} \quad \text{for all $i = 1, \, \dots$} \end{equation*}
In particular, the sequence $\mathbf{x}$ is uniquely determined by the value of the first coefficient $x_1$, i.e.
\begin{equation*}\mathbf{x} = (x_1, \, \frac{x_1}{\lambda}, \, \frac{x_1}{\lambda^2}, \, \dots), \end{equation*}
and therefore $\mathbf{x}$ cannot be an element of $\ell_2(\N)$ since
\begin{equation*}\| \mathbf{x} \|_{\ell_2(\N)} = \sum_{i = 1}^{+ \infty} \left| \frac{x_1}{\lambda^{i-1}} \right|^2 = |x_1|^2 \sum_{i = 0}^{+ \infty} |\lambda^{-i}|^2 = + \infty \end{equation*}
for all values of $\lambda \in \C \setminus \{0\}$ such that $|\lambda| \leq 1$, and this is exactly what we wanted to prove.
\end{proof}

\begin{exercise}\label{zerospectrum}Let $X := C^0\left([0, \, 1] \right)$, and let us consider the Volterra operator $T : C^0\left([0, \, 1] \right) \to C^0\left([0, \, 1] \right)$, defined by
\begin{equation*} u \longmapsto \int_{0}^x u(t) \, \mathrm{d}t. \end{equation*}
Prove that $T$ is a compact operator with no eigenvalues, i.e.,
\begin{equation*} \sigma_e(T) = \{0\}. \end{equation*}
Furthermore, find an explicit formula for the inverse operator $\left( \lambda \cdot \mathrm{id}_X - T\right)^{-1}$, as $\lambda$ ranges in the resolvent set $\rho(T)$.\end{exercise}

\begin{exercise}Let $H = L^2(X, \, \Sigma, \, \mu)$, and let $f \in L^\infty(X, \, \Sigma, \, \mu)$. The operator
\begin{equation*} M_f : H \longrightarrow H, \qquad g \mapsto f g\end{equation*}
is linear and continuous. Prove that: \mbox{}
\begin{enumerate}[label=\textbf{(\alph*)}]
\item The norm operator of $M_f$ is equal to $\|f\|_\infty$.
\item The spectrum of $M_f$ is given by
\begin{equation*} \sigma(M_f) = \bigcap_{ \text{$N$ null-set} } \overline{ g(X \setminus N) }, \end{equation*}
where $g \in [f]$ is any function in the equivalence class. More precisely, $\lambda \in \sigma(M_f)$ if and only if for any $\epsilon > 0$ the support of $g_\ast(\mu)$ intersects $B(\lambda, \, \epsilon)$ in a set of positive measure.
\item The eigenvalues spectrum of $M_f$ is given by
\begin{equation*} \sigma_e(M_f) = \left\{ \lambda \in \C \setminus \{0\} \: \left| \: \mu (f^{-1}(\lambda)) \neq 0 \right. \right\}. \end{equation*}
\end{enumerate} \end{exercise}
