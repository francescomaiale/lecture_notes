\chapter{Banach-Alaoglu and Closed Rank Theorem} \thispagestyle{empty}

In this chapter, we exploit the results we have proved so far to define a locally convex topology on a Banach space $X$ (which is weaker than the topology induced by the norm.)

In a similar fashion, we show that the dual space $X^\ast$ may be endowed with three (eventually) different topologies: the one induces by the dual norm, the weak one, and the weak-$\ast$ one.

In the second half of the chapter, we investigate some of the main properties of the weak-$\ast$ topology, e.g. the Banach-Alaouglu theorem (which asserts that the ball is weakly-$\ast$ compact.)

\section{Initial Topology}

Let $\mathscr{X}$ be a vector space, and let $\mathscr{F} \subset \mathscr{X}^\prime$ be a subfamily of the \textbf{algebraic dual}\footnote{\textbf{Definition.}\index{algebraic dual} Let $X$ be a vector space. The algebraic dual of $X$, denoted by $X^\prime$, is the set of all the linear functional $\varphi : X \longrightarrow \R$ - since a priori there is no topology on $X$.}. Notice that every linear functional induces a seminorm, defined by
\begin{equation*}\left| \varphi \right|(x) := \left| \varphi(x) \right|, \end{equation*}
and therefore the collection
\begin{equation*} \mathcal{P} :=  \left\{\left| \varphi \right| \: : \: \varphi \in \mathscr{F} \right\} \end{equation*}
generates a locally convex topology on $\mathscr{X}$. More precisely, let
\begin{equation*} \mathcal{B}_\varphi^r := \left\{x \in X \: \left| \: \left| \varphi(x) \right| < r \right. \right\} \end{equation*}
denote the ball associated to $\varphi \in \mathcal{P}$. The enlarged family, which is defined by
\begin{equation*} \widetilde{\mathcal{P}} := \left\{ \max_{i \in S} \left| \varphi_i \right| \: \left| \: S \subset \mathscr{F} \, \, \text{finite subset} \right. \right\}, \end{equation*}
has the following property: \textit{Every ball associated to an element of $\widetilde{\mathcal{P}}$ is equal to a finite intersection of balls associated to elements of $\mathcal{P}$}.

The family of seminorms $\mathcal{P}$ induces on $\mathscr{X}$ a locally convex topology $\tau$, which is the coarsest topology that makes any $\varphi \in \F$ continuous.

\begin{example}Let us consider a collection $(X_i, \, \tau_i)_{i \in I}$ of topological vector spaces, and let us denote by $X$ the product of the $X_i$'s, that is,
\begin{equation*}X := \prod_{i \in I} X_i. \end{equation*}
Our goal is to define a topology $\tau$ on $X$ that makes it a topological vector space. The natural way to do it is to require that a map $f : Y \longrightarrow X$ is continuous if and only if the components $p_i \circ f : Y \longrightarrow X_i$ are $\tau_i$-continuous for every $i \in I$, where
\begin{equation*} p_i : X := \prod_{i \in I} X_i \longrightarrow X_i \end{equation*}
denotes the $i$th canonical projection. Furthermore, one can easily prove that
\begin{equation*} \text{$(X_i, \, \tau_i)$ is locally convex for all $i \in I$} \implies \text{$(X, \, \tau)$ is locally convex.} \end{equation*}
For example, both the sum and the scalar product on $X$ are $\tau$-continuous operations since the composition with each projection $p_i$ is continuous.\end{example}

Back to the general case, we observe that the \textit{initial topology} $\tau$, generated by the family $\mathcal{P}$, is defined in such a way that it is the \textbf{coarsest} topology making every $\varphi \in \mathscr{F}$ continuous.

\begin{proposition}Let $\mathscr{X}$ be a vector space, let $\mathscr{F} \subset \mathscr{X}^\prime$ be a subfamily of the algebraic dual, and let $\tau$ be the initial topology\index{initial topology} relative to $\mathscr{F}$. Then continuous linear forms of $(\mathscr{X}, \, \tau)$ are precisely the elements of $\mathscr{F}$ and all of their linear combinations, that is,
\begin{equation*}\mathscr{X}_\tau^\ast = \mathrm{Span} \langle f \: : \: f \in \mathscr{F} \rangle. \end{equation*}  \end{proposition}

The proof of this result is an immediate consequence of the following technical lemma.

\begin{lemma} \label{lemma:algebaricicicc} Let $X$ be a vector space, and let $f, \, f_1, \, \dots, \, f_n \in X^\prime$. Then the following properties are equivalent: \mbox{} 
\begin{enumerate}[label=\textbf{(\alph*)}]
\item There are real numbers $\lambda_1, \, \dots, \, \lambda_n \in \R$ such that
\begin{equation*} f = \sum_{i = 1}^n \lambda_i  f_i . \end{equation*}
\item There exists a positive constant $C > 0$ such that
\begin{equation*} |f(x)| \leq C \max_{i= 1, \, \dots, \, n} |f_i(x)| \quad \text{for all $x \in X$}. \end{equation*}
\item The kernel of $f$ contains the intersection of the kernels, that is,
\begin{equation*} \mathrm{ker} \, f \supseteq \bigcap_{i = 1}^n \mathrm{ker} \, f_i. \end{equation*}
\end{enumerate}\end{lemma}

\begin{proof} The unique nontrivial implication is $\mathbf{(c)} \implies \mathbf{(a)}$. In the linear algebra course we have proved that there exists a unique $\lambda : \R^n \longrightarrow \R$ such that the diagram below is commutative
\begin{equation*}\begin{tikzcd}
X \arrow{r}{F} \arrow{dr}{f} & \R^n \arrow[dotted]{d}{\lambda} \\
 & \R
\end{tikzcd} \end{equation*}
where $F(x) := \left(f_1(x), \, \dots, \, f_n(x) \right)$.

On the other hand, the mapping $\lambda : \R^n \longrightarrow \R$ is clearly linear and satisfying the identity $\lambda \circ F(x) = f(x)$. It follows that $f$ is a linear combination of the components of $F$, that is,
\begin{equation*} f(x) = \sum_{i = 1}^{n} \lambda_i f_i(x). \end{equation*}  \end{proof} 

In particular, if $f$ is continuous with respect to the initial topology $\tau$, then $f$ is bounded in a neighborhood of the origin of the form
\begin{equation*} \bigcap_{i = 1}^{n} \left\{ x \in X \: \left| \: |f_i(x)| < \epsilon_i \right. \right\} \end{equation*}
Therefore, it follows from \hyperref[lemma:algebaricicicc]{Lemma \ref{lemma:algebaricicicc}} that $f$ is a linear combination of a finite number of elements $f_i \in \mathscr{F}$, precisely $f_1, \, \dots, \, f_n$, that is,
\begin{equation*}\mathscr{X}_\tau^\ast = \mathrm{Span} \langle f \: : \: f \in \mathscr{F} \rangle. \end{equation*} 

\section{Elementary Properties of the Weak Topologies}

Let $(X, \, \| \cdot \|)$ be a Banach space, and let $\mathscr{F} := X^\ast$ be the topological dual, that is, the set of all the linear forms continuous with respect to the strong convergence $\tau_s$.

The initial topology associated to the family $\mathscr{F}$ is called \textit{weak topology}\index{weak topology} of $X$, and it is usually denoted by $\tau_w$. A natural question arises: \textit{The weak topology $\tau_w$ and the strong topology $\tau_s$ are actually different?}

\paragraph{Answer.} The reader may readily check that the dual spaces are the same, that is,
\begin{equation*} X_{\tau_w}^\ast = X_{\tau_s}^\ast, \end{equation*}
and also that both topologies are separated. Luckily, this is not enough for them to coincide and, actually, if $X$ is an infinite-dimensional Banach space, the strong topology and the weak topology are always different, as a consequence of the following result.

\begin{lemma}Let $U$ be a neighborhood of the origin in $\tau_w$. Then there exists a finite-codimensional subspace $H \subset X$ contained in $U$. \end{lemma}

\begin{proof} A subbasis of the neighborhoods of the origin in the weak topology $\sigma \left(X, \, X^\ast \right)$ is given by the sets of the form
\begin{equation*} W_{f, \, \epsilon} = \left\{ x \in X \: \left| \: \left| \left\langle f, \, x \right\rangle \right| < \epsilon \right. \right\} \quad \text{for $f \in X^\ast$ and $\epsilon > 0$}. \end{equation*}
In particular, given $U \in \mathcal{U}_0(X)$ neighborhood of the origin, one can always find $f_1, \, \dots, \, f_N \in X^\ast$ and $\epsilon_1, \, \dots, \, \epsilon_N > 0$ such that
\begin{equation*} U \supseteq \bigcap_{i = 1}^{N} W_{f_i, \, \epsilon_i}. \end{equation*}
Furthermore, we may always assume that the finite intersection is made up of \textbf{linearly independent} elements. The reader can easily check that the finite-dimensional subspace
\begin{equation*} H = \left\{ x \in X \: \left| \: \text{$f_i(x) = 0$ for all $i = 1, \, \dots, \, N$} \right. \right\} \end{equation*}
is contained in $U$, and it is thus the sought subspace. \end{proof}

\paragraph{Weak-$\ast$ Topology.}  Let $\left(X^\ast, \, \|\cdot\|_{\ast} \right)$ be the topological dual of a Banach space $X$, endowed with the operator norm. The valuations $j_x : X^{\ast} \longrightarrow \R$ form a linear subspace of the algebraic bidual $X^{\prime \prime}$, which may be identified with $X$ in the obvious way.

\begin{definition}\index{weak-$\ast$ topology} The initial topology on $X^\ast$ associated to the family
\begin{equation*} \mathscr{F} := \left\{ j_x : X^\ast \longrightarrow \R \: \left| \: x \in X \right. \right\}  \end{equation*}
is called \textit{weak-$\ast$ topology}, and it is usually denoted by $\tau_w^\ast$. \end{definition}

In particular, on the dual space $X^\ast$ one can define three remarkable topologies satisfying the following chain of inclusions:
\begin{equation*}\tau_{w}^\ast \subseteq \tau_w \subseteq \tau_s. \end{equation*}  

\begin{notation} In the literature, the initial topology on a vector space $X$ associated to a family $\mathscr{F}$ is usually denoted by $\sigma(X, \, \mathscr{F})$. Therefore, from now on, we denote by $\sigma \left(X, \, X^\ast \right)$ the weak topology on $X$, by $\sigma \left(X^\ast, \, X^{\ast\ast} \right)$ the weak topology on $X^\ast$, and by $\sigma(X^\ast, \, X)$ is the weak-$\ast$ topology on $X^\ast$. \end{notation}

\begin{definition}[Separated] \index{separated topology} Let $X$ be a vector space, and let $\mathscr{F}$ be a subfamily of the algebraic dual. The initial topology $\sigma(X, \, \mathscr{F})$ is \textit{separated} if and only if for every $x \in X$, $x \neq 0$, there is an element $f \in \mathscr{F}$ such that $\langle f, \, x \rangle \neq 0$.\end{definition}

\begin{remark}Equivalently, the initial topology $\sigma(X, \, \mathscr{F})$ is separated if and only if for every $x \neq 0$ there are a neighborhood $U$ of the origin and a positive real number $c > 0$ such that 
\begin{equation*} x \notin c \cdot U. \end{equation*}  \end{remark}

\begin{lemma}Let $X$ be a Banach space. The weak topology $\sigma(X, \, X^\ast)$ and the weak-$\ast$ topology $\sigma(X^\ast, \, X)$ are separated. \end{lemma}

\begin{proof} Let $x \in X \setminus \{0\}$. It follows from \hyperref[ex:13]{Example \ref{ex:13}} that one can always find a continuous linear functional $f : X \longrightarrow \R$ such that
\begin{equation*} \|f\|_{X^\ast} = 1 \quad \text{and} \quad \langle f, \, x \rangle = \|x\| \neq 0. \end{equation*}
On the other hand, a functional $f \in X^\ast$ is nonzero if and only if there exists $x \in X$ such that $\langle f, \, x \rangle \neq 0$. Therefore, the valuation $j_x$ is different from zero in $f$, and this concludes the proof. \end{proof}

\begin{lemma} Let $X$ be a Banach space. \mbox{}
\begin{enumerate}[label=\textbf{(\alph*)}]
\item A subset $E \subset X$ is bounded in norm if and only if it is weakly bounded.
\item A subset $E \subset X^\ast$ is bounded in norm if and only if it is weakly-$\ast$ bounded.
\end{enumerate}
\end{lemma}

\begin{proof} One implication is clear in both the assertions: if $E$ is bounded w.r.t. the strong topology, then it is bounded in the weak (or weak-$\ast$) topology since there are less neighborhoods of $0$. \mbox{}
\begin{enumerate}[label=\textbf{(\alph*)}]
\item Assume that $E \subset X$ is weakly bounded. For every $f \in X^\ast$, the image $f(E)$ is bounded and hence there exists a positive constant $c_f > 0$ such that
\begin{equation*} \left|\langle f, \, x \rangle \right| \leq c_f \quad \text{for all $x \in E$}. \end{equation*}
If we identify $E$ with the subspace of the bidual $\left\{ j_x \: \left| \: x \in E \right. \right\}$, then the estimate above can be rewritten in terms of the valuation:
\begin{equation*} \left| j_x(f) \right| \leq c_f \quad \text{for all $x \in E$ and $f \in X^\ast$}. \end{equation*}
Then $E$ is pointwise bounded as a subset of the bidual $X^{\ast \ast}$, and hence the \hyperref[stein]{Banach-Steinhaus Theorem \ref{stein}} implies that $E$ is uniformly bounded as a subset of the bidual.

On the other hand, $X$ is a Banach space and so is $X^{\ast\ast}$. Consequently, the set $E$ is bounded w.r.t. the norm in $X^{\ast\ast}$, and therefore also w.r.t. the norm in $X$.
\item In this case the \hyperref[stein]{Banach-Steinhaus Theorem \ref{stein}} may be applied directly to the family $E$ since, by assumption, we have
\begin{equation*} \sup_{f \in E} \left| \langle f, \, x \rangle \right| < C \quad \text{for all $x \in X$}. \end{equation*}
\end{enumerate}
\end{proof}

\begin{remark} If $X$ is a locally convex topological vector space, the first assertion is still true while the second is, generally, not true. \end{remark}

\begin{lemma} \label{lemma:closedeness}Let $X$ be a Banach space. \mbox{}
\begin{enumerate}[label=\textbf{(\alph*)}]
\item If $C \subset X$ is a convex set, then $C$ is closed in norm if and only if $C$ is weakly closed.
\item If $C \subset X^\ast$ is a convex set, then $C$ is closed in norm if it is weakly-$\ast$ closed.
\end{enumerate}
\end{lemma}

\begin{proof} First, notice that $C$ closed in the weak (or weak-$\ast$) topology, always implies $C$ closed in norm. \mbox{}
\begin{enumerate}[label=\textbf{(\alph*)}]
\item Suppose that $C$ is a convex closed set w.r.t. the strong topology. We shall now prove that the complement $A := C^c$ is weakly open. Let $x_0 \notin C$. By \hyperref[theorem:hbg2]{Hahn-Banach Theorem \ref{theorem:hbg2}} there exist a linear continuous functional $f \in X^\ast$ and a real number $\alpha \in \R$ such that
\begin{equation*} \langle f, \, x_0 \rangle < \alpha < \langle f, \, x \rangle \quad \text{for all $x \in C$}. \end{equation*}
It follows that $V := \left\{ y \in X \: \left| \: \langle f, \, y \rangle < \alpha \right. \right\}$ is an open neighborhood of $x_0$, strictly contained in $A$, and this is enough to infer that $A$ is weakly open.
\item It follows easily from the definitions, but it is interesting to see a counterexample for the opposite implication. For example, a non-reflexive Banach space $X$ is closed in norm (as a subset of the bidual $X^{\ast\ast}$), but it is weakly-$\ast$ dense and thus it cannot be weakly-$\ast$ closed.

A beautiful, concrete example may be found at this \href{http://math.stackexchange.com/questions/218611/weak-star-closedness-in-dual-space}{Math Stackexchange} post.
\end{enumerate}\end{proof} 

\section{The Banach-Alaoglu Theorem}

In this section, we set the ground to state and prove a major result concerning the weak-$\ast$ topology. Namely, the closed ball is weakly-$\ast$ compact.

\vspace{1.5mm}
\noindent Let $\mathscr{X}$ be any set. The space of all the functions from $\mathscr{X}$ to $\C$, denoted by $\C^\mathscr{X}$, is a separated locally convex topological vector space, endowed with the product topology.

More precisely, one can easily prove that the product topology coincides with the initial topology associated to the family of the projections
\begin{equation*} \mathscr{F} := \left\{ \pi_x : \C^\mathscr{X} \longrightarrow \C_x \: \left| \: \C_x \cong \C, \, x \in \mathscr{X} \right. \right\}, \end{equation*}
and it is thus the coarsest that makes them continuous.

\begin{remark} There is a natural inclusion $\mathscr{X}^\ast \subset \C^\mathscr{X}$ and, surprisingly, the subspace topology coincides with the weak-$\ast$ topology $\sigma\left(\mathscr{X}^\ast, \, \mathscr{X} \right)$. \end{remark}

The assertion of the previous remark is, actually, a consequence of a more general fact. Let $\mathscr{Y} \subset \mathscr{X}$, and endow $\mathscr{X}$ with the initial topology relative to the family
\begin{equation*} \mathscr{F} := \{ f_\alpha : \mathscr{X} \longrightarrow \mathscr{Z}_\alpha\}_{\alpha \in \Gamma}, \end{equation*}
for some set of indices $\Gamma$. Then the subspace topology induces on $\mathscr{Y}$ is the initial topology relative to the restricted family
\begin{equation*} \F \, \big|_\mathscr{Y} := \{ f_\alpha : \mathscr{X} \cap \mathscr{Y} \longrightarrow \mathscr{Z}_\alpha\}_{\alpha \in \Gamma}. \end{equation*}

\begin{remark} The algebraic dual of $\mathscr{X}$ is closed w.r.t. the inclusion $\mathscr{X}^{\prime} \subset \C^\mathscr{X}$. Indeed, using the definitions, it is always possible to write it as the arbitrary intersection of closed sets:
\begin{equation*} \begin{aligned} \mathscr{X}^\prime & = \left\{ f : \mathscr{X} \longrightarrow \C \: \left| \: \text{$\langle f, \, \alpha  x+ \beta y \rangle = \alpha  \langle f,\, x\rangle + \beta \langle f,\, y\rangle$ for all $\alpha, \, \beta \in \C$ and $x, \, y \in \mathscr{X}$} \right. \right\} = \\[1em] & = \bigcap_{\alpha, \, \beta \in \C} \, \bigcap_{ x, \, y \in \mathscr{X}}  \left\{ f \in \C^\mathscr{X} \: \left| \: \langle f, \, \alpha  x+ \beta y \rangle = \alpha \langle f,\, x\rangle + \beta \langle f,\, y\rangle \right. \right\}. \end{aligned}\end{equation*}  \end{remark}

\begin{definition}[Polar] \index{polar set} Let $X$ be a topological vector space, and let $S \subset X$ be any subset. The \textit{polar} set associated to $S$ is defined by
\begin{equation*}S^\circ := \left\{ f \in X^\ast \: \left| \: \text{$\left| f(x) \right| \leq 1$ for all $x \in S$} \right. \right\} \end{equation*} \end{definition}

\begin{remark} Let $X$ be a topological vector space and let $S \subset X$ be any subset. The polar $S^\circ$ is clearly convex and weakly-$\ast$ closed since it is the intersection of weakly-$\ast$ closed convex sets:
\begin{equation*} S^\circ := \bigcap_{x \in S} \{x\}^\circ. \end{equation*}
Furthermore, if $V$ is a neighborhood of the origin in $X$, then the polar $V^\circ$ is convex, weakly-$\ast$ closed and also closed in the product space $\C^X$ (since it is closed in the algebraic dual $X^\prime$, which is closed in the product).\end{remark}

\begin{theorem}[Banach-Alaoglu]\index{Banach-Alaoglu Theorem} \label{weakastcop}Let $X$ be a topological vector space, and let $V$ be a neighborhood of the origin. Then the polar $V^\circ$ is convex, weakly-$\ast$ closed and weakly-$\ast$ compact.\end{theorem}

\begin{proof}First, notice that as a consequence of the definition of polar set we have
\begin{equation*} \left| \langle f, \, x \rangle \right| \leq 1 \quad \text{for all $x \in V$ and all $f \in V^\circ$}. \end{equation*}
In a topological vector space, a neighborhood of the origin is always absorbing. Therefore, there exists a function $\nu : X \longrightarrow \N$ such that, for any $f \in V^\circ$, we have
\begin{equation*} \left| \langle f, \, x \rangle \right| \leq \nu(x) \quad \text{for all $x \in X$}. \end{equation*}
Let $x \in X$ be a point. The image via $f \in V^\circ$ belongs to a ball, whose radius depends only on $x$ itself only, and this proves that
\begin{equation*} V^\circ \subset X^\prime \cap \prod_{x \in X} \overline{B_\C\left(0, \, \nu(x) \right)}. \end{equation*}
By \hyperref[tychonov]{Tychonov Theorem \ref{tychonov}}, it follows that
\begin{equation*} X^\prime \cap \prod_{x \in X} \overline{B_\C\left(0, \, \nu(x) \right)} \end{equation*}
is compact with respect to the product topology. On the other hand, the product topology coincides with the subspace topology induced by the inclusion
\begin{equation*}X^\prime \subset \C^X. \end{equation*}
In conclusion, note that $V^\circ$ is a weakly-$\ast$ closed subset of a Hausdorff weakly-$\ast$ compact set, and it is thus compact.
\end{proof}

\begin{corollary}Let $X$ be a Banach space. The closed ball $\overline{B_{X^\ast}(0, \, 1)}$ in $X^\ast$ is weakly-$\ast$ compact, that is, it is compact in the $\sigma(X^\ast, \, X)$ topology. \end{corollary}

\begin{proof}It follows from the \hyperref[weakastcop]{Banach-Alaoglu Theorem \ref{weakastcop}} since
\begin{equation*} V = B_X(0, \, 1) \implies V^\circ = \overline{B_{X^\ast}(0, \, 1)}. \end{equation*}
\end{proof}

\begin{theorem} Let $X$ be a infinite-dimensional Banach space. The topological spaces $\left(X, \, \sigma(X, \, X^\ast) \right)$ and $\left(X^\ast, \, \sigma(X^\ast, \, X) \right)$ are not metrizable, that is, they are not $\mathrm{A_1}$.  \end{theorem}

\begin{proof}[Optional Proof] We may always assume without loss of generality that $X$ is an infinite-dimensional space, that is, $\mathrm{dim} \, X \geq \aleph_0$. It is a well-known fact that the dual has (at least) the dimension of the continuum $\mathfrak{c}$, i.e., the dimension of $X^\ast$ is uncountable.

\paragraph{Step 1.} A subbasis of the neighborhoods of the origin in the weak topology $\sigma \left(X, \, X^\ast \right)$ is given by the sets of the form
\begin{equation*} W_{f, \, \epsilon} = \left\{ x \in X \: \left| \: \left| \left\langle f, \, x \right\rangle \right| < \epsilon \right. \right\} \quad \text{for some $f \in X^\ast$ and $\epsilon > 0$}. \end{equation*}
Clearly the family of the finite intersection of these elements is a local basis for the origin, since for any $U \in \mathcal{U}_0(X)$ there are $f_1, \, \dots, \, f_N \in X^\ast$ and $\epsilon_1, \, \dots, \, \epsilon_N > 0$ such that
\begin{equation*} U \supseteq \bigcap_{i = 1}^{N} W_{f_i, \, \epsilon_i}. \end{equation*}
As a consequence, we may always assume that the finite intersections are made up of \textbf{linearly independent} elements. 

\paragraph{Step 2.} Suppose that $X$ is metrizable, i.e. that $X$ is second-countable, and let
\begin{equation*}\mathcal{B} := \left\{ W_{f_i, \, \frac{1}{k}} \: \left| \: i, \, k \in \N \right. \right\} \end{equation*}
be a local countable basis made up of linearly independent functionals.

The dimension of $X^\ast$ is uncountable, and therefore we can always find a nontrivial functional $g \in X^\ast \setminus \mathrm{Span}\left< f_i \: \left| \: i \in \N \right. \right>$. Furthermore, by the \hyperref[theorem:hb]{Hahn-Banach Theorem \ref{theorem:hb}}, we can find a sequence of points $(x_n)_{n \in \N} \subset X$ such that
\begin{equation*} g(x_n) = 1 \quad \text{and} \quad f_i(x_n) = \frac{1}{n} \: \:  \text{for $i = 1, \, \dots, \, n$}. \end{equation*}
Any finite intersection of the form
\begin{equation*} W_{f_{i_1}, \frac{1}{k_1}} \cap \dots \cap W_{f_{i_N}, \frac{1}{k_N}} \end{equation*}
contains all but finitely many $x_n$'s, while the neighborhood $W_{g, \, 1}$ does not contain any point of the sequence. In particular
\begin{equation*} W_{f_{i_1}, \frac{1}{k_1}} \cap \dots \cap W_{f_{i_N}, \frac{1}{k_N}} \not \subset W_{g, \, 1} \end{equation*}
for any choice of the indices, and thus there exists a neighborhood of the origin that does not belong to $\mathcal{B}$: a contradiction.

\paragraph{Step 3.} Assume that the dimension of $X$ is uncountable, that is, $\mathrm{dim} \, X > \aleph_0$. A subbasis of the neighborhoods of the origin in the weak-$\ast$ topology $\sigma \left(X^\ast, \, X \right)$ is given by the sets of the form
\begin{equation*} W_{x, \, \epsilon} = \left\{ f \in X^\ast \: \left| \: \left| \left\langle f, \, x \right\rangle \right| < \epsilon \right. \right\} \quad \text{for some $x \in X$ and $\epsilon > 0$}. \end{equation*}
The family of the finite intersections of these elements is a local basis for the origin, since for any $U \in \mathcal{U}_0(X)$ there are $x_1, \, \dots, \, x_N \in X$ and $\epsilon_1, \, \dots, \, \epsilon_N > 0$ such that
\begin{equation*} U \supseteq \bigcap_{i = 1}^{N} W_{x_i, \, \epsilon_i}. \end{equation*}
As a consequence, we may always assume that the finite intersections are made up of \textbf{linearly independent} elements. 

\paragraph{Step 4.} Suppose that $X$ is metrizable, i.e. that $X$ is second-countable, and let
\begin{equation*}\mathcal{B} := \left\{ W_{x_i, \, \frac{1}{k}} \: \left| \: i, \, k \in \N \right. \right\} \end{equation*}
be a local countable basis made up of linearly independent functionals. The dimension of $X$ is uncountable by assumption, and hence there exists a point $y \in X \setminus \mathrm{Span}\left< x_i \: \left| \: i \in \N \right. \right>$. Furthermore, by the \hyperref[theorem:hb]{Hahn-Banach Theorem \ref{theorem:hb}}, we can always find a sequence of functionals $(f_n)_{n \in \N} \subset X$ such that
\begin{equation*} f_n(y) = 1 \quad \text{and} \quad f_n(x_i) = \frac{1}{n} \: \:  \text{for $i = 1, \, \dots, \, n$}. \end{equation*}
Any finite intersection of the form
\begin{equation*} W_{x_{i_1}, \frac{1}{k_1}} \cap \dots \cap W_{x_{i_N}, \frac{1}{k_N}} \end{equation*}
contains all but finitely many $f_n$'s, while the neighborhood $W_{y, \, 1}$ does not contain any point of the sequence. In particular
\begin{equation*} W_{x_{i_1}, \frac{1}{k_1}} \cap \dots \cap W_{x_{i_N}, \frac{1}{k_N}} \not \subset W_{y, \, 1} \end{equation*}
for any choice of the indices, and thus there is a neighborhood of the origin that does not belong to $\mathcal{B}$: a contradiction.

\paragraph{Step 5.} Assume now that the dimension of $X$ is countable, that is, $\mathrm{dim} \, X = \aleph_0$. Then the completion $\widetilde{X}$ has dimension at least $\mathfrak{c}$, and the weak-$\ast$ topologies of $X^\ast$ and $\widetilde{X}^\ast$ coincide. Therefore, the previous steps conclude the proof.
\end{proof} 

\begin{theorem}[Banach-Alaoglu] \index{Banach-Alaoglu Theorem} \label{baseq}Let $X$ be a separable Banach space. Then the closed ball
\begin{equation*}\overline{B_{X^\ast}(0, \, 1)} \subset X^\ast \end{equation*}
is weakly-$\ast$ metrizable and sequentially weakly-$\ast$ compact. \end{theorem}

\begin{proof}Let $\{ x_k \}_{k \in \N} \subset \overline{B_{X^\ast}(0, \, 1)}$ be a dense countable subset, and let us consider
\begin{equation*} D := \mathrm{Span}\left< x_k \: : \: k \in \N \right>. \end{equation*}

\paragraph{Metric.} For any $f, \, g \in \overline{B_{X^\ast}(0, \, 1)}$, let us set
\begin{equation} \label{bc:sdsdosdos} d(f, \, g) := \sum_{k \in \N} 2^{-k} \, \left| \langle f - g, \, x_k \rangle \right|. \end{equation}
We can easily check that \eqref{bc:sdsdosdos} defines an actual metric on the closed ball of radius one. Indeed, one can readily prove that: \mbox{}
\begin{enumerate}[label=\textbf{(\arabic*)}]
\item The function $d(-, \, \cdot)$ is positive and symmetric with respect to both variables.
\item The function $d(-, \, \cdot)$ is subadditive since
\begin{equation*} \begin{aligned} \sum_{k =0}^M 2^{-k} \, \left| \langle f - h, \, x_k \rangle \right| & = \sum_{k =0}^M 2^{-k} \, \left| \langle f - g + g - h, \, x_k \rangle \right| \leq  \\[1em] & \leq \sum_{k =0}^M 2^{-k} \, \left| \langle f - g, \, x_k \rangle \right|  + \sum_{k =0}^M 2^{-k} \, \left| \langle g - h, \, x_k \rangle \right|.  \end{aligned} \end{equation*}
\item If $f, \, g \in \overline{B_{X^\ast}(0, \, 1)}$ and $d(f, \, g) = 0$, then
\begin{equation*} \langle f - g, \, x_k \rangle = 0 \quad \forall \, k \in \N \implies  \langle f - g, \, d \rangle = 0 \qquad \forall \, d \in D, \end{equation*}
and we conclude that $f = g$ using the fact that $D$ is dense and $f - g$ is uniformly continuous.
\end{enumerate}

%\paragraph{Compactness.} We now prove that $(\overline{B_{X^\ast}(0, \, 1)}, \, d)$ is compact as a metric space. Namely, let $(f_n)_{n \in \N} \subset \overline{B_{X^\ast}(0, \, 1)}$ be a sequence of functionals. by definition, there is a subsequence $(f_{n_j})_{j \in \N}$ which converges at every $x_k \in D$, that is, a subsequence pointwisely converging on the whole subspace $D$.
%On the other hand, the $f_{n_j}$ are equicontinuous functions; hence the subsequence converges on the closure of $D$ as well, that is, it converges at every $x \in X$. More precisely, it turns out that
%\begin{equation*} f_{n_j} \stackrel{\ast}{\rightharpoonup} f \in \overline{B_{X^\ast}(0, \, 1)}. \end{equation*}

\paragraph{Equivalence.} To conclude the proof, we need to show that the weak-$\ast$ convergence is equivalent to the notion of convergence induced by the metric $d$. Let us consider
\begin{equation*} (f_n)_{n \in \N} \subset \overline{B_{X^\ast}(0, \, 1)}. \end{equation*}

\paragraph{Step 1.} Suppose that $f_n$ converges to some $f$ w.r.t. the weak-$\ast$ topology. It follows that
\begin{equation*} d(f_n, \, f) := \sum_{k \in \N} 2^{-k} \, \underbrace{\left|  \langle f - f_n, \, x_k \rangle\right|}_{\to 0}, \end{equation*}
as a consequence of the Lebesgue domination theorem.

\paragraph{Step 2.} Vice versa, if $f_n \xrightarrow{d} f$, then $d(f_n, \, f) \xrightarrow{n \to + \infty} 0$. In particular, it turns out that
\begin{equation*}  \sum_{k \in \N} 2^{-k} \left|  \langle f - f_n, \, x_k \rangle \right| \xrightarrow{n \to + \infty} 0 \iff \langle f - f_n, \, x_k \rangle \xrightarrow{n \to + \infty} \quad \text{for all $k \in \N$}, \end{equation*}
which means that $f_n -  f$ converges weakly at every $x_k$ to $0$. On the other hand, the $f_n$ are uniformly continuous, and therefore the sequence converges on the closure of $D$ as well, that is, it converges at every $x \in X$. More precisely, it turns out that
\begin{equation*} f_{n} \stackrel{\ast}{\rightharpoonup} f \in \overline{B_{X^\ast}(0, \, 1)}. \end{equation*}

\paragraph{Compactness.} We consider the identity mapping between two copies of the closed ball, endowed with different topologies, i.e.
\begin{equation*} \mathrm{id} : \left( \overline{B_{X^\ast}(0, \, 1)}, \, d \right) \longrightarrow \left( \overline{B_{X^\ast}(0, \, 1)}, \, \sigma(X^\ast, \, X) \right). \end{equation*}
This map is sequentially continuous, the domain is compact, and so is the codomain. We can finally infer that the closed ball is sequentially compact w.r.t. the weak-$\ast$ topology.
\end{proof}

\begin{theorem}[Separability Criterion] Let $X$ be a Banach space. If $\overline{B_{X^\ast}(0, \, 1)} \subset X^\ast$ is weakly-$\ast$ metrizable, then $X$ is separable. \end{theorem}

\begin{proof} Let $d$ be the metric on $\overline{B_{X^\ast}(0, \, 1)}$. For all $n \in \N$, we consider the metric ball of radius $1/n$, that is,
\begin{equation*} U_n := \left\{ f \in \overline{B_{X^\ast}(0, \, 1)} \: \left| \: d(f, \, 0) < \frac{1}{n} \right. \right\}. \end{equation*}
For all $n \in \N$, let $V_n$ be a neighborhood of the origin w.r.t. $\sigma\left(X^\ast, \, X\right)$ satisfying the inclusion $V_n \subset U_n$. We may assume that $V_n$ has the form
\begin{equation*} V_n := \left\{ f \in \overline{B_{X^\ast}(0, \, 1)}  \: \left| \: \text{$\left| \left\langle f, \, x \right\rangle \right| < \epsilon_n$ for all $x \in E_n$} \right. \right\}, \end{equation*}
where $\epsilon_n > 0$ and $E_n \subset X$ is some finite subset. Then
\begin{equation*} D := \bigcup_{n \in \N} E_n \subset X, \end{equation*}
is countable, and we now prove that $D$ is also dense in $X$. Indeed, if
\begin{equation*} \langle f, \, x \rangle = 0 \quad \text{for all $x \in D$}, \end{equation*}
then it turns out that $f \in V_n \subset U_n$ for any $n \in \N$, and thus $f \equiv 0$ since
\begin{equation*} \bigcap_{n \in \N} U_n = \{0\}. \end{equation*}\end{proof}

\begin{theorem} Let $(X, \, \tau)$ be a $\mathrm{T_0}$ topological vector space. Then $X$ is locally compact if and only if $X$ is a finite-dimensional space if and only if $X$ is linearly homeomorphic to $\mathbb{K}^n$. \end{theorem}

\begin{proof} We first prove that a finite-dimensional $\mathrm{T_0}$ topological vector space is linearly homeomorphic to $\mathbb{K}^n$ with the usual topology (which is locally compact), and then we show that a locally compact space is finite-dimensional.

\paragraph{Step 1.} Suppose that $X$ is a finite-dimensional $\mathrm{T_0}$ topological vector space, and let $\{e_1, \, \dots, \, e_n\}$ be a basis of $X$. There is linear isomorphism $\Phi : \mathbb{K}^n \longrightarrow X$, defined by setting
\begin{equation*} \mathbb{K}^n \ni (\lambda_1, \, \dots, \, \lambda_n) \longmapsto \sum_{i = 1}^{n} \lambda_i e_i \in X. \end{equation*}
We only need to prove that $\Phi$ is an open map to conclude that it is an homeomorphism since $\Phi$ is clearly continuous and bijective.

Let $B := \overline{B_{\mathbb{K}^n}(0, \, 1)}$ be the closed unit ball, and let $S := \partial B$ be its boundary. Then $S$ is compact in $\mathbb{K}^n$ and it does not contain the origin; hence $\Phi(S)$ is compact in $X$, and it does not contain the origin of $X$. In particular $\Phi(S)$ is closed\footnote{Recall that a $\mathrm{T}_0$ topological vector space is automatically $Hausdorff$, and a compact set in a Hausdorff space is always closed.} in $X$, and thus there exists $V \in \mathcal{U}_0(X)$ open and balanced neighborhood of the origin such that
\begin{equation*} V \cap \Phi(S) = \varnothing. \end{equation*}
Let $x \in V \setminus \Phi(B)$. The map is surjective, which means that we can find $\lambda \in \mathbb{K}^n$ such that $x = \Phi(\lambda)$, with $\| \lambda \| > 1$. The rescaling $\lambda / \| \lambda \|$ belongs to $S$, and therefore we find a contradiction since
\begin{equation*} \Phi \left( \frac{\lambda}{\| \lambda \|} \right) = \frac{x}{\| \lambda \|} \implies \Phi \left( \frac{\lambda}{\| \lambda \|} \right) \in \frac{1}{\| \lambda \|} \cdot V = V, \end{equation*}
that is, $x$ belongs to both $\Phi(S)$ and $V$ at the same time. It follows that $V \subseteq \Phi(B)$, and this implies that $\Phi(B)$ is a neighborhood of the origin in $X$, that is, $\Phi$ is an open mapping.

\paragraph{Step 2.} Conversely, assume that $X$ is locally compact, and let $V$ be a compact neighborhood of the origin of $X$. Clearly $1/2 \cdot V$ is also a neighborhood of the origin and, by compactness, there are finitely many points $x_i \in V$ such that
\begin{equation*} V \subseteq \bigcup_{i = 1}^{m}\left( x_i + \frac{1}{2} \cdot V \right).\end{equation*}
Let $Y$ be the linear span of the points $x_1, \, \dots, \, x_m$. Then
\begin{equation*} V \subseteq Y + \frac{1}{2} \cdot V \implies \dots \implies V \subseteq Y + \frac{1}{2^n} \cdot V. \end{equation*}
Notice that the local compactness of $X$ easily implies that the family $\left\{ 2^{-n} \cdot V \right\}_{n \in \N}$ is a local basis of the origin in $X$, and hence
\begin{equation*} V \subseteq \bigcap_{n \in \N} \left( Y + \frac{1}{2^n} \cdot V \right) = \overline{Y} = Y, \end{equation*}
since $Y$ is finite-dimensional, and thus closed. This concludes the proof since $V$ is absorbent and each rescaling is contained in $Y$, i.e.
\begin{equation*} X = \bigcup_{t > 0} t \cdot V \subseteq Y \implies X = Y. \end{equation*}
\end{proof} 

\begin{corollary} Let $(X, \, \tau)$ be a $\mathrm{T_0}$ topological vector space, and let $Y \subset X$ be a finite-dimensional subspace. Then $Y$ is closed. \end{corollary}

\section{Iteration Lemma}

In this brief section, we investigate the \textit{iteration lemma}, which we implicitly exploited to prove the open mapping theorem in the previous chapter.

\begin{lemma}[Iteration Lemma] \label{iterlemma} \index{iteration lemma} Let $T \in \mathcal{L}(X, \, Y)$ be a continuous linear mapping between two Banach spaces. Suppose that there exists a neighborhood of the origin $U \in \mathcal{U}_0(Y)$ and a positive real number $t \in (0, \, 1)$ satisfying the inclusion
\begin{equation} \label{eq.13.1.1.1} U \subseteq T  B_X + t \cdot U. \end{equation}
Then
\begin{equation*} (1 - t) \cdot U \subseteq T B_X \quad \text{and} \quad \text{$T$ is open and surjective.} \end{equation*}\end{lemma}

\begin{proof} The assumption \eqref{eq.13.1.1.1} implies that
\begin{equation*}(1-t) \cdot U \subseteq (1 - t) \cdot T  (B_X) + (1-t) t \cdot U \subseteq  T(B_X) + t \cdot T(B_X) + t^2 \cdot U. \end{equation*}
If we iterate this process $n$ times, we find the following inclusion:
\begin{equation*}(1-t) \cdot U \subseteq \left[ (1 - t) \sum_{k = 0}^{n-1} t^k \right] \cdot T (B_X) + (1-t) t^n \cdot U. \end{equation*}
Since $t$ belongs to $(0, \, 1)$, we can easily compute the sum of the series multiplying the first term on the right-hand side, that is,
\begin{equation*}\lim_{n \to + \infty}\left[ \sum_{k = 0}^{n-1} t^k \right] = \frac{1}{1 - t}.\end{equation*}
Therefore, if we take the limit as $n \to + \infty$ of the inclusion above, we find that
\begin{equation*}(1-t) \cdot U \subseteq \lim_{n \to + \infty} \left[ (1 - t) \sum_{k = 0}^{n-1} t^k \right] \cdot T (B_X) \subseteq T (B_X), \end{equation*}
where the last inclusion follows from the fact that $X$ is a Banach space (thus the sequence of points converges). Indeed, one can easily notice that
\begin{equation*}(1-t) \cdot U \subseteq T (B_X) \implies \text{$T$ open}, \end{equation*}
and
\begin{equation*}Y = \bigcup_{n \in \N} n \cdot U \implies Y \subseteq \bigcup_{n \in \N} \frac{n}{1-t} \cdot T (B_X) \implies \text{$T$ surjective}. \end{equation*}
\end{proof}

\paragraph{Applications of the Iteration Lemma.} In this paragraph, we state and prove two extension results that follow from an easy application of the iteration lemma.

\begin{theorem} Let $A$ be a closed subset of a metric space $M$, and let $f : A \longrightarrow E$ be a continuous map with values in a Banach space $E$. Then there exists a continuous map
\begin{equation*} \widetilde{f} : M \longrightarrow E \end{equation*}
which extends $f$, that is $\widetilde{f} \, \big|_{A} \equiv f$.\end{theorem}

\begin{proof}Notice that we may always assume, without loss of generality, that $f$ is a bounded map.

\paragraph{Step 0.} The thesis is clearly equivalent to the surjectivity of the operator
\begin{alignat*}{2}
 T :  C_b^0 \left(M, \, E \right) & \longrightarrow & \quad C_b^0 \left(A, \, E\right) & \\[1em]
  g &\longmapsto&  \quad g \, \big|_A. \quad &
\end{alignat*}
The idea here is to apply the \hyperref[iterlemma]{Iteration Lemma \ref{iterlemma}} to the operator $T$. In particular, it is enough to prove that for every $f \in C_b^0 \left(A, \, E\right)$ there is $g \in C_b^0\left(M, \, E \right)$ such that $\|g\| \leq 1$ and
\begin{equation*} \left\| g \, \big|_A - f \right\| \leq \frac{1}{2}. \end{equation*}
Indeed, any function $f \in C_b^0 \left(A, \, E\right)$ can be rewritten as
\begin{equation*} f = g \, \big|_A - \left( g \, \big|_A - f \right), \end{equation*}
and this is enough to infer that that the map is open and surjective.

\paragraph{Step 1.} The function $f$ is continuous at every point $p \in A$, which means that we can always find a neighborhood $U_p$ of $p$ in $M$ such that
\begin{equation*} f \, \big|_{U_p \cap A} \equiv f_p\, \big|_{U_p \cap A}, \end{equation*}
where $f_p : U_p \longrightarrow E$ is a continuous local approximation of $f$ at $p$.

\paragraph{Step 2.} Consider the open covering
\begin{equation*} \Omega := \left\{ U_p \right\}_{p \in A} \cup \{M \setminus A\} \end{equation*}
of the metric space $M$, and let let $\left\{ \psi_p \right\}_{p \in A} \cup \{ \psi_0\}$ be the associated partition of unity\footnote{Every metric space is paracompact, and therefore every metric space admits a partition of unity associated to an open covering.}.

\paragraph{Step 3.} For every $p \in A$ there are only finitely many functions $\psi_p$ that does not vanishing at $p$, and therefore we can define the approximation in the obvious way:
\begin{equation*} \widetilde{f}(x) := \sum_{p \in  A} \psi_p(x) f_p(x). \end{equation*}
The reader may check that, not only $\widetilde{f}(x)$ is the sought approximation, but also that it can be chosen to be zero outside of an arbitrarily small neighborhood of $A$.
\end{proof}

\begin{theorem} Let $E$ and $F$ be Banach spaces, and let $L \in \mathcal{L}(E, \, F)$ linear, continuous and surjective operator. If $(M, \, d)$ is a metric space and $f : M \longrightarrow F$ a mapping, then it can be lifted to a map $\widetilde{f} : M \longrightarrow E$ in such a way that the following diagram commutes:
\begin{equation*} \begin{tikzcd} & E \arrow{d}{L} \\ M \arrow{r}{f} \arrow[dotted]{ur}{\widetilde{f}} & F \end{tikzcd} \end{equation*}
\end{theorem}

\begin{proof} The argument is very much similar to the previous one. First, we assume, without loss of generality, that $f$ is a bounded mapping.

\paragraph{Step 0.} The thesis is clearly equivalent to the surjectivity of the operator
\begin{alignat*}{2}
 T :  C_b^0 \left(M, \, E \right) & \longrightarrow & \quad C_b^0 \left(M, \, F\right) & \\[1em]
  g &\longmapsto&  \quad L \circ g. \quad &
\end{alignat*}
In particular, the \hyperref[iterlemma]{Iteration Lemma \ref{iterlemma}} implies again that it is enough to show that for every $f \in C_b^0 \left(M, \, F\right)$ there is a $g \in C_b^0\left(M, \, E \right)$ such that $\|g\| \leq 1$ and
\begin{equation*} \left\| L \circ g - f \right\| \leq \frac{1}{2}. \end{equation*}
We shall not prove this result here because it is enough to produce minor changes to the argument used in the previous theorem. \end{proof}

\section{Closed Rank Theorem}

In this final section, we study the set of all operators between two Banach spaces in more depth; more precisely, we investigate necessary and sufficient condition for an operator to be injective (or surjective) and the relation with the dual operator.

\begin{proposition} Let $X$ and $Y$ be Banach spaces. The subspace of all surjective continuous linear mappings from $X$ to $Y$ is open. \end{proposition}

\begin{remark} More precisely, if $T \in \mathcal{L}(X, \, Y)$ is surjective, the operator
\begin{equation*} \widetilde{T} : \faktor{X}{\kers \, T} \longrightarrow Y \end{equation*}
is invertible, and the positive constant $k := \left\| \widetilde{T}^{-1} \right\| > 0$ is well-defined. Then, the proposition above can be restated in a compact way as follows:
\begin{equation*} k \cdot B_Y \subseteq T \left( B_X \right) \implies \text{$T + H$ surjective for all $H$ such that $\|H\| \leq k$}. \end{equation*}  \end{remark}

\begin{proof}The proof is a fairly simple consequence of the \hyperref[iterlemma]{Iteration Lemma \ref{iterlemma}}. Indeed, the inclusion
\begin{equation*} k \cdot B_Y \subseteq T \left( B_X \right)\end{equation*}
immediately implies that 
\begin{equation*} \begin{aligned} k \cdot B_Y & \subseteq \left(T + H \right)\left(B_X \right) + H \left(B_X \right) \subseteq \\[1em] & \subseteq \left(T + H \right) \left(B_X \right) + \|H\| \cdot B_Y \subseteq \\[1em] & \subseteq \left(T + H \right) \left(B_X \right) + \frac{\|H\|}{k} \cdot \left(k \cdot B_Y \right). \end{aligned} \end{equation*}
Since $k > \|H\|$ by assumption, the \hyperref[iterlemma]{Iteration Lemma \ref{iterlemma}} implies that
\begin{equation*}k \cdot \left(1 - \frac{\|H\|}{k} \right) \cdot B_Y \subseteq \left(T + H \right) \left(B_X \right) , \end{equation*}
and hence
\begin{equation*}\left(k - \| H \| \right) \cdot B_Y \subseteq \left(T + H \right) \left(B_X \right) , \end{equation*}
which means that $T + H$ is open and, hence, surjective.
\end{proof}

\begin{remark} If $X$ and $Y$ are infinite-dimensional Banach spaces, then the result does not hold for the set of all injective linear continuous operators. \end{remark}

\begin{lemma} Let $T \in \La(X, \, Y)$ be a linear continuous injective operator. Then
\begin{equation*} \mathrm{Ran}(T) = \overline{ \mathrm{Ran}(T) } \iff \exists \, c > 0 \: : \: \|Tx\| \geq c \|x\|. \end{equation*} \end{lemma}

\begin{proof}First, assume that $\mathrm{Ran}(T)$ is closed. Then $(\mathrm{Ran}(T), \, \| \cdot \|_Y)$ is a Banach space, and the \hyperref[opt]{Open Mapping Theorem} implies that
\begin{equation*} T^{-1} : \mathrm{Ran}(T) \longrightarrow X \end{equation*}
is a continuous well-defined operator. Therefore
\begin{equation*} \| T^{-1}x \|_Y \leq c_1 \| x \|_X \implies \| T x \|_Y \geq \frac{1}{c_1} \| x \|_X.\end{equation*}
Vice versa, notice that
\begin{equation*} \exists \, c > 0 \: : \: \|Tx\| \geq c \|x\| \implies \text{$T$ injective}. \end{equation*}
Let $(y_n)_{n \in \N} \subset \mathrm{Ran}(T)$ be a sequence converging to some $y \in Y$. Then
\begin{equation*} \|y_n - y_m\|_Y = \| TT^{-1} y_n - TT^{-1} y_m \|_Y \gtrsim \| T^{-1} y_n - T^{-1}y_m \|_X \end{equation*}
for all $n, \, m \in \N$. The sequence $(y_n)_{n \in \N}$ is Cauchy in $Y$, and thus the sequence $(T^{-1} y_n)_{n \in \N} \subset X$ is Cauchy in $X$, which means that
\begin{equation*}T^{-1} y_n \xrightarrow{n \to + \infty} x \in X.\end{equation*}
In conclusion, notice that
\begin{equation*}T(x) = \lim_{n \to + \infty} T(T^{-1}(y_n)) = \lim_{n \to + \infty} y_n = y,\end{equation*}
and this proves that $y \in \mathrm{Ran}(T)$.\end{proof}

On the other hand, under the simple assumption that the images are closed, we can find an equivalent proposition for injective operators.

\begin{proposition}\label{rmk:injclos} Let $X$ and $Y$ be Banach spaces. The subspace of injective continuous linear mappings from $X$ to $Y$, with closed rank, is open. \end{proposition} 

\begin{proof} Let $L \in \mathcal{L}(X, \, Y)$ be an injective operator and assume that $\mathrm{Ran}(T)$ is closed (in norm). The previous lemma implies that we can always find a positive constant $c > 0$ such that
\begin{equation*} \left\| Tx \right\| \geq c\|x\| \quad \text{for all $x \in X$}, \end{equation*}
and this is an open condition. Indeed, the triangular inequality implies that
\begin{equation*} \left\| \left(T + H \right) x \right\| \geq \left\| T x \right\| - \|H\| \|x\| \geq \left( c - \|H\| \right) \|x\|, \end{equation*}
and thus $T + H \in \mathcal{L}(X, \, Y)$ is injective and its rank is closed for every $H$ whose norm is strictly less than $c$. \end{proof}

\begin{proposition}\label{prop:sdaoskdq03291jiods}Let $X$ and $Y$ be Banach spaces, and let $T \in \mathcal{L}(X, \, Y)$ be an operator such that the adjoint $T^\ast : Y^\ast \longrightarrow X^\ast$ satisfies the estimate
\begin{equation*} \left\| T^\ast y^\ast \right\| \geq \| y^\ast \| \quad \text{for all $y^\ast \in Y^\ast$}. \end{equation*}
Then $T : X \longrightarrow Y$ is an open and surjective mapping, that is,
\begin{equation*} k \cdot B_Y \subseteq T \left(B_X \right). \end{equation*} \end{proposition}

\begin{proof}Let $t \in (0, \, 1)$. By the \hyperref[iterlemma]{Iteration Lemma \ref{iterlemma}}, it suffices to prove that
\begin{equation*} B_Y \subseteq T \left(B_X \right) + t \cdot B_Y. \end{equation*}
Suppose that there is $y_0 \in B_Y$ such that $y \notin T \left(B_X \right) + t \cdot B_Y$. Since $C := T \left(B_X \right) + t \cdot B_Y$ is an open convex set, the \hyperref[theorem:hbg1]{Hahn-Banach Theorem \ref{theorem:hbg1}} implies that there exists $y^\ast \in Y^\ast$ such that
\begin{equation*} \left\langle y^\ast, \, Tx + t \cdot y \right\rangle < \left\langle y^\ast, \, y_0 \right\rangle \quad \text{for all $x \in B_X$ and all $y \in B_Y$}. \end{equation*}
If we take $y = y_0$, then it turns out that
\begin{equation*} (1 - t) \cdot \left\langle y^\ast, \, y_0 \right\rangle > \left\langle y^\ast, \, T x \right\rangle = \left\langle T^\ast y^\ast, \, x \right\rangle \quad \text{for all $x \in B_X$}.\end{equation*}
We take the supremum w.r.t. $x \in B_X$ and $y \in B_Y$ respectively, and we obtain a contradiction with the assumption on $T^\ast$ since
\begin{equation*}  (1-t)\cdot \|y^\ast\| \geq (1 - t) \cdot \left\langle y^\ast, \, y_0 \right\rangle \geq \left\| T^\ast \, y^\ast \right\| \geq \|y^\ast\| \end{equation*}
for some $t \in (0, \, 1)$, which is clearly impossible.
\end{proof}

\begin{remark} The condition stated in the previous \hyperref[prop:sdaoskdq03291jiods]{Proposition \ref{prop:sdaoskdq03291jiods}} is not only sufficient for the operator $T$ to be surjective, but it is also necessary. \end{remark}

\begin{corollary} \label{corollary:skdsd}Let $X$ and $Y$ be Banach spaces and let $T \in \mathcal{L}(X, \, Y)$ be a linear continuous operator. If $T^\ast$ is injective and the rank $\mathrm{Ran} \, T^\ast$ is closed, then the map $T$ is open and surjective.\end{corollary}

\paragraph{Annihilator.} Let $X$ be a Banach space and let $M$ be a subspace of $X$. The \textit{annihilator}\index{annihilator} of $M$ is defined by
\begin{equation*} M^\perp = \left\{ f \in X^\ast \: \left| \:\text{$\langle f, \, x \rangle = 0$ for all $x \in M$} \right. \right\}. \end{equation*}
Notice that $M^\perp$ is a weakly-$\ast$ closed subspace of the dual $X^\ast$, since it is equal to an intersection  (possibly infinite) of closed sets:
\begin{equation*} M^\perp = \bigcap_{x \in M} \left\{ f \in X^\ast \: \left| \: \langle f, \, x \rangle = 0 \right. \right\}. \end{equation*}
In a similar fashion, given subspace $N$ of $X^\ast$, we define the \textit{pre-annihilator}\index{pre-annihilator} of $N$ as the set of points of $x$ which are zero against each element of $N$, that is,
\begin{equation*} N_\perp = \left\{ x \in X \: \left| \: \text{$\langle f, \, x \rangle = 0$ for all $f \in N$} \right. \right\}. \end{equation*}
Clearly, the pre-annihilator $N_\perp \subset X$ is closed w.r.t. the norm topology or, equivalently, w.r.t. the weak topology $\sigma(X, \, X^\ast)$ since it is convex (see \hyperref[lemma:closedeness]{Lemma \ref{lemma:closedeness}}). Indeed, we can always write it as the intersection of closed sets:
\begin{equation*} N_\perp = \bigcap_{f \in N} \mathrm{Ker} \, f. \end{equation*}
Notice that $^\perp$ and $_\perp$ are both inclusion-inversing operation, that is,
\begin{equation*} A \subset B \subset X \implies A^\perp \supset B^\perp \quad \text{and} \quad A \subset B \subset X^\ast \implies A_\perp \supset B_\perp. \end{equation*}

\begin{remark}If $N$ is a subspace of the dual $X^\ast$, the annihilator $N^\perp$ is a subset of the bidual $X^{\ast\ast}$, while the pre-annihilator $N_\perp$ is a subset of $X$, which can be identified with its image via the canonical immersion $\imath : X \hookrightarrow X^{\ast\ast}$. As a consequence, it turns out that
\begin{equation*} N_\perp = N^\perp \cap X = \imath^{-1} \left(N^\perp \right). \end{equation*}\end{remark} %%%%

\begin{lemma}\label{lemma:annhil}Let $X$ be a Banach space, and let $M \subset X$ and $N \subset X^\ast$ be two subspaces. \mbox{}
\begin{enumerate}[label=\textbf{(\alph*)}]
\item $M \subseteq \left(M^\perp \right)_\perp$.
\item $\left(M^\perp\right)_\perp = \overline{M}^{\|\cdot\|} = \overline{M}^{\tau_w}$.
\item $N \subseteq \left(N_\perp \right)^\perp$.
\item $ \left(N_\perp \right)^\perp = \overline{N}^{\tau_w^\ast}$.
\end{enumerate} \end{lemma}

\begin{proof}\mbox{}
\begin{enumerate}[label=\textbf{(\alph*)}]
\item This inclusion follows from the definitions.
\item By \hyperref[lemma:closedeness]{Lemma \ref{lemma:closedeness}}, it follows that $\left(M^\perp\right)_\perp$ is closed with respect to both the strong topology and the weak topology, and thus
\begin{equation*}\left(M^\perp\right)_\perp \supseteq \overline{M}^{\|\cdot\|} = \overline{M}^{\tau_w}. \end{equation*}
Let $x \notin \overline{M}^{\|\cdot\|}$. It follows from \hyperref[theorem:hb]{Hahn-Banach Theorem} there exists a functional $f$ such that
\begin{equation*} \left\langle f, \, x \right\rangle = 1 \quad \text{and}\quad f \in M^\perp, \end{equation*}
and thus $x \notin \left(M^\perp\right)_\perp$, which is enough to infer that the opposite inclusion holds.
\item This inclusion follows from the definitions.
\item We noted above that $\left(N_\perp\right)^\perp$ is closed with respect to both the weak-$\ast$ topology, and thus
\begin{equation*}\left(N_\perp\right)^\perp \supseteq \overline{N}^{\tau_w^\ast}. \end{equation*}
Let $f \notin \overline{N}^{\tau_w^\ast}$. It follows from \hyperref[theorem:hb]{Hahn-Banach Theorem} there exists a valuation $j_x$ such that
\begin{equation*} \left\langle j_x, \, f \right\rangle = 1 \quad \text{and}\quad x \in N_\perp, \end{equation*}
and thus $f \notin \left(N_\perp\right)^\perp$, which is enough to infer that the opposite inclusion holds.
\end{enumerate}\end{proof} 

\begin{lemma}\label{inclsuos}Let $X$ and $Y$ be Banach spaces and let $T \in \mathcal{L}(X, \, Y)$. Then the following inclusions hold true: \mbox{}
\begin{enumerate}[label=\textbf{(\alph*)}]
\item $\mathrm{Ker} \, T = \left( \mathrm{Ran}\, T^\ast \right)_\perp$.
\item $\mathrm{Ker} \, T^\ast = \left( \mathrm{Ran} \, T \right)^\perp$.
\item $\mathrm{Ran} \, T \subseteq \left( \mathrm{Ker} \, T^\ast  \right)_\perp$.
\item $\mathrm{Ran} \, T^\ast \subseteq \left( \mathrm{Ker} \, T\right)^\perp$.
\item $\overline{\mathrm{Ran} \, T}^{\|\cdot\|} = \overline{\mathrm{Ran}\, T}^{\tau_w} =  \left( \mathrm{Ker} \, T^\ast  \right)_\perp$.
\item $\overline{\mathrm{Ran}\, T^\ast}^{\tau_w^\ast} = \left( \mathrm{Ker} \, T  \right)^\perp$.
\end{enumerate} \end{lemma}

\begin{proof}\mbox{}
\begin{enumerate}[label=\textbf{(\alph*)}]
\item Let $x \in \mathrm{Ker} \, T$. For every $f \in Y^\ast$, it turns out that
\begin{equation*} \left\langle f, \, Tx \right\rangle = 0 \implies \left\langle T^\ast f, \, x \right\rangle = 0, \end{equation*}
that is, $x \in  \left( \mathrm{Ran} \, T^\ast \right)_\perp$. Vice versa, let $x \in  \left( \mathrm{Ran} \, T^\ast \right)_\perp$ and notice that
\begin{equation*} \left\langle f, \, Tx \right\rangle = 0 \impliedby \left\langle T^\ast f, \, x \right\rangle = 0 \end{equation*}
for all $f \in Y^\ast$, which means that $x \in \mathrm{Ker} \, T$.
\item Let $f \in \mathrm{Ker} \, T^\ast$. For every $x \in X$, it turns out that
\begin{equation*} \left\langle T^\ast f, \, x \right\rangle = 0 \implies \left\langle f, \, Tx \right\rangle = 0, \end{equation*}
that is, $x \in  \left( \mathrm{Ran} \, T \right)^\perp$. Vice versa, let $f \in  \left( \mathrm{Ran} \, T \right)^\perp$ and notice that
\begin{equation*} \left\langle T^\ast f, \, x \right\rangle = 0 \impliedby \left\langle f, \, Tx \right\rangle = 0, \end{equation*}
for all $x \in X$, which means that $f \in \mathrm{Ker} \, T^\ast$.
\item Let $y \in \mathrm{Ran} \, T$, and let $x \in X$ be a point such that $Tx = y$. Thus, for every $f \in Y^\ast$, it turns out that
\begin{equation*} \left\langle f, \, Tx \right\rangle = \left\langle f, \, y \right\rangle \implies \left\langle f, \, y \right\rangle = \left\langle T^\ast f, \, x \right\rangle \end{equation*}
and this concludes the proof.
\item This is similar to the previous point.
\item This equality follows from the point \textbf{(c)} and \hyperref[lemma:annhil]{Lemma \ref{lemma:annhil}}.
\item This equality follows from the point \textbf{(d)} and \hyperref[lemma:annhil]{Lemma \ref{lemma:annhil}}.
\end{enumerate}\end{proof}

\begin{lemma}Let $X$ and $Y$ be Banach spaces and let $T \in \mathcal{L}(X, \, Y)$ be an operator. \mbox{}
\begin{enumerate}[label=\textbf{(\alph*)}]
\item The operator $T$ is injective if and only if $\mathrm{Ran} \, T^\ast$ is weakly-$\ast$ dense.
\item The operator $T^\ast$ is injective if and only if $\mathrm{Ran} \, T$ is weakly/strongly dense.
\end{enumerate} \end{lemma}

\begin{proof} \mbox{}
\begin{enumerate}[label=\textbf{(\alph*)}]
\item The operator $T$ is injective if and only if $\mathrm{Ker} \, T = 0$ if and only if (\hyperref[inclsuos]{Lemma \ref{inclsuos}}) $\left( \mathrm{Ran}\, T^\ast \right)_\perp = 0$ if and only if the weak-$\ast$ closure of $\mathrm{Ran} \, T^\ast$ is the whole space $Y^\ast$, that is, the rank is $\tau_w^\ast$-dense.
\item The operator $T^\ast$ is injective if and only if $\mathrm{Ker} \, T^\ast = 0$ if and only if (\hyperref[inclsuos]{Lemma \ref{inclsuos}}) $\left( \mathrm{Ran} \, T \right)^\perp = 0$ if and only if the weak closure (or norm closure) of $\mathrm{Ran} \, T$ is the whole space $Y$, that is, the rank is $\tau_w$-dense (or dense in norm).
\end{enumerate}\end{proof}

\begin{theorem}[Closed Rank] \index{Closed Rank Theorem}\label{closedrank}Let $X$ and $Y$ be Banach spaces and let $T \in \mathcal{L}(X, \, Y)$ be a continuous linear operator. The following properties are equivalent: \mbox{}
\begin{enumerate}[label=\textbf{(\arabic*)}]
\item The rank $\mathrm{Ran} \, T$ is closed in norm.
\item The rank $\mathrm{Ran} \, T$ is weakly closed.
\item The rank $\mathrm{Ran} \, T$ is equal to $\left(\mathrm{Ker}\,T^\ast \right)_\perp$.
\item The rank $\mathrm{Ran}\, T^\ast$ is closed in norm.
\item The rank $\mathrm{Ran} \, T^\ast$ is weakly-$\ast$ closed.
\item The rank $\mathrm{Ran} \, T^\ast$ is equal to $\left( \mathrm{Ker} \, T \right)^\perp$.
\end{enumerate}
\end{theorem}

\begin{proof}Here we only prove the two nontrivial implications. The remaining ones follows easily from the theory we have developed so far.

\paragraph{"$\mathbf{(4)} \implies \mathbf{(1)}$"} Suppose that $\mathrm{Ran} \,T^\ast$ is closed w.r.t. the topology induced by the norm as a subset of $X^\ast$. Let $Z = \overline{\mathrm{Ran} \, T}^{\|\cdot\|}$ and consider the decomposition of the operator $T$ as follows:
\begin{equation*} \begin{tikzcd} X \arrow{rr}{T} \arrow{dr}{S} && Y \\ & Z \arrow[hookrightarrow]{ur}{j} & \end{tikzcd}\end{equation*}
We may equivalently prove that $S$ is a surjective operator. If we consider the dual diagram
\begin{equation*} \begin{tikzcd} X^\ast &&  \arrow{ll}{T^\ast}  Y^\ast \arrow{dl}{j^\ast}  \\ & Z^\ast \arrow{ul}{S^\ast} & \end{tikzcd}\end{equation*}
then it is easy to see that $j^\ast$ is the restriction operator, which is surjective by the \hyperref[theorem:hb]{Hahn-Banach Theorem \ref{theorem:hb}}. Consequently, we have the equality
\begin{equation*} \mathrm{Ran} \, S^\ast = \mathrm{Ran} \, T^\ast,\end{equation*}
which means that $\mathrm{Ran} \, S^\ast$ is closed w.r.t. the norm topology. In conclusion, since $S^\ast$ is an injective operator with closed rank, it follows from \hyperref[corollary:skdsd]{Corollary \ref{corollary:skdsd}} that $S$ is surjective, and thus $Z = \mathrm{Ran} \, T$.

\paragraph{"$\mathbf{(1)} \implies \mathbf{(6)}$"} Suppose that $\mathrm{Ran}\, T$ is closed w.r.t. the topology induced by the norm. We already know that the inclusion
\begin{equation*} \mathrm{Ran}(T^\ast) \subseteq \left( \mathrm{Ker}(T) \right)^\perp \end{equation*}
always holds. Therefore, it is enough to prove the opposite one. Let $f \in \left(\mathrm{Ker} \, T \right)^\perp$, and notice that 
\begin{equation*} \mathrm{Ker}\, f \supset \mathrm{Ker} \, T. \end{equation*}
The usual algebraic lemma proves that there exists $\widetilde{f} : \mathrm{Ran} \, T \longrightarrow \mathbb{K}$ such that the small triangle in the diagram below commutes:
\begin{equation*} \begin{tikzcd} X \ar[d, "f"] \ar[bend left, rr, "T"] \arrow[twoheadrightarrow]{r}{} & \mathrm{Ran}(T) \ar[r, hookrightarrow] \ar[dl, "\tilde{f}"] & Y \ar[dll, bend left, "F", dotted] \\ \mathbb{K} &  & \end{tikzcd}\end{equation*}
By \hyperref[theorem:hb]{Hahn-Banach Theorem \ref{theorem:hb}} there is an extension $F$ of $\widetilde{f}$ to the whole $Y$ satisfying the identity $F \circ T = f$. Since $F \circ T = T^\ast (F)$, this implies that $f$ belongs to the image of $T^\ast$, that is,
\begin{equation*} \mathrm{Ran} \, T^\ast \supseteq \left( \mathrm{Ker} \, T \right)^\perp. \end{equation*}
\end{proof}

\section{Appendix}

In this appendix, we introduce the notion of \textit{filter} and, specifically, \textit{ultrafilter} to give a relatively simple proof of the Tychonov Theorem, which is a result we used in the Banach-Alaoglu section.

\subsection{Tychonov Theorem. Ultrafilter Approach}

The main purpose of this section is to prove the Tychonov theorem using the basic definitions and the basic properties of \textbf{ultrafilters}.

\begin{theorem}[Tychonov] \label{tychonov} The topological product $\prod_i X_i$ of an arbitrary collection of compact topological spaces $X_i$ is compact. \end{theorem}

Recall that a set $E$ is compact (in the Heine-Borel sense) if and only if any open covering $\mathcal{U}$ admits a finite subfamily $\mathcal{V} \subset \mathcal{U}$ such that
\begin{equation*} E \subseteq \bigcup_{V \in \mathcal{V}} V. \end{equation*}

This notion of compactness can be easily reformulated in terms of closed coverings. Namely, a set $Y \subset X$ is compact if and only if every collection of closed set $\mathcal{F} = \left\{ F_i := U_i^c \right\}_{i \in I}$ has the finite intersection property, that is,
\begin{equation*} \bigcap_{i \in J} F_i \neq \varnothing \quad \text{for all $J \subset I$ finite} \implies \bigcap_{i \in I} F_i \neq \varnothing. \end{equation*}

\begin{definition}[Filter] Let $\mathfrak{X}$ be a set. A filter $\mathcal{F}$ on $\mathfrak{X}$ is a subset of $\mathcal{P}(\mathfrak{X})$ satisfying the following properties: \mbox{}
\begin{enumerate}[label=\textbf{(\alph*)}]
\item If $F \in \mathcal{F}$, then $F$ is nonempty.
\item The collection is increasing, that is, if $F \in \mathcal{F}$ and $F^\prime \supset F$, then $F^\prime \in \mathcal{F}$ as well.
\item The collection is closed under intersection, that is, for all $F, \, F^\prime \in \mathcal{F}$, the intersection $F \cap F^\prime$ still belongs to $\mathcal{F}$.
\end{enumerate}  \end{definition}

\begin{example} We now present a short list of filters. \mbox{}
\begin{enumerate}[label=\textbf{(\alph*)}]
\item If $(X, \, \tau)$ is a topological space and $x_0 \in X$ a point, then $\mathcal{F} = \mathcal{U}_{x_0}(X)$ is a filter.
\item If $X = \N$, the family $\mathcal{F} = \left\{ A \subset \N \: \left| \: \text{$A$ is finite} \right. \right\}$ is a filter.
\item If $A \subset X$ is nonempty, then the \textit{principal filter} associated to $A$ is defined by
\begin{equation*} \langle A \rangle := \left\{ B \in \mathcal{P}(X) \: \left| \: B \supseteq A \right. \right\}. \end{equation*}
\end{enumerate}
\end{example}

\begin{definition}[Filter Basis] Let $\mathfrak{X}$ be a set. A basis for a filter is a collection $\mathcal{B} \subset \mathcal{P}(\mathfrak{X})$ satisfying the following properties:
\begin{enumerate}[label=\textbf{(\alph*)}]
\item If $B \in \mathcal{B}$, then $B$ is nonempty.
\item The collection is closed under intersection, that is, for all $B, \, B^\prime \in \mathcal{B}$ there exists $B^{\prime \prime} \in \mathcal{B}$ such that $B^{\prime \prime} \subseteq B \cap B^\prime$.
\end{enumerate}
Furthermore, the filter generated by the basis $\mathcal{B}$ is defined by
\begin{equation*} \mathcal{F}_{\mathcal{B}} := \left\{ A \in \mathcal{P}(\mathfrak{X}) \: \left| \: \text{there is $B \in \mathcal{B}$ such that $B \subseteq A$} \right. \right\}. \end{equation*}\end{definition}

\begin{definition}[Coarser Filter] \index{coarser filter} Let $\mathfrak{X}$ be a set, and let $\mathcal{F}$ and $\mathcal{F}^\prime$ be filters on $\mathfrak{X}$. We say that $\mathcal{F}$ is coarser than $\mathcal{F}^\prime$ if $\mathcal{F} \subset \mathcal{F}^\prime$. \end{definition}

\begin{definition}[Coarser Basis] Let $\mathfrak{X}$ be a set, and let $\mathcal{B}$ and $\mathcal{B}^\prime$ be filter bases on $\mathfrak{X}$. We say that $\mathcal{B}$ is coarser than $\mathcal{B}^\prime$ if $\mathcal{F}_{\mathcal{B}} \subset \mathcal{F}_{\mathcal{B}^\prime}^\prime$. \end{definition}

\begin{remark} Let $\mathfrak{X}$ be a set, and let $\mathcal{F}$ and $\mathcal{F}^\prime$ be filters on $\mathfrak{X}$. The \textit{infimum} is defined as the maximum filter coarser than both, it always exists and it is equal to
\begin{equation*} \mathcal{F} \wedge \mathcal{F}^\prime := \mathcal{F} \cap \mathcal{F}^\prime. \end{equation*}
In a similar fashion, the \textit{supremum} is defined as the minimum filter finer than both, but it does not always exists. When it does, it is denoted by
\begin{equation*} \mathcal{F} \vee \mathcal{F}^\prime. \end{equation*} \end{remark}

\begin{lemma} Let $\mathfrak{X}$ be a set, and let $\mathcal{F}$ and $\mathcal{F}^\prime$ be filters on $\mathfrak{X}$. The following properties are equivalent: \mbox{}
\begin{enumerate}[label=\textbf{(\arabic*)}]
\item There exists a filter $\mathcal{G}$ finer than both.
\item The supremum $\mathcal{F} \vee \mathcal{F}^\prime$ exists.
\item For every $A \in \mathcal{F}$ and every $A^\prime \in \mathcal{F}^\prime$, the intersection $A \cap A^\prime$ is nonempty.
\end{enumerate}
\end{lemma}

\begin{proof} The chain of implications $\mathbf{(2)} \implies \mathbf{(1)} \implies \mathbf{(3)}$ is trivial. If $\mathbf{(3)}$ holds true, then the supremum may be directly defined as
\begin{equation*} \mathcal{F} \vee \mathcal{F}^\prime = \left\{ A \cap A^\prime \: \left| \: A \in \mathcal{F}, \, \, A^\prime \in \mathcal{F}^\prime \right. \right\}. \end{equation*} \end{proof}

Notice that the set of all the filters on $\mathfrak{X}$ is partially ordered by the inclusion $\subset$. Furthermore, it has the ascendant chain property, and thus (Zorn's lemma) each filter is contained in a maximal filter, called \textit{ultrafilter}.

\begin{lemma}Let $\mathfrak{X}$ be a set. The filter $\mathcal{M}$ on $\mathfrak{X}$ is maximal if and only if for every $A \in \mathcal{P}(\mathfrak{X})$ either $A \in \mathcal{M}$ or $A^c \in \mathcal{M}$. \end{lemma}

\begin{proof}We divide the proof into two steps.

\paragraph{$"\implies"$} Suppose that $\mathcal{M}$ is an ultrafilter. If $A$ is any subset of $\mathfrak{X}$, then it turns out that either $A \cap F \neq \varnothing$ for any $F \in \mathcal{M}$ or there exists $F \in \mathcal{M}$ such that the intersection is trivial.

In the first case, the previous lemma shows that the supremum between $\mathcal{M}$ and $\left\langle A \right\rangle$ is well-defined, and it is finer than $\mathcal{M}$, which is absurd. In the second case, $A \cap F = \varnothing$ implies that $F \subset A^c$, and thus $A^c \in \mathcal{M}$.

\paragraph{$"\impliedby"$} Vice versa, suppose that for any $A \in \mathcal{P}(\mathfrak{X})$ either $A \in \mathcal{M}$ or $A^c \in \mathcal{M}$. If $\mathcal{F} \supset \mathcal{M}$, then each $F \in \mathcal{F}$ either belongs also to $\mathcal{M}$ (which is fine) or the complement belongs to $\mathcal{M}$ (which is absurd, since $F^c \in \mathcal{M} \subset \mathcal{F}$ implies that $\varnothing = F \cap F^c \in \mathcal{F}$). \end{proof}

\paragraph{Image of a filter.} Let $f : \mathfrak{X} \longrightarrow \mathfrak{Y}$ be a function and let $\mathcal{F}$ be a filter on $\mathfrak{X}$. The image of $\mathcal{F}$ through $f$ is the filter $f(\mathcal{F})$ generated by the basis
\begin{equation*} \mathcal{B} = \left\{ f \left(F\right) \: \left| \: F \in \mathcal{F} \right. \right\}. \end{equation*} 

\begin{lemma} Let $f : \mathfrak{X} \longrightarrow \mathfrak{Y}$ be a function and let $\mathcal{F}$ be an ultrafilter on $\mathfrak{X}$. Then the image filter $f(\mathcal{F})$ is always an ultrafilter on $\mathfrak{Y}$. \end{lemma}

\paragraph{Topological Filters.} We are finally ready to restate the main topological definitions in terms of filters and ultrafilters.

\begin{definition}[Convergence] Let $(\mathfrak{X}, \, \tau)$ be a topological space. A filter $\mathcal{F}$ is said to be \textit{convergent} at $x \in \mathfrak{X}$ if and only if
\begin{equation*} \mathcal{F} \supset \mathcal{U}_x(\mathfrak{X}). \end{equation*}\end{definition}

\begin{definition}[Continuity] Let $(\mathfrak{X}, \, \tau)$ and $(\mathfrak{Y}, \, \sigma)$ be topological spaces. A function $f : \mathfrak{X} \longrightarrow \mathfrak{Y}$ is \textit{continuous} at $x \in \mathfrak{X}$ if and only if 
\begin{equation*}f \left( \mathcal{U}_{x}(\mathfrak{X}) \right) \supset \mathcal{U}_{f(x)}(\mathfrak{Y}).\end{equation*} 
In particular, a sequence $f : \N \longrightarrow X$ is convergent, and its limit is $x_0 \in \mathfrak{X}$, if and only if the image filter of the Frechèt filter $\mathcal{F}$ on $\N$ is convergent at $x_0$. \end{definition}

\begin{remark}Let $f : \mathfrak{X} \longrightarrow \mathfrak{Y}$ be a function, and let $\mathcal{F}$ be a filter on $\mathfrak{Y}$. If $f$ is \textit{surjective}, then
\begin{equation*}\left\{ f^{-1}(F) \: \left| \: F \in \mathcal{F} \right. \right\} \end{equation*}
is a filter basis on $\mathfrak{X}$, and we denote by $f^{-1}(\mathcal{F})$ the generated filter.  \end{remark}

\begin{remark}Let $\mathfrak{X}$ and $\mathfrak{Y}$ be sets, and let $\mathcal{F}$ and $\mathcal{G}$ be filters on $\mathfrak{X}$ and $\mathfrak{Y}$ respectively. If $f : \mathfrak{X} \longrightarrow \mathfrak{Y}$ is a function, then
\begin{equation*} f^{-1}(\mathcal{F}) \subseteq \mathcal{G} \iff \mathcal{F} \subseteq f(\mathcal{G}). \end{equation*}
Indeed, if $f^{-1}(\mathcal{F}) \subseteq \mathcal{G}$, then for every $F \in \mathcal{F}$ there is $G \in \mathcal{G}$ such that $G \subseteq f^{-1}(F)$. But this implies that $f(G) \subseteq F$, and thus $\mathcal{F} \subseteq f(\mathcal{G})$.
\end{remark}

\begin{lemma} Let $\left(\mathfrak{X}_i, \, \tau_i \right)_{i \in I}$ be topological spaces. A filter $\mathcal{F}$ on $\mathfrak{X} := \prod_{i \in I}X_i$ converges at $x \in X$ if and only if for each $i \in I$ the projection $\pi_i \left( \mathcal{F} \right)$ converges at $\pi_i(x)$.  \end{lemma}

\begin{proof}The filter of the neighborhoods of a point $x \in X$ can be expressed as the supremum of the filters $\pi_i^{-1} \left( \mathcal{U}_{x_i}(\mathfrak{X}_i) \right)$, where $x_i = \pi_i(x)$, and
\begin{equation*} \pi_i^{-1} \left( \mathcal{U}_{x_i}(\mathfrak{X}_i) \right) = \text{Filter generated by $\left\{ \prod_{j \neq i} \mathfrak{X}_j \times U_i \: \left| \: U_i \in \mathcal{U}_{x_i}(\mathfrak{X}_i) \right. \right\} $}. \end{equation*}
Therefore, a filter $\mathcal{F}$ on $\mathfrak{X} := \prod_{i \in I} \mathfrak{X}_i$ converges at $x \in \mathfrak{X}$ if and only if 
\begin{equation*} \mathcal{F} \supseteq \mathcal{U}_x = \sup_{i \in I} \pi_i^{-1} \left( \mathcal{U}_{x_i}(\mathfrak{X}_i) \right), \end{equation*}
which is equivalent to for any $i \in I$
\begin{equation*} \pi_i ( \mathcal{F} ) \supseteq  \mathcal{U}_{x_i}(\mathfrak{X}_i) \quad \text{for all $i \in I$}, \end{equation*}
also equivalent to $\pi_i \left( \mathcal{F} \right)$ converges at $\pi_i(x) = x_i$ for all $i \in I$.\end{proof}

\begin{definition}[Adherent] Let $(\mathfrak{X}, \, \tau)$ be a topological space and let $\mathcal{F}$ be a filter on $\mathfrak{X}$. A point $p \in \mathfrak{X}$ is \textit{adherent} to the filter $\mathcal{F}$ if and only if
\begin{equation*} \forall \, U \in \mathcal{U}_p(X), \, \, \forall \, F \in \mathcal{F} \rightsquigarrow U \cap F \neq \varnothing. \end{equation*} \end{definition}

\begin{remark}Equivalently, the set of the adherent point to a filter $\F$ can be expressed as the intersection of all the closures of the elements of $\F$. More precisely, it turns out that
\begin{equation*} \left\{ p \in \mathfrak{X} \: \left| \: \text{$p$ adherent to $\F$} \right. \right\} = \bigcap_{F \in \F} \overline{F}. \end{equation*}
Moreover, $p \in \mathfrak{X}$ is adherent to $\F$ if and only if there is a finer filter than both $\mathcal{U}_x(\mathfrak{X})$ and $\F$ if and only if there exists the supremum $\F \vee \mathcal{U}_x(\mathfrak{X})$ if and only if $\F$ admits a refinement convergent at $p$. \end{remark}

\begin{remark}Let $(\mathfrak{X}, \, \tau)$ be a compact topological space. As we have already pointed out, it is equivalent to require that every family of closed sets with nonempty finite intersection, has nonempty intersection. 

On the other hand, we can prove that $\mathfrak{X}$ is compact if and only if each filter on $\mathfrak{X}$ has adherent points (at least one) if and only if each ultrafilter is convergent if and only if each filter has adherent points and admits a convergent refinement. \end{remark}

\begin{proof}[Proof of Tychonov Theorem] Let $\mathcal{F}$ be an ultrafilter on the product $X := \prod_{i \in I} X_i$. Then for all $i \in I$ the image $\pi_i(\mathcal{F})$ is an ultrafilter on $X_i$ and thus, by compactness, it is also convergent. \end{proof}

\begin{remark}The Thychonov theorem requires the choice axiom to be proved, but it turns out that it is actually equivalent. More precisely, if we assume the Thychonov theorem to be true, then it turns out that for any family $(X_i)_{i \in I}$ of nonempty set, the product $\prod_{i \in I X_i}$ is nonempty.\end{remark}

\section{Exercises}

\begin{definition}[$\sigma$-convexity] The set $C \subseteq X$ is $\sigma$-convex if for any bounded sequence $(x_k)_{k \in \N} \subset C$ and any sequence $(\lambda_k)_{k \in \N}$ of positive numbers such that $\sum_k \lambda_k = 1$, it turns out that
\begin{equation*} \sum_{k \in\N} \lambda_k \cdot x_k \in C. \end{equation*} \end{definition}

\begin{exercise}Let $X, \, Y$ be Banach spaces and let $T \in \mathcal{L}(X, \, Y)$. \mbox{}
\begin{enumerate}[label=\textbf{(\alph*)}]
\item If $C$ is convex and closed, then $C$ is $\sigma$-convex.
\item If $C$ is convex and open, then $C$ is $\sigma$-convex.
\item If $C$ is $\sigma$-convex and bounded, then $T(C)$ is also $\sigma$-convex.
\item If $C$ is $\sigma$-convex, then also the translation and the preimage are $\sigma$-convexes.
\item If $C$ is $\sigma$-convex and bounded, then for any $t \in (0, \, 1)$ the inclusion $E \subseteq C + t \cdot E$ implies that $(1 - t)\cdot E \subseteq C$.

Moreover, this property characterize $\sigma$-convex sets in a Banach space.
\end{enumerate} \end{exercise}