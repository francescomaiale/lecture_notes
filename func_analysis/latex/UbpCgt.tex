\chapter{Banach-Steinhaus and Open Mapping Theorem} \thispagestyle{empty}

In the first half of this chapter, we formally introduce the notion of \textit{topological vector space}, and we investigate some of its fundamental properties (e.g., separability, special neighborhoods of the origin, and metrizability).

The next section is entirely devoted to state and prove the uniform boundedness theorem (due to Banach and Steinhaus) for families of operators between topological vector spaces.

In the final section, we exploit the boundedness principle to derive two fundamental results in functional analysis: the open mapping theorem, and the closed graph theorem (both widely used).

\section{Topological Vector Spaces}
\label{sec:tvs}

\begin{definition}[Topological Vector Space] \index{topological vector spaces} A \textit{topological vector space} is a $\mathbb{K}$ vector space $X$ endowed with a compatible topology $\tau$, that is, a topology such that the operations
\begin{equation*} + : X \times X \longrightarrow X \qquad \text{and} \qquad \cdot : \mathbb{K} \times X \longrightarrow X \end{equation*}
are both $\tau$-continuous. \end{definition}

\begin{remark} Let $(X, \, \tau)$ be a topological vector space. The topology is translation-invariant, which means that $\tau$ can be completely determined by a local basis of neighborhoods of the origin. \end{remark}

\begin{lemma} \label{lemma:svtpro} Let $(X, \, \tau)$ be a topological vector space. Then the following properties hold: \mbox{}
\begin{enumerate}[label = \textbf{(\arabic*)}]
\item Every neighborhood of the origin is absorbing.
\item Every neighborhood $U$ of the origin contains a neighborhood of the origin $V$ such that
\begin{equation*} V + V \subseteq U \quad \text{and} \quad V = - V. \end{equation*}
\item Every neighborhood $U$ of the origin contains a balanced\footnote{\textbf{Definition.} A subset $C \subset X$ of a vector space is balanced if and only if $\theta \cdot C \subseteq C$ for every $|\theta| \leq 1$.} neighborhood of the origin $V \subseteq U$.
\end{enumerate} \end{lemma}

\begin{proof} The assertion \textbf{(1)} is rather obvious since the product is continuous and
\begin{equation*} x \in X \leadsto \lim_{t \to 0^+} tx = 0 \implies tx \in U \quad \text{for all $t \in [0, \, \tau]$}. \end{equation*}
\mbox{}
\begin{enumerate}[label = \textbf{(\arabic*)}, start=2]
\item Recall that, in a topological vector space $(X, \, \tau)$, the sum is $\tau$-continuous. Consequently, there are $V_1$ and $V_2$ neighborhoods of $0$ such that
\begin{equation*} (x, \, y) \in V_1 \times V_2 \implies x + y \in U. \end{equation*}
In particular, we have the inclusion $V_1, \, V_2 \subseteq U$ and, since the family of the neighborhoods of a point is closed under intersection, we also have that $W := V_1 \cap V_2$ is a neighborhood of the origin. In conclusion, if we set $V := W \cap (-W)$, then it is easy to prove that that $V$ is the sought neighborhood of the origin, that is,
\begin{equation*} V + V \subseteq U \quad \text{and} \quad V = - V. \end{equation*}
\item Recall that, in a topological vector space, the product is $\tau$-continuous. In particular, there is a positive real number $\delta > 0$ and a neighborhood $W$ of $0$ such that
\begin{equation*} \alpha \cdot W \subseteq U, \qquad \forall \alpha \: : \: |\alpha| < \delta. \end{equation*}
Let us denote by $V$ the union of the scalings $\alpha \cdot W$, i.e.
\begin{equation*} V := \bigcup_{|\alpha| < \delta} \alpha \cdot W. \end{equation*}
It is easy to see that $V$ is a neighborhood of the origin, and also that $V \subseteq U$. In conclusion, to prove that $V$ is balanced, it suffices to take $|\beta| \leq 1$ and compute the product:
\begin{equation*} \beta \cdot V = \bigcup_{|\alpha| < \delta} (\beta \alpha) \cdot W, \end{equation*}
but $|\beta \alpha| < \delta$ for any $|\alpha| < \delta$, which is enough to infer that $V$ is balanced.
\end{enumerate} 
 \end{proof}

\begin{lemma} Let $X$ be a vector space, and let $\mathcal{U}$ be a collection of subsets satisfying the following properties:
\mbox{}
\begin{enumerate}[label = \textbf{(\arabic*)}]
\item The origin belongs to all $U \in \mathcal{U}$.
\item For all $U, \, V \in \mathcal{U}$, there exists $W \in \mathcal{U}$ such that $W \subseteq U \cap V$.
\item Every set $U \in \mathcal{U}$ is absorbing.
\item For all $U \in \mathcal{U}$ one can find $V \in \mathcal{U}$ such that
\begin{equation*} V + V \subseteq U \quad \text{and} \quad V = - V. \end{equation*}
\item For all $U \in \mathcal{U}$ one can find a balanced $V \in \mathcal{U}$ such that $V \subseteq U$.
\end{enumerate}
Then $\mathcal{B}_x := \left\{ x + \mathcal{U} \right\}$ is a local basis of neighborhoods for all $x \in X$. The generated topology $\tau$ on $X$ is compatible with the vector space operations, and it turns out that
\begin{equation*} A \in \tau \iff \forall \, x \in A,  \, \exists \, \, U \in \mathcal{U} \: : \: x \in U \subset A. \end{equation*} \end{lemma}

\subsection{Locally Convex Spaces}

In this brief subsection, we investigate some of the fundamental properties of locally convex topologies (since most of the spaces we will deal with are locally convex.)

\begin{definition}[Locally Convex] \index{topological vector space!locally convex}\label{def:lcsld} Let $(X, \, \tau)$ be a topological vector space. We say that $\tau$ is \textit{locally convex} if there exists a local basis of convex neighborhoods of $0$. \end{definition}

\begin{remark}Equivalently, a topological vector space $X$ is locally convex if and only if there exists a family $\mathcal{U}$ of subsets such that: \mbox{}
\begin{enumerate}[label = \textbf{(\arabic*)}]
\item The origin belongs to all $U \in \mathcal{U}$.
\item For all $U, \, V \in \mathcal{U}$, there exists $W \in \mathcal{U}$ such that $W \subseteq U \cap V$.
\item Every set $U \in \mathcal{U}$ is absorbing.
\item For all $U \in \mathcal{U}$ one can find $V \in \mathcal{U}$ such that
\begin{equation*} V + V \subseteq U \quad \text{and} \quad V = - V. \end{equation*}
\item For all $U \in \mathcal{U}$ one can find a convex balanced set $V \in \mathcal{U}$ such that $V \subseteq U$.
\end{enumerate}
\end{remark}

\begin{example}Here we give a short list of locally convex spaces, and we also furnish a brief explanation of what the notions of convergence are.
\begin{enumerate}[label=\textbf{\arabic*)}]
\item The space of all continuous functions on the real line $C^0(\R; \, \R)$ equipped with the family of seminorms $P_K := \|\cdot\|_{\infty, \, K}$, where $K \subset \R$ is a compact subset. The notion of convergence is the uniform convergence on compact subsets.
\item The space of all differentiable functions on a bounded subset $C^k(\Omega)$ equipped with the family of seminorms $P_{\alpha, \, K} := \|D^\alpha \, \cdot\|_{\infty, \, K}$, where $K \subset \Omega$ is a compact subset. The notion of convergence is the uniform convergence of \textbf{all} derivatives on compact subsets.
\item The space of distributions $\mathcal{D}(\Omega)$.
\item The space of all sequences $\R^{\N}$ (or its subset made up of compactly supported sequences) is a locally convex space with the finer topology that makes the inclusions continuous).
\end{enumerate} \end{example}

\begin{theorem}[Seminorm Characterization] \label{theorem:seminor} Let $X$ be a vector space, and let $\mathcal{P} = \{ p_\alpha \}_{\alpha \in I}$ be a family of seminorms defined on $X$. The following properties hold true: \mbox{}
\begin{enumerate}[label=\textbf{(\alph*)}]
\item The ball $B_{p_\alpha}$ is absorbing, convex and balanced for every $\alpha \in I$. 
\item The family of neighborhoods
\begin{equation*} \mathcal{U} := \left\{ \bigcap_{ \alpha \in S \subset I} B_{p_\alpha} \: \left| \:  \text{$S$ finite subset of indices of $I$} \right. \right\} \end{equation*}
satisfies the properties \textbf{(1)}-\textbf{(5)} described above, and hence it defines a locally convex topology $\tau$ on $X$.
\item The extended family of seminorms, i.e.
\begin{equation*} \widetilde{\mathcal{P}} := \left\{ \widetilde{p}_S := \max_{\alpha \in S} p_\alpha \: \left| \: \text{$S$ finite subset of indices of $I$} \right. \right\} \end{equation*}
induces a family of balls
\begin{equation*} \widetilde{\mathcal{U}} := \left\{B_{\widetilde{p}_S} \: \left| \:  \text{$S$ finite subset of indices of $I$} \right. \right\} \end{equation*}
which defines the same locally convex topology $\tau$ of \textbf{(b)}.
\item Let $\mathcal{U}$ be a local basis of the origin made up of absorbing, convex and balanced sets, and let $\sigma$ be the locally convex topology induced on $X$. Then the family of the Minkowski functionals
\begin{equation*} \mathcal{P}  := \left\{ p_U \: \left| \: U \in \mathcal{U} \right. \right\} \end{equation*}
is a family of seminorms, which generates the same topology $\sigma$.
\end{enumerate}\end{theorem}

\begin{remark} Let $X$ be a topological vector space, and let $\mathcal{U}$ be a local basis of neighborhoods of the origin. The following topological properties hold:\mbox{}
\begin{enumerate}[label = \textbf{(\alph*)}]
\item The space $X$ is $\mathrm{T_1}$ if and only if $X$ is $\mathrm{T_0}$.
\item The space $X$ is $\mathrm{T_1}$ if and only if $\{0\}$ is closed if and only if $\bigcap_{U \in \mathcal{U}} U = \{0\}$ or, equivalently, if and only if $p(x) = 0$ for any $p \in \mathcal{P}$ implies $x = 0$.
\item The space $X$ is $\mathrm{T_0}$ if and only if $X$ is $\mathrm{T_2}$ (i.e., Hausdorff).
\item Assume that $(X, \, \tau)$ is a locally convex $\mathrm{T_0}$ space, and assume also that there exists a countable separated family of seminorms generating $\tau$. Then $X$ is metrizable.
\end{enumerate}\end{remark}

\begin{proof} \mbox{}
\begin{enumerate}[label = \textbf{(\alph*)}]
\item Any topological group which is $\mathrm{T}_0$ is automatically $\mathrm{T}_1$. The converse is always true.
\item Recall that a topological space is $\mathrm{T}_1$ if and only if each singlet $\{x\}$ is closed. But if $\{0\}$ is closed in a topological vector space, then every $\{x\}$ is closed as $x$ ranges in $X$.
\item Fix $x, \, y \in X$. The singlet $\{ x\}$ is always compact, while the singlet $\{y\}$ is closed because of the $\mathrm{T_1}$ assumption. Therefore, the existence of open neighborhoods separating $x$ and $y$ follows immediately from \hyperref[lemma:sep]{Lemma \ref{lemma:sep}}.
\item Let us set
\begin{equation}\label{eq:disadosods} d(x, \, y) := \max_{k \in \mathbb{N}} \left[ 2^{-k} \frac{p_k(x - y)}{1 + p_k(x - y)} \right].\end{equation}
The reader may check by herself that \eqref{eq:disadosods} is a distance on $X$. For example, it is easy to see that
\begin{equation*} d(x, \, y) = 0 \implies p_k(x - y) = 0 \quad \text{for all $k \in \N$},\end{equation*}
and this is enough to infer that $x$ is necessarily equal to $y$, as a consequence of the fact that the family of seminorms is separated.

Furthermore, the function $d$ is translation-invariant. Thus we only need to prove that the collection of open balls
\begin{equation*} B_r(0) := \{x \in X \: \left| \: d(x, \, 0) < r \right. \} \end{equation*}
is a local basis of neighborhoods of the origin, which induces the very same topology $\tau$.

\paragraph{Step 1.} The condition $d(x, \, 0) < r$ easily implies that
\begin{equation*} \left(2^{-k} - r \right) p_k(x) < r, \end{equation*}
and this inequality is automatically satisfied for all $k > \log_2 (1/r) =: r^\ast$. The finite number of remaining indices satisfy the inequality
\begin{equation*} p_k(x) < \frac{r}{2^{-k}-r} =: r_k,\end{equation*}
and thus $B_r$ contains the intersection of a finite number of $\mathcal{P}$-balls, that is,
\begin{equation*} \bigcap_{k = 0}^{\floor{r^\ast}} r_k \cdot B_{p_k} \subseteq B_r(0). \end{equation*}
In particular, the metric ball $B_r(0)$ contains a $\tau$-neighborhood of the origin.

\paragraph{Step 2.} Vice versa, given a $\tau$-neighborhood $V$ of the origin, it follows from the definition of local basis that one can always find positive real numbers $r_j > 0$ such that
\begin{equation*} \bigcap_{j = 0}^{m} r_j \cdot B_{p_j} \subseteq V. \end{equation*}
If we take $r$ satisfying the inequality
\begin{equation*} 2 r < \max \left\{ 2^{-j} \cdot r_j \: \left| \: j = 1, \, \dots, \, m \right. \right\}, \end{equation*}
then any $x \in B_r(0)$ satisfies the inequality
\begin{equation*} d(x, \, 0) < r < \frac{2^{-j} \cdot r_j}{2} \implies p_j(x) < r_j \quad \text{for all $j = 1, \, \dots, \, m$}. \end{equation*}
Consequently, the metric ball $B_r(0)$ is contained in the finite intersection of the balls $r_j \cdot B_{p_j}$, and therefore it is also contained in $V$.
\end{enumerate}\end{proof}

\begin{theorem} A $\mathrm{T_0}$ topological vector space $X$ is second-countable if and only if it is metrizable.  \end{theorem}

\begin{proof}[Hint] Take a countable basis of neighborhoods of the origin such that each element is balanced and monotone (with respect to the inclusion) $U_{n+1} + U_{n+1} \subseteq U_n$. Then define
\begin{equation*}\rho(x) = \inf \left\{ \sum_{i=1}^{n} 2^{-k_i} \: \left| \: k_i \in \N, \, \, x \in U_{k_1}+\dots+U_{k_n} \right. \right\} \end{equation*}
and prove that $\rho(x - y) = d(x, \, y)$ is a metric on $X$.

The proof of this theorem is rather technical and I will not write it down, but it can be found e.g. in \textbf{W. Rudin, Functional Analysis} (Theorem 1.24).
\end{proof}

\section{The Uniform Boundedness Principle}

Let $X$ be a topological vector space. In this section, we shall denote by $\mathcal{U}_p(X)$ a local basis of neighborhoods around $p \in X$.

\begin{definition}[Bounded]\index{topological vector space!bounded set} A set $E \subset X$ is \textit{bounded} if and only if for every $U \in \mathcal{U}_0(X)$ there exists $r > 0$ such that $E \subseteq r \cdot U$. \end{definition}

\begin{remark} If $X$ is a metric space, a set is bounded in the sense above if and only if it is bounded with respect to the metric. \end{remark}

\begin{lemma} \label{lemma.closureclos} Let $X$ be a topological vector space and let $S \subset X$ be any subset. The closure of $S$ is given by the (possibly) infinite intersections of the $\mathcal{U}$-translations:
\begin{equation*} \overline{S} = \bigcap_{U \in \mathcal{U}_0(X)} S + U. \end{equation*} \end{lemma}

In particular, any point $x \in X$ admits a local basis of \textit{closed neighborhoods}. Indeed, it is straightforward to prove that
\begin{equation*} \left\{ V + x \: \left| \: V \in \mathcal{U}_0(X) \right. \right\} \end{equation*}
is a local basis of neighborhoods of $x$, and hence
\begin{equation*} \left\{ \overline{V} + x \: \left| \: V \in \mathcal{U}_0(X) \right. \right\} \end{equation*}
is also a local basis.

\begin{lemma} Let $X$ be a topological vector space, and let $E \subset X$ be a bounded subset. Then the closure $\overline{E}$ is also bounded.\end{lemma}

\begin{proof} By definition, for every $U \in \mathcal{U}_0(X)$ there exists a positive real number $r > 0$ such that $E \subset r \cdot U$. Hence, by taking the closure we obtain the chain of inclusions
\begin{equation*} \overline{E} \subseteq r \cdot \overline{U} \subset (r + \epsilon) \cdot U, \end{equation*}
for some $\epsilon > 0$, as a consequence of \hyperref[lemma.closureclos]{Lemma \ref{lemma.closureclos}}.
\end{proof}

\begin{lemma} Let $X$ and $Y$ be topological vector spaces. If $L \in \mathcal{L}(X, \, Y)$ is a linear and continuous operator, then it is bounded. The converse is true if, e.g., $X$ is locally bounded. \end{lemma}

\begin{theorem}A topological vector space $X$ is normable if and only if it is both locally convex and locally bounded.\end{theorem}

We now present a simple criterion to show whether or not a set is bounded looking at the behavior of all the sequences.

\begin{lemma} Let $X$ be topological vector space and let $E \subset X$. Then $E$ is bounded if and only if for every sequence $(\alpha_n)_{n \in \N} \subset \mathbb{K}$ converging to $0$ and every sequence $(x_n)_{n \in \N} \subset E$, it turns out that
\begin{equation*} \alpha_n x_n = o(1) \quad \text{for $n \to + \infty$}. \end{equation*}  \end{lemma}

\begin{proof} Suppose that $E$ is a bounded subset of $X$. Fix $U \in \mathcal{U}_0(X)$ neighborhood of the origin in $X$, and let $r > 0$ be the positive real number such that $E \subset r \cdot U$. It follows that
\begin{equation*} x_n \in r \cdot U \implies \frac{x_n}{r} \in U \quad \text{for $n \in \N$}. \end{equation*}
Therefore, we can always choose $N \in \N$ in such a way that $\alpha_N < 1/r$, that is,
\begin{equation*}  \alpha_n x_n \in U \quad \text{for all $n \geq N$}. \end{equation*}
The opposite implication is trivial (e.g., one could consider $x_n := x$ to be the constant sequence). \end{proof}

\begin{corollary} If $E$ is relatively compact (or relatively sequentially compact), then $E$ is bounded. \end{corollary}

\begin{remark} Let $(X, \, \|\cdot\|)$ be a normed space. Then: \mbox{}
\begin{enumerate}[label=\textbf{(\arabic*)}]
\item A subset $E \subset X$ is strongly bounded if and only if $\sup_{x \in E} \|x\| < +\infty$.
\item A subset $E \subset X$ is weakly bounded if and only if for all $f \in X^\ast$ we have $\sup_{x \in E} \langle f, \, x \rangle < +\infty$.
\item A subset $E \subset X^\ast$ is weakly-$\ast$ bounded if and only if for all $u \in X$ we have $\sup_{f \in E} \langle f, \, u \rangle < +\infty$.
\end{enumerate}\end{remark}

\begin{notation} Let $X$ and $Y$ be topological vector spaces, and let $\Gamma$ be a family (eventually uncountable) of linear and continuous applications from $X$ to $Y$. For every set $S \subset X$, we denote by $\Gamma(S)$ the union of the images, that is
\begin{equation*} \Gamma(S) := \bigcup_{T \in \Gamma} T(S), \end{equation*}
and, for every $R \subset Y$, we denote by $\Gamma^{-1}(R)$ the intersection of the preimages, that is
\begin{equation*} \Gamma^{-1}(R) := \bigcap_{T \in \Gamma} T^{-1}(R). \end{equation*}
In particular, the inclusion $\Gamma(S) \subset R$ is a compact way to express that
\begin{equation*} T(S) \subset R \quad \text{for every $T \in \Gamma$}, \end{equation*}
and, similarly, it is also equivalent to the inclusion
\begin{equation*} S \subset T^{-1}(R) \quad \text{for every $T \in \Gamma$}. \end{equation*}\end{notation}

\begin{definition}[Equicontinuous] \index{equicontinuous} Let $X$ and $Y$ be topological vector spaces. A family $\Gamma \subset \mathcal{L}(X, \, Y)$ of linear and continuous applications is \textit{equicontinuous} if and only if
\begin{equation*} \forall \, V \in \mathcal{U}_0(Y), \, \exists U \in \mathcal{U}_0(X)\: : \: \Gamma(U) \subset V. \end{equation*} \end{definition}

\begin{remark} If $X$ and $Y$ are metric spaces, this notion is completely equivalent to the equicontinuity in the sense of $\epsilon$-$\delta$.\end{remark}

\begin{remark}If $X$ and $Y$ are normed spaces, a family $\Gamma$ is equicontinuous if and only if $\Gamma$ is equibounded in $\mathcal{L}(X, \, Y)$ with respect to the operator norm.\end{remark}

\begin{definition}[Meager Set] \index{meager set} Let $(X, \, \tau)$ be a topological space, and let $S \subset X$. We say that $S$ is a \textit{meager} (or \textit{first-category}) set if and only if there exists a countable cover made up of nowhere dense subsets of $X$, that is,
\begin{equation*} S = \bigcup_{n \in \N} X_n, \qquad \mathrm{Int} \, \overline{X_n} = \varnothing. \end{equation*}
Furthermore, we say that $S$ is a \textit{second-category} set if it is not a first-category set. \end{definition}

\begin{theorem}[Baire] \index{Baire Theorem} \label{bairetheorem}Let $X$ be either a complete metric space or a locally compact topological space. Then each open nonempty subset of $X$ is a second-category set. \end{theorem}

\begin{theorem}[Banach-Steinhaus]\index{Banach-Steinhaus Theorem} \label{stein}Let $X$ and $Y$ be topological vector spaces and let $\Gamma \subset \mathcal{L}(X, \, Y)$ be a collection (eventually uncountable) of linear and continuous applications. If
\begin{equation*} E := \left\{ x \in X \: \left| \: \text{$\Gamma(\{x\})$ is bounded} \right. \right\} \end{equation*}
is a second-category set (i.e., $\Gamma$ is pointwise bounded in a second-category set), then $\Gamma$ is a equicontinuous family. \end{theorem}

\begin{proof} Fix $U \in \mathcal{U}_0(Y)$ neighborhood of the origin in $Y$. Recall that we can always find a neighborhood $V \in \mathcal{U}_0(Y)$ closed, balanced and satisfying the the inclusion $V + V \subset U$.

By assumption, for all $x \in E$ there exists a positive natural number $m(x) \in \N$ such that $\Gamma(x) \subset m(x) \cdot V$ (or, equivalently, $x \in m(x) \cdot \Gamma^{-1}(V)$). In particular, we have that
\begin{equation*} E \subseteq \bigcup_{n \in \N} n \cdot \Gamma^{-1}(V) \quad \text{and} \quad \text{$\Gamma^{-1}(V)$ closed}. \end{equation*}
Since $E$ is a second-category set, there exists $m \in \N$ such that $m \cdot \Gamma^{-1}(V)$ has nonempty internal part (and, hence, the same applies to $n \cdot \Gamma^{-1}(V)$ for every $n \in \N$). In particular, it turns out that
\begin{equation*}\mathfrak{V} := \Gamma^{-1}(V) - \Gamma^{-1}(V) \in \mathcal{U}_0(X), \end{equation*}
and by linearity of $\Gamma$ we also have that
\begin{equation*} \Gamma(\mathfrak{V}) = \Gamma(\Gamma^{-1}(V)) - \Gamma(\Gamma^{-1}(V)) = V - V \subseteq U. \end{equation*}
Therefore, the set $\mathfrak{V}$ is a neighborhood of the origin in $X$ whose image is contained in $U$, and this is exactly what we wanted to prove.\end{proof}

\begin{remark} The theorem is false if $\Gamma$ is a family of linear (but not necessarily continuous) operators from $X$ to $Y$. The easiest counterexample is the following: Consider a Banach space $X$, and let $Y := \R$ and $\Gamma := \{L\}$, for any $L$ discontinuous linear functional. \end{remark}

\paragraph{Applications.} In this brief paragraph, we investigate some basic results which may be obtained through a simple application of the uniform boundedness principle.

\begin{theorem} Let $X$ be a complete metric space, and let $Y$ and $Z$ be topological vector spaces. If
\begin{equation*} B : X \times Y \longrightarrow Z \end{equation*}
is a bilinear application separately sequentially continuous, then $B$ is jointly sequentially continuous (i.e., seq. continuous w.r.t. the couple). \end{theorem}

\begin{proof} We first prove a particular case, and then we generalize it with a simple algebraic trick.

\paragraph{Step 1.} Let $(x_n)_{n \in \N} \subset X$ be a sequence converging to $0$, and let $(y_n)_{n \in \N} \subset Y$ be a converging sequence. The linear mapping
\begin{equation*} B(\cdot, \, y_n) : X \longrightarrow Z \end{equation*}
is clearly continuous for every $n \in \N$, and it is pointwise bounded. Indeed, for $u \in X$ fixed, the subset
\begin{equation*} \left\{ B(u, \, y_n) \right\}_{n \in \N} \subset Z \end{equation*}
is bounded in $Z$, since converging sequences are bounded. The \hyperref[bairetheorem]{Baire Theorem \ref{bairetheorem}} holds in the complete metric space $X$, and hence the \hyperref[stein]{Banach-Steinhaus Theorem \ref{stein}} implies that for every $U \in \mathcal{U}_0(Z)$ there exists $V \in \mathcal{U}_0(X)$ such that
\begin{equation*} B \left( V, \, y_n \right) \subset U \quad \text{for every $n \in \N$}. \end{equation*}
The sequence $(x_n)_{n \in \N}$ converges to $0$; thus, $x_n \in V$ definitively, and this means that $B(x_n, \, y_n)$ belongs to $U$ for $n$ sufficiently large, that is, $B(x_n, \, y_n) \to 0$.

\paragraph{Step 2.} If $x_n \to x \in X$, then the thesis follows from the previous step if one notices that the difference $(x_n - x)_{n \in \N}$ converges to $0$. Then
\begin{equation*} B(x_n, \, y_n) = B(x_n - x, \, y_n) + B(x, \, y_n) \xrightarrow{n \to + \infty} 0 + B(x, \, y) = B(x, \, y).\end{equation*}
\end{proof}

\begin{corollary} Let $(T_n)_{n \in \N} \subset \mathcal{L}(X, \, Y)$ be a family of linear and continuous operators between two Banach spaces, and assume that the sequence is pointwise bounded, that is,
\begin{equation*}\sup_{n \in \N} \|T_n(x) \| < + \infty \quad \text{for all $x \in X$}. \end{equation*}
Then $T_n$ is equibounded,that is, there exists a positive constant $C > 0$ such that
\begin{equation*}\sup_{n \in \N} \|T_n\|_{\mathcal{L}(X, \, Y)} \leq C < + \infty. \end{equation*} \end{corollary}

\begin{remark} Let $(T_n)_{n \in \N} \subset \mathcal{L}(X, \, Y)$ be a family of linear and continuous operators between two Banach spaces, and suppose that
\begin{equation*} T_n(x) \xrightarrow{n \to + \infty} T(x) \quad \text{w.r.t. the strong convergence in $X$}. \end{equation*}
Then $T_n \to T \in \mathcal{L}(X, \, Y)$ in the operator norm, and the convergence is uniform on any compact set. \end{remark}

\begin{proof}The sequence $(T_n)_{n \in \N}$ is pointwise bounded. Therefore, the previous corollary asserts that it is also equibounded, that is, there exists a positive constant $c$ such that
\begin{equation*} \sup_{n \in \N} \|T_n \| = c < \infty. \end{equation*}
It follows that
\begin{equation*} \| T_n  x\| \leq c \| x\| \quad \text{for all $x \in X$}, \end{equation*}
which means that $T_n$ is also an equi-Lipschitz continuous sequence.

The Ascoli-Arzelà theorem allows us to extract a subsequence $(n_k)_{k \in \N}$ such that $T_{n_k}$ converges, uniformly on compact sets, to a linear operator $T$. Furthermore, the assumption
\begin{equation*} \| T_n  x\| \leq c \| x\| \quad \text{for all $x \in X$}, \end{equation*}
is preserved under the limit for $n \to + \infty$, and therefore $T$ is a continuous operator, which means that
\begin{equation*}T_{n_k} \xrightarrow{k \to + \infty} T \in \mathcal{L}(X, \, Y), \end{equation*}
uniformly on compact sets.\end{proof}

\section{The Open Mapping Theorem}

The statement and the proof of the open mapping theorem are both taken \textbf{verbatim} from the Brezis book \cite{brezis}.

\begin{theorem}[Open Mapping] \index{Open Mapping Theorem} \label{opt}Let $X$ and $Y$ be two Banach spaces, and let $T$ be a continuous linear operator from $X$ to $Y$ that is surjective (=onto). Then there exists a constant $c > 0$ such that
\begin{equation*} T \left( B_X(0, \, 1) \right) \supset B_Y(0, \, c). \end{equation*} \end{theorem}

\begin{remark} The conclusion of the open mapping theorem is equivalent to saying that $T$ is an open map. Indeed, let us suppose that $U$ is open in $X$ and let us prove that $T(U)$ is open in $Y$.

Fix any point $y_0 \in T(U)$, so that $y_0 = T \, x_0$ for some $x_0 \in U$. Let $r > 0$ be such that $B(x_0, \, r) \subset U$, i.e., $x_0 + B(0, \, r) \subset U$. It follows that
\begin{equation*} y_0 + T \left( B(0, \, r) \right) \subset T(U). \end{equation*}
Then by open mapping theorem it follows that
\begin{equation*} T \left( B(0, \, r) \right) \supset B(0, \, rc) \end{equation*}
and therefore
\begin{equation*} B(y_0, \, rc) \subset T(U). \end{equation*} \end{remark}

\begin{proof} We split the argument into two steps.

\paragraph{Step 1.} Assume that $T$ is a linear surjective operator from $X$ onto $Y$. Then there exists a constant $c > 0$ such that
\begin{equation*} \overline{T \left(B_X(0, \, 1) \right)} \supset B_Y(0, \, 2c). \end{equation*}
Set $X_n = n \cdot\overline{T \left(B_X(0, \, 1) \right)}$. Since $T$ is surjective, we clearly have
\begin{equation*} Y = \bigcup_{n \in \N} X_n, \end{equation*}
and by Baire category theorem there exists some $N$ such that $\mathrm{Int} X_N \neq \varnothing$. It follows that the same holds true (by homeomorphisms) for any other $X_n$ and thus
\begin{equation*} \mathrm{Int} \left(\overline{T \left(B_X(0, \, 1) \right)} \right) \neq \varnothing. \end{equation*}
Pick $c > 0$ and $y_0 \in Y$ such that
\begin{equation*} B_Y(y_0, \, 4 c) \subset \overline{T \left(B_X(0, \, 1) \right)}. \end{equation*}
In particular, $y_0 \in \overline{T \left(B_X(0, \, 1) \right)}$, and by symmetry, 
\begin{equation*} - y_0 \in \overline{T \left(B_X(0, \, 1) \right)}. \end{equation*}
Therefore
\begin{equation*} B_Y(0, \, 4c) \subset \overline{T \left(B_X(0, \, 1) \right)} + \overline{T \left(B_X(0, \, 1) \right)}. \end{equation*}
On the other hand, since $ \overline{T \left(B_X(0, \, 1) \right)}$ is convex, we have
\begin{equation*}\overline{T \left(B_X(0, \, 1) \right)} + \overline{T \left(B_X(0, \, 1) \right)} = 2 \cdot \overline{T \left(B_X(0, \, 1) \right)}, \end{equation*}
and the first claim follows.

\paragraph{Step 2.} Assume that $T$ is continuous linear operator from $X$ into $Y$ that satisfies the first claim. Then we have
\begin{equation*} T \left(B_X(0, \, 1) \right) \supset B_Y(0, \, c). \end{equation*}
Choose any $y \in Y$ with $\|y\| < c$. The goal is to find some $x \in X$ such that
\begin{equation*} \|x\| < 1 \qquad \text{and} \qquad T x = y. \end{equation*}
We know that
\begin{equation*} \text{$\forall \epsilon > 0, \:\: \exists \, z \in X$ such that $\|z\| < \frac{1}{2}$ and $\| y - T z \| < \epsilon$}.\end{equation*}
Choosing $\epsilon = c/2$, we find some $z_1 \in X$ such that
\begin{equation*} \| z_1 \| < \frac{1}{2} \qquad \text{and} \qquad \| y - T z_1 \| < \frac{c}{2}. \end{equation*}
Proceeding similarly, by induction we obtain a sequence $(z_n)_{n \in \N} \subset X$ such that
\begin{equation*} \| z_n \| < \frac{1}{2^n} \qquad \text{and} \qquad \| y - T (z_1 + \dots + z_n) \| < \frac{c}{2^n}, \quad \text{for any $n \in \N$}. \end{equation*}

It follows that the sequence $x_n = z_1 + \dots + z_n$ is a Cauchy sequence. Let $x$ be its limit (since $X$ is complete) with, clearly, $\|x\| < 1$ and $y = Tx$ (since $T$ is continuous).
\end{proof}

\begin{corollary} \label{corqeuodd}Let $X$ and $Y$ be Banach spaces, and let $T$ be a bijective continuous and linear operator from $X$ to $Y$. Then $T^{-1}$ is a continuous operator from $Y$ to $X$. \end{corollary}

\begin{proof} The \hyperref[opt]{Open Mapping Theorem \ref{opt}}, together with the assumption that $T$ is injective, is enough to infer that
\begin{equation*} x \in X \: : \: \|T x\|_Y < c \implies \|x \|_X < 1. \end{equation*}
By homogeneity, it turns out that
\begin{equation*} \|x\|_X \leq \frac{1}{c} \, \|T x \|_Y, \qquad \forall \, x \in X, \end{equation*}
and hence $T^{-1}$ is continuous.\end{proof}

\begin{corollary} \label{corwequ}Let $X$ be a vector space provided with two norms, $\| \cdot \|_1$ and $\| \cdot \|_2$. Assume that $X$ is Banach with respect to both norms, and assume that there exists a constant $C \geq 0$ such that
\begin{equation*} \|x\|_2 \leq C \, \|x\|_1, \qquad \forall \, x \in X. \end{equation*}
Then the two norms are \textbf{equivalent}, i.e., there is a constant $c > 0$ such that
\begin{equation*} \|x\|_1 \leq c \, \|x\|_2, \qquad \forall \, x \in X. \end{equation*} \end{corollary}

\begin{proof}It suffices to apply \hyperref[corqeuodd]{Corollary \ref{corqeuodd}} with
\begin{equation*} X:= (X, \, \| \cdot \|_1), \qquad Y := (X, \, \| \cdot \|_2) \quad \text{and} \quad T = \mathrm{id}_X. \end{equation*} \end{proof}

\begin{remark} The assumption that the two norms are comparable is critical. Indeed, if $(X, \, \|\cdot\|)$ is an infinite-dimensional Banach space, then one can always find a linear form $\ell : X \longrightarrow \R$ which is not continuous. Therefore, we may define a non-continuous operator $A : X \longrightarrow X$ as follows. Pick a point $x_0 \in X$ such that $T x_0 = 1$, and set
\begin{equation*} A(x) := x - 2 x_0 \, \langle \ell, \, x \rangle. \end{equation*}
We can easily define a norm
\begin{equation*} \|x\|_A := \| A x \|_X, \end{equation*}
which makes $A$ an isometry of $X$ (and thus continuous), if seen as an operator between $(X, \, \|\cdot\|_X)$ and $(X, \, \| \cdot \|_A)$.

The reader may check as an exercise that the corollary above fails since $(X, \, \| \cdot \|_A)$ is also a Banach space, but $\|\cdot\|_A$ and $\| \cdot \|$ are not equivalent norms.\end{remark}
 
\begin{theorem}[Closed Graph] \index{Closed Graph Theorem} Let $X$ and $Y$ be two Banach spaces, and let $T : X \longrightarrow Y$ be a linear operator. Then $T$ is continuous if and only if the graph of $T$, denoted by $\Gamma(T)$, is closed in the Cartesian product $X \times Y$. \end{theorem}

\begin{proof} First, we notice that the graph
\begin{equation*} \Gamma(T) := \left\{ \left(x, \, T x\right) \: \left| \: x \in X \right. \right\} \end{equation*}
is the kernel of the map
\begin{equation*} X \times Y \ni (x, \, y) \longmapsto T x - y. \end{equation*}

\paragraph{Step 1.} If $T$ is continuous, then the map $(x, \, y) \longmapsto T x - y$ is also continuous. Therefore, the preimage of the singlet $\{0\}$, which is closed, is exactly equal to the graph $\Gamma(T)$.

\paragraph{Step 2.} Vice versa, let us consider on $X$ the two norms
\begin{equation*} \|x\|_1 := \|x\|_X + \|T x\|_Y \qquad \text{and} \qquad \|x\|_2 := \|x\|_X. \end{equation*}
It is easy to check, using the assumption that $G(T)$ is closed, that $X$ is a Banach space for the norm $\| \cdot \|_1$. On the other hand, by assumption, $X$ is also a Banach space for $\|\cdot\|_2$ and clearly
\begin{equation*} \|x\|_1 \geq \|x\|_2 .\end{equation*}
It follows from \hyperref[corwequ]{Corollary \ref{corwequ}} that the two norms are equivalent; hence there exists a constant $c > 0$ such that $\|x\|_1 \leq c \, \|x\|_2$, from which it follows that
\begin{equation*} \|T x\|_Y \leq c^\prime \, \|x\|_X, \end{equation*}
i.e. $T$ is continuous.
\end{proof}

%\section{Exercises}

%\begin{exercise} Let $X$ and $Y$ be two Banach spaces and let $T$ be a continuous linear operator from $X$ to $Y$. Assume that $\mathrm{R}(T) \subset Y$ is a linear space of finite co-dimension in $Y$. Prove that $\mathrm{R}(T)$ is closed, and thus $T : X \to T(X)$ is an open map.
%\end{exercise}

%\begin{exercise}  Let $X$ and $Y$ be two Banach spaces and let $T$ be a surjective continuous linear operator from $X$ onto $Y$. Then
%\begin{equation*}\begin{tikzcd}%
%X \arrow{r}{T} \arrow{dr}{\pi} & Y \\
% & \sfrac{X}{\mathrm{Ker}(T)} \arrow[dotted]{u}{T^\prime} 
%\end{tikzcd} \end{equation*}
%the operator $T^\prime$ induced on the quotient by $T$ is \textbf{invertible}.
%\end{exercise}