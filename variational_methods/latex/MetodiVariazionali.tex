\title{Variational Methods  - Lecture Notes}
\author{Francesco Paolo Maiale
        }
        
\documentclass[a4paper,10 pt]{report}

%Graphics/Geometry Packages
\usepackage{graphicx}
\usepackage[dvipsnames]{xcolor}
\usepackage[labelfont=bf]{caption}
\usepackage[pass]{geometry}
\usepackage{setspace}
\usepackage{fancyhdr}

%AMS Packages
\usepackage{amsthm, amsmath, amssymb, amsfonts}

%Language Packages
\usepackage[english]{babel}
\usepackage[utf8]{inputenc}
\usepackage{etoolbox} 

\patchcmd{\part}{plain}{empty}{}{}

%Mathematics Packages
\usepackage{setspace}
%\usepackage{mathabx}
\usepackage{faktor}
\usepackage{mathrsfs}  
\usepackage{enumitem}
\usepackage{mathtools}
%\usepackage{accents}
%\usepackage{pifont}
%\usepackage{mwe}
%\usepackage{bbm}
\usepackage{PTSansNarrow}
\usepackage[T1]{fontenc}
%\usepackage{epigraph}

%Misc Packages
\usepackage{scalerel}[2014/03/10]
\usepackage[usestackEOL]{stackengine}
\usepackage{ragged2e}
\usepackage[framemethod=tikz]{mdframed}
\usepackage{marginnote}
\usepackage{xparse}
\usepackage{hyperref}
\usepackage{comment}

%TIKZ Packages
\usepackage{tikz-cd}
\usepackage{tikzpagenodes}
\usetikzlibrary{calc}
\usetikzlibrary{matrix}
\usetikzlibrary{plotmarks}

\usepackage{refcheck}

\usepackage{makeidx}

\usepackage[Sonny]{fncychap}

\makeindex %DDD

\hypersetup{
    colorlinks=true,
    linkcolor=cyan,
    filecolor=magenta,      
    urlcolor=cyan,
}


\DeclareRobustCommand\longtwoheadrightarrow
     {\relbar\joinrel\twoheadrightarrow}

\newcommand{\notimplies}{%
  \mathrel{{\ooalign{\hidewidth$\not\phantom{=}$\hidewidth\cr$\implies$}}}}
  
\pagestyle{plain}
\setlength{\topmargin}{0.0in}
\setlength{\headheight}{0.2in}
\setlength{\headsep}{0.2in}
\setlength{\footskip}{0.5in}
\setlength{\footnotesep}{0.15in}
\setlength{\textheight}{8.3in}
\setlength{\textwidth}{5.5in} % 6
\setlength{\oddsidemargin}{0.5in}
\setlength{\evensidemargin}{0.5in}
\setlength{\parindent}{0.2 in}
\setlength{\parskip}{0.1 in}
\setlength{\marginparwidth}{1.2 in}
\linespread{1.1} 


\newtheorem{theorem}{Theorem}[chapter]
\newtheorem{lemma}[theorem]{Lemma}
\newtheorem{proposition}[theorem]{Proposition}
\newtheorem{corollary}[theorem]{Corollary}
\theoremstyle{definition}
\newtheorem{definition}[theorem]{Definition}

\newtheorem{remark}[theorem]{Remark}
\newtheorem{example}[theorem]{Example}
\newtheorem*{notation}{Notation}
\newtheorem*{claim}{Claim}
\newtheorem{exercise}{Exercise}[chapter]

\newtheorem{innercustomthm}{Theorem}
\newenvironment{customthm}[1]
  {\renewcommand\theinnercustomthm{#1}\innercustomthm}
  {\endinnercustomthm}


\newcommand{\smallO}[1]{\scriptstyle\mathcal{O}}
\DeclarePairedDelimiter\floor{\lfloor}{\rfloor}
\newcommand*\conj[1]{\overline{#1}}
\newcommand{\R}{\mathbb R}

\definecolor{darkjunglegreen}{rgb}{0.1, 0.14, 0.13}
\newcommand{\C}{\mathbb C}
\newcommand{\cC}{\mathcal C}
\newcommand{\cN}{\mathcal N}
\newcommand{\scC}{\mathscr C}
\newcommand{\scH}{\mathscr H}
\newcommand{\N}{\mathbb N}
\newcommand{\G}{\mathcal G}
\newcommand{\Om}{\Omega}
\newcommand{\Z}{\mathbb Z}
\newcommand{\con}{\mathcal{C}\left(x, \, V, \, \alpha \right)}
\newcommand{\Q}{\mathbb Q}
\newcommand{\D}{\mathcal D}
\newcommand{\B}{\mathcal B}
\newcommand{\M}{\mathcal M}
\newcommand{\p}{\mathcal P}
\newcommand{\X}{\mathfrak X}
\newcommand{\cH}{\mathcal H}
\newcommand{\cA}{\mathcal A}
\newcommand{\e}{\rm e}
\newcommand{\cF}{\mathcal F}
\newcommand{\cG}{\mathcal G}
\newcommand{\cL}{\mathcal L}
\newcommand{\dr}{\mathrm d}
\newcommand{\scM}{\mathscr M}
\newcommand{\inv}{\mathrm {Inv}}
\newcommand{\homs}{\mathrm {Hom}}
\newcommand{\T}{\mathbb T}
\newcommand{\Y}{\mathfrak Y}
\newcommand{\Le}{\mathcal{L}}
\newcommand{\divs}{\mathrm{div}}
\newcommand{\can}{\symbol{35}}
\newcommand*{\double}[2][.1ex]{%
  \mathrel{\vcenter{\offinterlineskip%
  \hbox{$#2$}\vskip#1\hbox{$#2$}}}}
\newcommand*{\doublerightarrow}{\double{\longrightarrow}}

\newcommand{\restr}{%
  \,\raisebox{-.127ex}{\reflectbox{\rotatebox[origin=br]{-90}{$\lnot$}}}\,%
}

\newcommand{\bd}{\begin{definition}}
\newcommand{\ed}{\end{definition}}
\newcommand{\bl}{\begin{lemma}}
\newcommand{\el}{\end{lemma}}
\newcommand{\bthm}{\begin{theorem}}
\newcommand{\ethm}{\end{theorem}}
\newcommand{\bpr}{\begin{proposition}}
\newcommand{\epr}{\end{proposition}}
\newcommand{\brmk}{\begin{remark}}
\newcommand{\ermk}{\end{remark}}
\newcommand{\bcor}{\begin{corollary}}
\newcommand{\ecor}{\end{corollary}}
\newcommand{\bex}{\begin{example}}
\newcommand{\eex}{\end{example}}

\makeatletter
\renewcommand*\env@matrix[1][*\c@MaxMatrixCols c]{%
  \hskip -\arraycolsep
  \let\@ifnextchar\new@ifnextchar
  \array{#1}}
\makeatother

% Caution Form Code START

\newcounter{mycaution}
\newcommand\pointeranchor{}
\newcommand\boxanchor{}
\newlength\boxvshift
\newlength\uppertrianglecorner

\newcommand\tikzmark[1]{%
  \tikz[remember picture,overlay]\node[inner xsep=0pt,outer sep=0pt] (#1) {};}

\NewDocumentCommand{\caution}{O{c}O{BrickRed}O{Caution!}m}{%
\stepcounter{mycaution}%
\tikzmark{\themycaution}%
\if#1b\relax
\renewcommand\pointeranchor{mybox\themycaution.south east}%
\renewcommand\boxanchor{south east}%
\setlength\boxvshift{-10pt}%
\setlength\uppertrianglecorner{13pt}%
\else
\if#1t\relax
\renewcommand\pointeranchor{mybox\themycaution.north east}%
\renewcommand\boxanchor{north east}%
\setlength\boxvshift{10pt}%
\setlength\uppertrianglecorner{-7pt}%
\else
\if#1c\relax
\renewcommand\pointeranchor{mybox\themycaution.east}%
\renewcommand\boxanchor{east}%
\setlength\boxvshift{0pt}%
\setlength\uppertrianglecorner{3pt}%
\fi\fi\fi%
\begin{tikzpicture}[remember picture,overlay]
\node[draw=#2,anchor=\boxanchor,xshift=-\marginparsep,yshift=\boxvshift]   
  (mybox\themycaution)
  at ([yshift=3pt]current page text area.west|-\themycaution) 
  {\parbox{\marginparwidth}{\vskip10pt\RaggedRight\small#4}};
\node[fill=white,font=\color{#2}\sffamily,anchor=west,xshift=7pt]
  at (mybox\themycaution.north west) {\ #3\ };
\fill[#2]
  ([yshift=\uppertrianglecorner]\pointeranchor) --
  ([yshift=\uppertrianglecorner-3pt,xshift=3pt]\pointeranchor) --
  ([yshift=\uppertrianglecorner-6pt]\pointeranchor) -- cycle;
\end{tikzpicture}%
}

%% Caution Form Code END

\definecolor{deepjunglegreen}{rgb}{0.0, 0.29, 0.29}

\newcommand{\bsquare}{\item[\color{magenta}\ding{110}]} 
\newcommand{\barrow}{\item[\color{blue}\ding{228}]}
\newcommand{\bwarrow}{\item[\color{gray}\ding{227}]}

\def\dashint{\,\ThisStyle{\ensurestackMath{%
  \stackinset{c}{.2\LMpt}{c}{.5\LMpt}{\SavedStyle-}{\SavedStyle\phantom{\int}}}%
  \setbox0=\hbox{$\SavedStyle\int\,$}\kern-\wd0}\int}
\def\ddashint{\,\ThisStyle{\ensurestackMath{%
  \stackinset{c}{.2\LMpt}{c}{.5\LMpt+.2\LMex}{\SavedStyle-}{%
    \stackinset{c}{.2\LMpt}{c}{.5\LMpt-.2\LMex}{\SavedStyle-}{%
      \SavedStyle\phantom{\int}}}}\setbox0=\hbox{$\SavedStyle\int\,$}\kern-\wd0}\int}


\fancyhf{}
% Put the page number at the right edge of odd pages, and left edge of even pages.
\fancyhead[LO,RE]{\textbf \thepage}
% Custom text at the left edge of odd pages, and right edge of odd pages.
\fancyhead[RO]{ \slshape\nouppercase{\rightmark}}
\fancyhead[LE]{ \slshape\nouppercase{\leftmark}}

% Repeat for \fancyfoot if needed, e.g.
% Some decorative symbol at the centre of both odd and even pages
\fancyfoot[C]{ }

% Set this length to 0pt if you don't want any lines!
\renewcommand{\headrulewidth}{1pt}


% Apply the fancy header style
\pagestyle{fancy}

\begin{document}
\newpage \thispagestyle{empty}

\begin{center}

\begin{spacing}{1.5}
{\huge  \sf Lecture Notes}\\
\vspace*{\fill}
\end{spacing}
\begin{spacing}{2.5}
\textbf{\huge Variational Methods}\\[0.5cm]
\vspace*{\fill}
\begin{minipage}{5cm}
\centering {\textit{Course held by}}
\end{minipage}
\hspace*{\fill}
\begin{minipage}{5cm}
\centering {\textit{Notes written by}} \\
\end{minipage}
\end{spacing}

\begin{spacing}{1.3}

\begin{minipage}{5cm}
\centering {\textbf{\large Prof. Andrea Malchiodi}}
\end{minipage}
\hspace*{\fill}
\begin{minipage}{5cm}
\centering {\textbf{\large Francesco Paolo Maiale}}
\end{minipage}

\vspace*{\fill}

\textnormal{\large Scuola Normale, Pisa \\[0.4em] \today}

\end{spacing}
\end{center}

\newpage \thispagestyle{empty}
\begin{center}
\vspace*{1cm}
\begin{spacing}{2.5}
 {
\textbf{\huge Disclaimer}
}
\end{spacing} \end{center}

\begin{spacing}{1.2}
I wrote these notes to summarise the content of the course on Variational Methods, held by Professor Andrea Malchiodi at SNS.

I tried to include all the topics that were discussed in class and combine it with some additional information from several other courses to produce a self-contained document.

I will try to review them periodically, but I am sure that at the end there will be a large number of mistakes and oversights. To report them, feel free to send me an email at \textbf{francesco (dot) maiale (at) sns (dot) it}.
\end{spacing}


{ \setlength{\parskip}{0.05 in}

\clearpage                       % Otherwise \pagestyle affects the previous page.
{                                % Enclosed in braces so that re-definition is temporary.
  \pagestyle{empty}              % Removes numbers from middle pages.
  \fancypagestyle{plain}         % Re-definition removes numbers from first page.
  {
    \fancyhf{}%                       % Clear all header and footer fields.
    \renewcommand{\headrulewidth}{0pt}% Clear rules (remove these two lines if not desired).
    \renewcommand{\footrulewidth}{0pt}%
  }
  \tableofcontents
  \thispagestyle{empty}          % Removes numbers from last page.
}}

\part{Nonlinear Analysis}

\chapter{Introduction}

In this chapter, we introduce the main topics of the course and give a brief overview of what we will see and what we will be able to prove by the end of the course.

\section{Plateau's Problem}

The primary goal and the motivating example of this course is the \textbf{Plateau's problem}, that is, the problem to find the $d$-dimensional surface $\Sigma$ of the minimal area with prescribed $(d-1)$-dimensional boundary $\Gamma$.

By the end, we will be able to prove that a solution indeed exists, but we will not find it explicitly since it is a $NP$ (hard) numerical problem.

As of now, the problem is not well defined. In fact, the notions of \textit{surface}, \textit{area}, and \textit{boundary} make sense in the smooth setting but, as the examples below show, we need to work in a less regular setting.

More precisely, requiring the surface to be smooth is not enough for modeling reasons (e.g., dip a wire frame into a soap solution, form a soap film, and look for the minimal surface whose boundary is the wire frame), and also for existence reasons.

\begin{example} Here we give a list of Plateau's problems with prescribed boundary conditions, and we write down the correct solutions, without proving anything. \mbox{}
\begin{enumerate}[label=\textbf{(\alph*)}]
\item Let us identify $\R^4 \cong \C \times \C$ and, if $d = 2$, let us consider the smooth boundary given by
\begin{equation*} \Gamma_1 := \left( S^1 \times \{0\} \right) \cup \left( \{0\} \times S^1 \right). \end{equation*}
Surprisingly, every minimizing sequence of smooth surfaces converges to a surface which is not smooth at all. Indeed, the solution of the problem is
\begin{equation*} \Sigma_1 := \left( D^2 \times \{0\} \right) \cup \left( \{0\} \times D^2 \right). \end{equation*}
The surface $\Sigma_1$ is clearly singular at the origin, but the singularity may be removed (by factorizing it into two nonsingular surfaces).
\item Let us identify $\R^4 \cong \C \times \C$ and, if $d = 2$, let us consider the smooth boundary given by
\begin{equation*} \Gamma_2 := \left\{ (z^2, \, z^3) \: : \: z \in S^1 \right\}. \end{equation*}
The solution to the Plateau's problem is
\begin{equation*} \Sigma_2 := \left\{ (z^2, \, z^3) \: : \: z \in D^2 \right\}, \end{equation*}
which is a non-smooth surface, whose singularity cannot be removed (since the polynomial $z_1^3 = z_2^2$ cannot be factorized).
\item Let us identify $\R^8 \cong \R^4 \times \R^4$ and, if $d = 7$, let us consider the smooth boundary given by
\begin{equation*} \Gamma_3 := S^3 \times S^3. \end{equation*}
The minimal surface of prescribed boundary $\Gamma_3$ is
\begin{equation*} \Sigma_3 := \left\{ (x_1, \, x_2) \in \R^4 \times \R^4 \: : \: |x_1| = |x_2| \leq 1 \right\}. \end{equation*}
\end{enumerate}
\end{example}

To conclude this introductive chapter, we give a brief overview of the main approaches (studied in this course) to the Plateau's problem, as $d$ ranges between $1$ and $\infty$.

\section{Geodesics problem ($d=1$)}

The geodesics problem (that is, find the shortest curve connecting two points) is, surprisingly, still an open in the non-Riemannian setting. However, in the Riemannian setting, the geodesics problem is completely solved.

Indeed, if we consider the curves parametrized by paths, the \textit{length} is a well-defined notion, and the associated functional is lower semi-continuous and coercive; hence the compactness is easy to prove.

There are many possible approaches to the geodesics problem, e.g., the Steiner approach and the set theoretical approach, which we describe briefly in the remainder of the section.

\paragraph{Steiner Problem.} It is also called networks approach, and it is used to prove the existence of the geodesics and find the explicit expression for it. The reader may consult \cite{steinerpro} for a detailed dissertation on the topic.

\paragraph{Set Theoretical Approach.} The main idea is to find a closed and connected set $\Sigma$ of minimum \textit{length}, containing a given finite set $\Gamma$. As we shall see later in the course, in this case the length is a well defined concept: the \textit{Hausdorff distance}.

In fact, if $X$ is a suitable space (metric, endowed with Hausdorff distance, etc...), then the class defined by
\begin{equation*}\mathcal{X} := \left\{ K \subseteq X \: : \: K \, \, \text{compact and connected} \right\} \end{equation*}
is compact and, by Gotab theorem\footnote{\cite{falconer} Let $\mathscr{C}$ be an infinite collection of non-empty compact sets all lying in a bounded portion $B$ of $\R^n$. Then there exists a sequence $\{E_j\}$ of distinct sets of $\mathfrak{C}$ convergent in the Hausdorff metric to a non-empty compact set $E$.}, $\mathcal{H}^1$ is lower semi-continuous on $X$. 

\section{"Surface" Problem ($d>1$)}

\paragraph{3.} The Plateau's problem is much harder when $d = 2$, but there are still many approaches possible some of which relying, in a certain sense, on the work already done in the geodesics case.

\paragraph{Set Theoretical Approach.} This approach is highly nontrivial. For example, one may ask what does it mean that a compact set $\Sigma$ spans a boundary $\Gamma$? Moreover, there is another problem one should deal with: the $2$-dimensional Hausdorff measure $\mathcal{H}^2$ is, generally, not lower semi-continuous. The reader may consult \cite{Reifenberg1960} for a complete treatise of the topic.

\begin{remark}Suppose that $d = 2$, $n = 3$ and that $\Sigma$ is a surface with boundary $\Gamma$. If $\gamma$ is another closed curve, linked to $\Gamma$ (by a nonzero linking number), then $\gamma \cap \Sigma \neq \emptyset$. \end{remark}

\paragraph{Parametric Approach.} This method is essentially due to Douglas \cite{douglas}. The main idea is the following: since a parametrization $\phi : D^2 \to \R^n$ defines surfaces in $\R^n$, the area functional is well-defined and given by the formula
\begin{equation*}A(\phi) := \int_{D^2} \left| \frac{\partial \, \phi}{\partial \, s_1} \wedge \frac{\partial \, \phi}{\partial \, s_2} \right| \, \mathrm{d}s_1 \, \mathrm{d}s_2. \end{equation*}
On the other hand, the existence through lower semi-continuity and the compactness are a delicate matter, since coercivity is not an easy property to obtain (the integrand is similar to a determinant).

There is a trick which is similar to the one we can use to find geodesics in the differential geometry setting. More precisely, we consider the functional
\begin{equation*}E(\phi) := \frac{1}{2} \, \int_{D^2} \left| \nabla \phi \right|^2 \, \mathrm{d}s_1 \, \mathrm{d}s_2. \end{equation*}
If we find a minimal point $\phi$ for $E$, then $\phi$ will be a \textbf{conformal parametrized} minimum for $A$. This trick, on the other hand, heavily depends on a nontrivial theorem: every such $\Sigma$ admits a conformal re-parametrization.

The lack of conformal parametrization, though, is what stop us from extending the same trick to dimension $d$ strictly bigger than $2$.

\paragraph{Higher Dimension.} If the codimension of $\Sigma$ is equal to $1$ (that is, $n = d+1$), then finite perimeter sets generalize the notion of open $(d+1)$-dimensional sets with smooth boundary in $\R^n$.

The class of finite perimeter sets has excellent compactness properties and a notion of area lower semi-continuous. 

This approach is called "weak" surfaces approach, and it is essentially due to Caccioppoli \cite{cacio} and De Giorgi \cite{degi}. A different approach, working for any $d$ and $n$, referred to as \textit{integral currents}, was introduced by Federer and Flaming in their joint paper \cite{fed}.
\chapter{Local Inversion Theorems} \thispagestyle{empty}

%In this chapter, we continue with our research toward the extension of differential calculus to the abstract framework of Banach spaces. Recall that, for a function
%\[
%F : \R^n \to \R^n,
%\]
%being continuously differentiable with total derivative invertible at a point $p$ (i.e., the Jacobian determinant of $F$ at $p$ is nonzero) is enough to infer that $F$ is {\em locally invertible}\index{locally invertible}. The first part of this chapter is devoted to proving the same statement replacing $\R^n$ with Banach spaces $\X$ and $\Y$. More precisely, we have:

%\begin{customthm}{A}
%Let $F \in C^1(\X, \, \Y)$ with $F^\prime(u^\ast) \in \mathrm{Inv}(\X, \, \Y)$. Then $F$ is locally invertible at $u^\ast$ with $C^1$ inverse. Namely, there are neighbourhoods $U$ of $u^\ast$ and $V$ of $F(u^\ast)$ such that \mbox{}
%\begin{enumerate}[label=\textbf{(\roman*)}, leftmargin=2.5\parindent]
%\item The restriction $F \, \big|_{U} : U \longrightarrow V$ is a homomorphism.
%\item The inverse $F^{-1}$ belongs to $C^1(V, \, \X)$ and for all $v \in V$ there results
%\[
%\dr F^{-1}(v) := (F^\prime(u))^{-1},
%\]
%where $u = F^{-1}(v)$.
%\item If $F$ belongs to $C^k(\X, \, \Y)$, $k > 1$, then $F^{-1} \in C^k(\X, \, \Y)$.
%\end{enumerate}
%\end{customthm}

%In the second half of the chapter, we generalise a well-known result in Euclidean calculus: the {\em implicit function theorem}. The following statement holds:

%\begin{customthm}{B}
%Let $F \in C^k(\Lambda \times U, \, \Y)$, $k \geq 1$. Suppose that
%\[
%\text{$F(\lambda^\ast, \, u^\ast) = 0$ and that $F_u(\lambda^\ast, \, u^\ast)$ is invertible.}
%\]
%Then there are neighbourhoods $\Theta$ of $\lambda^\ast$ and $U^\ast$ of $u^\ast$ and a map $g \in C^k(\Theta, \, \X)$ such that: \mbox{}
%\begin{enumerate}[label=\textbf{(\roman*)}, leftmargin=2.5\parindent]
%\item For all $\lambda \in \Theta$ there results $F(\lambda, \, g(\lambda)) = 0$.
%\item If $(\lambda, \, u) \in \Theta \times U^\ast$ is such that $F(\lambda, \, u)= 0$, then $u = g(\lambda)$.
%\item If $\lambda \in \Theta$ and $p = (\lambda, \, g(\lambda))$, then
%\[
%g^\prime(\lambda) = - [F_u(p)]^{-1} \circ F_\lambda(p).
%\]
%\end{enumerate}
%\end{customthm}

[{\color{red}Da scrivere}].

\section{Local inversion theorem}

Although we will consider continuous maps from a Banach space $\X$ to another Banach space $\Y$, with few changes we can adapt everything to the case in which $F$ is defined on $U \subset \X$.

\bd[Inverse] \index{inverse linear operator}
Let $A \in \cL(\X, \Y)$ be a continuous linear operator. We say that $A$ is {\em invertible} if there exists $B \in \cL(\Y,  \X)$ such that
\[
B \circ A = \mathrm{Id}_\X \quad \text{and} \quad A \circ B = \mathrm{Id}_\Y.
\]
The map $B$ is unique and we will denote it, from now on, by $A^{-1}$. The set of all invertible continuous linear maps is denoted by
\[
\inv(\X, \Y) := \left\{ A \in \cL(\X, \Y) \: : \: \text{$A$ is invertible} \right\}.
\]
\ed

\bthm[Closed Graph] \index{closed graph theorem}
A linear operator $T$ between two Banach spaces\footnote{It is enough to require $\X$ and $\Y$ to be Fréchet spaces!} is continuous if and only if its graph $\cG(T)$ is closed, where
\[
\cG(T) = \{(x, y) \: : \: x \in \X,  y= T(x) \}.
\]
\ethm

\bcor
If $A$ is injective and has {\em range}\index{operator!range} equal to $\Y$, then $A \in \inv(\X, \Y)$.
\ecor

\bpr
Let $\X$ and $\Y$ be Banach spaces. Then the following hold: \mbox{}
\begin{enumerate}[label=\textbf{(\roman*)}]
\item Let $A \in \inv(\X, \Y)$. Then all operators $T \in \cL(\X, \Y)$ satisfying
\begin{equation} \label{eq.2.1.1}
\| T - A \|_{\cL(\X, \Y)} < \frac{1}{\| A^{-1} \|_{\cL(\X, \Y)}}
\end{equation}
are invertible. In particular, the set $\inv(\X, \, \Y)$ is open.
\item The map $J : \inv(\X, \Y) \to \cL(\Y, \X)$, $A \mapsto A^{-1}$, is smooth.
\end{enumerate}
\end{proposition}

\brmk
The continuity of $J^\prime$ is easy to deduce. Indeed, we know that $J$ is differentiable and its differential is given by
\[
\mathrm{d}J(A)[B] = - A^{-1} \circ B \circ A^{-1},
\]
and the right-hand side is a composition of continuous maps.
\ermk

\bd[Homomorphism]\index{homomorphism}
Let $U$ and $V$ be open subsets of $\X$ and $\Y$. A continuous map $F : U \to V$ is a {\em homeomorphism} if there exists $G : V \to U$ such that
\[
G \circ F(u) = u \quad \text{and} \quad F \circ G(v) = v
\]
for all $u \in U$ and all $v \in V$.  We denote by $\homs(U, V)$ the set of all homeomorphisms between $U$ and $V$.
\ed

\bd \index{locally invertible}
A continuous map $F \in C(\X, \Y)$ is {\em locally invertible} at $u^\ast \in \X$ if there are neighbourhoods $U$ of $u^\ast$ and $V$ of $F(u^\ast)$ such that 
\[
F \, \big|_U \in \mathrm{Hom}(U, V).
\]
The map $G : V \to U$ is called {\em local inverse} of $F$ and it is usually denoted by $F^{-1}$.
\ed

\bpr
\begin{enumerate}[label=(\alph*)]
\item If $F_1 \in C(\X, \Y)$ is locally invertible at $u$ and $F_2 \in C(\Y, \mathfrak{Z})$ is locally invertible at $F_1(u)$, then $F_2 \circ F_1$ is locally invertible at $u$.
\item If $F$ is locally invertible at $u$, then it is locally invertible at any point in a small neighbourhood of $u$.
\end{enumerate}
\epr

\brmk
Suppose that $F$ is a locally invertible map at $u^\ast$ with inverse $G$. If $F$ is differentiable at $u^\ast$ and $G$ at $v^\ast := F(u^\ast)$, then
\[
F \circ G = \mathrm{Id}_\Y \quad \text{and} \quad G \circ F = \mathrm{Id}_\X \implies \left( \dr F(u^\ast) \right)^{-1} = \dr G(v^\ast).
\]
\ermk

\bthm[Local Inverse]\index{local inversion theorem} \label{thm.2.2.1}
Let $F \in C^1(\X, \Y)$. Suppose that $F^\prime(u^\ast) \in \inv(\X, \Y)$. Then $F$ is locally invertible at $u^\ast$ with a $C^1$ inverse. Namely, there are neighbourhoods $U$ of $u^\ast$ and $V$ of $F(u^\ast) =: v^\ast$ satisfying the following properties: \mbox{}
\begin{enumerate}[label=\textbf{(\roman*)}]
\item The restriction $F \, \big|_{U} : U \to V$ is a homeomorphism.
\item The inverse $F^{-1}$ belongs to $C^1(V, \X)$ and for all $v \in V$ there results
\[
\dr F^{-1}(v) := (F^\prime(u))^{-1},
\]
where $u = F^{-1}(v)$.
\item If, in addition, $F$ belongs to $C^k(\X, \Y)$, $k > 1$, then $F^{-1} \in C^k(\X, \Y)$.
\end{enumerate}
\ethm

\begin{proof}
We can always assume (using translations) that $u^\ast = F(u^\ast) = 0$. According to the transitivity property, we can ultimately discuss the local invertibility of the function
\[
A \circ F,
\]
where $A$ is a linear continuous invertible map. We can choose $A := [F^\prime(0)]^{-1}$ so that it will be enough to prove the theorem for functions of the form
\[
F = \mathrm{Id}_\X + \Psi,
\]
where $\Psi \in C^1(\X, \X)$ and $\Psi^\prime(0) = 0$.

\paragraph{Step 1.} Since $\Psi^\prime$ is continuous, we can choose $r > 0$ such that
\[
\| p \|_\X < r \implies \| \Psi^\prime(p) \|_\X < \frac{1}{2}.
\]
It follows from \eqref{eq.2.4} that
\[ \begin{aligned}
\|\Psi(p) - \Psi(q) \|_\X & \leq \sup \{ \| \Psi^\prime(w) \| \: : \: w \in [p, q] \} \|p-q\|_\X \leq
\\ & \leq \frac{1}{2} \|p-q\|_\X,
\end{aligned} \end{equation*}
which means that $\Psi$ is a contraction and $\| \Psi(p) \|_\X \leq \frac{1}{2} \|p\|_\X$ for all $p \in B_\X(0, r)$.

\paragraph{Step 2.} Fix $v \in \X$ and define the function
\[
\Phi_v(u) := v - \Psi(u).
\]
We can verify that $\Phi_v$ is a contraction of $B_\X(0, r)$ for any fixed $v \in B_\X(0, \frac{r}{2})$. Indeed, a simple computation shows that
\[
\|\Phi_v(u) \|_\X \leq \|v\|_\X + \|\Psi(u)\|_\X \leq r.
\]
Applying the fixed-point theorem we can find a unique $u \in B_\X(0,r)$ that satisfies the equation
\[
u = v - \Psi(u).
\]
We can easily define a local inverse
\[
F^{-1} : B_\X\left(0, \frac{r}{2} \right) \longrightarrow B_\X(0, r)
\]
by setting $F^{-1}(v) = u$. To prove that $F^{-1}$ is continuous, let $u = F^{-1}(v)$ and $w = F^{-1}(z)$, and notice that these are given by
\[
\begin{cases} u + \Psi(u) = v, \\[.6em] w + \Psi(w) = z. \end{cases}
\]
It follows that
\[
\| u - w \|_\X \leq \|v - z\|_\X + \frac{1}{2} \| u - w \|_\X \implies \| F^{-1}(v) - F^{-1}(z) \|_\X \leq 2 \|v - z\|_\X, 
\]
which means that $F^{-1}$ is Lipschitz-continuous. In particular, letting $V$ be the ball of radius $\frac{r}{2}$ and $U = B_\X(0, r) \cap F^{-1}(V)$, we infer
\[
F \, \big|_U \in \homs(U, V). 
\]

\paragraph{Step 3.} Since $u = F^{-1}(v)$, where $u + \Psi(u) = v$, we have the identity
\[
F^{-1}(v) = v - \Psi(F^{-1}(v)).
\]
But $\Psi(u) = o(\|u\|_\X)$ and $F^{-1}$ is Lipschitz-continuous, so it must be that
\[
\Psi(F^{-1}(v))= o(\|v\|_\X).
\]
This shows that $F^{-1}$ is differentiable at $v = 0$ and
\[
\mathrm{d}F^{-1}(0) = \mathrm{Id}_\X.
\]
The formula for the differential at any other point follows easily because we can always compose with a translation, which is a linear map.

\paragraph{Step 4.} The continuity (and thus $F^{-1} \in C^1(V,U)$) is immediate as it is a composition of continuous maps. If $F \in C^k(\X,\Y)$ we simply iterate the same argument to higher-order differentials.
\end{proof}

\brmk
The assumption $F \in C^1(\X, \Y)$ cannot be removed, but we can drop the injectivity if both $\X$ and $\Y$ are finite-dimensional spaces.
\ermk

\bex
Consider the nondecreasing function $\varphi: \R \to \R$ defined by
\[
\varphi(s) = \begin{cases} \frac{1}{n}, & s \in \left[ \frac{1}{n} - \frac{1}{4n^2}, \, \frac{1}{n} + \frac{1}{4n^2} \right],
\\[1em] s + \mathcal{O}(s^2) & \text{as $s \to 0$}. \end{cases}
\]
This is a differentiable function with derivative at zero equal to $1$, but it is not injective in any neighbourhood of the origin.
\eex

On the other hand, in the infinite-dimensional setting we can easily construct an example of $F \notin C^1(\X, \Y)$ for which the local surjectivity fails.

\bex
Let $\varphi$ be as above. Let $\X = \Y = C^0([-1, 1])$, and consider the map
\[
F : \X \ni u \longmapsto \varphi \circ u \in \Y.
\]
Let $v_n \in \Y$ be the sequence defined by
\[
v_n(t) := \frac{1}{n} + \frac{t}{n^2}.
\]
It is easy to verify that $\|v_n\|_\infty \to 0$ and $v_n \notin F(\X)$. Indeed, if we could find a sequence $u_n \in \X$ such that $F(u_n) = v_n$, then one would find
\[
\varphi(u_n(t)) = \frac{1}{n} + \frac{t}{n^2}.
\]
However, it is easy to notice that
\[ \begin{cases}
\varphi(u_n(t)) > \frac{1}{n} & \text{if $t > 0$},
\\[.6em] \varphi(u_n(t)) < \frac{1}{n} & \text{if $t < 0$},
\end{cases} \]
and, using the monotonicity of $\varphi$, this would imply that
\[ 
u_n(t) \geq \frac{1}{n} + \frac{1}{4n^2}
\]
for $t > 0$, and
\[
u_n(t) \leq \frac{1}{n} - \frac{1}{4n^2}
\]
for $t < 0$. But $v_n$ is the image of a function $u_n$ that is not continuous $t = 0$ and this gives the sought contradiction.
\eex

\brmk
Notice that the $F$ in the previous example is differentiable at $u = 0$ with $F^\prime(0) = \mathrm{Id}_\X$, but it is not of class $C^1(\X, \Y)$.
\ermk

\subsection{Application to perturbed ODEs}

We will now show how the local invertibility theorem can be applied to obtain a solution of certain ODEs/PDEs with small norm.

\bex
Let $g \in C^1(\R\times\R,\R)$ and $h \in C^0(\R,\R)$. We are interested in $T$-periodic solution of the following ODE:
\[
\ddot{x}(t) + g(x, \, \dot{x}) = \epsilon h(t),
\]
where $\epsilon$ is a parameter which controls the contribution of the perturbation $h$. To apply the local invertibility theorem we consider the following Banach spaces
\[ \begin{aligned}
& \X := \left\{ x \in C^2(\R, \R) \: : \: \text{$x(t + T) = x(t)$ for all $t \in \R$}\right\},
\\ & \Y := \left\{ h \in C^0(\R, \R) \: : \: \text{$h(t + T) = h(t)$ for all $t \in \R$}\right\},
\end{aligned}\]
and the map $F : \X \to \Y$ defined by setting
\[
F(x(t)) := \ddot{x}(t) + g(x(t), \dot{x}(t)).
\]
We assume that $g(0, 0) = 0$ in such a way that for $\epsilon = 0$ the ODE admits the function identically zero as a solution. To apply \hyperref[thm.2.2.1]{Theorem \ref{thm.2.2.1}} with $u^\ast = 0$ we first notice that
\[
\dr F(x(t))[y(t)] =  \ddot{y}(t) + \left[g_{\dot{x}}(x(t), \dot{x}(t)) \dot{y}(t) + g_x(x(t), \dot{x}(t)) y(t) \right],
\]
which at $x(t) \equiv 0$ is equal to
\[
\dr F(0)[y(t)] = \ddot{y}(t) + \left[g_{\dot{x}}(0,0) \dot{y}(t)+g_x(0, 0) y(t) \right].
\]
By {\em Fredholm theory}, the differential $\dr F(0)$ is invertible if and only if the equation
\[
\ddot{y}(t) + g_{\dot{x}}(0,0) \dot{y}(t) + g_x(0, 0) y(t) = 0
\]
has $y(t) \equiv 0$ as the unique solution. In this case, we can find $\epsilon^\ast > 0$ and $\delta > 0$ such that for all $\epsilon < \epsilon^\ast$ the initial ODE has a unique solution $x(t)$ satisfying $\|x\|_\infty<\delta$.
\eex

\bex
Let $\Om \subset \R^n$ be a bounded odd set with smooth boundary and consider the associated boundary-value problem
\[ \begin{cases}
\Delta u - \lambda u + u^3 = h(x) & \text{if $x \in \Om$},
\\ u(x) = 0 & \text{if $x \in \partial \Om$}.
\end{cases} \]
We want to investigate the existence of a $C^{2,\alpha}$-solution under the assumption that $h$ is H\"{o}lder-continuous. Therefore, we consider the Banach spaces
\[
\X := \left\{ u \in C^{2, \alpha}(\bar{\Om}) \: : \: u \, \big|_{\partial \Omega} \equiv 0 \right\}, \quad \Y := C^{0, \alpha}(\bar{\Om}),
\]
and the map $F : \X \to \Y$ given by
\[
F(u) = \Delta u - \lambda u + u^3.
\]
A simple computation shows that the differential is
\[
\dr F(u)[v] = \Delta v - \lambda v + 3 u^2 v,
\]
which evaluated at the origin gives
\[
\dr F(0)[v] = \Delta v - \lambda v.
\]
If $\lambda \neq - \lambda_k(\Om)$ for all $k \in \N$, where $\{\lambda_k(\Om)\}_{k \in \N}$ are the eigenvalues of the Laplacian operator $-\Delta$ on $\Om$, then
\[
\dr F(0) \in \inv(\X,\Y).
\]
Consequently, we can apply the local invertibility theorem and conclude that for any $h \in \Y$ with $\|h\|_\Y < \delta$, there exists a unique solution $u \in \X$ with $\|u\|_\X < C(\delta)$.
\eex

\brmk
It can be proved that any solution $u$ of
\[ \begin{cases}
\Delta u - \lambda u = h(x) & \text{if $x \in \Om$},
\\ u(x) = 0 & \text{if $x \in \partial \Om$},
\end{cases} \]
where $h \in C^{0, \alpha}(\Om)$, satisfies an estimate of the type
\[
\|u\|_{2, \alpha} \leq C(n, \Om) ( \|h\|_{0, \alpha} + \|u\|_\infty).
\]
Removing the second term in the right-hand side is, morally, what happens when we apply the local invertibility theorem in the previous example.
\ermk

\bex
Let $\Om \subset \R^2$ be a smooth bounded connected set. Let $\gamma$ be a smooth function defined on $\partial \Om$ and taking values in $\R$. A smooth solution of
\[ \begin{cases}
\M(u) := (1+u_y^2) u_{xx} + (1 + u_x^2) u_{yy} - 2 u_x u_y u_{xy} = 0
\\ u \, \big|_{\partial\Om} \equiv \gamma
\end{cases}\]
is called minimal surface with boundary $\gamma$. To apply the local invertibility theorem we consider the Banach spaces
\[
\X := C^{2, \alpha}(\bar{\Om}), \quad \Y := C(\bar{\Om}) \times C^{2, \alpha}(\partial \Om),
\]
and the map $F:\X \to \Y$ defined by setting
\[
F(u) := \left(\M(u), u \, \big|_{\partial \Om}\right).
\]
It is easy to see that $F$ is $C^1$ with differential equal to
\[
\dr F(u)[v] = A v_{xx} + B v_{yy} - u(\dots),
\]
where $A = (1+u_y^2)$ and $B=(1+u_x^2)$. It turns out that
\[
\dr F(0)[v] = \left(\Delta v,  v \, \big|_{\partial \Om}\right),
\]
and by elliptic regularity theory the Dirichlet problem
\[ \begin{cases}
\Delta v = h(x) & \text{if $x \in \Om$}
\\ v(x) = \varphi(x) & \text{if $x \in \partial \Om$}
\end{cases} \]
admits a unique solution $v$, depending continuously on the initial data, provided that $(h, \varphi) \in \Y$. Now a simple application of \hyperref[thm.2.2.1]{Theorem \ref{thm.2.2.1}} shows that there are neighbourhoods $U$ and $V$ in $\X$ and $C^{2,\alpha}(\partial \Om)$ respectively such that
\[
\gamma \in V \implies \text{the system has a unique solution $u_\gamma \in U$,}
\]
and the correspondence $\gamma \mapsto u_\gamma$ is $C^1$.
\eex

\section{The implicit function theorem}

We will now show that we can extend the ``range of applicability'' of the local inversion theorem by adding an extra parameter. Namely, consider a map
\[
F : \Lambda \times U \longrightarrow \Y,
\]
where $U \subset \X$ and $\Lambda$, the set of parameters, is a subset of a Banach space $\mathfrak{T}$.

\bl \index{local inversion lemma}
Let $(\lambda^\ast, u^\ast) \in \Lambda \times U$. Suppose that the following properties hold: \mbox{}
\begin{enumerate}[label=(\roman*)]
\item The function $F$ and its partial derivative $F_u$ are continuous in $\Lambda \times U$.
\item The linear operator $F_u(\lambda^\ast, u^\ast)$ is invertible.
\end{enumerate}
Then the map $\Psi : \Lambda \times U \to \mathfrak{T} \times \Y$ given by
\begin{equation} \label{eq.13.1}
\Psi(\lambda, u) := \left(\lambda, F(\lambda, u)\right)
\end{equation}
is locally invertible at $(\lambda^\ast, u^\ast)$ and its inverse $\Phi$ is continuous.
\el

\brmk
The local inverse $\Phi$ belongs to $C^1$ if $F \in C^1$.
\ermk

\begin{proof}
Assume that $F \in C^1(\Lambda \times U, \Y)$ and introduce the symbols
\[
A:= F_\lambda(\lambda^\ast, u^\ast) \quad \text{and} \quad B:= F_u(\lambda^\ast, u^\ast).
\]
The map $\Psi$ belongs to $C^1$ (check!) and its differential is
\[
\dr\Psi(\lambda^\ast, u^\ast)[\xi, v] = (\xi, A\xi + B v).
\]
It is easy to verify that
\[
\dr \Psi(\lambda^\ast, u^\ast)[\xi, v] = (\eta, v) 
\]
yields $\eta = \xi$ and, since $B$ is invertible by assumption, there is a unique $v$ given by
\[
v = B^{-1}(v - A \eta).
\]
It follows that $\dr \Psi(\lambda^\ast, u^\ast)$ is invertible so the conclusion follows from a straightforward application of \hyperref[thm.2.2.1]{Theorem \ref{thm.2.2.1}} (which also gives $\Phi \in C^1)$.
\end{proof}

\brmk
The function $\Psi$ has a local inverse $\Phi$ defined in a neighbourhood $\Theta \times V$ of $(\lambda^\ast, F(\lambda^\ast, u^\ast))$ of the form
\begin{equation} \label{eq.13.2}
\Phi(\lambda, v) = (\lambda, \varphi(\lambda, v)).
\end{equation}
The function $\varphi : \Theta \times V \to \X$ is uniquely determined by
\begin{equation} \label{eq.13.3}
F(\lambda, \varphi(\lambda, v)) = v \quad \text{for all $\lambda \in \Theta$},
\end{equation}
and hence $\varphi \in C^1$ has partial derivatives given by
\[ \begin{aligned}
& F_\lambda + F_u \circ \varphi_\lambda = 0,
\\ & F_u \circ \varphi_v = \mathrm{Id},
\end{aligned} \quad \implies \quad
\begin{aligned} &\varphi_\lambda = - [F_u]^{-1} \circ F_\lambda,
\\ &\varphi_v = [F_u]^{-1}.
\end{aligned}\]
\ermk

\brmk
The existence of a local inverse $\Phi$ of $\Psi$ can be obtained in a more general setting, requiring e.g. $\mathfrak{T}$ topological space.
\ermk

\bthm[Implicit Function]\index{implicit function theorem} \label{thm.2.3.2}
Let $F \in C^k(\Lambda \times U, \Y)$, $k \geq 1$. Suppose that
\[
F(\lambda^\ast, u^\ast) = 0
\]
and $F_u(\lambda^\ast, u^\ast)$ invertible. Then there is a neighbourhood $\Theta \times U^\ast$ of $(\lambda^\ast,u^\ast)$ and $g \in C^k(\Theta, \X)$ satisfying the following properties: \mbox{}
\begin{enumerate}[label=(\roman*)]
\item For all $\lambda \in \Theta$ there results $F(\lambda, g(\lambda)) = 0$.
\item If $(\lambda, u) \in \Theta \times U^\ast$ is such that $F(\lambda, u)= 0$, then $u = g(\lambda)$.
\item If $\lambda \in \Theta$ and $p = (\lambda, g(\lambda))$, then
\[
g^\prime(\lambda) = - [F_u(p)]^{-1} \circ F_\lambda(p).
\]
\end{enumerate}
\ethm

\begin{proof}
Let $\Psi$ be the function defined by \eqref{eq.13.1}. Then $\Psi$ is locally invertible at $(\lambda^\ast, u^\ast)$ and it satisfies
\[
\Psi(\lambda^\ast, u^\ast) = (\lambda^\ast, F(\lambda^\ast, u^\ast)) = (\lambda^\ast, 0).
\]
The local inverse $\Phi$ satisfies \eqref{eq.13.2} and it is rather easy to verify that $\varphi$ is of class $C^k$, provided that $F$ is $C^k$. Now set
\[
g(\lambda) := \varphi(\lambda, 0)
\]
and use \eqref{eq.13.3} to infer that
\[
F(\lambda, g(\lambda)) = F(\lambda, \varphi(\lambda, 0)) = 0 \quad \text{for all $\lambda \in \Theta$}.
\]
This concludes the proof of ({\romannumeral 1}). The assertion ({\romannumeral 2}) follows from the fact that $\Phi$ is one-to-one and ({\romannumeral 3}) has been proved already in the previous remark.
\end{proof}

\subsection{Application to perturbed differential systems}

Let $f \in C^1(\R \times \R \times \R^n, \R^n)$ be a periodic function. Namely, there exists a positive time $T$ such that the following holds:
\[
f(\epsilon, t + T, x) = f(\epsilon, t, x).
\]
Our goal is to investigate periodic solutions of the $\epsilon$-perturbed differential system
\begin{equation} \label{eq.a.1} \tag{$P_\epsilon$}
\dot{x}(t) = f(\epsilon, t, x(t)).
\end{equation}
We shall always assume that for $\epsilon = 0$ there exists a $T$-periodic solution, which will be denoted by $y(t)$. Consider the Cauchy problem
\[ \begin{cases}
\dot{\alpha}(t) = f(\epsilon, t, \alpha(t)),
\\ \alpha(0) = \xi.
\end{cases}\]
We assumed $f$ to be differentiable so by Cauchy-Lipschitz we know we can always find a unique solution $\alpha$ which is defined in a small neighbourhood of the initial value, that is,
\[
|\xi - \xi^\ast| < \delta \quad \text{where $\xi^\ast = y(0)$}.
\]
Moreover, we know that
\[
A(\epsilon, t, \xi) := \partial_\xi \alpha
\]
is the $n\times n$ matrix solving the Cauchy problem
\[ \begin{cases}
\dot{A} = f_x(\epsilon, t, \alpha)A,
\\ A(\epsilon, 0, \xi) = \mathrm{Id}_{\R^n}.
\end{cases} \]
In what follows, we shall always denote by $A_0(t)$ the matrix $A(0, t, y(0))$.

\bthm
Under these assumptions, if $\lambda = 1$ is not in the spectrum of $A_0(t)$, then there are $\delta > 0$ and $\xi \in C^1( (-\delta, \, \delta))$, $\xi(0) = \xi^\ast$ such that
\[
|\epsilon| < \delta \implies \text{there exists a unique $T$-periodic solution of \eqref{eq.a.1}}.
\]
\ethm

\begin{proof}
The Cauchy problem \eqref{eq.a.1} has a $T$-periodic solution if and only if there exists $\xi \in \R^n$ such that $
\alpha(\epsilon, \, T, \, \xi) = \xi$. Thus, introducing the map $F:\R \times \R^n \to \R^n$ defined by
\[
F(\epsilon, \xi) := \alpha(\epsilon, T, \xi) - \xi,
\]
we are interested in solving the equation $F(\epsilon,\xi) = 0$. The function $F$ is $C^1$ and, since $\alpha(0, t, \xi^\ast) = y(t)$ and $y$ is $T$-periodic, it turns out that
\[
F(0, \xi^\ast) = \alpha(0, T, \xi^\ast) - \xi^\ast = y(T) - \xi^\ast = 0.
\]
We conclude applying \hyperref[thm.2.3.2]{Theorem \ref{thm.2.3.2}} since
\[
F_\xi(0, \xi^\ast) = \alpha_\xi(0, T, \xi^\ast) - \mathrm{Id} =A_0(t) - \mathrm{Id},
\]
and the right-hand side is invertible because $1$ is not in the spectrum of $A_0(t)$ by assumption.
\end{proof}

The autonomous case - $f$ does not depend on $t$ directly - is more delicate and requires additional efforts. Consider the system
\begin{equation} \label{eq.b.1}
\dot{x}(t) = f(\epsilon, x(t)),
\end{equation}
and notice that the period of a solution of \eqref{eq.b.1} is, a priori, unknown. Let $f(x) := f(0, x)$ and assume that $f \in C^1(\R^n, \R^n)$ satisfies the following property:
\[
\epsilon = 0 \implies \text{\eqref{eq.b.1} has a nonconstant $T$-periodic solution $y = y(t)$.}
\]
Without loss of generality we can assume $y(0) = 0$.

\brmk
It is important to notice that the previous theorem does not apply here because $1$ always belongs to the spectrum of $A_0(T)$. Indeed, $A_0$ satisfies
\[ \begin{cases}
\dot{A_0} = f^\prime(y(t))A_0,
\\ A_0(0) = \mathrm{Id}_{\R^n}.
\end{cases} \]
To see this, we differentiate the relation $y' = f(y)$ and find that
\[
y''(t) = f'(y(t)) y'(t),
\]
which can be rewritten via $v := y'$ as
\[
v' = f'(y) v.
\]
Let $v^\ast = v(0)$ and $w(t) = A_0(t) v^\ast$. It follows that
\[ \begin{cases}
w' = \dot{A_0}v^\ast = f'(y(t))A_0(t) v^\ast = f'(y) w,
\\ w(0) = v^\ast.
\end{cases}\]
By the uniqueness of the Cauchy problem, the only possibility is that $v \equiv w$. In particular, there results $w(T) = w(0)$ and consequently
\[
A_0(T) v^\ast = w(0) = v^\ast \implies 1 \in \sigma(A_0(T)). 
\]
\ermk

\bthm
Under these assumptions, if $\lambda = 1$ is a simple eigenvalue for $A_0(T)$, then there are continuous maps $h=h(\epsilon)$ and $\tau = \tau(\epsilon)$ such that
\[
h(0) = y(0), \quad \tau(0) = T,
\]
and \eqref{eq.b.1} has a $\tau(\epsilon)$-periodic solution $y_\epsilon$ satisfying $y_\epsilon(0) = h(\epsilon)$. \ethm
 \chapter{Global Inversion Theorems} \thispagestyle{empty}

[{\color{red}Da scrivere...}]

\section{The global inversion theorem}

The goal of this section is to investigate minimal conditions under which a map $F$ between metric spaces, $M$ and $N$, is a global homeomorphism.

\bd[Proper] \index{proper map}
A continuous map $F : M \to N$ between metric spaces is {\em proper} if the preimage
\[
F^{-1}(K) = \left\{u \in M \: : \: F(u) \in K \right\}
\]
is compact in $M$ whenever $K$ is compact in $N$.
\ed

\paragraph{N.B.} From now on, when we say that $F : M \to N$ is proper, we will also assume that $F$ is continuous with respect to the metric topologies $(M, \, d_M)$ and $(N, \, d_N)$.

\bl
Let $F : X \to Y$ be a proper map between topological spaces and suppose that $Y$ is locally compact and Hausdorff. Then $F$ is a closed map.
\el

\begin{proof}
Let $C$ be a closed subset of $X$. Take any $y \in Y \setminus F(C)$ and any neighbourhood $V \ni y$ with compact closure. It follows that
\[
\text{$F$ proper $\implies F^{-1}(\bar{V})$ compact in X}.
\]
Let $E := C \cap F^{-1}(\bar{V})$. Then $E$ is compact and, by continuity, so is $F(E)$. Since $Y$ is Hausdorff, $F(E)$ is also closed. Now consider
\[
U := V \setminus F(E).
\]
It is easy to verify that $U$ is an open neighbourhood of $y$ which is disjoint from $F(C)$, and this proves that $F(C)$ is closed as its complement is open.
\end{proof}

\bthm \index{cardinality map}
Let $F : M \to N$ be a proper locally invertible map. Then
\[
N \ni v \longmapsto [v] := \can F^{-1}(\{v\})
\]
is finite and locally constant.
\ethm

\begin{proof}
Using the fact that $F$ is proper and locally invertible we easily conclude that
\[
\text{$F^{-1}(\{v\}) \subset M$ is discrete and compact}, 
\]
which is possible if and only if it is finite. To show that the map is locally constant, fix $v \in N$ and let
\[
\{u_1, \dots, u_n\} = F^{-1}(v).
\]
By the local invertibility theorem we can find open neighbourhoods $U_i \ni u_i$ in $M$ and $V$ neighbourhood of $v$ in $N$ such that
\[
F \in \homs(U_i, V) \quad \text{for all $i = 1,  \dots, n$}.
\]
It follows that
\[
[w] \geq n \quad \text{for all $w \in V$}.
\]
We now claim that there exists an open neighbourhood $W \subset V$ of $v$ such that $[w]$ is identically equal to $n$ at all $w \in W$. If such $W$ does not exist, then we can find
\[
\{v_j\}_{j \in \N} \subset N \quad \text{and} \quad v_j \xrightarrow{j \to + \infty} v
\]
and a corresponding sequence of points $p_j \in M$ such that
\[
p_j \notin \bigcup_{i = 1}^n U_i \quad \text{and} \quad F(p_j) = v_j.
\]
Since $F$ is proper, we can find a subsequence $j_k$ such that $p_{j_k}$ converges to some $p$ that does not belong to $\bigcup_{i = 1}^n U_i$. However, by continuity of $F$ we have
\[
F(p_{j_k}) \xrightarrow{k \to + \infty} F(p) = v,
\]
which is the sought contradiction.
\end{proof}

\begin{figure}[h!]
\centering
\includegraphics[width = 14cm, height = 6cm]{images/mvar1.pdf}
\caption{Counterexamples to $[v]$ finite and locally constant.}
\label{fig:c1}
\end{figure}

\bcor
Let $F : M \to N$ be a proper locally invertible map. If $N$ is connected, then $[v]$ is globally constant.
\ecor

\bd[Singular] \index{singular point} \index{regular point}
A point $u \in M$ is said to be {\em singular} for $F$ if $F$ is not locally invertible at $u$ and {\em regular} if it is not singular.
\ed

Denote by $\Sigma$ the set of all singular points in $M$ and $\Sigma_0$ the preimage $F^{-1}(F(\Sigma))$. We want to work with regular points only, so we define $M_0 := M \setminus \Sigma_0$ and $N_0 := N \setminus F(\Sigma)$.

\brmk
The set $\Sigma$ is closed, so both $M_0$ and $N_0$ are open in $M$ and $N$ respectively.
\ermk

An obvious consequence of the definitions of singular points and $(M_0, \, N_0)$ is the following theorem, which asserts that $[v]$ is constant on connected components of $N_0$.

\bthm
Let $F : M \to N$ be a proper map. Then $[v]$ is constant on every connected component of $N_0$.
\ethm

We are now ready to state the main result of this section. The proof is simple but it requires some preliminary work to introduce a technical topological tool.

\bthm\label{git}
Let $F : M \to N$ be a proper map. Suppose that $N_0$ is simply connected and $M_0$ is arc-wise connected. Then $F$ is a global homeomorphism between $M_0$ and $N_0$.
\ethm

\bcor
Let $F : M \to N$ be a proper locally invertible map. Suppose that $N$ is simply connected and $M_0$ is arc-wise connected. Then $F \in \homs(M, N)$.
\ecor

The first step is to introduce and prove both existence and uniqueness of "paths that invert $F$ along another path".

\bd\index{inverting path}
Let $M$, $N$ be as above and let $\sigma : [a, \, b] \to N$ be a continuous path. We say that a path $\theta : [a, \, b] \to M$ \textit{inverts $F$ along $\sigma$} if the following diagram commutes:
\[ \begin{tikzcd}
  M \arrow[rr, "F"] &
    & N \\
  & {[a, \, b]} \arrow[ru, "\sigma"]\arrow[lu, "\theta"]
& \end{tikzcd}\]
\ed

\brmk\label{rmk.2.1.wde.}
Let $u \in M$ and $v \in N$ be such that $F(u) = v$ and $F \, \big|_{U} \in \homs(U, V)$, where $U$ and $V$ are respectively neighbourhoods of $u$ and $v$. Given a path
\[
\sigma : [a, b] \to N, \quad \sigma(a) = v \quad \text{and} \quad \sigma\left([a, b]\right) \subset V,
\]
it is easy to see that the equation $F(\theta(t)) = \sigma(t)$ defines the {\bf unique} path $\theta$ that inverts $F$ along $\sigma$ satisfying the initial condition $\theta(a) = u$.
\ermk

\brmk\label{gluing}
Let $\sigma : [a, \, b] \to N$ be a continuous path and suppose that there exists $c \in (a, b)$ such that $\theta_1$ inverts F along $\sigma \, \big|_{[a, c]}$ and $\theta_2$ along $\sigma \, \big|_{[c, b]}$ with $\theta_1(c) = \theta_2(c)$. Then
\[
\theta(t) := \begin{cases} \theta_1(t) & \text{if $t \in [a, c)$},
\\ \theta_2(t) & \text{if $t \in [c, b]$},\end{cases}
\]
is a well-defined (continuous) path that inverts $F$ along $\sigma$.
\ermk

\bl \label{lemmahomotopiesoneidim}
Let $u^\ast \in M_0$ and $v^\ast = F(u^\ast) \in N_0$. Then for any given path $\sigma : [0, 1] \to N$ with $\sigma(0) = v^\ast$ there exists a unique
\[
\theta : [0, 1] \to M_0
\]
that inverts $F$ along $\sigma$ satisfying the initial condition $\theta(0) = u^\ast$.
\el

\begin{proof} We first prove uniqueness, which is relatively easy, and then we exploit it to obtain the existence.

\paragraph{Uniqueness.} We argue by contradiction. Let $\theta_1$ and $\theta_2$ be two such paths and let
\[
\xi := \sup\left\{ s \in [0, 1] \: : \: \theta_1 \, \big|_{[0, s]} \equiv \theta_2 \, \big|_{[0, s]} \right\}.
\]
According to \hyperref[rmk.2.1.wde.]{Remark \ref{rmk.2.1.wde.}}, $\xi$ is well-defined and, since $u^\ast \in M_0$, it is also strictly bigger than zero. Moreover, by continuity one has that
\[
\theta_1(\xi) = \theta_2(\xi)
\]
so it is enough to prove that $\xi = 1$. Suppose that $\xi < 1$ and set
\[
u = \theta_1(\xi) = \theta_2(\xi) \quad \text{and} \quad v = F(u).
\]
Since $F$ is locally invertible in $M_0$, we can find neighbourhoods $U \ni u$ and $V \ni v$ such that $F \, \big|_{U} \in \homs(U, V)$. Now both paths are continuous so
\[
\theta_1\left( [\xi, \xi + \alpha] \right) \subset U \quad \text{and} \quad  \theta_2\left( [\xi, \xi + \alpha] \right) \subset U
\]
for some $\alpha > 0$ small enough. Consequently, we have
\[
\theta_1 \, \big|_{[0, \xi + \alpha]} \equiv \theta_2 \, \big|_{[0, \xi + \alpha]},
\]
and this is a contradiction with the definition of $\xi$ as the supremum.

\paragraph{Existence.} Let $\Xi$ be the set of all $s \in [0, 1]$ such that $F$ is invertible along $\sigma \, \big|_{[0, s]}$ with inverse given by
\[
\text{$\theta_s : [0, s] \longrightarrow M_0$ such that $\theta_s(0) = u^\ast$, $F(u^\ast) = \sigma(0)$}.
\]
It is enough to show that $\Xi$ is both closed and open in $[0, \, 1]$ since it is nonempty. \mbox{}
\begin{enumerate}[label=\textbf{(\alph*)}]
\item Let $\xi := \sup \Xi$. As before $\xi > 0$ and, by uniqueness, the resulting paths $\theta_s$ must coincide in the intersections of the intervals of definition. Therefore
\[
\theta(s) := \theta_s(s) \quad \text{for all $s \in [0, \, \xi)$}
\]
is a well-defined function. Let $s_n \nearrow \xi$ be a sequence such that $\sigma(s_n) \to v$. Since $\theta(s_n) = F^{-1}(\sigma(s_n))$ and $F$ is proper, we find that we have
\[
\theta(s_n) \to u \quad \text{and} \quad F(u) = v.
\]
Now let $U \ni u$ and $V \ni V$ be neighbourhoods such that $F \, \big|_U\in \homs(U, V)$. If $m \in \N$ is chosen in such a way that
\[
\theta(s_m) \in U  \quad \text{and} \quad \sigma\left([s_m, \xi] \right) \subset V,
\]
then $F$ can be inverted along $\sigma$ restricted to $[s_m, \xi]$ by a path $\theta_1$ which coincides with $\theta$ evaluated at $s_m$. Finally, the trick illustrated in \hyperref[gluing]{Remark \ref{gluing}} allows us to conclude that $\Xi$ is closed.
\item The idea is more or less the same. The reader might try to fill in the missing details as an exercise.
\end{enumerate}
\end{proof}

We can prove the same replacing paths with $2$-paths, namely continuous functions defined on $Q := [a, \, b]^2$ and taking values in $M$ or $N$.

\bd \index{inverting $2$-path}
Let $M$ and $N$ be as above and let $\sigma : Q \to N$ be a $2$-path. We say that the $2$-path $\theta : Q \to M$ \textit{inverts $F$ along $\sigma$} if the following diagram commutes:
\begin{equation*}\begin{tikzcd}
  M \arrow[rr, "F"] &
    & N \\
  & Q \arrow[ru, "\sigma"]\arrow[lu, "\theta"]
& \end{tikzcd} \end{equation*}
\ed

\bl \label{lemmadiomotopia}
Let $u^\ast \in M_0$ and $v^\ast = F(u^\ast) \in N_0$. Then given any $2$-path $\sigma : Q \to N$ such that $\sigma(0, 0) = v^\ast$, there exists a unique $2$-path
\[
\theta : Q \to M_0
\]
that inverts $F$ along $\sigma$ satisfying the initial condition $\theta(0, 0) = u^\ast$.
\el

\begin{proof} We divide into two steps as before, starting from uniqueness (where we exploit the result for paths!) which is once again needed to prove existence.

\paragraph{Uniqueness.} Let $\theta_1$ and $\theta_2$ be two such $2$-paths and let $(s, t) \in Q$. Define $\phi_1, \, \phi_2 : [0, 1] \to M_0$ and $\psi : [0, 1] \to N_0$ as follows:
\[ \begin{aligned}
& \phi_1(\lambda) = \theta_1(\lambda s, \lambda t),
\\ & \phi_2(\lambda) = \theta_2(\lambda s, \lambda t),
\\ & \psi(\lambda) = \sigma(\lambda s, \lambda t).
\end{aligned}\]
Then $\phi_1$ and $\phi_2$ are paths that invert $F$ along $\psi$ so they must coincide. Letting $\lambda = 1$ gives
\[
\theta_1(s, t) = \theta_2(s, t),
\]
which is enough to conclude as $(s, t) \in Q$ was chosen arbitrary.

\paragraph{Existence.} Consider the rectangle
\[
R_s = [0, s] \times [0, 1] \subset Q,
\]
and let $\Xi$ be the set of all $s \in (0, 1]$ for which there exists $\theta_s:R_s \to M_0$ that inverts $F$ along the restriction $\sigma \, \big|_{R_s}$ and $\theta_s(0, 0) = u^\ast$. Since $F$ is invertible along
\[
t \longmapsto \sigma(0,t)
\]
by \hyperref[lemmahomotopiesoneidim]{Lemma \ref{lemmahomotopiesoneidim}}, we have $0 \in \Xi$. Let $\xi := \sup \Xi$. As before $\xi > 0$ and, by uniqueness, the resulting $2$-paths $\theta_s$ must coincide in the intersections of the intervals so
\[
\theta(z, t) := \theta_s(z, t) \quad \text{for all $(z, t) \in R_s$}
\]
is a well-defined function. Fix $t \in [0, 1]$. Since $F$ is invertible along the path $s \mapsto \sigma(s, t)$ with inverse $s \mapsto \phi(s)$ satisfying the initial condition $\phi(0) = \theta(0, t)$, by uniqueness we have
\[
\phi(z) = \theta(z, t) \quad \text{for all $0 \leq z < \xi$}.
\]
If we set $\phi(\xi) =: u$ and $\sigma(\xi, t) =: v$, then we can find neighbourhoods $U \ni u$ and $V \ni v$ such that $F \, \big|_U \in \homs(U, V)$. Then we can find a rectangle $R^\prime$ centered at $(\xi, \, t)$ and
\[
\theta^\prime : R^\prime \cap Q \longrightarrow M_0
\]
such that $\theta^\prime$ inverts $F$ along $\sigma \, \big|_{R^\prime \cap Q}$ with $\theta^\prime(\xi, t) = u$. Since $\theta$ and $\theta^\prime$ coincide in $(0, \, \xi)$, we can extend $\theta$ to $R^\prime \cap Q$ and consequently (by continuity) to $R_\xi$ in such a way that
\[
F \circ \theta = \sigma
\]
holds at all points of $R_\xi$. It is now easy to check that $\xi = 1$ because we can always cover the segment $\{(\xi, t) \: : \: t \in [0, 1]\}$ with a family of rectangles $R^\prime$ and extend $\theta$ to $R_{\xi + \alpha}$ for some positive $\alpha$ which is absurd.
\end{proof}

\begin{proof}[Proof of \autoref{git}]
The map $v \mapsto [v]$ is constant on $N_0$ and by local invertibility $F$ is surjective ($[v]\geq1$) so we only need to show that 
\[ 
[v] = 1 \quad \text{at all $v \in N_0$.}
\]
Suppose that there are $u_0 \neq u_1 \in M_0$ such that $F(u_0) = F(u_1) = v$. Since $M_0$ is arcwise connected, we can always find a continuous path $\theta$ such that
\[
\theta(0) = u_0 \quad \text{and} \quad \theta(1) = u_1. 
\]
The corresponding path $\sigma := F \circ \theta$ is closed in the simply connected space $N_0$, and therefore homotopic to a constant one. Let $h \in C^0(Q, N_0)$ be the homotopy and assume that
\[
h(0, t) = h(1, t) = v \quad \text{for all $t \in [0, 1]$}.
\]
There exists a unique $2$-path $\Theta \in C(Q, \, M_0)$ that inverts $F$ along $h$ (see \hyperref[lemmadiomotopia]{Lemma \ref{lemmadiomotopia}}), which means that for all $(s,t) \in Q$ we have
\[
F(\Theta(s,  t)) = h(s, t).
\]
If we evaluate this at $t = 0$, from the identity $F(\Theta(s,0)) = h(s, 0) = \sigma(s)$ we easily deduce that $\Theta(s, 0) = \theta(s)$. It follows immediately that
\[
\Theta(1, 0) = \theta(1) = u_1.
\]
On the other hand, the assumption $h(0, t) = h(1, t) = v$ gives
\[
F(\Theta(0,t)) = F(\Theta(s, 1)) = F(\Theta(1, t)) = v.
\]
In particular, the restriction of $\Theta$ to the set
\[
\Gamma = \left(\{0\} \times [0, \, 1] \right) \cup \left([0, \, 1] \times \{1\} \right) \cup \left( \{1\} \times [0,\, 1] \right)
\]
is constant and, in particular, we obtain
\[
u_1 = \Theta(1, 0) = \Theta(0, 0) = u_0,
\]
and this concludes the proof.
\end{proof}

\subsection{Applications of global invertibility}

Let $\Om \subset \R^n$ be an open bounded set with smooth boundary and consider the Dirichlet problem
\begin{equation} \label{dirh} \begin{cases}
- \Delta u(x) = p(u(x)) + h(x) & \text{if $x \in \Om$},
\\ u(x) = 0 & \text{if $x \in \partial \Om$}.
\end{cases} \end{equation}
Let $(\lambda_k)_{k \in \N}$ denote the sequence of eigenvalues of the laplacian $-\Delta$ subject to Dirichlet boundary conditions and enumerate them in such a way that $0<\lambda_1 \leq \lambda_2 \leq \dots$ and
\[
\lim_{k \to \infty} \lambda_k = \infty.
\]

\bthm
Let $p \in C^1(\R)$ be a function of the form
\[
p(s) = as + b(s),
\]
where $|b(s)| \leq M$. Suppose that one of the following assumptions hold: \mbox{}
\begin{enumerate}[label={\textbf{(\alph*)}}, leftmargin=2.5\parindent]
\item For all $s \in \R$ there results $p^\prime(s) = a + b^\prime(s) < \lambda_1$.
\item There exists $k \in \N$ such that for all $s \in \R$ there results $\lambda_k < p^\prime(s) = a + b^\prime(s) < \lambda_{k+1}$.
\end{enumerate}
Then for any $h \in C^\alpha(\bar{\Om})$, $\alpha \in (0,1)$, there exists a unique $u \in C^{2, \alpha}(\bar{\Om})$ solution of \eqref{dirh}.
\ethm

\begin{proof}
Let $\X := \{ u \in C^{2, \alpha}(\bar{\Om}) \: : \: u \, \big|_{\partial \Om} \equiv 0\}$ and $\Y := C^{0,\alpha}(\bar{\Om})$. In view of \autoref{git}, it is sufficient to show that the map
\[
F(u) := - \Delta u - p(u) 
\]
is proper and locally invertible at all $u \in \X$.

\paragraph{Step 1.} A simple computation shows that
\[
\dr F(u)[v] := - \Delta v - p^\prime(u) v,
\]
and therefore $F$ is locally invertible at $u \in \X$ if and only if
\[
- \Delta v - p^\prime(u) v = 0 \iff v = 0.
\]
We now consider the bilinear form defined by the differential of $F$ at $u$, namely
\[
b: \X \times \X \to \Y, \quad b(u, v) := - \Delta v - p^\prime(u) v.
\]
Then $b$ is continuous,
\[
|b(u, v)| \leq \|u\|_\X \|v\|_\X,
\] 
and if the assumption {\textbf{(a)}} holds, then it is also coercive and Lax-Milgram theorem (see \autoref{thm.laxmilgram2}) applies. If {\textbf{(b)}} holds, we start off by considering the following eigenvalue problems:
\[ \begin{aligned} 
& \begin{cases} - \Delta v - \lambda_k v = \mu v & \text{if $x \in \Om$}, \\ u = 0 & \text{if $x \in \partial \Om$}, \end{cases}
\\[1em] & \begin{cases} - \Delta v - p^\prime(u) v = \tilde{\mu} v & \text{if $x \in \Om$}, \\ u = 0 & \text{if $x \in \partial \Om$}, \end{cases}
\\[1em] & \begin{cases} - \Delta v - \lambda_{k+1} v = \hat{\mu} v & \text{if $x \in \Om$}, \\ u = 0 & \text{if $x \in \partial \Om$}. \end{cases}
\end{aligned} \]
The assumption {\textbf{(b)}} implies that
\[
\hat{\mu}_j < \tilde{\mu}_j < \mu_j 
\]
for all $j \in \N$. However, we can compute these eigenvalues explicitly as
\[ 
\mu_j = \lambda_j - \lambda_k \quad \text{and} \quad \hat{\mu}_{j} = \lambda_j - \lambda_{k+1},
\]
and hence we conclude that
\[
\tilde{\mu}_k < 0 \quad \text{and} \quad \tilde{\mu}_{k+1} > 0.
\]
This shows that $\tilde{\mu}_j \neq 0$ for all $j \in \N$ and, as an immediate consequence, that $F$ is locally invertible.

\paragraph{Step 2.} To prove that $F$ is proper, suppose that $\|h_n- h\|_\Y \to 0$ and let $(u_n)_{n \in \N} \subset \X$ be a sequence of preimages, namely
\[
F(u_n) = h_n \quad \text{for all $n \in \N$.}
\]

\paragraph{Step 2.1} We claim that $\|u_n\|_\Y$ is bounded. If not, define $v_n := \frac{u_n}{\|u_n\|_\Y}$ and notice that it is well-defined and solves \eqref{dirh} with
\[
h = \frac{h_n}{\|u_n\|_\Y}.
\]
If we now apply the assumption $h(s) = as + b(s)$, we find that $v_n$ solves the equation
\[
- \Delta v_n + a v_n = U_n,
\]
where $U_n$ is a uniformly bounded function in $L^\infty(\Om)$ and, consequently, in every $L^p$-space for $1 \leq p < \infty$. On the other hand, the operator
\[
- \Delta + a \cdot \mathrm{Id}_\X
\]
is invertible for all admissible $a$'s, and hence $\|v_n\|_{W^{2,p}(\Om)} \leq C_p$ for all $p \in [1, \infty]$. By Sobolev embedding (see \hyperref[sobolevtheorem]{Theorem \ref{sobolevtheorem}}) the sequence $v_n$ is also bounded in $C^{0, \beta}(\Om)$ and, by Ascoli-Arzelà we also have that
\[
v_n \xrightarrow{n \to + \infty} v^\ast
\]
in $C^{0, \alpha}(\bar{\Om})$ for all $\alpha < \beta$ and $\| v^\ast \|_\Y = 1$. This is in contradiction with the fact that the sequence $U_n$ tends to zero in $\Y$ so $v^\ast$ must also satisfy
\[ \begin{cases}
- \Delta v^\ast + a v^\ast = 0,
\\ \|v^\ast\|_\Y = 1,
\end{cases} \]
which has $v^\ast \equiv 0$ as its unique solution.

\paragraph{Step 2.2.} We have proved that $u_n$ is bounded in $\Y$ so we only need to apply regularity theory to achieve boundedness in $\X$. Start by noticing that $u_n$ is a solution of
\[ \begin{cases}
- \Delta u_n(x) = \underbracket{p(u_n(x)) + h_n(x)}_{:= \theta_n} & \text{if $x \in \Om$},
\\ u_n(x) = 0 & \text{if $x \in \partial \Om$}.
\end{cases}\]
Since both $(u_n)_{n \in \N}$ and $(h_n)_{n \in \N}$ are bounded in $\Y$, we easily deduce that the sequence $\theta_n$ is bounded in $\Y$ as well. A well-known result in regularity theory gives
\[
\|u_n\|_\X \leq C,
\]
and by Ascoli-Arzelà we can find a subsequence $(u_{n_k})_{k \in \N}$ that converges to some $u^\ast$ in the topology $C^2(\bar{\Om})$. Finally, since $\theta_{n_k}$ converges in $\Y$, the elliptic regularity theory allows us to conclude that $u_{n_k}$ converges to $u \in \X$.
\end{proof}

To conclude this section, we recall the statement of the Lax-Milgram theorem in a more general form in which we do not require $b$ to be a bilinear form.

\bthm[Lax-Milgram]\index{Lax-Milgram theorem}
Let $H$ be a Hilbert space, and let $a : H \times H \to \R$ be a function satisfying the following properties: \mbox{}
\begin{enumerate}[label=(\roman*)]
\item $a(0, v) = 0$ for all $v \in H$ and $v \mapsto a(u, v)$ is linear for all $u \in H$.
\item For all $v \in H$ and all $(u_1, u_2) \in H \times H$ it turns out that
\[
\left| a(u_1, v) - a(u_2, v) \right| \leq M  \|u_1 - u_2\| \|v\|.
\]
\item There exists a constant $\nu > 0$ such that
\[
a(u_1, u_1 - u_2) - a(u_2, u_1 - u_2) \geq \nu  \|u_1 - u_2 \|^2 \quad \text{for all $(u_1, u_2) \in H \times H$}.
\]
\end{enumerate}
Then for all $F \in H^\ast$ there exists a unique element $u \in H$ such that
\[
a(u, v) = F(v) \quad \text{for all $v \in H$},
\]
and there exists a positive constant which only depends on $\nu$ such that
\[
\|u\| \leq \frac{1}{c(\nu)} \|F\|_{H^\ast}.
\]
\ethm

\brmk
If $a : H \times H \to \R$ is a bilinear form, then the last condition is equivalent to saying that $a$ is {\em coercive}.\index{coercivity}
\ermk

\section{Global inversion with singularities}

In this section, we will investigate {\em global invertibility} of maps when $\Sigma$ does not satisfy the assumptions of \autoref{git}. For this it will be convenient to deal with $C^2$-maps $F : \X \to \Y$, where $\X$ and $\Y$ are Banach spaces, and replace $\Sigma$ with a slightly larger set:
\[
\Sigma' := \{u \in \X \: : \: F'(u) \notin \inv(\X, \Y)\}.
\]
Let $F \in C^2(\X, \Y)$ and $u \in \Sigma'$. We assume that the following holds: \mbox{}
\begin{enumerate}[label=(\Alph*)]
\item The kernel of $F'(u)$ is one-dimensional, generated by some $\phi \in \X \setminus \{0\}$. The range is closed and has codimension one.
\item There exists $\tilde{\phi} \in \X$ such that $F''(u)[\tilde{\phi}, \phi] \notin \mathrm{Ran}(F^\prime(u))$.
\end{enumerate}
We say that a subset $M$ of $\X$ is a $C^1$-manifold of codimension one in $\X$ if for all $u^\ast \in M$ there exist $\delta > 0$ and a functional $\Gamma : B_\delta(u^\ast) \to \R$ of class $C^1$ such that
\[
M \cap B_\delta(u^\ast) = \{u \in B_\delta(u^\ast) \: : \: \Gamma(u) = 0\},
\]
and $\Gamma'(u^\ast) \neq 0$.

\bl
Suppose that for all $u\in \Sigma'$ the conditions $(A)$ and $(B)$ hold. Then $\Sigma'$ is a $C^1$-manifold of codimension one in $\X$.
\el

\bd \index{singular point!ordinary}
We say that $u \in \Sigma'$ is an {\em ordinary} singular point if $(A)$ holds and
\[
F''(u)[\phi, \phi] \notin \mathrm{Ran}(F'(u)),
\]
where $\phi$ is the element that generates the kernel (by $(A)$).
\ed

\bl
Let $u^\ast$ be an ordinary singular point. Then there exist $\epsilon > 0$ and a map $\Psi \in C^1(B_\epsilon(u^\ast),  \Y)$ such that \mbox{}
\begin{enumerate}[label=(\roman*), leftmargin=2.5\parindent]
\item $\Psi'(u^\ast) \in \inv(\X, \Y)$;
\item $\Psi(u) = F(u)$ for all $u \in \Sigma' \cap B_\epsilon(u^\ast)$.
\end{enumerate}
\el

%\begin{proof}
%First, notice that $\Sigma^\prime \cap B_\delta(u^\ast) = \Gamma^{-1}(0)$. Let $\Psi : B_\delta(u^\ast) \to \Y$ be the map defined by setting
%\[
%\Psi(u) := F(u) + \Gamma(u)z.
%\]
%Then $\Psi$ is $C^1$-regular, $\Psi(u)$ coincides with $F(u)$ for all $u \in \Sigma^\prime \cap B_\delta(u^\ast)$ and its differential is given by
%\[
%\Psi^\prime(u^\ast)u = F^\prime(u^\ast)u + \Gamma^\prime(u^\ast)(u)z.
%\]
%Setting $u = t \phi + w$, we find that
%\[ \begin{aligned}
%\Psi^\prime(u^\ast)u & = F^\prime(u^\ast)w + t \Gamma^\prime(u^\ast)(\phi)z + \Gamma^\prime(u^\ast)(w)z =
%\\[1em] & = F^\prime(u^\ast)w + t \langle \Psi, \, F^{\prime \prime}(u^\ast)[\phi, \, \phi] \rangle z + \langle \Psi, \, F^{\prime \prime}(u^\ast)[w, \, \phi] \rangle z.
%\end{aligned}\]
%Finally, observe that $\Psi^\prime(u^\ast)u = v$ has a unique solution when $\langle \Psi, \, F^{\prime \prime}(u^\ast)[\phi, \, \phi] \rangle \neq 0$; thus, if $u^\ast$ is an ordinary singular point, the map $\Psi^\prime(u^\ast)$ is invertible.
%\end{proof}

\bcor
If every $u \in \Sigma'$ is an ordinary singular point, then $F(\Sigma')$ is a $C^1$-manifold of codimension one in $\Y$.
\ecor

\bl
Let $u^\ast$ be an ordinary singular point with $\mathrm{Ker}(F'(u^\ast)) = \langle \phi \rangle_\R$. Assume that
\[
\langle \Psi, F^{''}(u^\ast)[\phi, \phi] \rangle > 0,
\]
and set $v^\ast := F(u^\ast)$. Then there are $\epsilon, \, \sigma > 0$ such that the equation
\[
F(u) = v^\ast + sz \quad \text{for $u \in B_\epsilon(u^\ast)$},
\]
has two solutions for all $0 < s < \sigma$ and none for $- \sigma < s < 0$. 
\el

\bthm \index{global inversion theorem!singularities}
Let $F \in C^2(\X, \Y)$ be a proper function. Assume that every $u \in \Sigma'$ is an ordinary singular point, the equation
\[
F(u) = v
\]
admits a unique solution for all $v \in F(\Sigma')$, and $\Sigma'$ is connected. Then there are two open connected subsets $\Y_0$ and $\Y_2$ of $\Y$ such that
\[
\Y = \Y_0 \cup \Y_2 \cup F(\Sigma'),
\]
and it turns out that
\[
[v] = \begin{cases}0 & \text{if $v\in \Y_0$}, \\ 1 & \text{if $v\in F(\Sigma')$}, \\ 2 & \text{if $v\in \Y_2$}. \end{cases}
\]
\ethm

\part{Variational Methods}

 \chapter{Critical Points and Natural Constraint} \thispagestyle{empty}

[{\color{red}Da scrivere}]

\section{Existence of extrema}

Let $\X$ be a Banach space and let $J$ be a functional, that is, a continuous mapping $J : \X \to \R$.

\bd
We say that $z \in \X$ is a {\em local minimizer}\index{local minimum} (resp. {\em local maximizer}) of $J$ if there exists a neighbourhood $U \ni z$ such that
\[
\text{$J(z) \leq J(u)$ (resp. $J(z) \geq J(u)$)} \quad \text{for all $u \in U$}.
\]
If the above inequality is strict (except at $z$), then we say that $u$ is a {\em strict local minimum} (resp. {\em strict local maximum}) of $J$.
\ed

\bd
We say that $z \in \X$ is a {\em global minimum}\index{global minimum} (resp. {\em global maximum}) of $J$ if
\[
J(z) \leq J(u) \quad \text{for all $u \in \X$ (resp. $J(z) \geq J(u)$)}.
\]
\ed

\bpr
Let $z$ be a local minimum and assume that $J$ is differentiable at $z$. Then $z$ is a stationary point, that is,
\[
\dr J(z) = 0.
\]
\epr

Our first goal is to investigate the connection between properties such as semicontinuity and coerciveness and existence of extrema, but we first need to recall a few definitions.

\bd\index{coercive}
We say that a functional $J : \X \to \R$ is {\em coercive} if
\[
\lim_{\|u\|\to \infty} J(u) =  \infty.
\]
\ed

\bd \index{lower semicontinuous}
We say that a functional $J : \X \to \R$ is {\em lower semicontinuous} if for every sequence $u_n \in \X$ such that $u_n \rightharpoonup u$, it turns out that
\[
J(u) \leq \liminf_{n \to \infty} J(u_n). 
\]
\ed

\bl
Let $\X$ be a reflexive Banach space and let $J$ be a coercive weakly lower semicontinuous functional. Then there exists $\alpha \in \R$ such that
\[
J(u) \geq \alpha \quad \text{for all $u \in \X$}.
\]
\el

\bthm\label{thm.d.1}
Let $\X$ be a reflexive Banach space and let $J$ be a coercive weakly lower semi-continuous functional. Then $J$ has a global minimum, that is,
\[
\exists z \in \X \: : \: J(z) \leq J(u) \quad \text{for all $u \in \X$}.
\]
\ethm

\begin{proof}
The previous lemma asserts that $m := \inf_{u \in \X} J(u)$ is finite. Let $u_n \in \X$ be a minimizing sequence sequence, that is,
\[
J(u_n) \xrightarrow{n \to + \infty} m.
\]
Since $J$ is coercive the sequence must be equibounded (i.e., $\|u_n\| \leq R^\prime$), and hence it admits a subsequence weakly converging to some $z \in \X$. However, we have
\[
J(z) \leq \lim_{k \to \infty} J(u_{n_k}) = m \implies J(z) = m
\]
by lower semicontinuity, and this concludes the proof.
\end{proof}

\subsection{Application to PDEs analysis}

We now show how to apply \autoref{thm.d.1} to find a solution to the following Dirac boundary problem under some assumptions on the nonlinearity:
\begin{equation} \mathbf{\tag{D}} \label{eq.pde.1}
\begin{cases} - \Delta u(x) = f(x, u) & \text{if $x \in \Om$}, \\
u(x) = 0& \text{if $x \in \partial \Om$}. \end{cases}
\end{equation}
Suppose that $\X$ is a Hilbert space and consider the functional
\[
J(u) = \frac{1}{2} \|u\|^2 - \Phi(u),
\]
where $\Phi(u) := \int_\Om F(x, u) \, \dr x$ and $F(x, u) = \int_0^u f(x, s) \, \dr s$.

\bthm
Assume that $\Phi \in C^1(\X, \R)$ is weakly continuous and satisfies the estimate
\[
|\Phi(u)| \leq a_1 + a_2 \|u\|^\alpha 
\]
for positive constants $a_1, a_2 > 0$ and $\alpha < 2$. Then $J$ achieves a global minimum at some $z \in \X$ and it turns out that
\[
\Phi'(z) = z.
\]
\ethm

\begin{proof}
The estimate on $\Phi$ readily gives
\[
J(u) \geq \frac{1}{2}\|u\|^2 - a_1 - a_2 \|u\|^\alpha,
\]
and it is easy to check that for $\alpha < 2$ the functional $J$ is always coercive. The function $\| \cdot \|^2$ is weakly lower semicontinuous and $\Phi$ is weakly continuous, so \hyperref[thm.d.1]{Theorem \ref{thm.d.1}} gives the existence of a critical point. Finally $J \in C^1(\X,\R)$ yields
\[
0 = J'(z) = z - \Phi'(z) \implies \Phi'(z) = z.
\]
\end{proof}

We would like to find assumptions on the nonlinearity rather than $\Phi$ itself so we start off by requiring that there are $a_1 \in L^2(\Om)$, $a_2 > 0$ and $0 < q < 1$ such that
\begin{equation} \label{eq.d.2} |f(x, u)| \leq a_1(x) + a_2 |u|^q. \end{equation}
Set $\X := H_0^1(\Om)$ and endow it with its homogeneous norm. Since $\X$ is compactly embedded in $L^2(\Om)$, we immediately conclude that
\[
\Phi(u) := \int_\Om F(x, u) \, \dr x
\]
is $C^1(\X,\R)$ and weakly continuous.

\bthm
Let $f$ be locally Hölder-continuous and suppose that \eqref{eq.d.2} holds. Then $J$ has a critical point, which is a weak solution of \eqref{eq.pde.1}. \ethm

\begin{proof}
Integrating the estimate \eqref{eq.d.2} leads to
\[
|\Phi(u)| \leq a_5 \|u\| + a_6 \|u\|^{q+1},
\]
and since $q + 1 < 2$ we can apply the result above to infer the existence of a critical point $z$ for $J$ such that
\[
J'(z) = z - \Phi'(z) = 0 \implies \Phi'(z) = z.
\]
\end{proof}

\brmk
One can prove that \eqref{eq.d.2} can be replaced with the assumption
\[
\lim_{|s|\to\infty} \frac{f(x, s)}{s} = 0
\]
uniformly with respect to $x$.
\ermk

\bex
Consider the boundary value problem
\begin{equation} \mathbf{\tag{$D$}} \label{eq.pde.2}
\begin{cases} - \Delta u = \lambda u - f(u) & \text{if $x \in \Om$},
\\ u = 0& \text{if $x \in \partial \Om$}, \end{cases}
\end{equation}
where $\lambda$ is a given parameter and $f : [0, \infty) \to \R$ is locally Hölder and satisfies
\[
\lim_{u \to 0^+} \frac{f(u)}{u} = 0, \quad \lim_{u \to + \infty} \frac{f(u)}{u} = + \infty.
\]
We claim that \eqref{eq.pde.2} has a positive solution for any $\lambda > \lambda_1$, the first (smallest) eigenvalue of the laplacian operator with DBC. Let $\xi := \xi_\lambda > 0$ be such that
\[
\lambda \xi = f(\xi) \quad \text{and} \quad \lambda u - f(u) > 0 
\]
for all $u \in (0,  \xi)$ and let $g_\lambda : \R \to \R$ be given by
\[
g_\lambda(x) := \begin{cases}  0 & \text{if $u < 0$ or $u > \xi$,}
\\ \lambda u - f(u) & \text{if $0 \leq u \leq \xi$.} \end{cases} 
\]
Consider the auxiliary boundary value problem
\begin{equation} \mathbf{\tag{$D_\lambda$}} \label{eq.pde.3}
\begin{cases} - \Delta u = g_\lambda(u) & \text{if $x \in \Om$},
\\ u = 0& \text{if $x \in \partial \Om$}. \end{cases}
\end{equation}
By the maximum principle any nontrivial solution $u$ of \eqref{eq.pde.3} must be positive and
\[
0 < u(x)< \xi_\lambda \quad \text{for all $x \in \Om$}
\]
so that $u$ is also a positive solution of \eqref{eq.pde.2}. Since $g_\lambda$ is locally Hölder-continuous and bounded, the theorem above applies to the functional
\[
J_\lambda(u) := \frac{1}{2} \|u\|^2 - \lambda \int_\Om G_\lambda(u) \, \dr x.
\]
We now claim that $\inf_{u \in \X} J_\lambda(u)$ is strictly less than zero. To prove this, let $\varphi_1 \in \X$ be the first (positive and unitary) eigenfunction of $-\Delta$, that is,
\[
- \Delta \varphi_1 = \lambda_1 \varphi_1, \quad \| \varphi_1 \|_{L^2(\Om)} = 1.
\]
For $t > 0$ small, one has $g_\lambda(t \varphi_1) = \lambda t \varphi_1 - f(t \varphi_1)$. Since $f(u) \in o(\|u\|)$, we immediately find that
\[
J_\lambda(t \varphi_1) = \frac{1}{2}( \lambda_1 - \lambda ) t^2 + o(t^2),
\]
which is strictly negative if we choose $t$ to be small enough.
\eex

\section{Constrained critical points}

Let $J : \X \to \R$ be a differentiable functional and let $\M$ be either a smooth Hilbert submanifold or a Hilbert space.

\bd
A \textit{constrained critical point}\index{constrained critical point} of $J$ on $\M$ is a point $z \in \M$ such that
\[
\dr J \, \big|_\M(z) = 0,
\]
which is equivalent to
\[
\dr J(z)[v] = 0 \quad \text{for all $v \in T_z \M$},
\]
where $T_z \M$ is the tangent space at $z$ to $\M$.
\ed

\brmk
Similarly, using the $\nabla_\M$ gradient, $z \in \M$ is a constrained critical point if
\[
\langle \nabla_\M J(z), v \rangle \quad \text{for all $v \in T_z \M$},
\]
which is as to say that $J'(z)$ is orthogonal to $T_z \M$.
\ermk

\brmk
Let $\gamma : [0, 1] \to \M$ be a smooth curve with $\gamma(0) = z$. Consider the real-valued function $\phi(t) := J \circ \gamma(t)$ and notice that
\[
\phi'(0) = J'(z)[\gamma'(0)].
\]
If $z$ is a critical point of $J$ constrained on $\M$, then $t = 0$ is a critical point of $\phi$. Vice versa, $z$ is a constrained critical point if
\[
\frac{\dr}{\dr t} \, \big|_{t = 0} J(\gamma(t)) = 0
\]
for all differentiable curves $\gamma$ satisfying $\gamma(0) = z$.
\ermk

Suppose that $\M$ has codimension one, that is, there exists $G: \X \to \R$ of class $C^1$ such that $\M = G^{-1}(0)$. We can always write
\[
\X = T_z \M \oplus \mathrm{Span}(\nabla G(z)),
\]
and using the Lagrange multiplier rule leads to
\[
\nabla J(z) = \lambda \nabla G(z) \implies \lambda = \frac{ \langle \nabla J(z), \nabla G(z) \rangle }{ \| \nabla G(z) \|^2}.
\]

\paragraph{Application to nonlinear eigenvalues}

Let $\Om \subset \R^n$ be a smooth bounded set and suppose that $f$ satisfies \eqref{eq.3.1}. Let $\X := H_0^1(\Om)$ and
\[
\Phi(u) := \int_\Om F(x, u) \, \dr x. 
\]
Consider the $C^1$-manifold given by
\[
\M = \{u \in \X \: : \: \|u\|^2 = 1 \} = G^{-1}(0),
\]
where $G(u) := \|u\|^2 - 1$. It is easy to verify this since $G \in C^1(\X,\R)$ and $\dr G(u) \equiv 2$. If $u$ is a constrained critical point of $\Phi$ on $\M$, then
\[
\nabla \Phi(u) = \lambda u \implies \lambda \int_\Om \nabla u \cdot \nabla v \, \dr x = \int_\Om f(x, u) v \, \dr x.
\]
It follows that $u$ is a weak solution of the boundary-value problem
\[
\begin{cases} - \lambda\Delta u(x) = f(x, u) & \text{if $x \in \Om$},
\\ u(x) = 0& \text{if $x \in \partial \Om$}. \end{cases}
\]
Notice that if $f$ is homogeneous, then one can consider the scaling $\lambda^{\frac{1}{p-1}} u$ that solves the same boundary-value problem with $\lambda = 1$.

\section{Natural constraint and Nehari manifolds}

Let $\X$ be a Hilbert space and let $J \in C^1(\X, \R)$.

\bd
A $C^1$-submanifold $\M$ is called a \textit{natural constraint}\index{natural constraint} for $J$ if there exists another functional $\tilde{J} \in C^1(\X, \R)$ such that
\[
\nabla_\M \tilde{J}(u) = 0 \iff J^\prime(u) = 0.
\]
\ed

\brmk
An example of a natural constraint is the so-called Nehari manifold\index{Nehari manifold} given by
\[
\M := \{u \in \X \setminus \{0\} \: : \: \langle J'(u), u \rangle = 0\}.
\]
\ermk

\bpr \label{prop.e.1}
Let $J \in C^2(\X, \R)$ and suppose that the corresponding Nehari manifold $\M$ is nonempty. Assume that the following conditions hold: \mbox{}
\begin{enumerate}[label=(\roman*)]
\item There exists $r > 0$ such that $\M \cap B_r(0) = \varnothing$.
\item For all $u \in \M$ it turns out that $\dr^2 J(u)[u, u] \neq 0$.
\end{enumerate}
Then $\M$ is a natural constraint for $J$ with $\tilde{J} = J$.
\epr

\begin{proof}
Let $G(u) := \langle J'(u), u \rangle$ so that $\M = G^{-1}(0)$ and notice that $G$ belongs to $C^1$. Since
\[
G'(u)[u] = \dr^2J(u)[u, u] + \underbracket{\dr J(u)[u]}_{= 0} = \dr^2J(u)[u, u] \neq 0,
\]
we immediately conclude that $\M$ is a $C^1$-submanifold. If $\nabla J\, \big|_\M (u) = 0$, then
\[
\nabla J(u) = \lambda \nabla G(u) \implies \langle \nabla J(u), u \rangle = \lambda \langle \nabla G(u), u \rangle.
\]
However, for any $u \in \M$ the left-hand side is zero and the right-hand side is nonzero so $\lambda$ must be equal to $0$ and $\M$ is a natural constraint for $J$.
\end{proof}

\subsection{Applications to PDEs}
\label{subsec:daosd}

Let $\Om \subset \R^n$ be an open bounded smooth set and consider the problem
\begin{equation} \label{e.1}
\begin{cases} - \Delta u = |u|^{p-1} u, & \text{if $x \in \Om$},
\\ u \, \big|_{\partial \Om} \equiv 0. \end{cases}
\end{equation}
We will need the compactness of the embedding
\[
L^{p+1}(\Om) \hookrightarrow H_0^1(\Om)
\]
so we must assume that $1 < p < \frac{n + 2}{n-2}$. Let $\X := H_0^1(\Om)$ equipped with the norm
\[
\|u\|_\X := \int_\Om |\nabla u|^2 \, \dr x.
\]
The variational formulation of \eqref{e.1} consists of finding critical points of the functional
\begin{equation*} J(u) = \frac{1}{2} \int_\Om |\nabla u|^2 \, \dr x - \frac{1}{p+1} \int_\Om |u|^{p+1} \, \dr x. \end{equation*}
It is easy to see that $J \in C^2(\X,\R)$ and that $J$ is unbounded on $\X$. Indeed, letting $\varphi_1$ be an eigenfunction of $\lambda_1(\Om)$, leads to
\[
\lim_{t \to + \infty} J(t \varphi_1) = - \infty. 
\]
In a similar fashion, one can prove that
\[
\sup_{u \in \X} J(u) = \infty,
\]
for example by taking the sequence $u_n(x) := \sin(nx) \chi(x)$, where $\chi$ is a cutoff function with support in $\Om$.

\bpr
The Nehari manifold
\begin{equation*} \M := \left\{ u \in \X \setminus \{0\} \: : \: \int_\Om |\nabla u|^2 \, \mathrm{d}x = \int_\Om |u|^{p+1} \, \mathrm{d}x \right\} \end{equation*}
is a natural constraint for $J$.
\epr

\begin{proof}
First, notice that
\[ 
\dr J(u)[v] = \int_\Om \nabla u \cdot \nabla v \, \dr x - \int_\Om |u|^{p-1} uv \, \dr x,
\]
so that $G(u) = \langle \nabla J(u), u\rangle$ is actually given by $\|u\|_\X^2 - \|u\|_{L^{p+1}(\Om)}^{p+1}$, which means that a nonzero $u \in \X$ belongs to $\M$ if and only if
\[
\|u\|_\X^2 = \int_\Om |u|^{p+1} \, \dr x.
\]
Using Sobolev embedding we can find a constant $C_{p, \Om} > 0$ such that
\[
\|u\|_{p+1} \leq C_{p, \Om} \|u\|_\X,
\]
and therefore for any $u \in \M$ it turns out that
\[
\|u\|_\X^2 = \|u\|_{p+1}^{p+1} \leq C_{p, \Om} \|u\|_\X^{p+1} \stackrel{p>1}{\implies} \|u\|_\X^{p-1} \geq \frac{1}{C_{p, \Om}} > 0.
\]
This proves ({\romannumeral 1}) of \hyperref[prop.e.1]{Proposition \ref{prop.e.1}} with $r$ equal to a negative power of $C_{p, \Om}$. A simple computation shows that the second differential is given by
\[
\dr^2J(u)[v, w] = \int_\Om \nabla w \cdot \nabla v \, \dr x - p \int_\Om |u|^{p-1} wv \, \dr x,
\]
which computed at $v=w=u$ gives
\[
\dr^2J(u)[u, u] = \|u\|_\X^2 - p \|u\|_{p+1}^{p+1}.
\]
However, taking into account that $p > 1$, for $u \in \M$ we have
\[
\dr^2J(u)[u, u] =(1 - p) \|u\|_\X^2 \neq 0,
\]
which shows that $\M$ is a natural constraint for $J$.
\end{proof}

\brmk
The functional is bounded from below on $\M$ since
\[
J \, \big|_{\M} (u) = \left( \frac{1}{2} - \frac{1}{p+1} \right) \|u\|_\X^2 \geq \left( \frac{1}{2} - \frac{1}{p+1} \right) r > 0.
\]
\ermk

We proved that $\M$ is a natural constraint for $J$ so it only remains to prove that $J$ attains a minimum point on $\M$, provided\footnote{We will not prove it here, but the assertion is false when $p$ is equal to the critical exponent.} that $1<p < \frac{n+2}{n-2}$. Let $u_n \rightharpoonup \bar{u} \in \X$ be a minimizing sequence. From the compactness of the Sobolev embedding it follows that
\[
\int_\Om |u_n|^{p+1} \, \dr x \xrightarrow{n \to + \infty}  \int_\Om |\bar{u}|^{p+1} \, \dr x.
\]
Notice that $u_n \in \M$ for all $n \in \N$, and hence we have
\[
 \int_\Om |\bar{u}|^{p+1} \, \dr x = \lim_{n \to + \infty} \int_\Om |u_n|^{p+1} \, \dr x \geq r^2 \implies \bar{u} \not \equiv 0.
 \]
There are now two possibilities that we need to discuss separately: \mbox{}
\begin{enumerate}[label=\textbf{(\alph*)}]
\item If $\|u_n\|_\X \to \|\bar{u}\|_\X$, then $\bar{u} \in \M$ and
\[
J \, \big|_\M (u) = \left( \frac{1}{2} - \frac{1}{p+1} \right) \|u\|_\X^2
\]
is lower semicontinuous. Therefore $\bar{u}$ is a minimizer for $J$ on $\M$.
\item If $\lim_{n \to \infty} \|u_n\|_\X > \|\bar{u}\|_\X$, then
\[
\|\bar{u}\|_\X^2 = \mu \lim_{n \to \infty} \|u_n\|_\X^2,
\]
for some $\mu \in (0, \, 1)$. However, we have
\[
\|\bar{u}\|_\X^2 = \mu \lim_{n \to \infty} \int_\Om |u_n|^{p+1} \, \mathrm{d}x = \mu \|\bar{u}\|_{L^{p+1}(\Om)}^{p+1}
\]
and if we take $\nu \in (0, 1)$ such that $\nu^{p-1} = \mu$, then $\nu \bar{u} \in \M$. This is a contradiction since $\bar{u}$ is the limit of a minimizing sequence.\end{enumerate}
 \chapter{Deformations and Palais-Smale Sequences} \thispagestyle{empty}

{\color{red}[...]}

\section{Introduction to deformations}

Let $J : U \subset \X \to \R$ be a functional defined on a open subset $U$ of a Banach space $\X$ and let $a \in \R$. We will always denote either by $\X^a$ or $J^a$ the sublevel\index{sublevel} 
\[
\{ u \in \X \: : \: J(u) \leq a \}.
\]

\bd[Deformation] \index{deformation}
A map $\eta \in C(A,\X)$ is a \textit{deformation} of $A \subset \X$ in $\X$ if it is homotopic to the identity. Namely, there exists a homotopy $H$ such that
\[
H(0, u) = u, \quad H(1, u) = \eta(u) \quad \text{for all $u \in \X$}.
\]
\ed

%The idea behind deforming a set into another one is the following. Since a deformation is a continuous map homotopic to the identity, we expect that $A$ and $\eta(A)$ have the same topological properties.

%More specifically, if $[a, \, b] \subset \R$ does not contain any critical point of $J$, then it can be proved that under some assumptions on $\X$ the sublevel $\X^b$ can be deformed into $\X^a$. On the other hand, the presence of an obstacle is often (but not always) a consequence of the existence in the given interval of a critical point.

\bex
Let $\M$ be a compact hypersurface in $\R^n$. Suppose that $b$ is not a critical level for $J$ on $\M$ and notice that
\[
J^{-1}(b) = \{ x \in M \: : \: J(x) = b \} 
\]
is a smooth submanifold of $\M$ and at any point the vector $- \nabla_\M J(x) \neq 0$. By compactness we have
\[
\min_{x \in J^{-1}(b)} |\nabla_\M J(x)| \geq C > 0, 
\]
and hence we can deform $J^{-1}(b)$ into $J^{-1}(b- \epsilon)$ for some $\epsilon$ small enough. Now, if there are no critical levels in $[a, \, b]$, we can repeat the same process over and over again, until we obtain $J^{-1}(a)$ as a deformation of $J^{-1}(b)$.
\eex

To better understand the change of topological properties after crossing critical levels, the following example is instructive.

\bex
Let $\M$ be the $2$-torus and let $J(x, y, z) = z$. The critical points of $J$ on $\M$ are the four points $p_i$ where the gradient of $J$ is orthogonal to $\M$. If we set
\[
c_i := J^{-1}(p_i),
\]
then it can be proved that
\[
M^a \cong \begin{cases}
\T^2 & \text{if $a > c_4$},
\\ \T^2 \setminus B_\varphi & \text{if $c_4 \geq a > c_3$},
\\ S^1 \times [0, 1] & \text{if $c_3 \geq a > c_2$},
\\ B_\varphi & \text{if $c_2 \geq a > c_1$},
\\ \varnothing & \text{if $a < c_1$}. \end{cases}
\]
\eex

\section{The steepest descent flow}\index{steepest descent flow}

Let $W \in C^{0, 1}(\X, \X)$ be a Lipschitz function defined on a Hilbert space $\X$ and let $\alpha(t, u)$ denote the solution of the Cauchy problem
\begin{equation}\label{eq.e.3}
\begin{cases} \alpha'(t) = W(\alpha(t,u)),
\\ \alpha(0) = u \in \X. \end{cases}
\end{equation}
The Lipschitz-Cauchy theorem gives the existence of a unique solution $\alpha(t, u)$, defined in a neighbourhood of $t = 0$, that depends continuously on the initial data. Let
\[
(t_u^-, \, t_u^+)
\]
denote the maximal interval of existence for a given $u \in \X$. We would like to find sufficient condition for $\alpha$ to be globally defined in $t > 0$ (i.e., $t_u^+ = + \infty$).

\bl
If $t_u^+ < + \infty$, then $\alpha(t, u)$ has no limit points as $t \nearrow t_u^+$.
\el

\begin{proof}
We argue by contradiction. If there exists $v \in \X$ such that $\alpha(t, \, u) \nearrow v$, then let $\beta$ denote the solution of the Cauchy problem \eqref{eq.e.3} with $u = v$. Then
\[
\text{$\beta$ is well-defined in a neighbourhood of $t^+$, $(t^+ - \epsilon, \, t^+ + \epsilon)$},
\]
and therefore the function
\[
\begin{cases} \alpha(t, \, u) & \text{if $t \in (t^-, \, t^+)$,}
\\ \beta(t, \, v) & \text{if $t \in [t^+, \, t^+ + \epsilon)$}, \end{cases}
\]
is a solution of \eqref{eq.e.3} with initial datum $u$, defined in a strictly bigger interval than the maximal one.
\end{proof}

\bl
Let $A \subset \X$ be closed and suppose that
\[
\|W(u)\| \leq C \quad \text{for all $u \in A$}.
\]
Let $u \in A$ be such that $\alpha(t, u) \in A$ for all $t \in [0, t_u^+)$. Then $t_u^+ = + \infty$.
\el

\begin{proof}
Suppose that $t_u^+ < \infty$. For all $t_i, t_j \in [0, \, t^+)$ we have
\[
\alpha(t_i, u) - \alpha(t_j, u) = \int_{t_j}^{t_i} \alpha'(s, u) \, \dr s= \int_{t_j}^{t_i} W(\alpha(s, u)) \, \dr s. 
\]
Since $W$ is bounded on $A$ and $\alpha(s,u) \in A$ for all $s$, it turns out that
\[
\| \alpha(t_i, u) - \alpha(t_j, u) \| \leq C |t_i - t_j|.
\]
Take the limit as $t_i \nearrow t_u^+$ and notice that the estimate implies $\alpha(t_i, u)$ Cauchy, which means that it converges to some point in $A$: a contradiction with the previous lemma.
\end{proof}

We are now ready to introduce the steepest descent flow. Assume that there exists a functional $G \in C^{1, 1}(\X, \R)$ such that
\[
\M = G^{-1}(0) \quad \text{and} \quad \text{$G'(u) \neq 0$ for $u \in \M$}. 
\]
Let $J \in C^{1, 1}(\X, \R)$ and consider the function
\[
W(u) = - \left[ J'(u) - \frac{ \langle J'(u), \, G'(u) \rangle }{\|G'(u)\|^2} G'(u) \right], 
\]
which is well-defined in a neighbourhood of $\M$, belongs to $C^{0, 1}$ as required, and coincides with $- \nabla_\M J(u)$ for all $u \in \M$.

The solution $\alpha$ of \eqref{eq.e.3} with this choice of $W$ is called {\em steepest descent flow} of $\M$ and it satisfies the following property:
\[
\alpha(0) \in \M \iff \text{$\alpha(t) \in \M$ for all $t \in (t_u^-, \, t_u^+)$}.
\]
Indeed, it is easy to verify that
\[ \begin{aligned}
\frac{\dr}{\dr t} G(\alpha(t)) & = \langle G'(\alpha(t)),  \alpha'(t) \rangle
\\ & = \langle G'(\alpha(t)), W(\alpha(t)) \rangle
\\ & = -\langle G'(\alpha(t)), J'(\alpha(t)) \rangle + \frac{\langle G'(\alpha(t)), \, J'(\alpha(t)) \rangle}{\|G'(\alpha(t))\|^2} \langle G'(\alpha(t)), G'(\alpha(t)) \rangle = 0,
\end{aligned}\]
which means that $G$ is constant along $\alpha$, and thus
\[
u \in \M \iff G(u) = 0 \iff G(\alpha(t)) = 0 \iff \alpha(t, u) \in \M. 
\]

\bl \label{lemma.f.4}
Under the assumptions above, the steepest descent flow $\alpha$ relative to $J$ satisfies the following properties: \mbox{}
\begin{enumerate}[label=\textbf{(\arabic*)}]
\item The function $t \mapsto J(\alpha(t, u))$ is nonincreasing for $t \in [0, t_u^+)$.
\item For $t, \tau \in [0, t_u^+)$ we have
\begin{equation} \label{eq.e.4}
J(\alpha(t, u)) - J(\alpha(\tau, u)) = - \int_\tau^t \| \nabla_\M J (\alpha(s, u)) \|^2 \, \dr s.
\end{equation}
\item If $J$ is bounded from below on $\M$, then $t_u^+ = \infty$ for all $u \in \M$.
\end{enumerate}
\el

\begin{proof}First, notice that
\[ \begin{aligned}
\frac{\dr}{\dr t} J(\alpha(t,u)) & = - \langle J'(\alpha(t,u)), \nabla_\M J(\alpha(t,u)) \rangle, 
\\ & - \| \nabla_\M J(\alpha(t,u)) \|^2
\end{aligned} \]
since $\nabla_\M J(\alpha)$ is the projection of $J'(\alpha)$ on $T_\alpha \M$. The first two properties follow easily from this so we only need to deal with the third. Let $u \in \M$ and suppose that
\[
t_u^+ < \infty.
\]
Apply \eqref{eq.e.4} with $\tau = 0$ to get
\[
J(\alpha(t,u)) - J(u) = - \int_0^t \| \nabla_\M J (\alpha(s, u)) \|^2 \, \dr s
\]
and, since $J$ is bounded from below on $M$, it follows that
\[
\int_0^t \| \nabla_\M J (\alpha(s, u)) \|^2 \, \dr s \leq a < + \infty
\]
for some positive constant $a$. Now let $t_i \nearrow t_u^+$ and notice that putting all together yields
\[
\| \alpha(t_i,u) - \alpha(t_j,u) \| \leq \int_0^t \| \nabla_\M J (\alpha(s, u)) \| \, \dr s.
\]
Using Hölder's inequality we get
\[
\| \alpha(t_i,u) - \alpha(t_j,u) \| \leq \sqrt{a} |t_i - t_j|^{\frac{1}{2}},
\]
which gives a contradiction with the fact that $\alpha(t_i,u)$ cannot have limit points.
\end{proof}

\brmk
If $J$ is $C^1$ only, the steepest descent flow might not be defined. However, there is a way to generalize the gradient vector field in such a way that \hyperref[lemma.f.4]{Lemma \ref{lemma.f.4}} holds.
\ermk

\bd[Pseudo-gradient] \index{pseudo-gradient vector field}
Let $J \in C^1(\X,\R)$. A \textit{pseudo-gradient} vector field (often referred to as $\Psi$-gradient v.f.) for $J$ on
\[
\X_0 := \left\{ u \in \X \: : \: \nabla J(u) \neq 0 \right\} 
\]
is a $C^{0, 1}(\X_0, \X)$ vector field $V$ satisfying the following properties for all $u \in \X_0$:
\begin{equation} \label{eq.f.1} \begin{aligned}
& \| V(u) \| \leq 2 \| \nabla J(u) \|,
\\[1em] & \langle V(u), \nabla J(u) \rangle \geq \| \nabla J(u) \|^2.
\end{aligned} \end{equation}
\ed

\brmk
If a $\Psi$-gradient vector field $V$ exists, then \hyperref[lemma.f.4]{Lemma \ref{lemma.f.4}} holds replacing $\alpha$ with the solution of
\[
\alpha'(t,u) = - V(\alpha(t, u)).
\]
\ermk

\bpr
Let $J \in C^1(\X, \R)$. Then a $\Psi$-gradient vector field $V$ always exists.
\epr

\begin{proof}
Fix $u \in \X_0$. Then there exists $w(u) := w \in \X$ such that
\[
\|w\|_\X = 1 \quad \text{and} \quad \langle \nabla J(u), w \rangle > \frac{2}{3} \| \nabla J(u) \|.
\]
Set
\[
\tilde{V}(u) := \frac{3}{2} \| \nabla J(u) \| w(u),
\]
and notice that \eqref{eq.f.1} holds since
\[ \begin{aligned}
& \| \tilde{V}(u) \| = \frac{3}{2} \| \nabla J(u) \| < 2 \| \nabla J(u) \|,
\\ & \langle \tilde{V}(u), \nabla J(u) \rangle = \frac{3}{2} \| \nabla J(u) \| \langle w(u), \nabla J(u) \rangle > \| \nabla J(u) \|^2.
\end{aligned}\]
Since $\nabla J$ is continuous, we can find $r := r(u) > 0$ such that
\[ \begin{aligned}
& \| \tilde{V}(u) \|  < 2 \| \nabla J(z) \|,
\\ & \langle \tilde{V}(u), \nabla J(z) \rangle > \| \nabla J(z) \|^2
\end{aligned} \]
hold for all $z \in B(u, r)$. We can cover $\X_0$ with balls, that is,
\[
\X_0 = \bigcup_{u\in\X_0} B(u, r(u))
\]
and we extract a locally finite covering $U_i := B(u_i, r(u_i))$. Now define
\[
d_i(u) := \mathrm{dist}(u, \X \setminus U_i) 
\]
and, to ease the notation, denote $\tilde{V}(u_i)$ by $\widetilde{V_i}$. The reader can check that
\[
V(u) := \sum_i \frac{ d_i(u) }{ \sum_j d_j(u)} \widetilde{V_i}
\]
is a well-defined locally Lipschitz $\Psi$-gradient vector field.
\end{proof}

\section{Deformation and compactness}

In this section, we will denote by $M$ either a Hilbert space or a $C^1$-submanifold of codimension one.

\begin{lemma} \label{lemma.f.2} Let $J \in C^1(M, \, \R)$ and suppose that there exist $c \in \R$ and $\delta > 0$ such that
\begin{equation*} \| \nabla_M J(u) \| \geq \delta \quad \text{for all $u$ such that $J(u) \in [c-\delta, \, c+\delta]$}. \end{equation*}
Then there exists $\eta$ deformation in $M$ such that
\begin{equation*} \eta(M^{c+\delta}) \subset M^{c-\delta}. \end{equation*} \end{lemma}

\begin{proof} Suppose first that $J$ is bounded from below. By \hyperref[lemma.f.4]{Lemma \ref{lemma.f.4}} the evolution above is globally defined. Let $T := \frac{2}{\delta}$ and set
\begin{equation*} \eta(u) := \alpha(T, \, u).\end{equation*}
It is easy to see that $\eta$ is a deformation since $(s, \, u) \mapsto \alpha(sT, \, u)$ is a homotopy between $\eta$ and the identity mapping. We now argue by contradiction so let $u \in M^{c + \delta}$ such that
\begin{equation*} J(\alpha(T, \, u)) > c - \delta. \end{equation*}
Since $J(\alpha(\cdot, \, u))$ is decreasing, we easily infer that
\begin{equation*} J(\alpha(t, \, u)) \in [c- \delta, \, c+ \delta] \end{equation*}
for all $t \in [0, \, T]$. We now apply the assumption to conclude that
\begin{equation*}  \| \nabla_M J( \alpha(t, \, u) ) \| \geq \delta \quad \text{for all $t \in [0, \, T]$}. \end{equation*}
We now use \hyperref[lemma.f.4]{Lemma \ref{lemma.f.4}} again and obtain
\begin{equation*}J(\alpha(T, \, u)) - \underbracket{J(\alpha(0, \, u))}_{= J(u)} = - \int_0^T \| \nabla_M J(\alpha(s, \, u)) \| \, \mathrm{d}x \geq \delta^2 T = 2 \delta, \end{equation*} 
from which we finally infer that
\begin{equation*} c - \delta < J(\alpha(T, \, u)) <  c + \delta - 2\delta = c - \delta \implies \text{absurd.} \end{equation*}
Now remove the assumption that $J$ is bounded from below. Define
\begin{equation*} \hat{J}(u) := h \circ J(u), \end{equation*}
where $h\in C^\infty(\R, \, \R)$ is given, for example, by
\begin{equation*} h(s) = \begin{cases}s & \text{if $s \geq c - \delta$},
\\[0.6em] \text{bounded below} & \text{at all $s \in \R$}. \end{cases} \end{equation*}
We conclude the proof using the argument above since $\hat{J}$ is bounded from below by construction and also
\begin{equation*} \{ \hat{J} \leq a \} = \{ J \leq a \} \end{equation*}
for all $a \geq c - \delta$ by construction.\end{proof}

\begin{remark}If $M$ is compact and $c$ is not a critical level for $J$, then we can always find a $\delta > 0$ satisfying the assumption of \hyperref[lemma.f.2]{Lemma \ref{lemma.f.2}}. \end{remark}

\begin{proof}[Hint] Argue by contradiction. \end{proof}

\begin{remark}Some kind of compactness is necessary even in finite-dimensional spaces. We can easily find a counterexample with $M = \R$; see \hyperref[fig.f.1]{Figure \ref{fig.f.1}}. \end{remark}

%%IMMAGINE!

\section{Palais-Smale sequences}

In this section, we introduce a notion of compactness which is weaker than the usual one but is rather useful when dealing with variational problems.

\begin{definition}\index{Palais-Smale!sequence} Let $c \in \R$ be a real number. We say that a sequence $(u_n)_{n \in \N} \subset M$ is Palais-Smale at the level $c$, denoted by $(u_n)_{n \in \N} \in (\mathrm{PS})_c$, if
 \begin{equation*} \begin{cases} f(u_n) \xrightarrow{n \to + \infty} c, \\[1em] \mathrm{grad} \, f(u_n) \xrightarrow{n \to + \infty} 0. \end{cases} \end{equation*}.\end{definition}

\begin{definition}\index{Palais-Smale!functional} A functional $J \in C^1(M, \, \R)$ is Palais-Smale at the level $c$ if
\begin{equation*} \forall (u_n)_{n \in \N} \in (\mathrm{PS})_c, \, \exists (n_k)_{k \in \N} \: : \: \text{$u_{n_k}$ converges}. \end{equation*}\end{definition}

\begin{remark} Let $J \in C^1(M, \,\R)$. \mbox{}
\begin{enumerate}[label=\textbf{(\roman*)}]
\item If $J$ satisfies the Palais-Smale condition at the level $c$, then any $(\mathrm{PS})_c$-sequence converges (up to subsequences) to some $u^\ast \in M$ such that
\begin{equation*} J(u^\ast) = c \quad \text{and} \quad \nabla_M J(u^\ast) = 0, \end{equation*}
which means that $u^\ast$ is a critical point (and thus $c$ a critical level).
\item The set
\begin{equation*} \{ z \in M \: : \: J(z) = c, \, \nabla J(z) = 0\} \end{equation*}
is compact.
\item If $J \in C^1(\R^n, \, \R)$ is bounded from below and coercive, then the Palais-Smale condition at the level $c$ holds for all $c$. This is false in the infinite-dimensional setting!
\end{enumerate} \end{remark}

\begin{lemma}\label{rmk:cris} Let $J \in C^1(M, \, \R)$ be a functional satisfying the $\left( \mathrm{PS} \right)_c$-condition at all $c \in [a, \, b]$ and assume that there are no critical levels in the interval. Then there exists $\delta > 0$ such that
\begin{equation*}\sigma := \inf_{ u \in J^{-1} \left( I_\epsilon \right) } \left\| \nabla J(u) \right\| > 0, \end{equation*}
where $I_\delta = \left[ a - \delta, \, b + \delta \right]$. \end{lemma}

\begin{proof}We argue by contradiction. There is a decreasing sequence $(\delta_n)_{n \in \N}$ that converges to $0$, and a sequence $(u_n)_{n \in \N} \subset M$ such that
\begin{equation*} \left\| \nabla J(u_n) \right\| \leq \delta_n \quad \text{and} \quad J(u_n) \in \left[ a- \delta_n, \, b + \delta_n \right]. \end{equation*}
Now, up to subsequences, $J(u_n) \to c$ and $ \nabla J(u) \to 0$ and by the Palais-Smale condition we know that $(u_n)_{n \in \N}$ is precompact. Thus $u_{n_k}$ converges to some $u^\ast$ which is a critical point with $J(u) \in [a, \, b]$, and this is the sought contradiction.
\end{proof}

\begin{lemma}Let $J \in C^1(M, \, \R)$ be a functional satisfying the $\left( \mathrm{PS} \right)_c$-condition at some noncritical level $c \in \R$. Then there exist $\delta > 0$ and a deformation $\eta$ such that
\begin{equation*} \eta(M^{c+\delta}) \subseteq M^{c-\delta}. \end{equation*} \end{lemma}

\begin{lemma}Let $J \in C^1(M, \, \R)$ be a functional satisfying the $\left( \mathrm{PS} \right)_c$-condition at all $c \in [a, \, b]$ and assume that there are no critical levels in the interval. Then there exists a deformation $\eta$ such that
\begin{equation*} \eta(M^{b}) \subseteq M^{a}. \end{equation*} \end{lemma}

\begin{proof}Simply apply the previous result a finite number of times since $[a, \, b]$, by compactness, can be covered by a finite number of intervals of length $\delta$. \end{proof}

\begin{theorem}\label{thm.o.1}Let $J \in C^1(M, \, \R)$, where $M$ is a $C^1$-submanifold of codimension one. Suppose that $J \, \big|_M$ is bounded from below and suppose that it satisfies the Palais-Smale condition at
\begin{equation*} m := \inf_{u \in M} J(u) > - \infty. \end{equation*}
Then $\inf_{u \in M} J(u)$ is achieved.\end{theorem}

\begin{proof}We argue by contradiction. If $m$ is not a critical value, then there exists $\epsilon > 0$ such that the following holds:
\begin{equation*} \text{$J^{(\alpha - \epsilon)}$ is a deformation retract of $J^{(\alpha + \epsilon)}$}. \end{equation*}
But this is impossible since the first set is empty, while the second one is not. \end{proof}

\section{Application to a superlinear Dirichlet problem}

In this section, we will exploit the theoretical results presented above to prove existence of a positive solution to a class of superlinear Dirichlet boundary-value problems:
\begin{equation} \mathbf{\tag{DSL}} \label{eq.m.1} \begin{cases} - \Delta u(x) = f(u(x)) & \text{if $x\in \Omega$}, \\[0.8em] u(x) = 0 & \text{if $x \in \partial \Omega$}. \end{cases} \end{equation}
We assume $\Omega$ to be a bounded domain in $\R^n$ and $f \in C^2(\R, \, \R)$ satisfies the following assumptions: there exist $a_1, \, a_2 > 0$ and $p \in (1, \, 2^\ast - 1)$ such that
\begin{equation} \label{eq.m.f} \begin{aligned} & |f(u)| \leq a_1 + a_2 |u|^p,
\\[0.8em] & |u f^\prime(u)| \leq a_1 + a_2|u|^p,
\\[0.8em] & |u^2 f^{\prime \prime}(u)| \leq a_1 + a_2|u|^p. \end{aligned} \end{equation}
Assume also that $f(u) = u h(u)$, where $h$ is a function satisfying the following assumptions:
\begin{enumerate}[label=\textbf{\color{orange}($h_{\arabic*}$)}]
\item $h(su) \leq s^\alpha h(u)$ for some $\alpha > 0$;
\item $u h^\prime(u) > 0$ for all $u \neq 0$;
\item $h(0) = 0$;
\item $\lim_{u \to + \infty} h(u) = + \infty$.
\end{enumerate}

\begin{example}The function $h(u) = |u|^{p-1}$ satisfies these properties and, indeed, we were able to obtain existence in \hyperref[subsec:daosd]{Section \ref{subsec:daosd}} looking for the minimum of
\begin{equation*} \int_\Omega |u|^{p+1} \, \mathrm{d}x \end{equation*}
on the manifold $ \{u \in H_0^1(\Omega) \: : \: \|u\|_{L^2(\Omega)} = 1 \}$.
\end{example}

\begin{theorem}\label{thm.m.1} Under these assumptions, the problem \eqref{eq.m.1} has a positive solution. \end{theorem}

The proof of this theorem will be attained through a sequence of technical lemmas, mostly relying on the theoretical aspects presented in this chapter. However, before we get to it, we need to introduce some notation. Namely, let $\X := H_0^1(\Omega)$ and denote by
\begin{equation*} \langle u, \, v \rangle := \int_\Omega \nabla u \cdot \nabla v \, \mathrm{d}x \end{equation*}
the standard scalar product and by $\| \cdot \|$ the norm on $\X$. Set
\begin{equation*} \begin{aligned} & F(u) := \int_0^u f(s) \, \mathrm{d}s = \int_0^1 f(su)u \, \mathrm{d}s,
\\[0.8em] & \Phi(u) = \int_\Omega F(u) \, \mathrm{d}x = \int_0^1 \mathrm{d}s \int_\Omega u f(su) \, \mathrm{d}x,
\\[0.8em] & \Psi(u) = \langle \Phi^\prime(u), \, u \rangle = \int_\Omega u f(u) \, \mathrm{d}x. \end{aligned} \end{equation*}
Now notice that
\begin{enumerate}[label=\textbf{\color{magenta}(\roman*)}]
\item The functional $\Phi$ and $\Psi$ respectively belong to $C^2(\X, \, \R)$ and $C^3(\X, \, \R)$. {\color{blue}This follows immediately from the regularity of $f$ and the definitions above.}
\item The functionals $\Phi$ and $\Psi$ are both weakly continuous.
\item The gradients $\nabla \Phi$ and $\nabla \Psi$ are compact operators. {\color{blue}This follows from the compactness of the Sobolev embedding (since $p < 2^\ast$) and it implies the previous point.}
\end{enumerate}

The solutions of \eqref{eq.m.1} are critical points of the following functional: 
\begin{equation*} J(u) := \frac{1}{2} \|u\|^2 - \Phi(u). \end{equation*}
The idea is to use Nehari manifolds together with the results on critical points obtained in this chapter. We thus introduce the natural functional
\begin{equation*} G(u) := \langle J^\prime(u), \, u \rangle = \|u\|^2 - \Psi(u), \end{equation*}
and the $C^2$-submanifold where $G$ vanishes, that is,
\begin{equation*} \mathcal{M} := \left\{ u \in \X \setminus \{0\} \: : \: G(u) = 0 \right\}. \end{equation*}
Our goal is to show that $\mathcal{M}$ is a natural constraint for $J$. In other words, we are looking for a functional of class $C^2$, $\widetilde{J}$, such that
\begin{equation*}\text{$\nabla_\mathcal{M} \widetilde{J}(u) = 0$ and $u \in M$} \iff J^\prime(u) = 0. \end{equation*}
We will then verify that $\widetilde{J}$ achieves a minimum on $\mathcal{M}$, which ends up giving a solution to the problem \eqref{eq.m.1}.

\begin{lemma} The functional $G$ belongs to $C^2(E, \, \R)$. Furthermore: \mbox{}
\begin{enumerate}[label=\textbf{(\roman*)}]
\item The set $\mathcal{M}$ is nonempty.
\item There exists $\rho > 0$ such that $\|u\| \geq \rho$ for all $u \in \mathcal{M}$.
\item The scalar product $\langle G^\prime(u), \, u \rangle$ is negative for all $u \in \mathcal{M}$.
\end{enumerate} \end{lemma}

\begin{proof}The regularity of $G$ is an easy consequence of the regularity of $\Psi$. \mbox{}
\begin{enumerate}[label=\textbf{(\roman*)}]
\item Take $u \in E$, $u > 0$, with $\|u \|=1$. Then
\begin{equation*}G(tu) = t^2 - t^2 \int_\Omega u^2 h(tu) \, \mathrm{d}x. \end{equation*}
Using $\mathbf{\color{orange}(h_3)}$ we find that
\begin{equation*} \lim_{t \to 0} \frac{G(tu)}{t^2} = 1, \end{equation*}
while, employing the property $\mathbf{\color{orange}(h_4)}$, we obtain
\begin{equation*}  \lim_{t \to + \infty} \frac{G(tu)}{t^2} = - \infty. \end{equation*} 
Putting these two together, we infer that there must be $\tilde{t} \in (0, \, \infty)$ such that $\tilde{t}u \in \mathcal{M}$.
\item This property, despite its simplicity, requires a lot of work because having $\|u\|$ small does not mean that the $L^\infty$-norm is also small (the embedding fails!).

Let $\|u\|$ be sufficiently small. Our goal is to prove that $G(u) > 0$ so that $u$ cannot belong to $\mathcal{M}$. First, take $\delta > 0$ and define
\begin{equation*} A_1^\delta := \{ x \in \Omega \: : \: |u(x)| \leq \delta \} \quad \text{and} \quad A_2^\delta = \Omega \setminus A_1^\delta. \end{equation*}
We claim that the volume of $A_2^\delta$ cannot be "too big". Recall that by Poincaré's inequality we can always find a positive constant $C_\Omega$ such that
\begin{equation*} \| u \|_{L^1(\Omega)} \leq C_\Omega \|u\|. \end{equation*}
It follows that
\begin{equation*} |A_2| \delta \leq \| u \|_{L^1(A_2)} \leq \| u \|_{L^1(\Omega)} \leq C_\Omega \|u\|, \end{equation*}
which means that the volume of $A_2$ is bounded by
\begin{equation*} |A_2| \leq \frac{C_\Omega}{\delta} \|u\|. \end{equation*}
We now employ Hölder's inequality to estimate the negative contribute to $G(u)$ on $A_2^\delta$. Namely, we have
\begin{equation*} \int_{A_2^\delta} u f(u) \, \mathrm{d}x \leq \|u\|_{p_1} \|f(u)\|_{p_2} |A_2^\delta|^{\frac{1}{p_3}} \end{equation*}
where
\[\frac{1}{p_1} + \frac{1}{p_2} + \frac{1}{p_3} = 1. \]
We want $p_3 > 1$, so the idea is to take the maximum $p_1$ and $p_2$ possible. However, we still need Sobolev embedding to estimate these terms with $\|u\|$. Let
\begin{equation*} p_1 := 2^\ast \quad \text{and} \quad p_2 := \frac{2^\ast}{p}. \end{equation*}
It is easy to see that
\begin{equation*}\frac{1}{p_1} + \frac{1}{p_2} = \frac{p + 1}{2^\ast} < 1, \end{equation*}
so $p_3 > 1$ as desired. We also use \eqref{eq.m.f} and $\mathbf{\color{orange}(h_4)}$ to conclude that $f(u)$ must satisfy a slightly different estimate
\begin{equation*} |f(u)| \lesssim |u| + |u|^p \end{equation*}
for $|u|$ small enough. Then
\begin{equation*}\begin{aligned} \|f(u)\|_{p_2} & \lesssim \left[ \int_\Omega \left( |u|^{p_2} + |u|^{p p_2} \right) \, \mathrm{d}x \right]^{\frac{1}{p_2}} \lesssim
\\[1em] & \lesssim \|u\| + \|u\|^p. \end{aligned} \end{equation*}
The right-hand side goes as $\|u\|$ when $\|u\|$ is sufficiently small (since $p > 1$) and therefore we conclude that
\begin{equation*} \left| \int_{A_2^\delta} u f(u) \, \mathrm{d}x \right| \lesssim \delta^{-\frac{1}{p_3}} \|u\|^{2 + \frac{1}{p_3}}. \end{equation*}
The estimate on $A_1^\delta$ is even easier since
\begin{equation*} \left| \int_{A_1^\delta} u f(u) \, \mathrm{d}x \right| = \left| \int_{A_1^\delta} u^2 h(u) \, \mathrm{d}x \right| \leq C_\Omega \|u\|^2 \sup_{|u| \in (0, \, \delta)} h(u).\end{equation*}
Fix $\delta > 0$ sufficiently small in such a way that $C_\Omega  \sup_{|u| \in (0, \, \delta)} h(u)$ is less than $\frac{1}{2}$. It follows that
\begin{equation*}G(u) \geq \|u\|^2 - \frac{1}{2} \|u\|^2 - \delta^{-\frac{1}{p_3}} \|u\|^{2 + \frac{1}{p_3}}, \end{equation*}
and the right-hand side is positive when we take the limit as $\|u\| \to 0$ since $2 + \frac{1}{p_3} > 2$. In particular, there exists $\rho>0$ such that for all $u \in B_\rho(0) \setminus \{0\}$ we have $G(u) > 0$.
\item First, notice that for $u \in \mathcal{M}$ we have
\begin{equation*}\begin{aligned} \langle G^\prime(u), \, u \rangle & = 2 \|u\|^2 - \langle \Psi^\prime(u), \, u \rangle =
\\[1em] & = 2 \Psi(u) - \langle \Psi^\prime(u), \, u \rangle. \end{aligned} \end{equation*}
One also has that
\begin{equation*} \begin{aligned} 2 \Psi(u) - \langle \Psi^\prime(u), \, u \rangle & = 2 \int_\Omega u f(u) \, \mathrm{d}x - \left[ \int_\Omega u f(u) \, \mathrm{d}x + \int_\Omega u^2 f^\prime(u) \, \mathrm{d}x \right] =
\\[1em] & = \int_\Omega u^2 h(u)\, \mathrm{d}x - \int_\Omega u^2(h(u) + u h^\prime(u)) \, \mathrm{d}x =
\\[1em] & = - \int_\Omega u^3 h^\prime(u) \, \mathrm{d}x. \end{aligned} \end{equation*}
Since $0 \notin \mathcal{M}$, using $\mathbf{\color{orange}(h_2)}$ that holds for all $u \neq 0$ we conclude that the scalar product must be negative.
\end{enumerate} \end{proof}

It follows from $\mathbf{(iii)}$ that $\mathcal{M}$ is a submanifold of class $C^2$ of codimension one in $E$. Now let $\widetilde{J} \in C^2(E, \, \R)$ be defined as
\begin{equation*} \widetilde{J}(u) = \frac{1}{2} \Psi(u) - \Phi(u). \end{equation*}
Notice that this functional coincides with $J$ on $\mathcal{M}$, but it is more convenient to deal with it since it is weakly continuous and its derivative is compact.

\begin{lemma}The submanifold $\mathcal{M}$ is a natural constraint for $J$ using $\widetilde{J}$, that is,
\begin{equation*} z \in \M, \, \, \nabla_\mathcal{M} \widetilde{J}(z) = 0 \implies J^\prime(z) = 0. \end{equation*}\end{lemma}

\begin{proof}If $z$ is such a point, then there exists $\lambda \in \R$ such that
\begin{equation*} \nabla \widetilde{J}(z) = \lambda \nabla G(z) \implies \langle\widetilde{J}(z), \, z \rangle = \lambda \langle \nabla G(z), \, z \rangle. \end{equation*}
On the other hand, we know that
\begin{equation*}\begin{aligned} \langle\widetilde{J}(z), \, z \rangle & = \frac{1}{2} \langle \Psi(z), \, z) - \langle \Phi(z), \, z \rangle =
\\[1em] & = \frac{1}{2} \langle \nabla \Psi(z), \, z \rangle - \Psi(z) =
\\[1em] & = - \frac{1}{2} \langle \nabla G(z), \, z \rangle\end{aligned} \end{equation*}
so $\lambda$ must be equal to $- \frac{1}{2}$. Then
\begin{equation*} \begin{aligned}
& \nabla G(z) = - \nabla \Psi(z) + 2z,
\\[0.8em] & \nabla \widetilde{J}(z) = \frac{1}{2} \Psi(z) - \nabla \Phi(z),
\end{aligned} \end{equation*}
and this immediately implies that $\nabla \Phi(z) = z$, which is completely equivalent to
\[\nabla J(z) = 0.\]\end{proof}

\begin{lemma}There exists $C_\alpha > 0$ such that $\widetilde{J}(u) \geq C_\alpha \|u\|^2$ for all $u \in \mathcal{M}$. \end{lemma}

\begin{proof} We use the definition of $\Psi$ and $\mathbf{\color{orange}(h_1)}$ to infer that
\begin{equation*} \begin{aligned}
\widetilde{J}(u) = \frac{1}{2} \int_\Omega u f(u) \, \mathrm{d}x - \int_0^1 \mathrm{d}s \int_\Omega u f(su) \, \mathrm{d}x & = \int_0^1 \mathrm{d}s \int_\Omega \left[su f(u) - u f(su) \right] \, \mathrm{d}x =
\\[1em] & = \int_0^1 \mathrm{d}s \int_\Omega \left[ su^2 \left( h(u) - h(su) \right) \right] \, \mathrm{d}x \geq
\\[1em] & \geq \int_0^1 s(1 - s^\alpha) \, \mathrm{d}s \int_\Omega u^2 h(u) \, \mathrm{d}x \geq
\\[1em] & \geq C_\alpha \underbracket{\int_\Omega u^2 h(u) \, \mathrm{d}x}_{= \Psi(u)} = C_\alpha \|u\|^2, \end{aligned} \end{equation*}
where the last equality follows from the fact that $u \in \mathcal{M}$ implies $\Psi(u) = \|u\|^2$.\end{proof}

\begin{lemma} Let $(u_i)_{i \in \N}$ be a Palais-Smale sequence at level $c > 0$ for $\widetilde{J}$ on $\mathcal{M}$. Then \mbox{}
\begin{enumerate}[label=\textbf{(\roman*)}]
\item $\|u_i\|$ is bounded and there exists $\bar{u} \neq 0$ such that $u_{i_\ell} \rightharpoonup \bar{u}$;
\item there exists $k > 0$ such that $\|\nabla \widetilde{J}(u_i) \| \geq k$.
\end{enumerate} \end{lemma}

\begin{proof} \mbox{}
\begin{enumerate}[label=\textbf{(\roman*)}]
\item By definition
\begin{equation*} \widetilde{J}(u_i) \xrightarrow{i \to + \infty} c, \end{equation*}
and using the previous result we also know that
\begin{equation*} \widetilde{J}(u_i) \geq c_\alpha \|u_i\|^2 \end{equation*}
so $\|u_i\|$ is bounded and $(u_i)_{i \in \N}$ converges weakly to some $\bar{u}$ up to subsequences. To prove that $\bar{u} \neq 0$, we notice that
\begin{equation*} u_i \in \mathcal{M} \implies \|u_i\| \geq \rho \implies \Psi(u_i) = \|u_i\|^2 \geq \rho^2. \end{equation*}
But $\Psi$ is weakly continuous so
\begin{equation*} \Psi(\bar{u}) \geq \rho^2 \implies \bar{u} \neq 0. \end{equation*}
\item We argue by contradiction. Suppose that $\nabla \widetilde{J}(u_i) \to 0$. The operator $\nabla \widetilde{J}$ is compact, so we can conclude that
\begin{equation*} \nabla \widetilde{J}(\bar{u}) = 0 \implies 0 = \frac{1}{2} \langle \nabla \Psi(\bar{u}), \, \bar{u} \rangle - \Psi(\bar{u}) = \frac{1}{2} \int_\Omega \bar{u}^3 h^\prime(\bar{u}) \, \mathrm{d}x. \end{equation*}
We know already that the right-hand side is strictly positive, so we obtained our contradiction.
\end{enumerate} \end{proof}

\begin{lemma} The function $\widetilde{J}$, restricted to $\mathcal{M}$, satisfies the Palais-Smale condition at all levels $c > 0$.\end{lemma}

\begin{proof}Let $(u_i)_{i \in \N}$ be a Palais-Smale sequence at level $c$ and let $\bar{u}$ be the weak limit of a subsequence $(u_{i_k})_{k \in \N}$. We have
\begin{equation*} \nabla_\mathcal{M} \widetilde{J}(u_i) = \nabla \widetilde{J}(u_i) - \alpha_i \nabla G(u_i), \end{equation*}
where
\begin{equation*} \alpha_i = \frac{ \langle \nabla \widetilde{J}(u_i), \, \nabla G(u_i) \rangle}{\| \nabla G(u_i)\|^2}.  \end{equation*}
We proved already that $\| \nabla \widetilde{J}(u_i) \| \geq k$ and it is easy to see that $\| \nabla G(u_i) \| \leq c$, so taking into account that
\begin{equation*} \underbracket{\nabla_\mathcal{M} \widetilde{J}(u_i)}_{\to 0} = \underbracket{\nabla \widetilde{J}(u_i)}_{\not \to 0} - \alpha_i \underbracket{\nabla G(u_i)}_{\mathrm{bounded}}, \end{equation*}
we must have $|\alpha_i| \geq c > 0$. It follows that
\begin{equation*} \nabla G(u_i) = \frac{1}{\alpha_i} \left[ \nabla \widetilde{J}(u_i)-\nabla_\mathcal{M} \widetilde{J}(u_i) \right], \end{equation*}
which easily translates to
\begin{equation*} 2 u_i = \underbracket{\nabla \Psi(u_i)}_{\mathrm{compact}} + \frac{1}{\alpha_i} \left[ \underbracket{\nabla \widetilde{J}(u_i)}_{\mathrm{compact}}- \underbracket{\nabla_\mathcal{M} \widetilde{J}(u_i)}_{\to 0} \right] \end{equation*}
and thus $u_{i_k}$ converges strongly to $\bar{u}$, concluding the proof. \end{proof}

\begin{proof}[Proof of Theorem \ref{thm.m.1}] Simply apply \hyperref[thm.o.1]{Theorem \ref{thm.o.1}} replacing $f$ with its positive part $f^+$. \end{proof}
 \chapter{Min-max Methods} \thispagestyle{empty}

In this chapter, we will discuss the existence of stationary point of a function $J$, defined on a Hilbert space $\X$, which can be found via different min-max procedures.

\section{The mountain pass theorem}

We proved that \eqref{eq.m.1} admits a positive solution, provided that $f$ satisfies certain assumptions including a growth condition
\[ h(su) \leq s^\alpha h(u), \]
that holds at all point $u \in \X$. A natural question is whether or not we can prove a similar result when the behaviour of $f$ is only known at the origin and at infinity. To deal with this problem, we consider the corresponding functional
\[ J(u) = \frac{1}{2} \|u\|^2 - \int_\Omega F(u) \, \mathrm{d}x, \]
with $\| \cdot \| = \| \cdot \|_\X$ and $\X = H_0^1(\Omega)$. It is easy to verify that $u = 0$ is a proper local minimum for $J$ since, assuming that $f^\prime(0) = 0$, we have
\[ f^\prime(0) = 0 \implies \langle J^{\prime \prime}(0) v, \, v)_\X = \|v\|^2. \]
On the other hand, if we assume that $F(u) \sim |u|^{p+1}$, $1 < p < \frac{n+2}{n-2}$, then for any $u \in \X$ that is different from zero we find that
\[ \lim_{t \to + \infty} J(tu) = \lim_{t \to + \infty} \left[ \frac{t^2}{2} \|u\|^2 - \int_\Omega F(tu) \, \mathrm{d}x \right] = - \infty. \]
In particular, the functional $J$ is not bounded from below on $\X$. We also notice that
\[ \sup_\X J = + \infty \]
since we can always consider a sequence of function $\|u_i\| \to + \infty$ with $\int_\Omega F(u_i) \, \mathrm{d}x$ uniformly bounded. Now to fix the ideas, consider the model nonlinearity so that
\[ J(tu) = \frac{1}{2} t^2 \|u\|^2 - \frac{1}{p+1} |t|^{p+1} \int_\Omega |u|^{p+1} \, \mathrm{d}x. \]
The real valued map $t \mapsto J(tu)$ achieves its maximum at a unique point $t =t_u > 0$ and, as expected, it is determined by the fact that
\[ tu \in \mathcal{M} := \{ u \in \X \setminus \{0\} \: : \: \langle J^\prime(u), \, u \rangle = 0 \}, \]
where $\mathcal{M}$ is the natural constraint introduced many times before. If $z$ is a critical point for $J$, we know that $J(z)$ is equal to the minimum value of $J$ achieved on $\mathcal{M}$, and thus
\[ J(z) = \min_{u \in \X \setminus \{0\} } \max_{t \in \R} J(tu). \]
The main goal of this section is to generalise this argument and to find optimal assumptions that allow one to find critical points of a functional $J$ via a max-min procedure.

In the sequel, to fix the notation, we will assume that $J$ has a local minimum at $u = 0$, but it is important to understand that this is a totally arbitrary choice. \mbox{}
\begin{enumerate}[label={\color{blue}$\mathbf{(\mathrm{MP}-\arabic*)}$},leftmargin=5\parindent]
\item The functional $J$ belongs to $C^1(\X,\, \R)$, $J(0) = 0$ and there are $r, \, \rho > 0$ such that $J(u) \geq \rho$ for all $u \in S_r$.
\item There exists $e\in \X$ with $\|e\| > r$ such that $J(e) \leq 0$.
\end{enumerate}
We will show that these assumptions on the geometry of $J$ are almost enough for the existence of a saddle point. Let
\[ \Gamma := \{ \gamma \in C([0, \, 1], \, \X) \: : \: \gamma(0) = 0, \, \gamma(1) = e \} \]
be the set of all continuous curves connecting $0$ and $e$ and notice that it is nonempty since
\[ t \longmapsto te \]
trivially belongs to $\Gamma$. We define the \textit{MP level} as
\begin{equation} \label{eq.n.1} c := \inf_{\gamma \in \Gamma} \max_{t \in [0, \, 1]} J(\gamma(t)). \end{equation}
If $J$ is a functional that has the \textit{MP geometry}, which means that it satisfies the two assumptions above, then it is easy to see that
\[ \gamma \in \Gamma \implies \gamma([0, \, 1]) \cap S_r \neq \varnothing \implies c \geq \min_{u \in S_r} J(u) \geq \rho > 0, \]
so if we were to find a critical point $u$ at the level $c$, we could immediately conclude that it is not trivial ($u \neq 0$). The following result is due to Ambrosetti and Rabinowitz in 1973.

\begin{theorem}[Mountain Pass] \label{thm.n.1}Let $J$ be a functional satisfying {\color{blue}$\mathbf{(\mathrm{MP}-1)}$} and {\color{blue}$\mathbf{(\mathrm{MP}-2)}$}. Suppose that the Palais-Smale condition at the level $c$ given by \eqref{eq.n.1} holds. Then
\[ \exists z \in \X \: : \: J(z) = c, \, \nabla J(z) = 0\]
and $z$ is nontrivial, that is, $z \neq 0$. \end{theorem}

To prove this result, we first need a technical lemma which gives us the existence of a particular deformation of the sublevels of $J$ that keeps a good portion of them fixed.

\begin{lemma} \label{lemma.n.1} Let $J \in C^1(\X, \, \R)$ and let $c \in \R$ be any noncritical value for $J$. Suppose that the Palais-Smale condition at the level $c$ holds for $J$. Then there are $\delta > 0$ with $c - 2 \delta > 0$ and $\eta$ deformation in $\X$ such that: \mbox{}
\begin{enumerate}[label=\textbf{(\alph*)}]
\item $\eta(J^{c+\delta}) \subseteq J^{c-\delta}$;
\item $\eta$ restricted to $J^{c-2\delta}$ coincides with the identity map.
\end{enumerate} \end{lemma}

\begin{proof}Recall that $J$ always admits a $\Psi$-gradient flow $V$ for $J$, that is defined at all points $u \in \X$ such that $\nabla J(u) \neq 0$, with the following properties: \mbox{}
\begin{enumerate}[label=\textbf{(\roman*)}]
\item $ \|V(u) \| \leq 2 \| \nabla J(u) \|$;
\item $ \langle V(u), \, \nabla J(u) \rangle_\X \geq \| \nabla J(u) \|^2$.
\end{enumerate}
Let $b \in C^{0, \, 1}(\R^+, \, \R^+)$ be the Lipschitz function defined by setting
\[ b(s) := \begin{cases} 1 & \text{if $s \in (0, \, 1]$},
\\[.8em] \frac{1}{s} & \text{if $s \geq 1$}. \end{cases}\]
Let $A := \{ u \in \X \: : \: J(u) \in [c-\delta, \, c + \delta]$, $B := \{u \in \X \: : \: J(u) \in (c-2 \delta, \, c + 2 \delta)^c \}$ and define the Lipschitz function from $\X$ to $\R$ given by
\[ g(u) := \frac{ d_\X(u, \, B) }{d_\X(u, \, A) + d_\X(u, \, B)} \in [0, \, 1].\]
Notice that $g$ is equal to zero if and only if $u \in B$ and equal to one if and only if $u \in A$. We can consider a slightly modified vector field as flow
\[ \widetilde{V}(u) := - g(u) b( \| \nabla J(u) \|) V(u). \]
There are several vantages in replacing $V$ with $\widetilde{V}(u)$. First, it is well-defined everywhere (even where the differential of $J$ vanishes), the boundedness of $b$ gives the global existence of $\eta$ and it is locally Lipschitz. Consider the solution of
\[ \begin{cases} \alpha^\prime(t) = \widetilde{V}(\alpha(t)), \\[0.6em] \alpha(0) = u, \end{cases} \]
and notice that the following properties are satisfied: \mbox{}
\begin{enumerate}[label=\textbf{(\roman*)}]
\item If $u \in B$, the $\widetilde{V}(u) = 0$ and thus $\alpha(t, \, u) = u$ at all times $t \in \R^+$.
\item The solution $\alpha$ is globally defined and $\| \widetilde{V}(u) \| \leq 2$.
\item The function $t \mapsto J(\alpha(t, \, u))$ is non-increasing since
\[ \begin{aligned} \frac{\mathrm{d}}{\mathrm{d}t} J(\alpha(t, \, u)) & = - g(\alpha) b( \| \nabla J(\alpha) \|) \langle \nabla J(\alpha), \, V(\alpha)\rangle_\X \leq
\\[1em] & \leq - g(\alpha) b(\| \nabla J(\alpha) \|) \| \nabla J(\alpha) \|^2 \leq 0. \end{aligned} \]
\end{enumerate}
Now let $\delta > 0$ be such that
\[ J(u) \in [c-\delta, \, c+ \delta] \implies \| \nabla J(u) \| \geq \delta, \]
and suppose that $c - 2 \delta > 0$. Let $T = \frac{2}{\delta}$ and define the deformation by setting
\[ \eta(u) := \alpha(T, \, u). \]
Then \textbf{(b)} trivially holds true, so we only need to check that $\eta$ satisfies \textbf{(a)}. For this, let $u \in J^{c+\delta}$ and suppose that $\eta(u) \notin J^{c-\delta}$. It follows that
\[ J(\alpha(t, \, u)) \in [c- \delta, \, c+\delta] \quad \text{for all $t \in [0, \, T]$}, \]
and hence $\alpha(t, \, u)$ belongs to $A$ for $t$ in the same interval. Using the definition of $g$ we infer that
\[ g(\alpha(t,\, u)) = 1 \quad \text{for all $t \in [0, \, T]$}, \]
so
\[ \begin{aligned} J(\eta(u)) - J(u) & = - \int_0^T b( \| \nabla J(\alpha(t, \, u)) \|) \langle \nabla J(\alpha(t, \, u)), \, V(\alpha(t,\, u)) \rangle_\X \, \mathrm{d}t \leq
\\[1em] & \leq - \int_0^T b( \| \nabla J(\alpha(t, \, u)) \|) \|\nabla J(\alpha(t, \, u)) \|^2 \, \mathrm{d}t \leq
\\[1em] & \leq - \int_0^T \delta^2 \, \mathrm{d}t = - 2 \delta, \end{aligned} \]
and this gives a contradiction since
\[ J(\eta(u)) \leq J(u) - 2\delta \leq c + \delta - 2\delta = c -\delta \implies \eta(u) \in J^{c-\delta}.\]\end{proof}

\begin{proof}[Proof of Theorem \ref{thm.n.1}] We argue by contradiction. Suppose that the MP level $c$ is not critical and let $\eta$ be the deformation given by \hyperref[lemma.n.1]{Lemma \ref{lemma.n.1}}. Now notice that
\[ 0, \, e \in J^0 \implies 0, \, e \in J^{c - 2 \delta} \implies \left( \gamma \in \Gamma \implies \eta \circ \gamma \in \Gamma \right), \]
so $\eta$ associates a curve in $\Gamma$ to any curve in $\Gamma$. Recall that
\[ c = \inf_{\gamma \in \Gamma} \max_{t \in [0, \, 1]} J(\gamma(t)), \]
so for any $\delta > 0$ we can find $\gamma \in \Gamma$ such that
\[ \max_{t \in [0, \, 1]} J(\gamma(t)) \leq c + \delta. \]
The deformation $\eta$ maps $\gamma([0, \, 1])$ into $J^{c-\delta}$ so
\[ \max_{t \in [0, \, 1]} J(\eta \circ \gamma(t)) \leq c - \delta, \]
and this is a contradiction since $c$ is the infimum value and yet $\eta \circ \gamma \in \Gamma$.
\end{proof}

\begin{remark}We cannot remove the assumption that $J$ satisfies the Palais-Smale condition at the MP level $c$. Indeed, it is easy to find a counterexample in $\R^2$ for which $J$ has the MP geometry but there are not critical points except for $(0, \, 0)$. Namely, let
\[ J(x, \, y) = x^2 + (1-x)^3 y^2, \]
and notice that {\color{blue}$\mathbf{(\mathrm{MP}-1)}$} is satisfied with $r = \frac{1}{2}$ and $\rho = \frac{1}{32}$, while {\color{blue}$\mathbf{(\mathrm{MP}-2)}$} is satisfied with $e = (2, \, 2)$.\end{remark}

\section{Application to the Dirichlet problem}

In this section, we will exploit the theoretical results presented above to prove existence of a positive solution to a class of Dirichlet boundary-value problems:
\begin{equation} \mathbf{\tag{D}} \label{eq.n.2} \begin{cases} - \Delta u(x) = f(u(x)) & \text{if $x\in \Omega$}, \\[0.8em] u(x) = 0 & \text{if $x \in \partial \Omega$}. \end{cases} \end{equation}
We assume $\Omega$ to be a smooth bounded domain in $\R^n$ and $f$ a function satisfying the following assumptions:
\begin{enumerate}[label=\textbf{\color{orange}($f_{\arabic*}$)},leftmargin=3\parindent]
\item $f$ is Carathéodory;
\item $|f(x, \, u)| \leq a + b |u|^p$ for some $1 < p < \frac{n+2}{n-2}$;
\item $ \lim_{|u|\to0^+} \frac{f(x, \, u)}{|u|} = \lambda \in \R$ uniformly with respect to $x \in \Omega$;
\item there exists $r > 0$ and $\theta \in (0, \, \frac{1}{2})$ such that
\begin{equation} \label{eq.n.3} 0 < F(x, \, u) \leq \theta u f(x, \, u) \end{equation}
for all $u$ with norm $\|u\| \geq R$.
\end{enumerate}

\begin{lemma} If $f$ satisfies the property \textbf{\color{orange}($f_{4}$)}, then
\begin{equation} \label{eq.n.4} F(u) \geq \frac{1}{c} u^{\frac{1}{\theta}} - c \quad \text{for all $u \geq R$}. \end{equation}  \end{lemma}

\begin{lemma} If $\lambda < \lambda_1(\Omega)$, then {\color{blue}$\mathbf{(\mathrm{MP}-1)}$} holds.\end{lemma}

\begin{proof}Fix $\epsilon := \frac{1}{2}(\lambda_1 - \lambda) > 0$. The assumptions on $f$ allows us to find a constant $A \in \R$ such that
\[ F(x, \, u) \leq \frac{1}{2}(\lambda + \epsilon) u^2 + A |u|^{p+1}.\]
Now integrate and use Sobolev embedding to infer that
\[ \begin{aligned} \left| \int_\Omega F(x, \, u) \, \mathrm{d}x \right| & \leq \frac{1}{2}(\lambda + \epsilon) \|u\|_{L^2(\Omega)}^2 + A \|u\|_{L^{p+1}(\Omega)}^{p+1} \leq
\\[1em] & \leq  \frac{1}{2}(\lambda + \epsilon) \|u\|_{L^2(\Omega)}^2 + A^\prime \|u\|^{p+1}\end{aligned}\]
so that the functional can be estimated by
\[ J(u) \geq \frac{1}{2} \|u\|^2 - A^\prime \|u\|^{p+1} -  \frac{1}{2}(\lambda + \epsilon) \|u\|_{L^2(\Omega)}^2. \]
We now recall that
\[  \|u\|_{L^2(\Omega)}^2 \geq \frac{1}{\lambda_1} \|u\|^2 \]
so
\[ J(u) \geq \frac{1}{2}\left( \frac{\lambda_1-\lambda-\epsilon}{\lambda_1}\right) \|u\|^2 - A^\prime \|u\|^{p+1}, \]
and the first term is multiplies a positive constant.\end{proof}

\begin{lemma}Under no extra assumpions {\color{blue}$\mathbf{(\mathrm{MP}-2)}$} holds.\end{lemma}

\begin{proof} Let $e \in \X$ smooth and positive on $\Omega$. Then for $t \in \R$ we have
\[ \begin{aligned} J(te) & = \frac{1}{2} t^2 \|e\|^2 - \int_\Omega F(x, \, te) \, \mathrm{d}x \geq
\\[1em] & \geq  \frac{1}{2} t^2 \|e\|^2 - \left( \frac{t^{\frac{1}{\theta}}}{c} \|e\|_{L^\theta(\Omega)} - c^\prime \right) |\Omega| \geq
\\[1em] & \geq  \frac{1}{2} t^2 \|e\|^2 - C_\Omega t^{\frac{1}{\theta}} \|e\|_{L^\theta(\Omega)} + C_\Omega \xrightarrow{t \to + \infty} - \infty. \end{aligned} \] 
Therefore, we can find $\tau \in \R^+$ such that $\tau e$ satisfies $J(\tau e) \leq 0$.\end{proof}

To apply the MP theorem, we only need to prove that $J$ satisfies the Palais-Smale condition at the level $c$. A standard argument shows that
\[ \text{$(u_n)_{n \in \N} \in (\mathrm{PS})_c$ for $J$ $\implies$ $u_n$ bounded}, \]
but the reader may try to prove this themselves as an exercise to get acquainted with the notion of Palais-Smale.

\begin{lemma}Under no extra assumpions, the functional $J$ satisfies the Palais-Smale condition at the level $c > 0$. \end{lemma}

\begin{proof}First, we evaluate $\Phi$ at $u_n$ and decompose the integral in such a way that we can use \textbf{\color{orange}($f_{4}$)}. Namely,
\[ \begin{aligned} \Phi(u_n) & = \int_{ u_n \leq R} F(x, \, u_n) \, \mathrm{d}x + \int_{ u_n \geq R} F(x, \, u_n) \, \mathrm{d}x \leq
\\[1em] & \leq C_{\Omega, \, R, \, f} + \theta \int_{u_n \geq R} u_n f(x, \, u_n) \, \mathrm{d}x \leq
\\[1em] & \leq C_{\Omega, \, R, \, f}^\prime + \theta \int_{\Omega} u_n f(x, \, u_n) \, \mathrm{d}x \leq
\\[1em] & \leq C_{\Omega, \, R, \, f}^\prime + \theta \left[ \int_{\Omega}|\nabla u_n|^2 \, \mathrm{d}x + o(\| u_n \|) \right], \end{aligned} \]
where the last inequality follows from the definition of differentiable:
\[ o(\|u_n\|) = \nabla J(u_n) [u_n] = \int_\Omega |\nabla u_n|^2 \, \mathrm{d}x - \int_\Omega u_n f(x, \, u_n). \]
Since $u_n$ is a Palais-Smale sequence, $|J(u_n)| \leq c$ and hence
\[ \int_\Omega|\nabla u|^2 \, \mathrm{d}x \leq C + 2 \Phi(u_n) \leq C^{\prime \prime} 2 \theta \left[ \int_\Omega |\nabla u|^2 \, \mathrm{d}x + o(\|u_n\|) \right]. \]
Recalling that $2 \theta < 1$, this implies that
\[ \int_\Omega |\nabla u_n|^2 \, \mathrm{d}x \leq \tilde{C} + o(\|u_n\|). \]
Now notice that $p < \frac{n+2}{n-2}$, so $\Phi$ is weakly continuous and its differential is a compact operator. From
\[ \nabla J(u_n)[v] = \langle u_n, \, v \rangle - \langle \nabla \Phi(u_n), \, v \rangle, \]
we conclude that
\[ \nabla J(u_n) = u_n - \nabla \Phi(u_n). \]
Since $u_n$ is Palais-Smale, $u_n$ is bounded and hence we can find a subsequence $u_{n_k}$ converging weakly to some $\bar{u}$. Furthermore,
\[ u_{n_k} = \nabla J(u_{n_k}) + \nabla \Phi(u_{n_k}), \]
and the first term $\nabla J(u_{n_k})$ converges strongly to zero, so by compactness $\nabla \Phi(u_{n_k})$ must converge strongly to $\nabla \Phi(\bar{u})$.\end{proof} 

This proves that $J$ satisfies the Palais-Smale condition at the level $c$. We can finally apply \autoref{thm.n.1} and conclude that \eqref{eq.n.2} admits a positive solution.

\section{Linking theorems}

Let $\scC$ be a nonempty class of subsets $A \subseteq \X$. Suppose that
\[ c:= \inf_{A \in \scC} \sup_{u \in A} J(u) > - \infty. \]
The idea is that, if $\scC$ is stable under deformations, we can do a sort of MP theorem for which $c$ is a candidate min-max level.

\begin{definition}[Link] \index{linking} Let $\cN$ be a compact manifold with nonempty boundary and let $C \subseteq \X$ be a subset. Consider the class of homotopies
\[ \scH := \left\{ h \in C(\cN, \, \X) \: : \: h \, \big|_{\partial \cN} \equiv \mathrm{id}_{\partial \cN} \right\}. \]
We say that $\partial \cN$ and $C$ \textit{link} if
\[ h(\cN) \cap C \neq \varnothing \quad \text{for all $h \in \scH$}. \] \end{definition}

%% PICTURE IPAD

\begin{example}The MP theorem is a linking-type theorem with $C = S_R$ and
\[ \cN := \{ te \: : \: t \in [0, \, 1] \}. \]
It is easy to verify that $C$ and $\partial \cN$ link using Bolzano's theorem. \end{example}

We will now investigate the linking property between slightly more complicated sets. From now on, we will make use of \textbf{degree theory} and, in particular, of the homotopy property. The reader that is not acquainted with it can find the formal construction and the main properties in \cite{nasep}.

\begin{proposition} Let $\X$ be a normed vector space and assume that $\X := V \oplus W$ with $V, \, W$ closed subspaces and $\mathrm{dim}(V) = k < \infty$. Then
\[ C := W \quad \text{and} \quad \cN := \{ v \in V \: : \: \|v\| \leq r \} \]
link. \end{proposition}

\begin{proof}Let $h \in \scH$ and let $p : \X \to V$ be the projection associated to the direct sum. Then
\[ \widetilde{h} := p \circ h : \cN \longrightarrow V \]
coincides with the identity on $\partial \cN$. It follows from degree theory that $\widetilde{h}$ vanishes at some $z \in \cN$ that does not belong to the boundary, and hence
\[ \widetilde{h}(z) = 0 \implies h(z) \in V^c = W = C. \] \end{proof}

\begin{proposition} Let $\X$ be a normed vector space and assume that $\X := V \oplus W$ with $V, \, W$ closed subspaces and $\mathrm{dim}(V) = k < \infty$. Given $e \in W$ and $R > 0$ define
\[ C := \{ w \in W \: : \: \|w\| \leq r, \]
and
\[ \cN := \{ u = v + se \: : \: v \in V, \, \, \|v\| \leq R, \, \, s \in [0, \, 1] \}. \]
Then $C$ and $\partial \cN$ link, provided that $\|e\| > r$. \end{proposition}

\begin{proof}Let $h \in \scH$ and let $p : \X \to V$ be the projection associated to the direct sum. Identify the manifold $\cN$ with
\[ \cN \cong \bar{B}_V(0, \, R) \times \{ se \: : \: s \in [0,\,1]\} \]
and define
\[ \widetilde{h}(u) := \left( p \circ h(u), \, \| h(u) - p \circ h(u) \| - r \right). \]
We now evaluate it at the boundary $\partial \cN$:
\[ \widetilde{h}(v, \, s) = (v, \, \|e\| - r) \neq (0, \, 0). \]
It follows that we can apply once again degree theory to find $(v, \, s) \in \cN$ such that $\widetilde{h}(v, \, s)$ vanishes. In particular,
\[ \begin{cases} p \circ h(v + se) \\[.6em] \| h(v + se) - p \circ h(v + se) \| = r \end{cases} \implies h(v+se) \in W \quad \text{and} \quad \| h(v + se) \| = r, \]
which means that $h(v + se) \in C$, and this concludes the proof. \end{proof}

We are now ready to generalise the MPT. Let $\X$ be a Hilbert space, $J \in C^1(\X, \, \R)$ and $\partial \cN, \, C \subset \X$ such that $\cN_\partial$ and $C$ link. Assume that \mbox{}
\begin{enumerate}[label={\color{blue}$\mathbf{(J\arabic*)}$}, leftmargin=2.5\parindent]
\item $J$ is bounded from below on $C$, that is, $\rho := \inf_{u \in C} J(u) > - \infty$;
\item $\rho > \beta := \sup_{u \in \partial \cN} J(u)$.
\end{enumerate}

\bd \index{linking level}
The number
\[
c := \inf_{h \in \scH} \sup_{u \in \cN} J \circ h(u)
\]
is called {\em linking level} associated to the function $J$.
\ed

\bl
Suppose that $\cN_\partial$ and $C$ link. If {\color{blue}$\mathbf{(J1)}$} holds, then $c \geq \rho$.
\el

\begin{proof}
By definition, for each $h \in \scH$ the intersection $h(\cN) \cap C$ is nonempty. Thus
\[
\sup_{u \in \cN} J(h(u)) \geq \inf_{u \in C} J(u) = \rho.
\]
\end{proof}

\bthm
Suppose that the following assumptions hold: \mbox{}
\begin{enumerate}[label=\textbf{(\alph*)}]
\item $\cN_\partial$ and $C$ link.
\item {\color{blue}$\mathbf{(J1)}$} and {\color{blue}$\mathbf{(J2)}$} hold.
\item The functional satisfies the Palais-Smale condition at the linking level $c$.
\end{enumerate}
Then $c$ is a critical value, that is, there exists $u \in \X$ such that $J(u) = c$ and $\nabla J(u) = 0$.
\ethm

\begin{proof}
Notice that $c \geq \rho > \beta$. Suppose that $c$ is not a critical value and use the {\em deformation lemma} to find a continuous deformation $\eta$ which satisfies
\begin{itemize}
\item $\eta(J^{c+\delta}) \subseteq J^{c-\delta}$ for $\delta$ such that $\beta < c - \delta$;
\item $\eta(u) = u$ for all $u \in J^\beta$.
\end{itemize}
Now let $h \in \scH$. It is easy to verify that $\eta \circ h \in \scH$ since it is composition of continuous mappings and also
\[
\text{$\eta(u) = u$ for all $u \in J^\beta$} \implies \eta \circ h \, \big|_{\partial \cN} =  \eta \, \big|_{\partial \cN} = \mathrm{id}_{\partial \cN}
\]
since $\cN \subset J^\beta$. Now let $\tilde{h} \in \scH$ be such that
\[
\sup_{u \in \cN} J(\tilde{h}(u)) < c + \delta.
\]
Then
\[
\sup_{u \in \cN} J(\eta \circ \tilde{h}(u)) < c - \delta,
\]
and this gives a contradiction since $c$ is the infimum. This concludes the proof.
\end{proof}

We now present three easy consequences of the theory developed in this section, which are incredibly interesting by themselves.

\bthm
Let $C$ be a manifold of codimension one in $\X$ and suppose that $u_0, \, u_1$ are points of $\X \setminus C$ belonging to two distinct connected components of $\X \setminus C$. Let $J \in C^1(\X, \, \R)$ satisfy the following assumptions: \mbox{}
\begin{enumerate}[label=\textbf{(L-\alph*)}, leftmargin=2.5\parindent]
\item $\inf_C J(u) > \max \{ J(u_0), \, J(u_1) \}$;
\item $J$ satisfies the Palais-Smale condition at the linking level $c$.
\end{enumerate}
Then $J$ has a critical point $\bar{u}$ at level $c$ and $\bar{u} \neq u_0, \, u_1$.
\ethm

\bthm \label{thm.t.1}
Let $\X = V \oplus W$, where $V$ and $W$ are closed subspaces and $\mathrm{dim}(V) < \infty$. Suppose $J \in C^1(\X, \, \R)$ satisfies: \mbox{}
\begin{enumerate}[label=\textbf{(L-\alph*)}, leftmargin=2.5\parindent]
\item There exist $r, \, \rho > 0$ such that
\[
J(w) \geq \rho \quad \text{for all $w \in W$ with $\|w\|=r$}.
\]
\item There exist  $R>0$ and $e \in W$, with $\|e\| > r$ such that, letting
\[
\cN = \{ u = v + te \: : \: v \in V, \,  \|v\| \leq R, \, t \in [0, \, 1] \},
\]
one has that
\[
J(u) < 0 \quad \text{for all $u \in \partial \cN$}.
\]
\end{enumerate}
If, in addition, $J$ satisfies the Palais-Smale condition at the linking level $c$, then $J$ has a critical point $\bar{u}$ at level $c > 0$. In particular, $\bar{u} \neq 0$.
\ethm

\bthm
Let $\X = V \oplus W$, where $V$ and $W$ are closed subspaces and $\mathrm{dim}(V) < \infty$. Suppose $J \in C^1(\X, \, \R)$ satisfies: \mbox{}
\begin{enumerate}[label=\textbf{(L-\alph*)}, leftmargin=2.5\parindent]
\item There exist $\rho > 0$ such that
\[
J(w) \geq \rho \quad \text{for all $w \in W$}.
\]
\item There exist  $r > 0$, $\beta < \rho$ such that
\[
J(u) \leq \beta \quad \text{for all $u \in V$ with $\|v\|=r$}.
\]
\end{enumerate}
If, in addition, $J$ satisfies the Palais-Smale condition at the linking level $c$, then $J$ has a critical point $\bar{u}$ at level $c > 0$.
\ethm

\subsection{Application of the saddle point theorem}

Let $\Omega$ be an open bounded subset of $\R^n$ with smooth boundary. Consider the Dirichlet problem
\begin{equation} \label{eq.t.1} \begin{cases}
- \Delta u - \lambda u = f(x, \, u) & \text{if $x \in \Omega$},
\\[.6em] u = 0 & \text{if $x \in \partial \Omega$}.
\end{cases} \end{equation}

\bthm \label{thm.t.2}
Suppose that \mbox{}
\begin{enumerate}[label=\textbf{(\roman*)}]
\item $\lambda$ is not an eigenvalue of $-\Delta$;
\item $f$ satisfies the Carathéodory condition;
\item $f$ is sublinear, that is, there is $\alpha < 1$ such that
\[
|f(x, \, s)| \leq a + b |s|^\alpha.
\]
\end{enumerate}
Then \textbf{(L-1)} and \textbf{(L-2)} hold. Furthermore, the functional associated to the problem,
\[
J(u) = \int_\Omega ( |\nabla u|^2 - \lambda |u|^2) \, \mathrm{d}x - \int_\Omega F(x, \, u) \, \mathrm{d}x,
\]
satisfies the Palais-Smale condition at any level. In particular, the linking level is critical.
\ethm

\begin{proof}
By assumption, there exists $k \in \N$ such that $\lambda \in (\lambda_k, \, \lambda_{k+1})$. If $\varphi_j$ denotes the $j$th eigenfunction, then we can take
\[
V := \mathrm{Span} \langle \varphi_1, \, \dots, \, \varphi_k \rangle.
\]
In this case, $W$ is the complementary subspace in $\X := H_0^1(\Omega)$. Notice that the quadratic form
\[
Q(u) = |\nabla u|^2 - \lambda |u|^2
\]
is definite negative on $V$ and definite negative on $W$, and also that the sublinearity of $f$ together with the Sobolev embedding implies that
\[
\left| \int_\Omega F(x, \, u) \, \mathrm{d}x \right| \leq A + B \|u\|^{\alpha + 1}.
\]
It follows that there exists $\gamma > 0$ such that
\[
u \in V \implies J(u) \leq - \gamma \|u\|^2 + A + B \|u\|^{\alpha + 1} \xrightarrow{\|u\| \to \infty} - \infty,
\]
which means that we can select $R$ big enough in \autoref{thm.t.1} for which \textbf{(L-2)} holds. In a similar fashion, notice that
\[
u \in W \implies J(u) \geq \gamma \|u\|^2 - A - B \|u\|^{\alpha + 1},
\]
which means that $\rho := \inf_W J(u) > - \infty$. Since $R$ is arbitrarily big, we can also require that $\beta < \rho$ and thus \textbf{(L-1)} holds as well.

\paragraph{Palais-Smale condition.} Write $u = u_V + u_W$. Then
\[
\nabla J(u)[u_V] = \nabla Q(u)[u_V] - \int_\Omega f(x, \, u) u_V \, \mathrm{d}x = 2 Q(u_V) -  \int_\Omega f(x, \, u) u_V \, \mathrm{d}x.
\]
Let $(u_n)_{n \in \N}$ be a Palais-Smale sequence. Then
\[
\nabla J(u_n)[(u_n)_V] = o(\|u_n\|)
\]
because $\nabla J(u_n)[(u_n)_V]$ converges to zero; on the other hand, the identity above suggests that
\[
\nabla J(u_n)[(u_n)_V] = 2 Q((u_n)_V) + \mathcal{O}(1 + \|u\|^{\alpha + 1}).
\]
Since $Q < 0$ on $V$, we can easily infer that
\[
\gamma \|(u_n)_V\|^2 \leq o(\|u_n\|) + \mathcal{O}(1 + \|u_n\|^{1 + \alpha}).
\]
In a similar fashion, we make the same computation on $W$ and find that
\[
\gamma \|(u_n)_W\|^2 \leq o(\|u_n\|) + \mathcal{O}(1 + \|u_n\|^{1 + \alpha}).
\]
Therefore, any Palais-Smale sequence for the functional $J$ is bounded in the $\| \cdot \|_\X$-norm. For the compactness, notice that $u_n$ bounded implies
\[
u_{n_k} \rightharpoonup \bar{u}
\]
and, using the fact that $V$ is finite-dimensional, we also have that
\[
(u_{n_k})_V \xrightarrow{k \to \infty} \bar{u}_V.
\]
Moreover, we have
\[
\nabla J(u_n)[v] = \int_\Omega (\nabla u_n \cdot \nabla v - \lambda u_n v ) \, \mathrm{d}x - \int_\Omega f(x, \, u_n) v \, \mathrm{d}x,
\]
which gives
\[
\int_\Omega \nabla u_n \cdot \nabla v \, \mathrm{d}x = \nabla J(u_n)[v] + \int_\Omega \lambda u_n v \, \mathrm{d}x + \int_\Omega f(x, \, u_n) v \, \mathrm{d}x.
\]
We conclude that the convergence is strong (up to subsequences) because the first addendum converges to zero, while the other two are compact linear operators by Sobolev embedding and sublinearity of $f$.
\end{proof}

\brmk
If $\lambda = \lambda_k$, then the existence of the solution is not guaranteed. Indeed, if we consider the problem
\[ \begin{cases}
- \Delta u - \lambda u = \varphi_k & \text{if $x \in \Omega$},
\\[.6em] u = 0 & \text{if $x \in \partial \Omega$},
\end{cases} \]
then it is easy to verify that it does not admit any solution since $u$ should be in the orthogonal of the linear space generated by $\varphi_k$.
\ermk

To conclude this section, we want to point out why any Palais-Smale sequence is bounded is enough to infer that the functional $J$ satisfies $(\mathrm{PS})_c$.

Let $\Omega \subset \R^n$ be a bounded set and let $f$ be a function that satisfies the Carathéodory condition and the growth condition
\[
|f(x, \, s)| \leq A + B |s|^p \quad \text{for $p < \frac{n+2}{n-2}$ if $n \geq 3$}.
\]
Let $F(x, \, u) := \int_0^u f(x, \, s) \, \mathrm{d}s$ and $\Phi(u) = \int_\Omega F(x, \, u) \, \mathrm{d}x$. We claim that $\nabla \Phi$ is compact as an operator from $\X := H_0^1(\Omega)$ to $\X$.

\begin{proof}
Let $u_n$ be a bounded sequence in $\X$ weakly converging to some $\bar{u}$. Then $u_{n_k}$ converges strongly to $\bar{u}$ in $L^{p+1}(\Omega)$ and by Nemitski theorem
\[
f(x, \, u_{n_k}) \to f(x, \, \bar{u}) \quad \text{strongly in $L^{\frac{p+1}{p}}(\Omega)$}.
\]
This implies that $\| \nabla \Phi(u_{n_k}) - \nabla \Phi (u) \| \to 0$ which is enough to infer that $\nabla \Phi$ is compact.
\end{proof}

\bcor
Consider the Dirichlet problem
\[ \begin{cases}
- \Delta u = f(x, \, u) \quad \text{if $x \in \Omega$},
\\[.6em] u = 0 & \text{if $x \in \partial \Omega$},
\end{cases} \]
and the associated function $J(u) = \frac{1}{2} \int_\Omega |\nabla u|^2 \, \mathrm{d}x - \Phi (u)$. Then the following properties hold: \mbox{}
\begin{enumerate}[label=\textbf{(\alph*)}]
\item If $u_n$ converges weakly to $\bar{u}$ and $\nabla J(u_n) \to 0$, then $u_{n_k}$ converges strongly to $\bar{u}$.
\item If Palais-Smale sequences at the level $c$ for $J$ are bounded, then $J$ satisfies the $(\mathrm{PS})_c$ condition.
\end{enumerate}
\ecor

\subsection{Application of linking-type theorems}


\bthm
Let $\Omega$ be a bounded subset of $\R^n$. Let $f \in \mathbb{F}_p$ with $1 < p < \frac{n+2}{n-2}$ for $n \geq 3$, and suppose that
\[
\lim_{s \to 0^+} \frac{f(x, \, s)}{s} = \lambda \quad \text{for a.e. $x \in \Omega$},
\]
for any $\lambda \in \R$, and
\[
\exists \, r > , \, \theta \in (0, \, \frac{1}{2}) \: : \: 0 < F(x, \, u) \leq \theta u f(x, \, u) \quad \text{for all $x \in \Omega$ and all $u \geq r$}.
\]
Then the Dirichlet problem \eqref{eq.t.1} admits a nontrivial solution.
\ethm

\begin{proof}We prove the result for the model problem
\begin{equation} \label{eq.t.3} \begin{cases}
- \Delta u = \lambda u + |u|^{p-1} u & \text{if $x \in \Omega$},
\\[.6em] u = 0 & \text{if $x \in \partial \Omega$}.
\end{cases} \end{equation}
Let $\X = H_0^1(\Omega)$ and consider the associated functional
\[
J(u) = \frac{1}{2} \|u\|^2 - \frac{1}{2} \lambda \|u\|_{L^2(\Omega)}^2  \frac{1}{p+1} \| u \|_{L^{p+1}(\Omega)}^{p+1}.
\]
If $\lambda < \lambda_1$, then we have proved already (see [REF]) that the MPT is enough to infer the existence of a nontrivial solution. So we can assume without loss of generality that
\[
\lambda_k \leq \lambda < \lambda_{k+1}
\]
for some $k \geq 1$; the idea is to apply \autoref{thm.t.1} with $V = \mathrm{Span}\langle \varphi_1, \, \cdots, \, \varphi_k\rangle$ and $W = V^\perp$, the $L^2$ complement of $V$. Indeed, if $w \in W$ we can always write
\[
w = \sum_{i = k+1}^\infty a_i \varphi_i.
\]
If $\|w\| \to 0$, then
\[
J(w) = \frac{1}{2} \sum_{i = k+1}^\infty a_i^2 \left( 1 - \frac{\lambda}{\lambda_i} \right) + o(\|w\|^2) \geq \frac{1}{2} \left(1 - \frac{\lambda}{\lambda_{k+1}} \right) \|w\|^2 + o(\|w\|^2),
\]
and the latter is always strictly positive since $\lambda < \lambda_{k+1}$ by assumption. In particular, the assumption $\mathbf{(L-a)}$ of \autoref{thm.t.1} holds with $r$ small enough. Now let $\tilde{V}$ be a finite-dimensional subspace of $\X$ and $\tilde{v} \in \tilde{V}$ be an element with unitary norm; it turns out that
\[
J(t \tilde{v}) = \frac{1}{2} t^2 - \frac{1}{2} \lambda^2 t^2 \| \tilde{v} \|_{L^2(\Omega)}^2 - \frac{1}{p+1}t^{p+1} \| \tilde{v} \|_{L^{p+1}(\Omega)}^{p+1}.
\]
Since $p > 1$ and $\tilde{V}$ finite-dimensional, it follows that we can always find $t > 0$ big enough such that the quantity above is strictly negative. In particular, we can find $R > r$ and $e \in W$, $\|e\| = R$, such that
\[
\|v + te \| \geq R \implies J(v + te) < 0.
\]
Then on the three sides of $\partial \cN$ given by $\{ u = v + te \: : \: \|v\| = R \} \cup \{ u = v + Re \}$ the functional $J$ is strictly negative. It remains to see what happens on the fourth side of $\partial \cN$, namely
\[
\{ v \in V \: : \: \|v\| \leq R \}.
\]
However, it is easy to verify that $v = \sum_{i = 1}^k a_i \varphi_i$ gives $\|v\|_{L^2(\Omega)}^2 = \sum_{i = 1}^k \lambda_i^{-1} a_i^2$; this implies that
\[
\|v\|_{L^2(\Omega)}^2 \geq \lambda_k^{-1} \|v\|^2,
\]
and hence
\[
J(v) \leq \frac{1}{2} \left( 1 - \frac{\lambda}{\lambda_k} \right) \|v\|^2 \leq 0.
\]
This shows that $\mathbf{(L-b)}$ holds as well. The Palais-Smale condition is obtained in the same way as \autoref{thm.t.2} so we can apply \autoref{thm.t.1} to conclude.
\end{proof}

\brmk
Notice that $J \, \big|_{C} > 0$ strictly, so a solution corresponding to a critical point at the level $c$ is necessarily nontrivial.
\ermk

\section{The Pohozaev identity}

Let $\Omega$ be an open bounded subset of $\R^n$ with smooth boundary. Consider the Dirichlet boundary value problem with nonlinearity independent of $x$, that is,
\begin{equation} \label{eq.t.3} \begin{cases}
- \Delta u = f(u) & \text{if $x \in \Omega$},
\\[.6em] u = 0 & \text{if $x \in \partial \Omega$},
\end{cases} \end{equation}
and let $F(u) = \int_0^u f(s) \, \mathrm{d}s$.

\bthm[Pohozaev] \index{Pohozaev identity}
Let $\nu$ denote the unit outer normal at $\partial \Omega$. If $u$ is any classical solution of \eqref{eq.t.3}, then the following identity holds:
\begin{equation}\label{eq.pohozaev}
n \int_\Omega F(u) \, \mathrm{d}s = \frac{1}{2} \int_{\partial \Omega} u_\nu^2(x \cdot \nu) \, \mathrm{d}\sigma + \frac{n-2}{2} \int_\Omega u f(u) \, \mathrm{d}x.
\end{equation}
\ethm

%%Ci sono dei remark da aggiungee sul file di testo

\begin{proof}
Set $\Theta(x) := (x \cdot \nabla u(x)) \nabla u(x)$. Then
\[ \begin{aligned}
\mathrm{div} \, \Theta & = \Delta u(x \cdot \nabla u) + \sum_k \frac{\partial u}{\partial x_k} \frac{\partial}{\partial x_k} \left( \sum_i x_i \frac{\partial u}{\partial x_i} \right) =
\\[1em] & = \Delta u(x \cdot \nabla u) + \sum_k \left( \frac{\partial u}{\partial x_k} \right)^2 + \sum_{i, \, k} \frac{\partial u}{\partial x_k} x_i \frac{\partial^2 u}{\partial x_i \partial x_k} = 
\\[1em] & = \Delta u(x \cdot \nabla u) + |\nabla u|^2 + \frac{1}{2} \sum_i x_i \frac{\partial}{\partial x_i} |\nabla u|^2.
\end{aligned} \]
Then an application of the divergence theorem shows that
\[
\int_\Omega \left[\Delta u(x \cdot \nabla u) + |\nabla u|^2 + \frac{1}{2} \sum_i x_i \frac{\partial}{\partial x_i} |\nabla u|^2 \right] \, \dr x = \int_{\partial \Omega} (x \cdot \nabla u)(\nabla u \cdot \nu) \, \dr \sigma.
\]
As for the boundary term, since $u = 0$ on $\partial \Omega$ one has that $\nabla u(x) = u_\nu \nu$ and thus the above equation becomes
\[
\int_\Omega \left[\Delta u(x \cdot \nabla u) + |\nabla u|^2 + \frac{1}{2} \sum_i x_i \frac{\partial}{\partial x_i} |\nabla u|^2 \right] \, \dr x = \int_{\partial \Omega} (x \cdot \nu) u_\nu^2 \, \dr \sigma.
\]
Now set $\Theta_1(x) := \frac{1}{2} |\nabla u|^2 x$. Since its divergence is
\[
\mathrm{div} \, \Theta_1 = \frac{n}{2} |\nabla u|^2 + \frac{1}{2} \sum_i x_i \frac{\partial}{\partial x_i} |\nabla u|^2,
\]
another application of the divergence theorem shows that
\[
\int_\Omega \left[ \frac{n}{2} |\nabla u|^2 + \frac{1}{2} \sum_i x_i \frac{\partial}{\partial x_i} |\nabla u|^2 \right] \, \dr x = \frac{1}{2} \int_{\partial \Omega} (x \cdot \nu) u_\nu^2 \, \dr \sigma.
\]
If we plug this into the previous identity we find that
\begin{equation} \label{eq.t.4}
\int_\Omega \Delta u(x \cdot \nabla u) \, \dr x + (1 - \frac{n}{2}) \int_\Omega |\nabla u|^2 \, \dr x = \frac{1}{2} \int_{\partial \Omega} (x \cdot \nu) u_\nu^2 \, \dr \sigma.
\end{equation}
The first integral can easily be rewritten using the equation \eqref{eq.t.3} of which $u$ is a solution; namely, we have
\[
- \int_\Omega \Delta u(x \cdot \nabla u) \, \dr x = \int_\Omega f(u)(x \cdot \nabla u) \, \dr x = \int_\Omega \sum_i x_i \frac{\partial F(u)}{\partial x_i} \, \dr x.
\]
Integrating by parts we obtain
\[
\int_\Omega \sum_i x_i \frac{\partial F(u)}{\partial x_i} \, \dr x = - n \int_\Omega F(u) \, \dr x,
\]
which implies that
\[
\int_\Omega \Delta u(x \cdot \nabla u) \, \dr x =n \int_\Omega F(u) \, \dr x.
\]
Once again, using \eqref{eq.t.3} we conclude that
\[
\int_\Omega |\nabla u|^2 \, \dr x = \int_\Omega u f(u) \, \dr x,
\]
which plugged into the identity \eqref{eq.t.4} leads to the Pohozaev identity.
\end{proof}

An immediate consequence is that the growth of the nonlinearity $f$ with exponent $p < \frac{n+2}{n-2}$ cannot be eliminated if we want to find nontrivial solutions of \eqref{eq.t.4}. There is a more precise statement which follows from the Pohozaev identity:

\bcor
If $\Omega$ is a star-shaped (w.r.t. the origin) domain in $\R^n$, then any smooth solution of \eqref{eq.t.4} satisfies
\[
n \int_\Omega F(u) \, \dr x - \frac{n-2}{2} \int_\Omega u f(u) \, \dr x > 0.
\]
In particular, if $f(u) = |u|^{p-1}u$, then we find
\[
\left( \frac{n}{p+1} - \frac{n-2}{2} \right) \int_\Omega |u|^{p+1} \, \dr x > 0,
\]
and hence $u \neq 0$ implies $p < \frac{n+2}{n-2}$.
\ecor

%%REMARK DA AGGIUNGERE

\chapter{Lusternik-Schnirelman Theory} \thispagestyle{empty}

In this chapter we aim to discuss the elegant theory of Lusternik and Schnirelman that connects critical points of functionals on manifolds to topological properties of the latter.

\section{Lusternik-Schnirelman category}

Throughout this chapter, $\M$ will always denote a Hilbert space or a $C^1$-submanifold modelled on a Hilbert space. 

\bd[Contractible] \index{contractible set}
Let $X$ be a topological space. A set $A \subset X$ is {\em contractible} in $X$ if the inclusion $\iota : A \hookrightarrow X$ is homotopic to a constant map. Namely, there exists
\[
H \in C \big([0, \, 1] \times A, \, X\big)
\]
such that $H(0, \, u) = u$ and $H(1,\, u) = p$ for all $u \in A$.
\ed

\bd[Category] \index{Lusternik-Schnirelman category}
Let $X$ be a topological space and $A \subset X$. The {\em (L-S) category} of $A$ with respect to $X$, denoted by $\mathrm{cat}(A, \, X)$, is the least integer $k\in \N$ such that
\[
A \subseteq \bigcup_{i = 1}^k A_i,
\]
where each $A_i$ is closed and contractible in $X$. If such an integer does not exist, we set $\mathrm{cat}(A, \, X) = \infty$ and if $A$ is empty we set $\mathrm{cat}(\varnothing, \, X) = 0$.
\ed

\brmk
The category of $A$ coincide with the category of its closure. Moreover
\[
\mathrm{cat}(A, \, X) \geq \mathrm{cat}(A, \, Y)
\]
provided that $A \subset X\subset Y$.
\ermk

\bex \mbox{}
\begin{enumerate}[label=\textbf{(\roman*)}]
\item The sphere $S^{m-1}$ is contractible in $\R^m$ so $\mathrm{cat}(S^{m-1}, \, \R^m) = 1$. However, it is not contractible in itself but can be covered by two closed hemispheres so $\mathrm{cat}(S^{m-1}) = 2$.
\item The sphere in a infinite-dimensional Hilbert space is always contractible so
\[
\mathrm{cat}(S_H, \, H) = 1.
\]
The reader interested in this property might refer to \cite{90}.
\item The category torus $T^2 = S^1 \times S^1 \subset \R^3$ in itself is equal to $3$. It is easy to verify that $\mathrm{cat}(T^2) \leq 3$ using $A_1, \, A_2, \, A_3$ as defined in Figure [REF].

The opposite inequality, however, is quite hard to obtain and we will only explain at the end of the section how to use a general result to prove it.
\end{enumerate}
\eex

%%DISEGNI IPAD



\bl
Let $A, \, B \subset \M$. \mbox{}
\begin{enumerate}[label=\textbf{(\alph*)}]
\item If $A \subset B$, then $\mathrm{cat}(A, \, \M) \leq \mathrm{cat}(B, \, \M)$.
\item $\mathrm{cat}(A \cup B, \, \M) \leq  \mathrm{cat}(A, \, \M) + \mathrm{cat}(B, \, \M)$.
\item If $A$ is closed and $\eta \in C(A, \, \M)$ is a deformation, then
\begin{equation} \label{def.111}
 \mathrm{cat}(A, \, \M) \leq  \mathrm{cat}\Big( \overline{\eta(A)}, \, \M\Big).
\end{equation}
\end{enumerate}
\el

\begin{proof}The only nontrivial assertion is $\mathbf{(c)}$. Let $k := \mathrm{cat}\Big( \overline{\eta(A)}, \, \M\Big)$ an assume that it is finite (otherwise there is nothing to prove). Then
\[
\eta(A) \subset \bigcup_{i = 1}^k C_i,
\]
where $C_i$ is closed and contractible in $\M$. Set
\[
A_i := \eta^{-1}(C_i)
\]
and observe that these are all closed because $\eta$ is continuous. Moreover, each $A_i$ is contractible because the composition of a contraction with $\eta$ gives another contraction. Since
\[
A \subset \bigcup_{i = 1}^k A_i,
\]
we easily deduce that \eqref{def.111} holds.\end{proof}

The strict inequality in \eqref{def.111} is possible to achieve. Indeed, let $\M = S^1$ and $A = S_+^1$ the hemisphere
\[
S_+^1 = \{ \e^{\imath \theta} \: : \: \theta \in [0, \, 2 \pi] \}.
\]
Let $\eta(\e^{\imath \theta}) := H(1, \, \theta)$, where
\[
H(t, \, \theta) = \e^{\imath(t + 1) \theta}
\]
is defined for all $t \in [0, \, 1]$. Then it is trivial to verify that $\mathrm{cat}(A, \, \M) <  \mathrm{cat}\Big( \overline{\eta(A)}, \, \M\Big)$.

\bl \label{lemma.z.z1}
Let $A \subset \M$ be compact. Then the following properties hold: \mbox{}
\begin{enumerate}[label=\textbf{(\roman*)}]
\item $\mathrm{cat}(A, \, \M) < \infty$.
\item There exists a neighbourhood $U_A$ of $A$ such that
\[
\mathrm{cat}(A, \, \M) = \mathrm{cat}\Big(\overline{U_A}, \, \M\Big).
\]
\end{enumerate}
\el

\begin{proof}
Suppose first that $\mathrm{cat}(A, \, \M) = 1$ and let $H : [0, \, 1] \times A \to \M$ be the contraction to the constant map $p$. We would like to extend $H$ to
\[
S := (\{0\}\times \M) \cup ([0, \, 1] \times A) \cup (\{1\} \times \M),
\]
and this is easily achieved by setting
\[
H(t, \,u) := \begin{cases}
u & \text{if $(t, \, u) \in \{0\} \times \M$},
\\[.6em] H(t, \,u) & \text{if $(t, \, u) \in [0, \, 1] \times A$},
\\[.6em] p & \text{if $(t, \, u) \in \{1\} \times \M$}.
\end{cases}
\]
Since $S$ is closed in $ Y := [0, \, 1] \times \M$ and $H$ is continuous from $S$ to $\M$, we can use the extension property to find a neighbourhood $N$ of $S$ in $Y$ and a function $\tilde{H} \in C(N, \, \M)$ such that
\[
\tilde{H} \, \big|_S \equiv H.
\]
Since $[0, \, 1] \times A$ is compact and the distance with $Y \setminus N$ is strictly positive, we can easily find a neighbourhood $U_A$ of $A$ in $\M$ such that
\[
[0,\,1] \times \overline{U_a} \subseteq N.
\]
It is easy to verify that $\overline{U_A}$ is contractible in $\M$ using the contraction $\tilde{H}$ appropriately restricted to a subset of its domain. In particular,
\[
\mathrm{cat}(A, \, \M) = 1 \implies \mathrm{cat}\Big(\overline{U_A}, \, \M\Big) = 1.
\]

\begin{enumerate}[label=\textbf{(\roman*)}]
\item Let $q \in A$ Then above we proved that there exists a contractible neighbourhood $U_q$ of category equal to one. Since we can always cover $A$ with finitely many $U_q$'s, we infer that the category of $A$ is finite.
\item Let $k = \mathrm{cat}(A,\, \M)$ and let $A_1,\, \cdots,\, A_k$ be the closed and contractible sets such that
\[
A \subseteq \bigcup_{i = 1}^k A_i.
\]
Observe that if we replace $A_i$ with $A \cap A_i$, we can assume without loss of generality that $A_i$'s are also compact. Since $\mathrm{cat}(A_i,\, \M) = 1$ we can find an open neighbourhood $U_i$ of $A_i$ such that
\[
\mathrm{cat}\Big(\overline{U_i},\, \M \Big) = 1
\]
for each $i = 1,\, \cdots, \, k$. Let $U_A := \bigcup_{i=1}^k U_i$ and notice that $U_A$ is an open neighbourhood of $A$ such that
\[
\mathrm{cat}\Big(\overline{U_A}, \, \M\Big) \geq k.
\]
Since $\overline{U_A} \subset \bigcup_{i = 1}^k \overline{U_i}$, we also get the opposite inequality and hence the equality holds.
\end{enumerate}
\end{proof}

%%FIG IPAD

\brmk
It can be proved that the category satisfies the inequality
\[
\mathrm{cat}(\M) \geq \mathrm{cup-length}(\M) + 1,
\]
where the {\em cup-length}\index{cup-length} of $\M$ is defined by
\[
\mathrm{cup-length}(\M) = \sup \left\{ k \in \N \: : \: \text{$\exists \, \alpha_1, \, \cdots,\, \alpha_k \in \M^\ast$ s.t. $\alpha_1 \cup \cdots \cup \alpha_k \neq 0$} \right\}.
\]
If $\M$ is a smooth manifold, then by De Ram's cohomology $\alpha_1 \cup \cdots \cup \alpha_k$ corresponds to the $\wedge$-product of differential forms. In particular
\[
\mathrm{d}x^1 \wedge \mathrm{d}x^2 \neq 0
\]
on the torus $\mathbb{T}^2$, so we obtain the bound $\mathrm{cat}(\mathbb{T}^2) \geq 3$.
\ermk

\section{Lusternik-Schnirelman theorems}

Let $\M$ be a Hilbert space or a $C^1$-submanifold modelled on a Hilbert space. Define
\[
\mathrm{cat}_K(\M) := \sup \left\{ \mathrm{cat}(A,\, \M) \: : \: \text{$A \subseteq M$ is compact} \right\}
\]
and introduce the corresponding class of sets that is preserved when we use deformations; namely, let
\[
C_m := \left\{ A \subseteq \M \: : \: \text{$A$ is compact and $\mathrm{cat}(A,\,\M) \geq m$} \right\}
\]
for $m \leq \mathrm{cat}_K(\M)$. Let $J \in C^1(\M,\,\R)$ and define
\[
c_m := \inf_{A \in C_m} \max_{u \in A} J(u).
\]
The following properties follows from the definition immediately: \mbox{}
\begin{enumerate}[label=\textbf{(\alph*)}]
\item The first level, $c_1$, coincide with $\inf_{u \in \M} J(u)$. 
\item The sequence of levels is increasing, that is,
\[
c_1 \leq c_2 \leq \cdots \leq c_k \leq \cdots
\]
\item For all $m \leq \mathrm{cat}_K(\M)$ there results $c_m < \infty$.
\item If $J$ is bounded from below on $\M$, then all $c_m$'s are finite.
\end{enumerate}

\bthm \label{thm.8..8}
Let $J \in C^1(\M,\, \R)$ be a functional bounded from below on $\M$ and satisfying the Palais-Smale condition at all $c \in \R$. Then $J$ has at least $\mathrm{cat}_K(\M)$ critical points and the following holds: \mbox{}
\begin{enumerate}[label=\textbf{(\arabic*)}]
\item For all $m \leq \mathrm{cat}_K(\M)$, $c_m$ is a critical value for $J$.
\item If there are integers $q, \, m \geq 1$ such that
\[
c := c_m = c_{m+1} = \cdots = c_{m + q},
\]
then $\mathrm{cat}(\mathcal{Z}_c,\, \M) \geq q + 1$.
\end{enumerate}
\ethm

\brmk
The category of a finite set of points $\{p_1,\, \dots, \, p_N\}$ in $\M$ is always equal to one (if $\M$ is connected). Consequently, $\mathbf{(2)}$ gives us an even more precise information than merely saying that there are infinite critical points at the level $c$.
\ermk

\bl[Deformation] \index{deformation lemma!critical points} \label{deformationlemma:secondo}
Let $J \in C^1(\M,\, \R)$ be a functional bounded from below on $\M$ and satisfying the Palais-Smale condition at all $c \in \R$. Then for each $U$ neighbourhood of $\mathcal{Z}_c$ there are $\delta = \delta(U) > 0$ and a deformation $\eta$ such that
\[
\eta(\M^{c+\delta} \setminus U) \subseteq \M^{c-\delta}.
\]
\el

\begin{proof}
We claim that for each $U$ neighbourhood of $\mathcal{Z}_c$ there exists $\bar{\delta} > 0$ such that
\[
\text{$u \notin U$ and $|J(u) - c | \leq \bar{\delta}$} \implies \| \nabla J(\alpha(t,\,u)) \| \geq 2 \bar{\delta} \quad \text{for all $t \in [0,\,1]$}.
\]
We argue by contradiction. Assume that there are sequences $t_k \in [0,\,1]$ and $u_k \notin U$ such that
\[
|J(u_k) - c| \leq \frac{1}{k} \quad \text{and} \quad \| \nabla J(\alpha(t_k,\,u)) \| \xrightarrow{k\to \infty} 0.
\]
Let $\bar{t} \in [0,\,1]$ be the limit (up to subsequences) of $t_k$ and set $\nu_k := \alpha(t_k, \, u_k)$. Then
\[
J(\nu_k) \leq J(u_k) \leq c + \frac{1}{k}
\]
and, since $J$ is bounded from below and satisfies the Palais-Smale condition, we find that $J(\nu_k)$ converges to $c$. Passing to the limit the inequality above shows that
\[
c = \lim_{k \to \infty} J(\nu_k) \leq  \lim_{k \to \infty} J(u_k) \leq c = \lim_{k \to \infty} (c + \frac{1}{k}),
\]
which means that $J(u_k)$ also converges to $c$, and hence it is enough to prove that $u_k$ converges to some $z$. We know that $\nu_k \to z$ and the flow $\alpha(t,\, z) = z$ for all $t \in [0,\,1]$. We can go backwards and obtain
\[
u_k = \alpha(-t_k, \, \nu_k),
\]
so by Cauchy's theorem we infer that $u_k \to z$. Since $z \in \mathcal{Z}_c$ we find a contradiction because $u_k$ does not belong to $U$ for all $k$ and $\mathcal{Z}_c$ is contained in $U$. The rest of the proof follows as in \hyperref[lemma.f.2]{Lemma \ref{lemma.f.2}}.
\end{proof}

%%DISEGNo

\begin{proof}[Proof of Theorem \ref{thm.8..8}]
The assertion $\mathbf{(1)}$ follows from the deformation lemma in the same fashion it did in the MPT. To prove $\mathbf{(2)}$ we argue by contradiction, i.e., we assume that
\[
\mathrm{cat}(\mathcal{Z}_c, \, \M) \leq q.
\]
Since $J$ satisfies the Palais-Smale condition, the critical set $\mathcal{Z}_c$ is compact and hence there exists an open neighbourhood $U$ of $\mathcal{Z}_c$ such that
\[
\mathrm{cat}\Big(\overline{U}, \, \M\Big) \leq q.
\]
By the second deformation \hyperref[deformationlemma:secondo]{Lemma \ref{deformationlemma:secondo}}, there are $\delta > 0$ and a deformation $\eta$ such that
\[
\eta \left( \M^{c+ \delta} \setminus U \right) \subseteq \M^{c-\delta}.
\]
Since $c = c_{m+q}$ we can find an element $A \in C_{m+q}$ such that
\[
\sup_{u \in A} J(u) \leq c + \delta \implies A \subseteq \M^{c+\delta}.
\]
Set $A^\prime := \overline{A \setminus U}$. Then
\[
\mathrm{cat} \left( A^\prime, \, \M \right) \geq \mathrm{cat}\left(A, \, \M\right)- \mathrm{cat}\left(\overline{U}, \, \M\right)\geq m + q - q = m,
\]
which means that $A^\prime \in C_m$. Therefore, the image of $A^\prime$ via $\eta$ is contained in $\M^{c-\delta}$ and, using the properties of the category, we also have that
\[
\mathrm{cat}\left( \eta(A^\prime),\, \M \right) \geq m.
\]
In particular, we have $A^\prime \in C_m$ and
\[
\sup_{A^\prime} J(u) \leq c - \delta,
\]
but this is a contradiction with the very definition of $c_m$.
\end{proof}

\bthm
Let $\M$ be a Hilbert space or a $C^{1,\, 1}$-manifold and let $J \in C^{1,\,1}(\M,\, \R)$ be bounded from below. Suppose that there exists $a \in \R$ such that the Palais-Smale condition holds at all levels $c \leq a$. Then
\[
\mathrm{cat}(\M^a) < \infty.
\]
\ethm

\begin{proof}
Let $Z = \{ \nabla J = 0 \}$ and set $Z^a := \M^a \cap Z$. The Palais-Smale condition implies that $Z^a$ is compact and by \hyperref[lemma.z.z1]{Lemma \ref{lemma.z.z1}} we can find an open neighbourhood $U^a$ of $Z^a$ such that
\[
\mathrm{cat}(\bar{U}^a,\,M^a) = \mathrm{cat}(Z^a,\,M^a) < \infty.
\]
We can assume without loss of generality that $\| \nabla J (u) \| \leq 1$ for all $u \in U_a$. Then there exists $V^a \subset U^a$, neighbourhood of $Z^a$, such that
\[
d := d(\bar{V}^a,\, \partial U^a) > 0.
\]
If a gradient flow of $J$ exits $V^a$ and enters the complement of $U^a$, then this has to happen in a time bigger than or equal to $d$. Observe that by $\mathrm{(PS_c)}$ there is $\delta > 0$ such that
\[
\| \nabla J(u) \| \geq \delta
\]
for all $u \in \M^a \setminus V^a$. Let $a' := a - \inf_{u \in \M} J(u)$ and $T > \frac{a'}{\delta^2}$. Recall that
\[
- \frac{\mathrm{d}}{\mathrm{d}t} J(\alpha(t,\,u)) = - \| \nabla J(\alpha(t,\,u)) \|^2,
\]
and thus if $\alpha(t,\,u)$ never enters $V^a$, we would have
\[
J(\alpha(t,\,u)) < J(u) - T \delta^2 < a - a' = \inf_{u \in \M} J(u),
\]
which is impossible. Now let $t_0 = 0 < t_1 < \dots < t_{n-1} < t_n = T$ be such that
\[
|t_i - t_{i-1}| \leq \frac{d}{2}.\]
Given $p \in \M^a$, there must be $\bar{t} \in [0,\,T]$ such that $\alpha(\bar{t},\, p) \in V^a$ and an index $i$ for which $|\bar{t}- t_i| \leq \frac{d}{2}$. Clearly
\[
\alpha(t_i,\, u) \in U^a,
\]
so we can consider the sets
\[
A_i = \{ p \in \M^a \: : \: \alpha(t_i,\,p) \in U^a\}.
\]
By what we proved above, $\M^a \subseteq \bigcup_{i = 0}^n A_i$ and therefore
\[
\mathrm{cat}(\M^a) \leq \sum_{i=0}^n \mathrm{cat}(A_i,\,\M^a).
\]
Since $A_i$ can be deformed in $U^a$ via $\eta_i := \alpha(t_i,\, \cdot)$, we use the properties of the category to infer that
\[
\mathrm{cat}(\M^a) \leq \sum_{i=0}^n \mathrm{cat}(\eta_i(A_i),\,\M^a) \leq (n+1) \mathrm{cat}(U^a,\, \M^a) < \infty.
\]
\end{proof}

\bcor
If $J$ is also bounded from above on $\M$, then $\mathrm{cat}(\M)$ is finite.
\ecor

\bcor
Let $J$ be bounded from below on $\M$. Suppose that $\mathrm{cat}(\M) = \infty$ and that $\mathrm{(PS)_a}$ holds for all $a < \sup_{u \in \M} J(u)$. Then
\[
c_m \xrightarrow{m \to \infty} \sup_{u \in \M} J(u),
\]
and hence $J$ has infinitely many critical points.
\ecor

\brmk[Relative category]\index{relative category}
Suppose that $A, \, Y \subseteq \M$ are closed. We define the relative category $\mathrm{cat}_{\M,Y}(A)$ as the least integer $k$ such that
\[
A \subseteq \bigcup_{i = 0}^k A_i,
\]
where $A_i$ is closed and contractible for all $i = 1,\, \dots, \, k$ in $\M$ and there exists a homotopy $h \in C([0,\,1]\times A_0,\, \M)$ satisfying
\[
h(0,\, \cdot) = \mathrm{id}_{A_0}, \qquad h(1,\, \cdot) \in Y \quad \text{and} \quad h(t, \, \cdot) \, \big|_Y \in Y.
\]
If $Y$ is empty, then the definition coincides with the one of category of $A$ in $\M$.

\bthm
If $J$ satisfies the Palais-Smale condition for all $c \in [a,\,b]$ then there are at least $\mathrm{cat}_{\M^b,\, \M^a}(\M^b)$ critical points in the energy strip $\overline{\M^b \setminus\M^a}$.
\ethm
\ermk

\section{Application to PDEs theory}

Let $\Omega$ be a bounded smooth subset of $\R^n$ and consider the Dirichlet-boundary problem
\[
\begin{cases} - \Delta u = \lambda u + f(u) & \text{if $x \in \Omega$},
\\[.6em] u = 0 & \text{if $x \in \partial \Omega$}, \end{cases}
\]
where $\lambda \in (\lambda_k,\, \lambda_{k+1})$ for some $k \geq 2$. Assume that $f$ is continuous and satisfies the following properties: \mbox{}
\begin{enumerate}[label={\color{magenta}(\arabic*)}]
\item $f$ is subcritical at $t = 0$, which means that $f(t) = \mathcal{O}(|t|^\alpha)$ for some $\alpha > 1$.
\item If $F(u) := \int_0^u f(s) \, \dr s$, then
\[
\lim_{|t|\to \infty} \frac{F(t)}{t^2} = - \infty.
\]
\item The function $t \mapsto f(t)$ is nonincreasing (thus $F(t) \leq 0$ for all $ \in \R$) and $F(t) = 0$ if and only if $t = 0$.
\end{enumerate}

\bthm
Under these assumptions, the Dirichlet-boundary problem admits at least two solutions.
\ethm

\begin{proof}Let $\X := H_0^1(\Omega)$ and let us consider the associated functional
\[
J(u) := \frac{1}{2} \int_\Omega |\nabla u|^2 \, \dr x - \frac{\lambda}{2} \int_\Omega |u|^2 \, \dr x - \int_\Omega F(u) \, \dr x.
\]
The assumption {\color{magenta}$(2)$} tells us that $- F(u) \geq Mu^2$ for all $M \in \R$ when $u$ is sufficiently big, while in the complement (which is compact) we can always find a constant $C_M > 0$ such that $F(u) \leq C_M$. It follows that
\[
-F(u) \geq M u^2 - C_M \quad \text{for all $u \in \R$}.
\]
We plug this inequality into $J$ and find that
\[
J(u_n) \geq \frac{1}{2} \int_\Omega |\nabla u_n|^2 \, \dr x + \left( M - \frac{\lambda}{2} \right) \int_\Omega |u_n|^2 \, \dr x - C_M |\Omega|
\]
and this goes to $\infty$ as $\|u_n\|_\X \to \infty$ since we can always pick $M > \frac{\lambda}{2}$. In particular, the functional $J$ is coercive on $\X$ and hence it satisfies the Palais-Smale condition at all levels\footnote{This assertion is not trivial, but one can show that coercivity gives the bounedness of Palais-Smale sequences and the subcriticality of $f$ allows one to write $\nabla J = \mathrm{Id} + \nabla \Phi$, where $\nabla \Phi$ is a compact operator.}.

\brmk
The $\Psi$-gradient decreases the value of $J$, so it is not restrictive to apply the min-max theory to a suitable sublevel $\X^a$. We introduce this apparent complication because we can collect more topological information as if we were in $\R^n$.
\ermk

We now substitute $\X$ with the sublevel $\X^{<0} := \{ u \in \X \: : \: J(u) < 0\}$, which is easy to see that it is nonempty using {\color{magenta}$(1)$}:
\[
J(t \varphi_1) \simeq_{t \to 0^+} \frac{1}{2} \underbrace{\left( \lambda_1 - \lambda \right)}_{< 0} t^2 + \mathcal{O}(|t|^{\alpha + 1}),
\]
where $\varphi_1$ is an eigenfunction of the first eigenvalue $\lambda_1$. Now notice that $C_1$ is nonempty and hence $c_1 < 0$ (since we are working in $\X^{<0}$) is a critical level and
\[
\mathcal{Z}_{c_1} \neq \varnothing.
\]
We claim that $C_2$ is nonempty. Let $V := \mathrm{Span}\langle \varphi_1,\, \dots, \, \varphi_k\rangle$ and, for $r$ small enough, notice that
\[
\sup_{S_r \cap V} J(u) < 0.
\]
If we can prove that the category of $S_r \cap V$ in $\X^{<0}$ is bigger than or equal to $2$, we will be able to conclude that $C_2 \neq \varnothing$. Let $\pi_V : \X \to V$ be the projection and let
\[
\pi_r(u) := r \frac{\pi_V(u)}{\| \pi_V(u) \|_\X}
\]
be the normalized projection which is the identity on $S_r \cap V$. Suppose that $\mathrm{cat}(S_r \cap V, \, \X^{<0}) = 1$ and let $A \supseteq S_r \cap V$ be the closed contractible set such that
\[
H(0,\, \cdot) \, \big|_A \equiv \mathrm{Id}_A \quad \text{and} \quad H(1,\, \cdot) \equiv p \in \X^{<0}
\]
for some contraction $H$. The assumption {\color{magenta}$(3)$} gives us that $\pi_V(u) = 0$ if $J(u) \geq 0$, which means that $\pi_V(u) \neq 0$ for all $u \in \X^{<0}$ and the restriction
\[
\pi_r(u) \, \big|_{S_r \cap V}
\]
is well-defined. We can consider the composition
\[
\pi_r \circ H : [0,\,1] \times A \to S_r \cap V,
\]
which, restricted to $[0,\,1] \times S_r \cap V$, gives a retraction of a $(k-1)$-dimensional sphere to a point in itself, and this is a contradiction as the sphere is non-contractible in itself.
\end{proof}

\brmk
If $c_1 = c_2 < 0$, then there are infinitely many critical points at level $c$ since the category of $\mathcal{Z}_c$ is at least two.
\ermk

\brmk
If $\lambda \in (\lambda_1,\, \lambda_2)$, then $V = \varphi_1 \R$ and the same argument leads to $S_r \cap V \cong S^0 = \{ \pm q \}$. The sublevels become disconnected, but it is still true that
\[
\mathrm{cat}(\{ \pm q\},\, \X^{<0}) = 2.
\]
\ermk
\chapter{The Krasnoselski Genus} \thispagestyle{empty}

\section{Introduction}

Let $E$ be a infinite-dimensional Hilbert space. We say that a subset $\Omega \subset E$ is {\em symmetric}\index{symmetric set} if it is symmetric with respect to the origin of $E$, that is,
\[
u \in \Omega \implies - u \in \Omega.
\]
Let $\Gamma$ be the class of all the symmetric subsets $A \subseteq E \setminus \{0\}$ which are {\bf closed} in $E \setminus \{0\}$.

\bd[Genus]\index{Krasnoselski genus}
Let $A \in \Gamma$. The {\em genus} of $A$, denoted by $\gamma(A)$, is the least integer number $k \in \N$ such that there exists a continuous odd map $\Phi : A \to \R^k$ satisfying
\[
\Phi(x) \neq 0 \quad \text{for all $x \in A$}.
\]
If such a number does not exist, then we define $\gamma(A) := \infty$. Similarly, if $A$ is an empty set, then we can set $\gamma(A) := 0$. 
\ed

\brmk
The genus of $A$ can be equivalently defined as the least integer number $k \in \N$ such that there exists a continuous odd map
\[
\Phi : A \to \R^k \setminus\{0\}.
\]
The reason is that we can always extend such a map to a continuous one taking values in $\R^k$ using {\em Dugundij's theorem} (selecting the odd part only).
\ermk

\brmk \label{rmk:genussphere}
The definition of genus does not change if we require $\Phi$ to be a function with values in the sphere $\mathbb{S}^{k-1}$ instead of $\R^k \setminus \{0\}$ since we can compose with the projection
\[
\pi(x) := \frac{x}{|x|}.
\]
\ermk

\bl \index{genus!of a sphere}
Let $E = L^2(\R^d)$ and let $A = S_E(0, \, 1)$ be the unit sphere in $L^2$. Then
\[
\gamma(A) = + \infty. 
\]
\el

\begin{proof}
Let $k \in \N$ be any positive integer and suppose that
\[
\Phi : S_E(0, \, 1) \to \R^k 
\]
is a continuous odd map. The infinite-dimensional sphere contains the $n$-dimensional sphere $\mathbb{S}^n(0, \, 1) \subset \R^{n + 1}$ for all $n \in \N$. By {\em Borsuk-Ulam theorem} it follows that, for $n > k$,
\[
0 \in \Phi \left(\mathbb{S}^n(0, \, 1) \right) \implies 0 \in \Phi\left(S_E(0, \, 1) \right).
\]
Since $S_E(0, \, 1)$ contains every finite-dimensional sphere, for every $k \in \N$ we can take $n = k + 1$ and obtain that $0$ is in the image. This shows that the genus is $+ \infty$.
\end{proof}

\brmk
In a similar fashion, one proves that $\gamma(\partial \Omega) = n$, where $\Omega \subset \R^n$ is an open bounded even subset such that $0 \in \Omega$. In particular,
\[
\gamma(\mathbb{S}^{n-1}) = n.
\]
\ermk

\begin{proof}
It is easy to verify that $\gamma \left( \partial \Omega \right) \leq n$. On the other hand, if
\[
\Phi : \partial \Omega \subseteq \R^n \longrightarrow \R^k
\]
is a continuous odd map, then {\em Borsuk-Ulam theorem} implies that $0 \in \Phi \left( \partial \Omega \right)$ for every $k < n$. It follows that
\begin{equation*}\gamma \left( \partial \Omega \right) \geq N \implies \gamma \left( \partial \Omega \right) = N. \end{equation*}
\end{proof}

\bl
The following properties hold: \mbox{}
\begin{enumerate}[label=\textbf{(\alph*)}]
\item If $A \in \Gamma$ is finite and nonempty, then $\gamma(A) = 1$.
\item If $A \subseteq \R^n$ and $0 \notin A$, then $\gamma(A) \leq n$.
\item If $0 \in A$, then $\gamma(A) = + \infty$.
\end{enumerate}
\el

\bpr\label{propgenus} Let $A$ and $B$ be elements of the class $\Gamma$. \mbox{}
\begin{enumerate}[label=\textbf{(\alph*)}]
\item The set $A$ is empty if and only if the genus $\gamma(A)$ is equal to $0$.
\item If $\Phi : A \to B$ is a continuous odd map, then $\gamma(A) \leq \gamma(B)$. In particular,
\[
A \subseteq B \implies \gamma(A) \leq \gamma(B).
\]
\item The genus is subadditive, that is,
\begin{equation} \label{genussub} \gamma (A \cup B) \leq \gamma(A) + \gamma(B). \end{equation}
\item There is an open neighborhood $U \supset A$ satisfying the following properties:
\mbox{}
\begin{enumerate}[label=\textbf{(\arabic*)}]
\item The set is symmetric, that is, if $u \in U$, then $- u \in U$.
\item The origin is not contained in the closure of the set, that is, $0 \notin \bar{U}$.
\item The genus coincides with the one of the set $A$, that is,
\[
\gamma \left( \bar{U} \right) = \gamma(A).
\]
\end{enumerate}
\end{enumerate}
\epr

\begin{proof} The first property is obvious. \mbox{}
\begin{enumerate}[label=\textbf{(\alph*)}]
\setcounter{enumi}{1}
\item If $\Psi : B \to \R^k \setminus \{0\}$ is a continuous odd map, then the composition
\[
\Psi \circ \Phi : A \to \R^k \setminus \{0\}
\]
is still continuous and odd. It follows that $\gamma(A) \leq \gamma(B)$.
\item Let $k, \, h \in \N$ be the least positive integers such that there are continuous odd maps $\Phi_1 : A \longrightarrow \R^k \setminus \{0\}$ and $\Phi_2 : B \longrightarrow \R^h \setminus \{0\}$ respectively. Let
\[
\widetilde{\Phi_i} : A \cup B \to \R^k
\]
be the continuous odd extensions of $\Phi_1$ and $\Phi_2$ respectively to $A \cup B$. Then
\[
\Psi(u) := \left(\widetilde{\Phi_1}(u), \, \widetilde{\Phi_2}(u) \right) : A \cup B \to \R^k \times \R^h \setminus \{(0,\,0)\}
\]
is a continuous odd map. Moreover, every point $u \in A \cup B$ belongs to either $A$ or $B$ so its image cannot be equal to $(0,\,0)$.
\item Let $k = \gamma(A)$. By \hyperref[rmk:genussphere]{Remark \ref{rmk:genussphere}} there exists a continuous odd map
\[
\Phi : A \to S^{k-1}.
\]
The set $A$ is closed in $E$ and hence there exists a continuous odd function $\widetilde{\Phi} : E \to \R^k$, which extends $\Phi$, but, a priori, $0$ may be in its image. Thus
\[
U_A := \left\{ u \in E \: \left| \: \left| \widetilde{\Phi}(u) \right| > \frac{1}{2} \right. \right\}
\]
is the desired neighborhood of $A$.
\end{enumerate} \end{proof}

\section{Genus in calculus of variations}

Suppose that $\X \subset E$, $\X$ Hilbert or $C^1$-submanifold, belongs to $\Gamma$. In this section, unless otherwise stated, every functional $J : \X \to \R$ is even and of class $C^1(\X, \, \R)$.

\bpr
Let $a < b$ be real numbers. Assume that $f : \X \to \R$ satisfies $\left( \mathrm{PS} \right)_c$ at every level $c \in [a, \, b]$. If there is a strict inequality
\[
\gamma \left(\X^a \right) < \gamma\left(\X^b \right),
\]
then there exists a critical value $c \in [a, \, b]$ for $f$.
\epr

\begin{proof}We argue by contradiction. If there are no critical values in $[a, \, b]$, then there is an odd retraction
$r: \X^{b} \to \X^{a}$, and we conclude using \hyperref[propgenus]{Proposition \ref{propgenus}}.
\end{proof}

\begin{notation}Let $k \in \N$ be a positive integer number such that $1 \leq k \leq \gamma(\X)$. We denote by $\gamma_k$ the infimum of all the sublevels such that the genus is at least $k$, that is,
\[
\gamma_k := \inf \left\{ c \in \R \: \left| \: \gamma \left( \X^c \right) \geq k \right. \right\}.
\]
It is possible that $\gamma \left( \X^c \right) \geq k$ is not satisfied for any real number $c \in \R$. In this case, the supremum of $J$ is $\infty$ and we set $\gamma_k = \infty$.
\end{notation}

\bl
\label{gen:fl}Let $1 \leq k \leq \gamma(\X)$. \mbox{}
\begin{enumerate}[label=\textbf{(\alph*)}]
\item The sequence is increasing, that is,
\[
\inf_{u \in \X} J(u) = \gamma_1 \leq \gamma_2 \leq \dots \leq \gamma_k \leq \sup_{u \in \X} J(u).
\]
\item If $\gamma_k \in \R$ and $J$ satisfies $\left( \mathrm{PS} \right)_{\gamma_k}$, then $\gamma_k$ is a critical value for the functional $J$. In particular,
\[
\gamma_1 \in \R \implies \gamma_1 = \min_{u \in \X} J(u).
\]
\item If $\gamma_k = \gamma_{k+1} = \dots = \gamma_{k + h}$ for some $h \geq 1$ and $f$ satisfies $\left( \mathrm{PS} \right)_{\gamma_k}$, then
\[
\gamma \left( \mathcal{Z}_{\gamma_k} \right) \geq h + 1,
\]
where $\mathcal{Z}_{\gamma_k}$ is the set of all singular points of $J$ at the level $\gamma_k$. In particular, if $0 \in \mathcal{Z}_{\gamma_k}$, then it is an infinite set.
\end{enumerate}
\el

\begin{proof}\mbox{}
\begin{enumerate}[label=\textbf{(\alph*)}]
\item The first identity follows from the fact that
\[
\gamma_1 := \inf \left\{ c \in \R \: \left| \: \gamma \left( \X^c \right) \geq 1 \right. \right\} = \inf \left\{ c \in \R \: \left| \: \X^c \neq \varnothing \right. \right\} = \inf_{u \in \X} J(u).
\]
Now notice that
\[
\left\{ c \in \R \: \left| \: \gamma \left(\X^c \right) \geq k \right. \right\} \supseteq \left\{ c \in \R \: \left| \: \gamma \left( \X^c \right) \geq k+1 \right. \right\},
\]
from which it follows that $\gamma_k \leq \gamma_{k+1}$ by taking the infimum of both sides.
\item If $\gamma_k$ is not a critical level for $J$, then there are $\delta > 0$ and
\[
r : \X^{\gamma_k + \delta} \to \X^{\gamma_k - \delta}
\]
odd retraction. By \hyperref[propgenus]{Proposition \ref{propgenus}} we have
\[
\gamma\left( \X^{\gamma_k + \delta} \right) \leq \gamma \left( \X^{\gamma_k - \delta} \right),
\]
but this is impossible since
\[
\gamma\left( \X^{\gamma_k - \delta} \right) \leq k -1 < k \leq \gamma \left( \X^{\gamma_k + \delta} \right).
\]
\item First, notice that $\mathcal{Z}_c$ is always a compact element of $\Gamma$. Thus by \hyperref[propgenus]{Proposition \ref{propgenus}} there exists a symmetric open neighborhood $U$ of $\mathcal{Z}_{\gamma_k}$ such that
\[
\bar{U} \in \Gamma \quad \text{and} \quad \gamma\left( \bar{U} \right) = \gamma \left( \mathcal{Z}_{\gamma_k} \right). 
\]
Then there are $\epsilon > 0$ and an odd retraction $r : \X^{\gamma_k + \epsilon} \setminus U \to \X^{\gamma_k - \epsilon}$, where the domain is closed, belongs to $\Gamma$ and it satisfies the inclusion
\begin{equation} \label{123123} \X^{\gamma_k + \epsilon} \subseteq \left(\X^{\gamma_k + \epsilon} \setminus U \right) \cup \bar{U}. \end{equation}
By assumption $k + h \leq \gamma \left( \X^{\gamma_k + \epsilon} \right)$, and by the subadditivity of the genus, it follows from \eqref{123123} that
\[
\begin{aligned}k + h & \leq \gamma \left( \X^{\gamma_k + \epsilon} \right) \leq
\\[1em] & \leq \gamma \left( \X^{\gamma_k + \epsilon} \setminus U \right) + \gamma \left( \bar{U} \right) \leq
\\[1em] & \leq \gamma \left( \X^{\gamma_k - \epsilon} \right) + \gamma \left( \bar{U} \right) \leq
\\[1em] & \leq k - 1 + \gamma\left( \mathcal{Z}_{\gamma_k} \right),
\end{aligned}\]
and this leads to the desired result.
\end{enumerate} \end{proof}

\bthm[Lusternik-Schnirelman]\index{Lusternik-Schnirelman theorem} \label{lussch}Let $J : \mathbb{S}^{n-1} \to \R$ be an even functional of class $C^1$. There are (at least) $n$ pairs of critical points for $J$ of the form
\[
(-u_i, \, u_i) \in \mathbb{S}^{n-1} \times \mathbb{S}^{n-1}.
\]
\ethm

\section{Application to nonlinear eigenvalues}

Let $\X$ be an infinite-dimensional Hilbert space and let $J\in C^1(\X)$ be an even functional satisfying the following assumptions: \mbox{}
\begin{enumerate}[label=(\roman*)]
\item $J(0) = 0$, $J(u) < 0$ for all $u \neq 0$ and $\sup_{u \in \X}J(u) = 0$.
\item $J$ is weakly continuous and $\nabla J$ is compact.
\item $\nabla J(u) \neq 0$ for all $u \in \X$.
\end{enumerate}

\bthm
Under these assumptions, the problem
\[
\nabla J(u) = \lambda u
\]
has infinitely many solutions $(\mu_k,\, z_k)$ with $z_k \in S := \{ u \in \X \: : \: \|u\|_2 = 1\}$ and $\mu_k \to 0$.
\ethm

\begin{proof}
To apply the general results with $S$, we need to prove that $J \, \big|_S$ is bounded below and $J$ satisfies the Palais-Smale condition at all $c < 0$.

\paragraph{Step 1.} This follows from the weak continuity of $J$. The reader might try to work out the details by herself as an exercise.

\paragraph{Step 2.} Let $u_n$ be a Palais-Smale sequence at the level $c < 0$. The weak continuity of $J$ implies (up to subsequences, which we ignore here) that
\[
u_n \rightharpoonup u \quad \text{and} \quad J(u) = c \implies u \neq 0.
\]
Now notice that $\nabla_\M J(u_k) = \nabla J(u_k) - \langle \nabla J(u_k),\,u_k \rangle u_k$ and
\[
\langle \nabla J(u_k),\, \nabla_\M J(u_k) \rangle = \| \nabla J(u_k) \|^2 - \left[ \langle \nabla J(u_k),\,u_k \rangle \right]^2,
\]
and by compactness of the gradient we have $\nabla J(u_k) \to \nabla J(u)$ strongly. Then
\[
0 = \| \nabla J(u) \|^2 - \left[ \langle \nabla J(u_k),\,u_k \rangle \right]^2 \implies \langle \nabla J(u),\, u \rangle \neq 0,
\]
and since $\langle \nabla J(u_k),\,u_k \rangle \to \langle \nabla J(u),\,u \rangle$, we can find $k$ sufficiently large such that
\[
u_k = \frac{1}{\langle \nabla J(u_k),\,u_k \rangle} \left[ \nabla J(u_k) - \nabla_\M J(u_k) \right]
\]
is well-defined. This shows that $u_k \to u$ strongly and concludes the proof.

\paragraph{Step 3.} Finally, $\gamma(S) = \infty$ implies that there are $z_k \in S$ critical points such that
\[
J(z_k) \to \sup_{u\in S} J(u) = 0.
\]
Since $z_k$ is a constrained critical point, we can always find $\mu_k$ such that $\nabla J(z_k) = \mu_k z_k$, and clearly it is given explicitly by
\[
\mu_k = \langle \nabla J(z_k),\,z_k\rangle.
\]
Finally $\nabla J(z_k)$ converges strongly to zero and $z_k$ weakly to zero, so $\mu_k \to 0$ and this concludes the proof.
\end{proof}

\bthm
Let $f$ be a Carathéodory function which is odd with respect to the second variable and that satisfies the $p$-growth
\[
|f(x,\,s)| \leq a + b |s|^p,
\]
where $1 < p < \frac{n+2}{n-2}$. Then the problem
\[
\begin{cases} - \lambda \Delta u = f(x,\,u) & \text{if $x \in \Omega$}, \\[.6em] u = 0 & \text{if $x \in \partial \Omega$}, \end{cases}
\]
has infinitely many solutions $(\mu_k,\, z_k)$ with $z_k \in S := \{ u \in \X \: : \: \|u\|_2 = 1\}$ and $\mu_k \searrow 0$.
\ethm

\section{Multiple critical points of even unbounded functionals}

Let $E$ be a Hilbert space, $J \in C^1(E,\,\R)$ a functional and define
\[
E_+ := \{ u \in E \: : \: J(u) \geq 0 \}.
\]
We now introduce two assumptions on $J$ that allows us, in some sense, to bypass the unboundedness both from above and below. These are similar to the ones necessary for the MPT, but the second one is ``stronger'': \mbox{}
\begin{enumerate}[label={\color{darkjunglegreen}(\roman*)}]
\item There are positive constants $r,\, \rho > 0$ such that $J(u) > 0$ for all $u \in B_r \setminus \{0\}$ and $J(u) \geq \rho$ for all $u \in S_r$. Furthermore, $J(0) = 0$.
\item For any $m$-dimensional subspace $E^m \subset E$, $E^m \cap E_+$ is bounded.
\end{enumerate}

Let $E^\ast$ be the class of maps $h \in C(E,\, E)$ which are odd homeomorphisms such that $h(\bar{B}_1) \subset E_+$. Notice that
\[
h_r(u) := ru \implies h_r \in E^\ast,
\]
where $r$ is given by ({\romannumeral 1}), so the class we introduced is never empty. Define
\[ \begin{aligned}
& \mathcal{A} := \left\{ A \subseteq E \setminus \{0\} \: : \: \text{$A$ is closed and even} \right\},
\\[1em] & \Gamma_m := \left\{ A \in \mathcal{A} \: : \: \text{$A$ is compact and $\gamma(A \cap h(S)) \geq n$ for all $h \in E^\ast$}\right\}.
\end{aligned} \]

\bl \label{kemmas.2s}
Let $J \in C^1(E,\,\R)$ be an even functional that satisfies ({\romannumeral 1}) and ({\romannumeral 2}). Then the following properties hold: \mbox{}
\begin{enumerate}[label=(\arabic*)]
\item $\Gamma_m \neq \varnothing$ for all $m$;
\item $\Gamma_{m+1} \subset \Gamma_m$;
\item if $A \in \Gamma_m$ and $U \in \mathcal{A}$, with $\gamma(U) \leq q < m$, then $\overline{A \setminus U} \in \Gamma_{m-q}$;
\item if $\eta$ is an odd homeomorphism in $E$ such that $\eta^{-1}(E_+) \subset E_+$, then $\eta(A) \in \Gamma_m$ whenever $A \in \Gamma_m$.
\end{enumerate}
\el

\begin{proof} \mbox{}
\begin{enumerate}[label=(\arabic*)]
\item By ({\romannumeral 2}) there exists $R > 0$ such that
\[
E^m \cap E_p \subset \bar{B}_R \cap E^m =: B_R^m.
\]
We claim that $B_R^m \in \Gamma_m$. Let $h \in E^\ast$ and notice that $h(B_1) \subset E_+$ implies
\[
E^m \cap h(B_1) \subset B_R^m.
\]
It follows that $E^m \cap h(S) \subset S_R^m \cap h(S)$ and, since one has the inclusion $B_R^m \cap h(S) \subset E^m \cap h(S)$, we infer that
\[
B_R^m \cap h(S) = E^m \cap h(S).
\]
Since $h$ is an odd homeomorphism, then $E^m \cap h(B_1)$ is a symmetric neighbourhood $\Omega$ of the origin. It is also easy to check that
\[
\partial \Omega = \partial(E^m \cap h(B_1))
\]
is contained in $E^m \cap h(S)$. Then
\[
\gamma(B_R^m \cap h(S)) = \gamma(E^m \cap h(S)) \geq \gamma(\partial \Omega) = m,
\]
which means that $B_R^m \in \Gamma_m$.
\item This follows immediately from the monotonicity property of the genus.
\item The set $\overline{A \setminus U} \in \mathcal{A}$ is compact and satisfies the identity
\[
\overline{A \setminus U} \cap h(S) = \overline{A \cap h(S) \setminus U}.
\]
Using the properties of the genus we infer that
\[ \begin{aligned}
\gamma \left(\overline{A \setminus U} \cap h(S) \right) & = \gamma \left( \overline{A \cap h(S) \setminus U} \right)
\\[1em] & \geq \gamma(A \cap h(S)) - \gamma(U) \geq m - q,
\end{aligned} \]
and this concludes the proof.
\item Let $A \in \Gamma_m$ and notice that $A' := \eta(A)$ is also compact. Our goal is to prove that for all $h \in E^\ast$ it turns out that
\[
\gamma(A' \cap h(S)) \geq m.
\]
It is easy to verify that $A' \cap h(S) = \eta \left[ A \cap \eta^{-1}(h(S)) \right]$. Since $\eta^{-1}(E_+) \subset E_+$, we infer that $\eta^{-1} \circ h$ belongs to $E^\ast$ and hence
\[ \begin{aligned}
\gamma \left(A' \cap h(S) \right) & = \gamma \left(\eta \left[ A \cap \eta^{-1}(h(S)) \right] \right)
\\[1em] & \geq \gamma \left( A \cap \eta^{-1}( h(S)) \right) \geq m.
\end{aligned} \]
\end{enumerate}
\end{proof}

\brmk
The condition $\eta(E_+) \subset E_+$ is natural if one thinks that deformations $\eta$ are usually induced by the $\Psi$-gradient flow.
\ermk

\bthm
Let $\Gamma_m$ be as above and set $b_m := \inf_{A \in \Gamma_m} \max_{u \in A} J(u)$. Suppose that $J \in C^1(E)$ satisfies ({\romannumeral 1}) and ({\romannumeral 2}). \mbox{}
\begin{enumerate}[label=(\arabic*)]
\item For all $m \in \N$ it turns out that $b_{m+1} \geq b_m \geq \rho > 0$.
\item If the Palais-Smale condition holds at the level $b_m$, then $b_m$ is critical.
\item If the Palais-Smale condition holds at all levels $c > 0$ and $b = b_m = \cdots = b_{m+q}$ for some $q \geq 1$, then
\[
\gamma \left( \mathcal{Z}_b \right) \geq q.
\]
\end{enumerate}
\ethm

\begin{proof}\mbox{}
\begin{enumerate}[label=(\arabic*)]
\item Let $r$ be given by ({\romannumeral 1}) and let $h_r \in E^\ast$ be the map defined above. If $A \in \Gamma_m$, then
\[
\gamma(A \cap h(S)) \geq m \quad \text{for all $h \in E^\ast$}
\]
and, since $h_r \in E^\ast$, we must have $A \cap S_r \neq \varnothing$ which means that $b_m \geq \rho$ for all $m \in \M$.
\item This assertion is proved in the usual way.
\item By the properties of the genus, we know that there exists an open neighbourhood $U$ of $\mathcal{Z}_b$ such that $\gamma(U) = \gamma(\mathcal{Z}_b)$. Recall that we can always find an odd deformation $\eta$ such that $\eta^{-1}(E^+) \subset E^+$ and a positive $\delta$ such that
\[
J(\eta(u)) \leq b - \delta \quad \text{for all $u \in J^{b+\delta} \setminus U$}.
\]
By definition of $b_{m+q}$, there is $A \in \Gamma_{m+q}$ with $\sup_A J(u) < b + \delta$. We proved above that $\overline{A \setminus U}$ also belongs to $\Gamma_{m+q}$ and thus
\[
A' := \eta \left(\overline{A \setminus U} \right) \in \Gamma_{m+q - q} = \Gamma_m.
\]
This leads to a contradiction because $\eta(A') \subset J^{b-\delta}$ and the genus of $J^{b-\delta}$ is necessarily strictly less than $m$.
\end{enumerate}
\end{proof}

\subsection{Application to nonlinear problems}

Let $\Omega \subset \R^n$ be a bounded set and consider the problem
\[ \begin{cases}
- \Delta u = f(x,\,u) & \text{if $x \in \Omega$},
\\[.6em] u = 0 & \text{if $x \in \partial \Omega$}.
\end{cases} \]
Suppose that $f$ is a function of with respect to the second variable which satisfies the Carathéodory condition and the $p$-growth condition
\[
|f(x,\,u)| \leq a + b|u|^p,
\]
where $1 < p < \frac{n+2}{n-2}$. Suppose also that there are $\lambda < \lambda_1(\Omega)$ such that
\[
f(x,\,u) = \lambda u + \mathcal{O}(|u|^{1+\alpha})
\]
for some $\alpha > 1$ and $\theta \in (0,\,\frac{1}{2})$ for which
\[
F(x,\,u) \leq \theta u f(x,\,u) \quad \text{for $|u| \geq r$}.
\]

\brmk
The latter condition implies that
\[
F(x,\,u) \geq c |u|^{\frac{1}{\theta}} + c',
\]
where $\frac{1}{\theta}$ is always strictly bigger than $2$.
\ermk

\begin{proof}[Proof of ({\romannumeral 2})]
Consider the functional
\[
J(u) = \frac{1}{2} \int_\Omega |\nabla u|^2 \, \dr x - \int_\Omega F(x,\,u) \, \dr x,
\]
let $H^m$ be a $m$-dimensional subspace and notice that $H^m \cap S$ is compact in $H_0^1(\Omega)$. We claim that there exists a positive $\delta := \delta(H^m)$ such that
\[
\left|\{ x \in \Omega \: : \: |u(x)| \geq \delta \right| \geq \delta \quad \text{for all $u \in H^m \cap S$}.
\]
If this were not true, then we could find a sequence $\delta_n \to 0$ and a sequence $u_n \in H^m \cap S$ such that
\[
\left|\{ x \in \Omega \: : \: |u_n(x)| \geq \delta_n \right| \leq \delta_n \quad \text{for all $u \in H^m \cap S$}.
\]
But then $u_n$ would converge to $0 \in S$ in $H_0^1(\Omega)$ and this is absurd because $u_n$ has norm equal to one. Now notice that, if we set $\Omega_u := \{ |u| \geq \delta\}$, then
\[
J(tu) = \frac{t^2}{2} - \int_{\Omega_u}  F(x,\,u) \, \mathrm{d} x - \int_{\Omega \setminus \Omega_u} F(x,\,u) \, \mathrm{d} x  \leq \frac{t^2}{2} - |\Omega_u| (c |t \delta|^{\frac{1}{\theta}} + c') + c'' |\Omega|.
\]
Since the right-hand side goes to $- \infty$ as $t \to \infty$ (as $\frac{1}{\theta} > 2$) uniformly with respect to $u$, we immediately infer that ({\romannumeral 2}) holds.
\end{proof}

We can exploit the same argument in combination with linking-type results to infer the existence of infinitely many critical points for linking geometry of even functionals.

\paragraph{Setting.} Let $H = V \oplus W$ be a Hilbert space with $\dim V < \infty$ and $W = V^\perp$. Let $J \in C^1(H,\,\R)$ be an even functional that satisfies ({\romannumeral 2}) and the linking conditions \mbox{}
\begin{enumerate}[label={\color{magenta}(\alph*)}]
\item $J(0)= 0$;
\item there are $r,\, \rho > 0$ such that
\[
J(u) > 0 \quad \text{for all $u \in (B_r(0) \setminus \{0\}) \cap W$},
\]
and
\[
J(u) \geq \rho \quad \text{for all $u \in S_r \cap W$}.
\]
\end{enumerate}
Let $\cH := \{h \in C(H,\,H) \: : \: \text{$h$ odd homeomorphism s.t. $h(B_1)\subseteq H_+ \cup \bar{B}_r$}\}$ and let
\[
\widetilde{\Gamma}_m := \left\{ A \in \cA \: : \: \text{$A$ is compact and $\gamma(A \cap h(S)) \geq m$ for all $h \in \cH$} \right\},
\]
where $\cA$ is the class of closed even sets disjoint from $\{0\}$.

\bl
Under these assumptions, the following assertions hold: \mbox{}
\begin{enumerate}[label=(\alph*)]
\item $\widetilde{\Gamma}_m \neq \varnothing$ for all $m$.
\item $\widetilde{\Gamma}_{m+1} \subset \widetilde{\Gamma}_m$ for all $m$.
\item If $A \in \widetilde{\Gamma}_m$ and $U\subset A$ satisfies $\gamma(U) \leq q < m$, then $\overline{A \setminus U} \in \widetilde{\Gamma}_{m-q}$.
\item If $\eta$ is a odd homeomorphism such that $\eta \, \big|_{\{J \leq 0\}}$ is the identity and $\eta(H_+) \subseteq H_+$, then
\[
\eta(\widetilde{\Gamma}_m) \subseteq \widetilde{\Gamma}_m.
\]
\end{enumerate}
\el

\begin{proof}
The linking conditions {\color{magenta}(a)} and {\color{magenta}(b)} implies that, given $H^m$ finite-dimensional vector space, there exists $R > 0$ such that
\[
H_+ \cap H^m \subseteq \overline{B_R} \cap H^m =: B_R^m.
\]
Taking $R$ large enough, we can also require that $(H_+ \cup \overline{B_r}) \cap H^m \subseteq B_R^n$. By definition, we have the inclusion
\[
B_R^n \supseteq h(B_1) \cap H^m
\]
for all $h \in \cH$, which gives us a set that contains (in the interior) the origin in $H^m$ and whose boundary has genus greater than or equal to $m$. This shows $(a)$, while $(b)$ and $(c)$ are similar to \hyperref[kemmas.2s]{Lemma \ref{kemmas.2s}}. For $(d)$ we need to check that
\[
\eta^{-1}\left( h(B_1) \right) \subseteq H_+ \cap \overline{B_r}.
\]
Let $u \in B_1$, $\nu = \eta^{-1} \circ h(u)$ and $h \in \cH$. Then $\eta(\nu) = h(u) \in H_+ \cup \overline{B_r}$ and there are two possibility to consider. If
\[
\eta(\nu) \in H_+,
\]
then $\nu \in H_+$. If, on the other hand, $\eta(\nu) \notin H_+$, then $\eta(\nu) = \nu \notin H_+$ using the property that $\eta$ is the identity where $J$ is nonpositive. In both cases
\[
\nu = \eta^{-1} \circ h(u) \in H_+ \cup \overline{B_r},
\]
and, since $\eta^{-1} \circ h$ is an odd homeomorphism, $\eta^{-1} \circ h \in \cH$ provided that $h \in \cH$.
\end{proof}

\bthm
Let $\Gamma_m$ be as above and set $\tilde{b}_m := \inf_{A \in \widetilde{\Gamma}_m} \max_{u \in A} J(u)$. Suppose that $J \in C^1(E)$ satisfies {\color{magenta}(a)}, {\color{magenta}(b)} and ({\romannumeral 2}). \mbox{}
\begin{enumerate}[label=(\arabic*)]
\item For all $m \in \N$ it turns out that $\tilde{b}_{m+1} \geq \tilde{b}_m \geq \rho > 0$.
\item If the Palais-Smale condition holds at the level $\tilde{b}_m$, then $\tilde{b}_m$ is critical.
\item If the Palais-Smale condition holds at all levels $c > 0$ and $\tilde{b} = \tilde{b}_m = \cdots = \tilde{b}_{m+q}$ for some $q \geq 1$, then
\[
\gamma \left( \mathcal{Z}_{\tilde{b}} \right) \geq q.
\]
\end{enumerate}
\ethm

We can use this theorem to prove that the problem
\[ \begin{cases}
- \Delta u = \lambda u + |u|^{p-1} u & \text{if $x \in \Omega$},
\\[.6em] u = 0 & \text{if $x \in \partial \Omega$},
\end{cases} \]
admits infinitely many solutions $u_j$ with $J(u_j) \to \infty$ for all $\lambda \in \R$.

\part{Applications to Differential Geometry}

\chapter{Geodesics on Riemannian Manifolds} \thispagestyle{empty}

\section{Introduction}

Let $(M^n, g)$ be a compact Riemannian manifold. We first start by defining closed curves $\gamma : S^1 \to M$ that belong to the Sobolev class $H^1(S^1,\, M)$. Recall that
\[
C^\infty(S^1,\,M) \subset H^1(S^1,\, M) \subset C(S^1,\,M),
\]
and closed curves in the smaller space are well-defined. We say\index{closed curve!Sobolev regularity} that $\gamma \in H^1(S^1,\,M)$ if $\gamma$ is {\em absolutely continuous} and
\[
\int g(\dot{c}, \dot{c}) < \infty.
\]
It is easy to verify that $H^1(S^1,\,M)$ is a Hilbert manifold\index{Hilbert manifold} (i.e., a separable topological manifold modeled on a Hilbert space rather than a Euclidean space). The manifold structure is induced by charts of the form
\[
\bar{c} \in C^\infty(S^1,\,M) \leadsto \bar{c}^\star ( TM ),
\]
where $TM$ is the tangent bundle of $M$ and $\bar{c}^\star$ is the pullback via $\bar{c}$. Given $c \in H^1(S^1,\,M)$, we can always find $\bar{c}\in C^\infty(S^1,\,M)$ and $X \in H^1(S^1,\,TM)$ such that
\[
c(t) = \exp_{\bar{c}(t)} X(t)
\]
since Sobolev-regular curves can always be approximated in $L^\infty$ via smooth ones. Furthermore, if $\varphi_{\bar{c}}$ is the above chart and $\bar{d}$ is a smooth curve close to $c$ (in $L^\infty$), then
\[
\varphi_{\bar{d}} \circ \varphi_{\bar{c}}^{-1}
\]
is a diffeomorphism between Hilbert spaces, which gives the differential structure of the manifold.

\bthm
The inclusion $H^1(S^1,\,M) \hookrightarrow C(S^1,\,M)$ is a homotopy equivalence.
\ethm

\paragraph{Tangent vectors.} Let $c(t) = \exp_{\bar{c}(t)} X(t)$ be a curve, $X$ section of class $H^1$, and consider
\[
c_\epsilon(t) = \exp_{\bar{c}(t)} \left( X(t) + \epsilon Y(t) \right).
\]
Then
\[
\frac{\mathrm{d}}{\mathrm{d}\epsilon} \, \Big|_{\epsilon = 0} c_\epsilon(t)
\]
is a tangent vector to $c(t)$, which means that $Y$ is a vector field along the curve $c$. We can thus define $T_c H^1(S^1,\,M)$ as the set of all vector fields $Y$ along $c$ such that
\[
\int g_{c(t)}(Y, Y) < \infty \quad \text{and} \quad \int g_{c(t)}(\nabla_{\dot{c}}Y, \nabla_{\dot{c}}Y) < \infty
\]

\bd
Let $c \in \Lambda(M)$ be a curve. The {\em energy}\index{energy of curves} is defined by
\[
E(c) := \frac{1}{2} \int_{S^1} g_{c(t)}(\dot{c}(t),\dot{c}(t)) \, \mathrm{d}t.
\]
\ed

\bthm
The functional $E$ is $C^1$ over $\Lambda(M)$ and it satisfies the Palais-Smale condition at all levels. Furthermore,
\[
\mathrm{d}E(c)[Y] = \int_{S^1} g_{c(t)}( \dot{c}(t), \nabla_{\dot{c}(t)} Y(t)) \, \mathrm{d}t
\]
and, if $c$ and $Y$ are smooth, then integrating by parts
\[
\mathrm{d}E(c)[Y] = - \int_{S^1} g_{c(t)}(\nabla_{\dot{c}(t)} \dot{c}(t), Y(t)) \, \mathrm{d}t.
\]
\ethm

\brmk
By regularity theory, critical points are smooth geodesics.
\ermk

\section{Critical points}

\bpr
There exists $\epsilon(M,g) := \epsilon > 0$ such that the only critical points $c$ of $E$ with energy $E(c) \leq \epsilon$ are constant curves. Moreover
\[
\{ E \leq 0 \}
\]
is a deformation of $\{ E \leq \epsilon \}$.
\epr

\begin{proof}[Hint]
For $\epsilon$ small, the length of the curve is $\sqrt{\epsilon}$ and thus small. The thesis is a consequence of Gauss lemma.
\end{proof}

To study critical points, we need to distinguish two cases since the fundamental group of $M$, $\pi_1(M)$, plays a critical role here. \mbox{}
\begin{enumerate}[label=(\roman*)]
\item If $\pi_1(M) \neq 0$, then $\Lambda(M)$ has a nontrivial component $\Theta$. We claim that
\[
c_\Theta := \inf_{c \in \Theta} E(c) > 0.
\]
This is a consequence of the result above because $\Theta$ is nontrivial and if $c_\Theta$ was equal to zero then the curve could be deformed to a trivial one (contradiction).

Since $E$ satisfies the Palais-Smale condition, we easily obtain a nontrivial geodesic at the level $c_\Theta$.
\item If $M$ is simply connected, we start by recalling a few facts in differential geometry.

\bthm
If $\pi_1(M) = 0$, then $\pi_k(\Lambda(M)) \cong \pi_k(M) \oplus \pi_{k+1}(M)$ for all $k \in \N$.
\ethm

\bpr
If $\pi_1(M) = 0$, then $\pi_k(M) \cong H_k(M)$ where $k$ is the least integer such that $\pi_k(M) \neq 0$.
\epr

As a consequence of these two facts, we can always find $k \in \N$ such that $\pi_k(\Lambda(M)) \neq 0$. Let $\Xi \subseteq \pi_k(\Lambda(M))$ and let $f \in \Xi$ be a nontrivial curve. Consider
\[
\cH = \{ h : S^k \to \Lambda(M) \: : \: \text{$h$ is homotopic to $f$} \}
\]
and
\[
c_f := \inf_{h \in \cH} \sup_{x \in S^k} E(h(x)) > 0,
\]
once again by contradiction. The Palais-Smale condition gives a nontrivial geodesic at the level $c_f$.

\end{enumerate}
\chapter{Allen-Cahn Energy} \thispagestyle{empty}

In 1977, Modica and Mortola considered the problem of {\em diffuse interfaces}. The {\em Allen-Cahn energy}, for example, describes metal alloys that are mixtures of two phases $\pm 1$. These alloys tend to form ``grais'' whose boundary evolves by the mean curvature flow (at least macroscopically). This means that the physical energy behaves at the main order like the area of the surface.

\section{Introduction}

The phases are described by a {\em double-well potential energy}\index{double-well potential energy}, namely a function $W(u)$ such that
\[
W(u) \simeq \frac{(1-u^2)^2}{4}.
\]
Minima of the functional $\int_\Om W(u) \, \dr x$ are perfectly separated phases, assuming everywhere the values $\pm 1$; more precisely,
\[
\min_{u} \int_\Omega W(u) \, \dr x = 0
\]
is attained by {\bf any} function $u$ that takes only the values $\pm 1$. However, functions of this type can be very ``wild'' and hence it makes sense to consider the slightly more regular functional:
\[
E_\epsilon(u) := \frac{\epsilon}{2} \int_\Omega |\nabla u|^2 \, \dr x + \frac{1}{\epsilon} \int_\Omega W(u) \, \dr x.
\]
The main idea (which can be proved with little effort) is that this functional $E_\epsilon$ tends to penalize oscillating functions.

\bex
Let $\Omega = \R$. Then critical points of $E_\epsilon(\cdot)$ satisfy the differential equation
\[
- \epsilon u'' + \frac{1}{\epsilon} W'(u) = 0.
\]
It can be proved that for $\epsilon$ small enough the transition between the phase $1$ and the phase $-1$ is smooth with order $\epsilon$. Now let $v \in H^1(\R)$ be a function satisfying the boundary conditions $v(a) = - 1$ and $v(b) = 1$ for some $a < b$. Then
\[
2xy \leq x^2 + y^2 \implies \frac{\epsilon}{2}(v')^2 + \frac{1}{\epsilon}W(v) \geq \sqrt{2 W(v)} v',
\]
which immediately leads to the energy estimate
\[
E_\epsilon(v) \geq 2 \int_{-1}^1 \sqrt{2 W(s)} \, \dr s =: C_W.
\]
The quantity $C_W$ is called {\em minimal transition energy}\index{minimal transition energy} and the equality holds if and only if
\[
v' = \sqrt{2 W(v)},
\]
but this is impossible (check!) if $a$ and $b$ are finite.
\eex

Before we can investigate what happens when $\Om$ is a subset of $\R^n$, $n \geq 2$, we need to recall a few definitions from geometric measure theory.

\bd[Caccioppoli Perimeter] \index{Caccioppoli Perimeter}
Let $E \subset \Om$ be a set. The perimeter of $E$ relative to $\Om$ is defined as
\[
\mathrm{Per}(E,\Om) := \sup_{\Phi \in C_c^\infty(\Om)} \int_E \divs \, \Phi \, \dr x.
\]
\ed

The next result holds even assuming that the boundary of $E$ is less regular (for example, Lipschitz should be enough), but for our purposes there is no need to go any further.

\bthm
Let $E \subset \Om$ be a set with smooth boundary. Then
\[
\mathrm{Per}(E,\Om) = | \partial E \cap \Om|.
\]
\ethm

\bthm[Modica-Mortola]
Let $\Om$ be an open subset of $\R^n$ and $E \subseteq \Om$ be a set with finite relative perimeter. Let $f_n$ be a sequence of functions such that
\[
\|f_n - g_E\|_{L^1(\Om)} \xrightarrow{n \to \infty} 0,
\]
where
\[
g_E(x) := \begin{cases} 1 & \text{if $x\in E$}, \\ -1 & \text{if $x \in \Om \setminus E$}. \end{cases}
\]
Then, given $\epsilon_n \to 0$, the following $\liminf$ inequality holds:
\[
\liminf_{n \to \infty} E_{\epsilon_n}(f_n) \geq C_W \mathrm{Per}(E,\Om).
\]
Moreover, there exists a sequence $f_n$ as above such that the $\limsup$ inequality also holds, namely
\[
\limsup_{n \to \infty} E_{\epsilon_n}(f_n) \leq C_W \mathrm{Per}(E,\Om).
\]
\ethm

\brmk
The result above can also be translated in terms of $\Gamma$-convergence as follows: the sequence of functionals $E_{\epsilon_n}(\cdot)$ $\Gamma$-converges to the functional $C_W \mathrm{Per}(\cdot,\Om)$.
\ermk

\section{Variational structure of $E_\epsilon$}

Let $(M,g)$ be a $n$-dimensional compact Riemannian manifold and let $u \in H^1(M)$. The Sobolev embedding
\[
H^1(M) \hookrightarrow L^{2^\ast}(M),
\]
where $2^\ast := \frac{2n}{n-2}$, implies that the integral
\[
\int_M W(u) \, \dr V
\]
is well-defined when $n \leq 4$\footnote{The case $n = 4$ is much more delicate since we lose the compactness of the embedding $H^1(M) \hookrightarrow L^{4}(M)$.}, but the same is not true when $n \geq 5$.

\bl \label{lemma.app.2.1}
Every solution of class $C^2(M)$ of the equation
\[
- \epsilon \Delta u = \frac{1}{\epsilon} W'(u)
\]
has the property $u(x) \in [-1,1]$ for all $x \in M$. Furthermore, unless $u$ is identically equal to either $1$, $-1$ or $0$, it has to change sign.
\el

\begin{proof}
Suppose that $\max_{x \in M} u(x) > 1$ and let $x_0$ denote a point where $u$ attains its maximum value. Then $W'(u) > 0$ in a neighbourhood $U$ of $x_0$ so we can consider a non-negative test function $\varphi$ supported in $U$. Then
\[
\dr E_\epsilon(u)[\varphi] = \int_U \varphi\left( - \epsilon \Delta u + \frac{1}{\epsilon} W'(u) \right) \, \dr V > 0,
\]
contradicting the minimality of $u$. If $\max_{x \in M} u(x) = 1$, by the maximum principle we would have $u$ identically equal to one so we can assume without loss of generality that $u$ is non-negative and not identically equal to zero. Then
\[
\dr E_\epsilon(u)[1] = \frac{1}{\epsilon} \int_\Om W'(u) \, \dr V < 0
\]
since $W'$ is negative in $(0,1)$. Since a similar argument holds for $u \leq 0$ (with $W'(u)$ being positive), we easily conclude that $u$ must change sign.
\end{proof}

We now define a slightly different potential energy which is subcritical, namely a function $W^\ast$ that satisfies the following properties: \mbox{}
\begin{enumerate}[label=(\roman*)]
\item $W^\ast(u) = W^\ast(-u)$ and $W^\ast(u) = Au^2$ in $[4, \infty)$ for some positive constant $A$;
\item $W^\ast \equiv W$ on $[-2,2]$ and $(W^\ast)' > 0$ in $[2,\infty)$.
\end{enumerate}
The energy with respect to this new potential is given by
\[
E_\epsilon^\ast(u) := \frac{\epsilon}{2} \int_M |\nabla u|^2 \, \dr V + \frac{1}{\epsilon} \int_M W^\ast(u) \, \dr V,
\]
and it is easy to see that its critical points are classical $C^2$ solutions of the equation
\[
- \epsilon \Delta u + \frac{1}{\epsilon} (W^\ast)'(u),
\]
which means that \hyperref[lemma.app.2.1]{Lemma \ref{lemma.app.2.1}} can be extended with no effort. Another advantage of using $E_\epsilon^\ast$ over $E_\epsilon$ is that the integral
\[
\int_M W^\ast(u) \, \dr V
\]
is always well-defined because outside of $(-4,4)$ the potential is quadratic.

\bpr
The functional $E_\epsilon^\ast : H^1(M) \to \R$ is coercive and belongs to $C^1$. Moreover, it satisfies the Palais-Smale condition at all levels.
\epr

\begin{proof}
The regularity follows from the general theory of Nemitski operators. The coercivity is also easy because we can always find a positive constant $c$ such that
\[
W^\ast(u) \geq \frac{1}{c} u^2 - c,
\]
from which it follows that
\[
E_\epsilon^\ast(u) \geq \min \left\{ \frac{\epsilon}{2}, \frac{1}{c \epsilon} \right\} \|u\|_{H^1(M)}^2 - \frac{c}{\epsilon} \cdot \mathrm{Vol}(M).
\]
The right-hand side goes to infinity as soon as $|u\|_{H^1(M)} \to \infty$ so $E_\epsilon^\ast$ is coercive. This implies (by a standard argument) that Palais-Smale sequences are bounded and everything follows as usual.
\end{proof}

\section{Mountain pass solutions}

Let $\Gamma = \{ \gamma : [0,1] \to H^1(M) \: : \: \gamma(0) = -1, \, \gamma(1) = 1 \}$ be the set of all admissible (continuous) curves and define the mountain-pass level
\[
c_\epsilon := \inf_{\gamma \in \Gamma} \sup_{t \in [0,1]} E_\epsilon^\ast(\gamma(t)).
\]
Since we would like to find a nontrivial solution at the limit (for $\epsilon \to 0^+$), the first step is proving that $c_\epsilon$ does not go to zero as $\epsilon$ does.

\bl \label{lemma.app1.1}
There exists $c > 0$ independent of $\epsilon$ such that $c_\epsilon \geq c$.
\el

To prove this lemma, we first need to present a technical result (of which we will only sketch the proof) due to De Giorgi.

\bl
Suppose that there are $a < b$ and $\delta > 0$ such that
\[
\min\left\{ |\{ u < a \}|, | \{ u < b \} | \right\} > \delta.
\]
Then there exists a constant $C = C(\delta,M)>0$ such that
\[
C(b-a) \leq \sqrt{ | \{ a \leq u \leq b \} | } \| \nabla u \|_{L^2(M)}.
\]
\el

\begin{proof}
Consider the {\em isoperimetrical profile}\index{isoperimetrical profile} defined by setting
\[
I(t) := \inf\{ \mathrm{Per}_M(\Om) \: : \: |\Om| = t, \, \Om \subseteq M\}.
\]
It can be proved that $I(t)$ is a continuous function which is even with respect to the point $\frac{\mathrm{Vol}(M)}{2}$ and strictly positive (except at $t=0$ and $t = \mathrm{Vol}(M)$). For $t \in (a,b)$ consider
\[
\Om_t := \{ u \leq t \},
\]
and notice that $\mathrm{Vol}(\Om_t) \in (\delta,\mathrm{Vol}(M)-\delta)$ so its isoperimetrical profile stays away from zero; namely, there exists a constant $C>0$ such that
\[
I(\mathrm{Vol}(\Om_t)) \geq C \quad \text{for all $t \in (a,b)$}.
\]
We now use the coarea formula
\[
\int_M f \, \dr V = \int_{\min u}^{\max u} \dr t \int_{ \{u = t \}} \frac{f}{|\nabla u|} \, \dr V
\]
with $f = |\nabla u|$ to infer that
\[
C(b-a) \leq \int_a^b \mathrm{Per}(\Om_t,M) \, \dr t = \int_{\{a \leq u \leq b\}} |\nabla u| \, \dr V.
\]
A simple application of H\"{o}lder inequality leads to the conclusion.
\end{proof}

\begin{proof}[Proof of Lemma \ref{lemma.app1.1}]
Suppose $c_\epsilon \to 0$ as $\epsilon \to 0^+$ and let $h \in \Gamma$ be such that
\[
\max_{t \in [0,1]}E_\epsilon^\ast(h(t)) \leq c_\epsilon + \epsilon.
\]
Select $t$ in such a way that $\int_M h(t) \, \dr V = 0$, $a \in (0,1)$  and let $c_a$ be a constant such that 
\[
W(u) \geq c_a > 0 \quad \text{on $[-a,a]$}.
\]
Notice that
\[
c_a \mathrm{Vol}(\{-a \leq u \leq a\}) \leq \epsilon(c_\epsilon + \epsilon)
\]
so the following two estimates hold.
\[
\begin{cases} 0 = \int_M u \, \dr V \leq a \mathrm{Vol}(\{u\geq a\}) -  a \mathrm{Vol}(\{u\leq -a\}) + \frac{\epsilon(c_\epsilon+\epsilon)}{c_a}, \\[.6em]
\mathrm{Vol}(M) \leq  \mathrm{Vol}(\{u\geq a\}) +  \mathrm{Vol}(\{u\leq - a\}) + \frac{\epsilon(c_\epsilon+\epsilon)}{c_a}. \end{cases}
\]
It follows that
\[
\mathrm{Vol}(\{ u \geq a\}) \geq \frac{a}{2} \mathrm{Vol}(M) - \frac{\epsilon(c_\epsilon+\epsilon)}{c_a} > \frac{a}{3} \mathrm{Vol}(M) =: \delta
\]
for $\epsilon$ small enough and, in a similar fashion, we can prove the same for $\mathrm{Vol}(\{ u \leq - a\})$ in place of $\mathrm{Vol}(\{ u \geq a\})$. Therefore, the exists a positive constant $c > 0$ such that
\[
0 < 2a c \leq \sqrt{\mathrm{Vol}(\{-a\leq u \leq a\})} \|\nabla u\|_{L^2(M)} \leq \sqrt{\frac{2}{c_a}} (c_\epsilon + \epsilon),
\]
which gives
\[
c_\epsilon + \epsilon \geq \frac{2ac}{\sqrt{2c_a^{-1}}} \implies c_\epsilon \geq \frac{2ac}{\sqrt{2c_a^{-1}}} ,
\]
a contradiction with $c_\epsilon \to 0$ as $\epsilon \to 0^+$.
\end{proof}

For the upper bound, we need to introduce a few definitions.

\bd
A {\em Morse function}\index{Morse function} is a function $f : M \to \R$ of class $C^2$ with finitely many non-degenerate critical points.
\ed

\brmk
Morse functions are dense in $C^2(M,\R)$.
\ermk

\bd
An {\em isotopy}\index{isotopy} on $M$ is a diffeomorphism which is homotopic to the identity map $\mathrm{id}_M$ via a family of diffeomorphisms.
\ed

\paragraph{Sweepoints.}\index{Sweepoints} Let $f : M \to [0,1]$ be a Morse function and define
\[
\Lambda := \left\{ \{ \Sigma_t = \Psi_t( f^{-1}(t),1) \}_{t \in [0,1]} \: : \: \Psi_t \in C^\infty([0,1], \mathrm{Isot}(M))\right\}.
\]
For $\{\Sigma_t\}_t \in \Lambda$, set
\[
F(\{\Sigma_t\}_t) := \max_{t \in [0,1]} |\Sigma_t|.
\]

\bd
The {\em width}\index{width} of $\Lambda$ is given by
\[
m_0(\Lambda) := \int_{\{\Sigma_t\}_t \in \Lambda} F(\{\Sigma_t\}_t).
\]
\ed

\bpr
The min-max level $c_\epsilon$ is bounded from above. More precisely, it turns out that
\[
\limsup_{\epsilon \to 0^+} c_\epsilon \leq C_W \cdot m_0(\Lambda).
\]
\epr

\begin{proof}
Let $d_{\Sigma_i}$ be the signed distance from $\Sigma_t$ and choose the sign in such a way that $d_{\Sigma_0} \geq 0$ and $d_{\Sigma_1} \leq 0$. Notice that
\[
|d_{\Sigma_t}| = 1 \quad \text{almost everywhere},
\]
$d_{\Sigma_t}$ is Lipschitz and $[0,1] \ni t \mapsto d_{\Sigma_t} \in H^1(M)$ is continuous. Let $v_0$ be the one-dimensional optimal profile solving the differential equation
\[
- v_0'' + W'(v_0) = 0.
\]
For $\epsilon, \delta > 0$ define
\[
v_{\epsilon,\delta}(x) := \begin{cases} v_0( \frac{d_\Sigma(x)}{\epsilon}) & \text{if $|d_\Sigma(x)| \leq \delta$},\\[.6em]  v_0\left( \frac{\delta}{\epsilon}  \frac{d_\Sigma(x)}{|d_\Sigma(x)|} \right) & \text{if $|d_\Sigma(x)| > \delta$}, \end{cases}
\]
where $d_{\Sigma} := d_{\Sigma_t}$ for some $t \in [0,1]$ so that $v_{\epsilon,\delta}$ also depends on $t$. The map $t \mapsto v_{\epsilon,\delta}$ is continuous from $[0,1]$ to $H^1(M)$ and, if $t = 0$ and $t = 1$, we have respectively $v_{\epsilon,\delta} \simeq 1$  and $v_{\epsilon,\delta} \simeq -1$. We would like these values to be constant so we need to adjust them; notice that\footnote{The argument for $\Sigma_1$ is the same modulo a few adjustments.} $\Sigma_0$ is a finite union of points and
\[
v_0 \geq 0
\]
by construction. Let $f_s^j := (1-s) + s v_j$ for $s \in [0,1]$ and notice that
\[
f_s^1 \ast v_t \ast f_s^0 =: h(t)
\]
is an admissible function for the minimum. We now claim that
\[
\max_{t\in[0,1]} E_\epsilon(h(t)) = \max_{t \in [0,1]} E_\epsilon(v_0(t)).
\]
We will now show that $\max_{s \in [0,1]} E_\epsilon(f_s) = E_\epsilon(v_0)$ implies the claim. Notice that $|\nabla f_s| \leq |\nabla v_0|$ and that
\[
v_0 = v_{\epsilon,\delta}(P,x).
\]
Since the distance $d_P$ does not change sign, we have $v_{\epsilon,\delta}(P,\cdot) \in [0,1)$, where $W$ is strictly decreasing. Then $W(f_s(x)) \leq W(v_{\epsilon,\delta}(P,x))$ and this implies that
\[
E_\epsilon(f_s) \leq E_\epsilon(v_{\epsilon,\delta}(\Sigma)) \leq E_\epsilon(v_0).
\]
Since the opposite inequality is trivially true by taking $s = 1$, we can now conclude the proof of the upper bound. We start by noticing that
\[
E_\epsilon(v_{\epsilon,\delta}(\Sigma)) = \frac{\epsilon}{2} \int |\nabla v_{\epsilon,\delta}(\Sigma,\cdot)|^2 + \frac{1}{\epsilon} \int W(v_{\epsilon,\delta}(\Sigma,\cdot)).
\]
At this point it makes sense to divide the integral interval using $|d_\Sigma| > \delta$ and $|d_\Sigma| < \delta$ since by definition we have
\[
\nabla v_{\epsilon,\delta}(\Sigma,x) = \frac{1}{\epsilon} v_0' \left( \frac{d_\Sigma(x)}{\epsilon} \right) \nabla d_\Sigma(x)
\]
if $|d_\Sigma| < \delta$ and zero otherwise. It follows that
\begin{equation} \label{eq....}
\int_{|d_\Sigma|>\delta} \dots \leq \frac{2}{\epsilon} \mathrm{Vol}_g(M) W\left( v_0 (\frac{\delta}{\epsilon}) \right).
\end{equation}
To compute the integral in the complement, taking into account that $|\nabla d_\Sigma|=1$, we use the coarea formula as follows:
\[ \begin{aligned}
\int_{|d_\Sigma|\leq \delta} \dots & = \int_{-\delta}^\delta \frac{1}{\epsilon} \left[ \frac{v_0'(s/\epsilon)^2}{2} + W(v_0(s/\epsilon)) \right] \cH^{n-1}\left( \{ d_\Sigma= s\}\right) \, \dr s
\\[1em] & = \int_{-\frac{\delta}{\epsilon}}^{\frac{\delta}{\epsilon}} \left[ \frac{v_0'(s)^2}{2} + W(v_0(s)) \right] \cH^{n-1}\left( \{ d_\Sigma= \epsilon s\}\right) \, \dr s.
\end{aligned} \]
It can be shown that for all $\eta > 0$ there exists $\delta_0 > 0$ such that
\[
\cH^{n-1} \left( \{ d_{\Sigma_t} =s\} \right) \leq (1+\eta) \cH^{n-1}(\Sigma_t) \quad \text{for all $|s|\leq \delta_0$ and all $t \in [0,1]$.}
\]
Therefore, we obtain that
\[
\int_{|d_\Sigma|\leq \delta} \dots \leq C_W(1+\eta) \cH^{n-1}(\Sigma)
\]
with $\eta \to 0$ as $\delta \to 0$. Since the convergence of $v_0$ to $\pm 1$ is exponential we have that $s W(v_0(s)) \to 0$ as $s \to \pm \infty$. Therefore the right-hand side in \eqref{eq....} tends to zero as $\epsilon \to 0$. So for all $\eta > 0$ there exists $\epsilon_0 > 0$ such that for all $\epsilon \in (0,\epsilon_0)$ we have
\[
c_\epsilon \leq 2 \sigma(1+\eta) \cF( \{\Sigma_t\}_t),
\]
which immediately implies
\[
\limsup_{\epsilon \to 0} c_\epsilon \leq C_W \cF( \{\Sigma_t\}_t).
\]
\end{proof}

\bd
Let $F : H \to \R$ be a map of class $C^2$, where $H$ is either a Hilbert space or a Hilbert manifold. Let $x_0 \in H$ be a critical point for $F$ such that the matrix
\[
\dr F^2 (x_0)
\]
is symmetric. The {\em Morse index}\index{Morse index} of $F$ at $x_0$ is the maximum dimension among all vector spaces $V \subseteq T_p M$ such that $\dr F^2 (x_0) \, \big|_V$ is negative definite.
\ed

\brmk
In most cases, for elliptic PDEs, the operator $\dr F^2 (x_0)$ is Fredholm (=compact perturbation of the identity) and hence its Morse index must be finite.
\ermk

\bthm
Suppose that the Palais-Smale condition holds and let $c_\epsilon$ be a min-max value of mountain pass type. Then there exists $x_0 \in \mathcal{Z}_{c_\epsilon}$ such that
\[
m(F,x_0) \leq 1.
\]
\ethm

\section{Convergence of interfaces}

Let $\epsilon_k \to 0$ and $u_k := u_{\epsilon_k}$ solution of the $\epsilon_k$-AC equation.

\bd
Let $U \subseteq M$ be an open set. We say that {\em $u_k$ is stable in $U$} if
\[
E_k'(u_k)[\varphi,\varphi] \geq 0 \quad \text{for all $\varphi \in C^1(U)$}.
\]
We also say that $u_k$ is {\em stable} if it is stable in $U = M$.
\ed

\bd
Let $U \subseteq M$ be an open set. A {\em varifold} is a finite Radon measure on the $(n-1)$-Grassmannian of unoriented planes $\mathcal{G}(U)$.
\ed

\brmk
The mass of a varifold is $\|v\|$, where
\[
\int_U \varphi(x) \, \dr \|v\|(x) = \int_{\mathcal{G}(U)} \varphi(x) \, \dr V(x,\pi).
\]
\ermk

\bex
If $\Sigma$ is a $(n-1)$-rectifiable set and $\theta : \Sigma \to Z$ a multiplicity map, then
\[
V_{\theta \Sigma}(\varphi) = \int \varphi(x,T_x \Sigma) \theta(x) \, \cH^{n-1}
\]
for all $\varphi \in C_c(\mathcal{G}(U))$.
\eex

Let $X$ be a smooth vector field on $U$ and denote by $\Psi_t$ the corresponding flow. Define the first variation of a varifold as
\[
[\delta V](x) := \frac{\dr}{\dr t} \, \big|_{t=0} \| (\Psi_t)_\ast V \|(U),
\]
and $V$ is {\em stationary} if $\delta V(x) = 0$ for all $x$.

\brmk
In this setting, let $\varphi(t) := \int_0^t \sqrt{2 W(s)} \, \dr s$ and let
\[
V_k(U) = \frac{2}{C_W} \int_\R V_{ \{ W_k = t \} }(U) \, \dr t,
\]
where $W_k := \varphi \circ u_k$.
\ermk

\bd
Let $V = V_{\theta \Sigma}$ be integer-rectifiable and stationary. Then $V$ has {\em optimal regularity} if $V = V_\Sigma$ and $\Sigma$ smooth embedded when $2 \leq n \leq 7$ or the same holds up to a set $S$ with $\cH^{n-8+\gamma}(S) = 0$ for all positive $\gamma$ when $n \geq 8$.
\ed

\bthm
Let $n \geq 2$. If $u_k$ are stable with $E_k(u_k) \leq c$, then $V_k \to V$ with $V$ stable stationary and with optimal regularity.
\ethm

\bthm
The same conclusion holds if $n \geq 3$ and $E_k(u_k) = \leq C$ and $m(E_k,u_k) \leq M$ (in place of stability).
\ethm

\bpr
If $m(E_k,u_k)\leq m$ and if $U_i$ are disjoint open sets for $i = 1,\dots,m+1$, then $u_k$ is stable in at least one of the $U_i$.
\epr

\brmk
For $n = 2$, the limit objects are geodesics with at most one self-intersection (so they might not be embedded).
\ermk

\brmk
Since $E_k(u) = E_k(-u)$, one can use genus theory to investigate multiplicity. For any $p \in \N$, there exists a min-max level $c_\epsilon$ (using sets of genus $\geq p$).
\ermk

\bthm
There exists $\tau(n)$ dimensional constant such that as $\epsilon \to 0$ it turns out that
\[
c_\epsilon(p) \to \ell_p(n),
\]
where $\ell_p(n) = \tau(n) \mathrm{Vol}_g(M)^{\frac{n}{n+1}} p^{\frac{1}{n+1}}$.
\ethm

\printindex

\bibliographystyle{siam}
\addcontentsline{toc}{chapter}{Bibliography} \markboth{Bibl}{}
\bibliography{bibs}{} % BIBLIOGRAFIA
\end{document}
