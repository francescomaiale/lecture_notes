\chapter{Special Orthogonal Group $\mathrm{SO}(4, \, \R)$} \thispagestyle{empty}

In this chapter, we study more in-depth the special unitary (Lie) group $\mathrm{SO}(4, \, \R)$ and the associated Lie algebra $\mathrm{so}(4, \, \R)$. Recall that the set of all orthogonal matrices
\begin{equation*} \mathrm{O}(N, \, \R) := \left\{ O \in \mathrm{GL}(N, \, \R) \: : \: O^T O = O O^T = \mathrm{Id}_{N \times N} \right\} \end{equation*}
is a group with respect to the matrix product. Similarly,
\begin{equation*} \mathrm{SO}(N, \, \R) := \left\{ O \in \mathrm{GL}(N, \, \R) \: : \: O^T O = O O^T = \mathrm{Id}_{N \times N}, \: \mathrm{det}(O) = 1 \right\} \end{equation*}
is also a group, and it is called \textit{special orthogonal group}.

\section{Representations of $\mathrm{SO}(4, \, \R)$}

The six generators of the $4$-dimensional special orthogonal group can be computed explicitly in the representation $\underline{4}$, and they are given by the following matrices:
\begin{equation*}\begin{aligned} & M^{23} = \begin{pmatrix} 0 & 0 & 0 & 0 \\ 0 & 0 & - \imath & 0 \\ 0 & \imath & 0 & 0 \\ 0 & 0 & 0 & 0 \end{pmatrix}, \qquad M^{31} = \begin{pmatrix} 0 & 0 & \imath & 0 \\ 0 & 0 & 0 & 0 \\ -\imath & 0 & 0 & 0 \\ 0 & 0 & 0 & 0 \end{pmatrix}, \qquad M^{12} = \begin{pmatrix} 0 & -\imath & 0 & 0 \\ \imath & 0 & 0 & 0 \\ 0 & 0 & 0 & 0 \\ 0 & 0 & 0 & 0 \end{pmatrix},
\\[1em] & M^{14} = \begin{pmatrix} 0 & 0 & 0 & -\imath \\ 0 & 0 & 0 & 0 \\ 0 & 0 & 0 & 0 \\ \imath & 0 & 0 & 0 \end{pmatrix}, \qquad M^{24} = \begin{pmatrix} 0 & 0 & 0 & 0 \\ 0 & 0 & 0 & -\imath \\ 0 & 0 & 0 & 0 \\ 0 & \imath & 0 & 0 \end{pmatrix}, \qquad M^{34} = \begin{pmatrix} 0 & 0 & 0 & 0 \\ 0 & 0 & 0 & 0 \\ 0 & 0 & 0 & -\imath \\ 0 & 0 & \imath & 0 \end{pmatrix} .\end{aligned} \end{equation*}
We immediately see that $\mathrm{SO}(4, \, \R)$ is generated by the rotations restricted to all the possible coordinate planes. More precisely, the matrix $M^{ij}$ is a rotation on the plane $\mathrm{Span} \langle i, \, \ j \rangle$.

The elements of $\mathrm{SO}(4, \, \R)$ can be easily computed via the exponentiation of the matrix $M^{ij}$ since we already know that
\begin{equation*} \mathrm{e}^{ \imath \theta {\tiny \begin{pmatrix} 0 & - \imath \\ \imath & 0 \end{pmatrix}} } = \begin{pmatrix} \cos \theta & \sin \theta \\ - \sin \theta & \cos \theta \end{pmatrix}. \end{equation*}
In particular, it turns out that
\begin{equation*} \left( \mathrm{e}^{ \imath \theta M^{ij} } \right)_{k, \, \ell} = \begin{cases} 1 & \text{if $k = \ell$ and $k \neq i$, $k \neq j$},\\[0.5em] \cos \theta & \text{if $k = \ell = i$ or $k=\ell=j$}, \\[0.5em] \sin \theta & \text{if $(k, \, \ell) = (i, \, j)$}, \\[0.5em] - \sin \theta & \text{if $(k, \, \ell) = (j, \, i)$}, \\[0.5em] 0 & \text{otherwise.}\end{cases} \end{equation*}
This computation is quite surprising because, for one thing, it implies that $\underline{4}$ is not the smallest nontrivial representation of $\mathrm{SO}(4, \, \R)$. Indeed, as we will be able to show soon, the associated Lie algebra $\mathrm{so}(4, \, \R)$ can be decomposed as the direct product of two invariant Lie algebras.

Our goal is now to find the relations between the generators, that is, $[M^{ij}, \, M^{k \ell}]$. To complete this task, we first rewrite the generators in a more compact manner in terms of the Dirac deltas, that is,
\begin{equation} \label{eq.12.1} (M^{ij})_{k \ell} = - \imath (\delta_{ik} \delta_{jl} - \delta_{il} \delta_{jk}). \end{equation}
It follows from \eqref{eq.12.1} that
\begin{equation*}[M^{ij}, \, M^{k \ell}] = \begin{cases} 0 & \text{if $\{i, \, j \} \cap \{k, \, \ell \} = \varnothing$}, \\[0.5em] - \imath M^{ik} & \text{if $\{i, \, j \} \cap \{k, \, \ell \} \neq \varnothing$}.\end{cases} \end{equation*}
The algebra $\mathrm{so}(4, \, \R)$ is thus given by the direct product of two invariant subalgebras if we consider a new set of generators defined by
\begin{equation*}\begin{aligned} & S^1 = \frac{1}{2} (M^{23} + M^{41}), \qquad \hat{S}^1 = \frac{1}{2} (M^{23} - M^{41}),
\\[1em] & S^2 = \frac{1}{2} (M^{31} + M^{42}), \qquad \hat{S}^2 = \frac{1}{2} (M^{31} - M^{42}),
\\[1em] & S^3 = \frac{1}{2} (M^{12} + M^{43}), \qquad \hat{S}^3 = \frac{1}{2} (M^{12} - M^{43}).\end{aligned} \end{equation*}
In fact, the reader can quickly verify that from \eqref{eq.12.1} it follows that
\begin{equation*}\begin{aligned} & [S^i, \, S^j] = \imath \epsilon^{ijk} S^k,
\\[1em] & [\hat{S}^i, \, \hat{S}^j] = \imath \epsilon^{ijk} \hat{S}^k,
\\[1em] & [S^i, \, \hat{S}^j] = 0,\end{aligned} \end{equation*}
which means that
\begin{equation*}\mathrm{so}(4, \, \R) \cong \mathrm{su}(2, \, \C) \otimes \mathrm{su}(2, \, \C) = \mathrm{Span}\langle S^1, \, S^2, \, S^3 \rangle \otimes \mathrm{Span} \langle \hat{S}^1, \, \hat{S}^2, \, \hat{S}^3 \rangle. \end{equation*}
In particular, the Lie algebra $\mathrm{so}(4, \, \R)$ is not simple, but one can easily show that it is semisimple since no invariant subalgebra is abelian (e.g., $\mathrm{su}(2, \, \C)$ is not commutative).

Consequently, there exists a bijective correspondence between the representations of $\mathrm{so}(4, \, \R)$ and the representation of the direct product $\mathrm{su}(2, \, \C) \otimes \mathrm{su}(2, \, \C)$. For example,
\begin{equation*} (\underline{1}, \, \underline{1}) \leftrightarrow \underline{1} \quad \text{or} \quad (\underline{2}, \, \underline{2}) \leftrightarrow \underline{4}, \end{equation*}
while the mixed representations $(\underline{2}, \, \underline{1})$ and $(\underline{1}, \, \underline{2})$ are usually called \textit{spin representation}\index{spin representation}.

Notice that these do not represent Dirac spinors because both $(\underline{2}, \, \underline{1})$ and $(\underline{1}, \, \underline{2})$ are given by elements with four components. Since only two of them are linearly independent, it turns out that these are Majorana spinors.