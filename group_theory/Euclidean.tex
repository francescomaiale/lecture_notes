\chapter{Euclidean Groups $E_n$} \thispagestyle{empty}
\label{ch:e2}

In this chapter, we study more in-depth the special unitary (Lie) group $E_n$ and the associated Lie algebra, denoted by $e_2$.

Recall that $E_n$ is the symmetry group of the $n$-dimensional Euclidean space ($\R^n$). Its elements are the isometries associated with the Euclidean distance and are called Euclidean isometries or Euclidean transformations.

\section{Two-Dimensional Euclidean Group $E_2$}

The two-dimensional Euclidean group $E_2$ is the symmetry group of the plane, and its elements are the transformations of the form
\begin{equation}\label{eq.15.1} \begin{pmatrix} x \\ y \end{pmatrix} \longmapsto \begin{pmatrix} \cos \theta & - \sin \theta \\ \sin \theta & \cos\theta \end{pmatrix} \begin{pmatrix} x \\ y \end{pmatrix} + \begin{pmatrix} a_1 \\ a_2 \end{pmatrix}, \end{equation}
where $\theta \in [0, \, 2 \pi)$ is the rotation angle, and $\mathbf{a} = (a_1, \, a_2)^T \in \R^2$ denotes the translation.

In the previous chapters, we proved that the generic transformation of the form \eqref{eq.15.1} may be equivalently rewritten as a $3$-dimensional transformation:
\begin{equation*} \begin{pmatrix} x \\ y \\ 1 \end{pmatrix} \longmapsto \left(\begin{array}{@{}c|c@{}}
  R(\theta) &
  \begin{matrix}
  a_1 \\
  a_2
  \end{matrix}
\\ \hline
  \begin{matrix}
  0 & 0
  \end{matrix}
  & 1
\end{array}\right) \begin{pmatrix} x \\ y \\ 1 \end{pmatrix}, \end{equation*}
where $R(\theta)$ denotes the rotation of angle $\theta$, that is,
\begin{equation*}R(\theta) = \begin{pmatrix} \cos \theta & - \sin \theta \\ \sin \theta & \cos\theta \end{pmatrix}. \end{equation*}
The generators of the Lie algebra (computed in the second chapter) are given by
\begin{equation*} \begin{cases} P^1 = - \imath \frac{\partial}{\partial x}, \\[1em] P^2 = - \imath \frac{\partial}{\partial y}, \\[1em] J = - \imath \left( x \frac{\partial}{\partial y} - y \frac{\partial}{\partial x} \right), \end{cases} \end{equation*}
since one can easily check that
\begin{equation*} \begin{aligned} & \mathrm{e}^{\imath a_1 P^1} \begin{pmatrix}x \\ y \end{pmatrix} \approx (\mathrm{id}_{2 \times 2} + \imath a_1 P^1) \begin{pmatrix}x \\ y \end{pmatrix} = \begin{pmatrix}x + a_1 \\ y \end{pmatrix},
\\[1em] & \mathrm{e}^{\imath a_2 P^2} \begin{pmatrix}x \\ y \end{pmatrix} \approx (\mathrm{id}_{2 \times 2} + \imath a_2 P^2) \begin{pmatrix}x \\ y \end{pmatrix} = \begin{pmatrix}x  \\ y + a_2 \end{pmatrix},
\\[1em] & \mathrm{e}^{\imath \theta J} \begin{pmatrix}x \\ y \end{pmatrix} \approx (\mathrm{id}_{2 \times 2} + \imath \theta J) \begin{pmatrix}x \\ y \end{pmatrix} = \begin{pmatrix}x \\ y \end{pmatrix} + \begin{pmatrix} - \theta y \\ \theta x \end{pmatrix}. \end{aligned} \end{equation*}
We can easily compute the value of the commutators
\begin{equation*} \begin{aligned} & [P^1, \, P^2] = 0, \\[1em] & [P^1, \, J] = - \imath P^2, \\[1em] & [P^2, \, J] = \imath P^1 \end{aligned} \end{equation*}
from which we infer that, for any $a \in \{1, \, 2, \, J\}$, we have
\begin{equation*} f^{12a} = 0 \qquad \text{and} \qquad f^{1 J 2} = - f^{2 J 1} = -1. \end{equation*}
Recall that the action of any element of the group $E_2$ can also be seen as the action of a $3$-dimensional matrix, which will be denoted from now on by $g(\mathbf{a}, \, \theta)$:
\begin{equation*} \begin{pmatrix} x \\ y \\ 1 \end{pmatrix} \longmapsto \left(\begin{array}{@{}c|c@{}}
  R(\theta) &
  \begin{matrix}
  a_1 \\
  a_2
  \end{matrix}
\\ \hline
  \begin{matrix}
  0 & 0
  \end{matrix}
  & 1
\end{array}\right) \begin{pmatrix} x \\ y \\ 1 \end{pmatrix} =: g(\mathbf{a}, \, \theta) \begin{pmatrix} x \\ y \\ 1 \end{pmatrix}. \end{equation*}
The product between any two elements is given by
\begin{equation*} g(\mathbf{a}, \, \theta) g(\mathbf{b}, \, \phi) = g\left( R(\theta)\mathbf{b} + \mathbf{a}, \, \theta + \phi \right), \end{equation*}
and this is the composition rule that characterize the two-dimensional Euclidean group $E_2$. Using this notation, we can easily write the generators $P^1$, $P^2$ and $J$ as follows:
\begin{equation*} J = \begin{pmatrix} 0 & - \imath & 0 \\ \imath & 0 & 0 \\ 0 & 0 & 0 \end{pmatrix}, \qquad P^1 = \begin{pmatrix} 0 & 0 & \imath \\ 0 & 0 & 0 \\ 0 & 0 & 0 \end{pmatrix}, \qquad P^2 = \begin{pmatrix} 0 & 0 & 0 \\ 0 & 0 & \imath \\ 0 & 0 & 0 \end{pmatrix}.\end{equation*}

\begin{remark} Notice that we \textbf{cannot} use the representation of the Euclidean group
\begin{equation*} \begin{pmatrix} x \\ y \\ 1 \end{pmatrix} \longmapsto \left(\begin{array}{@{}c|c@{}}
  R(\theta) &
  \begin{matrix}
  a_1 \\
  a_2
  \end{matrix}
\\ \hline
  \begin{matrix}
  0 & 0
  \end{matrix}
  & 1
\end{array}\right) \end{equation*}
to find the generators of $E_2$. The reason is that, although the form is useful to multiply the elements of $E_2$, it is not the right vector space (and we shall see soon that a similar thing happens for the Poincaré group.) \end{remark}

We can easily compute the elements in this particular representation by noticing that
\begin{equation*}(P^i)^2 = 0 \quad \text{for $i = 1, \, 2$} \implies (P^i)^{n} = 0 \quad \text{for $i = 1, \, 2$ and for all $n \geq 2$}, \end{equation*}
which means that the exponential power series has only two nonzero terms, i.e.,
\begin{equation*} \mathrm{e}^{\imath a_1 P^1}  = \mathrm{id}_{3 \times 3} - \imath a_1 P^1 = g( (a_1, \, 0), \, 0)  \quad \text{and} \quad \mathrm{e}^{\imath a_2 P^2}  = \mathrm{id}_{3 \times 3} - \imath a_2 P^2 = g( (0, \, a_2), \, 0) \end{equation*}
while the rotation is computed as usual:
\begin{equation*} \mathrm{e}^{\imath \theta J}  = \begin{pmatrix} \cos \theta &  \sin \theta & 0 \\ - \sin \theta & \cos \theta & 0 \\ 0 & 0 & 1 \end{pmatrix} = R(-\theta) = g(\mathbf{0}, \, - \theta). \end{equation*}
Notice also that
\begin{equation*} \mathfrak{t} := \mathrm{Span} \langle P^1, \, P^2 \rangle \end{equation*}
is an invariant subalgebra, generated by $P^1$ and $P^2$, which is also abelian. In particular, the algebra associated with $E_2$ is not semisimple (and thus, it is not simple), and we can always write
\begin{equation*} g(\mathbf{a}, \, \theta) = g(\mathbf{b}, \, 0) g(\mathbf{0}, \, \theta) \implies e_2 = \mathfrak{t} \otimes \mathrm{so}(2, \, \R). \end{equation*}

We now want to study the action of a rotation on the generators $P^i$, for $i = 1, \, 2$, in order to show that the translations form a normal subgroup of $E_2$. First, notice that
\begin{equation*}\begin{aligned} \mathrm{e}^{- \imath \theta J} P^1 \mathrm{e}^{\imath \theta J} & = g(\mathbf{0}, \, \theta) P^1 g(\mathbf{0}, \, -\theta) =
\\[1em] & = \begin{pmatrix} \cos \theta & - \sin \theta & 0 \\ \sin \theta & \cos \theta & 0 \\ 0 & 0 & 1 \end{pmatrix} \begin{pmatrix} 0 & 0 & \imath \\ 0 & 0 & 0 \\ 0 & 0 & 0 \end{pmatrix}\begin{pmatrix} \cos \theta &  \sin \theta & 0 \\ - \sin \theta & \cos \theta & 0 \\ 0 & 0 & 1 \end{pmatrix}  =
\\[1em] & = \begin{pmatrix} \cos \theta & - \sin \theta & 0 \\ \sin \theta & \cos \theta & 0 \\ 0 & 0 & 1 \end{pmatrix} \begin{pmatrix} 0 & 0 & \imath \\ 0 & 0 & 0 \\ 0 & 0 & 0 \end{pmatrix}  =
\\[1em] & = \begin{pmatrix} 0 & 0 & \imath \cos \theta \\ 0 & 0 & \imath \sin \theta \\ 0 & 0 & 0 \end{pmatrix} = P^1 \cos \theta + P^2 \sin \theta = P^i g(\mathbf{0}, \, \theta)_{i, \, 1},  \end{aligned} \end{equation*}
and, in a similar way, we also have that
\begin{equation*}\begin{aligned} \mathrm{e}^{- \imath \theta J} P^2 \mathrm{e}^{\imath \theta J} & = g(\mathbf{0}, \, \theta) P^1 g(\mathbf{0}, \, -\theta) =
\\[1em] & = \begin{pmatrix} \cos \theta & - \sin \theta & 0 \\ \sin \theta & \cos \theta & 0 \\ 0 & 0 & 1 \end{pmatrix} \begin{pmatrix} 0 & 0 & 0 \\ 0 & 0 & \imath \\ 0 & 0 & 0 \end{pmatrix}\begin{pmatrix} \cos \theta &  \sin \theta & 0 \\ - \sin \theta & \cos \theta & 0 \\ 0 & 0 & 1 \end{pmatrix}  =
\\[1em] & = \begin{pmatrix} \cos \theta & - \sin \theta & 0 \\ \sin \theta & \cos \theta & 0 \\ 0 & 0 & 1 \end{pmatrix} \begin{pmatrix} 0 & 0 & 0 \\ 0 & 0 & \imath \\ 0 & 0 & 0 \end{pmatrix}  =
\\[1em] & = \begin{pmatrix} 0 & 0 & - \imath \sin \theta \\ 0 & 0 & \imath \cos \theta \\ 0 & 0 & 0 \end{pmatrix} = - P^1 \sin \theta + P^2 \cos \theta = P^i g(\mathbf{0}, \, \theta)_{i, \, 2}.  \end{aligned} \end{equation*}
It follows that
\begin{equation*}\begin{aligned} g(\mathbf{0}, \, \theta) g(\mathbf{a}, \, 0) g(\mathbf{0}, \, \theta)^{-1} &= \mathrm{e}^{- \imath \theta R} \mathrm{e}^{\imath (a_1 P^1 + a_2 P^2)} \mathrm{e}^{\imath \theta R} =
\\[1em] & = \mathrm{e}^{- \imath \theta R} \left( \mathrm{id}_{3 \times 3} - \imath a_1 P^1 - \imath a_2 P^2  \right)\mathrm{e}^{\imath \theta R} =
\\[1em] & = \mathrm{id}_{3 \times 3} - \imath a_1 P^i g(\mathbf{0}, \, \theta)_{i, \, 1} - \imath a_2 P^i g(\mathbf{0}, \, \theta)_{i, \, 2} =
\\[1em] & = \begin{pmatrix} 1 & 0 & a_1 \cos \theta - a_2 \sin \theta \\ 0 & 1 & a_1 \sin \theta + a_2 \cos \theta \\ 0 & 0 & 1 \end{pmatrix} = g(\mathbf{a}^\prime, \, 0), \end{aligned} \end{equation*}
where $\mathbf{a}^\prime$ is a new translation vector defined by the product
\begin{equation*}\mathbf{a}^\prime := R(\theta) \mathbf{a}. \end{equation*}

\begin{theorem}The subgroup of all translation
\begin{equation*} \mathfrak{T} := \left\{ g(\mathbf{b}, \, 0) \: : \: \mathbf{b} \in \R^2 \right\} \subset E_2 \end{equation*}
is an invariant/normal subgroup of $E_2$.\end{theorem}

\begin{proof} Let $g(\mathbf{a}, \, 0) \in \mathfrak{T}$ be a translation, and let $g(\mathbf{b}, \, \theta)$ be an arbitrary element of $E_2$. From the identity proved above, it follows that
\begin{equation*} \begin{aligned} g(\mathbf{b}, \, \theta) g(\mathbf{a}, \, 0) g(\mathbf{b}, \, \theta)^{-1} & = g(\mathbf{b}, \, 0) g(\mathbf{0}, \, \theta) g(\mathbf{a}, \, 0) g(\mathbf{0}, \, \theta)^{-1} g(- \mathbf{b}, \, 0) =
\\[1em] & = g(\mathbf{b}, \, 0) g(R(\theta)\mathbf{a}, \, 0) g(-\mathbf{b}, \,0) =
\\[1em] & = g(R(\theta)\mathbf{a}, \, 0) \in \mathfrak{T},\end{aligned} \end{equation*}
and this concludes the proof. \end{proof}

In particular, we can consider the quotient group $\faktor{E_2}{\mathfrak{T}}$. It is interesting to notice that any element in the quotient depends on the rotation only since
\begin{equation*} \left[g(\mathbf{c}, \, \theta) \right] \in \faktor{E_2}{\mathfrak{T}} \implies g(\mathbf{a}, \, 0) \left[ g(\mathbf{c}, \, \theta) \right] = \left[ g(\mathbf{c}, \, \theta) \right],  \end{equation*}
and therefore
\begin{equation*}\begin{aligned} [g(\mathbf{c}, \, \theta)] & = \left\{ g(\mathbf{b}, \, \theta) \: : \: \mathbf{b} = \mathbf{c} + \mathbf{a}, \, \, \mathbf{a} \in \R^2 \right\} =
\\[1em] & = \left\{ g(\mathbf{b}, \, \theta) \: : \: \mathbf{b} \in \R^2 \right\}. \end{aligned}  \end{equation*}
In particular, the elements of the quotient group $\faktor{E_2}{\mathfrak{T}}$ depends on the rotation angle $\theta$ only, which means that
\begin{equation*} \faktor{E_2}{\mathfrak{T}} \longrightarrow \mathrm{SO}(2, \, \R), \qquad [g(\mathbf{b}, \, \theta)] \longmapsto R(\theta)  \end{equation*}
is an isomorphism, that is,
\begin{equation*} \faktor{E_2}{\mathfrak{T}} \cong \mathrm{SO}(2, \, \R). \end{equation*}

\subsection{Irreducible Representation of $E_2$}

In this section, our goal will be to determine all the finite-dimensional irreducible representations of the Euclidean group $E_2$.

\begin{remark}The Euclidean group $E_2$ is not compact since the norm of the elements $g(\mathbf{b}, \, \theta)$ is arbitrarily big (and, hence, not bounded). Unitary representations of non-compact non-abelian Lie groups (like $E_2$) tend to be infinite-dimensional, as we will see in a few moments.\end{remark}

\paragraph{Trivial Representation.} Fix $m \in \Z$. There is a "trivial" representation given by
\begin{equation} \label{eq.12.4} g(\mathbf{b}, \, \theta) \longmapsto U_m(\mathbf{b}, \, \theta) := \mathrm{e}^{\imath m \theta}. \end{equation}
To assure that \eqref{eq.12.4} actually defines a representation, we simply need to check the associativity property. Indeed, from the product formula
\begin{equation*} g(\mathbf{a}, \, \theta) g(\mathbf{b}, \, \phi) = g( R(\theta)\mathbf{b} + \mathbf{a}, \, \theta + \phi),  \end{equation*}
we immediately infer that
\begin{equation*} g(\mathbf{a}, \, \theta) g(\mathbf{b}, \, \phi) \longmapsto \mathrm{e}^{\imath m (\theta + \phi)} = U_m(\mathbf{a}, \, \theta) U_m(\mathbf{b}, \, \phi), \end{equation*}
which means that \eqref{eq.12.4} satisfies the associative property.

The problem with this particular representation is that it does not bring any information about the translations, and therefore it is not a \textit{faithful} representation\index{representation!faithful}. On the other hand, the map \eqref{eq.12.4} gives a faithful representation of the quotient space
\begin{equation} \label{eq.12.6} [g(\mathbf{0}, \, \theta)] \longmapsto U_m(\theta) := \mathrm{e}^{\imath m \theta}. \end{equation}

\paragraph{Infinite-Dimensional Representation.} Let us consider the Casimir operator relative to the translation invariant subalgebra, i.e.,
\begin{equation*} \mathbb{P}^2 := (P^1)^2 + (P^2)^2. \end{equation*}
Recall that $\mathbb{P}^2$ commutes with every other generator $P^1$, $P^2$ and $J$. Indeed, a straightforward computation shows that
\begin{equation*} \begin{aligned} [\mathbb{P}^2, \, J] & = P^1[P^1, \, J] + [P^1, \, J]P^1 + P^2[P^2, \, J] + [P^2, \, J]P^2 =
\\[1em] & = - \imath P^1P^2 - \imath P^2 P^1 + \imath P^2 P^1 + \imath P^1 P^2 = 0.  \end{aligned} \end{equation*}
We also consider the up/down operators $P_{\pm}$, defined by setting
\begin{equation*}P_{\pm} = P^1 \pm \imath P^2, \end{equation*}
and we notice that
\begin{equation*} [P_{\pm}, \, P_{\pm}] = 0 \quad \text{and} \quad [J, \, P_{\pm}] = \pm P_{\pm}. \end{equation*}
In particular, the Casimir operator $\mathbb{P}^2$ commutes with the generator $J$, which means that there exists a common basis of eigenstates (i.e., they are simultaneously diagonalizable). More precisely, let us consider an orthonormal basis $|p, \, m \rangle$ such that
\begin{equation*}\begin{cases} J \,| \, p, \, m \rangle = m \,| \, p, \, m \rangle, \\[0.4em] \mathbb{P}^2 \, | \, p, \, m \rangle = p^2 \,| \, p, \, m \rangle, \end{cases} \quad \text{and} \quad \begin{cases} \langle p, \, m \, | \,  p, \, m \rangle = 1, \\[0.4em] \langle p, \, m \, | \, p, \, m^\prime \rangle = 0. \end{cases} \end{equation*}
In order to understand how the up/down operators $P_{\pm}$ acts on the eigenstates, we first notice that
\begin{equation*} J \left( P_+ \, | \, p, \, m \rangle \right) = P_+ \, | \, p, \, m \rangle + m P_+ \, | \, p, \, m \rangle = (m + 1)P_+ \, | \, p, \, m \rangle, \end{equation*}
which means that there exists a complex $c \in \C$ such that $P_+ \, | \, p, \, m \rangle = c \, | \,p, \, m +1\rangle$. To compute the absolute value of $c$, we employ the orthogonality and the identity above as follows:
\begin{equation*}|c|^2 = \langle p, \, m + 1 \, | \, |c|^2 \, | \, p, \, m +1 \rangle = \langle p, \, m \, | \, P_- P_+ \, | \, p, \, m  \rangle = p^2 \underbracket{\langle p, \, m \,  | \, p, \, m  \rangle}_{=1}.  \end{equation*}
In particular, we have that
\begin{equation*}|c|^2 = p^2 \implies c = \pm \frac{p}{\imath} \implies P_{\pm} \, |p, \, m \rangle = \pm \frac{p}{\imath} \, | \, p, \, m \pm 1 \rangle, \end{equation*}
and therefore we need to consider every possible value of $m$ to have a basis of eigenstates $\{ |p, \, m \rangle \}_{m \in \Z}$, which clearly gives us a infinite-dimensional vector space. It follows that
\begin{equation*}g(\mathbf{0}, \, \theta) \, | \, p, \, m \rangle = \mathrm{e}^{- \imath \theta J} \, | \, p, \, m \rangle = \mathrm{e}^{- \imath m \theta} \, | \, p, \, m \rangle, \end{equation*}
which means that the "trivial" representation presented above gives the eigenvectors of the pure rotations in $E_2$. Therefore, if we employ the fact that the basis is orthonormal, we find that
\begin{equation*}\langle p, \, m^\prime \, | \, g(\mathbf{0}, \, \theta) \, | \, p, \, m \rangle = \delta_{m, \, m^\prime}  \mathrm{e}^{- \imath m \theta}. \end{equation*}
We now want to do the same computation with a translation element $g(\mathbf{a}, \, \theta)$, but, surprisingly, it requires quite a lot of work. First, recall that
\begin{equation*} g(\mathbf{0}, \, \theta) g(\mathbf{a}, \, 0) g(\mathbf{0}, \, \theta)^{-1} = g(\mathbf{a}^\prime, \, 0), \end{equation*}
and hence we can always write $\mathbf{a}$ as a suitable rotation.

Namely, let $\phi$ be the angle between the $x$-axis and the vector $\mathbf{a}$, and consider the projection $\mathbf{a}_0 := (a, \, 0)$ for $a$ equal to the length of $\mathbf{a}$, i.e., $a = |\mathbf{a}|$. Then
\begin{equation*} \mathbf{a} = R(\phi) \mathbf{a}_0 \implies \mathrm{e}^{- \imath P \cdot \mathbf{a}} = g(\mathbf{0}, \, \phi) g(\mathbf{a}_0, \, 0) g(\mathbf{0}, \, \phi)^{-1}, \end{equation*}
which yields to the following identity:
\begin{equation*} g(\mathbf{a}, \, \theta) = g(\mathbf{0}, \, \theta) g(\mathbf{0}, \, \phi) g(\mathbf{a}_0, \, 0) g(\mathbf{0}, \, \phi)^{-1}.\end{equation*}
A straightforward computation shows that
\begin{equation*}\begin{aligned} \langle p, \, m^\prime \, | \, g(\mathbf{a}, \, \theta) \, | \, p, \, m \rangle & = \langle p, \, m^\prime \, | \, g(\mathbf{0}, \, \theta) g(\mathbf{0}, \, \phi) g(\mathbf{a}_0, \, 0) g(\mathbf{0}, \, \phi)^{-1} \, | \, p, \, m \rangle =
\\[1em] & = \mathrm{e}^{- \imath m \theta} \langle p, \, m^\prime \, | \, g(\mathbf{0}, \, \phi) g(\mathbf{a}_0, \, 0) g(\mathbf{0}, \, \phi)^{-1} \, | \, p, \, m \rangle =
\\[1em] & = \mathrm{e}^{- \imath m \theta} \mathrm{e}^{\imath \phi(m - m^\prime)} \langle p, \, m^\prime \, | \,\mathrm{e}^{- \imath P \cdot \mathbf{a}_0} \, | \, p, \, m \rangle = 
\\[1em] & = \mathrm{e}^{- \imath m \theta} \mathrm{e}^{\imath \phi(m - m^\prime)} \langle p, \, m^\prime \, | \,\mathrm{e}^{- \imath P^1 a} \, | \, p, \, m \rangle =
\\[1em] & = \mathrm{e}^{- \imath m \theta} \mathrm{e}^{\imath \phi(m - m^\prime)} \langle p, \, m^\prime \, | \,\mathrm{e}^{\frac{- \imath a}{2} (P_+ + P_-)} \, | \, p, \, m \rangle =
\\[1em] & = \mathrm{e}^{- \imath m \theta} \mathrm{e}^{\imath \phi(m - m^\prime)} \sum_{k, \, \ell = 0}^{+ \infty}\frac{1}{k! \ell!} \left( \frac{-\imath a}{2} \right)^{k + \ell} \langle p, \, m^\prime \, | \,P_+^k P_-^\ell \, | \, p, \, m \rangle =
\\[1em] & = \mathrm{e}^{- \imath m \theta} \mathrm{e}^{\imath \phi(m - m^\prime)} \sum_{k, \, \ell = 0}^{+ \infty}\frac{1}{k! \ell!} \left( \frac{-\imath a}{2} \right)^{k + \ell} \left( \frac{p}{\imath} \right)^k \left(- \frac{p}{\imath} \right)^\ell \delta_{m^\prime, \, m + k - \ell } =
\\[1em] & = \mathrm{e}^{- \imath m \theta} \mathrm{e}^{\imath \phi(m - m^\prime)} \sum_{k, \, \ell = 0}^{+ \infty}\frac{(-1)^k}{k! \ell!} \left( \frac{p a}{2} \right)^{k + \ell} \delta_{m^\prime, \, m + k - \ell }. \end{aligned} \end{equation*}
Notice that for $m > m^\prime$ we can consider $\ell := m - m^\prime + k$, and obtain
\begin{equation*}\begin{aligned} \langle p, \, m^\prime \, | \, g(\mathbf{a}, \, \theta) \, | \, p, \, m \rangle & = \mathrm{e}^{- \imath m \theta} \mathrm{e}^{\imath \phi(m - m^\prime)} \sum_{k = 0}^{+ \infty}\frac{(-1)^k}{k! (m - m^\prime + k)!} \left( \frac{p a}{2} \right)^{2k + m - m^\prime} =
\\[1em] & = \mathrm{e}^{- \imath m \theta} \mathrm{e}^{\imath \phi(m - m^\prime)} \left( \frac{p a}{2} \right)^{m - m^\prime} \sum_{k = 0}^{+ \infty}\frac{(-1)^k}{k! (m - m^\prime + k)!} \left( \frac{p a}{2} \right)^{2k} =
\\[1em] & =  \mathrm{e}^{- \imath m \theta} \mathrm{e}^{\imath \phi(m - m^\prime)} J_{m - m^\prime}(ap), \end{aligned}  \end{equation*}
where $J_\nu(z)$ is the \textit{Bessel function}\index{Bessel function} defined by
\begin{equation} \label{bessel} J_\nu(z) := \left( \frac{z}{2} \right)^{\nu} \: \sum_{k = 0}^{+ \infty}\frac{(-1)^k \left( \frac{z}{2}\right)^{2k}}{k! \, \Gamma(\nu + k + 1)}. \end{equation}
Recall that the Bessel function is the solution, regular at $z = 0$, of the second-order differential equation
\begin{equation*} \frac{\mathrm{d}^2}{\mathrm{d}z^2} u(z) + \frac{1}{z} \frac{\mathrm{d}}{\mathrm{d}z} u(z) + \left(1 - \frac{\nu^2}{z^2} \right) u(z) = 0, \end{equation*}
while the \textit{Gamma function}\index{Gamma function} is defined by setting
\begin{equation*}\Gamma(z) := \int_0^{+ \infty} t^{z-1} \mathrm{e}^{-t} \, \mathrm{d}t \quad \text{for all $\mathfrak{Re}(z) > 0$}, \end{equation*}
which can be easily extended analytically to the whole complex plane except at the integers ($\Z_-$) less than $0$. Furthermore, the Bessel function satisfies a nontrivial symmetry property
\begin{equation} \label{bessel2} J_{-\nu}(z) = (-1)^\nu J_\nu(z) \quad \text{for all $\nu \in \Z$}.\end{equation}
Now, notice that for $m^\prime > m$ we can consider $k := m^\prime - m + \ell$, and obtain
\begin{equation*}\begin{aligned} \langle p, \, m^\prime \, | \, g(\mathbf{a}, \, \theta) \, | \, p, \, m \rangle & = \mathrm{e}^{- \imath m \theta} \mathrm{e}^{\imath \phi(m - m^\prime)} \sum_{\ell = 0}^{+ \infty}\frac{(-1)^{m^\prime - m + \ell}}{(m^\prime - m + \ell)! \ell!} \left( \frac{p a}{2} \right)^{2\ell + m^\prime - m} =
\\[1em] & = \mathrm{e}^{- \imath m \theta} \mathrm{e}^{\imath \phi(m - m^\prime)} \left( - \frac{p a}{2} \right)^{m^\prime - m} \sum_{\ell = 0}^{+ \infty}\frac{(-1)^\ell}{(m^\prime - m + \ell)! \ell!} \left( \frac{p a}{2} \right)^{2\ell} =
\\[1em] & =  \mathrm{e}^{- \imath m \theta} \mathrm{e}^{\imath \phi(m - m^\prime)} (-1)^{m^\prime - m} J_{m^\prime - m}(ap) =
\\[1em] & \stackrel{\eqref{bessel2}}{=} \mathrm{e}^{- \imath m \theta} \mathrm{e}^{\imath \phi(m - m^\prime)} J_{m - m^\prime}(ap), \end{aligned}  \end{equation*}
which means that for all $m, \, m^\prime \in \Z$ we have
\begin{equation*}\langle p, \, m^\prime \, | \, g(\mathbf{a}, \, \theta) \, | \, p, \, m \rangle = \mathrm{e}^{- \imath m \theta} \mathrm{e}^{\imath \phi(m - m^\prime)} J_{m - m^\prime}(ap).  \end{equation*}

\begin{remark} The representation presented here reduces to the "trivial" one when $p^2 = 0$. \end{remark}

\begin{remark} If $p^2$ is different from zero, we can choose a different system of coordinates. The Casimir operator $\mathbb{P}^2$ commutes with $P^1$ as well, and therefore we can find a simultaneous basis of eigenstates $| \, p, \, 0 \rangle$ in such a way that
\begin{equation*} P^1 \, | \, p, \, 0 \rangle = p \, | \, p, \, 0 \rangle \quad \text{and} \quad P^2 \, | \, p, \, 0 \rangle = 0, \end{equation*}
while the Casimir operator gives
\begin{equation*} \mathbb{P}^2 \, | \, p, \, 0 \rangle = p^2 \, | \, p, \, 0 \rangle. \end{equation*}
A straightforward computation also proves that
\begin{equation*}J \, | \, p, \, 0 \rangle = | \, R(\theta) \mathbf{p}_0 \rangle \quad \text{where $\mathbf{p}_0 = (p, \, 0)$}. \end{equation*} \end{remark}