
\documentclass[a4paper,10 pt, twoside]{report}

\usepackage{graphicx}
\usepackage[dvipsnames]{xcolor}
\usepackage{amsfonts}
\usepackage[labelfont=bf]{caption}
\usepackage[pass]{geometry}
\usepackage{amsthm}
\usepackage{amsmath, amssymb}
\usepackage{setspace}
\usepackage[english]{babel}
\usepackage{tikz-cd}
\usepackage{imakeidx} % permette di generare l'indice. La 'i' davanti esegue da solo il comando makeindex durante la compilazione
\usepackage{fancyhdr}
\usepackage[utf8]{inputenc}
\usepackage{mathtools}
\usepackage{enumitem}
\usepackage{accents}
\usepackage{faktor}
\usepackage{mathrsfs}  
\usepackage{pifont}
\usepackage{mwe}
\usepackage{young}
\usepackage[vcentermath]{youngtab}
\usepackage{hyperref}
\usepackage{scalerel}[2014/03/10]
\usepackage[usestackEOL]{stackengine}

\usepackage{bbm}
\usepackage{PTSansNarrow}
\usepackage[T1]{fontenc}

\usepackage{ragged2e}
\usepackage[framemethod=tikz]{mdframed}
\usepackage{marginnote}
\usepackage{xparse}
\usepackage{tikzpagenodes}


\usetikzlibrary{calc}
\usetikzlibrary{matrix}
\usetikzlibrary{plotmarks}

%%CAUTION
\newcounter{mycaution}
\newcommand\pointeranchor{}
\newcommand\boxanchor{}
\newlength\boxvshift
\newlength\uppertrianglecorner

\newcommand\tikzmark[1]{%
  \tikz[remember picture,overlay]\node[inner xsep=0pt,outer sep=0pt] (#1) {};}

\NewDocumentCommand{\caution}{O{c}O{BrickRed}O{Caution!}m}{%
\stepcounter{mycaution}%
\tikzmark{\themycaution}%
\if#1b\relax
\renewcommand\pointeranchor{mybox\themycaution.south east}%
\renewcommand\boxanchor{south east}%
\setlength\boxvshift{-10pt}%
\setlength\uppertrianglecorner{13pt}%
\else
\if#1t\relax
\renewcommand\pointeranchor{mybox\themycaution.north east}%
\renewcommand\boxanchor{north east}%
\setlength\boxvshift{10pt}%
\setlength\uppertrianglecorner{-7pt}%
\else
\if#1c\relax
\renewcommand\pointeranchor{mybox\themycaution.east}%
\renewcommand\boxanchor{east}%
\setlength\boxvshift{0pt}%
\setlength\uppertrianglecorner{3pt}%
\fi\fi\fi%
\begin{tikzpicture}[remember picture,overlay]
\node[draw=#2,anchor=\boxanchor,xshift=-\marginparsep,yshift=\boxvshift]   
  (mybox\themycaution)
  at ([yshift=3pt]current page text area.west|-\themycaution) 
  {\parbox{\marginparwidth}{\vskip10pt\RaggedRight\small#4}};
\node[fill=white,font=\color{#2}\sffamily,anchor=west,xshift=7pt]
  at (mybox\themycaution.north west) {\ #3\ };
\fill[#2]
  ([yshift=\uppertrianglecorner]\pointeranchor) --
  ([yshift=\uppertrianglecorner-3pt,xshift=3pt]\pointeranchor) --
  ([yshift=\uppertrianglecorner-6pt]\pointeranchor) -- cycle;
\end{tikzpicture}%
}

%%

\makeindex %DDD

\hypersetup{
    colorlinks=true,
    linkcolor=cyan,
    filecolor=magenta,      
    urlcolor=cyan,
}

\definecolor{amaranth}{rgb}{0.9, 0.17, 0.31}

\DeclareRobustCommand\longtwoheadrightarrow
     {\relbar\joinrel\twoheadrightarrow}

\newcommand{\notimplies}{%
  \mathrel{{\ooalign{\hidewidth$\not\phantom{=}$\hidewidth\cr$\implies$}}}}

\pagestyle{plain}
\setlength{\topmargin}{0.0in}
\setlength{\headheight}{0.2in}
\setlength{\headsep}{0.2in}
\setlength{\footskip}{0.5in}
\setlength{\footnotesep}{0.15in}
\setlength{\textheight}{8.3in}
\setlength{\textwidth}{5.5in} % 6
\setlength{\oddsidemargin}{0.5in}
\setlength{\evensidemargin}{0.5in}
\setlength{\parindent}{0.2 in}
\setlength{\parskip}{0.1 in}
\setlength{\marginparwidth}{1.2 in}

\newtheorem{theorem}{Theorem}[chapter]
\newtheorem{lemma}[theorem]{Lemma}
\newtheorem{proposition}[theorem]{Proposition}
\newtheorem{corollary}[theorem]{Corollary}
\theoremstyle{definition}
\newtheorem{definition}[theorem]{Definition}

\newtheorem{remark}{Remark}[chapter]
\newtheorem{example}{Example}[chapter]
\newtheorem*{notation}{Notation}
\newtheorem*{claim}{Claim}
\newtheorem{exercise}{Exercise}[chapter]

\newcommand{\smallO}[1]{\scriptstyle\mathcal{O}}
\DeclarePairedDelimiter\floor{\lfloor}{\rfloor}
\newcommand*\conj[1]{\overline{#1}}
\newcommand{\R}{\mathbb R}
\newcommand{\C}{\mathbb C}
\newcommand{\T}{\mathbb T}
\newcommand{\N}{\mathbb N}
\newcommand{\G}{\mathcal G}
\newcommand{\Ha}{\mathcal H}
\newcommand{\Na}{\mathcal N}
\newcommand{\g}{\mathfrak g}
\newcommand{\h}{\mathfrak h}
\newcommand{\Z}{\mathbb Z}
\newcommand{\con}{\mathcal{C}\left(x, \, V, \, \alpha \right)}
\newcommand{\Q}{\mathbb Q}
\newcommand{\p}{\mathbb P}
\newcommand{\Le}{\mathcal{L}}
\newcommand{\K}{\mathcal{K}}
\newcommand{\can}{\symbol{35}}
\newcommand*{\double}[2][.1ex]{%
  \mathrel{\vcenter{\offinterlineskip%
  \hbox{$#2$}\vskip#1\hbox{$#2$}}}}
\newcommand*{\doublerightarrow}{\double{\longrightarrow}}

\newcommand{\restr}{%
  \,\raisebox{-.127ex}{\reflectbox{\rotatebox[origin=br]{-90}{$\lnot$}}}\,%
}

\newcommand\subgroup{\leqslant}
\newcommand\normsubgroup{\leq}

\makeatletter
\renewcommand*\env@matrix[1][*\c@MaxMatrixCols c]{%
  \hskip -\arraycolsep
  \let\@ifnextchar\new@ifnextchar
  \array{#1}}
\makeatother

\definecolor{darkgreen}{rgb}{0.0, 0.2, 0.13}

\newcommand{\bsquare}{\item[\color{magenta}\ding{110}]} 
\newcommand{\barrow}{\item[\color{blue}\ding{228}]}
\newcommand{\bwarrow}{\item[\color{gray}\ding{227}]}

\def\dashint{\,\ThisStyle{\ensurestackMath{%
  \stackinset{c}{.2\LMpt}{c}{.5\LMpt}{\SavedStyle-}{\SavedStyle\phantom{\int}}}%
  \setbox0=\hbox{$\SavedStyle\int\,$}\kern-\wd0}\int}
\def\ddashint{\,\ThisStyle{\ensurestackMath{%
  \stackinset{c}{.2\LMpt}{c}{.5\LMpt+.2\LMex}{\SavedStyle-}{%
    \stackinset{c}{.2\LMpt}{c}{.5\LMpt-.2\LMex}{\SavedStyle-}{%
      \SavedStyle\phantom{\int}}}}\setbox0=\hbox{$\SavedStyle\int\,$}\kern-\wd0}\int}
      
      
\fancyhf{}
% Put the page number at the right edge of odd pages, and left edge of even pages.
\fancyhead[LO,RE]{\textbf \thepage}
% Custom text at the left edge of odd pages, and right edge of odd pages.
\fancyhead[RO]{ \rightmark}
\fancyhead[LE]{ \leftmark}

% Repeat for \fancyfoot if needed, e.g.
% Some decorative symbol at the centre of both odd and even pages
\fancyfoot[C]{ }

% Set this length to 0pt if you don't want any lines!
\renewcommand{\headrulewidth}{1pt}

% Apply the fancy header style
\pagestyle{fancy}


\begin{document}
\newpage \thispagestyle{empty}

\begin{center}

\begin{spacing}{1.5}
{\huge  \sf Lecture Notes}\\
\vspace*{\fill}
\end{spacing}
\begin{spacing}{2.5}
\textbf{\huge Group Theory}\\[0.5cm]
\vspace*{\fill}
\begin{minipage}{5cm}
\centering {\textit{Course held by}}
\end{minipage}
\hspace*{\fill}
\begin{minipage}{5cm}
\centering {\textit{Notes written by}} \\
\end{minipage}
\end{spacing}

\begin{spacing}{1.3}

\begin{minipage}{5cm}
\centering {\textbf{\large Prof. Kenichi Konishi}}
\end{minipage}
\hspace*{\fill}
\begin{minipage}{5cm}
\centering {\textbf{\large Francesco Paolo Maiale}}
\end{minipage}

\vspace*{\fill}

\textnormal{\large Department of Mathematics \\[0.4em] Pisa University \\[0.4em] \today}

\end{spacing}
\end{center}

\newpage \thispagestyle{empty}
\begin{center}
\vspace*{1cm}
\begin{spacing}{2.5}
 {
\textbf{\huge Disclaimer}
}
\end{spacing} \end{center}

\begin{spacing}{1.2}
These are the notes I have written during the \textit{Group Theory} course, held by Professor Kenichi Konishi in the first semester of the academic year 2017/2018.

These include all the topics that were discussed during the lectures, but I took the liberty to extend the Mathematical side (e.g., the homotopy group chapter).

To report any mistakes, misprints, or if you have any question, feel free to send me an email to \textbf{francescopaolo (dot) maiale (at) gmail (dot) com}.
\end{spacing}

\begin{center}
\vspace*{1cm}
\begin{spacing}{2.5}
 {
\textbf{\huge Acknowledgments}
}
\end{spacing} \end{center}

\begin{spacing}{1.2}
I would like to thank the user \href{https://tex.stackexchange.com/users/3954/gonzalo-medina}{Gonzalo Medina}, from StackExchange, for the code (available \href{https://tex.stackexchange.com/questions/120224/create-a-framed-environment-for-a-margin-note/120253#120253}{here}) of the margin notes frame.
\end{spacing}

{ \setlength{\parskip}{0.05 in}

\clearpage                       % Otherwise \pagestyle affects the previous page.
{                                % Enclosed in braces so that re-definition is temporary.
  \pagestyle{empty}              % Removes numbers from middle pages.
  \fancypagestyle{plain}         % Re-definition removes numbers from first page.
  {
    \fancyhf{}%                       % Clear all header and footer fields.
    \renewcommand{\headrulewidth}{0pt}% Clear rules (remove these two lines if not desired).
    \renewcommand{\footrulewidth}{0pt}%
  }
  \tableofcontents
  \thispagestyle{empty}          % Removes numbers from last page.
}}


\part{Representation Theory}
\chapter{Introduction}

In this chapter, we introduce the main topics of the course and give a brief overview of what we will see and what we will be able to prove by the end of the course.

\section{Plateau's Problem}

The primary goal and the motivating example of this course is the \textbf{Plateau's problem}, that is, the problem to find the $d$-dimensional surface $\Sigma$ of the minimal area with prescribed $(d-1)$-dimensional boundary $\Gamma$.

By the end, we will be able to prove that a solution indeed exists, but we will not find it explicitly since it is a $NP$ (hard) numerical problem.

As of now, the problem is not well defined. In fact, the notions of \textit{surface}, \textit{area}, and \textit{boundary} make sense in the smooth setting but, as the examples below show, we need to work in a less regular setting.

More precisely, requiring the surface to be smooth is not enough for modeling reasons (e.g., dip a wire frame into a soap solution, form a soap film, and look for the minimal surface whose boundary is the wire frame), and also for existence reasons.

\begin{example} Here we give a list of Plateau's problems with prescribed boundary conditions, and we write down the correct solutions, without proving anything. \mbox{}
\begin{enumerate}[label=\textbf{(\alph*)}]
\item Let us identify $\R^4 \cong \C \times \C$ and, if $d = 2$, let us consider the smooth boundary given by
\begin{equation*} \Gamma_1 := \left( S^1 \times \{0\} \right) \cup \left( \{0\} \times S^1 \right). \end{equation*}
Surprisingly, every minimizing sequence of smooth surfaces converges to a surface which is not smooth at all. Indeed, the solution of the problem is
\begin{equation*} \Sigma_1 := \left( D^2 \times \{0\} \right) \cup \left( \{0\} \times D^2 \right). \end{equation*}
The surface $\Sigma_1$ is clearly singular at the origin, but the singularity may be removed (by factorizing it into two nonsingular surfaces).
\item Let us identify $\R^4 \cong \C \times \C$ and, if $d = 2$, let us consider the smooth boundary given by
\begin{equation*} \Gamma_2 := \left\{ (z^2, \, z^3) \: : \: z \in S^1 \right\}. \end{equation*}
The solution to the Plateau's problem is
\begin{equation*} \Sigma_2 := \left\{ (z^2, \, z^3) \: : \: z \in D^2 \right\}, \end{equation*}
which is a non-smooth surface, whose singularity cannot be removed (since the polynomial $z_1^3 = z_2^2$ cannot be factorized).
\item Let us identify $\R^8 \cong \R^4 \times \R^4$ and, if $d = 7$, let us consider the smooth boundary given by
\begin{equation*} \Gamma_3 := S^3 \times S^3. \end{equation*}
The minimal surface of prescribed boundary $\Gamma_3$ is
\begin{equation*} \Sigma_3 := \left\{ (x_1, \, x_2) \in \R^4 \times \R^4 \: : \: |x_1| = |x_2| \leq 1 \right\}. \end{equation*}
\end{enumerate}
\end{example}

To conclude this introductive chapter, we give a brief overview of the main approaches (studied in this course) to the Plateau's problem, as $d$ ranges between $1$ and $\infty$.

\section{Geodesics problem ($d=1$)}

The geodesics problem (that is, find the shortest curve connecting two points) is, surprisingly, still an open in the non-Riemannian setting. However, in the Riemannian setting, the geodesics problem is completely solved.

Indeed, if we consider the curves parametrized by paths, the \textit{length} is a well-defined notion, and the associated functional is lower semi-continuous and coercive; hence the compactness is easy to prove.

There are many possible approaches to the geodesics problem, e.g., the Steiner approach and the set theoretical approach, which we describe briefly in the remainder of the section.

\paragraph{Steiner Problem.} It is also called networks approach, and it is used to prove the existence of the geodesics and find the explicit expression for it. The reader may consult \cite{steinerpro} for a detailed dissertation on the topic.

\paragraph{Set Theoretical Approach.} The main idea is to find a closed and connected set $\Sigma$ of minimum \textit{length}, containing a given finite set $\Gamma$. As we shall see later in the course, in this case the length is a well defined concept: the \textit{Hausdorff distance}.

In fact, if $X$ is a suitable space (metric, endowed with Hausdorff distance, etc...), then the class defined by
\begin{equation*}\mathcal{X} := \left\{ K \subseteq X \: : \: K \, \, \text{compact and connected} \right\} \end{equation*}
is compact and, by Gotab theorem\footnote{\cite{falconer} Let $\mathscr{C}$ be an infinite collection of non-empty compact sets all lying in a bounded portion $B$ of $\R^n$. Then there exists a sequence $\{E_j\}$ of distinct sets of $\mathfrak{C}$ convergent in the Hausdorff metric to a non-empty compact set $E$.}, $\mathcal{H}^1$ is lower semi-continuous on $X$. 

\section{"Surface" Problem ($d>1$)}

\paragraph{3.} The Plateau's problem is much harder when $d = 2$, but there are still many approaches possible some of which relying, in a certain sense, on the work already done in the geodesics case.

\paragraph{Set Theoretical Approach.} This approach is highly nontrivial. For example, one may ask what does it mean that a compact set $\Sigma$ spans a boundary $\Gamma$? Moreover, there is another problem one should deal with: the $2$-dimensional Hausdorff measure $\mathcal{H}^2$ is, generally, not lower semi-continuous. The reader may consult \cite{Reifenberg1960} for a complete treatise of the topic.

\begin{remark}Suppose that $d = 2$, $n = 3$ and that $\Sigma$ is a surface with boundary $\Gamma$. If $\gamma$ is another closed curve, linked to $\Gamma$ (by a nonzero linking number), then $\gamma \cap \Sigma \neq \emptyset$. \end{remark}

\paragraph{Parametric Approach.} This method is essentially due to Douglas \cite{douglas}. The main idea is the following: since a parametrization $\phi : D^2 \to \R^n$ defines surfaces in $\R^n$, the area functional is well-defined and given by the formula
\begin{equation*}A(\phi) := \int_{D^2} \left| \frac{\partial \, \phi}{\partial \, s_1} \wedge \frac{\partial \, \phi}{\partial \, s_2} \right| \, \mathrm{d}s_1 \, \mathrm{d}s_2. \end{equation*}
On the other hand, the existence through lower semi-continuity and the compactness are a delicate matter, since coercivity is not an easy property to obtain (the integrand is similar to a determinant).

There is a trick which is similar to the one we can use to find geodesics in the differential geometry setting. More precisely, we consider the functional
\begin{equation*}E(\phi) := \frac{1}{2} \, \int_{D^2} \left| \nabla \phi \right|^2 \, \mathrm{d}s_1 \, \mathrm{d}s_2. \end{equation*}
If we find a minimal point $\phi$ for $E$, then $\phi$ will be a \textbf{conformal parametrized} minimum for $A$. This trick, on the other hand, heavily depends on a nontrivial theorem: every such $\Sigma$ admits a conformal re-parametrization.

The lack of conformal parametrization, though, is what stop us from extending the same trick to dimension $d$ strictly bigger than $2$.

\paragraph{Higher Dimension.} If the codimension of $\Sigma$ is equal to $1$ (that is, $n = d+1$), then finite perimeter sets generalize the notion of open $(d+1)$-dimensional sets with smooth boundary in $\R^n$.

The class of finite perimeter sets has excellent compactness properties and a notion of area lower semi-continuous. 

This approach is called "weak" surfaces approach, and it is essentially due to Caccioppoli \cite{cacio} and De Giorgi \cite{degi}. A different approach, working for any $d$ and $n$, referred to as \textit{integral currents}, was introduced by Federer and Flaming in their joint paper \cite{fed}.
\chapter{Lie Groups and Lie Algebras} \thispagestyle{empty}

In this chapter, we introduce the notion of \textit{Lie groups} via representation theory, and we describe the local properties employing the concept of \textit{Lie algebras}.

\section{Definitions and Main Properties}

Let $\G$ be a continuous group, whose elements $A(\alpha)$ are expressed as a function of a set of continuous real-valued\footnote{In fact, if $\alpha \in \C$, then it is enough to consider $\alpha = \beta + \imath \gamma$ for $\beta, \, \gamma \in \R$. } parameters $\{\alpha \}_{\alpha \in \Delta} = \{(\alpha_1, \, \alpha_2, \, \dots, \, \alpha_k)\}_{\alpha \in \Delta}$. The parameters are chosen in such a way that
\begin{equation*} A(0, \, \dots, \, 0) = e \end{equation*}
is the identity element of $\G$. 

\begin{definition}[Lie Group] \index{Lie group}A \textit{Lie group} $\G$ is a continuous group, of parameter $\alpha$, satisfying the following properties: \mbox{}
\begin{enumerate}[label=\textbf{(\alph*)}]
\item \textbf{\scshape Closure.} For every $\alpha$ and $\beta$ it turns out that
\begin{equation*} A(\alpha) A(\beta) = A(\gamma), \end{equation*}
where $\gamma = f(\alpha, \, \beta)$ and $f$ is a differentiable function with respect to both variables such that $f(\gamma, \, 0) = \gamma$ and $f(0, \, \gamma) = \gamma$.
\item \textbf{\scshape Inverse.} For every $\alpha$ it turns out that
\begin{equation*} A(\alpha)^{-1} = A(\alpha^\prime), \end{equation*}
where $\alpha^\prime$ is a differentiable function of $\alpha$.
\item \textbf{\scshape Associativity.} For every $\alpha, \, \beta$ and $\gamma$ it turns out that
\begin{equation*} A(\alpha) \left( A(\beta) A(\gamma) \right) = \left( A(\alpha) A(\beta) \right) A(\gamma), \end{equation*}
\end{enumerate}
We say that the group has dimension $r$, since $\alpha$ is a $r$-dimensional vector.
\end{definition}


\begin{remark}Let $\G$ be a Lie group. The associative property immediately implies that
\begin{equation*} f(\alpha, \, f(\beta, \, \gamma)) = f( f(\alpha, \, \beta), \, \gamma) \quad \text{for every $\alpha, \, \beta, \, \gamma$}. \end{equation*} \end{remark}

\subsection{Local Behavior: Lie Algebras}

In this subsection, we shall consider $N$-dimensional representations $\mathfrak{R} := \{\rho(\alpha), \, V_N \}$ of a Lie group $\G$ of parameter $\alpha$ satisfying
\begin{equation*} \rho(0) = \mathrm{Id}_{N \times N}. \end{equation*}
Therefore, we may expand (using Taylor's formula) the representation $\rho(\alpha)$ for $\alpha$ near the null-vector $\mathbf{0}$. It turns out that
\begin{equation} \label{eq.7.3} \rho(\alpha) = \mathrm{Id}_{N \times N} + \imath \, \alpha_a T^a + \dots \quad \text{where $T^a = - \imath \frac{\partial}{ \partial \alpha_a}  \rho(\alpha)  \, \big|_{\{\alpha\} = 0}$}. \end{equation}
The elements $T^a$ are called \textit{generators} of the group $\G$ in the representation $\mathfrak{R}$. 

\begin{example} The trivial representation clearly gives
\begin{equation*} T^a  = 0 \quad \text{for every $a$}. \end{equation*} \end{example}

The set of generators $\{T^a\}_{a = 1, \, \dots}$ of the group $\G$ associated to the representation $\mathfrak{R}$ satisfies the following properties: \mbox{}
\begin{enumerate}[label=\arabic*)]
\item The set $\{T^a\}_{a = 1, \, \dots}$ form a basis of a vector space $\g = \mathrm{Alg}[\G]$, which means that
\begin{equation*}T^a, \, T^b \in \g \implies c_a T^a + c_b T^b \in \g \quad \text{for every $c_a, \, c_b \in \R$}.  \end{equation*}
\item The set $\{T^a\}_{a = 1, \, \dots}$ is closed under the commutations, that is,
\begin{equation*} [T^a, \, T^b] \in \g \quad \text{for every $a, \, b$}. \end{equation*}
\end{enumerate}
The closure under commutations may be rewritten in a different form by choosing the appropriate basis for the vector space $\g$, that is,
\begin{equation} \label{eq.3} [T^a, \, T^b] := \imath f^{abc} T^c. \end{equation}
The constants $f^{abc}$ are usually referred to as \textit{structure constants}\index{structure constants} of the group $\G$ in the literature. It follows from \eqref{eq.3} that
\begin{equation} \label{eq.3.2}  f^{abc} = - f^{bac}.\end{equation}
The basis $\{T^a\}_{a = 1, \, \dots}$ form the so-called \textit{Lie algebra}\index{Lie algebra} $\g$ of the group $\G$, and it is clearly uniquely characterized by the value of the structure constants.

The representation $\mathfrak{R}$ can be chosen among many, but a convenient choice is to consider the \textit{exponential representation}\index{exponential representation} given by
\begin{equation*} \rho(\alpha) = \mathrm{e}^{\imath \alpha_a T^a} =: \mathrm{e}^{\imath \alpha \cdot T}, \end{equation*}
which can be interpreted as the limit of $k$ iterations of the infinitesimal transformation, obtaining the following expression
\begin{equation*}\mathrm{e}^{\imath \alpha_a T_a} = \lim_{k \to + \infty} \left(1 + \imath \, \frac{\alpha_a T^a}{k} \right)^k \end{equation*}
so that formula \eqref{eq.7.3} makes sense.

We now want to prove the necessity of the condition \eqref{eq.3}, which follows from the associative property of $\G$. If we set $\alpha T := \alpha \cdot T = \alpha_a T^a$, then the property \textbf{(a)} of a Lie group implies that
\begin{equation} \label{eq.2} \mathrm{e}^{\imath \alpha T} \mathrm{e}^{\imath \beta T} = \mathrm{e}^{\imath \delta T}  \end{equation}
for some parameter $\delta$. In particular, it follows from \eqref{eq.2} that
\begin{equation*} \begin{aligned}\imath \delta T & = \log \left(\mathrm{e}^{\imath \alpha T} \mathrm{e}^{\imath \beta T}\right) =
\\[1em] & = \log \left(\mathrm{e}^{\imath \alpha T} \mathrm{e}^{\imath \beta T} \pm \mathrm{Id}_{N \times N} \right) \simeq
\\[1em] & \overset{ \alpha, \, \beta \sim 0}{\simeq} \log(\mathrm{Id}_{N \times N} + K) =
\\[1em] & = K - \frac{1}{2} K^2 + \frac{1}{3} K^3 + \dots,  \end{aligned} \end{equation*}
where $K$ denotes the matrix $\mathrm{e}^{\imath \alpha T} \mathrm{e}^{\imath \beta T} - \mathrm{Id}_{N \times N}$. On the other hand, we have
\begin{equation*} \begin{aligned}K & \simeq (\mathrm{Id}_{N \times N} + \imath \alpha T + \dots) (\mathrm{Id}_{N \times N} + \imath \beta T + \dots) - \mathrm{Id}_{N \times N} =
\\[1em] & = \imath \alpha T + \imath \beta T - (\alpha T)(\beta T) - \frac{1}{2} (\alpha T)^2 - \frac{1}{2} (\beta T)^2 + \dots, \end{aligned} \end{equation*}
which means that
\begin{equation*} \begin{aligned} \imath \delta T & = \imath \alpha T + \imath \beta T + \frac{1}{2} [\beta T, \, \alpha T] + \dots = \\[1em] & =  \imath \alpha T + \imath \beta T - \frac{1}{2} [\alpha T, \, \beta T] + \dots \end{aligned}\end{equation*}
since the quadratic terms $(\alpha T)^2$ and $(\beta T)^2$ vanish. In conclusion, for small parameters $\alpha, \, \beta$ and $\gamma$ it turns out that
\begin{equation*} [\alpha_a T^a, \, \beta_b T^b] = - 2 \imath(\delta_c - \alpha_c - \beta_c)T^c =: \imath \gamma_c T^c, \end{equation*}
and thus
\begin{equation*} \gamma_c = - 2(\delta_c - \alpha_c - \beta_c) = \alpha_a \beta_b f^{abc} \implies [T^a, \, T^b] = \imath f^{abc} \, T^c, \end{equation*}
which justifies the definition \eqref{eq.3}.

\paragraph{N.B.} The Lie algebras are subject to a consistency condition
\begin{equation} \label{jacobi} \left[ [T^a, \, T^b], \, T^c \right] + \left[ [T^b, \, T^c], \, T^a \right] + \left[ [T^c, \, T^a], \, T^b \right] = 0, \end{equation}
known as the \textit{Jacobi identity}.

\subsection{Adjoint Representation}\index{Adjoint Representation}

Let $T^a \in \g$ be a Lie algebra. The relation \eqref{eq.3} implies that
\begin{equation*}\left[ [T^a, \, T^b], \, T^c \right] = \imath f^{abd} \left[T^d, \, T^c \right] = \underbrace{\imath^2}_{= -1} f^{abd}f^{dce} T^e,\end{equation*}
and similarly
\begin{equation*}\left[ [T^b, \, T^c], \, T^a \right] = - f^{bcd}f^{dae} T^e \quad \text{and} \quad \left[ [T^c, \, T^a], \, T^b \right] = - f^{cad}f^{dbe} T^e.\end{equation*}
If we plug these identities into the Jacobi identity \eqref{jacobi}, we find that
\begin{equation} \label{eq.4.1} f^{abd}f^{dce} + f^{bcd}f^{dae} + f^{cad}f^{dbe} = 0 \quad \text{for every $a, \, b, \, c, \, e \in \{1, \, \dots, \, r\}$}.\end{equation}
We now consider the $r$ matrices defined by
\begin{equation*} \left( \mathbb{T}^a \right)_{b, \, c} := \imath f^{bac}, \end{equation*}
and we prove that $\{ \mathbb{T}^a \}_{a = 1, \, \dots, \, r}$ are the generators of the group $\G$ in the representation $\mathfrak{R}^\ast$, which is called \textit{adjoint representation} of $\mathfrak{R}$. Indeed, it follows from the definition and formula \eqref{eq.4.1} that
\begin{equation*} \begin{aligned} \left( \left[ \T^a, \, \T^b \right] \right)_{d, \, e} &= \left(\T^a \T^b \right)_{d, \, e} - \left(\T^b \T^a \right)_{d, \, e} =
\\[1em] & = (\T^a)_{d, \, c} (\T^b)_{c, \, e} - (\T^b)_{d, \, c} (\T^a)_{c, \, e} =
\\[1em] & = - f^{dac} f^{cbe} + f^{dbc} f^{cae} =
\\[1em] & \stackrel{(*)}{=} - f^{dac} f^{cbe} - f^{bdc} f^{cae} =
\\[1em] & \stackrel{(**)}{=} - f^{abc} f^{dce} =
\\[1em] & = \imath f^{abc} (\T^c)_{d, \, e},\end{aligned}\end{equation*}
which is exactly the relation \eqref{eq.3}.

The equality (*) follows immediately from the antisymmetric behavior of $f^{abc}$ with respect to the first two coordinates \eqref{eq.3.2}, while the equality (**) follows from \eqref{eq.4.1}.

\subsection{Examples}
In this brief section, we discuss the main examples proposed in the first chapter (e.g., the Euclidean group, the special unitary group, etc.)

\begin{example}[$\mathrm{SO}(2, \, \R)$] The special orthogonal group on $\R^2$ is given by the elements
\begin{equation*} R(\theta) := \begin{pmatrix} \cos \theta & \sin \theta \\ - \sin \theta & \cos \theta \end{pmatrix}, \end{equation*}
and therefore it is a one-parameter continuous group with $\theta \in [0, \, 2 \pi)$. For $\theta \sim 0$ it turns out that
\begin{equation*} R(\theta) \simeq \mathrm{Id}_{2 \times 2} + \imath \theta \begin{pmatrix} 0 & - \imath \\ \imath & 0\end{pmatrix}, \end{equation*}
which means that the unique generator is given by
\begin{equation*} T = \begin{pmatrix} 0 & - \imath \\ \imath & 0\end{pmatrix}.\end{equation*}
The representation $\rho(\theta) := R(\theta)$ is the fundamental one (i.e., the smallest that is not trivial), and it is easy to prove that
\begin{equation*} \mathrm{e}^{\imath \theta T} = R(\theta) \quad \text{for every $\theta \in [0, \, 2\pi)$}.\end{equation*} \end{example}

\begin{example}[$E_2$] Recall that the $2$-dimensional Euclidean transformations of the form \eqref{eq.1} can be easily rewritten in the following way
\begin{equation*} \begin{pmatrix} x \\ y \\ 1 \end{pmatrix} \longmapsto \left(\begin{array}{@{}c|c@{}}
  R(\theta) &
  \begin{matrix}
  b_1 \\
  b_2
  \end{matrix}
\\ \hline
  \begin{matrix}
  0 & 0
  \end{matrix}
  & 1
\end{array}\right) \begin{pmatrix} x \\ y \\ 1 \end{pmatrix}. \end{equation*}
The generators (see \hyperref[ch:e2]{Chapter \ref{ch:e2}}) are given by
\begin{equation*} \begin{cases} T^1 = - \imath \frac{\partial}{\partial x}, \\[1em] T^2 = - \imath \frac{\partial}{\partial y}, \\[1em] R = - \imath \left( x \frac{\partial}{\partial y} - y \frac{\partial}{\partial x} \right), \end{cases} \end{equation*}
since one can easily check that
\begin{equation*} \begin{aligned} & \mathrm{e}^{\imath b_1 T^1} \begin{pmatrix}x \\ y \end{pmatrix} \simeq (1 + \imath b_1 T^1) \begin{pmatrix}x \\ y \end{pmatrix} = \begin{pmatrix}x + b_1 \\ y \end{pmatrix},
\\[1em] & \mathrm{e}^{\imath b_2 T^2} \begin{pmatrix}x \\ y \end{pmatrix} \simeq (1 + \imath b_2 T^2) \begin{pmatrix}x \\ y \end{pmatrix} = \begin{pmatrix}x \\ y + b_2\end{pmatrix},
\\[1em] & \mathrm{e}^{\imath \theta R} \begin{pmatrix}x \\ y \end{pmatrix} \simeq \begin{pmatrix}x \\ y \end{pmatrix} + \begin{pmatrix} \theta y \\ - \theta x \end{pmatrix}. \end{aligned} \end{equation*}
The commutators between these generators are easy to compute,
\begin{equation*} \begin{aligned} & [T^1, \, T^2] = 0, \\[1em] & [T^1, \, R] = - \imath T^2, \\[1em] & [T^2, \, R] = \imath T^1 \end{aligned} \end{equation*}
and therefore, for any $a \in \{1, \, 2, \, R\}$, we have
\begin{equation*} f^{12a} = 0 \qquad \text{and} \qquad f^{1R2} = - f^{2 R 1} = -1. \end{equation*}
\end{example}

\begin{example}[$\mathrm{SU}(2, \, \C)$] The three generators of the special unitary group in the fundamental representation (=smallest nontrivial) are
\begin{equation*} T^a = \frac{1}{2} \tau^a,\end{equation*}
where $\tau^a$ denotes the $a$th Pauli matrix, that is,
\begin{equation*} \tau^1 = \begin{pmatrix} 0 & 1 \\ 1 & 0 \end{pmatrix}, \qquad \tau^2 = \begin{pmatrix} 0 & - \imath \\ \imath & 0 \end{pmatrix}, \qquad \tau^3 = \begin{pmatrix} 1 & 0 \\ 0 & - 1 \end{pmatrix}.\end{equation*}
A straightforward computation proves that $f^{abc}$ is equal to the well-known Levi-Civita tensor\index{Levi-Civita tensor} $\epsilon^{abc}$, that is,
\begin{equation*} f^{abc} = \epsilon^{abc} := \begin{cases} 1 & \text{if $(a b c)$ is an even permutation,} \\ - 1 & \text{if $(a b c)$ is an odd permutation,} \\ 0 & \text{otherwise}. \end{cases}\end{equation*}
\end{example}

\begin{example}[$\mathrm{SU}(N, \, \C)$] The generators of the Lie algebra in the fundamental representation are the $N \times N$ Hermitian matrices with zero trace
\begin{equation*} (T^a)^\dagger = T^a \quad \text{and} \quad \mathrm{Tr}(T^a) = 0 \quad \text{for $a = 1, \, \dots, \, N^2 - 1$}. \end{equation*}
\end{example}

\begin{example}[$\mathrm{SO}(N, \, \R)$] The generators of the Lie algebra in the fundamental  representation are the $N \times N$ antisymmetric matrices
\begin{equation*} (T^a)^T = - T^a \quad \text{for $a = 1, \, \dots, \, \frac{N(N-1)}{2}$.}\end{equation*}
\caution{The Lie group $\mathrm{SO}(3, \, \R)$ is \underline{not} isomorphic to the Lie group $\mathrm{SU}(2, \, \C)$.}The Lie algebra of $\mathrm{SO}(3)$ is isomorphic (=similar behavior near the identity element), as an algebra, to $\mathrm{SU}(2)$, and we denote this with the symbol\footnote{We shall always use the lower case for the Lie algebra associated to a given Lie group that is denoted by a capital symbol.}
\begin{equation*}\mathrm{su}(2) \sim \mathrm{so}(3). \end{equation*}
More precisely, the generators of the fundamental representation of $SO(3, \, \R)$ are
\begin{equation*} T^1 = \begin{pmatrix} 0 & 0 & 0 \\ 0 & 0 & - \imath \\ 0 & \imath & 0 \end{pmatrix}, \qquad T^2 = \begin{pmatrix} 0 & 0 & \imath \\ 0 & 0 & 0 \\ -\imath & 0 & 0 \end{pmatrix}, \qquad T^3 = \begin{pmatrix} 0 & -\imath & 0 \\ \imath & 0 & 0 \\ 0 & 0 & 0 \end{pmatrix},\end{equation*}
and it is immediate to check that
\begin{equation*} [T^a, \, T^b] = \imath \epsilon^{abc} T^c, \end{equation*}
where $\epsilon^{abc}$ is the Levi-Civita tensor introduced above.
\end{example}

\begin{example}[$\mathrm{SL}(N, \, \C)$] The generators of the Lie algebra in the fundamental representation are the $N \times N$ matrices with zero trace, that is,
\begin{equation*}\mathrm{Tr}(T^a) = 0 \quad \text{for $a = 1, \, \dots, \, N^2 - 1$}.\end{equation*}
\end{example}

\begin{example}[$\mathrm{S}_p(2N, \, \C)$] The generators of the Lie algebra in the fundamental (=smallest nontrivial) representation are the $2N \times 2N$ matrices satisfying the following property
\begin{equation*} (T^a)^\dagger J + T^a J = 0,\end{equation*}
where
\begin{equation*} J =
\left(
\begin{array}{c|c}
0 & \mathrm{Id}_{N \times N} \\
\hline
- \mathrm{Id}_{N \times N} & 0
\end{array}
\right). \end{equation*}
\end{example}

\section{Lie Algebra}

In this section, we focus more on the study of a Lie algebra $\g$ associated to a Lie group $\G$.

\subsection{SubAlgebras}
\index{Lie algebra!subalgebra}

Let $X^a \in \g$ be a Lie algebra, and consider a subset $\h \subset \g$. If $\{ Y^{\dot{a}} \}$ is the subset of the generators of $\g$ which belongs also to $\h$, then
\begin{equation*}\left[ Y^{\dot{a}}, \, Y^{\dot{b}} \right] \in \h \quad \text{for every $\dot{a}$ and $\dot{b}$} \implies \text{$\h$ is a subalgebra of $\g$}. \end{equation*}
Clearly, both $\mathbf{0}$ and $\g$ form trivial subalgebras of any algebra $\g$.

\begin{definition}[Ideal]\index{Lie algebra!abelian invariant subalgebra}\index{Lie algebra!invariant subalgebra} Let $X^a \in \g$ be a Lie algebra, and let $\{ Y^{\dot{a}} \}$ be the subset of the generators which belongs to a subalgebra $\h \subset \g$. If
\begin{equation}  \label{eq.4}\left[ Y^{\dot{a}}, \, X^b \right] = c_{\dot{d}} Y^{\dot{d}} \in \h \end{equation}
for every choice of $\dot{a}$ and $b$, then $\{Y^{\dot{a}} \}$ generate an \textit{invariant subalgebra/ideal} of $\g$. \end{definition}

\begin{definition}[Abelian] Let $X^a \in \g$ be a Lie algebra, and let $\h \subset \g$ be an invariant subalgebra. If
\begin{equation} \label{eq.4.4}  \left[ Y^{\dot{a}}, \, Y^{\dot{b}} \right] = 0 \quad \text{for every $Y^{\dot{a}}, \, Y^{\dot{b}} \in \h$}, \end{equation}
then $\h$ is said to be an \textit{abelian invariant subalgebra} (or an abelian ideal) of $\g$. \end{definition}

\begin{definition}[Simple Algebra] \index{Lie algebra!simple} A Lie algebra $\g$ is \textit{simple} if the only invariant subalgebras are the trivial ones, that is,
\begin{equation*} \text{$\h \subset \g$ invariant subalgebra} \implies \text{$\h = \{\mathbf{0}\}$ or $\h = \g$}.\end{equation*}
\end{definition}

\begin{definition}[Semisimple Algebra] \index{Lie algebra!semisimple} A Lie algebra $\g$ is \textit{semisimple} if no invariant subalgebra is abelian, except for the null one.
\end{definition}

\begin{definition}[Center]\index{Lie algebra!center} Let $\g$ be a Lie algebra. The \textit{center} of $\g$ is the set of all the elements $T^{\dot{a}} \in \g$ that commutes with every other element in $\g$, that is,
\begin{equation*} C(\g) := \left\{ T^{\dot{a}} \in \g \: : \: \text{$[T^{\dot{a}}, \, T^b] = 0$ for every $T^b \in \g$} \right\}. \end{equation*}
\end{definition}

\begin{lemma} Let $\g$ be a Lie algebra, and let $\h \subset \g$ be an invariant subalgebra. Then $\h$ generates the invariant subgroup $\Ha \subset \G$. \end{lemma}

\begin{proof} Let $h = \mathrm{e}^{\imath \alpha_{\dot{a}} Y^{\dot{a}} } \in \Ha$ and let $g = \mathrm{e}^{\imath \beta_b X^b } \in \G$: we need to prove that $g^{-1} h g \in \Ha$. By definition, the conjugate is given by
\begin{equation*} g^{-1} h g = g^{-1} \mathrm{e}^{\imath \alpha_{\dot{a}} Y^{\dot{a}} } g  = \mathrm{e}^{\imath \alpha_{\dot{a}}(g^{-1} Y^{\dot{a}} g)}, \end{equation*}
and thus it is enough to compute $g^{-1} Y^{\dot{a}} g$. Now
\begin{equation*} \begin{aligned} g^{-1} Y^{\dot{a}} g & = \mathrm{e}^{- \imath \beta_b X^b } Y^{\dot{a}} \mathrm{e}^{\imath \beta_b X^b } =
\\[1em] & = \left( 1 - \imath \beta X - \frac{1}{2} (\beta X)^2 + \dots \right) Y^{\dot{a}} \left( 1 + \imath \beta X - \frac{1}{2} (\beta X)^2 + \dots \right) =
\\[1em] & = Y^{\dot{a}} - \imath [\beta X, \, Y^{\dot{a}} ] + \frac{ (-\imath)^2 }{2} [\beta X, \, [\beta X, \, Y^{\dot{a}}]] + \dots + \frac{ (-\imath)^n }{n!} [\beta X, \, [ \dots [\beta X, \, Y^{\dot{a}}] \dots ]] + \dots =
\\[1em] & = \gamma_{\dot{c}} Y^{\dot{c}} \in \h,   \end{aligned} \end{equation*}
as a consequence of formula \eqref{eq.4}. To conclude the proof, it remains to give a formal justification of the last equality. Let us consider the function
\begin{equation*}G(t) := \mathrm{e}^{- \imath t \beta_b X^b} Y^{\dot{a}} \mathrm{e}^{\imath t \beta_b X^b}, \end{equation*}
so that the idea is to compute $G(1)$ by means of the Taylor's formula, that is,
\begin{equation*}G(1) = \sum_{n = 0}^{+ \infty} \frac{t^n}{n!} \, G^{(n)}(0). \end{equation*}
We already know that $G(0) = Y^{\dot{a}}$, and it is easy to prove that
\begin{equation*}G^\prime(t) =- \mathrm{e}^{- \imath t \beta_b X^b} \imath [\beta X, \, Y^{\dot{a}}] \mathrm{e}^{\imath t \beta_b X^b}, \end{equation*}
which means that
\begin{equation*}G^\prime(0) = - \imath [\beta X, \, Y^{\dot{a}}] = - \imath \beta_b [X^b, \, Y^{\dot{a}}], \end{equation*}
and the right-hand side is equal to $c_{\dot{d}} Y^{\dot{d}}$ by definition of invariant subalgebra \eqref{eq.4}. In a similar fashion, we notice that
\begin{equation*}G^{(2)}(t) \, \big|_{t = 0} = \frac{ (-\imath)^2 }{2} [\beta X, \, [\beta X, \, Y^{\dot{a}}]], \end{equation*}
and thus by induction we infer that $g^{-1} h g \in \Ha$.\end{proof}

\begin{corollary} The center of a Lie algebra $\g$ generates the center of the Lie group $\G$. \end{corollary}

\section{Killing Form}

In this section, we introduce a metric $g^{ab}$, also called \textit{killing form}, which gives us a compelling criterion to check whether a given Lie algebra is semisimple or not.

\begin{definition}[Killing Form] Let $\g$ be a Lie algebra. We define a metric by setting \index{killing form} 
\begin{equation} \label{kf} g^{ab} := f^{acd} f^{bdc} \quad \text{for every $a, \, b \in \{1, \, \dots, \, r\}$}, \end{equation}
where $r$ is the group dimension.
\end{definition}

\begin{theorem}[Cartan]\label{thm:cartan} A Lie algebra $\g$ is semisimple if $\mathrm{det} \left| g^{ab} \right|$ is nonzero. \end{theorem}

\begin{proof}We may equivalently prove the negation:
\begin{equation*} \textit{"If $\g$ is not a semisimple Lie algebra, then $\mathrm{det} \left| g^{ab} \right| = 0$."} \end{equation*}
Let $\h \subset \g$ be an invariant abelian subalgebra and $T^{\dot{a}} \in \h$ a generator. We have the identity
\begin{equation*} g^{\dot{a}b} = f^{\dot{a}cd} f^{bdc} \stackrel{(*)}{=} f^{\dot{a}c\dot{d}} f^{b\dot{d}c} \stackrel{(**)}{=} f^{\dot{a}\dot{c}\dot{d}}f^{b\dot{d}\dot{c}} = 0\end{equation*}
as a consequence of the following facts: \mbox{}
\begin{enumerate}[label=\alph*)]
\item The equality (*) is a consequence of \eqref{eq.4} because
\begin{equation*} f^{\dot{a}cd} = [T^{\dot{a}}, \, T^c] = c_{\dot{d}} T^{\dot{d}}. \end{equation*}
\item The equality (**) is also a consequence of \eqref{eq.4} applied to $f^{b\dot{d}c}$.
\item The subalgebra $\h$ is abelian by assumption; hence
\begin{equation*} f^{\dot{a}\dot{c}\dot{d}} T_{\dot{d}} = [T^{\dot{a}}, \, T^{\dot{c}}] = 0 \implies f^{\dot{a}\dot{c}\dot{d}}f^{b\dot{d}\dot{c}} = 0. \end{equation*}
\end{enumerate}
It follows that $g^{\dot{a}b} = 0$ for every $b \in \{1, \, \dots, \, r\}$, and therefore the metric form $g$ has at least a row ($\dot{a}$) that is equal to $(0, \, \dots, \, 0)$. \end{proof}

\subsection{Examples}

We now apply the \textit{Cartan criterion} introduced above to check whether the common algebras we are dealing with in this course are semi-simple or not.

\begin{example}[$\mathrm{su}(2, \, \C) \sim \mathrm{so}(3, \, \R)$]Recall that the structure constants of these Lie algebras are given by the Levi-Civita tensor, that is,
\begin{equation*} f^{abc} = \epsilon^{abc} \quad \text{for every $a, \, b, \, c \in \{1, \, 2, \, 3\}$}. \end{equation*}
It follows that
\begin{equation*} g^{ab} := f^{acd} f^{bdc} = \epsilon^{acd}\epsilon^{bdc} = \begin{cases}-2 & \text{if $a = b$}, \\[0.6em] 0 & \text{if $a \neq b$}, \end{cases} \end{equation*}
and thus the Killing form is given by
\begin{equation*} g= \begin{pmatrix} - 2 & 0 & 0 \\ 0 & - 2& 0 \\ 0 & 0 & -2 \end{pmatrix}.\end{equation*}
In conclusion, since $\mathrm{det}|g^{ab}| \neq 0$, it follows from the \hyperref[thm:cartan]{Cartan's criterion} that the Lie algebra $\mathrm{su}(2, \, \C) \sim \mathrm{so}(3, \, \R)$ is semisimple\caution[b][darkgreen][Note]{Actually, the algebra $\mathrm{su}(2, \, \C) \sim \mathrm{so}(3, \, \R)$ is simple, but we will not prove it here. }.  \end{example}

\begin{example}[$\mathrm{so}(2, \, 1)$] The indefinite special orthogonal group\index{indefinite special orthogonal group}, $\mathrm{SO}(2, 1)$ is the subgroup of $\mathrm{O}(2, 1)$ consisting of all elements with determinant 1. More precisely, given
\begin{equation*} g := \begin{pmatrix} 1 & 0 & 0 \\ 0 & 1 & 0 \\ 0 & 0 & - 1 \end{pmatrix}, \end{equation*}
the elements of $\mathrm{SO}(2, 1)$ are the transformations with determinant equal to $1$ that preserves the scalar product $\langle x, \, y \rangle := x^T g y$. The generators\footnote{We shall compute them explicitly later on the course.} of the algebra $\mathrm{so}(2, 1)$ satisfy the following relations
\begin{equation*} \begin{aligned} & [T^1, \, T^2] = \imath T^3, \\[0.8em] & [T^2, \, T^3] = - \imath T^1, \\[0.8em] & [T^3, \, T^1] = \imath T^2, \end{aligned} \end{equation*}
which means that the Killing form is given by
\begin{equation*} g= \begin{pmatrix} - 2 & 0 & 0 \\ 0 & - 2& 0 \\ 0 & 0 & 2 \end{pmatrix}.\end{equation*}
In particular, by \hyperref[thm:cartan]{Cartan's criterion} the algebra $\mathrm{so}(2, \, 1)$ is semisimple.\end{example}

\begin{example}[$E_2$] Recall that the generators of the Lie algebra $\g$ associated to the Euclidean group $E_2$ are given by
\begin{equation*} \begin{cases} T^1 = - \imath \frac{\partial}{\partial x}, \\[1em] T^2 = - \imath \frac{\partial}{\partial y}, \\[1em] R = - \imath \left( x \frac{\partial}{\partial y} - y \frac{\partial}{\partial x} \right), \end{cases} \end{equation*}
which means that the structure constants are
\begin{equation*} f^{12R} = 0, \qquad f^{2R1} = 1, \qquad f^{1R2} =- 1. \end{equation*}
We can easily compute the Killing form from the definition \eqref{kf}, obtaining
\begin{equation*} g^{ab} := f^{acd} f^{bdc} = \begin{pmatrix} - 2 & 0 & 0 \\ 0 & 0 & 0 \\ 0 & 0 & 0 \end{pmatrix} \implies \mathrm{det}|g^{ab}| = 0,\end{equation*}
which means that by \hyperref[thm:cartan]{Cartan's criterion} the algebra associated to $E_2$ is not semisimple. 

More precisely, the reader may check (using the definitions) that $\{T^1, \, T^2\}$ generate an invariant subalgebra $\h \in \g$ that is abelian (see \hyperref[ch:e2]{Chapter \ref{ch:e2}} for a detailed dissertation.)
\end{example}

\section{Casimir Operator}

In mathematics, a \textit{Casimir Operator} is a precise element which lies within the center of a Lie algebra (e.g., the square of the angular momentum modulus in $\mathrm{so}(3, \, \R)$).

Let $\g$ be a \textbf{semisimple} Lie algebra, and let $g^{ab}$ denote its Killing form \eqref{kf}. The matrix $g$ is invertible by Cartan's criterion, and therefore the inverse is well-defined:
\begin{equation*} g_{ab} := (g^{-1})_{ab} \end{equation*}

\begin{definition}\index{Casimir operator} The (quadratic) \textit{Casimir operator} of a semisimple Lie algebra $\g$ is defined by setting
\begin{equation} \label{caso} C := g_{ab}T^aT^b, \end{equation}
where $\{T^a\}$ is the set of generators of $\G$. \end{definition}

\begin{lemma} The Casimir operator $C$ is an element of the center $C(\g)$, that is,
\begin{equation*} [C, \, T^a] = 0 \quad \text{for every $T^a \in \g$}. \end{equation*} \end{lemma}

We also notice that, if we define $c_{abc} := g^{ae} f^{bce}$, then it turns out that $c_{abc} = c_{bca} = c_{cab}$. Moreover, the Casimir operator $C$ takes a constant value in a representation\footnote{We shall see later that this property is a simple consequence of the well-known Schur's Lemma.}, characterizing completely every other representation.

\begin{example}The Casimir operator of the Lie algebra $\mathrm{so}(3, \, \R) \sim \mathrm{su}(2, \, \C)$ is proportional to $T^aT^a$, which means that
\begin{equation*}C \propto (T^1)^2 + (T^2)^2 + (T^3)^2. \end{equation*}
The right-hand side is equal to the modulus squared of the angular momentum.\end{example}
\chapter{The Fundamental Group $\pi_1(M)$} \thispagestyle{empty}

In the previous chapter, we proved that the Lie algebra $\mathrm{so}(3, \, \R)$ is isomorphic to the Lie algebra $\mathrm{su}(2, \, \C)$, as they have the same universal constants.

The goal of this chapter is to introduce a powerful mathematical tool, called \textit{fundamental group}\index{fundamental group}, which will allow us to answer the question
\begin{equation*} \mathrm{SO}(3) \stackrel{?}{\cong} \mathrm{SU}(2). \end{equation*}
Furthermore, we shall explore the connection between the local behavior and the global behavior of these two groups employing the notion of \textit{covering space}\index{covering space}.

\section{Homotopy}

In this chapter, we develop the homotopy theory for a topological space\index{topological space} $M$ (i.e., a space equipped with a topology $\tau$), but the reader with no topology background may think of $M$ as a manifold.

In topology, there is a somewhat intuitive notion that measures, in a certain sense, how similar two (geometrical) objects are.

For example, the union between the circumference $S^1$ and its diameter is intuitively equivalent to the union of two tangent $S^1$. We shall soon be able to give a precise meaning to this statement, and we will also be able to prove it formally.

\begin{figure}[h]
        \centering{ \scalebox{.7}{ 
\begin{tikzpicture}
	\draw (0,0) circle (1.5cm);
	\node (A) at (-1.3, 1.3) {$S^1$};
	\draw[-,ultra thin, red] (-1.5,0)--(1.5,0) node[pos =0.1, below]{$d$};
	\node (B) at (2.5, 0) {$\sim$};
	\draw (5,0) circle (1.2cm);
	\draw (7.4,0) circle (1.2cm);
	\node (B) at (4, 1.2) {$S^1$};
	\node (C) at (8.5, 1.2) {$S^1$};
		[anchor=mid west,
			mark size=+2pt, mark color=red,  ball color=green]
			\foreach \plm[count=\cnt] in {ball}
			\draw[mark options={fill=red}]
			plot[mark=\plm] coordinates {(6.2, 0)} node[right]{$p$};
\end{tikzpicture}
}
	\caption{The union of a circumference with its diameter $S^1 \cup d$ is equivalent to the union $S^1 \cup S^1$ between tangent circumferences.}
	}
\end{figure}

\vspace{1cm}

\subsection{Path-Components}

The goal of this section is to introduce the $\pi_0(M)$, which is the set of all the path-connected components\index{path-connected components} of $M$. We first need to recall some basic topological notions (e.g., connectedness, path-connectedness, connected components, etc.)

\begin{definition}\index{connected set}\index{connectedness} A topological space $M$ is \textit{disconnected} if there are two nonempty proper open sets $A, \, B \subset M$ such that $A \cup B = M$. A topological space is \textit{connected} if it is not disconnected. \end{definition}

\vspace{1cm}

\begin{figure}[h]
        \centering{
        \scalebox{.8}{
\begin{tikzpicture}
	\draw (0,0) circle (1.5cm);
	\node (A) at (-1.3, 1.3) {$S^1$};
	\draw (5,0) circle (1.2cm);
	\draw (8,0) circle (1.2cm);
	\node (B) at (4, 1.2) {$S^1$};
	\node (C) at (9, 1.2) {$S^1$};
\end{tikzpicture}}
	\caption{The circumference $S^1$ is connected and path-connected, while the disjoint union $S^1 \sqcup S^1$ is disconnected and has two connected components.}
	}
\end{figure}

\vspace{1cm}

\begin{definition}\index{path-connected set}\index{path-connectedness} A topological space $M$ is \textit{path-connected} if for every $x, \, y \in M$ there exists a continuous path $\alpha : [0, \, 1] \longrightarrow M$ such that $\alpha(0) = x$ and $\alpha(1) = y$.\end{definition}

\begin{remark}A path-connected topological space $M$ is also connected. \end{remark}

\begin{definition}\index{connected component}Let $M$ be a topological space. A subset $C \subset M$ is a \textit{connected component} of $M$ if the following properties are satisfied: \mbox{}
\begin{enumerate}[label=\textbf{(\arabic*)}]
\item $C$ is connected.
\item $C$ is maximal with respect to the inclusion. Namely, if $C \subseteq A$ and $A$ is connected, then $A = C$.
\end{enumerate} \end{definition}

\begin{definition}[Locally Connected] \index{locally connected}A topological space $M$ is \textit{locally connected} if every point $x \in M$ admits a neighborhood basis made up of connected open sets.  \end{definition}

\begin{remark}A connected topological space $M$ needs not to be a locally connected space. \end{remark}

We are finally ready to introduce the notion of $\pi_0(M)$. Let $M$ be a topological space, and let us consider the equivalence relation defined by
\begin{equation*}x \sim y \iff \text{$\exists \: \alpha : [0, \, 1] \longrightarrow M$ continuous such that $\alpha(0) = x$ and $\alpha(1) = y$}. \end{equation*}
It is an easy exercise to prove that $\sim$ is actually an equivalence relation. The product between two paths can be defined by
\begin{equation*} \alpha \ast \beta(t) := \begin{cases} \alpha(2t) & \text{if $0 \leq t \leq 1/2$}, \\[0.8em] \beta(2t - 1) & \text{if $1/2 \leq t \leq 1$}, \end{cases} \end{equation*}
while the inverse of a path is given by
\begin{equation*} i(\alpha)(t) := \alpha(1 - t). \end{equation*}
The set $\pi_0(M)$ is the set of all the equivalence classes of $M$ with respect to $\sim$. \caution{The set $\pi_0(M)$, in general, \underline{is not a group!}}These equivalence classes are called \textit{path-connected components}\index{path-connected components} of $M$.

\begin{definition}[Locally Path-Connected] \index{locally path-connected}A topological space $M$ is \textit{locally path-connected} if every point $x \in M$ admits a neighborhood basis made up of path-connected open sets.  \end{definition}

In particular, it turns out that two topological spaces (or manifolds/Lie groups) $M$ and $N$ are not homeomorphic if the $\pi_0(\cdot)$'s are different. Unfortunately, both $\mathrm{SO}(3)$ and $\mathrm{SU}(2)$ are path-connected and locally path-connected topological groups, and hence
\begin{equation} \label{eq.4.5} \pi_0( \mathrm{SO}(3) ) = \pi_0 ( \mathrm{SU}(2) ), \end{equation}
which means that we need to introduce a more sophisticated tool to distinguish them, which will turn out to be the fundamental group $\pi_1( \cdot)$.

\subsection{Homotopy}

We now introduce an equivalence relation between continuous maps, called \textit{homotopy}, that will make more precise the meaning of \eqref{eq.4.5}. In the next section, we will refine the notion of homotopy to present the fundamental group finally.

\begin{definition}[Homotopy] \index{homotopic}\index{homotopy} Two continuous maps $f, \, g : M \longrightarrow N$ between topological space are \textit{homotopic} if there exists a continuous map
\begin{equation*} F : M \times [0, \, 1] \longrightarrow N, \end{equation*}
called \textit{homotopy}, such that $F(x, \, 0) = f(x)$ and $F(x, \, 1) = g(x)$ for every $x \in M$. \end{definition}

\begin{example}Two continuous maps $f$ and $g$, defined on a convex set $C \subset \R^n$, are always homotopic. Indeed, it suffices to consider the homotopy
\begin{equation*} F(x, \, t) := (1-t) f(x) + t g(x) \quad \text{for $x \in C$ and $t \in [0, \, 1]$}. \end{equation*} \end{example}

\paragraph{N.B.} The notion of homotopy defines on $C^0(M; \, N)$, the set of all continuous maps between $M$ and $N$, an equivalence relation $\sim$ which is given by
\begin{equation*} f \sim g \iff \text{there exists a continuous homotopy $F$ between $f$ and $g$} \end{equation*}
The formal proof that $\sim$ is actually an equivalence relation is very similar to the one we presented above for paths, and hence we will not write it down here. It is interesting to notice that the homotopy is stable under product (=composition), that is,
\begin{equation*} f_0 \sim f_1 \quad \text{and} \quad g_0 \sim g_1 \implies f_0 \ast g_0 \sim f_1 \ast g_1, \end{equation*}
where $f_0, \, f_1 : M \longrightarrow N$ and $g_0, \, g_1 : N \longrightarrow P$ are continuous mappings between topological spaces.

\begin{definition}[Homotopic Equivalence] \index{homotopic equivalence}A continuous mapping $f : M \longrightarrow N$ between topological spaces is a \textit{homotopic equivalence} if there exists a continuous map $g : N \longrightarrow M$ such that
\begin{equation*} f \circ g \sim \mathrm{id}_N \quad \text{and} \quad g \circ f \sim \mathrm{id}_M. \end{equation*}
Furthermore, two topological space are said to be \textit{homotopic equivalent} if there exists a homotopic equivalence between them. \end{definition}

If $M$ and $N$ are homeomorphic topological spaces (or, in our case, $\G$ and $\G^\prime$ are isomorphic topological groups), then they are also homotopic equivalent.

The fundamental result of this section is the following one. If $f : M \longrightarrow N$ is a homotopic equivalence, then it turns out that $\pi_0(M)$ is isomorphic to $\pi_0(N)$, which means that
\begin{equation*} \pi_0(M) \neq \pi_0(N) \implies \text{$M$ and $N$ are not homotopic equivalent} \implies M \not \cong N, \end{equation*}
and this explains the importance of this notion.

\begin{definition}[Contractible] \index{contractible topological space}A topological space $X$ is \textit{contractible} if it is homotopic equivalent to a point. Equivalently, $X$ is contractible if the identity map $\mathrm{id}_X$ is homotopic to a constant map.\end{definition}

\section{The Fundamental Group}

Let $\alpha : [0, \, 1] \longrightarrow M$ be a closed path (that is, $\alpha(0) = \alpha(1) = x_0$.) In this section, we shall finally refine the notion of homotopy for closed paths of base point $x_0$, and introduce the fundamental group $\pi_1(M, \, x_0)$. 

\subsection{Path Homotopy}

Let $x, \, y \in M$ be two points in a topological space. We define the set of all continuous paths between $x$ and $y$ as follows:
\begin{equation*} \Omega(M, \, x, \, y) := \left\{ \alpha : [0, \, 1] \longrightarrow M \: : \: \text{$\alpha$ is continuous, $\alpha(0) = x$ and $\alpha(1) = y$} \right\}. \end{equation*}

\begin{definition}[Path Homotopy] \index{path homotopic}\index{path homotopy} Two continuous paths $\alpha, \, \beta \in \Omega(M, \, x, \, y)$ are \textit{path homotopic} if there exists a continuous map
\begin{equation*} F : [0, \, 1] \times [0, \, 1] \longrightarrow M, \end{equation*}
called \textit{path homotopy}, such that
\begin{equation*} \begin{aligned} & \text{$F(t, \, 0) = \alpha(t)$ and $F(t, \, 1) = \beta(t)$} \quad \text{for every $t \in [0, \, 1]$}, \\[1em]
& \text{$F(0, \, s) = x$ and $F(1, \, s) = y$} \quad \text{for every $s \in [0, \, 1]$}.\end{aligned} \end{equation*} \end{definition}

The second condition can be easily rewritten as follows. If we let $F_s(\cdot) := F(\cdot, \, s)$, then we are requiring that the continuous path $F_s$ belongs to $\Omega(M, \, x, \, y)$ for every $s \in [0, \, 1]$.

The reader may prove easily that the existence of a path homotopy is also an equivalence relation, denoted by $\sim$, in the set $\Omega(M, \, x, \, y)$. Moreover, both the product of (concatenation) paths and the inverse element commute with the homotopy equivalence relation, which means that
\begin{equation*} \alpha_0 \sim \alpha_1 \quad \text{and} \quad \beta_0 \sim \beta_1 \implies \alpha_0 \ast \beta_0 \sim \alpha_1 \ast \beta_1, \end{equation*}
where $\alpha_i \in \Omega(M, \, x, \, y)$ and $\beta_i \in \Omega(M, \, y, \, z)$, and
\begin{equation*} \alpha_0 \sim \alpha_1 \implies i(\alpha_0) \sim i(\alpha_1). \end{equation*}
The product $\ast$ is associative up to homotopy, that is, given $\alpha \in \Omega(M, \, x, \, y)$, $\beta \in \Omega(M, \, y, \, z)$, and $\gamma \in \Omega(M, \, z, \, w)$, it turns out that
\begin{equation*}(\alpha \ast \beta) \ast \gamma \sim \alpha \ast (\beta \ast \gamma),\end{equation*}
which means that we have the equality as equivalence classes:
\begin{equation*}\left[ (\alpha \ast \beta) \ast \gamma \right] = \left[ \alpha \ast (\beta \ast \gamma) \right]\end{equation*}
In a similar fashion, one can prove that given $\alpha \in \Omega(M, \, x, \, y)$ and $\beta \in \Omega(M, \, x, \, y)$ it turns out that
\begin{equation*} \begin{aligned} & \mathbf{x} \ast \alpha \sim \alpha \ast \mathbf{y} \sim \alpha \implies [\mathbf{x} \ast \alpha] = [\alpha \ast \mathbf{y}] = [\alpha], \\[1em] & \alpha \ast i(\alpha) \sim \mathbf{x}  \implies [\alpha \ast i(\alpha)] = [\mathbf{x}], \end{aligned} \end{equation*}
where $\mathbf{x}$ and $\mathbf{y}$ denote the constant mappings $\mathbf{x}(t) := x$ and $\mathbf{y}(t) := y$ for all $t \in [0, \, 1]$ respectively.

\subsection{The Fundamental Group}

The \textit{fundamental group}\index{fundamental group} of a topological space (or manifold) $M$ with base point $x_0 \in M$ is given by the set of all the equivalence classes $[\alpha]$ for $\alpha \in \Omega(M, \, x_0, \, x_0)$ closed path, and $\sim$ path homotopic equivalence.

\begin{theorem}The set $\pi_1(M, \, x_0)$ endowed with the path product $\ast$ is a group, where the identity element is the path $\mathbf{x_0}$, and the inverse element is given by $i(\cdot)$. \end{theorem}

The group $\pi_1(M, \, x_0)$ does not depend on $x_0$ as an individual point, but rather on the entire path-connected component containing $x_0$. Hence, if $M$ is a path-connected space
\begin{equation*} \pi_1(M, \, x_0) \cong \pi_1(M, \, y_0) \quad \text{for every $x_0, \, y_0 \in M$}, \end{equation*}
and thus we can drop the notation $\pi_1(M, \, x_0)$ and simply write $\pi_1(M)$.

Since both $\mathrm{SO}(3)$ and $\mathrm{SU}(2)$ are path-connected, this simple remark will be extremely useful in the last part of this chapter.

\begin{definition}[Simply Connected]\index{simply connected space}A topological space $M$ is \textit{simply connected} if $M$ is path-connected and $\pi_1(M) = \{e\}$ is the trivial group. \end{definition}

\subsection{Examples}

We will develop more the theory of the fundamental group in the next section after we have introduced the notion of \textit{covering space}. Here we present some examples of $\pi_1(-)$ that can be computed explicitly with some efforts.

\begin{theorem}The fundamental group of $S^1$ is isomorphic to $\Z$. \end{theorem}

\begin{proof}This result is highly nontrivial, and a proof can be found in \cite[pp. 29--32]{hatcher}. \end{proof}

\begin{theorem}\label{thm.sc1} The fundamental group of $S^n$ is trivial for every $n \geq 2$. \end{theorem}

\begin{proof}This assertion follows from a straightforward application of the Van Kampen's theorem. The interested reader may consult \cite[pp. 43--52]{hatcher} for a more detailed discussion. \end{proof}

\begin{theorem}The fundamental group of the torus $\T$ is isomorphic to $\Z \times \Z$. \end{theorem}

\begin{proof}The torus $\T$ is isomorphic to $S^1 \times S^1$, and therefore it is enough to show that
\begin{equation*} \pi_1(M \times N) \cong \pi_1(M) \times \pi_1(N) \end{equation*}
for path-connected topological spaces $M$ and $N$.\end{proof}

\begin{theorem}Let $M$ be a contractible manifold (or topological space). Then the fundamental group of $M$ is trivial. \end{theorem}

Before we can talk about our last, fundamental, example, we need to briefly introduce the $n$-dimensional real projective space\footnote{The projective space is a smooth manifold, i.e. a manifold of class $C^\infty$.} $\R \p^n$. We consider the equivalence relation
\begin{equation*} x \sim_\imath y \iff \exists \: \lambda \in \R \setminus \{0\} \: : \: x = \lambda y, \end{equation*}
and we define the real projective space as the quotient
\begin{equation*} \R \p^n := \faktor{\R^{n + 1} \setminus \{0\} }{\sim_\imath}. \end{equation*}
Intuitively, the projective space is the set of all the lines through the origin in $\R^{n + 1}$, and therefore one can easily prove that we also have
\begin{equation*} \R \p^n := \faktor{S^n }{\sim_\imath}, \end{equation*}
where $S^n := \left\{x \in \R^{n + 1} \: : \: |x| = 1 \right\}$ and $x \sim_\imath y$ if and only if $x = - y$.

\begin{theorem} \label{thm.sc2} The fundamental group of the real projective space $\R \p^n$ is isomorphic to $\Z_2$ (=$\mathcal{C}_2$) for every $n \geq 2$. \end{theorem}

\begin{proof}The reader may consult \cite[pp. 71--73]{hatcher}.  \end{proof}

\section{Covering Space}

\begin{definition}[Covering Space] \index{covering space} A \textit{topological covering} is a continuous surjective map of topological spaces
\begin{equation*} p : \widetilde{M} \longrightarrow M \end{equation*}
 such that, for every $x \in M$, there exists an open neighborhood $U_x \subset M$ of $x$, such that
\begin{equation*} p^{-1} (U_x) = \bigsqcup_{i \in \mathcal{I}} U_i, \end{equation*}
where $\{U_i\}_{i \in \mathcal{I}}$ is a disjoint collection (eventually infinite) of open sets $U_i \subset \widetilde{M}$ such that
\begin{equation*} p \, \big|_{U_i} : U_i \longrightarrow U_x \end{equation*}
is a homeomorphism\footnote{A homeomorphism $f : X \longrightarrow Y$ is an invertible continuous map between topological spaces such that $f^{-1} : Y \longrightarrow X$ is also continuous. } for every $i \in \mathcal{I}$.  \end{definition}

The space $\widetilde{M}$ is called \textit{total space}, the space $M$ is the \textit{base space}, and the sets $p^{-1}(x)$ are the \textit{fibers} of the covering $p$.

\begin{definition}[Degree] \index{covering space!degree} If every fiber of $p : \widetilde{M} \longrightarrow M$ is finite and of cardinality $d$, we say that $p$ is a covering of degree $d$. If the cardinality is infinite, we simply say that $p$ is a covering of infinite ($\infty$) degree. \end{definition}

\begin{example}[Circle] The (universal) covering of $S^1$ is given by
\begin{equation*}\R \ni t \longmapsto \mathrm{e}^{2 \pi \imath \cdot t} \in S^1, \end{equation*}
and it degree is equal to infinity.  \end{example}

\begin{example}[Complex Polynomial] The map
\begin{equation*}\C \setminus \{0\} \ni z \longmapsto z^n \in \C \setminus \{0\}, \end{equation*}
is a covering of degree $n$ for every $n \geq 1$.  \end{example}

\begin{example}[Projective Space] \label{ex.4.1} Let $n \geq 2$. The natural projection
\begin{equation*}\pi : S^n \longrightarrow \faktor{S^n}{\sim_\imath} = \R \p^n\end{equation*}
that sends a point $x$ to its equivalence class $[x] := \{x, \, -x\}$ is a covering of degree two. \end{example}

\section{Application: $\mathrm{SU}(2, \, \C) \not \cong \mathrm{SO}(3, \, \R)$}

In this final section, the goal is to prove that $\mathrm{SU}(2, \, \C)$ is isomorphic to the sphere $S^3$ and $\mathrm{SO}(3, \, \R)$ is isomorphic to the real projective space $\R \p^3$. From \hyperref[thm.sc1]{Theorem \ref{thm.sc1}} and \hyperref[thm.sc2]{Theorem \ref{thm.sc2}} it will follow easily that
\begin{equation*} \pi_1 \left( \mathrm{SU}(2, \, \C) \right) = \{e\} \neq \Z_2 \cong \pi_1 \left( \mathrm{SO}(3, \, \R) \right), \end{equation*}
and thus $\mathrm{SU}(2, \, \C) \not \cong \mathrm{SO}(3, \, \R)$. Moreover, we will also be able to find a covering
\begin{equation*}\pi : \mathrm{SU}(2, \, \C) \longrightarrow\mathrm{SO}(3, \, \R) \end{equation*}
of degree $2$, that is, a surjective map between the two groups which is also $2$-to-$1$.

We will not give an entirely formal proof of the following results, but we will only describe the main ideas behind them and leave it to the reader to fill in the details.

\begin{theorem}The special orthogonal group $\mathrm{SO}(3, \, \R)$ is isomorphic to $\R \p^3$. \end{theorem} \caution[b][darkgreen][Note]{Actually, the same proof shows that $\mathrm{SO}(3, \, \R)$ is diffeomorphic to $\R \p^3$.}

\begin{proof} The group $\mathrm{SO}(3, \, \R)$ consists in all the rotations of $\R^3$, and these are characterized uniquely by the choice of an oriented vector $\vec{v} \in S^2$ and an angle $\theta \in [0, \, 2\pi)$. If we denote by $R(\vec{v}, \,\theta)$ a rotation, it is easy to prove that
\begin{equation*} R(\vec{v}, \, \pi) = R(- \vec{v}, \, \pi) \quad \text{and} \quad R(\vec{v}, \, 0) = R(\vec{w}, \, 0), \end{equation*}
which means that we can define a mapping
\begin{equation*} \varphi : \mathrm{SO}(3, \, \R) \longrightarrow \faktor{S^3}{\sim_\imath} \end{equation*}
that sends $R(\vec{v}, \, \theta)$ to the equivalence class $[v \cdot \theta]$. The reader can easily prove using the definition that $\varphi$ is continuous, bijective, and its inverse $\varphi^{-1}$ is also continuous. \end{proof}

\begin{theorem}\label{thm.5.2} The special unitary group $\mathrm{SU}(2, \, \C)$ is isomorphic to $S^3$. \end{theorem}
\caution[b][darkgreen][Note]{Actually, the same proof shows that $\mathrm{SU}(2, \, \C)$ is diffeomorphic to $S^3$.}

\begin{proof} The $3$-sphere $S^3$ in $\R^4$ can be identified with the complex sphere
\begin{equation*} S_\C^1 := \left\{ (z, \, w) \in \C^2 \: : \: |z|^2 + |w|^2 = 1 \right\}, \end{equation*}
and thus it suffices to prove that $S_\C^1 \cong \mathrm{SU}(2, \, \C)$. On the other hand, one can easily check that
\begin{equation} \label{eq.4.100} U \in \mathrm{SU}(2) \implies U = \begin{pmatrix}z & - \bar{w} \\ w & \bar{z} \end{pmatrix} \quad \text{and} \quad \mathrm{det}(U) = |z|^2 +|w|^2 = 1. \end{equation}
If we identify the unitary matrix \eqref{eq.4.100} with the symbol $U_{z,  \,w}$, then we can easily define a mapping
\begin{equation*} \psi : \mathrm{SU}(2, \, \C) \longrightarrow S_\C^1 \end{equation*}
that sends $U_{z, \, w}$ to $(z, \, w) \in \C^2$. Clearly, $\psi$ is invertible and its inverse is given by
\begin{equation*} \psi^{-1} : S_\C^1 \ni (z, \, w) \longmapsto \begin{pmatrix}z & - \bar{w} \\ w & \bar{z} \end{pmatrix} \in \mathrm{SU}(2, \, \C), \end{equation*}
and, as the reader may check by herself, both are continuous.\end{proof}

\section{Monodromy Group}

Let $p : M\longrightarrow N$ be a covering, and let $x, \, y \in N$ be any two points in the base space. The \textit{monodromy}\index{monodromy mapping} mapping associated to $p$ is given by
\begin{equation*}\mathrm{Mon} : p^{-1}(x) \times \Omega(N, \, x, \, y) \longrightarrow p^{-1}(y), \qquad \mathrm{Mon}(e, \, \alpha) := \alpha_e(1), \end{equation*}
where $\alpha_e : I \longrightarrow M$ is the unique path lifting\index{path lifting}\footnote{Let $p : M \longrightarrow N$ be a covering and $\alpha : I \longrightarrow N$ a path. A path $\gamma : I \longrightarrow M$ is a lifting of $\alpha$ if and only if the diagram is commutative, that is $p \circ \gamma = \alpha$.} of $\alpha$ such that $\alpha_e(0) = e$. For any path $\beta \in \Omega(N, \, y, \, z)$ it is easy to prove that
\begin{equation*}(\alpha \ast \beta)_e = \alpha_e \ast \beta_{\alpha_e(1)}, \end{equation*}
where $\ast$ is the path multiplication. It follows that
\begin{equation} \label{eq.5.1} \mathrm{Mon}(e, \, \alpha \ast \beta) = \mathrm{Mon} \left( \mathrm{Mon}(e, \, \alpha), \, \beta \right). \end{equation}
We will not prove it here, but the monodromy $\mathrm{Mon}(e, \, \alpha)$ depends only on the homotopy class of the path $\alpha$, and hence
\begin{equation*} \mathrm{Mon} \left(e, \, \alpha \ast i(\alpha) \right) = \mathrm{Mon}(e, \, \mathbf{x}) = e. \end{equation*}
It follows that for any $\alpha \in \Omega(N, \, x, \, y)$, the mappings
\begin{equation*}p^{-1}(x) \longrightarrow p^{-1}(y), \qquad e \longmapsto \mathrm{Mon}(e, \, \alpha) \end{equation*}
is bijective, whose inverse is given by
\begin{equation*}p^{-1}(y) \longrightarrow p^{-1}(x), \qquad e \longmapsto \mathrm{Mon}(e, \, i(\alpha)) \end{equation*}
If $x = y$, then the monodromy mapping associated to $p$ acts on the $\pi_1(N, \, x)$, that is,
\begin{equation*}\mathrm{Mon} : p^{-1}(x) \times \pi_1(N, \, x) \longrightarrow p^{-1}(y), \qquad \mathrm{Mon}(e, \, [\alpha]) := \alpha_e(1), \end{equation*}
where $[\alpha]$ denotes the equivalence class of $\alpha$. The monodromy mapping sends the constant path $\mathbf{x}$ to the identity element in $p^{-1}(x)$, and similarly from \eqref{eq.5.1} we infer that
\begin{equation*}\mathrm{Mon}(e, \, [\alpha \ast \beta]) = \mathrm{Mon} \left( \alpha_e(1), \, [\beta] \right) = \beta_{\alpha_e(1)}(1). \end{equation*}
More intuitively, the monodromy map permutes the points in the fiber $p^{-1}(x) = \{ a_j \}_{j \in \mathcal{J}}$. In fact, given a closed loop $[\alpha] \in \pi_1(N, \, x)$, the lifting $\alpha_{a_j}$, for some $j \in \mathcal{J}$, does not necessarily satisfies $\alpha_{a_j}(1) = a_j$, but it could happen that $\alpha_{a_j}(1) = a_k$ for a $k \in \mathcal{J}$ different from $j$.

In particular, we associate a permutation matrix $\sigma(\alpha)$, called \textit{monodromy matrix}\index{monodromy matrix}, to the path $\alpha$ in such a way that
\begin{equation*} \begin{pmatrix} a_1 \\ a_2 \\ \vdots \end{pmatrix} \xrightarrow{\alpha_{a_j}}\begin{pmatrix} a_{\sigma(1)} \\ a_{\sigma(2)} \\ \vdots \end{pmatrix} = \begin{pmatrix} & & \\ & \sigma(\alpha) & \\ & & \end{pmatrix} \begin{pmatrix} a_1 \\ a_2 \\ \vdots \end{pmatrix}. \end{equation*}
The square matrix $\sigma(\alpha)$ has a number of row/column equal to the grade of the covering $p$, which means that it could be infinite-dimensional. The monodromy matrix depends only on the homotopy class of $\alpha$, and we also have that
\begin{equation*} \sigma( \alpha \ast \beta ) = \sigma(\alpha) \sigma(\beta) \end{equation*}
as a consequence of \eqref{eq.5.1}. It follows that
\begin{equation*} \left\{ \sigma(\alpha) \: \left| \: [\alpha] \in \pi_1(N, \, x) \right. \right\} \end{equation*}
is a matrices group, which gives us a representation of the fundamental group $\pi_1(N, \, x)$. We denote it by $\sigma(\pi_1(N, \, x))$, and we call it \textit{monodromy group}\index{monodromy group}.

\subsection{Examples}

In this section, we investigate some simple examples of covering spaces, and we compute the respective monodromy groups.

\begin{example}[Circle] Recall that the (universal) covering of $S^1$ is given by
\begin{equation*}\R \ni t \longmapsto \mathrm{e}^{2 \pi \imath \cdot t} \in S^1, \end{equation*}
and the fundamental group of $S^1$ is isomorphic to $\Z$. The integer $m \in \Z$ corresponding to a closed loop $[\alpha] \in \pi_1(S^1)$ is, intuitively, equal to the "number of laps", which means that the monodromy matrix is simply given by a translation
\begin{equation*} \sigma(m) : x \longmapsto x + m \quad \text{for $m \in \Z \cong \pi_1(S^1)$}.\end{equation*} 
Therefore, the monodromy group is the additive group of integer translation, i.e. $\sigma(\pi_1(S^1)) \cong \Z$ once again.\end{example}

\begin{example}We consider the covering of degree $n$
\begin{equation*}\C \setminus \{0\} \ni z \longmapsto z^n \in \C \setminus \{0\}, \end{equation*}
and we identify $\C \setminus \{0\}$ with $S^1$ in the usual way. The fundamental group of $S^1$ is still $\Z$, but in this case the monodromy matrix is slightly different since an entire "lap" ($m = 1$) corresponds to a rotation of $2 \pi/n$ of the $n$th roots of the unity. Namely, we have
\begin{equation*}\sigma(1) = \begin{pmatrix} 0 & \dots & 0 & 1 \\ 1 & \dots & 0 & 0 \\ \vdots & \ddots & \vdots& \vdots \\ 0 & \dots & 1 & 0 \end{pmatrix}, \quad \dots, \quad  \sigma(n-1) = \begin{pmatrix} 0 & 1 & \dots & 0  \\ \vdots & \vdots & \ddots& \vdots \\ 0 & 0 & \dots & 1 \\1 & 0 & \dots & 0  \end{pmatrix},\end{equation*} 
and it is easy to prove that $\sigma(i + n) = \sigma(i)$ for every $i \in \{0, \, \dots, \, n-1\}$. In particular, the monodromy group $\sigma(\pi_1(S^1))$ is isomorphic to the cyclic group $\Z_n$ (=$\mathcal{C}_n$).\end{example}

\begin{example}Let $n \geq 2$. We notice in the previous sections that the natural projection
\begin{equation*}\pi : S^n \longrightarrow \faktor{S^n}{\sim_\imath} = \R \p^n\end{equation*}
that sends a point $x$ to its equivalence class $[x] := \{x, \, -x\}$ is a covering of degree two. The fundamental group of $\R \p^n$ is isomorphic to $\Z_2$, which means that the monodromy group consist in only two elements:
\begin{equation*} \sigma(0)(x) = x \quad \text{and} \quad \sigma(1)( \pm x) = \mp x. \end{equation*}
In particular, the monodromy group $\sigma(\pi_1(\R \p^n))$ is also isomorphic to $\Z_2$. \end{example}

\section{Higher-Order Homotopy Groups}

Homotopy theory begins with the notion of generalized homotopy group $\pi_n(M)$ for $n \geq 2$. These allow us to "distinguish", in a certain sense, the class of homeomorphism of topological spaces (i.e., Lie groups) as follows. If $M$ and $N$ are two topological groups such that
\begin{equation*} \exists \, i \geq 0 \: : \: \pi_i(M) \not \cong \pi_i(N), \end{equation*}
then one can infer that $M \not \cong N$. Unfortunately, the opposite assertion
\begin{equation*}  \pi_i(M) \cong \pi_i(N) \quad \text{for all $i \geq 0$} \implies M \cong N \end{equation*}
is false, as one can easily check by taking $M := \R$ and $N = \{x_0\}$. Furthermore, the higher-order homotopy groups are usually hard to compute than the fundamental group, due to the fact that the Van Kampen's theorem does not hold for $\pi_n$, $n \geq 2$.

Before we give the formal definition of $\pi_n(M)$ for any $n \geq 2$, we take a look at some of the computations of the group $\pi_i(S^n)$ presented in \cite{toda}.

\begin{figure}[h!]
\begin{flushleft}
\begin{tikzpicture}
\clip node (m) [matrix,matrix of nodes,
fill=black!20,inner sep=0pt,
nodes in empty cells,
nodes={minimum height=1cm,minimum width=1cm,anchor=center,outer sep=0,font=\sffamily},
row 1/.style={nodes={fill=black,text=white}},
column 1/.style={nodes={fill=gray,text=white,align=center,text width=1.5cm,text depth=0.5ex}},
column 2/.style={text width=0.5cm,align=center,every even row/.style={nodes={fill=white}}},
column 3/.style={text width=0.5cm,align=center,every even row/.style={nodes={fill=white}},},
column 4/.style={text width=0.5cm,align=center,every even row/.style={nodes={fill=white}},},
column 5/.style={text width=0.5cm,align=center,every even row/.style={nodes={fill=white}},},
column 6/.style={text width=0.5cm,align=center,every even row/.style={nodes={fill=white}},},
column 7/.style={text width=0.5cm,align=center,every even row/.style={nodes={fill=white}},},
column 8/.style={text width=1.5cm,align=center,every even row/.style={nodes={fill=white}},},
column 9/.style={text width=1.5cm,align=center,every even row/.style={nodes={fill=white}},},
column 10/.style={text width=1.5cm,align=center,every even row/.style={nodes={fill=white}},},
column 11/.style={text width=1.5cm,align=center,every even row/.style={nodes={fill=white}},},
column 12/.style={text width=1cm,align=center,every even row/.style={nodes={fill=white}},},
column 13/.style={text width=1.5cm,align=center,every even row/.style={nodes={fill=white}},},
row 1 column 1/.style={nodes={fill=gray}},
prefix after command={[rounded corners=4mm] (m.north east) rectangle (m.south west)}
] {
     $n/i$     & 1 & 2 &3 & 4 & 5 & 6 & 7 & 8 & 9 & 10 & 11 & 12  \\
$1$     & $\Z$ & $0$ & $0$ & $0$ &  $0$ & $0$ & $0$ & $0$ & $0$ & $0$ & $0$ & $0$  \\
$2$      &  $0$ & $\Z$ & $\Z$ & $\Z_2$ & $\Z_2$ & $\Z_{12}$ & $\Z_2$ &$\Z_2$ & $\Z_3$& $\Z_{15}$& $\Z_2$ & $\Z_2 \times \Z_2$ \\
$3$      &  $0$ & $0$ & $\Z$ & $\Z_2$ & $\Z_2$ & $\Z_{12}$ & $\Z_2$ &$\Z_2$ & $\Z_3$& $\Z_{15}$& $\Z_2$ & $\Z_2 \times \Z_2$ \\
$4$      &  $0$ & $0$ & $0$ & $\Z$ & $\Z_2$ & $\Z_{2}$ & $\Z \times \Z_{12}$ &$\Z_2 \times \Z_2$ & $\Z_2 \times \Z_2 $& $\Z_{24} \times \Z_3$& $\Z_{15}$ & $\Z_2$ \\
$5$      &  $0$ & $0$ & $0$ & $0$ & $\Z$ & $\Z_{2}$ & $\Z_2$ &$\Z_{24}$ & $\Z_2 $& $\Z_{2}$& $\Z_{2}$ & $\Z_{30}$ \\
$6$      &  $0$ & $0$ & $0$ & $0$ & $0$ & $\Z$ & $\Z_2$ &$\Z_{2}$ & $\Z_{24} $& $0$& $\Z$ & $\Z_{2}$ \\
$7$      &  $0$ & $0$ & $0$ & $0$ & $0$ & $0$ & $\Z$ &$\Z_{2}$ & $\Z_{2} $& $\Z_{24}$& $0$ & $0$ \\
$8$      &  $0$ & $0$ & $0$ & $0$ & $0$ & $0$ & $0$ &$\Z$ & $\Z_{2} $& $\Z_2$& $\Z_{24}$ & $0$ \\
};
\end{tikzpicture}
\end{flushleft}
\caption{The \LaTeX code of this table can be found \href{https://tex.stackexchange.com/questions/67586/how-to-create-comparison-tables-in-latex}{here}.}
\end{figure}

\vspace{1.5mm}
We shall follow closely \cite[Section 4.1]{hatcher} from now on. The table above shows a lot of peculiar properties, e.g., the subdiagonal is zero, and indeed $\pi_i(S^n) = 0$ for all $i < n$. Also, the diagonal is given by a sequence of $\Z$, as a consequence of the Hurewicz theorem, which asserts that for a simply-connected space ($\pi_1(S^n) = 0$ for all $n \geq 2$), the first nonzero homotopy group $\pi_n(S^n)$ is isomorphic to the homology group $H_n(S^n)$.

Another interesting property is that along each diagonal the groups $\pi_{n + k}(S^n)$ with $k$ fixed and $n$ varying eventually become independent of $n$ for a large enough $n$.

\subsection{Definitions and Basic Properties}

Let $I^n$ be the $n$-dimensional unit cube $[0, \, 1]^n$. The boundary $\partial I^n$ is the set of all point $p \in I^n$ such that at least one of the coordinates is either $0$ or $1$.

For a topological space $X$ and a base point $x_0 \in X$, we define the $n$th homotopy group $\pi_n(X, \, x_0)$ to be the set of homotopy classes of continuous maps
\begin{equation*} \alpha : I^n \longrightarrow X, \qquad \partial I^n \longmapsto x_0, \end{equation*}
where a homotopy $H$ is admissible if and only if $H_t(\partial I^n) = x_0$ for every $t \in [0, \, 1]$.

It is immediate to verify that, if we take $n := 1$, then we obtain the definition of the fundamental group $\pi_1(X, \, x_0)$. For $n \geq 2$, a sum operation in $\pi_n(X, \, x_0)$ is defined, as a generalization of the path product $\alpha \ast \beta$ in the fundamental group, as follows:
\begin{equation*}(\alpha + \beta)(x_1, \, \dots, \, x_n) = \begin{cases} \alpha(2x_1, \, x_2, \, \dots, \, x_n) & \text{if $x_1 \in [0, \, 1/2]$}, \\ \beta(2x_1 - 1, \, x_2, \, \dots, \, x_n) & \text{if $x_1 \in [1/2, \, 1]$}. \end{cases} \end{equation*}
The sum is well-defined on homotopy classes and, since there is only a coordinate involved ($x_1$), it easily turns out that $\pi_n(X)$ is a group with inverse element
\begin{equation*}i(\alpha)(x_1, \, \dots, \, x_n) = \alpha(1 - x_1, \, x_2, \, \dots, \, x_n). \end{equation*}
We use the additive notation ($i(\alpha) = - \alpha$ and $\alpha + \beta$) because the homotopy group $\pi_n(X)$ is abelian for every $n \geq 2$. Namely, we have that
\begin{equation*} \alpha + \beta \sim \beta + \alpha \quad \text{i.e.} \quad [\alpha + \beta] = [\beta + \alpha] \end{equation*}
via the following homotopy (see \cite[pp 340]{hatcher}). Assume that $\mathrm{dom}(\alpha) = [0, \, 1/2] \times [0, \, 1]^{n-1}$ and $\mathrm{dom}(\beta) = [1/2, \, 1] \times [0, \, 1]^{n-1}$. 

The homotopy begins with $\alpha + \beta$ by shrinking the domain of $\alpha$ and $\beta$ to subcubes of $I^n$ that are well-separated, with the region outside these domains mapping to the base point $x_0$. After this, there is room to slide the two subcubes anywhere around $I^n$ as long as they stay disjoint, so if $n \geq 2$ they can be slid past each other, interchanging their initial positions. In conclusion, the domains of $\alpha$ and $\beta$ can be enlarged to their original size obtaining $\beta + \alpha$.

\subsection{Equivalent Definitions and Base Point}

A continuous map
\begin{equation*} \alpha : I^n \longrightarrow X, \qquad \partial I^n \longmapsto x_0, \end{equation*}
can easily be identified with quotient maps
\begin{equation*} \alpha : S^n =\faktor{I^n}{\partial I^n} \longrightarrow X, \qquad s_0 = \faktor{\partial I^n}{\partial I^n} \longmapsto x_0.\end{equation*}
This means that we can also view $\pi_n(X, \, x_0)$ as homotopy classes of maps $(S^n, \, s_0) \to (X,\, x_0)$, where homotopies are through maps of the same form $(S^n, \, s_0) \to (X,\, x_0)$. In this equivalent interpretation, the sum is given by the composition
\begin{equation*}S^n \xrightarrow{c} S^n \vee S^n \xrightarrow{f \vee g} X, \end{equation*}
where $c$ collapses the equator $S^{n - 1} \subset S^n$ to a point $s_0$, obtaining the wedge $S^n \vee S^n$, and $f \vee g$ is the wedge map.

We shall now enlist a few intuitive and useful properties of higher-order homotopy group. The reader interested in the proof of these statements and in a more systematic investigation of homotopy groups, may consult \cite[Chapter 4]{hatcher}.

\begin{proposition} Let $X$ be a path-connected topological space. Then
\begin{equation*} \pi_n(X, \, x_0) \cong \pi_n(X, \, y) \quad \text{for every $y \in X$ and $n \in \N$}. \end{equation*}
In particular, if $X$ is path-connected we shall always write $\pi_n(X)$ in place of $\pi_n(X, \, x_0)$. \end{proposition}

\begin{proposition} Let $\{X_\alpha\}_{\alpha}$ be a collection of path-connected topological spaces. Then
\begin{equation*} \pi_n( \prod_\alpha X_\alpha ) \cong \prod_\alpha \pi_n(X_\alpha) \quad \text{for every $n \in \N$}. \end{equation*} \end{proposition}

\subsection{Higher-Order Homotopy Groups in Physics}

In this final section, we give a table of explicitly computed higher-order homotopy groups related to Lie groups of fundamental importance in physics (e.g., $\mathrm{SO}(n, \, \R)$). These are stable groups, for which the homotopy groups repeat themselves periodically, as one can see from the table below:

\begin{figure}[h]
\centering
\begin{tikzpicture}
\clip node (m) [matrix,matrix of nodes,
fill=black!20,inner sep=0pt,
nodes in empty cells,
nodes={minimum height=1cm,minimum width=1cm,anchor=center,outer sep=0,font=\sffamily},
row 1/.style={nodes={fill=black,text=white}},
column 1/.style={nodes={fill=gray,text=white,align=center,text width=1.5cm,text depth=0.5ex}},
column 2/.style={text width=0.5cm,align=center,every even row/.style={nodes={fill=white}}},
column 3/.style={text width=0.5cm,align=center,every even row/.style={nodes={fill=white}},},
column 4/.style={text width=0.5cm,align=center,every even row/.style={nodes={fill=white}},},
column 5/.style={text width=0.5cm,align=center,every even row/.style={nodes={fill=white}},},
column 6/.style={text width=0.5cm,align=center,every even row/.style={nodes={fill=white}},},
column 7/.style={text width=0.5cm,align=center,every even row/.style={nodes={fill=white}},},
column 8/.style={text width=0.5cm,align=center,every even row/.style={nodes={fill=white}},},
column 9/.style={text width=0.5cm,align=center,every even row/.style={nodes={fill=white}},},
row 1 column 1/.style={nodes={fill=gray}},
prefix after command={[rounded corners=4mm] (m.north east) rectangle (m.south west)}
] {
     $i$ mod $8$     & 1 & 2 &3 & 4 & 5 & 6 & 7 & 8  \\
$\pi_i \mathrm{O}(n)$     &$\Z_2$&$\Z_2$&$0$&$\Z$&$0$&$0$&$0$&$\Z$  \\
$\pi_i \mathrm{U}(n)$     &$0$&$\Z$&$0$&$\Z$&$0$&$\Z$&$0$&$\Z$  \\
$\pi_i \mathrm{S}_p(n)$     &$0$&$0$&$0$&$\Z$&$\Z_2$&$\Z_2$&$0$&$\Z$  \\
};
\end{tikzpicture}
\caption{The \LaTeX code of this table can be found \href{https://tex.stackexchange.com/questions/67586/how-to-create-comparison-tables-in-latex}{here}.}
\end{figure}
\chapter{Other measures and dimensions}

In this chapter, we first introduce the \textit{integralgeometric} measure, and then we investigate the Haar invariant $k$-dimensional measure.

In the second half, we show how the Haar measure may be used to define an invariant measure on the Grassmannian manifold $G(n, \, m)$, which will be extremely useful to study rectifiable sets.

\section{Geometric Integral Measure}

In this first section, we propose an alternative $k$-dimensional measure to the Hausdorff one and, at the same time, we exhibit a motivational example for introducing the notion of \textit{invariant measures}.

\paragraph{Definition ($1$-dimensional).} Let $E \subseteq \R^n$ be a subset, and fix a projection
\begin{equation*} P_L : \R^n \longrightarrow L \end{equation*}
onto a $1$-dimensional linear subspace (i.e., a line). We may easily define (see \hyperref[fig:012dkos02]{Figure \ref{fig:012dkos02}}) a measure, which is invariant under translation but not under rotations, as follows:
\begin{equation} \label{mea1} \int_{x \in L} \can \left( P_L^{-1}(x) \cap E \right) \, \mathrm{d} \mathcal{H}^1(x), \end{equation}
where $\can ( \cdot )$ denotes the cardinality function.

In order to define a rotation-invariant measure, the simplest idea that comes in mind is to consider the average value of \eqref{mea1}, as $L$ ranges in the set of all the $1$-dimensional subspaces of $\R^n$. More precisely, we "define" the $1$-dimensional integralgeometric\index{measure!integralgeometric} measure as follows:
\begin{equation} \label{mea11} \mathcal{I}^1(E) := \dashint_{L \in G(n, \, 1)} \left[ \int_{x \in L} \can \left( P_L^{-1}(x) \cap E \right) \, \mathrm{d} \mathcal{H}^1(x) \right]. \end{equation}
The measure \eqref{mea11} is not well-defined, since we do not know which measure we need to use to integrate over all $L \in G(n, \, 1)$; we will come back to this issue later.

\newpage

\begin{figure}[h]
\centering
\includegraphics[width=14cm, height=8cm]{images/TGMMMS1.png}
\label{fig:012dkos02}
\caption{One-dimensional translation-invariant measure.}
\end{figure}

\paragraph{Definition ($k$-dimensional).} The same construction can be easily generalized to a $k$-dimensional measure. Indeed, if we denote by $G(n, \, k)$ the Grassmannian manifold (i.e, the set of all $k$-dimensional subspaces of $\R^n$) then, it turns out that
\begin{equation} \label{meak} \mathcal{I}^k(E) := c_{k, \, n} \dashint_{V \in G(n, \, k)} \left[ \int_{x \in V} \can \left( P_V^{-1}(x) \cap E \right) \, \mathrm{d} \mathcal{H}^k(x) \right] \end{equation}
is a measure, which is invariant under translations and rotations. The reader may prove that, if the renormalization constant $c_{k, \, n}$ is chosen properly, then
\begin{equation*} \mathcal{I}^k(E) = \mathcal{L}^k(E) = \mathcal{H}^k(E), \end{equation*}
provided that $E$ is contained in a $k$-hyperplane of $\R^n$.

\paragraph{Definition Issue.} The $k$-dimensional integralgeometric measure \eqref{meak} is not well-defined (as we have already mentioned in the $1$-dimensional case.) Indeed, we are taking the average integral over all the elements of the Grassmannian manifold $(n, \, k)$, but we have not introduced yet a measure on that space that is also invariant.

This issue is the main reason why we are so interested in developing (at least partially) the theory of \textit{invariant measures}. At the end of the chapter, we will be able to prove the existence of an invariant measure $\gamma_{n, \, k}$ on the Grassmannian manifold $(n, \, k)$.

\paragraph{Basic Properties.} To conclude this introduction we give a list of relevant and compelling facts about the $k$-dimensional geometric integral measure and leave them as exercises for the reader.

\begin{lemma} Let $E$ be a subset of a $h$-dimensional surface $\Sigma$. Assume that $h > k$ strictly, and that the $h$-dimensional volume of $\Sigma$ is positive. Then $\mathcal{I}^k(E) = + \infty$. \end{lemma}

\begin{remark}The $k$-dimensional Hausdorff measure is, in general, different from the $k$-dimensional geometric integral measure.

More precisely, it may happen that for some subset $E \subset \R^n$, the Hausdorff dimension is $\mathrm{dim}_{\mathcal{H}}(E) > k$, but $\mathcal{I}^k(E) = 0$. \end{remark}

\begin{example}[Cantor Set] Let us consider the set $\mathcal{C}_2 \subset \R^2$ given by the product of two Cantor-type set with scaling factor equal to $1/4$ (see \hyperref[fig:cs]{Figure \ref{fig:cs}}.) 

In particular, we have the similarities $\Phi_1, \, \dots, \, \Phi_4$, all with scaling factor equal to $1/4$, in such a way that $\mathcal{C}_2$ is a self-similar fractal set in the sense of Hutchinson, that is,
\begin{equation*} \mathcal{C}_2 = \bigcup_{i =1}^4 \Phi_i(\mathcal{C}_2). \end{equation*}
By \hyperref[th:hutchisnds]{Theorem \ref{th:hutchisnds}}, the Hausdorff dimension of $\mathcal{C}_2$ is the unique solution to the equation
\begin{equation*} 4 \left(\frac{1}{4} \right)^d = 1 \implies d = 1, \end{equation*}
but the reader may prove, as an exercise, that $\mathcal{I}^1( \mathcal{C}_2) = 0$. \end{example}

\begin{figure}[h]
\centering
\includegraphics[width=12cm, height=6cm]{images/CS.png}
\label{fig:cs}
\caption{Cantor Square}
\end{figure}

\section{Invariant Measures on Topological Groups}

In this section, we examine the assumptions needed for the existence and uniqueness of an invariant measure defined on a topological group $\G$.

\begin{definition}[Push-Forward] \index{measure!push-forward} Let $\mu$ be a positive measure on $X$, and let $f : X \longrightarrow Y$ be a Borel function. The \textit{push-forward} measure of $\mu$ via $f$ is defined by setting
\begin{equation*}f_{\can}\mu(E) := \mu \left( f^{-1}(E) \right), \qquad \forall \, E \in \mathcal{B}(Y). \end{equation*} \end{definition}

\begin{lemma} Let $\left(X, \, \mathcal{B}(X) \right)$ and $\left(Y, \, \mathcal{B}(Y) \right)$ be measurable spaces, and let $\mu$ be a positive measure on $X$. Then the push-forward $f_\can \mu$ is a well-defined measure on the the Borel $\sigma$-algebra of $Y$. \end{lemma}

\paragraph{Topological Groups.} Let $\G$ be a topological group. For any $y \in \G$, we denote by $\tau_y$ the left-multiplication ($x \mapsto y \cdot x$) and by $\tau_y^\ast$ the right-multiplication ($x \mapsto x \cdot y$).

\begin{definition}[Invariant Measure] \index{measure!invariant} Let $\mu$ be a measure defined on a topological group $\G$. The measure $\mu$ is left-invariant on $\G$ if and only if
\begin{equation*} \left( \tau_y \right)_{\can} \mu = \mu\qquad \forall \, y \in \G. \end{equation*} 
In a similar fashion, the measure $\mu$ is right-invariant if and only if
\begin{equation*} \left( \tau_y^\ast \right)_{\can} \mu = \mu\qquad \forall \, y \in \G, \end{equation*} 
and, clearly, $\mu$ is \textit{invariant} if and only if $\mu$ is both left-invariant and right-invariant.
\end{definition}

We are now ready to state the existence and uniqueness results. We will not prove the second theorem in this course, but we will give, at the end of the section, a complete sketch of the proof of the first result (which is the only one we shall be using in this course).

\begin{theorem}\label{theorem:1das} Let $\G$ be a compact group. Then there exists a unique invariant probability measure on $\G$, called Haar measure. \end{theorem}

\begin{theorem}Let $\G$ be a locally compact and separable group. Then there exists a locally finite invariant measure on $\G$, which is unique up to a multiplicative constant. \end{theorem}

\paragraph{Grassmannian Manifold.} The Grassmannian $G(n, \, k)$ is, in general, not a group. Therefore, we cannot apply to it the theorem stated above, but there is another way around since, as we shall see soon, we can identify $G(n, \, k)$ with a quotient $\faktor{\G}{\Gamma}$, where $\Gamma$ satisfies certain properties.

Let $\G$ be a topological group acting on a set $X$, and let $\tau : \G \times X \longrightarrow X$ be the left action map, that is,
\begin{equation*} \tau(g, \, x) = g \cdot x. \end{equation*}
In order to be coherent with the previous paragraph, we denote by $\tau_g(x)$ the element $\tau(g, \, x)$ so that
\begin{equation*} \tau_{g_1 \cdot g_2} = \tau_{g_1} \tau_{g_2}. \end{equation*}
A measure $\mu$ defined on $X$ is \textit{invariant} under the action of $\G$ if and only if
\begin{equation*} \left( \tau_g \right)_{\can} \mu = \mu \end{equation*}
for every $g \in \G$. However, in this general setting, the existence (let alone the uniqueness) of an invariant measure is not guaranteed and, actually, we can exhibit an easy counterexample assuming both $\G$ and $X$ compact.

\begin{example}Let $X = \p^1(\R) \cong \R \cup \{0\}$ and let $\G$ be the group of all the projective transformation of $X$. Since the translation group is contained in $\G$, the only possible invariant measure is the $1$-dimensional Lebesgue measure. On the other hand, the Lebesgue measure is not invariant under homothety, and hence there are no invariant measures. \end{example}

\begin{theorem}\label{theorem:3osds}Let $\G$ be a topological group acting on a set $X$. An invariant measure (not necessarily unique) exists if one of the following conditions is satisfied: \mbox{}
\begin{enumerate}[label=\textbf{(\arabic*)}]
\item The group $\G$ is abelian (or, more generally, it satisfies the Weyl condition\footnote{We do not need to deal with this delicate condition, but the interested reader may find more information \href{https://en.wikipedia.org/wiki/Weyl_group}{here}.}.)
\item The set $X$ is isomorphic to the quotient $\faktor{\G}{H}$, where $H$ is a closed subspace\footnote{Notice that $H$ does not need to be a normal subgroup of $\G$.} of $\G$.
\end{enumerate} \end{theorem}

\begin{remark} If $\mu$ is a left-invariant measure on $\G$ and $\pi : \G \longrightarrow \faktor{\G}{H}$ is a projection onto a closed subset, then the push-forward $\pi_{\can} \mu$ is also an invariant measure.\end{remark}

The non-oriented Grassmannian manifold $G(n, \, k)$ is diffeomorphic to a certain quotient, so that, by \hyperref[theorem:3osds]{Theorem \ref{theorem:3osds}}, an invariant measure $\gamma_{n, \, k}$ exists. In particular, we will finally be able to show that the integralgeometric $k$-dimensional measure \eqref{meak} is well-defined.

\begin{lemma} Let $O(m)$ denote the group of the orthogonal $m\times m$ matrices. Then there is a diffeomorphism
\begin{equation*} G(n, \, k) \cong \faktor{O(n)}{\left(O(k) \times O(n - k) \right)}, \end{equation*}
where $O(k) \times O(n - k)$ is the space of orthogonal matrices made up of a $k\times k$ orthogonal block and a $(n-k)\times (n-k)$ orthogonal block. \end{lemma}

The reader may work out the details of the proof by themselves, but the intuitive idea behind it is simple: Given $W \in G(n, \, k)$ element of the Grassmannian manifold, consider an orthonormal basis $w_1, \, \dots, \, w_k$ for it. Complete it to an orthonormal basis $w_1, \, \dots, \, w_n$ of $\R^n$ and, at this point, define an equivalence class naturally (that is, up to change of orthonormal basis for $W$ and the complement of $W$ separately.)

In conclusion, as promised, we sketch the proof of \hyperref[theorem:1das]{Theorem \ref{theorem:1das}} in the special case of $\G$ Lie group, and we give a complete proof (of the existence, at least) in the case of $\G$ commutative).

\begin{proof}[Proof 1] Let $\G$ be a $k$-dimensional Lie group. The idea is to define a left-invariant $k$-form $\omega$, that is, a $k$-form such that the pull-back according to $\tau_y$ is $\omega$ itself. Then it suffices to check that
\begin{equation*} \mu(E) := \int_E \omega \end{equation*}
is the sought invariant measure, and also that it is unique. \end{proof}

\begin{proof}Let $\G$ be a commutative group, and let $\mathcal{P}$ be the space of probability measures defined on $\G$. For any $g \in \G$, set
\begin{equation*} \mathcal{P}_g := \left\{ \mu \in \mathcal{P} \: \left| \: \left(\tau_g\right)_{\can} \mu = \mu \right. \right\} \end{equation*}
be the subset of $\mathcal{P}$ containing all the $g$-invariant probability measures defined on $X$.

\paragraph{Step 1.} We want to prove that, for every $g \in \G$, the subset $\mathcal{P}_g$ is nonempty. Fix $\mu_0 \in \mathcal{P}$ and let us consider, for every $n \in \N$, the probability measure defined by setting
\begin{equation*} \mu_n := \frac{\mu_0 + \left(\tau_{g} \right)_{\can} \mu_0 + \dots + \left(\tau_{g^n} \right)_{\can} \mu_0}{n+1} \in \mathcal{P}, \end{equation*}
where $g^{n}$ denotes the product of $n$ copies of $g$.

By compactness there exists a subsequence $\mu_{n_k}$ weakly-$\ast$ converging to a measure $\mu_\infty$. We now claim that $\mu_\infty$ is a $\tau_g$-invariant probability measure. Indeed, by definition of $\mu_n$, it follows that
\begin{equation*} \left(\tau_g \right)_{\can} \mu_{n_k} \to \mu_\infty \implies \left(\tau_g\right)_{\can} \mu_\infty = \mu_\infty.  \end{equation*}

\paragraph{Step 2.} We want to prove that the intersection of all the $\mathcal{P}_g$ is nonempty, which is clearly enough to infer the existence of an invariant measure.

Let $g, \, h \in \G$ be two elements, let $\mu_0 \in \mathcal{P}_g$ be an invariant measure, and let $\mu_\infty$ be the weakly-$\ast$ limit of the sequence
\begin{equation*} \mu_n := \frac{\mu_0 + \left(\tau_{h} \right)_{\can} \mu_0 + \dots + \left(\tau_{h^n} \right)_{\can} \mu_0}{n+1} \in \mathcal{P}. \end{equation*}
The set $\mathcal{P}_g$ is weakly-$\ast$ closed; therefore $\mu_\infty \in \mathcal{P}_g \cap \mathcal{P}_h$. By induction we can prove that the family $\{\mathcal{P}_g\}_{g \in \G}$ has the finite intersection property, and thus, by compactness of $\G$, it immediately follows that
\begin{equation*} \bigcap_{g \in \G} \mathcal{P}_g \neq \varnothing. \end{equation*}

\paragraph{Step 3.} We want to prove that the intersection above contains only one element. In order to do that, we define the convolution product of two measures by setting
\begin{equation*} \mu_1 \ast \mu_2 (E) := \left(\mu_1 \times \mu_2 \right) \left( \{ (x_1, \, x_2) \: \left| \: x_1 + x_2 \in E \right. \} \right). \end{equation*}
The reader may prove that the convolution is commutative, and also that
\begin{equation*} \mu_1 \ast \mu_2 = \mu_1,\end{equation*}
if $\mu_1$ is an invariant measure.

This is enough to infer that the invariant measure is unique. Indeed, if $\lambda, \, \mu \in \cap_{g \in \G} \mathcal{P}_g$ are two invariant measures, then the properties above of the convolution implies that
\begin{equation*} \mu = \mu \ast \lambda = \lambda \ast \mu = \lambda \implies \mu = \lambda. \end{equation*}
\end{proof}
\chapter{Representation Theory} \thispagestyle{empty}

In this chapter, we develop the theory of representations we need later to investigate in-depth properties of $\mathrm{SU}(2, \, \C)$, $\mathrm{SU}(3, \, \C)$, $\mathrm{SO}(4, \, \R)$, the Euclidean groups, the Lorentz group, the Poincaré group, etc.

\section{Introduction}

First, we recall some definitions we have already introduced in the introductive chapter.

\begin{definition}[Representation] \index{representation} A \textit{representation} of a group $\G$ on a $N$-dimensional vector space $V$ over a field $\mathbb{K}$ is a group homomorphism
\begin{equation*} \rho : \G \longrightarrow \mathrm{GL}(N, \, \mathbb{K}; \; V), \end{equation*}
that is, a mapping $\rho$ such that $\rho(g_1 g_2) = \rho(g_1) \cdot \rho(g_2)$ for every $g_1, \, g_2 \in \G$, where $\cdot$ denotes the matrices product. \end{definition}

In particular, the identity element $e \in \G$ is represented by the identity matrix $\mathrm{Id}_{N \times N}$, while the inverse element $g^{-1}$ is represented by the inverse matrix $\rho(g)^{-1}$.

Moreover, every group $\G$ admits a trivial $1$-dimensional representation\index{trivial representation}, which is defined in the following way:
\begin{equation*}\rho(g) := (1)\quad\text{for all $g \in \G$}. \end{equation*}

\begin{definition}[Similarity] \index{representation!equivalence} Let $\G$ be a group, and let $\{\rho, \, V\}$ and $\{\widetilde{\rho}, \, \widetilde{V} \}$ be two $N$-dimensional representations of $\G$. We say that they are \textit{similar} (or, equivalent) if and only if there exists a regular (=invertible) matrix $S$ such that
\begin{equation*} \widetilde{\rho}(g) = S \rho(g) S^{-1} \quad \text{for every $g \in \G$}. \end{equation*}
\end{definition}

\begin{definition}[Unitary Representation] \index{representation!unitary} A $N$-dimensional representation $\{ \rho, \, V \}$ of a group $\G$ is \textit{unitary} if and only if $\rho(g) \in \mathrm{U}(N, \, \C)$ for every $g \in \G$. \end{definition}

\begin{example}Recall that $\mathfrak{S}_3$ is the group of all permutations of $3$ elements, that is,
\begin{equation*} \mathfrak{S}_3 = \left\{e, \, (123), \, (132), \, (12), \, (13), \, (23) \right\}. \end{equation*}
There is an obvious three dimensional representation $\rho$ that acts as
\begin{equation*} \rho( (123) ) \begin{pmatrix}a \\ b \\ c \end{pmatrix} = \begin{pmatrix} b \\ c \\ a \end{pmatrix}, \end{equation*}
and does a similar work with any other element of $\mathfrak{S}_3$. They are explicitly given by the usual permutation matrices, that is,
\begin{equation*} \begin{aligned} & \rho(e) = \begin{pmatrix} 1 & 0 & 0 \\ 0 & 1 & 0 \\ 0 & 0 & 1 \end{pmatrix}, \qquad  \rho((123)) = \begin{pmatrix} 0 & 0 & 1 \\ 1 & 0 & 0 \\ 0 & 1 & 0 \end{pmatrix}, \qquad  \rho((132)) = \begin{pmatrix} 0 & 1 & 0 \\ 0 & 0 & 1 \\ 1 & 0 & 0 \end{pmatrix},
\\[1.5em] & \rho((12)) = \begin{pmatrix} 0 & 1 & 0 \\ 1 & 0 & 0 \\ 0 & 0 & 1 \end{pmatrix}, \qquad  \rho((13)) = \begin{pmatrix} 0 & 0  & 1 \\ 0 & 1 & 0 \\ 1 & 0 & 0 \end{pmatrix}, \qquad  \rho((23)) = \begin{pmatrix} 1 & 0 & 0 \\ 0 & 0 & 1 \\ 0 & 1 & 0 \end{pmatrix}.\end{aligned} \end{equation*} \end{example}

\begin{example}[Direct Product]\index{representation!direct product} Let $\{ \rho, \, V \}$ and $\{ \sigma, \, W \}$ be  representations of two groups $\G$ and $\G^\prime$ respectively. Recall that the direct product $\G \otimes \G^\prime$ is defined as the group with underlying set the Cartesian product $\G \times \G^\prime$ and component-wise multiplication, i.e.,
\begin{equation*} (g, \, h) \oplus (g^\prime, \, h^\prime) = (g g^\prime, \, h h^\prime) \in \G \times \G^\prime. \end{equation*}
There is an obvious representation of the direct product that is given by $\{ \rho \oplus \sigma, \, V \oplus W \}$, where $V \oplus V$ is the direct sum of vector spaces, and
\begin{equation*}\rho \oplus \sigma(g, \, h) := \rho(g) \oplus \sigma(h). \end{equation*}
The reader may quickly check that $\rho \oplus \sigma$ is a group homeomorphism, i.e. it preserves the group structure. \end{example}

\section{Irreducible Representations}

A \textit{diagonal block matrix}\index{diagonal block matrix} is a $n \times n$ matrix $M$ of the form
\begin{equation*}M = \left( \begin{array}{c|c|c|c} M_1  & 0 & 0 & 0 \\ \hline 0 & M_2 & 0 & 0 \\ \hline 0 & 0 & \ddots & 0 \\ \hline0 & 0 & 0 & M_k \end{array} \right) \end{equation*}
where $M_i$ is a $n_i \times n_i$ matrix for all $i \in \{1, \, \dots, \,k \}$ and $\sum_{i = 1}^k n_i = n$.

\begin{definition}[Irreducible Representation] \index{representation!irreducible}\index{representation!reducible} A representation $\{ \rho, \, V\}$ of a group $\G$ is \textit{reducible} if it is equivalent (via a regular matrix $S$) to a diagonal representation, that is,
\begin{equation*}S \rho(g) S^{-1} = M(g) = \left( \begin{array}{c|c|c|c} M_1(g)  & 0 & 0 & 0 \\ \hline 0 & M_2(g) & 0 & 0 \\ \hline 0 & 0 & \ddots & 0 \\ \hline0 & 0 & 0 & M_{k}(g) \end{array} \right) \quad \text{for every $g \in \G$}.\end{equation*}
A non-reducible representation is usually referred to as \textit{irreducible representation}.\end{definition}

\begin{definition}[$\G$-Invariant] \index{invariant $\G$-space} Let $\{ \rho, \, V \}$ be a representation of $\G$. A linear subspace $W \subset V$ is \textit{$\G$-invariant} if
\begin{equation*} \rho(g)w \in W \quad \text{for every $g \in \G$ and $w \in W$}. \end{equation*}   \end{definition}

\begin{definition}[Subrepresentation] \index{subrepresentation} Let $\{ \rho, \, V \}$ be a representation of $\G$, and let $W \subset V$ be a $\G$-invariant subspace. The representation $\{ \rho \, \big|_{W}, \, W \}$ is called \textit{subrepresentation} of $\{\rho, \, V\}$.\end{definition}

\begin{remark}A representation $\{ \rho, \, V\}$ of a group $\G$ is reducible if and only if there are $\G$-invariant nontrivial subspaces $W_1, \, \dots, \, W_k \subset V$ such that
\begin{equation*} W_1 \oplus \dots \oplus W_k = V. \end{equation*}\end{remark} 

\begin{definition}[Complex Conjugate]\index{representation!complex conjugate} Let $\{ \rho, \, V\}$ be a representation of a group $\G$ over a complex vector space $V$. The \textit{complex conjugate representation} $\{ \rho^\ast, \, V^\ast\}$ is defined by
\begin{equation*} g \longmapsto \rho^\ast(g) := - (\rho(g))^\ast, \end{equation*} 
where $M^\ast$ is the complex conjugate of the matrix $M$.\end{definition}

\begin{definition}[Complex/Real Representation] \index{representation!complex}\index{representation!real} A representation $\{\rho, \, V\}$ is \textit{real} if it is equivalent via a \textbf{unitary} matrix $S$ to the representation $\{ \rho^\ast, \, V^\ast\}$, that is,
\begin{equation*}S \rho(g) S^{-1} = \rho^\ast(g) \quad \text{for every $g \in \G$}.\end{equation*}
Furthermore, we say that a representation is \textit{complex} if it is not real.\end{definition}

\subsection{Schur Lemmas}

In this section, we prove the fundamental theorem due to Schur, known as \textit{Schur lemma}, for irreducible representations, and we use it to show interesting properties of irreducible (complex) representations of a group $\G$.

\begin{lemma}[Schur] \label{schur1} Let $\{ \rho, \, V\}$ and $\{ \rho^\prime, \, W \}$ be two irreducible representations of a group $\G$. If there exists a linear mapping $A : V \longrightarrow W$ such that
\begin{equation} \label{eq.5.1} A \rho(g) = \rho^\prime( Ag ) \quad \text{for every $g \in \G$}, \end{equation}
then $A$ is either $0$ or an isomorphism of vector spaces. \end{lemma}

\begin{proof}We assume that $A \neq 0$, and we show that $A$ is both injective and surjective.

\paragraph{Injective.} The kernel $K := \mathrm{ker} A$ is clearly $\G$-invariant since
\begin{equation*}x \in K \implies A \rho(x) = \rho^\prime(Ax) = \rho^\prime(0) = 0 \implies \rho(x) \in K. \end{equation*}
The representation $\{ \rho, \, V\}$ is irreducibile; therefore $K$ is either $\{0\}$ or $V$. The linear application $A$ is not constantly zero, which means that $K$ cannot be equal to the whole vector space $V$, and hence $A$ is injective.

\paragraph{Surjective.} The rank $R := \mathrm{ran} A = A(V)$ is clearly $\G$-invariant since
\begin{equation*}x \in R \implies x = Ay \implies \rho^\prime(x) = A \rho(y) \implies \rho^\prime(x) \in R \end{equation*}
The representation $\{ \rho^\prime, \, W\}$ is irreducibile; therefore $R$ is either $\{0\}$ or $W$. The linear application $A$ is not constantly zero, which means that $R$ cannot be equal to $\{0\}$, and hence $A$ is surjective.\end{proof}

\begin{lemma} \label{schur2} Let $\{ \rho, \, V\}$ be an irreducible representation of a group $\G$ over a complex vector space $V$. If there exists a linear mapping $A : V \longrightarrow V$ such that
\begin{equation} \label{eq.5.2} A \rho(g) = \rho( Ag ) \quad \text{for every $g \in \G$}, \end{equation}
then there exists $\lambda \in \C$ such that $A = \lambda \cdot \mathrm{Id}_V= \lambda \cdot \mathrm{id}_{n \times n}$.  \end{lemma}

\begin{proof} Let $\lambda \in \C$ be an eigenvalue of $A$. Then $B := A - \lambda \cdot \mathrm{id}_{n \times n}$ commutes with $\rho(g)$ for all $g \in \G$, and thus we infer from \hyperref[schur1]{Schur Lemma} that $A - \lambda \cdot \mathrm{id}_{n \times n} = 0$. \end{proof}

\subsection{Representations of Finite and Compact Groups}

As a consequence of Haar measures theory (see \hyperref[sec:haar]{Chapter \ref{sec:haar}}), we can easily show that every finite-dimensional representation of a compact group $\G$ is equivalent to some unitary representation.

\begin{theorem} \label{thm.5.2} Every representation of a finite group $\G$ is equivalent to some unitary representation. \end{theorem}

\caution{The proof we present here is taken, almost verbatim, from \cite{hassani}. }

\begin{proof} Let $\{ \rho, \, V \}$ be a representation of $\G$. Consider the hermitian operator
\begin{equation*}\mathbf{R} := \sum_{g \in \G} \rho(g)^\dagger \rho(g), \end{equation*}
and notice that for every $h \in \G$ we have
\begin{equation*}\begin{aligned} \rho(h)^\dagger \mathbf{R} \rho(h) &= \sum_{g \in \G} \rho(h)^\dagger \rho(g)^\dagger \rho(g) \rho(h) =
\\[1em] & = \sum_{g \in \G} (\rho(g) \rho(h))^\dagger \rho(g) \rho(h) =
\\[1em] & = \sum_{g \in \G} \rho(gh)^\dagger \rho(gh) =
\\[1em] & = \sum_{x \in \G} \rho(x)^\dagger \rho(x) = \mathbf{R}, \end{aligned} \end{equation*} 
where the equality $\stackrel{(*)}{=}$ follows from the fact that $\G$ is a finite group, and hence
\begin{equation*} \{ gh \: : \: g \in \G \} = \G \quad \text{for every fixed $h\in \G$}. \end{equation*}
Let $\mathbf{S} := \sqrt{ \mathbf{R} }$. Multiply the identity above by $\mathbf{S}^{-1}$ on the left and by $\rho(g)^{-1} \mathbf{S}^{-1}$ on the right to obtain
\begin{equation*}\begin{aligned} \rho(g)^\dagger \mathbf{R} \rho(g) = \mathbf{R} & \implies \mathbf{S}^{-1} \rho(g)^\dagger \mathbf{S} = \mathbf{S} \rho(g)^{-1} \mathbf{S}^{-1}
\\[1em] & \implies  \left( \mathbf{S} \rho(g) \mathbf{S}^{-1} \right)^\dagger = \left( \mathbf{S} \rho(g) \mathbf{S}^{-1} \right)^{-1} \quad \text{for every $g \in \G$}, \end{aligned} \end{equation*}
which means that the representation $\rho^\prime(g) := \mathbf{S} \rho(g) \mathbf{S}^{-1}$ is unitary and equivalent to $\{ \rho, \, V \}$. \end{proof}

\begin{theorem} \label{thm.5.3} Every finite-dimensional representation of a compact group $\G$ is equivalent to some unitary representation.  \end{theorem}

\begin{proof}Let $\mu$ be the Haar probability measure given by \hyperref[theorem:1das]{Theorem \ref{theorem:1das}}, and let $\{ \rho, \, V \}$ be a representation of $\G$. Let $b : V \times V \longrightarrow \C$ be any inner product and define
\begin{equation*} \langle u, \, v \rangle := \int_\G b( \rho(g)u, \, \rho(g)v) \, \mathrm{d}\mu(g) . \end{equation*}
The reader may check by herself that $\langle \cdot, \, \cdot \rangle$ is a $\G$-invariant inner product, that is, a inner product such that $\langle \rho(g)u, \, \rho(g)v \rangle = \langle u, \, v \rangle$ for all $u, \, v \in V$ and $g \in \G$.

Let $\mathcal{V} := \{v_1, \, \dots, \, v_n\}$ be a basis of the vector space $V$, and denote by $S$ the matrix associated to $\langle \cdot, \, \cdot \rangle$ with respect to $\mathcal{V}$. It turns out that
\begin{equation*} \langle v_i, \, v_j \rangle = \langle \rho(g) v_i, \, \rho(g) v_j \rangle \implies  S^\dagger S = (S \rho(g))^\dagger S \rho(g) \quad \text{for every $g \in \G$},\end{equation*}
and therefore the representation
\begin{equation*} \G \ni g \longmapsto S \rho(g) S^{-1} \in \mathrm{GL}(V, \, \C)\end{equation*}
is unitary because
\begin{equation*}\begin{aligned} \left(S \rho(g) S^{-1}\right)^\dagger S \rho(g) S^{-1} & = (S^{-1})^\dagger \rho(g)^\dagger S^\dagger S \rho(g) S^{-1} =
\\[1em] & = (S^{-1})^\ast \underbracket{\left( S \rho(g) \right)^\dagger \left( S \rho(g)\right)}_{= S^\dagger S} S^{-1} =
\\[1em] & = (S^{-1})^\dagger S^\dagger S S^{-1} = \mathrm{id}_{n\times n}. \end{aligned}\end{equation*} \end{proof}

\begin{proposition}Every irreducible representation of an abelian group $\G$ is one-dimensional. \end{proposition}

\begin{proof}Let $\mathfrak{R} := \{ \rho, \, V\}$ be any representation of $\G$. The group is abelian, hence \eqref{eq.5.2} holds with $A := \rho(g)$ for every $g \in \G$. It follows from \hyperref[schur2]{Schur Lemma \ref{schur2}} that
\begin{equation*} \forall g \in \G, \, \exists \lambda(g) \in \C \: : \: \rho(g) = \lambda(g) \cdot \mathrm{id}_{n \times n}, \end{equation*}
which means that $\mathfrak{R}$ is a $1$-dimensional representation given by
\begin{equation*} \G \ni g \longmapsto \lambda(g) \in \C. \end{equation*} \end{proof}

\begin{example}[$\mathrm{SO}(2, \, \R)$] Fix $m \in \Z$. Every element of $\mathrm{SO}(2, \, \R)$ is a rotation that can be represented as a complex exponential
\begin{equation*} \psi_m(\varphi) := \mathrm{e}^{\imath m \varphi} \end{equation*}
for some $\varphi \in [0, \, 2 \pi)$. For every $m \in \Z$ we have a $1$-dimensional unitary representation given by the mapping
\begin{equation*}\mathrm{SO}(2, \, \R) \ni \psi_m(\varphi) \longmapsto \varphi \in \R.  \end{equation*}\end{example}
\part{Fundamental Groups in Physics}
\chapter{Special Unitary Group $\mathrm{SU}(2, \, \C)$} \thispagestyle{empty}
\label{su2ch}

In this chapter, we study more in-depth the special unitary (Lie) group $\mathrm{SU}(2, \, \C)$ and the associated Lie algebra $\mathrm{su}(2, \, \C)$. Recall that the set of all unitary matrices
\begin{equation*} \mathrm{U}(N, \, \C) := \left\{ U \in \mathrm{GL}(N, \, \C) \: : \: U^\dag U = U U^\dag = \mathrm{Id}_{N \times N} \right\} \end{equation*}
is a group with respect to the matrix product. Similarly,
\begin{equation*} \mathrm{SU}(N, \, \C) := \left\{ U \in \mathrm{GL}(N, \, \C) \: : \: U^\dag U = U U^\dag = \mathrm{Id}_{N \times N}, \: \mathrm{det}(U) = 1 \right\} \end{equation*}
is also a group, and it is called \textit{special unitary group}. Moreover, we proved that
\begin{equation*} \mathrm{SU}(2, \, \C) \cong S^3, \end{equation*}
where $S^3$ is the $3$-dimensional sphere in $\R^4$ (or $\C^2$). The three generators of the special unitary group in the fundamental representation (=smallest nontrivial) are
\begin{equation*} J^a = \frac{1}{2} \tau^a,\end{equation*}
where $\tau^a$ denotes the $a$th Pauli matrix, that is,
\begin{equation*} \tau^1 = \begin{pmatrix} 0 & 1 \\ 1 & 0 \end{pmatrix}, \qquad \tau^2 = \begin{pmatrix} 0 & - \imath \\ \imath & 0 \end{pmatrix}, \qquad \tau^3 = \begin{pmatrix} 1 & 0 \\ 0 & - 1 \end{pmatrix}.\end{equation*}
Moreover, we also proved that $f^{abc}$ is equal to the Levi-Civita tensor $\epsilon^{abc}$, that is,
\begin{equation*} f^{abc} = \epsilon^{abc} := \begin{cases} 1 & \text{if $(a b c)$ is an even permutation,} \\ - 1 & \text{if $(a b c)$ is an odd permutation,} \\ 0 & \text{otherwise}. \end{cases}\end{equation*}

\begin{remark} Let $A, \, B, \, C$ be arbitrary operators (e.g., $n \times n$ matrices). The reader can easily show that the following identity for the commutator of a product holds:
\begin{equation} \label{eq.6.1} [AB, \, C] = A[B, \, C] + [A, \, C]B.\end{equation}   \end{remark}

\paragraph{Casimir.} Recall that the Casimir operator\index{quadratic Casimir operator} of this representation is
\begin{equation*} \mathbb{J}^2 := (J^1)^2 + (J^2)^2 + (J^3)^2. \end{equation*}
We can apply \eqref{eq.6.1} to show that $\mathbb{J}^2$ commutes with all the generators $J^i$. Indeed, a straightforward computation proves that
\begin{equation*}\begin{aligned} [ \mathbb{J}^2, \, J^1 ] & = \underbrace{[ (J^1)^2, \, J^1 ]}_{=0} + [ (J^2)^2, \, J^1 ] + [ (J^3)^2, \, J^1 ] =
\\[1em] & = J^2[J^2, \, J^1] + [J^2, \, J^1] J^2 + J^3[J^3, \, J^1] + [J^3, \, J^1] J^3 =
\\[1em] & = - \imath J^2 J^3 - \imath J^3 J^2 + \imath J^3 J^2 + \imath J^2 J^3 = 0.  \end{aligned} \end{equation*}
In a similar way, one can prove that $[ \mathbb{J}^2, \, J^2 ] = [ \mathbb{J}^2, \, J^3 ] = 0$. 

\begin{remark} The unitary group preserves the complex scalar product
\begin{equation*}\langle z, \, w \rangle_{\C} = z^\dag \cdot w := z_1^\ast w_1^\ast + \dots + z_N^\ast w_N^\ast \quad \text{for all $z, \, w \in V$}. \end{equation*}
In fact, it is enough to notice that $U^\dag U = \mathrm{Id}_{N \times N}$, and plug it into the scalar product:
\begin{equation*} \langle z, \, w\rangle_{\C} = z^\dag \cdot w = z^\dag (U^\dag U) w = \underbrace{(z^\dag U^\dag)}_{= (Uz)^\dag} (U  w) = \langle Uz, \, Uw \rangle_{\C} \quad \text{for all $z, \, w \in V$}. \end{equation*} \end{remark}

\section{Finite Irreducible Representations}

The primary goal is to find all the irreducible representations of $\mathrm{SU}(2, \, \C)$ and $\mathrm{su}(2, \, \C)$, starting here with the finite-dimensional\footnote{A representation\index{representation!finite-dimensional} is finite dimensional if and only if the carrying vector space $V$ has finite dimension.} ones.

\subsection{Introduction}

From \cite{wiki}: "\textit{In quantum mechanics, when a Hamiltonian has a symmetry, that symmetry manifest itself via a set of states at the same energy, i.e. degenerate states.}

\textit{In particle physics, the near mass-degeneracy of the neutron and proton points to an approximate symmetry of the Hamiltonian describing the strong interactions. The neutron does have a slightly higher mass due to isospin breaking; this is due to the difference in the masses of the up and down quarks and the effects of the electromagnetic interaction.}

\textit{It was Heisenberg, the scientist who noticed that the mathematical formulation of this symmetry was in certain respects similar to the mathematical formulation of spin, whence the name "isospin" derives. To be precise, the isospin symmetry is given by the invariance of the Hamiltonian of the strong interactions under the action of the Lie group $\mathrm{SU}(2, \, \C)$. The neutron and the proton are assigned to the doublet (the spin $- 1/2$, $2$, or fundamental representation) of $\mathrm{SU}(2, \, \C)$, which is described above in terms of Pauli's matrices.}" Namely, the nucleon transforms as follows:
\begin{equation*}\begin{pmatrix}p \\ n \end{pmatrix} \longmapsto U(\alpha) \begin{pmatrix}p \\ n \end{pmatrix} \quad \text{where $U(\alpha) = \mathrm{e}^{\imath \frac{\tau^a}{2} \alpha_a} \in \mathrm{SU}(2, \, \C)$}.  \end{equation*}

The pions, on the other hand, are assigned to the adjoint representation of $\mathrm{SU}(2, \, \C)$. Recall that the adjoint representation is defined by setting
\begin{equation*} \left( \mathbb{T}^a \right)_{b, \, c} := \imath f^{bac}, \end{equation*}
where $a, \, b, \, c = 1, \, \dots, \, 3$. It follows that we can also represent the elements of the special unitary group as $3 \times 3$ matrices
\begin{equation*}\mathbb{U}(\alpha) = \mathrm{e}^{\imath \mathbb{T}^a \alpha_a} \in \mathrm{SU}(2, \, \C) \end{equation*}
in such a way that the pions triplet transforms as follows:
\begin{equation} \label{eq.6.2} \begin{pmatrix}\pi^1 \\ \pi^2 \\ \pi^3 \end{pmatrix} \longmapsto \mathbb{U}(\alpha) \begin{pmatrix}\pi^1 \\ \pi^2 \\ \pi^3 \end{pmatrix}.  \end{equation}

\begin{remark}The adjoint representation of $\mathrm{SU}(2, \, \C)$ has dimension $3$. A straightforward computation shows that the generators of $\mathrm{SO}(3, \, \R)$
\begin{equation*} T^1 = \begin{pmatrix} 0 & 0 & 0 \\ 0 & 0 & - \imath \\ 0 & \imath & 0 \end{pmatrix}, \qquad T^2 = \begin{pmatrix} 0 & 0 & \imath \\ 0 & 0 & 0 \\ -\imath & 0 & 0 \end{pmatrix}, \qquad T^3 = \begin{pmatrix} 0 & -\imath & 0 \\ \imath & 0 & 0 \\ 0 & 0 & 0 \end{pmatrix},\end{equation*}
are equal to the generators of the adjoint representation of $\mathrm{SU}(2, \, \C)$. For example, we have
\begin{equation*} \left( \mathbb{T}^1 \right)_{b, \, c} = \imath f^{b1c} = \begin{pmatrix} 0 & 0 & 0 \\ 0 & 0 & - \imath \\ 0 & \imath & 0 \end{pmatrix} = T^1, \end{equation*}
and, similarly, for the other generators.\end{remark}

Recall that the triplet of pions $\pi^i$ for $i = 1, \, 2, \, 3$, is connected to the observable triplet $\pi^0$ and $\pi^{\pm}$, by the following relations:
\begin{equation*}\pi^3 := \pi^0, \qquad \pi^+ := \frac{\pi^1 - \imath \pi^2}{\sqrt{2}}, \qquad \pi^- := \frac{\pi^1 + \imath \pi^2}{\sqrt{2}}.\end{equation*}
There is a different transformation for the triplet of pions through the ($2$-dimensional) fundamental representation, defined in the following way:
\begin{equation} \label{eq.6.3} \pi^a \frac{\tau^a}{2} \longmapsto U(\alpha)\pi^a \frac{\tau^a}{2} U(\alpha)^{-1} =: (\pi^\prime)^a \frac{\tau^a}{2}. \end{equation}

\begin{proposition} The transformation \eqref{eq.6.2} is equivalent to the transformation \eqref{eq.6.3}. \end{proposition}

\begin{proof}[Hint] It is enough to use Taylor's formula (w.r.t. $\alpha$) up to order one of both, i.e.,
\begin{equation*}\mathbb{U}(\alpha) = \mathrm{id}_{3 \times 3} + \imath \mathbb{T}^a \alpha_a + \dots,  \end{equation*}
and
\begin{equation*}U(\alpha) = \mathrm{id}_{2 \times 2} + \imath \frac{\tau^a}{2} \alpha_a + \dots \quad \text{and} \quad U(\alpha)^{-1} = \mathrm{id}_{2 \times 2} - \imath \frac{\tau^a}{2} \alpha_a + \dots. \end{equation*} \end{proof}

As a consequence, we find a simple formula for the \textit{Yukawa interaction}\index{Yukawa interaction}
\begin{equation*} \begin{aligned} V_Y & := g_Y \begin{pmatrix} \tilde{p} & \tilde{n} \end{pmatrix} \pi^a \frac{\tau^a}{2} \begin{pmatrix} p \\ n \end{pmatrix} =
\\[1em] & = \left( \tilde{p} p - \tilde{n} n \right) \pi^0 + \sqrt{2} \left( \tilde{p}n \pi^+ + \tilde{n}p \pi^{-} \right), \end{aligned}\end{equation*}
which describes the nuclear force between nucleons, mediated by pions.

\subsection{Admissible Dimensions of Finite Representations}
\label{sdaosdwkd}

In this section, we assume that $J^1, \, J^2$ and $J^3$ are the generators of a finite-dimensional representation $\mathfrak{R}$ of $\mathrm{SU}(2, \, \C)$, whose dimension is unknown at the moment. Let $\mathbb{J}^2$ be the quadratic Casimir operator, and set
\begin{equation*} J_{\pm} := J^1 \pm \imath J^2 \quad \text{and} \quad J_3 := J^3. \end{equation*}
Note that $\{J_+, \, J_-, \, J_3\}$ is also a generator basis, called \textit{Cartan basis}\index{Cartan basis}, satisfying the following commutators relations:
\begin{equation*}[J_3, \, J_+] = J_+, \qquad [J_3, \, J_-] = - J_- \quad \text{and} \quad [J_+, \, J_-] = 2 J_3.  \end{equation*}
The Casimir quadratic operator $\mathbb{J}^2$ commutes with the generator $J_3$, which means that they are simultaneously diagonalizable and $\mathbb{J}^2 = c \cdot \mathrm{Id}_{N \times N}$ as a consequence of Schur's Lemma.

From now on, we shall denote by $\{ | \, c, \, m \rangle \}$ the common basis of eigenstates for both $J_3$ and $\mathbb{J}^2$, that is, we require that
\begin{equation*}J_3 \, | \, c, \, m \rangle = m \, | \, c, \, m \rangle \qquad \text{and} \qquad \mathbb{J}^2 \, | \, c, \, m \rangle = c \, | \, c, \, m \rangle. \end{equation*}
Let $j := \max \{m \: : \: J_3 \, | \, c, \, m \rangle = m \, | \, c, \, m \rangle \}$ be the maximum eigenstate\footnote{The maximum is well-defined because the representation $\mathfrak{R}$ is finite-dimensional by assumption! } in the representation $\mathfrak{R}$ w.r.t. the operator $J_3$. A straightforward computation shows that
\begin{equation*} \begin{aligned} J_3 \left( J_+ \, | \, c, \, m \rangle \right) & \stackrel{(*)}{=} J_+ \left( J_3 \, | \, c, \, m \rangle \right) + J_+ \, | \, c, \, m \rangle  =
\\[1em] & = m \, J_+ \, | \, c, \, m \rangle + J_+ \, | \, c, \, m \rangle =
\\[1em] & = (m + 1) \,J_+ \, | \, c, \, m \rangle, \end{aligned} \end{equation*}
where $(\ast)$ follows from the commutator identity $[J_3, \, J_+] = J_+$. Similarly, we find that
\begin{equation*} \begin{aligned} J_3 \left( J_- \, | \, c, \, m \rangle \right) & \stackrel{(*)}{=} J_- \left( J_3 \, | \, c, \, m \rangle \right) - J_- \, | \, c, \, m \rangle=
\\[1em] & = m \, J_- \, | \, c, \, m \rangle - J_- \, | \, c, \, m \rangle =
\\[1em] & = (m - 1) \,J_- \, | \, c, \, m \rangle, \end{aligned} \end{equation*}
which means that
\begin{equation*} \begin{aligned} & J_- \, | \, c, \, m \rangle \quad \propto \quad | \, c, \, m -1 \rangle,
\\[1em] & J_+ \, | \, c, \, m \rangle \quad \propto \quad  | \, c, \, m + 1 \rangle. \end{aligned} \end{equation*}
In particular, since $j$ is the maximum eigenstate, we have that $J_+ \, | \, c, \, j \rangle = 0$.

\vspace{1.2mm}
The goal is now to find the relation between $c$ and $j$, and use it to derive an upper bound (resp. lower bound) to the number of "jumps" via $J_+$ (resp. $J_-$).

\begin{remark} The quadratic Casimir operator can be easily rewritten in terms of $J_+ J_-$ as
\begin{equation} \label{eq.6.4} \mathbb{J}^2 = J_+ J_- + J_3^2 - J_3, \end{equation}
\caution{Here $J_3^2$ denotes the square of the operator $J_3$.}and, similarly, in terms of $J_- J_+$ as
\begin{equation} \label{eq.6.5}  \mathbb{J}^2 = J_- J_+ + J_3^2 + J_3.\end{equation}
The proof does not require any idea, but it suffices to evaluate the right-hand side of both identities plugging in the formulas defining $J_+$ and $J_-$.
\end{remark}

Recall that a linear algebra result shows that the eigenstates $\{ | \, c, \, m \rangle \}$ are orthogonal; thus, we can always assume without loss of generality that $\{ | c, \, m \rangle \}$ is a orthonormal basis, which means that
\begin{equation*}\langle c, \, m \, | \, c, \, m \rangle = 1 \qquad \text{and} \qquad \langle c, \, m \, | \, c, \, m^\prime \rangle = 0 \quad \text{for all $m \neq m^\prime$}.\end{equation*}
As a consequence of the normalization, we find that
\begin{equation*}\begin{aligned} c = \langle c, \, j \, | \, \mathbb{J}^2 \, | \, c, \, j \rangle &  \stackrel{(*)}{=} \langle c, \, j \, | \, J_- J_+ + J_3^2 + J_3 \, | \, c, \, j \rangle =
\\[1em] & = \langle c, \, j \, | \, J_- J_+ \, | \, c, \, j \rangle + \langle c, \, j \, | \, J_3^2 \, | \, c, \, j \rangle + \langle c, \, j \, | \, J_3 \, | \, c, \, j \rangle =
\\[1em] & = 0 + j^2 \, \underbrace{\langle c, \, j \, | \, c, \, j \rangle}_{=1} + j \, \underbrace{\langle c, \, j \, | \, c, \, j \rangle}_{=1} = j(j+1),\end{aligned} \end{equation*}
where $(\ast)$ follows from a direct application of formula \eqref{eq.6.5}.

In particular, $c$ depends on $j$, and thus, from now on, we shall denote by $| \, j, \, m\rangle$ the eigenstate $| \,c, \, m\rangle$. The representation $\mathfrak{R}$ is finite-dimensional, which means that also $J_-$ cannot go all the way down; let $n$ be the minimum eigenstate, that is,
\begin{equation*} J_- \, | \, j, \, j - n+ 1 \rangle = | \, j, \, j - n \rangle \quad \text{and} \quad J_- \, | \, j, \, j - n \rangle = 0. \end{equation*}
If we plug \eqref{eq.6.4} into the previous computation, we find that
\begin{equation*}\begin{aligned} c & = \langle j, \, j-n \, | \, \mathbb{J}^2 \, | \, j, \, j - n \rangle 
\\[1em] & = \langle j, \, j  - n\, | \, J_- J_+ + J_3^2 - J_3 \, | \, j, \, j-n \rangle =
\\[1em] & = \langle j, \, j - n \, | \, J_+ J_- \, | \, j, \, j-n \rangle + \langle j, \, j-n \, | \, J_3^2 \, | \, j, \, j-n \rangle - \langle j, \, j-n \, | \, J_3 \, | \, j, \, j-n \rangle =
\\[1em] & = 0 + (j - n) \, \underbrace{\langle j, \, j-n \, | \, j, \, j-n \rangle}_{=1} + (j - n - 1) \,  \underbrace{\langle j, \, j-n \, | \, j, \, j-n \rangle}_{=1} = j(j+1),\end{aligned} \end{equation*}
from which we infer that
\begin{equation*} c = j(j + 1) = (j - n)(j - n - 1) \implies n = 2j. \end{equation*}
In particular, it turns out that the dimension of an irreducible representation $\mathfrak{R}$ is $N := 2j + 1$ (since the eigenvalues range from $- j$ to $+ j$), where
\begin{equation*} j \in \frac{\N}{2} := \left\{ \frac{n}{2}\: : \: n \in \N \right\}. \end{equation*}
Recall that the nucleon corresponds to the fundamental representation of $\mathrm{SU}(2, \, \C)$, which means that $j = 1/2$ and $N = 2$. It turns out that
\begin{equation*}\begin{pmatrix} p \\ n \end{pmatrix} \sim | \, \frac{1}{2}, \, \pm \frac{1}{2}\rangle. \end{equation*}
In a similar fashion, the triplet of pions is associated with the fundamental representation ($j = 1$), and therefore we have
\begin{equation*}\begin{pmatrix} \pi^+ \\ \pi^0 \\ \pi^- \end{pmatrix} \sim \begin{aligned} & | \, 1, \, 1\rangle, \\ & | \, 1, \, 0\rangle, \\ & | \, 1, \, -1\rangle, \end{aligned}\end{equation*}
where the right-hand side vectors correspond to $\pi^1, \, \pi^2$ and $\pi^3$ respectively. The $\pi N$ scattering shows a strong resonance at the kinetic energy about $200$ MeV; it occurs in the $P$-wave ($\ell = 1$) with total angular momentum $J = 3$, which means $j = 3/2$. In this case, we have
\begin{equation*}\begin{pmatrix} \Delta^{++} \\  \Delta^{+} \\ \Delta^{0} \\ \Delta^{-} \end{pmatrix} \sim \begin{aligned} & | \, \frac{3}{2}, \, \frac{3}{2}\rangle, \\ & | \, \frac{3}{2}, \, \frac{1}{2}\rangle, \\ & | \, \frac{3}{2}, \, - \frac{1}{2}\rangle, \\ & | \, \frac{3}{2}, \, - \frac{3}{2}\rangle. \end{aligned} \end{equation*}


\section{Fundamental Representation of $\mathrm{SU}(2, \, \C)$}

Recall that the generators of the fundamental representation $\mathfrak{R} := \{ \rho, \, V \}$ of $\mathrm{SU}(2, \, \C)$ are given by
\begin{equation*} T^a = \frac{1}{2} \tau^a \quad \text{for $a = 1, \, 2, \, 3$},\end{equation*}
where $\tau^a$ denotes the $a$th Pauli matrix\index{Pauli matrices}, that is,
\begin{equation*} \tau^1 = \begin{pmatrix} 0 & 1 \\ 1 & 0 \end{pmatrix}, \qquad \tau^2 = \begin{pmatrix} 0 & - \imath \\ \imath & 0 \end{pmatrix}, \qquad \tau^3 = \begin{pmatrix} 1 & 0 \\ 0 & - 1 \end{pmatrix}.\end{equation*}
Note that Pauli matrices are Hermitian matrices ($(\tau^a)^\dag = \tau^a$) with null-trace. Furthermore, a straightforward computation proves that $f^{abc}$ is equal to the Levi-Civita tensor\index{Levi-Civita tensor}, i.e.,
\begin{equation*} f^{abc} = \epsilon^{abc} := \begin{cases} 1 & \text{if $(a b c)$ is an even permutation,} \\ - 1 & \text{if $(a b c)$ is an odd permutation,} \\ 0 & \text{otherwise}, \end{cases}\end{equation*}
which means that
\begin{equation*}[T^a, \, T^b] = \imath \epsilon^{abc} T^c \implies \begin{aligned} & [T^1, \, T^2] = - [T^2, \, T^1] = \imath T^3, \\[0.8em] & [T^2, \, T^3] = - [T^3, \, T^2] = \imath T^1, \\[0.8em] & [T^3, \, T^1] = - [T^1, \, T^3] = \imath T^2. \end{aligned} \end{equation*}

\begin{definition}[Quaternion Group] \index{quaternion group} The \textit{quaternion group} $Q_8$ is one of the two non-commutative group of order $8$. More precisely, it is given by
\begin{equation*} Q_8 := \left\{  \pm 1, \, \pm i, \,  \pm j, \, \pm k \right\},  \end{equation*}
endowed with a product satisfying the following rules, known as Hamilton's rules\index{Hamilton's rules}:
\begin{equation*}i^2 = j^2 = k^2 = ijk = - 1 \quad \text{and} \quad \begin{cases} ij = - ji = k, \\ jk = - kj = i, \\ ki = - ik = j. \end{cases} \end{equation*} \end{definition}

We now prove that the Pauli matrices form a quaternion group, that is, we show that
\begin{equation*} \left\{ \pm \mathrm{id}_{2 \times 2}, \, \pm \imath \tau^1, \,\pm \imath \tau^2, \, \pm \imath \tau^3 \right\},  \end{equation*}
endowed with the matrix product, is isomorphic to $Q_8$. Recall that the anticommutator\index{anticommutator} is denoted by $\{ \cdot, \, \cdot \}$, and is defined by setting
\begin{equation*} \{A, \, B\} := AB + BA. \end{equation*}
A direct computation proves that
\begin{equation*}(\tau^a)^2 = \mathrm{id}_{2 \times 2} \implies (\imath \tau^a)^2 = - \mathrm{id}_{2 \times 2} \quad \text{for all $a = 1, \, 2, \, 3$}, \end{equation*}
and also that
\begin{equation*}\{ \tau^a, \, \tau^b \} = 0 \quad \text{for all $a \neq b \in \{ 1, \, 2, \, 3 \}$}. \end{equation*}
It follows that, using the generator formula $[T^a, \, T^b] = \imath \epsilon^{abc} T^c$, we have
\begin{equation*}(\imath \tau^a)(\imath \tau^b) = - \tau^a \tau^b = \frac{1}{2}(\tau^b \tau^a - \tau^a \tau^b) = \imath \tau^c, \end{equation*}
which means that $\left\{ \pm \mathrm{id}_{2 \times 2}, \, \pm \imath \tau^1, \,\pm \imath \tau^2, \, \pm \imath \tau^3 \right\}$ is, actually, a quaternion group.

\subsection{Elements in the Fundamental Representation}

In this section, we want to compute the element $U(\alpha)$ in the fundamental representation, where $\alpha$ is a (real) parameter in $\R^3$. Let $\beta_a := \frac{\alpha_a}{2}$ and $\beta := (\beta_1, \, \beta_2, \, \beta_3)$, and notice that
\begin{equation*}U(\alpha) = \mathrm{e}^{\imath T^a \alpha_a} = \mathrm{e}^{\sum_{a =1}^3 \imath \tau^a \beta_a} \neq \prod_{a = 1}^3 \mathrm{e}^{\imath \tau^a \beta_a} \end{equation*}
because, as we proved earlier, the Pauli matrices do not commute between themselves.

The idea is thus to apply the quaternion rules and the definition, as a series, of the exponential function. Namely, let us denote by $\tau \cdot \beta$ the scalar product $\tau^a \beta_a$. By definition
\begin{equation} \label{eq.7.1} \begin{aligned} \mathrm{e}^{\imath \tau \cdot \beta} & = \sum_{n \in \N} \frac{1}{n!} \imath^n (\tau \cdot \beta)^n = \\[1em] & = \sum_{m \in \N} \frac{1}{(2m)!} (-1)^m \left[ (\tau \cdot \beta)^2 \right]^m + \sum_{m \in \N} \frac{1}{(2m + 1)!} (-1)^m \left[ (\tau \cdot \beta)^2 \right]^m (\tau \cdot \beta), \end{aligned} \end{equation}
and using the anticommutative property of the quaternion algebra, we also have that
\begin{equation*}\begin{aligned} (\tau \cdot \beta)^2 & = \beta_a \beta_b \tau^a \tau^b = 
\\[1em] & = \frac{1}{2} \beta_a \beta_b \left( \tau^a \tau^b + \tau^b \tau^a \right) =
\\[1em] & = \beta_a^2 (\tau^a)^2 = |\beta|^2, \end{aligned} \end{equation*}
since $(\tau^a)^2 = \mathrm{id}_{2 \times 2}$ for all $a = 1, \, 2, \, 3$. Plugging this identity into \eqref{eq.7.1} we find that
\begin{equation} \label{eq.7.2} \begin{aligned} \mathrm{e}^{\imath \tau \cdot \beta} & = \sum_{m \in \N} \frac{1}{(2m)!} (-1)^m |\beta|^{2m} + \sum_{m \in \N} \frac{1}{(2m + 1)!} (-1)^m \frac{|\beta|^{2m+1}}{|\beta|} (\tau \cdot \beta) =
\\[1em] & = \cos |\beta| \cdot \mathrm{id}_{2 \times 2} + \frac{\imath \sin |\beta|}{|\beta|} (\tau \cdot \beta). \end{aligned} \end{equation}
We can easily compute the matrix $\tau \cdot \beta$ explicitly as
\begin{equation*} \tau \cdot \beta = \begin{pmatrix} 0 & \beta_1 \\ \beta_1 & 0 \end{pmatrix} + \begin{pmatrix} 0 & - \imath \beta_2 \\ \imath \beta_2 & 0 \end{pmatrix} + \begin{pmatrix} \beta_3 & 0 \\0 & - \beta_3 \end{pmatrix} = \begin{pmatrix} \beta_3 & \beta_1 - \imath \beta_2 \\ \beta_1 + \imath \beta_2 & - \beta_3 \end{pmatrix}, \end{equation*}
and thus we obtain from \eqref{eq.7.2} that the element of parameter $\alpha$ in the fundamental representation is given by
\begin{equation*}\begin{aligned} U(\alpha) & = \mathrm{e}^{\imath \tau \cdot \beta} = \begin{pmatrix} \cos |\beta| & 0 \\ 0 & \cos |\beta| \end{pmatrix} + \frac{\imath \sin |\beta|}{|\beta|} \begin{pmatrix} \beta_3 & \beta_1 - \imath \beta_2 \\ \beta_1 + \imath \beta_2 & - \beta_3 \end{pmatrix} =
\\[1em] & =\mathrm{e}^{\imath \tau \cdot \frac{\alpha}{2}} = \begin{pmatrix} \cos \frac{|\alpha|}{2} & 0 \\ 0 & \cos\frac{|\alpha|}{2} \end{pmatrix} + \frac{ \imath \sin \frac{|\alpha|}{2}}{|\alpha|} \begin{pmatrix} \frac{\alpha_3}{\sqrt{2}} & \frac{\alpha_1 - \imath \alpha_2}{\sqrt{2}} \\ \\ \frac{\alpha_1+ \imath \alpha_2}{\sqrt{2}} & - \frac{\alpha_3}{\sqrt{2}} \end{pmatrix}.
\end{aligned} \end{equation*}

\subsection{Pseudo-Real (Fundamental) Representation}

In this section, we introduce a refinement of the notion of \textit{real representation}, and we prove that the fundamental representation of $\mathrm{SU}(2, \, \C)$ is, actually, pseudo-real (or quaternionic).

\begin{definition}[Real Representation]\index{representation!real} A representation $\mathfrak{R} = \{\rho, \, V\}$ of a group $\G$ is \textit{real} if it is equivalent via a unitary matrix $S$ to the representation $\{ \rho^\ast, \, V^\ast\}$, that is,
\begin{equation*}S \rho(g) S^{-1} = \rho(g)^\ast \quad \text{for every $g \in \G$}.\end{equation*}\end{definition}

\begin{definition}[Pseudo-Real Representation]\index{representation!pseudo-real} A representation $\mathfrak{R} = \{\rho, \, V\}$ of a group $\G$ is said to be \textit{pseudo-real} if it is equivalent via an antisymmetric unitary matrix $S$ to the complex conjugate representation $\{ \rho^\ast, \, V^\ast\}$, that is,
\begin{equation*}S \rho(g) S^{-1} = \rho(g)^\ast \quad \text{for every $g \in \G$}.\end{equation*}\end{definition}

Let $\mathfrak{R} := \{ U(\alpha), \, V \}$ denote the fundamental representation of $\mathrm{SU}(2, \, \C)$. We know that the Pauli matrices $\tau^a$ are Hermitian, and hence
\begin{equation*}(\tau^a)^\dagger = \tau^a \implies (\tau^a)^\ast = (\tau^a)^T \quad \text{for all $a \in \{1, \, 2, \, 3\}$}.\end{equation*}
Therefore
\begin{equation*}U(\alpha)^\ast = S U(\alpha) S^{-1} \iff \mathrm{e}^{- \imath (t^a)^\ast \alpha_a } = S \mathrm{e}^{ \imath t^a \alpha_a } S^{-1} = \mathrm{e}^{\imath S t^a \alpha_a S^{-1} },\end{equation*}
which means that the fundamental representation is real if and only if one can find a regular matrix $S$ such that
\begin{equation} \label{eq.7.3} S t^a S^{-1} = - (t^a)^\ast \quad \text{for all $a \in \{1, \, 2, \, 3\}$}.\end{equation}
The reader can quickly check that $S := \tau^2$ is the sought unitary matrix since
\begin{equation*}\begin{aligned} & \tau^2 \frac{\tau^1}{2} (\tau^2)^{-1} = \frac{1}{2} \tau^2 \tau^1 \tau^2 = - \frac{1}{2} \underbrace{(\tau^2)^2}_{= \mathrm{id}_{2 \times 2}} \tau^1 = - \frac{1}{2} \underbrace{\tau^1}_{= (\tau^1)^\ast }
\\[1em] & \tau^2 \frac{\tau^2}{2} (\tau^2)^{-1} = \frac{1}{2} \tau^2 = - \frac{1}{2} (\tau^2)^\ast,
\\[1em] & \tau^2 \frac{\tau^3}{2} (\tau^2)^{-1} = \frac{1}{2} \tau^2 \tau^3 \tau^2 = - \frac{1}{2} \underbrace{(\tau^2)^2}_{= \mathrm{id}_{2 \times 2}} \tau^3 = - \frac{1}{2} \underbrace{\tau^3}_{= (\tau^3)^\ast}.\end{aligned}\end{equation*}
The matrix $\tau^2$ is antisymmetric since $(\tau^2)^T = (\tau^2)^\ast = - \tau^2$, therefore the fundamental representation is pseudo-real.

Note that it is not necessary to know that $S = \tau^2$ to prove that the representation is pseudo real since we can use \eqref{eq.7.3} and the fact that $\tau^a$ is Hermitian for every $a$. Namely, we have that
\begin{equation*} \eqref{eq.7.3} \implies S t^a S^{-1} = - (t^a)^\ast = - (t^a)^T \implies t^a = - (S^{-1})^T (t^a)^T S^T, \end{equation*}
and therefore
\begin{equation*} t^a = (S^{-1})^T S t^a S^{-1} S^T \implies S^{-1} S^{T} t^a = t^a S^{-1} S^T.\end{equation*}
In particular, the matrix $S^{-1} S^T$ commutes with each element of the Lie group $\mathrm{SU}(2, \, \C)$, and thus by \hyperref[schur1]{Schur Lemma} it follows that
\begin{equation*}\exists \, \lambda \in \R \: : \: S^{-1} S^T = \lambda. \end{equation*}
In particular, we have that $S^T = \lambda S$ and, by taking the square of the identity, we also find that $\lambda^2 = 1$, i.e. $S = S^T$ is symmetric or $S = - S^T$ is antisymmetric, which is exactly what we wanted to prove (i.e., the representation is either real or pseudo real, depending on the matrix $S$.)

\subsection{Simpleness of $\mathrm{SU}(2, \, \C)$}

In the fundamental representation, the Lie algebra $\mathrm{su}(2, \, \C)$ has three subalgebras, which are nothing but the ones generated by $\tau^a$ for each $a \in \{1, \, 2, \, 3\}$. If $\h := \{ \tau^3 \}$, then one can check that
\begin{equation*}[\tau^3, \, \tau^3] = 0 \quad \text{and} \quad [\tau^3, \, \tau^1] = \imath \tau^2 \notin \h, \end{equation*}
and similarly with any subalgebra $\h \subset \mathrm{su}(2, \, \C)$.

In particular, the Lie algebra $\mathrm{su}(2, \, \C)$ has no nontrivial invariant subalgebras, which means that the only normal subgroups $\mathcal{H}$ of $\mathrm{SU}(2, \, \C)$ that are abelian, are the trivial ones.

\begin{theorem} The Lie group $\mathrm{SU}(2, \, \C)$ is simple. \end{theorem}

\begin{proof}The reader can consult \cite{thskda} for a formal proof of this fact. \end{proof}

\section{Elements in Irreducible Representations}
\label{sec:oqweofdks}

Let $J_1, \, J_2$ and $J_3$ be the generators of a $N = 2j + 1$ dimensional irreducible representation of $\mathrm{SU}(2, \, \C)$. Recall that by \eqref{eq.6.4} we have
\begin{equation*}\begin{aligned} c = \langle j, \, m \, | \, \mathbb{J}^2 \, | \, j, \, m \rangle & = \langle j, \, m \, | \, J_+ J_- + J_3^2 - J_3 \, | \, j, \, m \rangle =
\\[1em] & = \langle j, \, m \, | \, J_+ J_- \, | \, j, \, m \rangle + m(m - 1).\end{aligned} \end{equation*}

\begin{remark}Let $\{ v^{(m)} \}_{m = j, \, j-1, \, \dots, \, -j}$ be an orthonormal basis of eigenstates for $J_3$. The eigenvalues $\lambda^{(m)}$ are real, and it is easy to see that
\begin{equation*} J_3 v^{(m)} = \lambda^{(m)} v^{(m)} \implies (v^{(m)})^\dag J_3^\dag = \lambda^{(m)} (v^{(m)})^\dag, \end{equation*}
which yields to the so-called \textit{completeness identity}\index{completeness identity}
\begin{equation} \label{eq.7.4} \sum_{m}  v^{(m)} \cdot (v^{(m)})^\dag= \mathrm{id}_{N \times N}. \end{equation}  \end{remark}

\noindent It follows from \eqref{eq.7.4} that
\begin{equation*}\begin{aligned} c & = \langle j, \, m \, | \, J_+ J_- \, | \, j, \, m \rangle + m(m - 1) =
\\[1em] & = \sum_{m^\prime} \langle j, \, m \, | \, J_+ \, | \, j, \, m^\prime \rangle\langle j, \, m^\prime \, | \, J_- \, | \, j, \, m \rangle + m(m-1) =
\\[1em] & = \langle j, \, m \, | \, J_+ \, | \, j, \, m - 1 \rangle\langle j, \, m - 1 \, | \, J_- \, | \, j, \, m \rangle + m(m-1) =
\\[1em] & = \left| \langle j, \, m-1 \, | \, J_- \, | \, j, \, m \rangle \right|^2 + m(m - 1).\end{aligned} \end{equation*}

In a similar fashion, one can employ formula \eqref{eq.6.5} to prove the equivalent identity for $J_+$, which yields immediately to a complete characterization of the matrices $J_+$ and $J_-$ as follows:
\begin{equation*}\begin{aligned} & \langle j, \, m - 1 \, | \, J_- \, | \, j, \, m \rangle = \sqrt{ (j + m)(j - m + 1) },
\\[1em] & \langle j, \, m+1 \, | \, J_+ \, | \, j, \, m \rangle = \sqrt{ (j - m)(j + m + 1) }.\end{aligned} \end{equation*}
In particular, the matrices $J_+$ and $J_-$ have a peculiar form as the only nonzero elements are the supdiagonal and the subdiagonal respectively, that is,
\begin{equation*} J_- = \begin{pmatrix}0       &0 &\ldots  &0\\
\ast & 0      &\ddots  &\vdots\\
\vdots  &\ddots  &0       &0\\
0 &\ldots  &\ast &0 \end{pmatrix} \quad \text{and} \quad J_+ = \begin{pmatrix}0       &\ast &\ldots  &0\\
0 & 0      &\ddots  &\vdots\\
\vdots  &\ddots  &0       &\ast\\
0 &\ldots  &0 &0 \end{pmatrix}. \end{equation*}
It follows that
\begin{equation*} J_1 = \frac{1}{2}(J_+ + J_-) = \begin{pmatrix}0       &\ast &\ldots  &0\\
\ast & 0      &\ddots  &\vdots\\
\vdots  &\ddots  &0       &*\\
0 &\ldots  &\ast &0 \end{pmatrix} \quad \text{and} \quad J_2 = \frac{1}{2\imath}(J_+ - J_-) = \begin{pmatrix}0       &\ast &\ldots  &0\\
\ast & 0      &\ddots  &\vdots\\
\vdots  &\ddots  &0       &*\\
0 &\ldots  &\ast &0 \end{pmatrix},\end{equation*}
and $J_3$ is the diagonal $(2j+1)\times(2j+1)$ matrix $\mathrm{diag}(j, \, j - 1, \, \dots, \, - j)$.

\paragraph{Adjoint Representation.} One can easily apply the general formulas provided above to compute the generators of the adjoint representation ($j = 1$), and the exponentials $\mathbb{U}(\alpha)$ for $\alpha \in \R^3$.

\section{Tensor Product of Representations}

The content of this section is mostly a summary of \cite[Chapter 24.8]{hassani}. The reader interested in a better understanding of this topic may start by consulting that book.

\vspace{1.2mm}
A quantum mechanical system possessing a group of symmetry is described by vectors that transform according to an irreducible representation $\mathfrak{R}$. For example, a rotationally invariant system can be characterized by an eigenstate of angular momentum, the generator of the rotation.

Often irreducible states are combined to form new states. For example, the state of two noninteracting particles is described by a two-particle state, labeled by the combined eigenvalues of the two sets of operators that define each particle separately.

In the case of the angular momentum, the single-particle states may be labeled as $| \, j_1, \, m_1 \rangle$ and $| \, j_2, \, m_2 \rangle$. Then the combined state is labeled by
\begin{equation*} \text{$| \, j_1, \, m_1 \rangle | \, j_2, \, m_2 \rangle$ or $| \, j_1, \, m_1; \; j_2, \, m_2 \rangle$},\end{equation*}
and one can define an action of the rotation group on the vector space spanned bu these combined states to construct the so-called \textit{tensor product representation}\index{tensor product representation}. We now recall the way in which one can construct such a representation.

\paragraph{Kronecker Product.} Let $\mathfrak{R} := \{ \rho, \, V \}$ and $\mathfrak{S} := \{ \rho^\prime, \, W \}$ be two representations of a group $\G$. We can easily define an action of the group $\G$ on the tensor product $V \otimes W$ via the representation $\rho \otimes \rho^\prime : \G \longrightarrow \mathrm{GL}(V \otimes W)$ given by
\begin{equation}\label{eq.8.1} (\rho \otimes \rho^\prime)(g)( |v\rangle, \, |w\rangle ) := \left( \rho(g) |v\rangle, \, \rho^\prime(g) |w\rangle \right). \end{equation}
The reader can easily check that \eqref{eq.8.1} gives a representation on the tensor product since the associativity is an immediate consequence of the associativity of both $\mathfrak{R}$ and $\mathfrak{S}$.

\begin{notation}In this course, we shall often denote the direct product element $( |v\rangle, \, |w\rangle )$ by $|v, \, w\rangle$, or simply $ | vw \rangle$. Similarly, if $\{ |v_i\rangle\}$ is an orthonormal basis for $V$ and $\{ |w_a\rangle \}$ is an orthonormal basis for $W$, we define an inner product on $V \otimes W$ by setting
\begin{equation}\label{eq.8.2} \langle v, \, w \, | \, v^\prime, \, w^\prime \rangle := \langle v \, | \, v^\prime \rangle \langle w \, | \, w^\prime \rangle. \end{equation} \end{notation}

An important special case is the tensor product of a representation with itself. For such a representation, the matrix elements satisfy the symmetry relation
\begin{equation*} (\rho \otimes \rho)(g)_{i a, \, jb} = (\rho \otimes \rho)(g)_{ai, \, bj}. \end{equation*} 
The symmetry can be used to decompose the tensor product space into two $\G$-invariant subspaces. To do this, take the span of all symmetric vectors $|v_i w_j\rangle + |v_j w_i \rangle$ and denote it by $(V \otimes V)_s$. Similarly, take the span of all antisymmetric vectors $|v_i w_j\rangle - |v_j w_i \rangle$ and denote it by $(V \otimes V)_a$. It is easy to see that every vector in $V \otimes V$ can be written as the sum of a symmetric and an antisymmetric vector, i.e.
\begin{equation*}|v_i w_j \rangle = \frac{1}{2} ( |v_i w_j \rangle + |v_j w_i\rangle) + \frac{1}{2} ( |v_i w_j \rangle - |v_j w_i\rangle). \end{equation*} 
It follows that
\begin{equation*} V \otimes V = (V \otimes V)_s \oplus (V \otimes V)_a \end{equation*}
since the unique common vector is the zero vector. It follows that the Kronecker product of a representation with itself is always reducible into two representations, the symmetric and the antisymmetric ones.

\subsection{Clebsh-Gordan Decomposition}

Let $j_1$ and $j_2$ be irreducible representations of $\mathrm{SU}(2, \, \C)$. The tensor product $j_1 \otimes j_2$ is, clearly, not irreducible anymore by definition. The idea is to decompose it as a sum of irreducible representations
\begin{equation*}j_1 \otimes j_2 = (j_1 + j_2) \otimes (j_1 + j_2 - 1) \otimes \dots \otimes |j_1 - j_2| \end{equation*}
in such a way that
\begin{equation*}|j_1, \, m_1\rangle |j_2, \, m_2 \rangle = \sum_{M = j_1 + j_2}^{|j_1 - j_2|} |J, \, M\rangle \langle J, \, M| j_1, \, m_1; \; j_2, \, m_2 \rangle, \end{equation*}
where $J = j_1 + j_2$. The coefficients of the sum $\langle J, \, M| j_1, \, m_1; \; j_2, \, m_2 \rangle$ are known as \textit{Clebsh-Gordan coefficients}\index{Clebsh-Gordan decomposition}.

\section{Comparison: $\mathrm{SO}(3, \, \R)$ and $\mathrm{SU}(2, \, \C)$}

In this final section, we discuss a little bit more about the relationship between the Lie groups $\mathrm{SO}(3, \, \R)$ and $\mathrm{SU}(2, \, \C)$. First, recall that
\begin{equation*} \mathrm{SO}(2, \, \R) \cong \mathrm{U}(1, \, \C) \end{equation*}
via the isomorphism
\begin{equation*} \mathrm{SO}(2, \, \R) \ni \begin{pmatrix} \cos \theta & \sin \theta \\ - \sin \theta & \cos \theta \end{pmatrix} \longmapsto \mathrm{e}^{\imath \theta} \in \mathrm{U}(1, \, \C). \end{equation*}
The irreducible unitary representations of $\mathrm{U}(1, \, \C)$ are all one-dimensional as a consequence of Schur's Lemma; namely, we have that
\begin{equation*} \psi_m : U(1, \, \C) \longrightarrow \R, \qquad \mathrm{e}^{\imath m \theta} \longmapsto \theta \end{equation*}
is a one-dimensional unitary irreducible representation for every $m \in \Z$. Consider the fundamental representation ($m = 1$), given by
\begin{equation*} \mathrm{SO}(2, \, \R) \ni \begin{pmatrix} \cos \theta & \sin \theta \\ - \sin \theta & \cos \theta \end{pmatrix} \longmapsto \theta \in \R, \end{equation*}
and consider the change of basis via a regular matrix
\begin{equation*} S = \frac{1}{\sqrt{2}} \begin{pmatrix}1 & \imath \\ \imath & 1 \end{pmatrix} \quad \text{and} \quad S^{-1} = \frac{1}{\sqrt{2}} \begin{pmatrix} 1 & -\imath \\ -\imath & 1\end{pmatrix}. \end{equation*}
It turns out that
\begin{equation*} S \begin{pmatrix} x \\ y \end{pmatrix} = \frac{1}{\sqrt{2}} \begin{pmatrix} x + \imath y \\ \imath x + y\end{pmatrix} =: \frac{1}{\sqrt{2}} \begin{pmatrix} z \\ \imath \bar{z} \end{pmatrix}, \end{equation*}
and
\begin{equation*} S R_\theta S^{-1} = \begin{pmatrix}\mathrm{e}^{- \imath \theta} & 0 \\ 0 & \mathrm{e}^{\imath \theta}\end{pmatrix}, \end{equation*}
which means that the matrix $S$ transforms the fundamental representation $m = 1$ into the one-dimensional representation given by $m = - 1$ (since $\mathrm{e}^{-\imath \theta} z = - \imath \mathrm{e}^{\imath \theta}\bar{z}$).

\paragraph{Subgroups.} We now show that $\mathrm{SO}(2, \, \R)$ is a subgroup of $\mathrm{SU}(2, \, \C)$, and we explain the intuitive reason behind the $2$-degree covering
\begin{equation*}\mathrm{SU}(2, \, \C) \longrightarrow  \mathrm{SO}(3, \, \R). \end{equation*}
Recall that the third Pauli matrix $\tau^3$ is diagonal, and therefore the computation of the exponential is extremely easy. In particular, we obtain that the subgroup generated by $\tau^3$ only is given by
\begin{equation*}U((0, \, 0, \, \theta)) = \mathrm{e}^{\imath \frac{\tau^3}{2} \theta} = \begin{pmatrix}\mathrm{e}^{- \imath \frac{\theta}{2}} & 0 \\ 0 & \mathrm{e}^{\imath \frac{\theta}{2}}\end{pmatrix}, \end{equation*}
which means that $\tau^3$ generates the $\mathrm{U}(1, \, \C)$ as a subgroup of $\mathrm{SU}(2, \, \C)$.

Note that there is a factor $2$ on the rotation angle, and this is the main reason behind the existence of a $2$-degree covering. We shall not compute it directly, but one can prove that
\begin{equation*}U((0, \, \theta, \, 0)) = \mathrm{e}^{\imath \frac{\tau^2}{2} \theta} = R_{ \frac{\theta}{2} }, \end{equation*}
and similarly for the subgroup generated by the first Pauli matrix $\tau^1$. Intuitively, for any fixed axis of rotation $\vec{v}$, there is a subgroup isomorphic to $\mathrm{SO}(2, \, \R)$ that consist in all the rotations of the plane perpendicular to $\vec{v}$.

\paragraph{Rotation Group.} Recall that the generators of the Lie group $\mathrm{SO}(3, \, \R)$ are given by
\begin{equation*} T^1 = \begin{pmatrix} 0 & 0 & 0 \\ 0 & 0 & - \imath \\ 0 & \imath & 0 \end{pmatrix}, \qquad T^2 = \begin{pmatrix} 0 & 0 & \imath \\ 0 & 0 & 0 \\ -\imath & 0 & 0 \end{pmatrix}, \qquad T^3 = \begin{pmatrix} 0 & -\imath & 0 \\ \imath & 0 & 0 \\ 0 & 0 & 0 \end{pmatrix},\end{equation*}
that are nothing else but the rotations on the coordinate planes $xy$, $yz$ and $xz$. Let $(\alpha, \, \beta, \, \gamma)$ be the Euler angle of rotation, so that a generic element of $\mathrm{SO}(3, \, \R)$ in this representation is given by
\begin{equation*}U( (\alpha, \, \beta, \, \gamma) ) = U_3(\gamma) U_2(\beta) U_1(\alpha), \end{equation*}
where $U_i(\theta)$ is a rotation around the $i$th coordinate axis of angle $\theta$. For example, we have
\begin{equation*}U_1(\alpha) = \begin{pmatrix} \cos \alpha & \sin \alpha & 0 \\ - \sin \alpha & \cos \alpha & 0 \\ 0 & 0 & 1 \end{pmatrix} \end{equation*}
as an element of $\mathrm{SO}(3, \, \R)$, and
\begin{equation*}\tilde{U}_1(\alpha) = \begin{pmatrix}\mathrm{e}^{- \imath \frac{\alpha}{2}} & 0 \\ 0 & \mathrm{e}^{\imath \frac{\alpha}{2}}\end{pmatrix} \end{equation*}
as an element of $\mathrm{SU}(2, \, \C)$. It is easy to check that
\begin{equation*}U_1(2 \pi) = \mathrm{id}_{3 \times 3} \quad \text{and} \quad \tilde{U}_1(4\pi) = \mathrm{id}_{2 \times 2}, \end{equation*}
which means that in $\mathrm{SU}(2, \, \C)$ we need to complete "two laps" to go back to the identity matrix, and this is to be expected as a consequence of the existence of the degree-two covering.
\chapter{Special Unitary Group $\mathrm{SU}(3, \, \C)$} \thispagestyle{empty}
\label{chwk}

In this chapter, we study more in-depth the special unitary (Lie) group $\mathrm{SU}(3, \, \C)$ and the associated Lie algebra $\mathrm{su}(3, \, \C)$. Recall that the set of all unitary matrices
\begin{equation*} \mathrm{U}(N, \, \C) := \left\{ U \in \mathrm{GL}(N, \, \C) \: : \: U^\dag U = U U^\dag = \mathrm{Id}_{N \times N} \right\} \end{equation*}
is a group with respect to the matrix product. Similarly,
\begin{equation*} \mathrm{SU}(N, \, \C) := \left\{ U \in \mathrm{GL}(N, \, \C) \: : \: U^\dag U = U U^\dag = \mathrm{Id}_{N \times N}, \: \mathrm{det}(U) = 1 \right\} \end{equation*}
is also a group, and it is called \textit{special unitary group}.

The eight generators of the $3$-special unitary group can be computed explicitly in the fundamental representation (=smallest nontrivial), and they are given by the \textit{Gell-Mann matrices}\index{Gell-Mann matrices}:
\begin{equation*}\begin{aligned} & \lambda^1 = \frac{1}{2} \begin{pmatrix} 0 & 1 & 0 \\ 1 & 0 & 0 \\ 0 & 0 & 0 \end{pmatrix}, \qquad \lambda^2 = \frac{1}{2} \begin{pmatrix} 0 & - \imath & 0 \\ \imath & 0 &0 \\ 0 & 0 & 0 \end{pmatrix}, \qquad \lambda^3 = \frac{1}{2} \begin{pmatrix} 1 & 0 &0 \\ 0 & - 1 & 0 \\ 0 & 0 & 0\end{pmatrix},
\\[1em] & \lambda^4 = \frac{1}{2} \begin{pmatrix} 0 & 0 & 1 \\ 0 & 0 & 0 \\ 1 & 0 & 0 \end{pmatrix}, \qquad \lambda^5 = \frac{1}{2} \begin{pmatrix} 0 & 0 & - \imath \\ 0 & 0 & 0 \\ \imath & 0 & 0 \end{pmatrix}, \qquad \lambda^6 = \frac{1}{2} \begin{pmatrix} 0 & 0 & 0 \\ 0 & 0 & 1 \\ 0 & 1 & 0 \end{pmatrix},
\\[1em] & \lambda^7 = \frac{1}{2} \begin{pmatrix} 0 & 0 & 0 \\ 0 & 0 & -\imath \\ 0 & \imath & 0 \end{pmatrix}, \qquad \lambda^8 = \frac{1}{2\sqrt{3}} \begin{pmatrix} 1 & 0 & 0 \\ 0 & 1 & 0\\ 0 & 0 & -2  \end{pmatrix} .\end{aligned} \end{equation*}
We immediately see that $\mathrm{SU}(2, \, \C)$ is generated by $\lambda^1$, $\lambda^2$ and $\lambda^3$, and it is thus a subgroup of $\mathrm{SU}(3, \, \C)$. Moreover, one can easily prove that
\begin{equation*}\mathrm{tr}(\lambda^a \lambda^b) = \frac{1}{2} \delta_{ab} \quad \text{and} \quad f^{abc} = \begin{cases} 1 & \text{if $(a b c) = (1 2 3)$,} \\[0.5em] 1/2 & \text{if $(a b c) \in \{(345), \, (147), \, (246), \, (257) \}$ ,} \\[0.5em] - 1/2 & \text{if $(abc) \in \{(156), \, (367)\}$,} \\[0.5em] \sqrt{3}/2 & \text{if $(abc) \in \{ (458), \, (678) \}$}, \end{cases}\end{equation*}
which determines all the possible values of $f^{abc}$ since it is a completely antisymmetric tensor.

\paragraph{Killing Form.} We can easily check that the killing form \eqref{kf} is given by
\begin{equation*} g^{ab} := f^{acd} f^{bdc} = - \frac{3}{2} \delta_{ab} \quad \text{for all $a, \, b = 1, \, \dots, \, 8$}. \end{equation*}
Recall that a \textit{Casimir Operator} is a precise element which lies within the center of a Lie algebra (e.g., the square of the angular momentum modulus in $\mathrm{so}(3, \, \R)$).

Let $\g$ be a \textbf{semisimple} Lie algebra, and let $g^{ab}$ denote its metric \eqref{kf}. The matrix $g$ is invertible by Cartan's criterion, and therefore we can introduce the following notation:
\begin{equation*} g_{ab} := (g^{-1})_{ab} \end{equation*}
The (quadratic) \textit{Casimir operator} of the semisimple Lie algebra $\mathrm{su}(3, \, \C)$ is defined by setting
\begin{equation} \label{caso1} C := g_{ab}\lambda^a\lambda^b. \end{equation}

\begin{lemma} The Casimir operator $C$ is an element of the center $C(\g)$, that is,
\begin{equation*} [C, \, T^a] = 0 \quad \text{for every $T^a \in \g$}. \end{equation*} \end{lemma}

Recall also that, if we define $c_{abc} := g^{ae} f^{bce}$, then it turns out that $c_{abc} = c_{bca} = c_{cab}$ and $c$ is totally antisymmetric. In particular, in the case of the special unitary group we have that
\begin{equation*} c_{abc} := - \frac{3}{2} f^{bca} \implies \text{$f^{bca}$ is also totally antisymmetric}.\end{equation*}

\section{Finite Irreducible Representations}

The primary goal is to find all the irreducible representations of $\mathrm{SU}(3, \, \C)$ and $\mathrm{su}(3, \, \C)$, starting here with the finite-dimensional\footnote{A representation\index{representation!finite-dimensional} is finite dimensional if and only if the carrying vector space $V$ has finite dimension.} ones.

\subsection{Construction via Weight Diagrams}

Let us consider the fundamental ($3$-dimensional) representation
\begin{equation*} \underline{3} \: : \: \begin{pmatrix} q_1 \\ q_2 \\ q_3 \end{pmatrix} \quad \text{where $| \, q_1 \rangle = \begin{pmatrix} 1 \\ 0 \\ 0 \end{pmatrix}$, $|\, q_2 \rangle = \begin{pmatrix} 0 \\ 1 \\ 0 \end{pmatrix}$ and $|\,q_3 \rangle = \begin{pmatrix} 0 \\ 0 \\ 1 \end{pmatrix}$.} \end{equation*}
The generators $\lambda^3$ and $\lambda^8$ are both diagonal, and thus $\{ |\,q_i \rangle \: : \: i \in \{1, \, 2, \, 3\}\}$ is a basis for the both of them. Precisely, we have that
\begin{equation*} \begin{aligned} & \lambda^3 \,|\, q_1 \rangle = \frac{1}{2} \,|\, q_1 \rangle & \text{and} \qquad \lambda^8 \,|\,q_1 \rangle = \frac{1}{2 \sqrt{3}} \,|\,q_1 \rangle,
\\[1em] & \lambda^3 \,|\, q_2 \rangle = - \frac{1}{2} \,|\,q_2 \rangle & \text{and} \qquad \lambda^8 \,|\,q_2 \rangle = \frac{1}{2 \sqrt{3}} \,|\,q_2 \rangle,
\\[1em] & \lambda^3 \,|\,q_3 \rangle = 0 \,|\,q_3 \rangle & \text{and} \qquad \lambda^8 \,|\,q_3 \rangle = - \frac{1} {\sqrt{3}} \,|\,q_3 \rangle. \end{aligned} \end{equation*}
The vectors $q_i$ are $2$-dimensional vectors\footnote{The rank\index{group rank} of a group $\G$ is defined as the number of diagonal generators in the fundamental representation.} in the space generated by $\lambda^3$ and $\lambda^8$, whose coordinate are given by the relations above:
\begin{equation*} q_1 = \begin{pmatrix} \frac{1}{2} & \frac{1}{2 \sqrt{3}} \end{pmatrix}, \qquad q_2 = \begin{pmatrix} - \frac{1}{2} & \frac{1}{2 \sqrt{3}} \end{pmatrix}, \qquad q_3 = \begin{pmatrix} 0 & - \frac{1}{\sqrt{3}} \end{pmatrix}. \end{equation*}
If we consider the hypercharge operator\index{hypercharge operator} $Y := \frac{2}{\sqrt{3}} \lambda^8$, then the points $q_i$ delimits a regular triangle as in the figure below:
\begin{figure}[!htbp]
        \centering
        \mbox{%
\begin{minipage}{.40\textwidth}
\begin{tikzpicture}
\draw[help lines, color=gray!30, dashed] (-2.3,-2.3) grid (2.3,2.3);
\draw[->,ultra thin] (-2.4,0)--(2.4,0) node[right]{$\lambda^3$};
\draw[->,ultra thin] (0,-2.4)--(0,2.4) node[above]{$Y$};
  [anchor=mid west,
  mark size=+2pt, mark color=red,  ball color=green]
  \foreach \plm[count=\cnt] in {ball}
    \draw[mark options={fill=red}]
      plot[mark=\plm] coordinates {(1, 2/3) (-1, 2/3) (0, -4/3) (1, 2/3)};
	\node[] at (-1, 2/3+0.3) {$q_2$}; \node[] at (1, 2/3 + 0.3) {$q_1$}; \node[] at (0.3, -4/3) {$q_3$}; 
\end{tikzpicture}
          \end{minipage}%
          \qquad
          \begin{minipage}{.40\textwidth}
      \begin{tikzpicture}
\draw[help lines, color=gray!30, dashed] (-2.3,-2.3) grid (2.3,2.3);
\draw[->,ultra thin] (-2.4,0)--(2.4,0) node[right]{$\lambda^3$};
\draw[->,ultra thin] (0,-2.4)--(0,2.4) node[above]{$Y$};
  [anchor=mid west,
  mark size=+2pt, mark color=red,  ball color=green]
  \foreach \plm[count=\cnt] in {ball}
    \draw[mark options={fill=red}]
      plot[mark=\plm] coordinates {(1, -2/3) (-1, -2/3) (0, 4/3) (1, -2/3)};
	\node[] at (-1, -2/3 - 0.3) {$q_2$}; \node[] at (1, -2/3 - 0.3) {$q_1$}; \node[] at (0.3, 4/3) {$q_3$}; 
\end{tikzpicture}
          \end{minipage}
    }
    \caption{\textbf{Left.} Fundamental Representation $\underline{3}$. \textbf{Right.} Complex conjugate representation $\underline{3}^\ast$.}
\end{figure}

We can now employ \textit{weight diagrams}\index{weight diagram} to compute the tensor product representation $\underline{3} \otimes \underline{3}^\ast$ ($|q \bar{q}\rangle$), and prove that it is equivalent to the representation $\underline{8} \oplus \underline{1}$. We consider the vector sum
\begin{equation*}q_{ij} := q_i + q_j^\ast = q_i - q_j \quad \text{for all $i, \, j =1, \, 2, \, 3$} \end{equation*}
and we obtain the following figure:

\begin{figure}[!htbp]
        \centering
        \mbox{%
\begin{minipage}{.45\textwidth}
\begin{tikzpicture}
\draw[->,ultra thin] (-2.5,0)--(2.5,0) node[right]{$\lambda^3$};
\draw[->,ultra thin] (0,-4.1)--(0,4.1) node[above]{$Y$};
  [anchor=mid west,
  mark size=+2pt, mark color=red,  ball color=green]
  \foreach \plm[count=\cnt] in {ball}
    \draw[mark options={fill=red}]
      plot[mark=\plm] coordinates {(-2, 0) (-1, 8/3) (1, 8/3) (2, 0) (1, -8/3) (-1, -8/3) (-2, 0)};
       [anchor=mid west,
  mark size=+2pt, mark color=red,  ball color=green]
  \foreach \plm[count=\cnt] in {ball}
    \draw[mark options={fill=red}]
      plot[mark=\plm] coordinates {(0, 0)};
	\node[] at (-2.3, 0.3) {$q_{22}$}; \node[] at (2.3, 0.3) {$q_{11}$}; \node[] at (-1.05, 8/3 + 0.3) {$q_{23}$};  \node[] at (1.05, 8/3 + 0.3) {$q_{13}$}; \node[] at (-1.05, -8/3-0.3) {$q_{32}$};  \node[] at (1.05, -8/3-0.3) {$q_{31}$}; \node[] at (-0.3, 0.2) {$q_{21}$}; \node[] at (0.3, 0.2) {$q_{12}$};
\end{tikzpicture}
          \end{minipage}%
          \qquad
          \begin{minipage}{.45\textwidth}
      \begin{tikzpicture}
\draw[->,ultra thin] (-2.1,0)--(2.1,0) node[right]{$\lambda^3$};
\draw[->,ultra thin] (0,-2.1)--(0,2.1) node[above]{$Y$};
  [anchor=mid west,
  mark size=+2pt, mark color=red,  ball color=green]
  \foreach \plm[count=\cnt] in {ball}
    \draw[mark options={fill=red}]
      plot[mark=\plm] coordinates {(0, 0)};
	\node[] at (0.3, 0.3) {$q_{33}$};
\end{tikzpicture}
          \end{minipage}
    }
    \caption{The decomposition of the tensor product representation $\underline{3} \otimes \underline{3}^\ast$. }
\end{figure}

In particular, the representation $\underline{3} \otimes \underline{3}^\ast$ is decomposed in a octet $\underline{8}$ with a degenerate state at the origin, and a singlet which represents the direct scalar product:
\begin{equation*}| \underline{1} \rangle = | q_1^\ast \rangle |q_1\rangle + | q_2^\ast \rangle |q_2\rangle + | q_3^\ast \rangle |q_3\rangle. \end{equation*}
In a similar fashion, one can prove that the tensor product representation $\underline{3} \otimes \underline{3}$ ($|q q\rangle$) is equivalent to the representation $\underline{6} \oplus \underline{3}^\ast$.

\begin{figure}[!htbp]
        \centering
        \mbox{%
\begin{minipage}{.45\textwidth}
\begin{tikzpicture}
\draw[->,ultra thin] (-2.1,0)--(2.1,0) node[right]{$\lambda^3$};
\draw[->,ultra thin] (0,-3.5)--(0,2.4) node[above]{$Y$};
  [anchor=mid west,
  mark size=+2pt, mark color=red,  ball color=green]
  \foreach \plm[count=\cnt] in {ball}
    \draw[mark options={fill=red}]
      plot[mark=\plm] coordinates {(-2, 4/3) (0, 4/3) (2, 4/3) (1, -2/3) (0, -8/3) (-1, -2/3) (-2, 4/3)};
\end{tikzpicture}
          \end{minipage}%
          \qquad
          \begin{minipage}{.45\textwidth}
      \begin{tikzpicture}
\draw[->,ultra thin] (-2.1,0)--(2.1,0) node[right]{$\lambda^3$};
\draw[->,ultra thin] (0,-2.1)--(0,2.5) node[above]{$Y$};
  [anchor=mid west,
  mark size=+2pt, mark color=red,  ball color=green]
  \foreach \plm[count=\cnt] in {ball}
    \draw[mark options={fill=red}]
      plot[mark=\plm] coordinates {(1, -2/3) (-1, -2/3) (0, 4/3) (1, -2/3)};

\end{tikzpicture}
          \end{minipage}
    }
    \caption{The decomposition of the tensor product representation $\underline{3} \otimes \underline{3}$. }
\end{figure}

Using the decompositions we derived above we can construct more higher-dimensional representations, e.g.,
\begin{equation*}\underline{3} \otimes \underline{3} \otimes \underline{3} = (\underline{6} \oplus \underline{3}^\ast) \otimes \underline{3} = \underline{10} \oplus \underline{8} \oplus \underline{8} \oplus \underline{1}, \end{equation*}
as the reader can easily check by a direct computation.

\subsection{Generators of the Special Unitary Group $\mathrm{SU}(2, \, \C)$}

The generators $\lambda^1$, $\lambda^2$ and $\lambda^3$ are, essentially, the Pauli matrices $\tau^a$. Therefore
\begin{equation*}\mathrm{Span}< \lambda^1, \, \lambda^2, \, \lambda^3 > \cong \mathrm{su}(2, \, \C), \end{equation*}
that is, they generate the algebra $\mathrm{su}(2, \, \C)$ inside $\mathrm{su}(3, \, \C)$. We also introduce the \textit{isospin} operators
\begin{equation*}T_{\pm} := \lambda^1 \pm \imath \lambda^2, \end{equation*}
and notice that $\{ T_{\pm}, \, \lambda^3 \}$ is still a basis of $\mathrm{su}(2, \, \C)$. Similarly, we introduce the $U$-spin operators
\begin{equation*}U_{\pm} := \lambda^6 \pm \imath \lambda^7, \end{equation*}
and notice that
\begin{equation*}[U_+, \, U_-] = \left( \sqrt{3} \lambda^8 - \lambda^3 \right) =: 2 U_3.\end{equation*}
One can easily check that, similarly to the case of $\mathrm{su}(2, \, \C)$, we have the following relations:
\begin{equation*}[U_3, \, U_+] =U_+ \quad \text{and} \quad [U_3, \, U_-] =- U_- .\end{equation*}
In a similar fashion, we introduce the $V$-spin operators
\begin{equation*}V_{\pm} := \lambda^4 \pm \imath \lambda^5, \end{equation*}
and notice that
\begin{equation*}[V_+, \, V_-] = \left( \sqrt{3} \lambda^8 + \lambda^3 \right) =: 2 V_3.\end{equation*}
One can easily check that, similarly to the case of $\mathrm{su}(2, \, \C)$, we have the following relations:
\begin{equation*}[V_3, \, V_+] =V_+ \quad \text{and} \quad [V_3, \, V_-] =- V_- .\end{equation*}
We can use these new spin operators to get a better understanding of what happens when we decompose the representation $\underline{3} \otimes \underline{3}^\ast$. Indeed, a straightforward computation proves that the points of the hexagon are given by $T_{\pm} \mathbf{0}$, $U_{\pm} \mathbf{0}$ and $V_{\pm} \mathbf{0}$.

\begin{figure}[!htbp]
        \centering
        
\begin{tikzpicture}
\draw[->,ultra thin] (-3.1,0)--(3.1,0) node[right]{$\lambda^3$};
\draw[->,ultra thin] (0,-4.1)--(0,4.1) node[above]{$Y$};
\draw[->, red] (0,0)--(-2, 0);
\draw[->, red] (0,0)--(2, 0);
\draw[->, red] (0,0)--(-1, 8/3);
\draw[->, red] (0,0)--(1, 8/3);
\draw[->, red] (0,0)--(1, -8/3);
\draw[->,red] (0,0)--(-1, -8/3);
  [anchor=mid west,
  mark size=+2pt, mark color=red,  ball color=green]
  \foreach \plm[count=\cnt] in {ball}
    \draw[mark options={fill=red}]
      plot[mark=\plm] coordinates {(-2, 0) (-1, 8/3) (1, 8/3) (2, 0) (1, -8/3) (-1, -8/3) (-2, 0)};
       [anchor=mid west,
  mark size=+2pt, mark color=red,  ball color=green]
  \foreach \plm[count=\cnt] in {ball}
    \draw[mark options={fill=red}]
      plot[mark=\plm] coordinates {(0, 0)};
	\node[] at (-2.3, 0.3) {$T_- \mathbf{0}$}; \node[] at (2.3, 0.3) {$T_+ \mathbf{0}$}; \node[] at (-1.05, 8/3 + 0.3) {$U_+ \mathbf{0}$};  \node[] at (1.05, 8/3 +0.3) {$V_+ \mathbf{0}$}; \node[] at (-1.05, -8/3-0.3) {$V_- \mathbf{0}$};  \node[] at (1.05, -8/3 -0.3) {$U_- \mathbf{0}$}; \node[] at (-0.3, 0.2) {$\mathbf{0}$};
\end{tikzpicture}
     
\end{figure}

The generators $\lambda^2$, $\lambda^5$ and $\lambda^7$, on the other hand, are essentially rotations with fixed axis $z$, $y$ and $x$ respectively. Therefore
\begin{equation*}\mathrm{Span}< \lambda^2, \, \lambda^5, \, \lambda^7 > \cong \mathrm{so}(3, \, \R), \end{equation*}
and the isomorphism is given by sending $\lambda^2$ to $\frac{1}{2} \tau^3$, $\lambda^5$ to $\frac{1}{2} \tau^2$ and $\lambda^7$ to $\frac{1}{2} \tau^1$. We notice that
\begin{equation*}\mathrm{tr}( \lambda^a \lambda^b ) = \frac{1}{2} \delta_{ab} \quad \text{and} \quad \mathrm{tr}( \tau^a \tau^b ) = 2 \delta_{ab}, \end{equation*}
which means that this is a minimal embedding of $\mathrm{so}(3, \, \R)$ into $\mathrm{su}(2, \, \C)$.

\section{Quark Model}

In physics an hadron is a subatomic particle (not elementary) that is subject to the strong nuclear force and is formed by quarks, sometimes associated to anti-quarks. Denote
\begin{equation*} |\, q\rangle = \begin{pmatrix} u \\ d \\ s \end{pmatrix} \quad \text{and} \quad |\, \bar{q}\rangle = \begin{pmatrix} \bar{u} \\ \bar{d} \\ \bar{s} \end{pmatrix} \end{equation*}
the quark and the anti-quark respectively. 

\paragraph{Baryons}\index{baryons} All known baryons are made of three valence quarks, so they are fermions, i.e., they have half-integer spin. In particular, we denote by
\begin{equation*} p, \, n, \, \Sigma^{-}, \, \Sigma^0, \, \Sigma^+, \, \Lambda, \, \Xi^0, \, \Xi^- \end{equation*}
the collection of the eight baryons of the form $| \, qqq \rangle$.

\paragraph{Mesons.}\index{mesons} Mesons are hadrons composed of a quark-antiquark pair. They are bosons, meaning they have integer spin, i.e., $0, 1$, or $-1$. In particular, we denote by
\begin{equation*} \pi^-, \, \pi^0, \, \pi^{+}, \, \kappa^+, \, \kappa^0, \, \bar{\kappa}^0, \, \kappa^-, \, \eta \end{equation*}
the collection of the eight mesons of the form $|\, q\bar{q} \rangle$.

\paragraph{Representation.} We can employ the weight diagrams introduced in the previous section to describe these octets in such a way to reveal some of their symmetries.

\begin{figure}[!htbp]
        \centering
        \mbox{%
\begin{minipage}{.45\textwidth}
\begin{tikzpicture}
\draw[->,ultra thin] (-2.6,0)--(2.6,0) node[right]{$\lambda^3$};
\draw[->,ultra thin] (0,-3.6)--(0,3.6) node[above]{$Y$};
  [anchor=mid west,
  mark size=+2pt, mark color=red,  ball color=green]
  \foreach \plm[count=\cnt] in {ball}
    \draw[mark options={fill=red}]
      plot[mark=\plm] coordinates {(-2, 0) (-1, 8/3) (1, 8/3) (2, 0) (1, -8/3) (-1, -8/3) (-2, 0)};
       [anchor=mid west,
  mark size=+2pt, mark color=red,  ball color=green]
  \foreach \plm[count=\cnt] in {ball}
    \draw[mark options={fill=red}]
      plot[mark=\plm] coordinates {(0, 0)};
	\node[] at (-2.3, 0.3) {$\pi^-$}; \node[] at (2.3, 0.3) {$\pi^+$}; \node[] at (-1.05, 8/3+0.3) {$\kappa^0$};  \node[] at (1.05, 8/3+0.3) {$\kappa^+$}; \node[] at (-1.05, -8/3-0.3) {$\kappa^-$};  \node[] at (1.05, -8/3-0.3) {$\bar{\kappa}^0$}; \node[] at (-0.3, 0.2) {$\pi^0$}; \node[] at (0.3, 0.2) {$\eta$};
\end{tikzpicture}
          \end{minipage}%
          \qquad
          \begin{minipage}{.45\textwidth}
     \begin{tikzpicture}
\draw[->,ultra thin] (-2.6,0)--(2.6,0) node[right]{$\lambda^3$};
\draw[->,ultra thin] (0,-3.6)--(0,3.6) node[above]{$Y$};
  [anchor=mid west,
  mark size=+2pt, mark color=red,  ball color=green]
  \foreach \plm[count=\cnt] in {ball}
    \draw[mark options={fill=red}]
      plot[mark=\plm] coordinates {(-2, 0) (-1, 8/3) (1, 8/3) (2, 0) (1, -8/3) (-1, -8/3) (-2, 0)};
       [anchor=mid west,
  mark size=+2pt, mark color=red,  ball color=green]
  \foreach \plm[count=\cnt] in {ball}
    \draw[mark options={fill=red}]
      plot[mark=\plm] coordinates {(0, 0)};
	\node[] at (-2.3, 0.3) {$\Sigma^-$}; \node[] at (2.3, 0.3) {$\Sigma^+$}; \node[] at (-1.05, 8/3+0.3) {$n$};  \node[] at (1.05, 8/3+0.3) {$p$}; \node[] at (-1.05, -8/3-0.3) {$\Xi^-$};  \node[] at (1.05, -8/3-0.3) {$\Xi^0$}; \node[] at (-0.3, 0.2) {$\Sigma^0$}; \node[] at (0.3, 0.2) {$\Lambda$};
\end{tikzpicture}
          \end{minipage}
    }
    \caption{On the left the mesons, and on the right the baryons.}
\end{figure}

The weight diagrams representation is coherent with the fact that, for example, the couple $(n, \, p)$ is an iso-duplet and $(\pi^-, \, \pi^0, \, \pi^+)$ is an iso-triplet, as we have already proved earlier. 

\newpage

The symmetry of $\mathrm{SU}(2, \, \C)$ is a "good symmetry" in nature, but the $\mathrm{SU}(3, \, \C)$ symmetry is broken by the masses since $m_u$ and $m_d$ are very near, while $m_s$ is comparable with $\Lambda_{QCD}$. More precisely, assuming $c = 1$, we have that
\begin{equation*}\begin{rcases} m_u \sim 5 \, \text{MeV} \\ m_d \sim 10 \, \text{MeV}  \end{rcases} \ll \Lambda_{QCD} \sim 200 \, \text{MeV} \quad \text{while $m_s \sim 200 \, \text{MeV}$}. \end{equation*}

\subsection{Young Tableaux}
\index{Young tableaux}

In mathematics, a Young tableau is a combinatorial object useful in representation theory. It provides a convenient way to describe the group representations of the symmetric and general linear groups $\mathfrak{S}_n$ and to study their properties.

We will not develop this topic here. The interested reader may consult \cite[Chapter 25.5]{hassani}. We only note that Young diagrams are in one-to-one correspondence with irreducible representations of the symmetric group over the complex numbers.

There is a nice way to introduce Young diagrams as a natural consequence of the symmetry properties of a system of $n$ particles interacting between themselves. We consider
\begin{equation*}\psi_{a_1}(1) \dots \psi_{a_n}(n), \end{equation*}
where $a_i \in \{1, \, \dots, \, p\}$ is, in a certain sense, the set of all properties.

The idea is to consider the symmetrization, anti-symmetrization and a mix of both of this expression with respect to the exchange of particles, and denote them via Young diagrams. For example, if $n = 2$ we have
\begin{equation*} \yng(2) = \frac{1}{\sqrt{2}} \left( \psi_{a_1}(1)\psi_{a_2}(2) + \psi_{a_2}(1)\psi_{a_1}(2) \right), \end{equation*}
and
\begin{equation*} \yng(1,1) = \frac{1}{\sqrt{2}} \left( \psi_{a_1}(1)\psi_{a_2}(2) - \psi_{a_2}(1)\psi_{a_1}(2) \right). \end{equation*}
Denote by $\psi_{1, \, 2}^S$ the symmetrization with respect to the indices $(1, \, 2)$, and denote by $\psi_{1, \, 2}^A$ the anti-symmetrization with respect to the indices $(1, \, 2)$. If $n = 3$ we have either a totally symmetric tensor, a totally antisymmetric tensor or a mix as follows:
\begin{equation*} \yng(3) = \frac{1}{\sqrt{3!}} \sum_{\sigma \in \mathfrak{S}_3} \psi_{a_1}(\sigma(1))\psi_{a_2}(\sigma(2)) \psi_{a_3}(\sigma(3))= \psi_{1, \, 2, \, 3}^S, \end{equation*}
\begin{equation*} \yng(1,1,1) = \frac{1}{\sqrt{3!}} \sum_{\sigma \in \mathfrak{S}_3}(-1)^{\mathrm{sgn}(\sigma)} \psi_{a_1}(\sigma(1))\psi_{a_2}(\sigma(2)) \psi_{a_3}(\sigma(3)) = \psi_{1, \, 2, \, 3}^A, \end{equation*}
and
\begin{equation*} \yng(2,1) = \sum_{\sigma \in \mathfrak{S}_2}(-1)^{\mathrm{sgn}(\sigma)} \psi_{1, \, \sigma(2)}^S \psi_{a_3}(\sigma(3)). \end{equation*}
The total symmetrization and anti-symmetrization can be computed explicitly for any value of $n$ and $p$ as follows:
\begin{equation*} \yng(3) \dots \yng(1) = \frac{1}{\sqrt{n!}} \sum_{\sigma \in \mathfrak{S}_n} \psi_{a_1}(\sigma(1))\psi_{a_2}(\sigma(2))\dots \psi_{a_n}(\sigma(n)), \end{equation*}
and
\begin{equation*} \begin{matrix} \yng(1,1,1) \\ \vdots \\ \yng(1) \end{matrix} = \begin{cases} \frac{1}{\sqrt{n!}} \sum_{\sigma \in \mathfrak{S}_n} (-1)^{\mathrm{sgn}(\sigma)} \psi_{a_1}(\sigma(1))\psi_{a_2}(\sigma(2))\dots \psi_{a_n}(\sigma(n)) & \text{if $n \leq p$},
\\[1em] 0 & \text{if $p > n$}. \end{cases} \end{equation*}

\paragraph{Special Unitary Group.} Let us consider neutrons of the form
\begin{equation*} | \, N \rangle := \begin{pmatrix} p \\ n \end{pmatrix} \end{equation*}
subject to the action of the special unitary group $\mathrm{SU}(2, \, \C)$. It is easy to check that
\begin{equation} \label{eq.9.1} \frac{p_1n_2 - n_1p_2}{\sqrt{2}} = \yng(1,1) \sim \underline{1}, \end{equation}
that is, the antisymmetric form corresponds to the singlet $\underline{1}$ as a consequence of the fact that the relation above is invariant under the action of $\mathrm{SU}(2, \, \C)$. Indeed, consider the vectors
\begin{equation*} \begin{pmatrix} p_1^\prime \\ n_1^\prime \end{pmatrix} = \begin{pmatrix} a & -b^\ast \\b & a^\ast \end{pmatrix} \begin{pmatrix} p_1 \\ n_1 \end{pmatrix} = \begin{pmatrix} a p_1 - b^\ast n_1 \\ b p_1 + a^\ast n_1\end{pmatrix}, \end{equation*}
and
\begin{equation*} \begin{pmatrix} p_2^\prime \\ n_2^\prime \end{pmatrix} = \begin{pmatrix} a & -b^\ast \\ b & a^\ast \end{pmatrix} \begin{pmatrix} p_2 \\ n_2 \end{pmatrix} = \begin{pmatrix} a p_2 - b^\ast n_2 \\ -b p_2 + a^\ast n_2\end{pmatrix}. \end{equation*}
A straightforward computation yields to
\begin{equation*}\begin{aligned} p_1^\prime n_2^\prime - n_1^\prime p_2^\prime & = (a p_1 - b^\ast n_1)(b p_2 + a^\ast n_2) - ( b p_1 + a^\ast n_1)(a p_2 - b^\ast n_2) =
\\[1em] & = - b b^\ast n_1 p_2 - a a^\ast n_1 p_2 + a a^\ast p_1 n_2 + b b^\ast p_1 n_2 =
\\[1em] & = p_1 n_2 - n_1 p_2, \end{aligned}\end{equation*}
and this proves the invariance under the action of $\mathrm{SU}(2, \, \C)$. The antisymmetric form corresponds to the so-called \textit{deuteron}, whose electric charge is one. In a similar way, one can prove that
\begin{equation*} \yng(2) \sim \underline{3}, \qquad \yng(3) \sim \underline{4}, \qquad \text{etc...} \end{equation*}
On the other hand, here $n$ is equal to $2$, and this means that the unique totally antisymmetric setting is the invariant one described above, that is,
\begin{equation*} \yng(1,1,1) = 0. \end{equation*}
In particular, a Young diagram for the group $\mathrm{SU}(2, \, \C)$ is equivalent to a totally symmetric one since
\begin{equation*} \yng(2,1) = \yng(1) \qquad \text{or} \qquad \yng(3,1) = \yng(2) \end{equation*}

There is a simple rule, that can be found in \cite[Chapter 25.5]{hassani}, that allows us to find the decomposition of a direct product of two representations.

In the following example, we show that what we have already proved for $\mathrm{SU}(3, \, \C)$ may be obtained via Young diagrams in a simple and coherent way. We shall now describe that rule, but we do not present any proof here.

\caution[b][green][Note]{The following result and the first application is taken, almost verbatim, from \cite[Chapter 25.5]{hassani}.}

\begin{theorem}[Young Rule \cite{hassani}] To find the components of Young frames in the product of two Young frames, draw one of the frames. In the other frame, assign the same symbol, say $a$, to all boxes in the first row, the same symbol $b$ to all the boxes in the second row, etc. Now attach the first row to the first frame, and enlarge in all possible ways subject to the restriction that no two $a$'s appear in the same column, and that the result graph be regular. Repeat with the $b$'s etc., making sure in each step that as we read from right to left and top to bottom no symbols counted fewer times than the symbol that came after it. The product is the sum of all the diagrams obtained in this way \end{theorem}

To illustrate this procedure, we shall compute the product $\underline{8} \otimes \underline{8}$ between representations of $\mathrm{SU}(3, \, \C)$. More precisely, consider the product
\begin{equation*} \Yvcentermath1 \yng(2,1) \otimes \young(11,2)  \end{equation*}
We now apply the first row $\young(11)$ to the frame on the left, and we obtain the following four Young diagrams:
\begin{equation*} \Yvcentermath1 \young(\empty\empty11,\empty) \qquad \young(\empty\empty1,\empty1) \qquad \young(\empty\empty1,\empty,1) \qquad \young(\empty\:,\empty1,1)   \end{equation*}
Now we apply the second row $\young(2)$ to each of these graphs separately.

\paragraph{First Diagram.} We cannot put a $2$ to the right of the $1$'s, because in that case, as we count from right to left, we would start with a $2$ without having counted any $1$'s. The allowed graphs obtain from the first diagram are thus given by
\begin{equation*} \Yvcentermath1 \young(\empty\empty11,\empty2) \qquad \young(\empty\empty11,\empty,2)   \end{equation*}

\paragraph{Second Diagram.} Applying the $\young(2)$ to the second graph yields to
\begin{equation*} \Yvcentermath1 \young(\empty\empty1,\empty12) \qquad \young(\empty\empty1,\empty1,2) \end{equation*}

\paragraph{Third Diagram.} Applying the $\young(2)$ to the third graph gives
\begin{equation*} \Yvcentermath1 \young(\empty\empty1,\empty2,1) \qquad \young(\empty\empty1,\empty,1,2) \end{equation*}

\paragraph{Fourth Diagram.} Applying the $\young(2)$ to the fourth graph yields to
\begin{equation*} \Yvcentermath1  \young(\empty\:,\empty1,12) \qquad  \young(\empty\:,\empty1,1,2) \end{equation*}

In particular, the entire process described above can be easily written in terms of frames as follows:
\begin{equation*} \Yvcentermath1 \begin{aligned} \yng(2,1) \otimes \yng(2,1) & = \yng(4,2) + \yng(4,1,1) + \yng(3,3) \: + \dots \\[1em] & \dots + 2 \: \yng(3,2,1) + \yng(3,1,1,1) + \yng(2,2,2) + \yng(2,2,1,1) \end{aligned} \end{equation*}
On the other hand, we are dealing with two representations of $\mathrm{SU}(3, \, \C)$, which means that the Young column of length $3$ is the singlet. The result is thus given by
\begin{equation*} \Yvcentermath1 \begin{aligned}\underline{8} \otimes \underline{8} & = \yng(4,2) + \yng(3) + \yng(3,3) \: + \dots \\[1em] & \dots + 2 \: \yng(2,1) + \yng(2,2,2) =
\\[1em] & = \underline{27} + \underline{10} + \underline{10}^\ast + \underline{8} + \underline{8} + \underline{1} \end{aligned} \end{equation*}
The fifth addendum and the seventh addendum are zero because $N = p = 3$, and therefore every column of length $4$ or more is automatically zero (as we already mentioned above).

\begin{example}The fundamental representation of $\mathrm{SU}(3, \, \C)$ is given by
\begin{equation*} \underline{3} \sim \yng(1) \end{equation*}
while its adjoint is given by
\begin{equation*} \underline{3}^\ast \sim \yng(1,1) \end{equation*}
The reader may check, as an exercise, that the following computations are correct and compare them with the results we already know.
\begin{equation*} \Yvcentermath1 \begin{aligned} & \underline{3} \otimes \underline{3} \sim  \yng(1) \otimes \yng(1) = \yng(2) + \yng(1,1)  \sim \underline{6} \oplus \underline{3}^\ast
\\[1em] & \underline{6} \otimes \underline{3} \sim \yng(2) \otimes \yng(1) =  \yng(3) + \yng(2,1) \sim \underline{10} \oplus \underline{8}
\\[1em] & \underline{6} \otimes \underline{6} \sim \yng(2) \otimes \yng(2) =  \yng(4) + \yng(3,1) + \yng(2,2)
\\[1em] &\underline{3} \otimes \underline{3}^\ast \sim \yng(1) \otimes \yng(1,1) = \yng(2,1) + \yng(1,1,1)
	 \sim \underline{8} \oplus \underline{1}
\\[1em] & \underline{3}^\ast \otimes \underline{3}^\ast \sim \yng(1,1) \otimes \yng(1,1) = \yng(1,1,1,1) + \yng(2,1,1) + \yng(2,2) \end{aligned}\end{equation*}

 \end{example}
 
We now want to generalize the relation \eqref{eq.9.1}, that is, in the special unitary group $\mathrm{SU}(2, \, \C)$ the totally antisymmetric form corresponds to the singlet $\underline{1}$.

More precisely, we shall prove that the Young diagram with $N$ rows and $1$ column is the singlet in $\mathrm{SU}(N, \, \C)$ for all $N \in \N$. First, notice that we have
\begin{equation*} \begin{matrix} \yng(1,1,1) \\ \vdots \\ \yng(1) \end{matrix} = \frac{1}{\sqrt{n!}} \sum_{\sigma \in \mathfrak{S}_n} (-1)^{\mathrm{sgn}(\sigma)} \psi_{a_1}(\sigma(1))\psi_{a_2}(\sigma(2))\dots \psi_{a_n}(\sigma(n)), \end{equation*}
where the $a_i$s takes value in the set $\{1, \, \dots, \, n\}$. Let $U \in \mathrm{SU}(N, \, \C)$ be an arbitrary unitary transformation, and consider
\begin{equation*}\psi_i^\prime = U \psi_i \implies (\psi_i^\prime)_j = U_{j, \, \ell} (\psi_i)_\ell, \end{equation*}
where $(\psi_i)_\ell$ denotes the $\ell$th component of the $i$th vector. It suffices to prove that
\begin{equation*}\sum_{\sigma \in \mathfrak{S}_n} (-1)^{\mathrm{sgn}(\sigma)} \psi_{1}(\sigma(1))\psi_{2}(\sigma(2))\dots \psi_{n}(\sigma(n)) = \sum_{\sigma \in \mathfrak{S}_n} (-1)^{\mathrm{sgn}(\sigma)} \psi_{1}^\prime(\sigma(1))\psi_{2}^\prime(\sigma(2))\dots \psi_{n}^\prime(\sigma(n)), \end{equation*}
where $\psi_{i}(\sigma(j))$ denotes the component $(\psi_i)_j$. A simple computation yields to
\begin{equation*}\begin{aligned} \sum_{\sigma \in \mathfrak{S}_n} (-1)^{\mathrm{sgn}(\sigma)} \psi_{1}^\prime(\sigma(1))\dots \psi_{n}^\prime(\sigma(n)) & = \sum_{\sigma \in \mathfrak{S}_n} (-1)^{\mathrm{sgn}(\sigma)} U_{\sigma(1), \, \ell} (\psi_1)_\ell \dots U_{\sigma(n), \, \ell} (\psi_{n})_\ell =
\\[1em] & = \epsilon^{\sigma(1) \dots \sigma(n)} U_{\sigma(1), \, \ell} (\psi_1)_\ell \dots U_{\sigma(n), \, \ell} (\psi_{n})_\ell =
\\[1em] & = \epsilon^{\sigma(1) \dots \sigma(n)}  \mathrm{det}(U)  (\psi_1)_{\sigma(1)} \dots  (\psi_{n})_{\sigma(n)} =
\\[1em] & = \sum_{\sigma \in \mathfrak{S}_n} (-1)^{\mathrm{sgn}(\sigma)} \psi_{1}(\sigma(1)) \dots \psi_{n}(\sigma(n)), \end{aligned}\end{equation*}
which is exactly what we wanted to prove.

\subsection{Adjoint Representation}

The Young frame of the adjoint representation of a given representation is extremely easy to find, especially when we are dealing with $\mathrm{SU}(N, \, \C)$.

Indeed, we know that the Young column of length $N$ corresponds to the singlet $\underline{1}$ in $\mathrm{SU}(N, \, \C)$, and therefore, given a representation $\mathscr{R}$ with a Young frame, we can find the Young frame of $\mathscr{R}^\ast$ as the "smallest" one that attached to the one of $\mathscr{R}$ gives a singlet.

\begin{example}The adjoint of the fundamental representation $\underline{3} \sim \yng(1)$ is given by $\underline{3}^\ast \sim \yng(1,1)$ since
\begin{equation*} \begin{matrix} \yng(1,1) \\ \updownarrow \\ \yng(1) \end{matrix} = \yng(1,1,1) \sim \underline{1}. \end{equation*}
The adjoint of the representation $\underline{6} \sim \yng(2)$ is given by the square, i.e.
\begin{equation*} \underline{6}^\ast \sim \yng(2,2) \end{equation*}
since
\begin{equation*} \begin{matrix} \yng(2) \\ \updownarrow \\ \yng(2,2) \end{matrix} = \yng(2,2,2) \sim \underline{1}. \end{equation*} \end{example}

\subsection{Multiplicity}

There is an easy way to find, given an arbitrary Young frame, the corresponding representation in $\mathrm{SU}(3)$. Indeed, if we consider the number $p_1$ of squares in the first row that have no other square attached below them, and the number $p_2$ of squares in the first row that have no other square attached below them, then
\begin{equation*} N_3 = \frac{1}{2} (p_1 + 1)(p_1 + p_2 + 2)(p_2 + 1) \end{equation*}
is the corresponding representation. For example, we have that
\begin{equation*} \yng(3,1) \sim \underline{15} \end{equation*}
because $p_1 = 2$ and $p_2 = 1$.

\subsection{Baryons, Resonances and Colors Model}

Recall that the known baryons are made of three valence quarks, so they are fermions, i.e., they have half-integer spin. In particular, we denote by
\begin{equation*} p, \, n, \, \Sigma^{-}, \, \Sigma^0, \, \Sigma^+, \, \Lambda, \, \Xi^0, \, \Xi^- \end{equation*}
the collection of the eight baryons of the form $| \, qqq \rangle$. Using Young diagrams we infer that
\begin{equation*} \Yvcentermath1 
    |\,qqq \rangle \sim  \yng(1) \otimes \yng(1) \otimes \yng(1) = \yng(3) + \yng(2,1) + \yng(2,1) + \yng(1,1,1)  \sim \underline{10} \oplus \underline{8} \oplus \underline{8} \oplus \underline{1},\end{equation*}
and therefore we would like to know something more specific about the composition of the decuplet. We will not give any details, as this topic will most likely be presented in a better way in a Quantum Physics course, but we simply recall some basic facts.

The four $\Delta$ baryons form a quartet in the weight diagram of $\underline{10}$, and they are given by $\Delta^{++}$ (constituent quarks: $| \,uuu\rangle$), $\Delta^+$ ($|\,uud\rangle$), $\Delta^0$ ($|\,udd\rangle$), and $\Delta^-$ ($|\,ddd\rangle$), which respectively carry an electric charge of $+2$, $+1$, $0$, and $-1$. They have spin and isospin $\frac{3}{2}$, and mass $\sim 1240 \, \text{$MeV$/$c^2$}$.
\newpage

\begin{figure}[!htbp]
        \centering
        \mbox{%
\begin{tikzpicture}
\draw[->,ultra thin] (-4.1,0)--(4.1,0) node[right]{$\lambda^3$};
\draw[->,ultra thin] (0,-7.1)--(0,4.1) node[above]{$Y$};
  [anchor=mid west,
  mark size=+2pt, mark color=red,  ball color=green]
  \foreach \plm[count=\cnt] in {ball}
    \draw[mark options={fill=red}]
      plot[mark=\plm] coordinates {(-1, 0) (-3/2,3) (-1/2, 3) (1/2, 3) (3/2, 3) (1, 0) (1/2, -3) (0, -6) (-1/2, -3) (-1, 0)};
       [anchor=mid west,
  mark size=+2pt, mark color=red,  ball color=green]
  \foreach \plm[count=\cnt] in {ball}
    \draw[mark options={fill=red}]
      plot[mark=\plm] coordinates {(0, 0)};
	\node[] at (-1.55, 3.3) {$\Delta^-$}; \node[] at (-1.5, 0.3) {$\Sigma^{\ast -}$}; \node[] at (1.5, 0.3) {$\Sigma^{\ast+}$}; \node[] at (-0.55, 3.3) {$\Delta^0$};  \node[] at (0.55, 3.3) {$\Delta^+$};  \node[] at (1.55, 3.3) {$\Delta^{++}$}; \node[] at (-0.85, -3.3) {$\Xi^{\ast-}$};  \node[] at (0.85, -3.3) {$\Xi^{\ast 0}$}; \node[] at (-0.3, 0.3) {$\Sigma^{ \ast 0}$}; \node[] at (0.4, -6.1) {$\Omega^-$}; 
\end{tikzpicture}
    }
    \caption{The decuplet of baryons. Note that the charge is constant on each diagonal and takes value in $\{-1, \, 0, \, 1, \, 2\}$.}
\end{figure}


The baryon $\Omega^-$ is given by $|\, sss \rangle$ and, as we will see in a few moments, this is one of the reasons why we need to introduce the notion of \textit{colors}. First, notice that
\begin{equation*} m_{\Omega^-} \sim 1650 \, \text{$MeV$/$c^2$}, \end{equation*}
which is way different from the energy of the $\Delta$ baryons since the symmetry in $\mathrm{SU}(3, \, \C)$ is broken by the introduction of the strange quark $s$, which brings a mass comparable to $\Lambda_{QCD}$, and this does not happen for $u$ and $d$.

Furthermore, the baryon $|\, sss \rangle$ is totally symmetric with respect to the space, and this is a direct contradiction with the Fermi-Dirac statistic, which asserts that it should be totally antisymmetric. 

\section{Fundamental Representation in $\mathrm{SU}(2, \, \C) \times \mathrm{U}(1, \, \C)$}

In this section, we want to decompose the irreducible representation $\underline{8}$ of $\mathrm{SU}(3, \, \C)$ in the product of irreducible representations of $\mathrm{SU}(2, \, \C) \times \mathrm{U}(1, \, \C)$. Recall that
\begin{equation*} Y = \frac{2}{\sqrt{3}} T^8 \quad \text{and} \quad \underline{3} \sim \yng(1) \sim \left(2, \, \frac{1}{3} \right) \oplus \underbracket{\left(1, \, - \frac{2}{3} \right)}_{\sim \: \young(3)}\end{equation*}
It follows that
\begin{equation*} \begin{aligned} \underline{8} & \sim \yng(2,1) = \young(\:3,\:) + \young(33,\empty) + \yng(2,1) + \young(\:\:,3) =
\\[1em] & = (\underline{1}, \, 0) +  (\underline{2}, \, -1) +  (\underline{2}, \, 1) +  (\underline{3}, \, 0). \end{aligned} \end{equation*}
since both ${\tiny \young(33,3)}$ and ${\tiny \young(3\:,\:)}$ are zero.
\chapter{Special Orthogonal Group $\mathrm{SO}(4, \, \R)$} \thispagestyle{empty}

In this chapter, we study more in-depth the special unitary (Lie) group $\mathrm{SO}(4, \, \R)$ and the associated Lie algebra $\mathrm{so}(4, \, \R)$. Recall that the set of all orthogonal matrices
\begin{equation*} \mathrm{O}(N, \, \R) := \left\{ O \in \mathrm{GL}(N, \, \R) \: : \: O^T O = O O^T = \mathrm{Id}_{N \times N} \right\} \end{equation*}
is a group with respect to the matrix product. Similarly,
\begin{equation*} \mathrm{SO}(N, \, \R) := \left\{ O \in \mathrm{GL}(N, \, \R) \: : \: O^T O = O O^T = \mathrm{Id}_{N \times N}, \: \mathrm{det}(O) = 1 \right\} \end{equation*}
is also a group, and it is called \textit{special orthogonal group}.

\section{Representations of $\mathrm{SO}(4, \, \R)$}

The six generators of the $4$-dimensional special orthogonal group can be computed explicitly in the representation $\underline{4}$, and they are given by the following matrices:
\begin{equation*}\begin{aligned} & M^{23} = \begin{pmatrix} 0 & 0 & 0 & 0 \\ 0 & 0 & - \imath & 0 \\ 0 & \imath & 0 & 0 \\ 0 & 0 & 0 & 0 \end{pmatrix}, \qquad M^{31} = \begin{pmatrix} 0 & 0 & \imath & 0 \\ 0 & 0 & 0 & 0 \\ -\imath & 0 & 0 & 0 \\ 0 & 0 & 0 & 0 \end{pmatrix}, \qquad M^{12} = \begin{pmatrix} 0 & -\imath & 0 & 0 \\ \imath & 0 & 0 & 0 \\ 0 & 0 & 0 & 0 \\ 0 & 0 & 0 & 0 \end{pmatrix},
\\[1em] & M^{14} = \begin{pmatrix} 0 & 0 & 0 & -\imath \\ 0 & 0 & 0 & 0 \\ 0 & 0 & 0 & 0 \\ \imath & 0 & 0 & 0 \end{pmatrix}, \qquad M^{24} = \begin{pmatrix} 0 & 0 & 0 & 0 \\ 0 & 0 & 0 & -\imath \\ 0 & 0 & 0 & 0 \\ 0 & \imath & 0 & 0 \end{pmatrix}, \qquad M^{34} = \begin{pmatrix} 0 & 0 & 0 & 0 \\ 0 & 0 & 0 & 0 \\ 0 & 0 & 0 & -\imath \\ 0 & 0 & \imath & 0 \end{pmatrix} .\end{aligned} \end{equation*}
We immediately see that $\mathrm{SO}(4, \, \R)$ is generated by the rotations restricted to all the possible coordinate planes. More precisely, the matrix $M^{ij}$ is a rotation on the plane $\mathrm{Span} \langle i, \, \ j \rangle$.

The elements of $\mathrm{SO}(4, \, \R)$ can be easily computed via the exponentiation of the matrix $M^{ij}$ since we already know that
\begin{equation*} \mathrm{e}^{ \imath \theta {\tiny \begin{pmatrix} 0 & - \imath \\ \imath & 0 \end{pmatrix}} } = \begin{pmatrix} \cos \theta & \sin \theta \\ - \sin \theta & \cos \theta \end{pmatrix}. \end{equation*}
In particular, it turns out that
\begin{equation*} \left( \mathrm{e}^{ \imath \theta M^{ij} } \right)_{k, \, \ell} = \begin{cases} 1 & \text{if $k = \ell$ and $k \neq i$, $k \neq j$},\\[0.5em] \cos \theta & \text{if $k = \ell = i$ or $k=\ell=j$}, \\[0.5em] \sin \theta & \text{if $(k, \, \ell) = (i, \, j)$}, \\[0.5em] - \sin \theta & \text{if $(k, \, \ell) = (j, \, i)$}, \\[0.5em] 0 & \text{otherwise.}\end{cases} \end{equation*}
This computation is quite surprising because, for one thing, it implies that $\underline{4}$ is not the smallest nontrivial representation of $\mathrm{SO}(4, \, \R)$. Indeed, as we will be able to show soon, the associated Lie algebra $\mathrm{so}(4, \, \R)$ can be decomposed as the direct product of two invariant Lie algebras.

Our goal is now to find the relations between the generators, that is, $[M^{ij}, \, M^{k \ell}]$. To complete this task, we first rewrite the generators in a more compact manner in terms of the Dirac deltas, that is,
\begin{equation} \label{eq.12.1} (M^{ij})_{k \ell} = - \imath (\delta_{ik} \delta_{jl} - \delta_{il} \delta_{jk}). \end{equation}
It follows from \eqref{eq.12.1} that
\begin{equation*}[M^{ij}, \, M^{k \ell}] = \begin{cases} 0 & \text{if $\{i, \, j \} \cap \{k, \, \ell \} = \varnothing$}, \\[0.5em] - \imath M^{ik} & \text{if $\{i, \, j \} \cap \{k, \, \ell \} \neq \varnothing$}.\end{cases} \end{equation*}
The algebra $\mathrm{so}(4, \, \R)$ is thus given by the direct product of two invariant subalgebras if we consider a new set of generators defined by
\begin{equation*}\begin{aligned} & S^1 = \frac{1}{2} (M^{23} + M^{41}), \qquad \hat{S}^1 = \frac{1}{2} (M^{23} - M^{41}),
\\[1em] & S^2 = \frac{1}{2} (M^{31} + M^{42}), \qquad \hat{S}^2 = \frac{1}{2} (M^{31} - M^{42}),
\\[1em] & S^3 = \frac{1}{2} (M^{12} + M^{43}), \qquad \hat{S}^3 = \frac{1}{2} (M^{12} - M^{43}).\end{aligned} \end{equation*}
In fact, the reader can quickly verify that from \eqref{eq.12.1} it follows that
\begin{equation*}\begin{aligned} & [S^i, \, S^j] = \imath \epsilon^{ijk} S^k,
\\[1em] & [\hat{S}^i, \, \hat{S}^j] = \imath \epsilon^{ijk} \hat{S}^k,
\\[1em] & [S^i, \, \hat{S}^j] = 0,\end{aligned} \end{equation*}
which means that
\begin{equation*}\mathrm{so}(4, \, \R) \cong \mathrm{su}(2, \, \C) \otimes \mathrm{su}(2, \, \C) = \mathrm{Span}\langle S^1, \, S^2, \, S^3 \rangle \otimes \mathrm{Span} \langle \hat{S}^1, \, \hat{S}^2, \, \hat{S}^3 \rangle. \end{equation*}
In particular, the Lie algebra $\mathrm{so}(4, \, \R)$ is not simple, but one can easily show that it is semisimple since no invariant subalgebra is abelian (e.g., $\mathrm{su}(2, \, \C)$ is not commutative).

Consequently, there exists a bijective correspondence between the representations of $\mathrm{so}(4, \, \R)$ and the representation of the direct product $\mathrm{su}(2, \, \C) \otimes \mathrm{su}(2, \, \C)$. For example,
\begin{equation*} (\underline{1}, \, \underline{1}) \leftrightarrow \underline{1} \quad \text{or} \quad (\underline{2}, \, \underline{2}) \leftrightarrow \underline{4}, \end{equation*}
while the mixed representations $(\underline{2}, \, \underline{1})$ and $(\underline{1}, \, \underline{2})$ are usually called \textit{spin representation}\index{spin representation}.

Notice that these do not represent Dirac spinors because both $(\underline{2}, \, \underline{1})$ and $(\underline{1}, \, \underline{2})$ are given by elements with four components. Since only two of them are linearly independent, it turns out that these are Majorana spinors.
\chapter{Euclidean Groups $E_n$} \thispagestyle{empty}
\label{ch:e2}

In this chapter, we study more in-depth the special unitary (Lie) group $E_n$ and the associated Lie algebra, denoted by $e_2$.

Recall that $E_n$ is the symmetry group of the $n$-dimensional Euclidean space ($\R^n$). Its elements are the isometries associated with the Euclidean distance and are called Euclidean isometries or Euclidean transformations.

\section{Two-Dimensional Euclidean Group $E_2$}

The two-dimensional Euclidean group $E_2$ is the symmetry group of the plane, and its elements are the transformations of the form
\begin{equation}\label{eq.15.1} \begin{pmatrix} x \\ y \end{pmatrix} \longmapsto \begin{pmatrix} \cos \theta & - \sin \theta \\ \sin \theta & \cos\theta \end{pmatrix} \begin{pmatrix} x \\ y \end{pmatrix} + \begin{pmatrix} a_1 \\ a_2 \end{pmatrix}, \end{equation}
where $\theta \in [0, \, 2 \pi)$ is the rotation angle, and $\mathbf{a} = (a_1, \, a_2)^T \in \R^2$ denotes the translation.

In the previous chapters, we proved that the generic transformation of the form \eqref{eq.15.1} may be equivalently rewritten as a $3$-dimensional transformation:
\begin{equation*} \begin{pmatrix} x \\ y \\ 1 \end{pmatrix} \longmapsto \left(\begin{array}{@{}c|c@{}}
  R(\theta) &
  \begin{matrix}
  a_1 \\
  a_2
  \end{matrix}
\\ \hline
  \begin{matrix}
  0 & 0
  \end{matrix}
  & 1
\end{array}\right) \begin{pmatrix} x \\ y \\ 1 \end{pmatrix}, \end{equation*}
where $R(\theta)$ denotes the rotation of angle $\theta$, that is,
\begin{equation*}R(\theta) = \begin{pmatrix} \cos \theta & - \sin \theta \\ \sin \theta & \cos\theta \end{pmatrix}. \end{equation*}
The generators of the Lie algebra (computed in the second chapter) are given by
\begin{equation*} \begin{cases} P^1 = - \imath \frac{\partial}{\partial x}, \\[1em] P^2 = - \imath \frac{\partial}{\partial y}, \\[1em] J = - \imath \left( x \frac{\partial}{\partial y} - y \frac{\partial}{\partial x} \right), \end{cases} \end{equation*}
since one can easily check that
\begin{equation*} \begin{aligned} & \mathrm{e}^{\imath a_1 P^1} \begin{pmatrix}x \\ y \end{pmatrix} \approx (\mathrm{id}_{2 \times 2} + \imath a_1 P^1) \begin{pmatrix}x \\ y \end{pmatrix} = \begin{pmatrix}x + a_1 \\ y \end{pmatrix},
\\[1em] & \mathrm{e}^{\imath a_2 P^2} \begin{pmatrix}x \\ y \end{pmatrix} \approx (\mathrm{id}_{2 \times 2} + \imath a_2 P^2) \begin{pmatrix}x \\ y \end{pmatrix} = \begin{pmatrix}x  \\ y + a_2 \end{pmatrix},
\\[1em] & \mathrm{e}^{\imath \theta J} \begin{pmatrix}x \\ y \end{pmatrix} \approx (\mathrm{id}_{2 \times 2} + \imath \theta J) \begin{pmatrix}x \\ y \end{pmatrix} = \begin{pmatrix}x \\ y \end{pmatrix} + \begin{pmatrix} - \theta y \\ \theta x \end{pmatrix}. \end{aligned} \end{equation*}
We can easily compute the value of the commutators
\begin{equation*} \begin{aligned} & [P^1, \, P^2] = 0, \\[1em] & [P^1, \, J] = - \imath P^2, \\[1em] & [P^2, \, J] = \imath P^1 \end{aligned} \end{equation*}
from which we infer that, for any $a \in \{1, \, 2, \, J\}$, we have
\begin{equation*} f^{12a} = 0 \qquad \text{and} \qquad f^{1 J 2} = - f^{2 J 1} = -1. \end{equation*}
Recall that the action of any element of the group $E_2$ can also be seen as the action of a $3$-dimensional matrix, which will be denoted from now on by $g(\mathbf{a}, \, \theta)$:
\begin{equation*} \begin{pmatrix} x \\ y \\ 1 \end{pmatrix} \longmapsto \left(\begin{array}{@{}c|c@{}}
  R(\theta) &
  \begin{matrix}
  a_1 \\
  a_2
  \end{matrix}
\\ \hline
  \begin{matrix}
  0 & 0
  \end{matrix}
  & 1
\end{array}\right) \begin{pmatrix} x \\ y \\ 1 \end{pmatrix} =: g(\mathbf{a}, \, \theta) \begin{pmatrix} x \\ y \\ 1 \end{pmatrix}. \end{equation*}
The product between any two elements is given by
\begin{equation*} g(\mathbf{a}, \, \theta) g(\mathbf{b}, \, \phi) = g\left( R(\theta)\mathbf{b} + \mathbf{a}, \, \theta + \phi \right), \end{equation*}
and this is the composition rule that characterize the two-dimensional Euclidean group $E_2$. Using this notation, we can easily write the generators $P^1$, $P^2$ and $J$ as follows:
\begin{equation*} J = \begin{pmatrix} 0 & - \imath & 0 \\ \imath & 0 & 0 \\ 0 & 0 & 0 \end{pmatrix}, \qquad P^1 = \begin{pmatrix} 0 & 0 & \imath \\ 0 & 0 & 0 \\ 0 & 0 & 0 \end{pmatrix}, \qquad P^2 = \begin{pmatrix} 0 & 0 & 0 \\ 0 & 0 & \imath \\ 0 & 0 & 0 \end{pmatrix}.\end{equation*}

\begin{remark} Notice that we \textbf{cannot} use the representation of the Euclidean group
\begin{equation*} \begin{pmatrix} x \\ y \\ 1 \end{pmatrix} \longmapsto \left(\begin{array}{@{}c|c@{}}
  R(\theta) &
  \begin{matrix}
  a_1 \\
  a_2
  \end{matrix}
\\ \hline
  \begin{matrix}
  0 & 0
  \end{matrix}
  & 1
\end{array}\right) \end{equation*}
to find the generators of $E_2$. The reason is that, although the form is useful to multiply the elements of $E_2$, it is not the right vector space (and we shall see soon that a similar thing happens for the Poincaré group.) \end{remark}

We can easily compute the elements in this particular representation by noticing that
\begin{equation*}(P^i)^2 = 0 \quad \text{for $i = 1, \, 2$} \implies (P^i)^{n} = 0 \quad \text{for $i = 1, \, 2$ and for all $n \geq 2$}, \end{equation*}
which means that the exponential power series has only two nonzero terms, i.e.,
\begin{equation*} \mathrm{e}^{\imath a_1 P^1}  = \mathrm{id}_{3 \times 3} - \imath a_1 P^1 = g( (a_1, \, 0), \, 0)  \quad \text{and} \quad \mathrm{e}^{\imath a_2 P^2}  = \mathrm{id}_{3 \times 3} - \imath a_2 P^2 = g( (0, \, a_2), \, 0) \end{equation*}
while the rotation is computed as usual:
\begin{equation*} \mathrm{e}^{\imath \theta J}  = \begin{pmatrix} \cos \theta &  \sin \theta & 0 \\ - \sin \theta & \cos \theta & 0 \\ 0 & 0 & 1 \end{pmatrix} = R(-\theta) = g(\mathbf{0}, \, - \theta). \end{equation*}
Notice also that
\begin{equation*} \mathfrak{t} := \mathrm{Span} \langle P^1, \, P^2 \rangle \end{equation*}
is an invariant subalgebra, generated by $P^1$ and $P^2$, which is also abelian. In particular, the algebra associated with $E_2$ is not semisimple (and thus, it is not simple), and we can always write
\begin{equation*} g(\mathbf{a}, \, \theta) = g(\mathbf{b}, \, 0) g(\mathbf{0}, \, \theta) \implies e_2 = \mathfrak{t} \otimes \mathrm{so}(2, \, \R). \end{equation*}

We now want to study the action of a rotation on the generators $P^i$, for $i = 1, \, 2$, in order to show that the translations form a normal subgroup of $E_2$. First, notice that
\begin{equation*}\begin{aligned} \mathrm{e}^{- \imath \theta J} P^1 \mathrm{e}^{\imath \theta J} & = g(\mathbf{0}, \, \theta) P^1 g(\mathbf{0}, \, -\theta) =
\\[1em] & = \begin{pmatrix} \cos \theta & - \sin \theta & 0 \\ \sin \theta & \cos \theta & 0 \\ 0 & 0 & 1 \end{pmatrix} \begin{pmatrix} 0 & 0 & \imath \\ 0 & 0 & 0 \\ 0 & 0 & 0 \end{pmatrix}\begin{pmatrix} \cos \theta &  \sin \theta & 0 \\ - \sin \theta & \cos \theta & 0 \\ 0 & 0 & 1 \end{pmatrix}  =
\\[1em] & = \begin{pmatrix} \cos \theta & - \sin \theta & 0 \\ \sin \theta & \cos \theta & 0 \\ 0 & 0 & 1 \end{pmatrix} \begin{pmatrix} 0 & 0 & \imath \\ 0 & 0 & 0 \\ 0 & 0 & 0 \end{pmatrix}  =
\\[1em] & = \begin{pmatrix} 0 & 0 & \imath \cos \theta \\ 0 & 0 & \imath \sin \theta \\ 0 & 0 & 0 \end{pmatrix} = P^1 \cos \theta + P^2 \sin \theta = P^i g(\mathbf{0}, \, \theta)_{i, \, 1},  \end{aligned} \end{equation*}
and, in a similar way, we also have that
\begin{equation*}\begin{aligned} \mathrm{e}^{- \imath \theta J} P^2 \mathrm{e}^{\imath \theta J} & = g(\mathbf{0}, \, \theta) P^1 g(\mathbf{0}, \, -\theta) =
\\[1em] & = \begin{pmatrix} \cos \theta & - \sin \theta & 0 \\ \sin \theta & \cos \theta & 0 \\ 0 & 0 & 1 \end{pmatrix} \begin{pmatrix} 0 & 0 & 0 \\ 0 & 0 & \imath \\ 0 & 0 & 0 \end{pmatrix}\begin{pmatrix} \cos \theta &  \sin \theta & 0 \\ - \sin \theta & \cos \theta & 0 \\ 0 & 0 & 1 \end{pmatrix}  =
\\[1em] & = \begin{pmatrix} \cos \theta & - \sin \theta & 0 \\ \sin \theta & \cos \theta & 0 \\ 0 & 0 & 1 \end{pmatrix} \begin{pmatrix} 0 & 0 & 0 \\ 0 & 0 & \imath \\ 0 & 0 & 0 \end{pmatrix}  =
\\[1em] & = \begin{pmatrix} 0 & 0 & - \imath \sin \theta \\ 0 & 0 & \imath \cos \theta \\ 0 & 0 & 0 \end{pmatrix} = - P^1 \sin \theta + P^2 \cos \theta = P^i g(\mathbf{0}, \, \theta)_{i, \, 2}.  \end{aligned} \end{equation*}
It follows that
\begin{equation*}\begin{aligned} g(\mathbf{0}, \, \theta) g(\mathbf{a}, \, 0) g(\mathbf{0}, \, \theta)^{-1} &= \mathrm{e}^{- \imath \theta R} \mathrm{e}^{\imath (a_1 P^1 + a_2 P^2)} \mathrm{e}^{\imath \theta R} =
\\[1em] & = \mathrm{e}^{- \imath \theta R} \left( \mathrm{id}_{3 \times 3} - \imath a_1 P^1 - \imath a_2 P^2  \right)\mathrm{e}^{\imath \theta R} =
\\[1em] & = \mathrm{id}_{3 \times 3} - \imath a_1 P^i g(\mathbf{0}, \, \theta)_{i, \, 1} - \imath a_2 P^i g(\mathbf{0}, \, \theta)_{i, \, 2} =
\\[1em] & = \begin{pmatrix} 1 & 0 & a_1 \cos \theta - a_2 \sin \theta \\ 0 & 1 & a_1 \sin \theta + a_2 \cos \theta \\ 0 & 0 & 1 \end{pmatrix} = g(\mathbf{a}^\prime, \, 0), \end{aligned} \end{equation*}
where $\mathbf{a}^\prime$ is a new translation vector defined by the product
\begin{equation*}\mathbf{a}^\prime := R(\theta) \mathbf{a}. \end{equation*}

\begin{theorem}The subgroup of all translation
\begin{equation*} \mathfrak{T} := \left\{ g(\mathbf{b}, \, 0) \: : \: \mathbf{b} \in \R^2 \right\} \subset E_2 \end{equation*}
is an invariant/normal subgroup of $E_2$.\end{theorem}

\begin{proof} Let $g(\mathbf{a}, \, 0) \in \mathfrak{T}$ be a translation, and let $g(\mathbf{b}, \, \theta)$ be an arbitrary element of $E_2$. From the identity proved above, it follows that
\begin{equation*} \begin{aligned} g(\mathbf{b}, \, \theta) g(\mathbf{a}, \, 0) g(\mathbf{b}, \, \theta)^{-1} & = g(\mathbf{b}, \, 0) g(\mathbf{0}, \, \theta) g(\mathbf{a}, \, 0) g(\mathbf{0}, \, \theta)^{-1} g(- \mathbf{b}, \, 0) =
\\[1em] & = g(\mathbf{b}, \, 0) g(R(\theta)\mathbf{a}, \, 0) g(-\mathbf{b}, \,0) =
\\[1em] & = g(R(\theta)\mathbf{a}, \, 0) \in \mathfrak{T},\end{aligned} \end{equation*}
and this concludes the proof. \end{proof}

In particular, we can consider the quotient group $\faktor{E_2}{\mathfrak{T}}$. It is interesting to notice that any element in the quotient depends on the rotation only since
\begin{equation*} \left[g(\mathbf{c}, \, \theta) \right] \in \faktor{E_2}{\mathfrak{T}} \implies g(\mathbf{a}, \, 0) \left[ g(\mathbf{c}, \, \theta) \right] = \left[ g(\mathbf{c}, \, \theta) \right],  \end{equation*}
and therefore
\begin{equation*}\begin{aligned} [g(\mathbf{c}, \, \theta)] & = \left\{ g(\mathbf{b}, \, \theta) \: : \: \mathbf{b} = \mathbf{c} + \mathbf{a}, \, \, \mathbf{a} \in \R^2 \right\} =
\\[1em] & = \left\{ g(\mathbf{b}, \, \theta) \: : \: \mathbf{b} \in \R^2 \right\}. \end{aligned}  \end{equation*}
In particular, the elements of the quotient group $\faktor{E_2}{\mathfrak{T}}$ depends on the rotation angle $\theta$ only, which means that
\begin{equation*} \faktor{E_2}{\mathfrak{T}} \longrightarrow \mathrm{SO}(2, \, \R), \qquad [g(\mathbf{b}, \, \theta)] \longmapsto R(\theta)  \end{equation*}
is an isomorphism, that is,
\begin{equation*} \faktor{E_2}{\mathfrak{T}} \cong \mathrm{SO}(2, \, \R). \end{equation*}

\subsection{Irreducible Representation of $E_2$}

In this section, our goal will be to determine all the finite-dimensional irreducible representations of the Euclidean group $E_2$.

\begin{remark}The Euclidean group $E_2$ is not compact since the norm of the elements $g(\mathbf{b}, \, \theta)$ is arbitrarily big (and, hence, not bounded). Unitary representations of non-compact non-abelian Lie groups (like $E_2$) tend to be infinite-dimensional, as we will see in a few moments.\end{remark}

\paragraph{Trivial Representation.} Fix $m \in \Z$. There is a "trivial" representation given by
\begin{equation} \label{eq.12.4} g(\mathbf{b}, \, \theta) \longmapsto U_m(\mathbf{b}, \, \theta) := \mathrm{e}^{\imath m \theta}. \end{equation}
To assure that \eqref{eq.12.4} actually defines a representation, we simply need to check the associativity property. Indeed, from the product formula
\begin{equation*} g(\mathbf{a}, \, \theta) g(\mathbf{b}, \, \phi) = g( R(\theta)\mathbf{b} + \mathbf{a}, \, \theta + \phi),  \end{equation*}
we immediately infer that
\begin{equation*} g(\mathbf{a}, \, \theta) g(\mathbf{b}, \, \phi) \longmapsto \mathrm{e}^{\imath m (\theta + \phi)} = U_m(\mathbf{a}, \, \theta) U_m(\mathbf{b}, \, \phi), \end{equation*}
which means that \eqref{eq.12.4} satisfies the associative property.

The problem with this particular representation is that it does not bring any information about the translations, and therefore it is not a \textit{faithful} representation\index{representation!faithful}. On the other hand, the map \eqref{eq.12.4} gives a faithful representation of the quotient space
\begin{equation} \label{eq.12.6} [g(\mathbf{0}, \, \theta)] \longmapsto U_m(\theta) := \mathrm{e}^{\imath m \theta}. \end{equation}

\paragraph{Infinite-Dimensional Representation.} Let us consider the Casimir operator relative to the translation invariant subalgebra, i.e.,
\begin{equation*} \mathbb{P}^2 := (P^1)^2 + (P^2)^2. \end{equation*}
Recall that $\mathbb{P}^2$ commutes with every other generator $P^1$, $P^2$ and $J$. Indeed, a straightforward computation shows that
\begin{equation*} \begin{aligned} [\mathbb{P}^2, \, J] & = P^1[P^1, \, J] + [P^1, \, J]P^1 + P^2[P^2, \, J] + [P^2, \, J]P^2 =
\\[1em] & = - \imath P^1P^2 - \imath P^2 P^1 + \imath P^2 P^1 + \imath P^1 P^2 = 0.  \end{aligned} \end{equation*}
We also consider the up/down operators $P_{\pm}$, defined by setting
\begin{equation*}P_{\pm} = P^1 \pm \imath P^2, \end{equation*}
and we notice that
\begin{equation*} [P_{\pm}, \, P_{\pm}] = 0 \quad \text{and} \quad [J, \, P_{\pm}] = \pm P_{\pm}. \end{equation*}
In particular, the Casimir operator $\mathbb{P}^2$ commutes with the generator $J$, which means that there exists a common basis of eigenstates (i.e., they are simultaneously diagonalizable). More precisely, let us consider an orthonormal basis $|p, \, m \rangle$ such that
\begin{equation*}\begin{cases} J \,| \, p, \, m \rangle = m \,| \, p, \, m \rangle, \\[0.4em] \mathbb{P}^2 \, | \, p, \, m \rangle = p^2 \,| \, p, \, m \rangle, \end{cases} \quad \text{and} \quad \begin{cases} \langle p, \, m \, | \,  p, \, m \rangle = 1, \\[0.4em] \langle p, \, m \, | \, p, \, m^\prime \rangle = 0. \end{cases} \end{equation*}
In order to understand how the up/down operators $P_{\pm}$ acts on the eigenstates, we first notice that
\begin{equation*} J \left( P_+ \, | \, p, \, m \rangle \right) = P_+ \, | \, p, \, m \rangle + m P_+ \, | \, p, \, m \rangle = (m + 1)P_+ \, | \, p, \, m \rangle, \end{equation*}
which means that there exists a complex $c \in \C$ such that $P_+ \, | \, p, \, m \rangle = c \, | \,p, \, m +1\rangle$. To compute the absolute value of $c$, we employ the orthogonality and the identity above as follows:
\begin{equation*}|c|^2 = \langle p, \, m + 1 \, | \, |c|^2 \, | \, p, \, m +1 \rangle = \langle p, \, m \, | \, P_- P_+ \, | \, p, \, m  \rangle = p^2 \underbracket{\langle p, \, m \,  | \, p, \, m  \rangle}_{=1}.  \end{equation*}
In particular, we have that
\begin{equation*}|c|^2 = p^2 \implies c = \pm \frac{p}{\imath} \implies P_{\pm} \, |p, \, m \rangle = \pm \frac{p}{\imath} \, | \, p, \, m \pm 1 \rangle, \end{equation*}
and therefore we need to consider every possible value of $m$ to have a basis of eigenstates $\{ |p, \, m \rangle \}_{m \in \Z}$, which clearly gives us a infinite-dimensional vector space. It follows that
\begin{equation*}g(\mathbf{0}, \, \theta) \, | \, p, \, m \rangle = \mathrm{e}^{- \imath \theta J} \, | \, p, \, m \rangle = \mathrm{e}^{- \imath m \theta} \, | \, p, \, m \rangle, \end{equation*}
which means that the "trivial" representation presented above gives the eigenvectors of the pure rotations in $E_2$. Therefore, if we employ the fact that the basis is orthonormal, we find that
\begin{equation*}\langle p, \, m^\prime \, | \, g(\mathbf{0}, \, \theta) \, | \, p, \, m \rangle = \delta_{m, \, m^\prime}  \mathrm{e}^{- \imath m \theta}. \end{equation*}
We now want to do the same computation with a translation element $g(\mathbf{a}, \, \theta)$, but, surprisingly, it requires quite a lot of work. First, recall that
\begin{equation*} g(\mathbf{0}, \, \theta) g(\mathbf{a}, \, 0) g(\mathbf{0}, \, \theta)^{-1} = g(\mathbf{a}^\prime, \, 0), \end{equation*}
and hence we can always write $\mathbf{a}$ as a suitable rotation.

Namely, let $\phi$ be the angle between the $x$-axis and the vector $\mathbf{a}$, and consider the projection $\mathbf{a}_0 := (a, \, 0)$ for $a$ equal to the length of $\mathbf{a}$, i.e., $a = |\mathbf{a}|$. Then
\begin{equation*} \mathbf{a} = R(\phi) \mathbf{a}_0 \implies \mathrm{e}^{- \imath P \cdot \mathbf{a}} = g(\mathbf{0}, \, \phi) g(\mathbf{a}_0, \, 0) g(\mathbf{0}, \, \phi)^{-1}, \end{equation*}
which yields to the following identity:
\begin{equation*} g(\mathbf{a}, \, \theta) = g(\mathbf{0}, \, \theta) g(\mathbf{0}, \, \phi) g(\mathbf{a}_0, \, 0) g(\mathbf{0}, \, \phi)^{-1}.\end{equation*}
A straightforward computation shows that
\begin{equation*}\begin{aligned} \langle p, \, m^\prime \, | \, g(\mathbf{a}, \, \theta) \, | \, p, \, m \rangle & = \langle p, \, m^\prime \, | \, g(\mathbf{0}, \, \theta) g(\mathbf{0}, \, \phi) g(\mathbf{a}_0, \, 0) g(\mathbf{0}, \, \phi)^{-1} \, | \, p, \, m \rangle =
\\[1em] & = \mathrm{e}^{- \imath m \theta} \langle p, \, m^\prime \, | \, g(\mathbf{0}, \, \phi) g(\mathbf{a}_0, \, 0) g(\mathbf{0}, \, \phi)^{-1} \, | \, p, \, m \rangle =
\\[1em] & = \mathrm{e}^{- \imath m \theta} \mathrm{e}^{\imath \phi(m - m^\prime)} \langle p, \, m^\prime \, | \,\mathrm{e}^{- \imath P \cdot \mathbf{a}_0} \, | \, p, \, m \rangle = 
\\[1em] & = \mathrm{e}^{- \imath m \theta} \mathrm{e}^{\imath \phi(m - m^\prime)} \langle p, \, m^\prime \, | \,\mathrm{e}^{- \imath P^1 a} \, | \, p, \, m \rangle =
\\[1em] & = \mathrm{e}^{- \imath m \theta} \mathrm{e}^{\imath \phi(m - m^\prime)} \langle p, \, m^\prime \, | \,\mathrm{e}^{\frac{- \imath a}{2} (P_+ + P_-)} \, | \, p, \, m \rangle =
\\[1em] & = \mathrm{e}^{- \imath m \theta} \mathrm{e}^{\imath \phi(m - m^\prime)} \sum_{k, \, \ell = 0}^{+ \infty}\frac{1}{k! \ell!} \left( \frac{-\imath a}{2} \right)^{k + \ell} \langle p, \, m^\prime \, | \,P_+^k P_-^\ell \, | \, p, \, m \rangle =
\\[1em] & = \mathrm{e}^{- \imath m \theta} \mathrm{e}^{\imath \phi(m - m^\prime)} \sum_{k, \, \ell = 0}^{+ \infty}\frac{1}{k! \ell!} \left( \frac{-\imath a}{2} \right)^{k + \ell} \left( \frac{p}{\imath} \right)^k \left(- \frac{p}{\imath} \right)^\ell \delta_{m^\prime, \, m + k - \ell } =
\\[1em] & = \mathrm{e}^{- \imath m \theta} \mathrm{e}^{\imath \phi(m - m^\prime)} \sum_{k, \, \ell = 0}^{+ \infty}\frac{(-1)^k}{k! \ell!} \left( \frac{p a}{2} \right)^{k + \ell} \delta_{m^\prime, \, m + k - \ell }. \end{aligned} \end{equation*}
Notice that for $m > m^\prime$ we can consider $\ell := m - m^\prime + k$, and obtain
\begin{equation*}\begin{aligned} \langle p, \, m^\prime \, | \, g(\mathbf{a}, \, \theta) \, | \, p, \, m \rangle & = \mathrm{e}^{- \imath m \theta} \mathrm{e}^{\imath \phi(m - m^\prime)} \sum_{k = 0}^{+ \infty}\frac{(-1)^k}{k! (m - m^\prime + k)!} \left( \frac{p a}{2} \right)^{2k + m - m^\prime} =
\\[1em] & = \mathrm{e}^{- \imath m \theta} \mathrm{e}^{\imath \phi(m - m^\prime)} \left( \frac{p a}{2} \right)^{m - m^\prime} \sum_{k = 0}^{+ \infty}\frac{(-1)^k}{k! (m - m^\prime + k)!} \left( \frac{p a}{2} \right)^{2k} =
\\[1em] & =  \mathrm{e}^{- \imath m \theta} \mathrm{e}^{\imath \phi(m - m^\prime)} J_{m - m^\prime}(ap), \end{aligned}  \end{equation*}
where $J_\nu(z)$ is the \textit{Bessel function}\index{Bessel function} defined by
\begin{equation} \label{bessel} J_\nu(z) := \left( \frac{z}{2} \right)^{\nu} \: \sum_{k = 0}^{+ \infty}\frac{(-1)^k \left( \frac{z}{2}\right)^{2k}}{k! \, \Gamma(\nu + k + 1)}. \end{equation}
Recall that the Bessel function is the solution, regular at $z = 0$, of the second-order differential equation
\begin{equation*} \frac{\mathrm{d}^2}{\mathrm{d}z^2} u(z) + \frac{1}{z} \frac{\mathrm{d}}{\mathrm{d}z} u(z) + \left(1 - \frac{\nu^2}{z^2} \right) u(z) = 0, \end{equation*}
while the \textit{Gamma function}\index{Gamma function} is defined by setting
\begin{equation*}\Gamma(z) := \int_0^{+ \infty} t^{z-1} \mathrm{e}^{-t} \, \mathrm{d}t \quad \text{for all $\mathfrak{Re}(z) > 0$}, \end{equation*}
which can be easily extended analytically to the whole complex plane except at the integers ($\Z_-$) less than $0$. Furthermore, the Bessel function satisfies a nontrivial symmetry property
\begin{equation} \label{bessel2} J_{-\nu}(z) = (-1)^\nu J_\nu(z) \quad \text{for all $\nu \in \Z$}.\end{equation}
Now, notice that for $m^\prime > m$ we can consider $k := m^\prime - m + \ell$, and obtain
\begin{equation*}\begin{aligned} \langle p, \, m^\prime \, | \, g(\mathbf{a}, \, \theta) \, | \, p, \, m \rangle & = \mathrm{e}^{- \imath m \theta} \mathrm{e}^{\imath \phi(m - m^\prime)} \sum_{\ell = 0}^{+ \infty}\frac{(-1)^{m^\prime - m + \ell}}{(m^\prime - m + \ell)! \ell!} \left( \frac{p a}{2} \right)^{2\ell + m^\prime - m} =
\\[1em] & = \mathrm{e}^{- \imath m \theta} \mathrm{e}^{\imath \phi(m - m^\prime)} \left( - \frac{p a}{2} \right)^{m^\prime - m} \sum_{\ell = 0}^{+ \infty}\frac{(-1)^\ell}{(m^\prime - m + \ell)! \ell!} \left( \frac{p a}{2} \right)^{2\ell} =
\\[1em] & =  \mathrm{e}^{- \imath m \theta} \mathrm{e}^{\imath \phi(m - m^\prime)} (-1)^{m^\prime - m} J_{m^\prime - m}(ap) =
\\[1em] & \stackrel{\eqref{bessel2}}{=} \mathrm{e}^{- \imath m \theta} \mathrm{e}^{\imath \phi(m - m^\prime)} J_{m - m^\prime}(ap), \end{aligned}  \end{equation*}
which means that for all $m, \, m^\prime \in \Z$ we have
\begin{equation*}\langle p, \, m^\prime \, | \, g(\mathbf{a}, \, \theta) \, | \, p, \, m \rangle = \mathrm{e}^{- \imath m \theta} \mathrm{e}^{\imath \phi(m - m^\prime)} J_{m - m^\prime}(ap).  \end{equation*}

\begin{remark} The representation presented here reduces to the "trivial" one when $p^2 = 0$. \end{remark}

\begin{remark} If $p^2$ is different from zero, we can choose a different system of coordinates. The Casimir operator $\mathbb{P}^2$ commutes with $P^1$ as well, and therefore we can find a simultaneous basis of eigenstates $| \, p, \, 0 \rangle$ in such a way that
\begin{equation*} P^1 \, | \, p, \, 0 \rangle = p \, | \, p, \, 0 \rangle \quad \text{and} \quad P^2 \, | \, p, \, 0 \rangle = 0, \end{equation*}
while the Casimir operator gives
\begin{equation*} \mathbb{P}^2 \, | \, p, \, 0 \rangle = p^2 \, | \, p, \, 0 \rangle. \end{equation*}
A straightforward computation also proves that
\begin{equation*}J \, | \, p, \, 0 \rangle = | \, R(\theta) \mathbf{p}_0 \rangle \quad \text{where $\mathbf{p}_0 = (p, \, 0)$}. \end{equation*} \end{remark}
\chapter{Lorentz Group} \thispagestyle{empty}

The \textit{Lorentz group}, denoted by $\mathrm{O}(1, \, 3)$, is the group of all the invertible matrices $\Lambda \in \mathrm{GL}(4, \, \R)$ preserving the following transformation
\begin{equation}\label{eq.20.1} \Lambda_\mu^\eta g_{\eta \lambda} \Lambda_\nu^\lambda = g_{\mu \nu}, \end{equation}
where
\begin{equation*} g \equiv (g_{\mu \nu})_{\mu, \, \nu = 0, \, \dots, \, 3} = \begin{pmatrix} 1 & 0 & 0 & 0 \\ 0 & - 1 & 0 & 0 \\ 0 & 0 & -1 & 0 \\ 0 & 0 & 0 & -1 \end{pmatrix} \end{equation*}
is the Minkowski space-time metric. The transformations of the Lorentz group are thus given by
\begin{equation} \label{eq.20.2} p^\mu \longmapsto \Lambda_\nu^\mu p^\nu, \end{equation}
where $\Lambda$ is a regular matrix satisfying \eqref{eq.20.1} and $p^\mu$ is a contravariant $4$-vector\index{contravariant $4$-vector}. 

\section{Introduction}

In this chapter we will mainly focus on a class of equivalence of the Lorentz group $\mathrm{O}(1, \, 3)$, namely the connected component $\mathrm{SO}^+(1, \, 3)$ (usually called \textit{proper orthochronous Lorentz group}\index{proper orthochronous Lorentz group}), which will be introduced shortly.

We now introduce a particular notation for the inverse of $g$ in such a way that the Einstein convention works just fine, that is, 
\begin{equation*}  g^{\mu \nu} := (g^{-1})_{\mu \nu} \quad \text{for $\mu, \, \nu = 0, \, \dots, \, 3$}. \end{equation*}
Recall also that $p^\mu$ denotes a contravariant $4$-vector, and $p_\mu$ a covariant $4$-vector\index{covariant $4$-vector}. It follows that the scalar product induced by the Minkowski metric $g_{\mu \nu}$ can be compactly written as
\begin{equation} \label{eq.20.3} p^\mu q_\nu := p^\mu q^\nu g_{\mu \nu} = p_\mu q_\nu g^{\mu \nu}. \end{equation}
Notice also that from \eqref{eq.20.1} we necessarily have
\begin{equation*} \left[ \mathrm{det}(\Lambda)\right]^2 = 1 \implies \mathrm{det}(\Lambda) = \pm 1, \end{equation*}
and we decide (arbitrarily) to work in the component of the space with
\begin{equation} \label{eq.20.c1} \mathrm{det}(\Lambda) = 1. \end{equation}
Similarly, notice that
\begin{equation*}\Lambda_\mu^\eta g_{\eta \lambda} \Lambda_\nu^\lambda = g_{\mu \nu} \implies (\Lambda_0^0)^2 - \sum_{i = 1}^3 (\Lambda_0^i)^2 = 1 \implies (\Lambda_0^0)^2 \geq 1, \end{equation*}
which means that either $\Lambda_0^0$ is greater than or equal to $1$ or less than or equal to $- 1$. We decide (arbitrarily) to work in the component of the space where
\begin{equation} \label{eq.20.c2} \Lambda_0^0 \geq 1. \end{equation}
The connected component of $\mathrm{O}(1, \, 3)$ satisfying the properties \eqref{eq.20.1}, \eqref{eq.20.c1} and \eqref{eq.20.c2} is called \textit{proper orthochronous Lorentz group}, and it is usually denoted by $\mathfrak{T}_+$.

\begin{remark} Recall that the Levi-Civita tensor $\epsilon_{\mu \nu \lambda \sigma}$ is defined by setting
\begin{equation*}\epsilon_{\mu \nu \lambda \sigma} := \begin{cases} 1 & \text{if $(\mu \nu \lambda \sigma)$ even permutation of $(0 1 2 3)$}, \\[0.8em] - 1 & \text{if $(\mu \nu \lambda \sigma)$ odd permutation of $(0 1 2 3)$}, \\[0.8em] 0 & \text{otherwise}. \end{cases} \end{equation*}
The reader may check that the matrices $\Lambda \in \mathfrak{T}_+$ satisfies the following useful property
\begin{equation*}\epsilon_{\mu \nu \lambda \sigma} = \Lambda_\mu^\alpha \Lambda_\nu^\beta \Lambda_\lambda^\gamma \Lambda_\sigma^\delta \epsilon_{\alpha \beta \gamma \delta} \end{equation*}
by means of the following well-known formula for the determinant:
\begin{equation*}1 = \mathrm{det}(\Lambda) = \epsilon_{\mu \nu \lambda \sigma} \Lambda_0^\mu \Lambda_0^\nu \Lambda_0^\lambda \Lambda_0^\sigma.  \end{equation*}
In particular, the Levi-Civita tensor is called \textit{invariant tensor} for the proper Lorentz group (similarly to the Levi-Civita tensor $\epsilon_{ijk}$ for the group $\mathrm{SU}(3, \, \C)$).\end{remark}

\section{Irreducible Representations of $\mathfrak{T}_+$: Part I}

In this section, we investigate the irreducible representations of the group $\mathfrak{T}_+$, and we take a closer look at the Lie algebra.

\subsection{Generators}

Recall that the Lie group $\mathrm{SO}(4, \, \R)$ is generated by the six matrices $M^{ij}$, where the couple $(i, \, j)$ denotes the rotation plane.

Using the same notation, we now introduce the generators of $\mathfrak{T}_+$, and we show that - accordingly to what happens in $\mathrm{SO}(4, \, \R)$ - only $6$ of them are actually necessary. More precisely, we start with the generators
\begin{equation*} (J_{\lambda \sigma})_\mu^\nu := - \imath \left( \delta_\lambda^\nu g_{\sigma \mu} - \delta_\sigma^\nu g_{\lambda \mu} \right) \quad \text{for $\lambda, \, \sigma = 0, \, \dots, \, 3$}, \end{equation*}
and we notice that
\begin{equation*} J_{\lambda \sigma} = - J_{\sigma \lambda}, \end{equation*}
so that only $6$ of them are actually necessary to generate the Lie group $\mathfrak{T}_+$. We define
\begin{equation*} K_m := J_{m, \, 0} \quad \text{for $m = 1, \, 2, \, 3$}, \end{equation*}
and these are nothing more than the \textit{boost}\index{Lorentz boosts} in the directions $\hat{x}$, $\hat{y}$ and $\hat{z}$ respectively. In a similar fashion, we define
\begin{equation*} J_k := \frac{1}{2} \epsilon^{kmn} J_{m, \, n} \quad \text{for $k = 1, \, 2, \, 3$}, \end{equation*}
and these are the generators of the rotation subgroup isomorphic to $\mathrm{SO}^+(3, \, \R)$ (i.e., the rotations with respect to the space component of the Minkowski space-time). Clearly, these six matrices generate the Lie group $\mathfrak{T}_+$, and it is easy to see that
\begin{equation*} \Lambda_{\mathrm{rot}}^{\alpha \theta \phi} := \begin{pmatrix}1 & 0 & 0 & 0 \\ 0 & \cos \alpha \cos \theta \cos \phi - \sin \alpha \sin \phi & - \cos \alpha \cos\theta \sin \phi - \sin \alpha \cos \phi & \cos\alpha \sin \theta  \\ 0 & \sin \alpha \cos \theta \cos \phi - \cos \alpha \sin \phi & - \sin \alpha \cos\theta \sin \phi + \cos \alpha \cos \phi & \sin\alpha \sin \theta  \\ 0 & - \sin \theta \cos \phi & \sin \theta \sin \phi & \cos \theta \end{pmatrix} \end{equation*}
is the transformation in $\mathfrak{T}_+$ associated to $\{J_k\}_{k = 1, \, 2, \, 3}$, while
\begin{equation*} \Lambda_{\mathrm{boost}}^{\gamma \beta \hat{z}} := \begin{pmatrix} \gamma & 0 & 0 & \beta \gamma \\ 0 & 1 & 0  & 0 \\ 0 & 0 & 1 & 0 \\ \beta \gamma & 0 & 0 & \gamma \end{pmatrix} \end{equation*}
is, for example, the boost transformation along the $z$-axis. The parameters of the boost elements are nothing else than the Lorentz transformations parameters, that is,
\begin{equation*} \beta = \frac{v}{c} \in [-1, \, 1] \quad \text{and} \quad \gamma = \frac{1}{\sqrt{1 - \beta^2}} \geq 1. \end{equation*}

\begin{remark} There is an alternative parametrization for the boost elements\index{Lorentz boosts alternative}. Namely, we replace $\beta$ with the hyperbolic tangent of another parameter, that is, we set
\begin{equation*} \beta := \tanh \omega. \end{equation*}
Then $\omega$ takes value in $(- \infty, \, + \infty)$, and therefore
\begin{equation*} \gamma = \frac{1}{\sqrt{1 - \tanh^2 \omega}} = \cosh \omega \geq 1, \end{equation*}
which gives us the boost transformation
\begin{equation*} \Lambda_{\mathrm{boost}}^{\gamma \omega \hat{z}} := \begin{pmatrix} \cosh \omega & 0 & 0 & \sinh \omega \\ 0 & 1 & 0  & 0 \\ 0 & 0 & 1 & 0 \\ \sinh \omega & 0 & 0 & \cosh \omega \end{pmatrix}. \end{equation*}
\end{remark}

\subsection{Lie Algebra of $\mathfrak{T}_+$}

In this section, we investigate the Lie algebra associated to $\mathfrak{T}_+$, denoted by $\mathrm{so}(3, \, 1)$, by means of different sets of generators. First, a simple computation shows that
\begin{equation*}\begin{aligned}( J_{\mu \nu} )_\beta^\alpha (J_{\sigma \lambda})_\gamma^\beta & = (- \imath)^2 \left( \delta_\mu^\alpha g_{\nu \beta} - \delta_\nu^\alpha g_{\mu \beta} \right)\left( \delta_\sigma^\beta g_{\lambda \gamma} - \delta_\lambda^\beta g_{\sigma \gamma} \right) =
\\[1em] & = - \delta_\mu^\alpha g_{\nu \beta} \delta_\sigma^\beta g_{\lambda \gamma} + \delta_\mu^\alpha g_{\nu \beta} \delta_\lambda^\beta g_{\sigma \gamma} + \delta_\nu^\alpha g_{\mu \beta} \delta_\sigma^\beta g_{\lambda \gamma} - \delta_\nu^\alpha g_{\mu \beta} \delta_\lambda^\beta g_{\sigma \gamma} =
\\[1em] & = - \delta_\mu^\alpha g_{\nu \sigma} g_{\lambda \gamma} + \delta_\mu^\alpha g_{\nu \lambda} g_{\sigma \gamma} + \delta_\nu^\alpha g_{\mu \sigma} g_{\lambda \gamma} - \delta_\nu^\alpha g_{\mu \lambda} g_{\sigma \gamma} =
\\[1em] & = \imath (J_{\mu \nu})_\lambda^\alpha g_{\sigma \gamma} - \imath (J_{\mu \nu})_\sigma^\alpha g_{\lambda \gamma}, \end{aligned}\end{equation*}
and therefore
\begin{equation*}\begin{aligned} [J_{\mu \nu}, \, J_{\sigma \lambda}]_{\gamma}^\alpha & = ( J_{\mu \nu} )_\beta^\alpha (J_{\sigma \lambda})_\gamma^\beta - ( J_{\sigma \lambda} )_\beta^\alpha (J_{\mu \nu})_\gamma^\beta =
\\[1em] & =  \imath \left[ (J_{\mu \nu})_\lambda^\alpha g_{\sigma \gamma} - (J_{\mu \nu})_\sigma^\alpha g_{\lambda \gamma} - \left( (J_{\sigma \lambda})_\nu^\alpha g_{\mu \gamma} -  (J_{\sigma \lambda})_\mu^\alpha g_{\nu \gamma} \right) \right] =
\\[1em] & = \imath \left[ (J_{\mu \nu})_\lambda^\alpha g_{\sigma \gamma} - (J_{\mu \nu})_\sigma^\alpha g_{\lambda \gamma} - (J_{\sigma \lambda})_\nu^\alpha g_{\mu \gamma} + (J_{\sigma \lambda})_\mu^\alpha g_{\nu \gamma} \right]. \end{aligned}\end{equation*}
In order to simplify the Lie algebra, it is convenient to use the $3 + 3$ generators introduced above, that is, $\{J_k\}_{k = 1, \, 2, \, 3}$. and $\{ K_m \}_{m = 1, \, 2, \, 3}$. It turns out that
\begin{equation*}\begin{aligned} & [J_m, \, J_n] = \imath \epsilon_{mnk} J_k, \\[1em] & [K_m, \, J_n] = \imath \epsilon_{mn \ell} K_\ell, \\[1em] & [K_m, \, K_n] = - \imath \epsilon_{mn\ell} J_\ell, \end{aligned}\end{equation*}
and therefore the set of generators $\{J_k\}_{k = 1, \, 2, \, 3}$ forms an invariant subalgebra isomorphic to $\mathrm{so}(3, \, \R)$, that is,
\begin{equation*} \langle J_1, \, J_2, \, J_3 \rangle \cong \mathrm{so}(3, \, \R) \subset \mathrm{so}(3, \, 1). \end{equation*}
Now let
\begin{equation*} M_n := \frac{1}{2}(J_n + \imath K_n) \quad \text{and} \quad M_n := \frac{1}{2} (J_n - \imath K_n) \end{equation*}
for $n = 1, \, 2, \, 3$, and notice that
\begin{equation*}\begin{aligned} & [M_m, \, M_n] = \imath \epsilon_{mn \ell} M_\ell, \\[1em] & [N_m, \, N_n] =- \imath \epsilon_{mn \ell} N_\ell, \\[1em] & [M_m, \, N_n] =0, \end{aligned}\end{equation*}
and therefore the sets of generators form two invariant subalgebra isomorphic to $\mathrm{su}(2, \, \C)$. It follows that the Lie algebra of the Lorentz group $\mathfrak{T}_+$ is isomorphic to the product of two copies of the Lie algebra associated to $\mathrm{SU}(2, \, \C)$, that is,
\begin{equation*} \mathrm{so}(3, \, 1) \sim \langle M_n, \, N_m \rangle_{m, \, n = 1, \, 2, \, 3} \sim \mathrm{su}(2, \, \C) \times \mathrm{su}(2, \, \C). \end{equation*}
It follows that the irreducible finite-dimensional representations of the Lorentz group $\mathfrak{T}_+$ are in correspondence with the couples $(j_1, \, j_2)$, where $j_1, \, j_2 \in 1/2 \Z$ denote the irreducible representation of $\mathrm{SU}(2, \, \C)$ we have already investigated thoroughly. For example
\begin{equation*} \text{$(0, \, 0)$ is the trivial one-dimensional representation}, \end{equation*}
while 
\begin{equation*} \text{$(\frac{1}{2}, \, 0)$ and $(0, \, \frac{1}{2})$} \end{equation*}
are the spinorial representations of chirality left $(L)$ and right $(R)$ respectively\index{representation!spinorial}.

\section{Spinorial Representations}

The goal of this brief section is to expand a little bit on the last notion we introduced in the previous discussion: the spinorial representations.

\begin{remark} The Lorentz group $\mathfrak{T}_+$ is not compact since the boost's parameter $\gamma$ can be arbitrarily big ($\gamma \geq 1$), and therefore finite-dimensional representations need not to be unitary. \end{remark}

First, recall that the special linear group $\mathrm{SL}(2, \, \C)$ is a $6$-parameters Lie group because an arbitrary element is given by
\begin{equation*} \begin{pmatrix} \alpha_1 + \imath \beta_1 & \alpha_2 + \imath \beta_2 \\ \alpha_3 + \imath \beta_3 & \alpha_4 + \imath \beta_4 \end{pmatrix} \quad \text{for $\alpha_i, \, \beta_i \in \R$}, \end{equation*}
satisfying the constraint $\mathrm{det} A = 1$, which gives us two equations (real and imaginary part).

In the first half of this section, the main goal is to prove that every transformation in the Lorentz group $\mathfrak{T}_+$ corresponds to an element of the group $\mathrm{SL}(2, \, \C)$, and infer that
\begin{equation*}\mathfrak{T}_+ \cong \mathrm{SL}(2, \, \C). \end{equation*}
Denote by $\sigma^i$ the $i$th Pauli's matrix. We introduce an useful notation that takes into account the particular form of the Minkowski metric, that is,
\begin{equation*}\sigma^\mu := (- \mathrm{id}_{2 \times 2}, \, \sigma^i)_{i = 1, \, 2, \, 3}, \end{equation*}
and
\begin{equation*}\bar{\sigma}^\mu := (- \mathrm{id}_{2 \times 2}, \, - \sigma^i)_{i = 1, \, 2, \, 3}. \end{equation*}
Let $p^\mu$ be a given $4$-vector. We associate to $p^\mu$ the matrix
\begin{equation*}P := p^\mu \sigma_\mu, \end{equation*}
where $\sigma_\mu = \sigma^\nu g_{\mu \nu}$. A straightforward computation shows that
\begin{equation*}P  = p^0 \sigma^0 - p^i \sigma^i = \begin{pmatrix}- p^0 - p^3 & - p^1 + \imath p^2 \\ - p^1 - \imath p^2 & - p^0 + p^3 \end{pmatrix}, \end{equation*}
and, therefore, we can easily reverse this operation and find that
\begin{equation*}p^\mu = \frac{1}{2} \mathrm{Tr} \left[ \bar{\sigma}^\mu P \right], \end{equation*}
where $P$ is the matrix above. In particular, every $4$-vector $p^\mu$ corresponds to a matrix, denoted by the capital $P$, and therefore it suffices to show that $\mathrm{SL}(2, \, \C)$ acts on $P$ in the same way $\mathfrak{T}_+$ acts on $p^\mu$ to infer that
\begin{equation*}\mathfrak{T}_+ \cong \mathrm{SL}(2, \, \C). \end{equation*}
Let $M \in \mathrm{SL}(2, \, \C)$. The action on $P$ is defined by
\begin{equation*}P \longmapsto M P M^\dagger \equiv P^\prime, \end{equation*}
and this shows immediately that
\begin{equation*}p^\mu p_\mu = \mathrm{det} P = \mathrm{det} P^\prime = (p^\prime)^\mu p_\mu^\prime. \end{equation*}

\begin{example} Let $\phi \in [0, \, 2 \pi)$. The matrix $M$ could be chosen in such a way that
\begin{equation*} M_\theta := \mathrm{e}^{\frac{\imath}{2} \phi \sigma_3} = \begin{pmatrix} \mathrm{e}^{\imath \frac{\phi}{2}} & 0 \\ 0 & \mathrm{e}^{-\imath \frac{\phi}{2}} \end{pmatrix} \quad \text{or} \quad M = \mathrm{e}^{\frac{\imath}{2} \phi \sigma_2} = \begin{pmatrix} \cos \frac{\phi}{2} & \sin \frac{\phi}{2} \\ - \sin \frac{\phi}{2} & \cos \frac{\phi}{2} \end{pmatrix}. \end{equation*}
Furthermore, if $\omega \in (- \infty, \, + \infty)$ is the parameter introduced above, it makes sense to consider the real exponential
\begin{equation*} M_\omega := \mathrm{e}^{\frac{1}{2} \omega \sigma_3} = \begin{pmatrix} \mathrm{e}^{ \frac{\omega}{2}} & 0 \\ 0 & \mathrm{e}^{- \frac{\omega}{2}} \end{pmatrix} \end{equation*}
since, given a $3$-dimensional vector $\underline{v}$, it is easy to see that
\begin{equation*} \text{$\mathrm{Tr}(\sigma^i) = 0$ for all $i = 1, \, 2, \, 3$} \implies \mathrm{det} \left[ \mathrm{e}^{\frac{1}{2} \underline{v} \cdot \underline{\sigma} } \right] = 1. \end{equation*} \end{example}

Let us focus on the complex exponential transformation induced by the matrix $M_\theta$ introduced above. A straightforward computation shows that
\begin{equation*}\begin{aligned} P \longmapsto P^\prime & = \begin{pmatrix} \mathrm{e}^{\imath \frac{\phi}{2}} & 0 \\ 0 & \mathrm{e}^{-\imath \frac{\phi}{2}} \end{pmatrix} \begin{pmatrix}- p^0 - p^3 & - p^1 + \imath p^2 \\ - p^1 - \imath p^2 & - p^0 + p^3 \end{pmatrix} \begin{pmatrix} \mathrm{e}^{- \imath \frac{\phi}{2}} & 0 \\ 0 & \mathrm{e}^{\imath \frac{\phi}{2}} \end{pmatrix} =
\\[1em] & \begin{pmatrix}- (p^\prime)^0 - (p^\prime)^3 & - (p^\prime)^1 + \imath (p^\prime)^2 \\ - (p^\prime)^1 - \imath (p^\prime)^2 & - (p^\prime)^0 + (p^\prime)^3 \end{pmatrix}, \end{aligned}\end{equation*}
where
\begin{equation*} \begin{cases} (p^0)^\prime=p^0, \\[0.7em] (p^3)^\prime = p^3, \end{cases} \quad \text{and} \quad \begin{cases} (p^1)^\prime=p^1 \cos \theta + p^2 \sin \theta, \\[0.7em] (p^2)^\prime = - p^1 \sin \theta + p^2 \cos \theta \end{cases} \implies \begin{pmatrix} (p^1)^\prime \\ (p^2)^\prime \end{pmatrix} = R(-\theta) \begin{pmatrix} p^1 \\ p^2 \end{pmatrix}.  \end{equation*}
In a similar fashion, if we focus on the real exponential transformation induced by the matrix $M_\omega$ introduced above, we have that
\begin{equation*}\begin{aligned} P \longmapsto P^\prime & = \begin{pmatrix} \mathrm{e}^{ \frac{\omega}{2}} & 0 \\ 0 & \mathrm{e}^{- \frac{\omega}{2}} \end{pmatrix} \begin{pmatrix}- p^0 - p^3 & - p^1 + \imath p^2 \\ - p^1 - \imath p^2 & - p^0 + p^3 \end{pmatrix} \begin{pmatrix} \mathrm{e}^{ \frac{\omega}{2}} & 0 \\ 0 & \mathrm{e}^{- \frac{\omega}{2}} \end{pmatrix} =
\\[1em] & \begin{pmatrix}- (p^\prime)^0 - (p^\prime)^3 & - (p^\prime)^1 + \imath (p^\prime)^2 \\ - (p^\prime)^1 - \imath (p^\prime)^2 & - (p^\prime)^0 + (p^\prime)^3 \end{pmatrix}, \end{aligned}\end{equation*}
where
\begin{equation*} \begin{cases} (p^1)^\prime=p^1, \\[0.7em] (p^2)^\prime = p^2, \end{cases} \quad \text{and} \quad \begin{cases} (p^0)^\prime=p^0 \cos \theta - p^3 \sin \theta, \\[0.7em] (p^3)^\prime = p^0 \sin \theta + p^3 \cos \theta \end{cases} \implies \begin{pmatrix} (p^0)^\prime \\ (p^3)^\prime \end{pmatrix} = R(\theta) \begin{pmatrix} p^0 \\ p^3 \end{pmatrix}.  \end{equation*}
In particular, the complex exponentials (w.r.t. the Pauli's matrices) give us the transformations of the form $\Lambda_{\mathrm{rot}}^{\alpha \theta \phi}$, while the real exponentials give us the boost transformations, and therefore we can finally infer that
\begin{equation*}\mathfrak{T}_+ \cong \mathrm{SL}(2, \, \C). \end{equation*}
We are now ready to investigate the spinorial representations introduced above. Consider a Weyl spinor (L) with two components $\psi = \begin{pmatrix} \cdot \\ \cdot \end{pmatrix} \sim ( \frac{1}{2}, \, 0)$ that transforms like
\begin{equation*} \psi \longmapsto M \psi, \end{equation*}
and notice that $\bar{\psi} \sim (0, \, \frac{1}{2})$ and
\begin{equation*} \bar{\psi} \longmapsto M^\ast \bar{\psi}. \end{equation*}
It follows that
\begin{equation*} \psi \longmapsto \psi^\prime \implies (\psi^\prime)_\alpha = M_\alpha^\beta \psi_\beta, \end{equation*}
and, if we set $\psi^\alpha := \epsilon^{\alpha \beta} \psi_\beta$, then one can easily prove that
\begin{equation*} (\psi^\prime)^\alpha = (M^{-1})_\beta^\alpha \psi^\beta. \end{equation*}
In a similar fashion, one could also prove that the conjugate transforms in a similar way, that is,
\begin{equation*}(\bar{\psi}^\prime)_\alpha = (M^\ast)_\alpha^\beta \bar{\psi}_\beta \implies (\bar{\psi}^\prime)^\alpha = \left[ (M^{-1})^\ast \right]_\beta^\alpha \bar{\psi}^\beta. \end{equation*}
Let $\psi$ and $\chi$ be two Weyl $(L)$-spinor. The scalar product is commutative as a consequence of the following, straightforward, computation:
\begin{equation*}\begin{aligned} \psi_\alpha \chi^\alpha & = \psi^\beta \epsilon_{\alpha \beta} \chi_\gamma \epsilon^{\alpha \gamma} =
\\[1em] & = \epsilon_{\alpha \beta} \epsilon^{\alpha \gamma} \psi^\beta \chi_\gamma =
\\[1em] & =  \delta_\beta^\gamma \psi^\beta \chi_\gamma = \psi^\alpha \chi_\alpha. \end{aligned} \end{equation*} 
Notice that, for a two-component spinor $\psi$, it is equivalent to be a Weyl spinor or a Majorana spinor since we can write it in the form
\begin{equation*} \psi_M = \begin{pmatrix} \psi \\ \overline{\psi} \end{pmatrix}, \end{equation*}
which means that one is the antiparticle of the other one. The Dirac spinor
\begin{equation*} \psi_D = \begin{pmatrix} \psi \\ \chi \end{pmatrix}, \end{equation*}
on the other hand, is not equivalent because the two components are independent.

\begin{example}[QCD] Let us consider the two-component spinor
\begin{equation*} \psi = \begin{pmatrix} q_L \\ q_R \end{pmatrix} \sim \underline{3}, \end{equation*}
where $\underline{3}$ is the fundamental representation of $\mathrm{SU}(3, \, \C)$ (color), and
\begin{equation*} \begin{aligned} & \psi_L \sim q_L \: : \: \substack{ \longrightarrow \\ \longleftarrow } \\[1em] & \psi_R \sim q_R \: : \: \substack{ \longrightarrow \\ \longrightarrow } \end{aligned} \qquad \text{and} \qquad \lambda = \frac{p \cdot \mathfrak{J}}{|p|}. \end{equation*}
The fermions "condense" in the empty space and, in particular, we have that
\begin{equation*}\langle \psi \bar{\psi} \rangle = \langle q_R^c q_L \rangle + \langle q_L^c q_R \rangle. \end{equation*} \end{example}

\paragraph{Dirac Spinor.} Recall that the Dirac matrices\index{Dirac matrix}\index{Lorentz group!Dirac representation} $\gamma^\mu$'s are defined by means of the Pauli matrices $\sigma^\mu$'s as follows:
\begin{equation*} \gamma^\mu = \begin{pmatrix} 0 & \sigma^\mu \\ \bar{\sigma}^\mu & 0 \end{pmatrix} \quad \text{and} \quad \gamma^5 = \gamma^0\gamma^1\gamma^2\gamma^3 = \begin{pmatrix} \mathrm{id}_{2 \times 2} & 0 \\ 0 & - \mathrm{id}_{2 \times 2} \end{pmatrix}. \end{equation*}
Let $(m, \, \bar{\psi}_D, \, \psi_D)$ be a Dirac mass triplet. We introduce the Dirac conjugate $\psi_D^+$ as
\begin{equation*} \bar{\psi}_D = \psi_D^+ \gamma_0, \end{equation*}
and we notice that
\begin{equation*} m \bar{\psi}_D \psi_D = \psi_R^c \psi_L + hc. \end{equation*}
The Dirac spinor representation is \textbf{not} an irreducible representation of the Lorentz group $\mathfrak{T}_+$, but it is the direct sum of two irreducible representations. More precisely, we have that
\begin{equation*} \psi_D \sim ( \frac{1}{2}, \, 0 ) \oplus (0, \, \frac{1}{2}), \end{equation*}
while the Weyl spinor representation is obviously irreducible and given by
\begin{equation*} \psi_M \sim ( \frac{1}{2}, \, 0 ). \end{equation*}
The tensor product representation $\psi_M \otimes \psi_M$ is clearly equal to the trivial representation $\underline{0}$ since it contains the singlet only. For example, the kinetic term 
\begin{equation*}\mathscr{L}_c = \bar{\psi} \imath \bar{\gamma}^\mu \partial_\mu \psi + \psi \leftrightarrow \chi, \end{equation*}
where $\bar{\psi} \equiv \psi^+$, is given by a $(1/2, \, 1/2)$ vector. Indeed, one can easily prove that
\begin{equation*} \bar{\psi} \sim (0, \, \frac{1}{2}) \quad \text{and} \quad \bar{\gamma}^\mu \psi \sim ( \frac{1}{2},\, 0) \quad \text{and} \quad \partial_\mu \sim (\frac{1}{2}, \, \frac{1}{2}) \implies \mathscr{L}_c \sim (\frac{1}{2}, \, \frac{1}{2}). \end{equation*}
In the neutrino theory ($m = 0$) the equation of motion is the well-known Weyl equation\index{Weyl equation}, which asserts that
\begin{equation} \label{eq.weyl} \imath \bar{\sigma}^\mu \partial_\mu \psi = 0. \end{equation}
On the other hand, in the Dirac theory ($m_D \neq 0$) the energy is given by
\begin{equation*} \mathscr{L}_c + m_D(\bar{\chi} \psi + \bar{\psi} \chi), \end{equation*}
and therefore the equation of motion is given by
\begin{equation*}\underbrace{\imath \bar{\sigma}^\mu \partial_\mu \psi}_{\text{Left (L) spinor}} + \underbrace{m_D \chi}_{\text{Right (R) spinor}} = 0. \end{equation*}
In any case, if we write explicitly the Weyl equation \eqref{eq.weyl}, we find that
\begin{equation*} (- \imath \partial_0 - \imath \sigma^i \partial_i ) \psi = 0, \end{equation*}
and therefore, if we set $p_i := - \imath \partial_i$ for $i = 1, \, 2, \, 3$, then
\begin{equation*} E :=  \imath \partial_0 \implies E = \vec{p} \cdot \vec{\sigma}.  \end{equation*}
In conclusion, since the Einstein's equation asserts that $E$ is equal to the modulus of $\vec{p}$, i.e. $\sqrt{p^2}$, we infer that
\begin{equation*} \frac{\vec{p} \cdot \vec{\sigma}}{|p|} = 1.  \end{equation*}

\section{Irreducible Representations of $\mathfrak{T}_+$: Part II}

In this finals section, the primary goal is to conclude the discussion of the possible set of generators of the Lorentz group and to find the finite-dimensional representations similarly to what we have already done for the groups introduced earlier.

\subsection{Generators of $\mathfrak{T}_+$}

Recall that the generators are defined by
\begin{equation*} (J_{\lambda \sigma})_\mu^\nu := - \imath \left( \delta_\lambda^\nu g_{\sigma \mu} - \delta_\sigma^\nu g_{\lambda \mu} \right) \quad \text{for $\lambda, \, \sigma = 0, \, \dots, \, 3$}. \end{equation*}
If we consider the Dirac matrices
\begin{equation*} \gamma_\mu = \begin{pmatrix} 0 & \sigma_\mu \\ \bar{\sigma}_\mu & 0 \end{pmatrix} \quad \text{and} \quad \gamma_5 = \gamma_0\gamma_1\gamma_2\gamma_3 , \end{equation*}
then it is easy to see that
\begin{equation*} J_{\mu \nu} = \frac{\imath}{4} [\gamma_\mu, \, \gamma_\nu] = \frac{\imath}{4} \begin{pmatrix} \sigma_\mu \bar{\sigma}_\nu - \sigma_\nu \bar{\sigma}_\mu & 0 \\ 0 &  \bar{\sigma}_\mu \sigma_\nu - \bar{\sigma}_\nu \sigma_\mu \end{pmatrix}, \end{equation*}
and, hence, we refer to $\{\gamma_\mu\}_{\mu = 0, \, \dots, \, 3}$ as \textit{chiral basis}, and to $\gamma_5$ as \textit{chiral operator}\index{chiral basis}\index{chiral operator}. Recall that
\begin{equation*}  \gamma_5= \begin{pmatrix} \mathrm{id}_{2 \times 2} & 0 \\ 0 & - \mathrm{id}_{2 \times 2} \end{pmatrix}, \end{equation*}
and this implies that the chiral operator commutes with each generator $J_{\mu \nu}$, that is,
\begin{equation*} [J_{\mu \nu}, \, \gamma_5] = 0 \implies \text{invariant chirality of the Lorentz group}, \end{equation*}
and it does not depend on the reference system.

\subsection{Finite Irreducible Representations of $\mathfrak{T}_+$}

In the first half of the chapter, we proved that the Lie algebra associated to $\mathfrak{T}_+$ is isomorphic to the Lie algebra $\mathrm{su}(2, \, \C) \times \mathrm{su}(2, \, \C)$. We also introduced the operators
\begin{equation*} \begin{cases} K_m := J_{m, \, 0}  & \text{for $m = 1, \, 2, \, 3$}, \\[0.8em] J_k := \frac{1}{2} \epsilon^{kmn} J_{m, \, n} & \text{for $k = 1, \, 2, \, 3$}, \end{cases} \quad \text{and} \qquad \begin{cases} M_n := \frac{1}{2}(J_n + \imath K_n) & \text{for $n = 1, \, 2, \, 3$}, \\[0.8em] M_n := \frac{1}{2} (J_n - \imath K_n) & \text{for $n = 1, \, 2, \, 3$}, \end{cases} \end{equation*}
in such a way that the following commutator relations hold true:
\begin{equation*}\begin{aligned} & [M_m, \, M_n] = \imath \epsilon_{mn \ell} M_\ell, \\[1em] & [N_m, \, N_n] =- \imath \epsilon_{mn \ell} N_\ell, \\[1em] & [M_m, \, N_n] =0. \end{aligned}\end{equation*}
It follows that the (second) sets of generators form two invariant subalgebra isomorphic to $\mathrm{su}(2, \, \C)$, which means that the Lie algebra of the Lorentz group $\mathfrak{T}_+$ is semisimple, but not simple.

Let $j \in \frac{1}{2}\Z$ be a representative of a finite-dimensional representation of $\mathrm{su}(2, \, \C)$, and let $\mathbb{J}^2 = j(j + 1)$ denote the Casimir operator. Consider the usual basis of eigenstates
\begin{equation*} \{ |\, j, \, m \rangle \}_{m = -j, \, \dots, \, j} \end{equation*}
of the irreducible representation of $\mathrm{su}(2, \, \C)$. Furthermore, recall that the irreducible finite-dimensional representations of the Lorentz group $\mathfrak{T}_+$ are in correspondence with the couples $(j_1, \, j_2)$, where $j_1, \, j_2 \in 1/2 \Z$ denote the irreducible representations of $\mathrm{SU}(2, \, \C)$.

We can define, as we have already done in the previous chapters, the up/down operators as follows (obtaining the six generators of the Lorentz group):
\begin{equation*} \begin{cases} K_\pm := \imath( \mathbbm{1} \otimes N_{\pm} - M_{\pm} \otimes \mathbbm{1}), \\[0.8em] K_3 = \imath(N_3 - M_3), \end{cases} \quad \text{and} \qquad  \begin{cases} J_\pm := M_{\pm} \otimes \mathbbm{1} + \mathbbm{1} \otimes N_{\pm}, \\[0.8em] J_3 = M_3 + N_3. \end{cases} \end{equation*}
Consider now a general representation $(j_1, \, j_2)$, and let us denote by $|\, j_1, \, m_1\rangle |\, j_2, \, m_2\rangle$ the eigenstates of the two irreducible representations of $\mathrm{SU}(2, \, \C)$. Then
\begin{equation*} (J_3)_{(m_1^\prime, \, m_2^\prime); \; (m_1, \, m_2)} = \delta_{m_1^\prime, \, m_1} \delta_{m_2^\prime, \, m_2} (m_1 + m_2), \end{equation*}
and (see \hyperref[sec:oqweofdks]{Section \ref{sec:oqweofdks}})
\begin{equation*} \begin{aligned} (J_+)_{(m_1^\prime, \, m_2^\prime); \; (m_1, \, m_2)} & = \delta_{m_1^\prime, \, m_1+1} \delta_{m_2^\prime, \, m_2} \sqrt{(j_1 - m_1)(j_1 + m_1 + 1)} + \dots
\\[1em] & \dots + \delta_{m_1^\prime, \, m_1} \delta_{m_2^\prime, \, m_2 + 1} \sqrt{(j_2 - m_2)(j_2 + m_2 + 1)}. \end{aligned} \end{equation*}
In a similar fashion, it turns out that
\begin{equation*} (K_3)_{(m_1^\prime, \, m_2^\prime); \; (m_1, \, m_2)} = \imath \delta_{m_1^\prime, \, m_1} \delta_{m_2^\prime, \, m_2} (m_2 - m_1), \end{equation*}
and
\begin{equation*} \begin{aligned} (K_+)_{(m_1^\prime, \, m_2^\prime); \; (m_1, \, m_2)} & = \imath \Big[ \delta_{m_1^\prime, \, m_1} \delta_{m_2^\prime, \, m_2 + 1} \sqrt{(j_2 - m_2)(j_2 + m_2 + 1)}  - \dots
\\[1em] & \dots - \delta_{m_1^\prime, \, m_1+1} \delta_{m_2^\prime, \, m_2} \sqrt{(j_1 - m_1)(j_1 + m_1 + 1)} \Big]. \end{aligned} \end{equation*}

\begin{remark}The representation $(j_1, \, j_2)$ is finite-dimensional, but it is not unitary. In fact, the Lorentz group $\mathfrak{T}_+$ is not compact, and hence unitary representations need to be infinite-dimensional. \end{remark}

\section{Chirality}

In this final section, we briefly discuss the physic part concerning the Lorentz group, and we show that the theory developed so far is strongly correlated with the Higgs mechanism.

\subsection{Standard Model}

The Gauge group in the \textit{standard model}\index{standard model} is given by
\begin{equation*} \mathfrak{G} := \mathrm{SU}_{QED}(3, \, \C) \times \mathrm{SU}_L(2, \, \C) \times \mathrm{U}_Y(1, \, \C). \end{equation*}
The second term $\mathrm{SU}_L(2, \, \C)$ is profoundly related to Higgs mechanism\index{Higgs mechanism}. Indeed, the group acts on the bosons $W^\mu$, and the couple $(W^3, \, Y)$ is transformed in the couple $(Z, \, Y)$, where $Z$ is a different boson. The Higgs mechanism breaks the symmetry of $\mathrm{U}_{em}(1, \, \C)$ since the boson $Z$ acquires a whole lot of mass. It follows that the radius is small, and the potential has an exponential growth, that is,
\begin{equation*} V_Y = g \frac{ \mathrm{e}^{- m \tau}}{\tau} \qquad \text{"\textit{Yukawa Potential}"}, \end{equation*}
and it is thus a non-Coulombian type of potential. We also notice that
\begin{equation*} \mathrm{SU}_L(2, \, \C) \implies \text{the symmetry is broken (left only).} \end{equation*}
There are different possible quarks, and, more precisely, we have
\begin{equation*} \begin{aligned} & \begin{pmatrix} u \\ d \end{pmatrix}_L, \quad \begin{pmatrix} c \\ s \end{pmatrix}_L, \quad \begin{pmatrix} t \\ b \end{pmatrix}_L \sim \text{$\underline{3} \times ( \underline{2}, \, \frac{1}{3} ) \times \begin{pmatrix} \frac{2}{3} \\[0.8em] - \frac{1}{3} \end{pmatrix}$ in $\mathfrak{G}$},
\\[1em] & \begin{pmatrix} u \\ d \end{pmatrix}_R \sim \text{$\underline{3} \times ( \begin{pmatrix} \underline{1} \\ \underline{1} \end{pmatrix}, \, \begin{pmatrix} \frac{4}{3} \\[0.8em] - \frac{2}{3} \end{pmatrix} ) \times \begin{pmatrix} \frac{2}{3} \\[0.8em] - \frac{1}{3} \end{pmatrix}$ in $\mathfrak{G}$}, \end{aligned} \end{equation*}
and, clearly, there are some differences between left and right. It also turns out that
\begin{equation*} \begin{aligned} & \begin{pmatrix} \nu_e \\ e^- \end{pmatrix}_L, \quad \begin{pmatrix} \nu_\mu \\ \mu^- \end{pmatrix}_L, \quad \begin{pmatrix} \nu_\tau \\ \tau^- \end{pmatrix}_L \sim \text{$\underline{3} \times \underline{2} \times (-1) \implies Q_{em} = -1$},
\\[1em] & e_R^-, \quad \mu_R^-, \quad\tau_R^- \sim \text{$\underline{1} \times \underline{0} \times (-2) \implies Q_{em} = -1$},
\\[1em] & (\nu_e)_R, \quad (\nu_\mu)_R, \quad (\nu_\tau)_R \sim \text{$\underline{1} \times \underline{1} \times 0 \implies Q_{em} = 0$}. \end{aligned} \end{equation*}
Notice that we thought that neutrino did not have any mass, but later (neutrino oscillations) we found that the mass is different from zero.

The fermions' mass mixes up the left (L) and right (R), and it is not separable. The mass of an elementary particle given by condensation is called \textit{Yukawa mass}, and we have that
\begin{equation*} \mathscr{L}_Y = g_Y (\bar{\psi}_L)(\phi) \psi_R + \dots \end{equation*}
Notice that the masses of the elementary particles are not all given as a result of the Higgs mechanism; for example, we do not know it yet for the neutrino. Even if the (R) neutrino $\nu_R$ does not exist, the (L) neutrino does, and it turns out that
\begin{equation*}m_L ( \psi^\prime \psi^2 - \psi^2 \psi^\prime) = 2 m_L \psi^\prime \psi^2 \neq 0, \end{equation*}
which means that the mass $m_L$ of the (L) neutrino should be different from zero. A similar argument works for the (R) neutrino $\nu_R$, and the Higgs mechanics (may..) gives us the second addendum of
\begin{equation*}m_R\nu_R \nu _R + m_D \bar{\nu}_R \nu_L + hc, \end{equation*}
while the first comes out from the fact that $\nu_R$ is unbiased with respect to everything else.
\chapter{Poincaré Group} \thispagestyle{empty}

The Poincaré group, usually denoted\footnote{We shall explain the meaning of this particular notation soon. For the time being, one could think of it as a simple notation and nothing more.} by $\R^{1, \, 3} \rtimes \mathrm{SO}^+(1, \, 3)$, is the group of all the isometries of the Minkowski space-time. More precisely, the transformations are all of the form
\begin{equation} \label{eq.21.1} p^\mu \longmapsto \Lambda_\nu^\mu\,  p^\nu + b^\mu, \end{equation}
where $\Lambda$ denotes a matrix in $\mathrm{SO}^+(1, \, 3) = \mathfrak{T}_+$ - the proper orthochronous Lorentz group -, and $b^\mu \in \R^4$ is a space-time translation vector.

\section{Introduction}

We first want to give meaning to the notation $\R^{1, \, 3} \rtimes \mathrm{SO}^+(1, \, 3)$, and explain why it coincides with the Poincarè group. To achieve this, we first need to introduce the notion of \textit{semidirect product}\index{group!semidirect product} between two groups.

\begin{definition}[Semidirect Product] Let $\G$ and $\G^\prime$ be two groups, and let $\varphi : \G^\prime \longrightarrow \mathrm{Aut}(\G)$ be a group homomorphism. The (outer) \textit{semidirect product} of $\G$ and $\G^\prime$ with respect to $\varphi$ is a new group, denoted by $\G \rtimes_\varphi \G^\prime$, is defined as follows.
\begin{enumerate}[label=\textbf{(\alph*)}]
\item The underlying set is the Cartesian product $\G \times \G^\prime$.
\item The group operation $\cdot$ is defined by means of $\varphi$. Namely, we set
\begin{equation*} (g, \, g^\prime) \cdot (h, \, h^\prime) := (g \varphi_{g^\prime}(h), \, g^\prime h^\prime),\end{equation*}
where $\varphi_{g^\prime} := \varphi(g^\prime) \in \mathrm{Aut}(\G)$. \end{enumerate} \end{definition}

\begin{exercise}Let $\G$ and $\G^\prime$ be two groups, and let $\varphi : \G^\prime \longrightarrow \mathrm{Aut}(\G)$ be a group homomorphism. Find, explicitly, the identity and the inverse of the group $\G \rtimes_\varphi \G^\prime$. \end{exercise}

\begin{remark} The direct product (introduced in \hyperref[sec.first]{Section \ref{sec.first}}) is nothing but a particular case of semidirect product, which is achieved when $\varphi$ sends each element of $\G^\prime$ to the identity $\mathrm{id}_{\G} \in \mathrm{Aut}(\G)$. \end{remark}

\begin{theorem}[Semidirect Decomposition] Let $\G$ be a group, and let $\Ha, \, \mathcal{K} \subgroup \G$. Suppose that the following properties hold true: \mbox{}
\begin{enumerate}[label=\textbf{(\roman*)}]
\item The subgroup $\Ha$ is normal, that is, $\Ha \trianglelefteq \G$.
\item The intersection between $\Ha$ and $\mathcal{K}$ is trivial, that is, $\Ha \cap \mathcal{K} = \varnothing$.
\item Every element $g \in \G$ can be written as the product of an element $h \in \Ha$ and an element $k \in \mathcal{K}$, that is, $\G = \Ha \mathcal{K}$.
\end{enumerate}
Then $\G$ is isomorphic to the semidirect product $\Ha \rtimes_\varphi \mathcal{K}$, where $\varphi(k) \in \mathrm{Aut}(\Ha)$ is the conjugation map, which is defined by
\begin{equation*} \varphi(k) := \varphi_k : \Ha \longrightarrow \Ha, \qquad h \longmapsto k h k^{-1}. \end{equation*}\end{theorem}

The transformations of the form \eqref{eq.21.1} are usually denoted by $g(\mathbf{b}, \, \Lambda)$, in such a way that the elements of the form $g(\mathbf{0}, \, \Lambda) = \Lambda$ generate the Lorentz group $\mathfrak{T}_+$, while the elements of the form $g(\mathbf{b}, \, 0) = T(\mathbf{b})$ generate the translation group $\mathscr{T} := \R^{1, \, 3}$.

We shall prove this later, but the abelian group of translation $\mathscr{T}$ is normal, while the proper orthochronous Lorentz group $\mathrm{SO}^+(1, \, 3)$ is a subgroup acting on the Minkowski vector space $\mathscr{T} := \R^{1, \, 3}$ as follows:
\begin{equation*} g(\mathbf{a}, \, \Lambda)\mathbf{v} = \mathbf{a} + \Lambda \mathbf{v}. \end{equation*}
It follows that the Poincarè group is isomorphic to the semidirect product $\R^{1, \, 3} \rtimes \mathrm{SO}^+(1, \, 3)$ with respect to the group homomorphism $\varphi : \mathrm{SO}^+(1, \, 3) \longrightarrow \mathrm{Aut}(\R^{1, \, 3})$ defined by setting
\begin{equation*} \varphi(\Lambda)(\mathbf{v}) :=  \Lambda \mathbf{v}. \end{equation*}

\section{Irreducible Representations of $\mathrm{P}(1, \, 3)$}

In this section, we investigate the irreducible representations of the Poincaré group $\R^{1, \, 3} \rtimes \mathrm{SO}^+(1, \, 3)$, and we take a closer look to its Lie algebra, and the connection with the Lorentz Lie algebra.

\subsection{Generators}

Let $J_{\mu \nu}$ denote the generators of the Lie algebra associated to $\mathfrak{T}_+$, and let $P_\mu$ be the generators of the translations, that is, we require that
\begin{equation} \label{eq.21.2} T(\mathbf{b}) = \mathrm{e}^{- \imath b^\mu P_\mu}, \end{equation}
where $T(\mathbf{b})$ denotes the space-time translation given by the vector $\mathbf{b}$. Notice that $P_\mu = - \imath \partial_{x^\mu}$, and therefore we may equivalently denote the generators of the translation subgroup as follows:
\begin{equation*} P_\mu = - \imath \frac{\partial}{\partial x^\mu} \leadsto \begin{cases} H^0 = \imath \frac{\partial}{\partial t}, \\[0.8em] P^i = - \imath \frac{\partial}{\partial x^i}.\end{cases} \end{equation*}
Furthermore, the set of generators $\{P_\mu\}_{\mu = 0, \, \dots, \, 3}$ generates an \underline{normal and abelian} subalgebra $\mathfrak{P}$ of $\mathrm{p}(1, \, 3)$, and this implies that the Lie algebra of the Poincaré group is not semisimple. In particular, we have the following commutator identites:
\begin{equation*}\begin{aligned} & [P_\mu, \, P_\nu] = 0,
\\[1em] & [P_\mu, \, J_{\lambda \sigma}] = \imath (P_\lambda g_{\mu \sigma} - P_\sigma g_{\mu \lambda}),
\\[1em] & [J_{\mu \nu}, \, J_{\sigma \lambda}]_{\gamma}^\alpha = \imath \left[ (J_{\mu \nu})_\lambda^\alpha g_{\sigma \gamma} - (J_{\mu \nu})_\sigma^\alpha g_{\lambda \gamma} - (J_{\sigma \lambda})_\nu^\alpha g_{\mu \gamma} + (J_{\sigma \lambda})_\mu^\alpha g_{\nu \gamma} \right].  \end{aligned} \end{equation*}
The Poincarè transformation $g(\mathbf{b}, \, \Lambda)$ is also equal to the composition between a translation and a Lorentz transformation, that is, $T(\mathbf{b}) \Lambda$, as a consequence of formula \eqref{eq.21.1}. It follows that
\begin{equation} \label{eq.21.3} \Lambda T( \mathbf{b} )\Lambda^{-1} = T(\Lambda \mathbf{b}), \end{equation}
and this implies that the translation subgroup $\mathscr{T}$ (which is associated to the subalgebra $\mathfrak{P}$) is also a normal abelian subgroup of $\R^{1, \, 3} \rtimes \mathrm{SO}^+(1, \, 3)$, as we claimed in the previous section. 

The proof of \eqref{eq.21.3} follows immediately from the multiplicative rule given by the isomorphism with the semidirect product, i.e.,
\begin{equation} \label{eq.21.4} g(\mathbf{b}, \, \Lambda) g(\mathbf{c}, \, \Gamma) = g( \Lambda \mathbf{c} + \mathbf{b}, \, \Lambda \Gamma). \end{equation}

Now, recall that in the previous chapter we proved that the generators $J_{\mu \nu}$ of the Lorentz group $\mathfrak{T}_+$ can be replaced by the following ones
\begin{equation*} K_m := J_{m, \, 0} \quad \text{and} \quad J_k := \frac{1}{2} \epsilon^{kmn} J_{m, \, n}, \end{equation*}
for $m, \, k \in \{1, \, 2, \, 3\}$. Therefore, the set of generators $\{ P^\mu, \, J_{\mu \nu} \}$ may be replaced by the equivalent set of generators
\begin{equation*} \{ H^0, \, P^i, \, J_m, \, K_m \}_{i, \, m = 1, \, 2, \, 3}. \end{equation*}
The Lie algebra generated by this new set of generators is characterized by the following commutator identities, whose proof is left to the reader:
\begin{equation*}\begin{aligned} & [P^i, \, J_m] = 0,
\\[1em] & [P^m, \, J_n] = \imath \epsilon^{mnk} P_k,
\\[1em] & [P^m, \, K_n] = \imath \epsilon^{mn} P^0,
\\[1em] &  [P^0, \, K_n] = \imath P_n,
\\[1em] & [J_m, \, J_n] = \imath \epsilon^{mnk} J_k,
\\[1em] & [K_m, \, J_n] = \imath \epsilon^{mnk} K_k,
\\[1em] & [K_m, \, K_n] = - \imath \epsilon^{mnk} J_k.\end{aligned} \end{equation*}
Note that the third and the fourth ones are \textbf{nontrivial}, as they contain the boost component of the Lorentz group $\mathfrak{T}_+$.

{\centering \subsection*{The Casimir Operators}}

\noindent The Casimir invariants of this Lie algebra are
\begin{equation*} c_1 := P^\mu P_\mu = (H^0)^2 - (P^i)^2, \end{equation*}
and
\begin{equation*} c_2 = W^\mu W_\mu = (W^0)^2 - (W^i)^2, \end{equation*}
where
\begin{equation*} W^\mu := \frac{1}{2} \epsilon^{\mu \nu \lambda k} K_{\nu \lambda} P_k \end{equation*}
is the so-called \textit{Pauli–Lubanski pseudovector}\index{Pauli-Wbanski pseudovector}, which is used to describe the spin states of moving particles. In any case, the operator $W^\mu$ is orthogonal to $P_\nu$ and they also commute, that is,
\begin{equation*} W^\mu P_\mu = 0 \quad \text{and} \quad [W^\mu, \, P_\nu] = 0. \end{equation*}
Similarly, the reader can easily prove that
\begin{equation*} [W^\lambda, \, J_{\mu \nu}] = \imath ( W^\mu g^{\lambda \nu} - W^\nu g^{\lambda \mu} ) \quad \text{and} \quad [W^\lambda, \, W^\nu] = \imath \epsilon^{\lambda \nu \rho \sigma} W_\rho P_\sigma. \end{equation*}

Recall that, if the system is given by a unique particle, then $c_1$ gives us the mass of that particle (otherwise, it is the square of the total momentum.) Both $c_1$ and $c_2$ induce irreducible representations of the Poincaré group $\R^{1, \, 3} \rtimes \mathrm{SO}^+(1, \, 3)$, and they commute with the generators, that is,
\begin{equation*}\begin{aligned} & [c_1, \, J_{\mu \nu}] = 0,
\\[1em] & [c_1, \, P^\mu] = 0, \end{aligned} \qquad \text{and} \qquad
\begin{aligned} & [c_2, \, J_{\mu \nu}] = 0,
\\[1em] & [c_2, \, P^\mu] = 0.\end{aligned} \end{equation*}
We shall investigate more in-depth the main differences between the following possibilities for the Casimir operator value: $c_1 > 0$, $c_1 = 0$ or $c_1 < 0$.  If, for example, we have $p^\mu = (M, \, \mathbf{0})$, then $c_1 = M^2 > 0$, and therefore the Pauli-Wbanski operator is given by
\begin{equation*} W^i := - M J_i \in \mathrm{so}(3, \, \C), \end{equation*}
which means that it is the generator (along the $i$th axis) of the angular momentum. 

\section{Little Group}
\label{little group}

In this final section, our goal is to introduce the so-called \textit{little group} associated to the Poincarè group $\R^{1, \, 3} \rtimes \mathrm{SO}^+(1, \, 3)$ and the space-time vector $\mathbf{b} \in \R^{1, \, 3}$.

\begin{definition}Let $\G$ be a group, let $X$ be a set, and let $\varphi : \G \times X \longrightarrow X$ be a group action. The \textit{little group} of $x \in X$, denoted by $\G(x)$, is the set of all the elements of $g \in \G$ such that
\begin{equation*}\varphi(g, \, x) = x. \end{equation*}  \end{definition}

\begin{theorem}Let $P^\mu \equiv p^\mu$ be the impulse $4$-vector. Then, the operators $W^\mu$ generate the stability group (=little group) $G(p^\mu)$ that stabilizes $p^\mu$. \end{theorem}

\begin{example}In $\mathrm{SO}(3, \, \R)$, the little group associated to $\begin{pmatrix} 0 & 0 & 1 \end{pmatrix}^T$ is isomorphic to the group $\mathrm{SO}(2, \, \R)$, acting on the first two components only. \end{example}

\begin{theorem}Let $P^\mu \equiv p^\mu$ be the impulse $4$-vector. The irreducible representations of the Poincaré group $\R^{1, \, 3} \rtimes \mathrm{SO}^+(1, \, 3)$ are in correspondence with the irreducible representations of $G(p^\mu)$ via the action of the Lorentz group $\mathfrak{T}_+$. \end{theorem}

\begin{remark} \mbox{}
\begin{enumerate}[label=\textbf{(\alph*)}]
\item If $p^\mu = \mathbf{0}$, then the little group is $G(\mathbf{0}) = \mathfrak{T}_+$.
\item If $c_1 := p^\mu p_\mu > 0$, then we can always assume\footnote{It suffices to consider the frame of reference where the particle is not moving or, if there are more than a single particle, the center of mass.} that the $4$-impulse is given by $p^\mu = (M, \, \mathbf{0})$ for some positive constant $M > 0$. In this case, we have
\begin{equation*} W^\mu = \frac{1}{2} \epsilon^{\mu \nu \lambda \sigma} J_{\nu \lambda} P_\sigma, \end{equation*}
and a straightforward computation shows that
\begin{equation*} W^0 = 0 \qquad \text{and} \qquad \text{$W^i = M J_i$} \quad \text{for $i = 1, \, 2, \, 3$}. \end{equation*}
It follows that the little group $G(p^\mu)$ is given by $\mathrm{SO}(3, \, \R)$, whose generators are the rotations $J_1, \, J_2$ and $J_3$.

The irreducible representations of $\mathrm{SO}(3, \, \R)$ correspond to the irreducible representations of $\mathrm{SU}(2, \, \C)$, and therefore we consider the eigenstates basis $\{ | \, j, \, m \rangle\}$. If we denote by $J_z$ the simultaneously diagonalizable generator, then it is easy to check that
\begin{equation*}\begin{aligned} & P^\mu \, | \, 0, \, m \rangle = p^\mu \, | \, 0, \, m \rangle,
\\[1em] & \mathbb{J}^2 \, | \, 0, \, m \rangle = j(j+1) \, | \, 0, \, m \rangle,
\\[1em] & J_z \, | \, 0, \, m \rangle = m \, | \, 0, \, m \rangle, \end{aligned} \end{equation*}
where $\mathbb{J}^2$ is the Casimir operator. Now set $| \, p, \, m \rangle \equiv H(p) \, | \, 0, \, m \rangle$, and notice that
\begin{equation*}| \, p, \, m \rangle \equiv H(p) \, | \, 0, \, m \rangle = R(\alpha, \, \beta, \, 0) L_z(\xi) \, | \, 0, \, m \rangle,  \end{equation*}
where $R(\alpha, \, \beta, \, 0)$ is the space-rotation w.r.t. the $z$-axis, and $L_z(\xi)$ is the Lorentz boost in the $\hat{z}$ direction, i.e.,
\begin{equation*}L_z(\xi) = \begin{pmatrix} \cosh(\xi) & 0 & 0 & \sinh(\xi) \\ 0 & 1 & 0 &0  \\ 0 & 0 & 1 & 0 \\ \sinh(\xi) & 0 & 0 & \cosh(\xi) \end{pmatrix} \end{equation*}
We now claim that $\{ | \, p, \, m \rangle\}$ is a basis for the irreducible representation of the Poincaré group induced by the one of $\mathrm{SO}(3, \, \R)$. Indeed, it is easy to check that
\begin{equation*}\begin{aligned} & T(\mathbf{b}) \, | \, p, \, m \rangle = \mathrm{e}^{-\imath b_\mu p^\mu} \, | \, p, \, m \rangle,
\\[1em] & \Lambda \, | \, p, \, m \rangle = | \, p^\prime, \, m^\prime \rangle = D_{m, \, m^\prime}^j( R(\Lambda, \, p) ), \end{aligned} \end{equation*}
where $D_{m, \, m^\prime}^j$ is the rotation matrix in $\mathrm{SU}(2, \, \C)$. More precisely, we have
\begin{equation*} D_{m,\, m^\prime}^j(R^z(\theta)) = \mathrm{e}^{\imath \frac{\theta}{2} \tau_3} = \begin{pmatrix} \mathrm{e}^{\imath \frac{\theta}{2}} & 0 \\ 0 & \mathrm{e}^{- \imath \frac{\theta}{2}} \end{pmatrix}, \end{equation*}
and
\begin{equation*} D_{m,\, m^\prime}^j(R^\lambda(\theta)) = \mathrm{e}^{\imath \frac{\theta}{2} \tau_2} = \begin{pmatrix}\cos \frac{\theta}{2} & \sin \frac{\theta}{2} \\ \\ - \sin \frac{\theta}{2} & \cos \frac{\theta}{2} \end{pmatrix}. \end{equation*}
The reader may prove, as an exercise, that in general we have
\begin{equation*} R(\Lambda, \, p) = H^{-1}(p^\prime) \Lambda H(p). \end{equation*}
\item If $c_1 := p^\mu p_\mu = 0$, then we can always assume that the $4$-impulse is given by $p^\mu = (\omega_0, \, 0, \, 0, \, \omega_0)$ for a constant $\omega_0$. In this case, we have
\begin{equation*} \begin{aligned} & W^0 = - W^3 = \omega_0 J_{12} = \omega_0 J_3,
\\[1em] & W^1 = \omega_0 (J_{23} + J_{20}) = \omega_0(- J_1 + K_2),
\\[1em] & W^2 = \omega_0 (J_{31} - J_{10}) = \omega_0(- J_2 - K_1).\end{aligned} \end{equation*}
It follows that $c_2 := W^\mu W_\mu = - (W_1)^2 - (W_2)^2$, and we also have that the associated Lie algebra is
\begin{equation*} \begin{aligned} & [W^1, \, W^2] = 0,
\\[1em] & [W_2, \, J_3] = \imath W^1,
\\[1em] & [W^1, \, J_3] = - \imath W^2, \end{aligned} \end{equation*}
and clearly it is isomorphic to the Lie algebra of the Euclidean group $E_2$.

More precisely, the generators $W^1$ and $W^2$ act like translations (i.e., like $P^1$ and $P^2$ in the Euclidean group) in the $xy$ plane, while $J_3$ generates the rotations.

Let $| \, p, \, \lambda \rangle$ be a basis of eigenstates for which $P^\mu$ and $J_3$ are simultaneously diagonalizable, and let $\lambda$ represent the eigenvalues of $J_3$, that is,
\begin{equation*} \lambda = 0, \, \pm \frac{1}{2}, \, \dots \end{equation*}
By definition, we have that $|\, p, \, \lambda \rangle$ is an eigenstate for both $P^\mu$ and $J_3$, which means that
\begin{equation*}\begin{aligned} & P^\mu \, | \, p, \, \lambda \rangle = P_1^\mu \, | \, p, \, \lambda \rangle,
\\[1em] & J_3 \, | \, p, \, \lambda \rangle = \lambda \, | \, p, \, \lambda \rangle, \end{aligned} \end{equation*}
and therefore
\begin{equation*}| \, p, \, \lambda \rangle \equiv H(p) \, | \, p, \, \lambda \rangle = R(\theta, \, \varphi, \, 0) L_z(\xi) \, | \, p, \, \lambda \rangle = R(\theta, \, \varphi, \, 0) \, | \, p\hat{z}, \, \lambda \rangle.  \end{equation*}
Denote by $p_1$ the $4$-vector $(\omega_0, \, 0, \, 0, \, \omega_0)$. Then the identity above yields to
\begin{equation*}| \, p, \, \lambda \rangle \equiv H(p) \, | \, p_1, \, \lambda \rangle. \end{equation*}
Furthermore, the reader may check that
\begin{equation*}\begin{aligned} & T(\mathbf{b}) \, | \, p, \, \lambda \rangle = \mathrm{e}^{- \imath b_\mu p^\mu} \, | \, p, \, \lambda \rangle,
\\[1em] & \Lambda \, | \, p, \, \lambda \rangle = \mathrm{e}^{-\imath \lambda R(\Lambda, \, p)} \, | \, \Lambda p, \, \lambda \rangle = \langle p, \, \lambda \, | \, H^{-1}(\Lambda p) \Lambda H(p) \, | \, p_1, \, \lambda \rangle, \end{aligned} \end{equation*}
from which it follows that the helicity\index{helicity} - invariant under Lorentz transformations if the mass of the particle is zero - is given by
\begin{equation*} \lambda \sim \frac{\mathbf{p} \cdot \mathbf{s}}{|\mathbf{p} \cdot \mathbf{s}|}, \end{equation*} 
where $\mathbf{s}$ denotes the spin vector and $\mathbf{p}$ is the $4$-impulse mentioned above.
\end{enumerate}
\end{remark}
\part{Applications to Quantum Mechanics}
\chapter{Roots and Weights} \thispagestyle{empty}

In this chapter, we introduce the last theoretical tools of this course: \textit{weights and roots}. We shall see that a semisimple Lie algebra $\g$ is entirely characterized by its simple roots, and we will also show that Dynkin diagrams, for example, are a valuable tool when it comes to representing an algebra in terms of its roots. 

\section{Introduction}

In the whole chapter, we shall always denote by $\g$ a semisimple Lie algebra endowed with a Cartan basis\index{Cartan basis}
\begin{equation*} \left\{H_i, \, E_\alpha \: : \: i = 1, \, \dots, \, m, \, \alpha \in \Delta \right\}, \end{equation*}
\caution[b][darkgreen][Note]{The parameters $\alpha \in \Delta$ are, actually, vectors with $m$ components. These are called roots\index{roots}.} where $\Delta$ is the system of all non-zero roots\index{roots} of $\g$ with respect to $\h$, the Cartan subalgebra generated by the diagonal generators $H_i$, and $m$ is the so-called rank\index{algebra!rank} of $\g$.

We proved during the course that the Cartan subalgebra $\h$ is invariant, and therefore the commutator between $H_i$ and $H_j$ is zero, that is,
\begin{equation*} [H_i, \, H_j] = 0 \quad \text{for all $i, \, j \in \{1, \, \dots, \, m\}$}. \end{equation*}
Furthermore, we require that the following commutator identities hold true for suitable choices of coefficients $N_{\alpha \beta}$:
\begin{equation} \label{eq.f.1} \begin{aligned} & [H_i, \, E_\alpha] = \alpha_i E_\alpha,
\\[1em] & [E_\alpha,\, E_\beta] = N_{\alpha \beta} E_{\alpha + \beta},
\\[1em] & [E_\alpha, \, E_{- \alpha}] = \sum_{i = 1}^m \alpha_i H_i. \end{aligned} \end{equation}
The number of roots depends both on $\g$ and on $\h$. The Jacobi identity \eqref{jacobi} proves that
\begin{equation*}[ H_i, \, [E_\alpha, \, E_\beta]] = \beta_i [E_\alpha, \, E_\beta] + \alpha_i [E_\alpha, \, E_\beta] = (\alpha_i + \beta_i) [E_\alpha, \, E_\beta], \end{equation*}
and therefore the Lie algebra can be easily characterized by means of the sums $\alpha + \beta$, provided that $\alpha + \beta$ belongs to $\Delta$ for all $\alpha$ and $\beta$ root vectors.

\begin{remark} The relations \eqref{eq.f.1} are well-defined. Indeed, to prove that $\alpha \in \Delta$ root vector implies $- \alpha$ root vector, it suffices to notice that
\begin{equation*} H_i = H_i^\dag \quad \text{and} \quad \left([H_i, \, E_\alpha]\right)^\dag = \left(\alpha_i E_\alpha \right)^\dag \implies E_{- \alpha} = E_\alpha^\dag. \end{equation*}  \end{remark}

\begin{example}[$\mathrm{SU}(2, \, \C)$] Recall that the Lie algebra $\mathrm{su}(2, \, \C)$ is semisimple, and its Cartan basis is given by
\begin{equation*} \left\{H_1 := J_3, \, E_\alpha := J_+, \, E_{- \alpha} := J_- \right\}, \end{equation*}
where $\alpha$ is a single-valued vector. It is easy to check that it satisfies the relations \eqref{eq.f.1}.\end{example}

In the general case, we can always reduce to $\mathrm{su}(2, \, \C)$ by introducing a new generator basis. Fix a root vector $\alpha \in \Delta$, and set \caution{Here $\alpha^2$ is defined as $\alpha \cdot \alpha$, scalar product between two $m$-vectors.}
\begin{equation*} J_\alpha^1 := \frac{1}{\sqrt{2 \alpha^2}} (E_\alpha + E_{- \alpha}), \qquad J_\alpha^2 := - \frac{\imath}{\sqrt{2 \alpha^2}} (E_\alpha - E_{- \alpha}), \qquad J_\alpha^3 := \alpha^\ast \cdot \mathbf{H}, \end{equation*}
where $\mathbf{H}$ is the vector $(H_1, \, \dots, \, H_m)$ and\footnote{The product $\cdot$ is the usual scalar product, which means that $\alpha \cdot \beta := \sum_{i = 1}^m \alpha_i \beta_i$.} $\alpha^\ast = \frac{\alpha}{\alpha \cdot \alpha}$ is the so-called \textit{dual root}\index{dual root} of $\alpha$.

Surprisingly, for $\alpha \in \Delta$ fixed but arbitrary, these three operators generate a Lie algebra isomorphic to $\mathrm{su}(2, \, \C)$. Indeed, a straightforward computation shows that
\begin{equation*} \begin{aligned} & [J_\alpha^1, \, J_\alpha^2] = \frac{\imath}{\alpha^2} [E_\alpha, \, E_{- \alpha}] = \imath \,\frac{\alpha \cdot \mathbf{H}}{\alpha^2} = \imath J_\alpha^3,
\\[1em] & [J_\alpha^1, \, J_\alpha^3] = - \imath J_\alpha^2 \quad \text{and} \quad [J_\alpha^2, \, J_\alpha^3] = \imath J_{\alpha}^1, \end{aligned} \end{equation*}
where $\alpha \cdot \mathbf{H} := \alpha_1 H_1 + \dots + \alpha_m H_m$.

\begin{definition}[Weight Vector] \index{weight vector}\index{weight} Let $H_1, \, \dots, \, H_m$ be the generators of the Cartan subalgebra $\h$. A \textit{weight} is a vector $\mu$ such that
\begin{equation*} H_i \, | \, \mu \rangle = \Lambda_i \, | \, \mu \rangle \quad \text{for all $i \in \{1, \, \dots, \, m\}$}, \end{equation*}
that is, a simultaneous eigenstate for all the $H_i$'s. We shall also denote it by
\begin{equation*}\mu = \left( \Lambda_1, \, \dots, \, \Lambda_m \right) \end{equation*}
to emphasize the fact that $\mu$ is an eigenstate of the operator $\mathbf{H}$. \end{definition}

\begin{theorem} \label{thm.f.1} Let $\alpha, \, \beta \in \Delta$ be roots. \mbox{}
\begin{enumerate}[label=\textbf{(\alph*)}]
\item The ratio of the scalar products $\alpha \cdot \beta$ and $\alpha^2$ is either integer or semi-integer, that is,
\begin{equation*} 2\, \frac{\alpha \cdot \beta}{\alpha \cdot \alpha} \in \Z. \end{equation*}
\item The Weyl reflection\index{Weyl reflection}, given by
\begin{equation*} \beta - \frac{2 \alpha( \alpha \cdot \beta)}{\alpha \cdot \alpha}, \end{equation*}
is also a root vector.
\item Let $\mu$ be a weight. Then
\begin{equation*} 2 \, \frac{\mu \cdot \alpha}{\alpha^2} \in \Z. \end{equation*}
\end{enumerate} \end{theorem}

\begin{proof} \mbox{}
\begin{enumerate}[label=\textbf{(\alph*)}]
\item The argument here is similar to the one we use to prove $\mathbf{(c)}$. The reader may try to fill in the details as a simple exercise.
\item This assertion will be proved later.
\item Let $\mu$ be a weight vector. From the definitions, it follows that $\mu$ is an eigenstate for $J_\alpha^3$, and therefore we must have
\begin{equation*} J_\alpha^3 \, | \, \mu \rangle = \frac{\alpha \cdot \mu}{\alpha^2} \, | \, \mu \rangle \end{equation*}
The eigenvalues of $J_\alpha^3$ are semi-integer numbers (see, for example, \hyperref[sdaosdwkd]{Subsection \ref{sdaosdwkd}}), which means that
\begin{equation*} 2 \, \frac{\mu \cdot \alpha}{\alpha^2} \in \Z. \end{equation*}
\end{enumerate} \end{proof}

The argument used in the point $\mathbf{(c)}$ of the previous Theorem tells us much more than what we needed to conclude. More precisely, we have
\begin{equation*} J_\alpha^3 \left( J_\alpha^1 \, | \, \mu , \, \mathfrak{R}\rangle \right) = J_\alpha^1\, | \, \mu , \, \mathfrak{R}\rangle +  J_\alpha^1 J_\alpha^3 \, | \, \mu , \, \mathfrak{R}\rangle = \left( \frac{\alpha \cdot \mu}{\alpha^2} + 1 \right) J_\alpha^3 \, | \, \mu , \, \mathfrak{R}\rangle, \end{equation*}
where $\mathfrak{R}$ is a representation, and
\begin{equation*} j_3 := \left( \frac{\alpha \cdot \mu}{\alpha^2} + 1 \right)  \in \{- j, \, \dots, \, j\}. \end{equation*}
Therefore, the exists $p \in \Z$ such that
\begin{equation*}j_3(p) := \left( \frac{\alpha \cdot \mu}{\alpha^2} + p \right) = j, \end{equation*}
and, using a similar argument with $J_\alpha^2$, we also infer that there must be $q \in \Z$ such that
\begin{equation*}j_3(q) := \left( \frac{\alpha \cdot \mu}{\alpha^2} - q \right) = -j. \end{equation*}
In particular, from the proof of the point $\mathbf{(c)}$ we also infer that there are $p, \, q \in \Z$, satisfying the properties above, and such that
\begin{equation} \label{eq.f.2} 2 \, \frac{\alpha \cdot \mu}{\alpha^2}+ p - q = 0.\end{equation}

\section{Weight and Roots in $\mathrm{SU}(3, \, \C)$}

Recall that the Lie algebra $\mathrm{su}(3, \, \C)$ is semisimple, and it is generated by eight matrices: the up-down operators $T_\pm$, $U_\pm$, $V_\pm$, and the diagonal generators $\lambda^3$ and $\lambda^8$.

We have proved already that these generators define a Cartan basis, and clearly the Cartan subalgebra is given by
\begin{equation*} \h := \mathrm{Span} \langle \lambda^3, \, \lambda^8 \rangle, \end{equation*}
while the system of all non-zero roots of $\mathrm{su}(3, \, \C)$ with respect to $\h$ is given by
\begin{equation*}\begin{aligned} & E_{\alpha_1} = T_+ \quad \text{and} \quad E_{- \alpha_1} = T_-,
\\[1em] &  E_{\alpha_2} = U_+ \quad \text{and} \quad E_{- \alpha_e} = U_-,
\\[1em] & E_{\alpha_3} = V_+ \quad \text{and} \quad E_{- \alpha_3} = V_-. \end{aligned} \end{equation*}
In \hyperref[chwk]{Chapter \ref{chwk}} we studied entirely the Lie algebra $\mathrm{su}(3, \, \C)$ and, in particular, the commutators between the generators. We have
\begin{equation*}\begin{aligned} & [\lambda^3, \, T_\pm] = \pm T_\pm \quad \text{and} \quad [\lambda^8, \, T_\pm] = 0,
\\[1em] & [\lambda^3, \, U_\pm] = \mp \frac{1}{2} U_\pm \quad \text{and} \quad [\lambda^8, \, U_\pm] = \pm \frac{\sqrt{3}}{2} U_\pm,
\\[1em] & [\lambda^3, \, V_\pm] = \pm \frac{1}{2} V_\pm \quad \text{and} \quad [\lambda^8, \, V_\pm] = \pm \frac{\sqrt{3}}{2} V_\pm, \end{aligned} \end{equation*}
which means that the root vectors are given by
\begin{equation*} \alpha_1 = (1, \, 0), \qquad \alpha_2 = \left( - \frac{1}{2}, \, \frac{\sqrt{3}}{2}\right), \qquad \alpha_3 = \left(\frac{1}{2}, \, \frac{\sqrt{3}}{2} \right). \end{equation*}
In the Cartan basis, we have that
\begin{equation*} \underline{3} \quad : \quad |\, q_1 \rangle = \begin{pmatrix} 1 \\ 0 \\ 0 \end{pmatrix} \sim z_1, \qquad | \, q_2 \rangle = \begin{pmatrix} 0 \\ 1 \\ 0 \end{pmatrix} \sim z_2, \qquad |\, q_3 \rangle = \begin{pmatrix} 0 \\ 0 \\ 1 \end{pmatrix} \sim z_3. \end{equation*}
We can equivalently consider $|\, q_1 \rangle$, $|\, q_2 \rangle$ and $|\, q_3 \rangle$ either as quantum states or complex vectors\footnote{This has nothing to do with Quantum Mechanics but, rather, one can chose the best way to consider them, depending on the context.}. In a similar fashion, one can check that the weight vectors are\caution{The states $|\, q_1 \rangle$, $|\, q_2 \rangle$ and $|\, q_3 \rangle$ are orthogonal, but the associated points $\alpha_i$ and $\mu_i$ are not orthogonal vectors. These are totally unrelated concepts!}
\begin{equation*}\underline{3} \quad : \quad \mu_1 = \left(\frac{1}{2}, \, \frac{1}{2 \sqrt{3}} \right), \qquad \mu_2 = \left(- \frac{1}{2}, \, \frac{1}{2 \sqrt{3}} \right), \qquad \mu_3 = \left(0, \, -\frac{1}{\sqrt{3}} \right), \end{equation*}
where $| \, q_i \rangle \leadsto \mu_i$. The adjoint representation $\underline{3}^\ast$ can be easily found using the theory developed in this chapter since
\begin{equation*}\underline{3}^\ast \quad : \quad - \mu_1 = \left(-\frac{1}{2}, \, -\frac{1}{2 \sqrt{3}} \right), \qquad -\mu_2 = \left(\frac{1}{2}, \, -\frac{1}{2 \sqrt{3}} \right), \qquad -\mu_3 = \left(0, \, \frac{1}{\sqrt{3}} \right). \end{equation*}
Furthermore, we can easily compute the weight vectors associated to the representation $\underline{8}$, using the well-known fact that
\begin{equation*} \underline{3} \otimes \underline{3}^\ast = \underline{8} \oplus \underline{1}. \end{equation*}
In fact, the representation $\underline{8}$ is given by the couples $|  q_i \rangle \, | q_j \rangle^\ast$ of quantum states, for $i \neq j \in \{1, \, 2, \, 3\}$, and therefore the weight vectors are
\begin{equation*}\mu_{i, \, j} := \mu_i - \mu_j \quad \text{for all $i \neq j \in \{1, \, 2, \, 3\}$},\end{equation*}
so that the corresponding figure in the $(\lambda^3, \, Y)$-plane is an hexagon. Note that this gives us a total of six points, and thus we need two more to complete the octet. We consider the combinations
\begin{equation*} \frac{1}{\sqrt{2}} \left( |  q_1 \rangle \, | q_1 \rangle^\ast - |  q_2 \rangle \, | q_2 \rangle^\ast\right) \quad \text{and} \quad \frac{1}{\sqrt{2}} \left( |  q_1 \rangle \, | q_1 \rangle^\ast + |  q_2 \rangle \, | q_2 \rangle^\ast - 2|  q_3 \rangle \, | q_3 \rangle^\ast \right), \end{equation*}
both of which are associated with the weight vector $\mathbf{0}$. Note that these are nothing but $q_{21}$ and $q_{12}$ in the figure below.

\begin{figure}[!htbp]
        \centering
        \scalebox{.8}{
        \mbox{%
\begin{minipage}{.40\textwidth}
\begin{tikzpicture}
\draw[help lines, color=gray!30, dashed] (-2.3,-2.3) grid (2.3,2.3);
\draw[->,ultra thin] (-2.4,0)--(2.4,0) node[right]{$\lambda^3$};
\draw[->,ultra thin] (0,-2.4)--(0,2.4) node[above]{$Y$};
  [anchor=mid west,
  mark size=+2pt, mark color=red,  ball color=green]
  \foreach \plm[count=\cnt] in {ball}
    \draw[mark options={fill=red}]
      plot[mark=\plm] coordinates {(1, 2/3) (-1, 2/3) (0, -4/3) (1, 2/3)};
	\node[] at (-1, 2/3+0.3) {$q_2$}; \node[] at (1, 2/3 + 0.3) {$q_1$}; \node[] at (0.3, -4/3) {$q_3$}; 
\end{tikzpicture}
          \end{minipage}%
          \qquad
          \begin{minipage}{.40\textwidth}
      \begin{tikzpicture}
\draw[help lines, color=gray!30, dashed] (-2.3,-2.3) grid (2.3,2.3);
\draw[->,ultra thin] (-2.4,0)--(2.4,0) node[right]{$\lambda^3$};
\draw[->,ultra thin] (0,-2.4)--(0,2.4) node[above]{$Y$};
  [anchor=mid west,
  mark size=+2pt, mark color=red,  ball color=green]
  \foreach \plm[count=\cnt] in {ball}
    \draw[mark options={fill=red}]
      plot[mark=\plm] coordinates {(1, -2/3) (-1, -2/3) (0, 4/3) (1, -2/3)};
	\node[] at (-1, -2/3 - 0.3) {$q_2$}; \node[] at (1, -2/3 - 0.3) {$q_1$}; \node[] at (0.3, 4/3) {$q_3$}; 
\end{tikzpicture}
          \end{minipage}
    }}
    \caption{\textbf{Left.} Fundamental Representation $\underline{3}$. \textbf{Right.} Complex conjugate representation $\underline{3}^\ast$.}
\end{figure}

\begin{figure}[!htbp]
        \centering
        \scalebox{.8}{
        \mbox{%
\begin{minipage}{.45\textwidth}
\begin{tikzpicture}
\draw[->,ultra thin] (-2.5,0)--(2.5,0) node[right]{$\lambda^3$};
\draw[->,ultra thin] (0,-4.1)--(0,4.1) node[above]{$Y$};
  [anchor=mid west,
  mark size=+2pt, mark color=red,  ball color=green]
  \foreach \plm[count=\cnt] in {ball}
    \draw[mark options={fill=red}]
      plot[mark=\plm] coordinates {(-2, 0) (-1, 8/3) (1, 8/3) (2, 0) (1, -8/3) (-1, -8/3) (-2, 0)};
       [anchor=mid west,
  mark size=+2pt, mark color=red,  ball color=green]
  \foreach \plm[count=\cnt] in {ball}
    \draw[mark options={fill=red}]
      plot[mark=\plm] coordinates {(0, 0)};
	\node[] at (-2.3, 0.3) {$q_{22}$}; \node[] at (2.3, 0.3) {$q_{11}$}; \node[] at (-1.05, 8/3 + 0.3) {$q_{23}$};  \node[] at (1.05, 8/3 + 0.3) {$q_{13}$}; \node[] at (-1.05, -8/3-0.3) {$q_{32}$};  \node[] at (1.05, -8/3-0.3) {$q_{31}$}; \node[] at (-0.3, 0.2) {$q_{21}$}; \node[] at (0.3, 0.2) {$q_{12}$};
\end{tikzpicture}
          \end{minipage}%
          \qquad
          \begin{minipage}{.45\textwidth}
      \begin{tikzpicture}
\draw[->,ultra thin] (-2.1,0)--(2.1,0) node[right]{$\lambda^3$};
\draw[->,ultra thin] (0,-2.1)--(0,2.1) node[above]{$Y$};
  [anchor=mid west,
  mark size=+2pt, mark color=red,  ball color=green]
  \foreach \plm[count=\cnt] in {ball}
    \draw[mark options={fill=red}]
      plot[mark=\plm] coordinates {(0, 0)};
	\node[] at (0.3, 0.3) {$q_{33}$};
\end{tikzpicture}
          \end{minipage}
    }}
    \caption{The decomposition of the tensor product representation $\underline{3} \otimes \underline{3}^\ast$. }
\end{figure}

\newpage

The six quantum states $|  q_i \rangle \, | q_j \rangle^\ast$ and the two quantum states defined above are all orthonormal vectors, and thus they form an orthonormal frame of $\C^8$.

The weight vectors associated to the last two of them are the zero vector $\mathbf{0}$, and these are usually referred to as \textit{degenerate states}\index{degenerate states}.

\begin{theorem} Let $\mathfrak{R}$ be a representation of $\mathrm{SU}(3, \, \C)$. Then
\begin{equation} \label{eq.f.4} E_\alpha \, | \, \mu, \, \mathfrak{R} \rangle \propto | \, \mu + \alpha, \, \mathfrak{R} \rangle. \end{equation} \end{theorem}

\begin{remark} We use the symbol $\propto$ (=proportional to) because there is no guarantee that the left-hand side of \eqref{eq.f.4} is nonzero. \end{remark}

\begin{proof}It follows from \eqref{eq.f.1} that
\begin{equation*} H_i \left( E_\alpha \, | \, \mu, \, \mathfrak{R} \rangle \right) = [H_i, \, E_\alpha] \, | \, \mu, \, \mathfrak{R} \rangle  + E_\alpha \left( H_i \, | \, \mu, \, \mathfrak{R} \rangle  \right) = (\alpha_i + \mu_i) \cdot \left( E_\alpha \, | \, \mu, \, \mathfrak{R} \rangle \right).\end{equation*}
In particular, $E_\alpha \, | \, \mu, \, \mathfrak{R} \rangle$ is an eigenstate of $H_i$ for all $i = 1, \, \dots, \, m$, and therefore it must be proportional to $| \, \mu + \alpha, \, \mathfrak{R} \rangle$. \end{proof} 

\begin{remark} Fix $\alpha \in \Delta$, and consider the generators $J_\alpha^1$, $J_\alpha^2$ and $J_\alpha^3$ of the $\mathrm{su}(2, \, \C)$ Lie algebra. If $j$ is the maximum eigenvalue of the diagonal operator $J_\alpha^3$, then
\begin{equation*} J_\alpha^1 \, | \, j \rangle = 0, \end{equation*}
and this implies that \eqref{eq.f.4} is a proportionality relation only. \end{remark}

We now need to use the theory of irreducible representations of $\mathrm{SU}(2, \, \C)$, developed in \hyperref[su2ch]{Chapter \ref{su2ch}}, to prove that the Weyl reflection\index{Weyl reflection} of two root vectors is also a root vector.

Let $j$ denote the maximum eigenvalue of $J_\alpha^3$, and let $m \in \{- j, \, \dots, \, j\}$ be the set of all eigenvalues. Recall that, if $\mu$ is a weight vector, then
\begin{equation*} J_\alpha^3 \, | \, \mu \rangle = \frac{\mu \cdot \alpha}{\alpha^2} \, | \, \mu \rangle. \end{equation*}
Note that the eigenvalues of $J_\alpha^3$ are either integer or semi-integer (depending on the value of $j$), and therefore the same goes for the scalar product, that is,
\begin{equation*} \text{$\frac{\mu \cdot \alpha}{\alpha^2}$ is either semi-integer or integer}.  \end{equation*}
The operators $E_{\pm \alpha}$ can be easily recovered using the definitions of $J_\alpha^1$, $J_\alpha^2$ and, in particular, we have
\begin{equation*}E_\alpha = \left( \frac{\sqrt{2}}{\alpha} \right)^{-1} (J_\alpha^1 + \imath J_\alpha^2) \quad \text{and} \quad E_{-\alpha} = \left( \frac{\sqrt{2}}{\alpha} \right)^{-1} (J_\alpha^1 - \imath J_\alpha^2). \end{equation*}
We are now ready to prove the point $\mathbf{(b)}$ of \hyperref[thm.f.1]{Theorem \ref{thm.f.1}}, i.e., the Weyl reflection of two root vectors is a root vector. First, notice that
\begin{equation*} J_\alpha^3 \, | \, \beta \rangle = \frac{ \alpha \cdot \beta}{\alpha^2} \, | \, \beta \rangle = \frac{\alpha \cdot \mathbf{H}}{\alpha^2} \, | \, \beta \rangle, \end{equation*}
and, as a consequence of formula \eqref{eq.f.4}, we also have that
\begin{equation*} E_{- \alpha} \, | \, \beta \rangle \propto \alpha - \beta. \end{equation*}
The commutator identities between the $\mathrm{SU}(2, \, \C)$ generators yield to the following chain of equalities:
\begin{equation*} \begin{aligned} J_\alpha^3 \left( E_{- \alpha} \, | \, \beta \rangle \right) & = \frac{\alpha \cdot \mathbf{H}}{\alpha^2} \left( E_{- \alpha} \, | \, \beta \rangle \right) =
\\[1em] & = \frac{\alpha_i}{\alpha^2} H_i \left( E_{- \alpha} \, | \, \beta \rangle \right) =
\\[1em] & = \frac{\alpha_i}{\alpha^2} \big\{ [H_i, \, E_{-\alpha}]\, | \, \beta \rangle + E_{- \alpha}  \left( H_i \, | \, \beta \rangle \right) \big\}=
\\[1em] & = - E_{- \alpha} \, | \, \beta \rangle + \frac{\alpha \cdot \beta}{\alpha^2} E_{- \alpha} \, | \, \beta \rangle =
\\[1em] & = \left( \frac{\alpha \cdot \beta}{\alpha^2} - 1 \right) \cdot E_{- \alpha} \, | \, \beta \rangle.
\end{aligned} \end{equation*}
Suppose that $\frac{\alpha \cdot \beta}{\alpha^2} > 0$. Then, as a consequence of the point $\mathbf{(a)}$, the operator $(E_{-\alpha})^{ \frac{2 \alpha \cdot \beta}{\alpha^2}}$ is well-defined because the exponent is an integer number, and therefore the computation above shows that
\begin{equation*} (E_{-\alpha})^{ \frac{2 \alpha \cdot \beta}{\alpha^2}} \, | \, \beta \rangle \propto | \, \beta - \frac{2 \alpha \cdot \beta}{\alpha^2} \alpha \rangle. \end{equation*}
In particular, the vector $| \, \beta - \frac{2 \alpha \cdot \beta}{\alpha^2} \alpha \rangle$ is an eigenstate of $J_\alpha^3$ with associated eigenvalue $- \frac{\alpha \cdot \beta}{\alpha^2}$, and this is exactly what we wanted to prove for the point $\mathbf{(b)}$.

Let $| \, \mu,\, \mathfrak{R} \rangle$ be a generic weight vector in the representation $\mathfrak{R}$. We can always write it as a linear combination of representations $j$ of the group $\mathrm{SU}(2, \, \C)$. Consequently, there must be an integer $p \in \Z_{\geq 0}$ such that
\begin{equation*} (J_\alpha^1)^p \, | \, \mu,\, \mathfrak{R} \rangle \neq 0 \quad \text{and} \quad (J_\alpha^1)^{p + 1} \, | \, \mu,\, \mathfrak{R} \rangle = 0.\end{equation*}
Using the usual commutator identity, we also find that \caution{Here $(J_\alpha^1)^p$ denotes the $p$th power of the operator $J_\alpha^1$.}
\begin{equation*} J_\alpha^3 \left\{ (J_\alpha^1)^p \, | \, \mu,\, \mathfrak{R} \rangle \right\} = \left( \frac{\alpha \cdot \mu}{\alpha^2} + p \right) \cdot (J_\alpha^1)^p \, | \, \mu,\, \mathfrak{R} \rangle,\end{equation*}
and therefore
\begin{equation} \label{eq.21.1}  \left( \frac{\alpha \cdot \mu}{\alpha^2} + p \right) = j.\end{equation}
Similarly, there exists $q \in \Z_{\geq 0}$ such that
\begin{equation*} (J_\alpha^2)^q \, | \, \mu,\, \mathfrak{R} \rangle \neq 0 \quad \text{and} \quad (J_\alpha^2)^{q + 1} \, | \, \mu,\, \mathfrak{R} \rangle = 0.\end{equation*}
Using the usual commutator identity, we also find that
\begin{equation*} J_\alpha^3 \left\{ (J_\alpha^2)^q \, | \, \mu,\, \mathfrak{R} \rangle \right\} = \left( \frac{\alpha \cdot \mu}{\alpha^2} - q \right) \cdot (J_\alpha^2)^q \, | \, \mu,\, \mathfrak{R} \rangle,\end{equation*}
and therefore
\begin{equation} \label{eq.21.2} \left( \frac{\alpha \cdot \mu}{\alpha^2} - q \right) = - j.\end{equation}
If we sum \eqref{eq.21.1} and \eqref{eq.21.2}, we obtain the well-known relation
\begin{equation} \label{eq.21.3} 2 \, \frac{\alpha \cdot \mu}{\alpha^2} + p - q = 0.\end{equation}
Note that $p$ and $q$ characterize the position of the multiplets $\mathrm{SU}(2, \, \C)$ generated by $J_\alpha^1$, $J_\alpha^2$ and $J_\alpha^3$. If we apply \eqref{eq.21.3} for $\beta = \mu$, then
\begin{equation*} \eqref{eq.21.3} \implies \frac{\alpha \cdot \beta}{\alpha^2} = - \frac{1}{2}(p - q).\end{equation*}
Similarly, if we do the same with the $\mathrm{SU}(2, \, \C)$ generated by $J_\beta^1$, $J_\beta^2$ and $J_\beta^3$, then
\begin{equation*} \eqref{eq.21.3} \implies \frac{\alpha \cdot \beta}{\beta^2} = - \frac{1}{2}(p^\prime - q^\prime)\end{equation*}
for some (possibly different) positive integers $p^\prime, \, q^\prime \in \Z$. If we multiply the two identities above, we find that
\begin{equation*}  \frac{(\alpha \cdot \beta)^2}{\alpha^2 \beta^2} = \frac{1}{4}(p - q)(p^\prime - q^\prime) \leq 1,\end{equation*}
which means that there exists $\theta_{\alpha, \, \beta} \in [0, \, 2 \pi)$ such that
\begin{equation*}  \cos^2(\theta_{\alpha, \, \beta}) = \frac{1}{4}(p - q)(p^\prime - q^\prime) \leq 1.\end{equation*}
Clearly, there are only four possibilities for the right-hand side, and therefore there are only four possibilities for the angles between the roots.

\begin{figure}[h!]
\label{table:1}
\begin{flushleft}
\begin{tikzpicture}
\clip node (m) [matrix,matrix of nodes,
fill=black!20,inner sep=0pt,
nodes in empty cells,
nodes={minimum height=1cm,minimum width=1cm,anchor=center,outer sep=0,font=\sffamily},
row 1/.style={nodes={fill=black,text=white}},
column 1/.style={nodes={fill=gray,text=black,align=center,text width=4.5cm,text depth=0.5ex}},
column 2/.style={text width=4.5cm,align=center,every even row/.style={nodes={fill=white}}},
row 1 column 1/.style={nodes={fill=black}},
column 3/.style={text width=4.5cm,align=center,every even row/.style={nodes={fill=white}}},
prefix after command={[rounded corners=4mm] (m.north east) rectangle (m.south west)}
] {
     $(p - q)(p^\prime - q^\prime)$  & $\cos(\theta_{\alpha, \, \beta})$ & $\theta_{\alpha, \, \beta}$  \\
$0$   & $0$ & $\frac{\pi}{2}$  \\
$1$   & $ \pm \frac{1}{2}$ & $\frac{\pi}{3}$  \\
$2$   & $ \pm \frac{1}{\sqrt{2}}$ & $\frac{\pi}{4}$ or $\frac{3\pi}{4}$  \\
$3$   & $ \pm \frac{3}{\sqrt{2}}$ & $\frac{\pi}{6}$ or $\frac{5\pi}{6}$  \\
};
\end{tikzpicture}
\end{flushleft}
\caption{The {\LaTeX} code of this table can be found \href{https://tex.stackexchange.com/questions/67586/how-to-create-comparison-tables-in-latex}{here}.}
\end{figure}

Note that the value $(p - q)(p^\prime - q^\prime) = 4$ is not admissible because the angle $\theta_{\alpha, \, \beta}$ would be equal to $0$ or $\pi$, that is, the two roots are parallel (which is impossible).

\section{Simple Roots}

\begin{definition}[Positive Weight] \index{weight!positive} A weight vector $\mu$ is said to be \textit{positive} if and only if the first nonzero component is positive. \end{definition}

\begin{definition} \index{roots!order} Let $\mu$ and $\nu$ be two weight vectors. Then
\begin{equation*} \mu \geq \nu \iff \mu - \nu := (\mu_1 - \nu_1, \, \dots, \, \mu_m - \nu_m) \geq 0. \end{equation*} \end{definition}

The relation $\geq$ defined here is a \textit{partial order}\index{partial order} in the sense of the following definition.

\begin{definition}[Partial Order Relation] \index{partial order relation} Let $M$ be a set. A \textit{partial order} $\leq$ is a subset of the product $M \times M$ satisfying the following properties:
\begin{enumerate}[label=\textbf{(\roman*)}]
\item \textbf{\scshape Reflexive.} For every $a \in M$ it turns out that $a \leq a$.
\item \textbf{\scshape Antisymmetric.} For every couple $(a, \, b) \in M^{\otimes 2}$ it turns out that $a \leq b$ and $b \leq a$ if and only if $a = b$.
\item \textbf{\scshape Transitive.} For every triple $(a, \, b, \, c) \in M^{\otimes 3}$ satisfying $a \leq b$ and $b \leq c$, it turns out that $a \leq c$.
\end{enumerate}\end{definition}

\begin{definition}[Highest Weight] \index{highest weight} Let $\mathfrak{R}$ be a representation of $\g$. The \textit{highest weight} is defined as the weight vector $\mu$ such that $\mu > \nu$ for all other weight vectors $\nu \neq \mu$. \end{definition}

From now on, we shall denote by $\mu_{\mathrm{highest}}$ the highest weight of a semisimple Lie algebra $\g$ w.r.t. $\h$.

\begin{theorem}A representation $\mathfrak{R}$ of $\g$ is entirely characterized by the weight $\mu_{\mathrm{highest}}$. \end{theorem}

\begin{proof}The interested reader can find a detailed proof of this result in \cite[Chapter 7]{thskda}. \end{proof}

\begin{definition}[Positive Root] \index{roots!positive} A root $\alpha \in \Delta$ is said to be \textit{positive} if and only if the first nonzero component is positive. \end{definition}

\begin{definition}[Simple Root] \index{roots!simple} A root $\alpha \in \Delta$ is said to be a \textit{simple root} if and only if $\alpha$ is positive and cannot be expressed as a sum of positive roots. \end{definition}

\caution[b][darkgreen][Note]{From now on, we shall denote by $\mathfrak{L}$ the subset of $\Delta$ containing all the simple roots of $\g$.}\begin{theorem} Let $| \, \mu \rangle$ be a weight vector. Then
\begin{equation*} E_\alpha \, | \, \mu \rangle = 0 \quad \text{for all $\alpha \in \Delta$} \implies | \, \mu \rangle = | \, \mu_{\mathrm{highest}} \rangle.\end{equation*} \end{theorem}

We now show how to apply the notions introduced in this section to a concrete example. Consider the Lie algebra $\mathrm{su}(3, \, \C)$, and consider the representation $\underline{8}$:

\begin{figure}[h!]
 \centering{
\scalebox{.6}{ 
\begin{tikzpicture}
\draw[->,ultra thin] (-2.5,0)--(2.5,0) node[right]{$\lambda^3$};
\draw[->,ultra thin] (0,-4.1)--(0,4.1) node[above]{$Y$};
  [anchor=mid west,
  mark size=+2pt, mark color=red,  ball color=green]
  \foreach \plm[count=\cnt] in {ball}
    \draw[mark options={fill=red}]
      plot[mark=\plm] coordinates {(-2, 0) (-1, 8/3) (1, 8/3) (2, 0) (1, -8/3) (-1, -8/3) (-2, 0)};
       [anchor=mid west,
  mark size=+2pt, mark color=red,  ball color=green]
  \foreach \plm[count=\cnt] in {ball}
    \draw[mark options={fill=red}]
      plot[mark=\plm] coordinates {(0, 0)};
	\node[] at (-2.3, 0.3) {$q_{22}$}; \node[] at (2.3, 0.3) {$q_{11}$}; \node[] at (-1.05, 8/3 + 0.3) {$q_{23}$};  \node[] at (1.05, 8/3 + 0.3) {$q_{13}$}; \node[] at (-1.05, -8/3-0.3) {$q_{32}$};  \node[] at (1.05, -8/3-0.3) {$q_{31}$}; \node[] at (-0.3, 0.2) {$q_{21}$}; \node[] at (0.3, 0.2) {$q_{12}$};
\end{tikzpicture}
          }
         }
\end{figure}

Then the highest weight vector is given by $q_{11}$ since the first component is not only positive, but also bigger than any other. Note that the strictly positive weight vectors here are $q_{11}$, $q_{13}$ and $q_{31}$, while $q_{21} = q_{12}$ is exactly equal to zero.

\begin{theorem} Let $\alpha, \, \beta \in \Delta$ be simple roots. Then $\alpha - \beta \notin \Delta$. \end{theorem}

\begin{proof} We may assume without loss of generality that $\alpha > \beta$. If $\alpha - \beta$ is nonzero, then it is necessarily a positive root. This yields to a contradiction since
\begin{equation*}\alpha = \beta + (\alpha - \beta) \end{equation*}
is sum of positive roots, and therefore $\alpha$ would not be simple. \end{proof}

{\centering \subsubsection*{Angle Between \underline{Simple} Roots}}

\noindent Let $\alpha, \, \beta \in \Delta$ be two fixed simple roots. Then
\begin{equation*}\begin{aligned} &\frac{\alpha \cdot \beta}{\alpha^2} + p = j \quad \text{and} \quad \frac{\alpha \cdot \beta}{\alpha^2}  = - j,
\\[1em] & \frac{\alpha \cdot \beta}{\beta^2} + p^\prime = j^\prime \quad \text{and} \quad \frac{\alpha \cdot \beta}{\beta^2}  = - j^\prime,
\\[1em] & \frac{\alpha \cdot \beta}{\alpha^2} = - \frac{p}{2} \quad \text{and} \quad \frac{\alpha \cdot \beta}{\beta^2}  = - \frac{p^\prime}{2}. \end{aligned}\end{equation*}
The value of the angle $\theta_{\alpha, \, \beta}$ is thus given by
\begin{equation*} \cos( \theta_{\alpha, \, \beta} ) = - \frac{ \sqrt{p p^\prime} }{2}, \end{equation*}
and consequently
\begin{equation*}  \theta_{\alpha, \, \beta} \in \left[ \frac{\pi}{2}, \, \pi \right) \quad \text{and} \quad \left| \frac{\beta}{\alpha} \right| = \sqrt{ \frac{p}{p^\prime} }. \end{equation*}

\begin{remark} The maximum weight vector is the highest member of the multiplets generated by all the simple roots. More precisely,
\begin{equation*} E_\alpha \, | \, \mu \rangle = 0 \quad \text{for all $\alpha \in \Delta$ simple} \implies | \, \mu \rangle = | \, \mu_{\mathrm{highest}} \rangle.\end{equation*}  \end{remark}

\begin{theorem}Simple roots are linearly independent as vectors. \end{theorem}

\begin{proof} Let us consider a linear combination of simple roots
\begin{equation*} \gamma := \sum_{\alpha \in \mathfrak{L}} x_\alpha \alpha, \end{equation*}
and suppose that $\gamma = 0$. The roots $\alpha \in \mathfrak{L}$ are necessarily positive, and therefore the coefficients $x_\alpha$ cannot possibly be all positive. We write
\begin{equation*} \gamma = \sum_{\alpha \in \mathfrak{L}_1} x_\alpha \alpha - \sum_{\alpha \in \mathfrak{L}_2} y_\alpha \alpha =: \mu - \nu, \end{equation*}
where $\mathfrak{L}_1 \sqcup \mathfrak{L}_2 = \mathfrak{L}$, and the two linear combinations are both positive with $y_\alpha := - x_\alpha$. If we take the square of $\gamma$, we obtain
\begin{equation*} \gamma^2 = \mu^2 + \nu^2 - \sum_{(\alpha, \, \beta) \in \mathfrak{L}_1 \times \mathfrak{L}_2} x_\alpha x_\beta \alpha \beta \geq 0.\end{equation*}
The cosine of the angle between simple roots is always negative (second quarter), and hence $\alpha \cdot \beta \leq 0$. It follows that
\begin{equation*} 0 = \mu^2 + \nu^2 - \sum_{(\alpha, \, \beta) \in \mathfrak{L}_1 \times \mathfrak{L}_2} x_\alpha x_\beta \alpha \beta \geq 0 \iff x_{\alpha} = x_{\beta} = 0,\end{equation*}
which means that simple roots are linearly independent. \end{proof}

\begin{theorem} Let $\alpha \in \Delta$ be a positive root. Then $\alpha$ can always be written as a positive linear combination of simple roots.\end{theorem}

\begin{theorem} The collection $\mathfrak{L}$ of all simple roots is a complete set of vectors. Furthermore, there are exactly $m := \mathrm{rank}(\g)$ simple roots. \end{theorem}

\begin{proof} We argue by contradiction. \caution{The vector $w$ is orthogonal to every root $\alpha \in \Delta$ if it is orthogonal to the simple ones!}Let $w$ be a nonzero vector orthogonal to all $\alpha \in \mathfrak{L}$. Then
\begin{equation*} [ \xi \cdot \mathbf{H}, \, E_\alpha] = \xi_i [H_i, \, E_\alpha] = \xi \alpha_i E_{\alpha_i} = 0, \end{equation*}
which means that $\xi \cdot \mathbf{H}$ is an abelian subalgebra of $\g$ that commutes with all the generators. This is a contradiction with the fact that $\g$ is a semisimple algebra.
\end{proof}

{ \centering \subsubsection*{Simple Roots $\leadsto$ Lie Algebra $\g$} }

\noindent We are finally ready to show what we have anticipated above: the whole Lie algebra $\g$ is characterized by its simple roots, at least for rank-two algebras. First, we notice that for $m  =2$ there are only four possible choices for $p$ and $p^\prime$, that is,
\begin{equation*} \begin{aligned} & p = p^\prime = 0 \implies \text{$\frac{\beta}{\alpha}$  is indeterminate} \leadsto \mathrm{so}(4, \, \R),
\\[1em] & p = p^\prime = 1 \implies \text{$\theta_{\alpha, \, \beta} = \frac{2 \pi}{3}$ and $\frac{\beta}{\alpha} = 1$} \leadsto \mathrm{su}(3, \, \C),
\\[1em] & p = 1, \, p^\prime = 2 \implies \text{$\theta_{\alpha, \, \beta} = \frac{7 \pi}{12}$ and $\frac{\beta}{\alpha} = \frac{1}{\sqrt{2}}$} \leadsto \mathrm{so}(5, \, \R) \sim \mathrm{usp}(4, \, \C),
\\[1em] & p = 2, \, p^\prime = 3 \implies \text{$\theta_{\alpha, \, \beta} = \frac{5 \pi}{6}$ and $\frac{\beta}{\alpha} = \frac{1}{\sqrt{3}}$} \leadsto \g_2. \end{aligned} \end{equation*}

In the next couple of pages, we picture the weight diagram of the groups of rank $2$ mentioned above, and show which ones are the simple roots and why they are enough to characterize the algebra itself.

\newpage

\begin{figure}[h!]
 \centering{   
\scalebox{.8}{ 
\begin{tikzpicture}
\draw[->, thin, red] (0, 0)--(2,0) node[right]{$h$};
\draw[->, thin, blue] (0, 0)--(1,-4/3) node[right]{$p, \, s$};
\draw[->, thin, blue] (0, 0)--(1,4/3) node[right]{$p, \, s$};
\draw[->, thin] (0, 0)--(-2,0);
\draw[->, thin] (0, 0)--(-1,-4/3);
\draw[->, thin] (0, 0)--(-1,4/3);

\draw[dashed, ultra thin] (2, 0)--(1, 4/3);
\draw[dashed, ultra thin] (1, 4/3)--(-1, 4/3);
\draw[dashed, ultra thin] (-1, 4/3)--(-2, 0);
\draw[dashed, ultra thin] (-2, 0)--(-1, -4/3);
\draw[dashed, ultra thin] (-1, -4/3)--(1,- 4/3);
\draw[dashed, ultra thin] (1, -4/3)--(2, 0);
\end{tikzpicture}
          }
         }
         \caption{The Lie algebra $\mathrm{su}(3, \, \C)$.}
\end{figure}

\begin{figure}[h!]
 \centering{   
\scalebox{.8}{ 
\begin{tikzpicture}
\draw[->, thin, blue] (0, 0)--(1,-4/3) node[right]{$p, \, s$};
\draw[->, thin, blue] (0, 0)--(1,4/3) node[right]{$p, \, s$};
\draw[->, thin] (0, 0)--(-1,-4/3);
\draw[->, thin] (0, 0)--(-1,4/3);
\end{tikzpicture}
          }
         }
         \caption{The Lie algebra $\mathrm{so}(4, \, \R) \sim \mathrm{su}(2, \, \C) \times \mathrm{su}(2, \, \C)$.}
\end{figure}

\begin{figure}[h!]
  \centering
        \mbox{%
\begin{minipage}{.30\textwidth}
\scalebox{.8}{ 
\begin{tikzpicture}
\draw[->, thin] (0, 0)--(2,0);
\draw[->, thin] (0, 0)--(0,-2);
\draw[->, thin] (0, 0)--(-2,0);
\draw[->, thin] (0, 0)--(2, 2);
\draw[->, thin] (0, 0)--(-2, 2);
\draw[->, thin] (0, 0)--(-2, -2);
\draw[->, thin, red] (0, 0)--(0,2);
\draw[->, thin, red] (0, 0)--(2, -2);

\draw[dashed, ultra thin] (2, 2)--(-2,2);
\draw[dashed, ultra thin] (-2, 2)--(-2, -2);
\draw[dashed, ultra thin] (-2, -2)--(2,-2);
\draw[dashed, ultra thin] (2, -2)--(2,2);
\end{tikzpicture}
          }
          \end{minipage}%
          \qquad \qquad
         \begin{minipage}{.30\textwidth}
\scalebox{.8}{ 
\begin{tikzpicture}
\draw[->, thin] (0, 0)--(2,0);
\draw[->, thin] (0, 0)--(0,-2);
\draw[->, thin] (0, 0)--(-2,0);
\draw[->, thin] (0, 0)--(1,1);
\draw[->, thin] (0, 0)--(-1, 1);
\draw[->, thin] (0, 0)--(-1, -1);
\draw[->, thin, red] (0, 0)--(0,2);
\draw[->, thin, red] (0, 0)--(1, -1);

\draw[dashed, ultra thin] (2, 0)--(0,2);
\draw[dashed, ultra thin] (0, 2)--(-2, 0);
\draw[dashed, ultra thin] (-2, 0)--(0,-2);
\draw[dashed, ultra thin] (0, -2)--(2,0);
\end{tikzpicture}
          }
          \end{minipage}%

         }
         \caption{The Lie algebras $\mathrm{so}(5, \, \R)$ and $\mathrm{usp}(4, \, \C)$. The color {\color{red}red} denotes the two simple roots. }
\end{figure}

\section{Dynkin Diagram} \index{Dynkin diagram}

The $2$-dimensional Dynkin diagram is a valuable tool to picture the rank-two algebras studied above, without relying on the weight diagram. We introduce the following notation: \mbox{}
\begin{enumerate}[label=\textbf{(\arabic*)}]
\item We denote by $\circ$ the bigger simple root, and by $\bullet$ the smaller simple root.
\item We denote the angle between $\alpha$ and $\beta$ simple roots with a number of segments equal to the number in the first column of \hyperref[table:1]{Figure 12.3}.
\end{enumerate}

The following Dynking Diagrams are an easy consequence of what we have proved so far in this chapter and, especially, in the last section.

\begin{equation*} \begin{aligned} & \mathrm{so}(4, \, \R) \: : \qquad
  \begin{tikzpicture}[scale=.4]
    \draw[thick] (0, 0) circle (2 mm);
     \draw[thick] (3, 0) circle (2 mm);
  \end{tikzpicture}
  \\[1em] & \mathrm{su}(3, \, \C) \: : \qquad
  \begin{tikzpicture}[scale=.4]
    \draw[thick] (0, 0) circle (2 mm);
     \draw[thick] (3, 0) circle (2 mm);
     \draw[thin] (0.2, 0) -- (2.8,0);
  \end{tikzpicture}
    \\[1em] & \mathrm{su}(4, \, \C) \: : \qquad
  \begin{tikzpicture}[scale=.4]
    \draw[thick] (0, 0) circle (2 mm);
     \draw[thick] (3, 0) circle (2 mm);
     \draw[thick] (6, 0) circle (2mm);
     \draw[thin] (0.2, 0) -- (2.8,0);
     \draw[thin] (3.2, 0) -- (5.8,0);
  \end{tikzpicture}
     \\[1em] & \mathrm{su}(n + 1, \, \C) \: : \qquad
  \begin{tikzpicture}[scale=.4]
    \draw[thick] (0, 0) circle (2 mm);
     \draw[thick] (3, 0) circle (2 mm);
     \draw[thin] (0.2, 0) -- (2.8,0);
  \end{tikzpicture} \dots \begin{tikzpicture}[scale=.4]
    \draw[thick] (0, 0) circle (2 mm);
     \draw[thick] (3, 0) circle (2 mm);
     \draw[thin] (0.2, 0) -- (2.8,0);
  \end{tikzpicture} 
\end{aligned}\end{equation*}
We can also use the Dynking diagrams to show that $\mathrm{so}(5, \, \R)$ is isomorphic (as a Lie algebra) to $\mathrm{usp}(4, \, \C)$, and $\mathrm{su}(4, \, \C)$ is isomorphic to $\mathrm{so}(4, \, \R)$. Namely, we have
\begin{center}
  \begin{tikzpicture}[scale=.4]
    \draw (-1,0) node[anchor=east]  {$\mathrm{so}(4, \, \R)$};
  
    \draw[thick] (0 ,0) circle (.3 cm);
    \draw[thick] (30: 17 mm) circle (.3cm);
    \draw[thick] (-30: 17 mm) circle (.3cm);
    \draw[thick] (30: 3 mm) -- (30: 14 mm);
    \draw[thick] (-30: 3 mm) -- (-30: 14 mm);
  \end{tikzpicture}
\end{center}
which is clearly equivalent to $\mathrm{su}(4, \, \C)$. We also have
\begin{center}
  \begin{tikzpicture}[scale=.4]
    \draw (-1,0) node[anchor=east]  {$\mathrm{so}(2n, \, \R)$};
    \foreach \x in {0,...,4}
    \draw[xshift=\x cm,thick] (\x cm,0) circle (.3cm);
    \draw[xshift=8 cm,thick] (30: 17 mm) circle (.3cm);
    \draw[xshift=8 cm,thick] (-30: 17 mm) circle (.3cm);
    \draw[dotted,thick] (0.3 cm,0) -- +(1.4 cm,0);
    \foreach \y in {1.15,...,3.15}
    \draw[xshift=\y cm,thick] (\y cm,0) -- +(1.4 cm,0);
    \draw[xshift=8 cm,thick] (30: 3 mm) -- (30: 14 mm);
    \draw[xshift=8 cm,thick] (-30: 3 mm) -- (-30: 14 mm);
  \end{tikzpicture}
\end{center}
while, for odd natural numbers, we have
\begin{center}
  \begin{tikzpicture}[scale=.4]
    \draw (-1,0) node[anchor=east]  {$\mathrm{so}(2n + 1, \, \R)$};
    \foreach \x in {0,...,4}
    \draw[xshift=\x cm,thick] (\x cm,0) circle (.3cm);
    \draw[xshift=5 cm,thick,fill=white] (5 cm, 0) circle (.3 cm);
    \draw[dotted,thick] (0.3 cm,0) -- +(1.4 cm,0);
    \draw[xshift=5 cm,thick,fill=black] (3 cm, 0) circle (.3 cm);
    \foreach \y in {1.15,...,3.15}
    \draw[xshift=\y cm,thick] (\y cm,0) -- +(1.4 cm,0);
    \draw[thick] (8.3 cm, 0 cm) -- +(1.4 cm,0);
  \end{tikzpicture}
\end{center}
Since
\begin{center}
  \begin{tikzpicture}[scale=.4]
    \draw (-1,0) node[anchor=east]  {$\mathrm{so}(5, \, \R)$};
    \draw[thick] (0 ,0) circle (.3 cm);
    \draw[thick,fill=black] (2 cm,0) circle (.3 cm);
    \draw[thick] (0: 3 mm) -- +(1.4 cm, 0);
  \end{tikzpicture}
\end{center}
and
\begin{center}
 \begin{tikzpicture}[scale=.4]
    \draw (-1,0) node[anchor=east]  {$\mathrm{usp}(4, \, \C)$};
    \draw[thick, fill=black] (0 ,0) circle (.3 cm);
    \draw[thick,fill=white] (2 cm,0) circle (.3 cm);
    \draw[thick] (0: 3 mm) -- +(1.4 cm, 0);
  \end{tikzpicture}
\end{center}
we can easily infer that $\mathrm{so}(5, \, \R)$ is isomorphic (as a Lie algebra) to $\mathrm{usp}(4, \, \C)$.

{ \centering \subsubsection*{Dynking Diagram of Exceptional Groups} }

\begin{center}
  \begin{tikzpicture}[scale=.4]
    \draw (-1,0) node[anchor=east]  {$G_2$};
    \draw[thick] (0 ,0) circle (.3 cm);
    \draw[thick,fill=black] (2 cm,0) circle (.3 cm);
    \draw[thick] (30: 3mm) -- +(1.5 cm, 0);
    \draw[thick] (0: 3 mm) -- +(1.4 cm, 0);
    \draw[thick] (-30: 3 mm) -- +(1.5 cm, 0);
  \end{tikzpicture}
\end{center}

\begin{center}
  \begin{tikzpicture}[scale=.4]
    \draw (-3,0) node[anchor=east]  {$F_4$};
    \draw[thick] (-2 cm ,0) circle (.3 cm);
    \draw[thick] (0 ,0) circle (.3 cm);
    \draw[thick,fill=black] (2 cm,0) circle (.3 cm);
    \draw[thick,fill=black] (4 cm,0) circle (.3 cm);
    \draw[thick] (15: 3mm) -- +(1.5 cm, 0);
    \draw[xshift=-2 cm,thick] (0: 3 mm) -- +(1.4 cm, 0);
    \draw[thick] (-15: 3 mm) -- +(1.5 cm, 0);
    \draw[xshift=2 cm,thick] (0: 3 mm) -- +(1.4 cm, 0);
  \end{tikzpicture}
\end{center}

\begin{center}
  \begin{tikzpicture}[scale=.4]
    \draw (-1,1) node[anchor=east]  {$E_6$};
    \foreach \x in {0,...,4}
    \draw[thick,xshift=\x cm] (\x cm,0) circle (3 mm);
    \foreach \y in {0,...,3}
    \draw[thick,xshift=\y cm] (\y cm,0) ++(.3 cm, 0) -- +(14 mm,0);
    \draw[thick] (4 cm,2 cm) circle (3 mm);
    \draw[thick] (4 cm, 3mm) -- +(0, 1.4 cm);
  \end{tikzpicture}
\end{center}

\begin{center}
  \begin{tikzpicture}[scale=.4]
    \draw (-1,1) node[anchor=east]  {$E_7$};
    \foreach \x in {0,...,5}
    \draw[thick,xshift=\x cm] (\x cm,0) circle (3 mm);
    \foreach \y in {0,...,4}
    \draw[thick,xshift=\y cm] (\y cm,0) ++(.3 cm, 0) -- +(14 mm,0);
    \draw[thick] (4 cm,2 cm) circle (3 mm);
    \draw[thick] (4 cm, 3mm) -- +(0, 1.4 cm);
  \end{tikzpicture}
\end{center}

\begin{center}
  \begin{tikzpicture}[scale=.4]
    \draw (-1,1) node[anchor=east]  {$E_8$};
    \foreach \x in {0,...,6}
    \draw[thick,xshift=\x cm] (\x cm,0) circle (3 mm);
    \foreach \y in {0,...,5}
    \draw[thick,xshift=\y cm] (\y cm,0) ++(.3 cm, 0) -- +(14 mm,0);
    \draw[thick] (4 cm,2 cm) circle (3 mm);
    \draw[thick] (4 cm, 3mm) -- +(0, 1.4 cm);
  \end{tikzpicture}
\end{center}

\section{Cartan Matrices} \index{Cartan matrix}

Let $\mathfrak{L}$ be the set of simple roots of a semisimple Lie algebra $\g$. The \textit{Cartan matrix} is defined \caution{The non-diagonal elements of the Cartan matrix represent the angles between the simple roots of $\g$.}by setting
\begin{equation*} (A^\g)_{i, \, j} := 2 \frac{\alpha_i \alpha_j}{\alpha_i^2} \quad \text{for $i, \, j \in \{1, \, \dots, \, m \}$}, \end{equation*}
where $m$ is the rank of $\g$. 

\begin{example} In the case of $\mathrm{su}(3, \, \C)$, the matrix is given by
\begin{equation*} A = \begin{pmatrix} 2 & - 1\\-1 & 2 \end{pmatrix}. \end{equation*}
 \end{example}
 
 \begin{example} In the case of $\mathrm{su}(4, \, \C)$, the matrix is given by
\begin{equation*} A = \begin{pmatrix} 2 & - 1 & 0 \\-1 & 2 & - 1 \\ 0 & - 1 & 2\end{pmatrix}. \end{equation*}
 \end{example}
 
{\centering \subsubsection*{The construction of the Lie algebra $\mathrm{su}(3, \, \C)$}}

The simple roots of $\mathrm{su}(3, \, \C)$ are
\begin{equation*} \alpha_1 = \left( \frac{1}{2}, \, \frac{\sqrt{3}}{2} \right) \quad \text{and} \quad \alpha_2 = \left( \frac{1}{2}, \, - \frac{\sqrt{3}}{2} \right) \end{equation*}
and we have already proved that $\alpha_1 + \alpha_2$ is also a root. It is easy to check that
\begin{equation*} \frac{ \alpha_1 \cdot \alpha_2 }{\alpha_1^2} = - \frac{1}{2} \implies J_{\alpha_1}^3 \, | \, \alpha_2 \rangle = - \frac{1}{2} \, | \, \alpha_2 \rangle, \end{equation*}
and therefore $- j = 1/2$. We are now ready to show the construction of the Lie algebra:
\begin{equation*}\begin{aligned} [E_{\alpha_1}, \, E_{\alpha_2}] &= (T^{\alpha_1})_{\beta \alpha_2} E_\beta =
\\[1em] & = \langle \beta \, | \, T^{\alpha_1} \, | \, \alpha_2 \rangle E_\beta =
\\[1em] & = \langle \beta \, | \, J_1^{\alpha_1} \, | \, \alpha_2 \rangle E_\beta =
\\[1em] & = \frac{1}{\sqrt{2}} E_{\alpha_1 + \alpha_2}. \end{aligned}  \end{equation*}
Using this relation and the Jacobi identity \eqref{jacobi}, we infer that
\begin{equation*}\begin{aligned} [E_{-\alpha_1}, \, E_{\alpha_1 + \alpha_2}] &= \sqrt{2} \left[E_{-\alpha_1}, \, [E_{\alpha_1}, \, E_{\alpha_2}] \right] =
\\[1em] & = - \sqrt{2} \left[E_{\alpha_1}, \, \underbracket{[E_{\alpha_2}, \, E_{-\alpha_1}]}_{= 0} \right] - \sqrt{2} \left[E_{\alpha_2}, \, [E_{-\alpha_1}, \, E_{\alpha_1}] \right]  =
\\[1em] & = \sqrt{2} [ \alpha_1 \cdot \mathbf{H}, \, E_{\alpha_2} ] =
\\[1em] & = - \sqrt{2} \alpha_1 \cdot \alpha_2 E_{\alpha_2} = \frac{1}{\sqrt{2}} E_{\alpha_2}. \end{aligned}  \end{equation*}
In a similar fashion, one can prove that
\begin{equation*}\begin{aligned} [E_{-\alpha_2}, \, E_{\alpha_1 + \alpha_2}] &= \sqrt{2} \left[E_{-\alpha_2}, \, [E_{\alpha_1}, \, E_{\alpha_2}] \right] =
\\[1em] & = - \sqrt{2} \left[E_{\alpha_1}, \, [E_{\alpha_2}, \, E_{-\alpha_2}] \right] - \sqrt{2} \left[E_{\alpha_2}, \, \underbracket{[E_{-\alpha_2}, \, E_{\alpha_1}]}_{= 0} \right]  =
\\[1em] & = -\sqrt{2} [E_{\alpha_1}, \, \alpha_2 \cdot \mathbf{H} ] =
\\[1em] & = \sqrt{2} \alpha_1 \cdot \alpha_2 E_{\alpha_1} = - \frac{1}{\sqrt{2}} E_{\alpha_1}. \end{aligned}  \end{equation*}
In particular, the coefficients $N_{\alpha \beta}$ of \eqref{eq.f.1} have been entirely determined, and it is not hard to see that this is the Lie algebra $\mathrm{su}(3, \, \C)$.

\subsection{Weyl Reflection Group} \index{Weyl reflection group}

Let $\alpha, \, \beta \in \Delta$ be roots. We have proved in \hyperref[thm.f.1]{Theorem \ref{thm.f.1}} that the Weyl reflection, given by
\begin{equation*} \beta - \frac{2 \alpha( \alpha \cdot \beta)}{\alpha \cdot \alpha}, \end{equation*}
is also a root. The set of Weyl reflections preserves the weights diagram, as one can easily check by computing the action of the operators $J_\alpha^i$ against $\beta$.

\begin{theorem}The trace of any generator of any representation of a compact simple Lie group is zero \end{theorem}

\begin{proof}See \cite{alsd} for a detailed dissertation on the topic. \end{proof}
\chapter{Quantum Physics Applications} \thispagestyle{empty}

In this final chapter, the primary goal is to show how we can apply the theory developed during the whole course to prove specific properties of physical systems in, for example, quantum mechanics.

\section{Harmonic Oscillator $3$D}
\label{sec:apsadk}
The Hamiltonian of a $3$-dimensional harmonic oscillator\index{harmonic oscillator} is given by
\begin{equation} \label{qa.1} H = \frac{\mathbf{p}^2}{2m} + \frac{m \omega^2 \mathbf{r}^2}{2}, \end{equation}
where $\mathbf{p} := - \imath \hbar \nabla$ is the momentum operator, and $\mathbf{r}$ the position vector. Let $\mathbf{L} := \mathbf{r} \times \mathbf{p}$ be the angular moment. Then
\begin{equation*} [\mathbf{L},\, H] = 0 \implies \text{$\mathbf{L}$ is preserved}. \end{equation*}
Let us denote $| \, N \rangle$ by $| \, n_1, \, n_2, \, n_3 \rangle$, where $N = n_1 + n_2 + n_3$, the eigenvectors, and let us consider the respectively (energy) eigenvalues\index{energy eigenvalues}
\begin{equation*} E_N := \omega \hbar \left( N + \frac{3}{2} \right) = \omega \hbar \left( 2k + \ell + \frac{3}{2} \right) . \end{equation*}
Since $k$ is a non-negative integer, we have that
\begin{equation*} \ell = \begin{cases}0, \, 2, \, 4, \, \dots & \text{if $N$ is even}, \\[0.8em] 1, \, 3, \, 5, \, \dots & \text{if $N$ is odd}. \end{cases} \end{equation*}
The magnetic quantum number\index{magnetic quantum number} $m$ is an integer satisfying $-\ell \leq m \leq \ell$, so for every $N$ and $\ell$ there are $2 \ell + 1$ different quantum states, labeled by $m$. Thus, the degeneracy at level $N$ is
\begin{equation*} \sum_{\substack{\ell=0 \\ \text{$\ell$ even}}}^{N} (2\ell + 1) = \sum_{k = 0}^{N/2} (4 k + 1) = \frac{(N + 1)(N + 2)}{2} \end{equation*}
if $N$ is even, and
\begin{equation*} \sum_{\substack{\ell=0 \\[0.1em] \text{$\ell$ odd}}}^{N} (2\ell + 1) = \sum_{k = 0}^{(N-1)/2} (4 k + 3) = \frac{(N + 1)(N + 2)}{2} \end{equation*}
if $N$ is odd.

In order to have a better understanding of why there always is degeneration at the $N$ level, we introduce the \textit{ladder operator}\index{ladder operator} formalism. Following this approach, we define the operators $a$ and its adjoint $a^\dag$,
\begin{equation*}a := \sqrt{ \frac{m \omega}{2 \hbar} } \left( \mathbf{x}+\frac{\imath}{m\omega} \mathbf{p} \right) \quad \text{and} \quad a^\dag = \sqrt{ \frac{m \omega}{2 \hbar} } \left( \mathbf{x}-\frac{\imath}{m\omega} \mathbf{p} \right).\end{equation*}
The Hamiltonian \eqref{qa.1} can be easily rewritten in terms of these new operators as
\begin{equation} \label{qa.2} H = \omega \hbar \left( a_1^\dag a_1 + a_2^\dag a_2 + a_3^\dag a_3 + \frac{3}{2} \right), \end{equation}
where $a_i$ and $a_j^\dag$ denote, respectively, the components of the ladder operators. The Hamiltonian \eqref{qa.2} is invariant both under the action of $\mathrm{SO}(3, \, \R)$ and $\mathrm{SU}(3, \, \C)$ since the latter preserves the complex scalar product. To prove this, we introduce the operators
\begin{equation*} Q^a := a_i^\dag (\lambda^a)_{ij} a_j \quad \text{for $a = 1, \, \dots, \, 8$}, \end{equation*}
where $\lambda^a$ denotes the $a$th generator of $\mathrm{SU}(3, \, \C)$. 

\begin{lemma} The operators $Q^a$ commute with the Hamiltonian given by \eqref{qa.2}. \end{lemma}

\begin{proof} A straightforward computation shows that
\begin{equation*} [H, \, Q^a] = [a_i^\dag a_j, \, a_k^\dag a_k] = - a_i^\dag a_j + a_i^\dag a_j = 0.\end{equation*} \end{proof}

In particular, the Hamiltonian \eqref{qa.2} is invariant under the action of $\mathrm{SU}(3, \, \C)$. Furthermore, if we denote by $\psi_N$ the eigenstate relative to the energy $E_N$, that is,
\begin{equation*} H \, | \, \psi_N \rangle = E_N \, | \, \psi_N \rangle, \end{equation*}
then $Q^a \psi_N \neq 0$ yields to degeneration\footnote{The dimension of the eigenspace is strictly bigger than one.}. Indeed, using the fact that $H$ and $Q^a$ commutes,
\begin{equation*} H\left( Q^a \, | \, \psi_N \rangle \right) = Q^a \left( H \, | \, \psi_N \rangle \right) = E_N Q^a \, | \, \psi_N \rangle,\end{equation*}
which means that $Q^a \, | \, \psi_N \rangle$ is also an eigenstate associated to the same eigenvalue. To find the degeneration order, it suffices to compute the number of $ | \, \psi_N \rangle$, and therefore it is given by
\begin{equation*} \sum_{n_3 = 0}^N \sum_{n_2 = 0}^{N - n_3} 1 = \sum_{n_3 = 0}^N (N - n_3 + 1) = \frac{(N + 1)(N+2)}{2}, \end{equation*}
which is the same result that we were able to show above by different means. Also, note that
\begin{equation*} | \, N \rangle = (a_1^\dag)^{n_1}(a_2^\dag)^{n_2}(a_3^\dag)^{n_3} \, |  \,0 \rangle. \end{equation*}

\section{Hydrogen Atom}
\index{hydrogen atom}

The Hamiltonian of the hydrogen atom is given by
\begin{equation} \label{qa.3} H = \frac{\mathbf{p}^2}{2m} + \frac{e^2}{\mathbf{r}^2}, \end{equation}
where $\mathbf{p} := - \imath \hbar \nabla$ is the momentum operator and $e$ the electronic charge. Let $\mathbf{L} := \mathbf{r} \times \mathbf{p}$ be the angular moment. Then
\begin{equation*} [\mathbf{L},\, H] = 0 \implies \text{$\mathbf{L}$ is preserved}. \end{equation*}
Also, the energy eigenvalue associated to the eigenvectors $\psi_{n \ell m}$ is given by
\begin{equation*} E_n := - \frac{e^2}{2 r_B n^2} \quad \text{for $n = 1, \, 2, \, \dots$}\end{equation*}
where $r_B$ is the Bohr radius\index{Bohr radius}. In this case $\ell$ can take all the possible values between $0$ and $n - 1$ so that the degeneracy at the level $n$ is given by
\begin{equation*} \sum_{\ell = 0}^{n - 1} (2\ell + 1) = (n - 1)n + n = n^2. \end{equation*}
This degeneration can be explained by the symmetries of the hydrogen atoms. Hence, we introduce the \textit{Lenz vector}\index{Lenz vector}, which is defined by
\begin{equation} \label{qa.4} \mathbf{A} := \frac{e^2}{r} \mathbf{r} - \frac{1}{2m} \left( \mathbf{p}\times \mathbf{L} - \mathbf{L}\times \mathbf{p} \right), \end{equation}
where $r$ denotes the length of the vector $\mathbf{r}$. We now notice that
\begin{equation*} \begin{aligned} & [L_i, \, p_j] = \imath \epsilon_{ijk} \hbar p_k,
\\[1em] & [L_i, \, L_j] = \imath\epsilon_{ijk} \hbar L_k, 
\\[1em] & [L_i, \, x_j] = \imath \epsilon_{ijk} \hbar x_k, \end{aligned} \end{equation*}
where $L_i$, $p_j$ and $x_k$ are respectively the components of the vectors $\mathbf{L}$, $\mathbf{p}$ and $\mathbf{r}$. Recall that, for all functions $F$, we have
\begin{equation*} [p_i, \, F(\mathbf{r})] = - \imath \hbar \frac{\partial}{\partial x_i} F(\mathbf{r}), \end{equation*}
and hence it is not hard to check that
\begin{equation} \label{qa.5} [\mathbf{A}, \, H] = 0. \end{equation}
\caution{From now on, we shall assume that $$\hbar = e = m = 1.$$}In a similar fashion, one can show that the following relations hold:
\begin{equation*} \begin{aligned} & [L_i, \, A_j] = \imath \epsilon_{ijk} A_k,
\\[1em] & [A_i, \, A_j] = - 2 H_i \epsilon_{ijk}  L_k, 
\\[1em] & [L_i, \, L_j] = \imath \epsilon_{ijk} L_k. \end{aligned} \end{equation*}
Fix $n$. Then $-2E := - 2 E_n > 0$, and therefore the operator of components
\begin{equation*} u_i := \frac{A_i}{\sqrt{- 2E}} \end{equation*}
is well-defined. The commutator identities proved above imply that
\begin{equation*} \begin{aligned} & [L_i, \, u_j] = \imath \epsilon_{ijk} u_k,
\\[1em] & [L_i, \, L_j] = \imath \epsilon_{ijk}  L_k, 
\\[1em] & [u_i, \, u_j] = \imath \epsilon_{ijk} L_k. \end{aligned} \end{equation*}
Therefore, if we introduce $\mathbf{j}_1 := \frac{\mathbf{L} + \mathbf{u}}{2}$ and $\mathbf{j}_2 := \frac{\mathbf{L} - \mathbf{u}}{2}$, then the commutator relations above yield to
\begin{equation*} \begin{aligned} & [j_{1, \, i}, \, j_{2, \, j}] = 0,
\\[1em] & [j_{1, \, i}, \, j_{1, \, j}] = \imath \epsilon_{ijk}  j_{1, \, k}, 
\\[1em] & [j_{2, \, i}, \, j_{2, \, j}] = \imath \epsilon_{ijk}  j_{2, \, k},  \end{aligned} \end{equation*}
which means that these are the generators of the $\mathrm{su}(2, \, \C) \times \mathrm{su}(2, \, \C) \sim \mathrm{so}(4, \, \R)$ Lie algebra.

Furthermore, the angular moment $\mathbf{L}$ and the vector $\mathbf{u}$ both commute with the Hamiltonian $H$, but $[ \mathbf{L}, \, \mathbf{u} ] \neq 0$, and this is the reason why there is degeneracy at every level $n$. A simple computation shows (note that $\mathbf{r}$ and $\mathbf{p}$ do not commute) that
\begin{equation*}\mathbf{A}^2 = 2 H \left( \mathbf{L}^2 + 1 \right) + 1, \end{equation*}
and thus
\begin{equation*}\mathbf{A}^2 = (- 2E) \mathbf{u}^2 \implies \mathbf{u}^2 + \mathbf{L}^2 = - 1 - \frac{1}{2E}. \end{equation*}
Since
\begin{equation*}\mathbf{L} \cdot \mathbf{u} = 0 \quad \text{and} \quad \mathbf{L} \cdot \mathbf{A} = L \cdot \left[ \frac{\mathbf{r}}{r} - \frac{1}{2} \left( \mathbf{p}\times \mathbf{L} - \mathbf{L}\times \mathbf{p} \right) \right], \end{equation*}
we easily infer that
\begin{equation*}\begin{aligned} & (\mathbf{j}_1)^2 = \frac{1}{4} (\mathbf{L}^2 + \mathbf{u}^2) = \frac{1}{4} \left(- 1 - \frac{1}{2E} \right) = j(j + 1),
\\[1em] & (\mathbf{j}_2)^2 = \frac{1}{4} (\mathbf{L}^2 + \mathbf{u}^2) = \frac{1}{4} \left(- 1 - \frac{1}{2E} \right) = j(j + 1). \end{aligned}\end{equation*}
In particular, the value of $j$ is given by
\begin{equation*}(2j + 1)^2 = - \frac{1}{2E} > 0, \end{equation*}
and therefore we have
\begin{equation*}E = - \frac{1}{2n^2} = - \frac{e^2}{2 r_B n^2}, \end{equation*}
and the degeneracy number is $(2j + 1)(2j + 1) = n^2$, as expected.

\section{Wigner-Eckart Theorem in $\mathrm{SU}(2, \, \C)$ - $\mathrm{SO}(3, \, \R)$}

The \textit{Wigner–Eckart theorem} is a fundamental result in quantum mechanics, which relies a lot on representation theory.

It states that matrix elements of spherical tensor operators, on the basis of angular momentum eigenstates, can be expressed as the product of two factors, one of which is \underline{independent} of angular momentum orientation, and the other a \textit{Clebsch–Gordan coefficient}.

Namely, let us consider a basis $\{ | \, j, \, m \rangle \}$ of eigenstates that simultaneously diagonalize the Casimir operator $\mathbb{J}^2$ and the $z$-rotation operator $J_z$, in such a way that
\begin{equation*} \mathbb{J}^2 \, | \, j, \, m \rangle = j(j+1) \, | \, j, \, m \rangle \quad \text{and} \quad J_z \, | \, j, \, m \rangle = m \, | \, j, \, m \rangle. \end{equation*}
We are mainly interested in transformations of the form
\begin{equation*} | \, j, \, m \rangle \longmapsto U(\omega) \, | \, j, \, m \rangle := \sum_{m^\prime} D_{m^\prime, \, m}^j(\omega) \, | \, j, \, m^\prime \rangle, \end{equation*}
where $D_{m^\prime, \, m}^j(\cdot)$ denotes the rotation matrix.

\begin{example} For example, we have
\begin{equation*} D_{m^\prime, \, m}^{1/2}(\phi, \, \hat{z}) := \begin{pmatrix} \mathrm{e}^{\imath \frac{\phi}{2} } & 0 \\ 0 & \mathrm{e}^{- \imath \frac{\phi}{2} } \end{pmatrix} \quad \text{and} \quad D_{m^\prime, \, m}^{1/2}(\psi, \, \hat{y}) := \begin{pmatrix} \cos \frac{\psi}{2} & \sin \frac{\psi}{2} \\ \\ - \sin \frac{\psi}{2} & \cos \frac{\psi}{2} \end{pmatrix}, \end{equation*}
and, similarly, we have
\begin{equation*} D_{m^\prime, \, m}^{1}(\phi, \, \hat{z}) := \begin{pmatrix} \cos \phi & \sin \phi & 0 \\ - \sin \phi & \cos \phi & 0 \\ 0 &0 & 1 \end{pmatrix}.  \end{equation*}\end{example}

\begin{remark} We also consider the transformations of the form
\begin{equation*} | \, j_1, \, m_1 \rangle | \, j_2, \, m_2 \rangle \longmapsto \mathbb{U}(\omega) \, | \, j_1, \, m_1 \rangle | \, j_2, \, m_2 \rangle := \mathrm{e}^{\imath j_1 \omega} \mathrm{e}^{\imath j_2 \omega} \, | \, j_1, \, m_1 \rangle | \, j_2, \, m_2 \rangle, \end{equation*}
where
\begin{equation*}\mathrm{e}^{\imath j_1 \omega} \mathrm{e}^{\imath j_2 \omega} \, | \, j_1, \, m_1 \rangle | \, j_2, \, m_2 \rangle = \sum_{m_1^\prime, \, m_2^\prime} D_{m_1^\prime, \, m_1}^{j_1} D_{m_2^\prime, \, m_2}^{j_2} \, | \, j_1, \, m_1^\prime \rangle | \, j_2, \, m_2^\prime \rangle. \end{equation*} \end{remark}

{\centering \subsubsection*{Spherical Tensor Operators} }
\noindent The idea now is to rearrange the operators
\begin{equation*} \begin{aligned}& \mathbf{r} \longmapsto \mathrm{e}^{\imath \mathbf{L} \cdot \mathbf{\omega}} \, \mathbf{r} \, \mathrm{e}^{- \imath \mathbf{L} \cdot \mathbf{\omega}},
\\[1em] & \mathbf{p} \longmapsto \mathrm{e}^{\imath \mathbf{L} \cdot \mathbf{\omega}} \, \mathbf{p} \, \mathrm{e}^{- \imath \mathbf{L} \cdot \mathbf{\omega}}, \end{aligned}  \end{equation*}
in such a way to have spherical tensor operators\index{spherical tensor operators}. For example, for an operators of rank equal to one, i.e. $\mathbf{A} = (A_x, \, A_y, \, A_z)$, we may rearrange as follows:
\begin{equation*} \begin{cases} T_0^1 := A_z, \\[0.8em] T_1^1 := - \frac{1}{\sqrt{2}} (A_x + \imath A_y), \\[0.8em] T_{-1}^1 := \frac{1}{\sqrt{2}} (A_x - \imath A_y). \end{cases} \end{equation*}
For an operator of rank equal to two, we may rearrange as follows:
\begin{equation*} \begin{cases} T_0^2 := \frac{1}{\sqrt{6}} (A_{xx} + A_{yy} - 2 A_{zz}), \\[0.8em] T_{\pm 1}^2 := \mp (A_{xz} \pm \imath A_{yz}), \\[0.8em] T_{\pm 2}^2 := - \frac{1}{2}(A_{xx} - A_{yy} \pm 2 \imath A_{xy}). \end{cases} \end{equation*}
In general, though, spherical tensor operators can be defined using the Clebsch–Gordan coefficients as follows
\begin{equation*}T_Q^P := \langle p_1, \, q_1; \; p_2, \, q_2 \, | \, P, \, Q \rangle \, T_{q_1}^{p_1}T_{q_2}^{p_2}, \end{equation*}
and the transformations above are simply given by
\begin{equation*} T_q^p \longmapsto U(\omega) T_q^p U(\omega)^{-1} = \sum_{q^\prime} D_{q^\prime, \, q}^p(\omega) \, T_{q^\prime}^p.\end{equation*}
In particular, for spherical tensor operators, we have that
\begin{equation*}\begin{aligned} T_q^p | \, j, \, m \rangle \longmapsto \mathrm{e}^{\imath j \omega} T_q^p \, | \, j, \, m \rangle & = \mathrm{e}^{\imath j \omega} T_q^p \mathrm{e}^{-\imath j \omega} \mathrm{e}^{\imath j \omega} \, | \, j, \, m \rangle =
\\[1em] & = \sum_{q^\prime, \, m^\prime} D_{q^\prime, \, q}^p(\omega)D_{m^\prime, \, m}^j(\omega) \, T_{q^\prime}^p \, | \, j, \, m^\prime \rangle, \end{aligned} \end{equation*}
which can be compactly rewritten as
\begin{equation*}| \, p, \, q \rangle | \, j, \, m \rangle \longmapsto \sum_{q^\prime, \, m^\prime} D_{q^\prime, \, q}^p(\omega)D_{m^\prime, \, m}^j(\omega)  | \, p, \, q^\prime \rangle  | \, j, \, m^\prime \rangle. \end{equation*}
Let us now multiply for an arbitrary eigenstate $| \, J, \, M \rangle$ on both the left and the right-hand side of the identity above. Then
\begin{equation*}\langle J, \, M \, | \, \mathrm{e}^{\imath j \omega} T_q^p \, | \, j, \, m \rangle = \sum_{q^\prime, \, m^\prime} D_{q^\prime, \, q}^p(\omega)D_{m^\prime, \, m}^j(\omega) \langle J, \, M \, | \,  T_{q^\prime}^p \, | \, j, \, m^\prime \rangle,  \end{equation*}
and, equivalently,
\begin{equation*}\langle J, \, M \, | \, \mathrm{e}^{\imath j \omega} | \, p, \, q \rangle \, | \, j, \, m \rangle = \sum_{q^\prime, \, m^\prime} D_{q^\prime, \, q}^p(\omega)D_{m^\prime, \, m}^j(\omega) \langle J, \, M \, | \, p, \, q \rangle \, | \, j, \, m^\prime \rangle.  \end{equation*}
It follows that
\begin{equation*} \sum_{M^\prime} (D_{M, \, M^\prime}^J)^\ast \langle J, \, M^\prime \, | \, T_q^p \, | \, j, \, m \rangle  = \sum_{q^\prime, \, m^\prime} D_{q^\prime, \, q}^p(\omega)D_{m^\prime, \, m}^j(\omega) \langle J, \, M \, | \,  T_{q^\prime}^p \, | \, j, \, m^\prime \rangle, \end{equation*}
and
\begin{equation*} \sum_{M^\prime} (D_{M, \, M^\prime}^J)^\ast \langle J, \, M^\prime \, | \, | \, p, \, q \rangle \, | \, j, \, m \rangle  = \sum_{q^\prime, \, m^\prime} D_{q^\prime, \, q}^p(\omega)D_{m^\prime, \, m}^j(\omega) \langle J, \, M \, | \, | \, p, \, q \rangle \, | \, j, \, m^\prime \rangle.  \end{equation*}
We now use the orthogonality relation
\begin{equation*} \int D_{m, \, m^\prime}^j(\omega^\ast)D_{q, \, q^\prime}^p(\omega) \, \mathrm{d}\omega = \frac{4 \pi^2}{2j + 1} \delta_{jp} \delta_{m^\prime q^\prime} \delta_{m q} \end{equation*}
to infer the thesis of the Wigner–Eckart theorem, that is,
\begin{equation} \label{finale} \langle J, \, M_\alpha \, | \, T_q^p \, | \, j, \, m_\beta \rangle = \langle J, \, M \, | p, \, q \,\rangle \,| \, j, \, m \rangle \cdot \langle J_\alpha \| T^p \| j_\beta \rangle, \end{equation}
where $\alpha$ and $\beta$ are quantum numbers, $\langle J, \, M \, | p, \, q \,\rangle \,| \, j, \, m \rangle$ is an universal object that does not depend on $\alpha$ and $\beta$ (the Clebsch–Gordan coefficient), and $\langle J_\alpha \| T^{(p)} \| j_\beta \rangle $ denotes some value that does not depend on $m$, $m^\prime$, nor $q$ and is referred to as the \textbf{reduced matrix element}\index{reduced matrix element}.
\printindex

\bibliography{Bibliografia}{} % BIBLIOGRAFIA
\bibliographystyle{plain}
\end{document}
