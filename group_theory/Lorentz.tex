\chapter{Lorentz Group} \thispagestyle{empty}

The \textit{Lorentz group}, denoted by $\mathrm{O}(1, \, 3)$, is the group of all the invertible matrices $\Lambda \in \mathrm{GL}(4, \, \R)$ preserving the following transformation
\begin{equation}\label{eq.20.1} \Lambda_\mu^\eta g_{\eta \lambda} \Lambda_\nu^\lambda = g_{\mu \nu}, \end{equation}
where
\begin{equation*} g \equiv (g_{\mu \nu})_{\mu, \, \nu = 0, \, \dots, \, 3} = \begin{pmatrix} 1 & 0 & 0 & 0 \\ 0 & - 1 & 0 & 0 \\ 0 & 0 & -1 & 0 \\ 0 & 0 & 0 & -1 \end{pmatrix} \end{equation*}
is the Minkowski space-time metric. The transformations of the Lorentz group are thus given by
\begin{equation} \label{eq.20.2} p^\mu \longmapsto \Lambda_\nu^\mu p^\nu, \end{equation}
where $\Lambda$ is a regular matrix satisfying \eqref{eq.20.1} and $p^\mu$ is a contravariant $4$-vector\index{contravariant $4$-vector}. 

\section{Introduction}

In this chapter we will mainly focus on a class of equivalence of the Lorentz group $\mathrm{O}(1, \, 3)$, namely the connected component $\mathrm{SO}^+(1, \, 3)$ (usually called \textit{proper orthochronous Lorentz group}\index{proper orthochronous Lorentz group}), which will be introduced shortly.

We now introduce a particular notation for the inverse of $g$ in such a way that the Einstein convention works just fine, that is, 
\begin{equation*}  g^{\mu \nu} := (g^{-1})_{\mu \nu} \quad \text{for $\mu, \, \nu = 0, \, \dots, \, 3$}. \end{equation*}
Recall also that $p^\mu$ denotes a contravariant $4$-vector, and $p_\mu$ a covariant $4$-vector\index{covariant $4$-vector}. It follows that the scalar product induced by the Minkowski metric $g_{\mu \nu}$ can be compactly written as
\begin{equation} \label{eq.20.3} p^\mu q_\nu := p^\mu q^\nu g_{\mu \nu} = p_\mu q_\nu g^{\mu \nu}. \end{equation}
Notice also that from \eqref{eq.20.1} we necessarily have
\begin{equation*} \left[ \mathrm{det}(\Lambda)\right]^2 = 1 \implies \mathrm{det}(\Lambda) = \pm 1, \end{equation*}
and we decide (arbitrarily) to work in the component of the space with
\begin{equation} \label{eq.20.c1} \mathrm{det}(\Lambda) = 1. \end{equation}
Similarly, notice that
\begin{equation*}\Lambda_\mu^\eta g_{\eta \lambda} \Lambda_\nu^\lambda = g_{\mu \nu} \implies (\Lambda_0^0)^2 - \sum_{i = 1}^3 (\Lambda_0^i)^2 = 1 \implies (\Lambda_0^0)^2 \geq 1, \end{equation*}
which means that either $\Lambda_0^0$ is greater than or equal to $1$ or less than or equal to $- 1$. We decide (arbitrarily) to work in the component of the space where
\begin{equation} \label{eq.20.c2} \Lambda_0^0 \geq 1. \end{equation}
The connected component of $\mathrm{O}(1, \, 3)$ satisfying the properties \eqref{eq.20.1}, \eqref{eq.20.c1} and \eqref{eq.20.c2} is called \textit{proper orthochronous Lorentz group}, and it is usually denoted by $\mathfrak{T}_+$.

\begin{remark} Recall that the Levi-Civita tensor $\epsilon_{\mu \nu \lambda \sigma}$ is defined by setting
\begin{equation*}\epsilon_{\mu \nu \lambda \sigma} := \begin{cases} 1 & \text{if $(\mu \nu \lambda \sigma)$ even permutation of $(0 1 2 3)$}, \\[0.8em] - 1 & \text{if $(\mu \nu \lambda \sigma)$ odd permutation of $(0 1 2 3)$}, \\[0.8em] 0 & \text{otherwise}. \end{cases} \end{equation*}
The reader may check that the matrices $\Lambda \in \mathfrak{T}_+$ satisfies the following useful property
\begin{equation*}\epsilon_{\mu \nu \lambda \sigma} = \Lambda_\mu^\alpha \Lambda_\nu^\beta \Lambda_\lambda^\gamma \Lambda_\sigma^\delta \epsilon_{\alpha \beta \gamma \delta} \end{equation*}
by means of the following well-known formula for the determinant:
\begin{equation*}1 = \mathrm{det}(\Lambda) = \epsilon_{\mu \nu \lambda \sigma} \Lambda_0^\mu \Lambda_0^\nu \Lambda_0^\lambda \Lambda_0^\sigma.  \end{equation*}
In particular, the Levi-Civita tensor is called \textit{invariant tensor} for the proper Lorentz group (similarly to the Levi-Civita tensor $\epsilon_{ijk}$ for the group $\mathrm{SU}(3, \, \C)$).\end{remark}

\section{Irreducible Representations of $\mathfrak{T}_+$: Part I}

In this section, we investigate the irreducible representations of the group $\mathfrak{T}_+$, and we take a closer look at the Lie algebra.

\subsection{Generators}

Recall that the Lie group $\mathrm{SO}(4, \, \R)$ is generated by the six matrices $M^{ij}$, where the couple $(i, \, j)$ denotes the rotation plane.

Using the same notation, we now introduce the generators of $\mathfrak{T}_+$, and we show that - accordingly to what happens in $\mathrm{SO}(4, \, \R)$ - only $6$ of them are actually necessary. More precisely, we start with the generators
\begin{equation*} (J_{\lambda \sigma})_\mu^\nu := - \imath \left( \delta_\lambda^\nu g_{\sigma \mu} - \delta_\sigma^\nu g_{\lambda \mu} \right) \quad \text{for $\lambda, \, \sigma = 0, \, \dots, \, 3$}, \end{equation*}
and we notice that
\begin{equation*} J_{\lambda \sigma} = - J_{\sigma \lambda}, \end{equation*}
so that only $6$ of them are actually necessary to generate the Lie group $\mathfrak{T}_+$. We define
\begin{equation*} K_m := J_{m, \, 0} \quad \text{for $m = 1, \, 2, \, 3$}, \end{equation*}
and these are nothing more than the \textit{boost}\index{Lorentz boosts} in the directions $\hat{x}$, $\hat{y}$ and $\hat{z}$ respectively. In a similar fashion, we define
\begin{equation*} J_k := \frac{1}{2} \epsilon^{kmn} J_{m, \, n} \quad \text{for $k = 1, \, 2, \, 3$}, \end{equation*}
and these are the generators of the rotation subgroup isomorphic to $\mathrm{SO}^+(3, \, \R)$ (i.e., the rotations with respect to the space component of the Minkowski space-time). Clearly, these six matrices generate the Lie group $\mathfrak{T}_+$, and it is easy to see that
\begin{equation*} \Lambda_{\mathrm{rot}}^{\alpha \theta \phi} := \begin{pmatrix}1 & 0 & 0 & 0 \\ 0 & \cos \alpha \cos \theta \cos \phi - \sin \alpha \sin \phi & - \cos \alpha \cos\theta \sin \phi - \sin \alpha \cos \phi & \cos\alpha \sin \theta  \\ 0 & \sin \alpha \cos \theta \cos \phi - \cos \alpha \sin \phi & - \sin \alpha \cos\theta \sin \phi + \cos \alpha \cos \phi & \sin\alpha \sin \theta  \\ 0 & - \sin \theta \cos \phi & \sin \theta \sin \phi & \cos \theta \end{pmatrix} \end{equation*}
is the transformation in $\mathfrak{T}_+$ associated to $\{J_k\}_{k = 1, \, 2, \, 3}$, while
\begin{equation*} \Lambda_{\mathrm{boost}}^{\gamma \beta \hat{z}} := \begin{pmatrix} \gamma & 0 & 0 & \beta \gamma \\ 0 & 1 & 0  & 0 \\ 0 & 0 & 1 & 0 \\ \beta \gamma & 0 & 0 & \gamma \end{pmatrix} \end{equation*}
is, for example, the boost transformation along the $z$-axis. The parameters of the boost elements are nothing else than the Lorentz transformations parameters, that is,
\begin{equation*} \beta = \frac{v}{c} \in [-1, \, 1] \quad \text{and} \quad \gamma = \frac{1}{\sqrt{1 - \beta^2}} \geq 1. \end{equation*}

\begin{remark} There is an alternative parametrization for the boost elements\index{Lorentz boosts alternative}. Namely, we replace $\beta$ with the hyperbolic tangent of another parameter, that is, we set
\begin{equation*} \beta := \tanh \omega. \end{equation*}
Then $\omega$ takes value in $(- \infty, \, + \infty)$, and therefore
\begin{equation*} \gamma = \frac{1}{\sqrt{1 - \tanh^2 \omega}} = \cosh \omega \geq 1, \end{equation*}
which gives us the boost transformation
\begin{equation*} \Lambda_{\mathrm{boost}}^{\gamma \omega \hat{z}} := \begin{pmatrix} \cosh \omega & 0 & 0 & \sinh \omega \\ 0 & 1 & 0  & 0 \\ 0 & 0 & 1 & 0 \\ \sinh \omega & 0 & 0 & \cosh \omega \end{pmatrix}. \end{equation*}
\end{remark}

\subsection{Lie Algebra of $\mathfrak{T}_+$}

In this section, we investigate the Lie algebra associated to $\mathfrak{T}_+$, denoted by $\mathrm{so}(3, \, 1)$, by means of different sets of generators. First, a simple computation shows that
\begin{equation*}\begin{aligned}( J_{\mu \nu} )_\beta^\alpha (J_{\sigma \lambda})_\gamma^\beta & = (- \imath)^2 \left( \delta_\mu^\alpha g_{\nu \beta} - \delta_\nu^\alpha g_{\mu \beta} \right)\left( \delta_\sigma^\beta g_{\lambda \gamma} - \delta_\lambda^\beta g_{\sigma \gamma} \right) =
\\[1em] & = - \delta_\mu^\alpha g_{\nu \beta} \delta_\sigma^\beta g_{\lambda \gamma} + \delta_\mu^\alpha g_{\nu \beta} \delta_\lambda^\beta g_{\sigma \gamma} + \delta_\nu^\alpha g_{\mu \beta} \delta_\sigma^\beta g_{\lambda \gamma} - \delta_\nu^\alpha g_{\mu \beta} \delta_\lambda^\beta g_{\sigma \gamma} =
\\[1em] & = - \delta_\mu^\alpha g_{\nu \sigma} g_{\lambda \gamma} + \delta_\mu^\alpha g_{\nu \lambda} g_{\sigma \gamma} + \delta_\nu^\alpha g_{\mu \sigma} g_{\lambda \gamma} - \delta_\nu^\alpha g_{\mu \lambda} g_{\sigma \gamma} =
\\[1em] & = \imath (J_{\mu \nu})_\lambda^\alpha g_{\sigma \gamma} - \imath (J_{\mu \nu})_\sigma^\alpha g_{\lambda \gamma}, \end{aligned}\end{equation*}
and therefore
\begin{equation*}\begin{aligned} [J_{\mu \nu}, \, J_{\sigma \lambda}]_{\gamma}^\alpha & = ( J_{\mu \nu} )_\beta^\alpha (J_{\sigma \lambda})_\gamma^\beta - ( J_{\sigma \lambda} )_\beta^\alpha (J_{\mu \nu})_\gamma^\beta =
\\[1em] & =  \imath \left[ (J_{\mu \nu})_\lambda^\alpha g_{\sigma \gamma} - (J_{\mu \nu})_\sigma^\alpha g_{\lambda \gamma} - \left( (J_{\sigma \lambda})_\nu^\alpha g_{\mu \gamma} -  (J_{\sigma \lambda})_\mu^\alpha g_{\nu \gamma} \right) \right] =
\\[1em] & = \imath \left[ (J_{\mu \nu})_\lambda^\alpha g_{\sigma \gamma} - (J_{\mu \nu})_\sigma^\alpha g_{\lambda \gamma} - (J_{\sigma \lambda})_\nu^\alpha g_{\mu \gamma} + (J_{\sigma \lambda})_\mu^\alpha g_{\nu \gamma} \right]. \end{aligned}\end{equation*}
In order to simplify the Lie algebra, it is convenient to use the $3 + 3$ generators introduced above, that is, $\{J_k\}_{k = 1, \, 2, \, 3}$. and $\{ K_m \}_{m = 1, \, 2, \, 3}$. It turns out that
\begin{equation*}\begin{aligned} & [J_m, \, J_n] = \imath \epsilon_{mnk} J_k, \\[1em] & [K_m, \, J_n] = \imath \epsilon_{mn \ell} K_\ell, \\[1em] & [K_m, \, K_n] = - \imath \epsilon_{mn\ell} J_\ell, \end{aligned}\end{equation*}
and therefore the set of generators $\{J_k\}_{k = 1, \, 2, \, 3}$ forms an invariant subalgebra isomorphic to $\mathrm{so}(3, \, \R)$, that is,
\begin{equation*} \langle J_1, \, J_2, \, J_3 \rangle \cong \mathrm{so}(3, \, \R) \subset \mathrm{so}(3, \, 1). \end{equation*}
Now let
\begin{equation*} M_n := \frac{1}{2}(J_n + \imath K_n) \quad \text{and} \quad M_n := \frac{1}{2} (J_n - \imath K_n) \end{equation*}
for $n = 1, \, 2, \, 3$, and notice that
\begin{equation*}\begin{aligned} & [M_m, \, M_n] = \imath \epsilon_{mn \ell} M_\ell, \\[1em] & [N_m, \, N_n] =- \imath \epsilon_{mn \ell} N_\ell, \\[1em] & [M_m, \, N_n] =0, \end{aligned}\end{equation*}
and therefore the sets of generators form two invariant subalgebra isomorphic to $\mathrm{su}(2, \, \C)$. It follows that the Lie algebra of the Lorentz group $\mathfrak{T}_+$ is isomorphic to the product of two copies of the Lie algebra associated to $\mathrm{SU}(2, \, \C)$, that is,
\begin{equation*} \mathrm{so}(3, \, 1) \sim \langle M_n, \, N_m \rangle_{m, \, n = 1, \, 2, \, 3} \sim \mathrm{su}(2, \, \C) \times \mathrm{su}(2, \, \C). \end{equation*}
It follows that the irreducible finite-dimensional representations of the Lorentz group $\mathfrak{T}_+$ are in correspondence with the couples $(j_1, \, j_2)$, where $j_1, \, j_2 \in 1/2 \Z$ denote the irreducible representation of $\mathrm{SU}(2, \, \C)$ we have already investigated thoroughly. For example
\begin{equation*} \text{$(0, \, 0)$ is the trivial one-dimensional representation}, \end{equation*}
while 
\begin{equation*} \text{$(\frac{1}{2}, \, 0)$ and $(0, \, \frac{1}{2})$} \end{equation*}
are the spinorial representations of chirality left $(L)$ and right $(R)$ respectively\index{representation!spinorial}.

\section{Spinorial Representations}

The goal of this brief section is to expand a little bit on the last notion we introduced in the previous discussion: the spinorial representations.

\begin{remark} The Lorentz group $\mathfrak{T}_+$ is not compact since the boost's parameter $\gamma$ can be arbitrarily big ($\gamma \geq 1$), and therefore finite-dimensional representations need not to be unitary. \end{remark}

First, recall that the special linear group $\mathrm{SL}(2, \, \C)$ is a $6$-parameters Lie group because an arbitrary element is given by
\begin{equation*} \begin{pmatrix} \alpha_1 + \imath \beta_1 & \alpha_2 + \imath \beta_2 \\ \alpha_3 + \imath \beta_3 & \alpha_4 + \imath \beta_4 \end{pmatrix} \quad \text{for $\alpha_i, \, \beta_i \in \R$}, \end{equation*}
satisfying the constraint $\mathrm{det} A = 1$, which gives us two equations (real and imaginary part).

In the first half of this section, the main goal is to prove that every transformation in the Lorentz group $\mathfrak{T}_+$ corresponds to an element of the group $\mathrm{SL}(2, \, \C)$, and infer that
\begin{equation*}\mathfrak{T}_+ \cong \mathrm{SL}(2, \, \C). \end{equation*}
Denote by $\sigma^i$ the $i$th Pauli's matrix. We introduce an useful notation that takes into account the particular form of the Minkowski metric, that is,
\begin{equation*}\sigma^\mu := (- \mathrm{id}_{2 \times 2}, \, \sigma^i)_{i = 1, \, 2, \, 3}, \end{equation*}
and
\begin{equation*}\bar{\sigma}^\mu := (- \mathrm{id}_{2 \times 2}, \, - \sigma^i)_{i = 1, \, 2, \, 3}. \end{equation*}
Let $p^\mu$ be a given $4$-vector. We associate to $p^\mu$ the matrix
\begin{equation*}P := p^\mu \sigma_\mu, \end{equation*}
where $\sigma_\mu = \sigma^\nu g_{\mu \nu}$. A straightforward computation shows that
\begin{equation*}P  = p^0 \sigma^0 - p^i \sigma^i = \begin{pmatrix}- p^0 - p^3 & - p^1 + \imath p^2 \\ - p^1 - \imath p^2 & - p^0 + p^3 \end{pmatrix}, \end{equation*}
and, therefore, we can easily reverse this operation and find that
\begin{equation*}p^\mu = \frac{1}{2} \mathrm{Tr} \left[ \bar{\sigma}^\mu P \right], \end{equation*}
where $P$ is the matrix above. In particular, every $4$-vector $p^\mu$ corresponds to a matrix, denoted by the capital $P$, and therefore it suffices to show that $\mathrm{SL}(2, \, \C)$ acts on $P$ in the same way $\mathfrak{T}_+$ acts on $p^\mu$ to infer that
\begin{equation*}\mathfrak{T}_+ \cong \mathrm{SL}(2, \, \C). \end{equation*}
Let $M \in \mathrm{SL}(2, \, \C)$. The action on $P$ is defined by
\begin{equation*}P \longmapsto M P M^\dagger \equiv P^\prime, \end{equation*}
and this shows immediately that
\begin{equation*}p^\mu p_\mu = \mathrm{det} P = \mathrm{det} P^\prime = (p^\prime)^\mu p_\mu^\prime. \end{equation*}

\begin{example} Let $\phi \in [0, \, 2 \pi)$. The matrix $M$ could be chosen in such a way that
\begin{equation*} M_\theta := \mathrm{e}^{\frac{\imath}{2} \phi \sigma_3} = \begin{pmatrix} \mathrm{e}^{\imath \frac{\phi}{2}} & 0 \\ 0 & \mathrm{e}^{-\imath \frac{\phi}{2}} \end{pmatrix} \quad \text{or} \quad M = \mathrm{e}^{\frac{\imath}{2} \phi \sigma_2} = \begin{pmatrix} \cos \frac{\phi}{2} & \sin \frac{\phi}{2} \\ - \sin \frac{\phi}{2} & \cos \frac{\phi}{2} \end{pmatrix}. \end{equation*}
Furthermore, if $\omega \in (- \infty, \, + \infty)$ is the parameter introduced above, it makes sense to consider the real exponential
\begin{equation*} M_\omega := \mathrm{e}^{\frac{1}{2} \omega \sigma_3} = \begin{pmatrix} \mathrm{e}^{ \frac{\omega}{2}} & 0 \\ 0 & \mathrm{e}^{- \frac{\omega}{2}} \end{pmatrix} \end{equation*}
since, given a $3$-dimensional vector $\underline{v}$, it is easy to see that
\begin{equation*} \text{$\mathrm{Tr}(\sigma^i) = 0$ for all $i = 1, \, 2, \, 3$} \implies \mathrm{det} \left[ \mathrm{e}^{\frac{1}{2} \underline{v} \cdot \underline{\sigma} } \right] = 1. \end{equation*} \end{example}

Let us focus on the complex exponential transformation induced by the matrix $M_\theta$ introduced above. A straightforward computation shows that
\begin{equation*}\begin{aligned} P \longmapsto P^\prime & = \begin{pmatrix} \mathrm{e}^{\imath \frac{\phi}{2}} & 0 \\ 0 & \mathrm{e}^{-\imath \frac{\phi}{2}} \end{pmatrix} \begin{pmatrix}- p^0 - p^3 & - p^1 + \imath p^2 \\ - p^1 - \imath p^2 & - p^0 + p^3 \end{pmatrix} \begin{pmatrix} \mathrm{e}^{- \imath \frac{\phi}{2}} & 0 \\ 0 & \mathrm{e}^{\imath \frac{\phi}{2}} \end{pmatrix} =
\\[1em] & \begin{pmatrix}- (p^\prime)^0 - (p^\prime)^3 & - (p^\prime)^1 + \imath (p^\prime)^2 \\ - (p^\prime)^1 - \imath (p^\prime)^2 & - (p^\prime)^0 + (p^\prime)^3 \end{pmatrix}, \end{aligned}\end{equation*}
where
\begin{equation*} \begin{cases} (p^0)^\prime=p^0, \\[0.7em] (p^3)^\prime = p^3, \end{cases} \quad \text{and} \quad \begin{cases} (p^1)^\prime=p^1 \cos \theta + p^2 \sin \theta, \\[0.7em] (p^2)^\prime = - p^1 \sin \theta + p^2 \cos \theta \end{cases} \implies \begin{pmatrix} (p^1)^\prime \\ (p^2)^\prime \end{pmatrix} = R(-\theta) \begin{pmatrix} p^1 \\ p^2 \end{pmatrix}.  \end{equation*}
In a similar fashion, if we focus on the real exponential transformation induced by the matrix $M_\omega$ introduced above, we have that
\begin{equation*}\begin{aligned} P \longmapsto P^\prime & = \begin{pmatrix} \mathrm{e}^{ \frac{\omega}{2}} & 0 \\ 0 & \mathrm{e}^{- \frac{\omega}{2}} \end{pmatrix} \begin{pmatrix}- p^0 - p^3 & - p^1 + \imath p^2 \\ - p^1 - \imath p^2 & - p^0 + p^3 \end{pmatrix} \begin{pmatrix} \mathrm{e}^{ \frac{\omega}{2}} & 0 \\ 0 & \mathrm{e}^{- \frac{\omega}{2}} \end{pmatrix} =
\\[1em] & \begin{pmatrix}- (p^\prime)^0 - (p^\prime)^3 & - (p^\prime)^1 + \imath (p^\prime)^2 \\ - (p^\prime)^1 - \imath (p^\prime)^2 & - (p^\prime)^0 + (p^\prime)^3 \end{pmatrix}, \end{aligned}\end{equation*}
where
\begin{equation*} \begin{cases} (p^1)^\prime=p^1, \\[0.7em] (p^2)^\prime = p^2, \end{cases} \quad \text{and} \quad \begin{cases} (p^0)^\prime=p^0 \cos \theta - p^3 \sin \theta, \\[0.7em] (p^3)^\prime = p^0 \sin \theta + p^3 \cos \theta \end{cases} \implies \begin{pmatrix} (p^0)^\prime \\ (p^3)^\prime \end{pmatrix} = R(\theta) \begin{pmatrix} p^0 \\ p^3 \end{pmatrix}.  \end{equation*}
In particular, the complex exponentials (w.r.t. the Pauli's matrices) give us the transformations of the form $\Lambda_{\mathrm{rot}}^{\alpha \theta \phi}$, while the real exponentials give us the boost transformations, and therefore we can finally infer that
\begin{equation*}\mathfrak{T}_+ \cong \mathrm{SL}(2, \, \C). \end{equation*}
We are now ready to investigate the spinorial representations introduced above. Consider a Weyl spinor (L) with two components $\psi = \begin{pmatrix} \cdot \\ \cdot \end{pmatrix} \sim ( \frac{1}{2}, \, 0)$ that transforms like
\begin{equation*} \psi \longmapsto M \psi, \end{equation*}
and notice that $\bar{\psi} \sim (0, \, \frac{1}{2})$ and
\begin{equation*} \bar{\psi} \longmapsto M^\ast \bar{\psi}. \end{equation*}
It follows that
\begin{equation*} \psi \longmapsto \psi^\prime \implies (\psi^\prime)_\alpha = M_\alpha^\beta \psi_\beta, \end{equation*}
and, if we set $\psi^\alpha := \epsilon^{\alpha \beta} \psi_\beta$, then one can easily prove that
\begin{equation*} (\psi^\prime)^\alpha = (M^{-1})_\beta^\alpha \psi^\beta. \end{equation*}
In a similar fashion, one could also prove that the conjugate transforms in a similar way, that is,
\begin{equation*}(\bar{\psi}^\prime)_\alpha = (M^\ast)_\alpha^\beta \bar{\psi}_\beta \implies (\bar{\psi}^\prime)^\alpha = \left[ (M^{-1})^\ast \right]_\beta^\alpha \bar{\psi}^\beta. \end{equation*}
Let $\psi$ and $\chi$ be two Weyl $(L)$-spinor. The scalar product is commutative as a consequence of the following, straightforward, computation:
\begin{equation*}\begin{aligned} \psi_\alpha \chi^\alpha & = \psi^\beta \epsilon_{\alpha \beta} \chi_\gamma \epsilon^{\alpha \gamma} =
\\[1em] & = \epsilon_{\alpha \beta} \epsilon^{\alpha \gamma} \psi^\beta \chi_\gamma =
\\[1em] & =  \delta_\beta^\gamma \psi^\beta \chi_\gamma = \psi^\alpha \chi_\alpha. \end{aligned} \end{equation*} 
Notice that, for a two-component spinor $\psi$, it is equivalent to be a Weyl spinor or a Majorana spinor since we can write it in the form
\begin{equation*} \psi_M = \begin{pmatrix} \psi \\ \overline{\psi} \end{pmatrix}, \end{equation*}
which means that one is the antiparticle of the other one. The Dirac spinor
\begin{equation*} \psi_D = \begin{pmatrix} \psi \\ \chi \end{pmatrix}, \end{equation*}
on the other hand, is not equivalent because the two components are independent.

\begin{example}[QCD] Let us consider the two-component spinor
\begin{equation*} \psi = \begin{pmatrix} q_L \\ q_R \end{pmatrix} \sim \underline{3}, \end{equation*}
where $\underline{3}$ is the fundamental representation of $\mathrm{SU}(3, \, \C)$ (color), and
\begin{equation*} \begin{aligned} & \psi_L \sim q_L \: : \: \substack{ \longrightarrow \\ \longleftarrow } \\[1em] & \psi_R \sim q_R \: : \: \substack{ \longrightarrow \\ \longrightarrow } \end{aligned} \qquad \text{and} \qquad \lambda = \frac{p \cdot \mathfrak{J}}{|p|}. \end{equation*}
The fermions "condense" in the empty space and, in particular, we have that
\begin{equation*}\langle \psi \bar{\psi} \rangle = \langle q_R^c q_L \rangle + \langle q_L^c q_R \rangle. \end{equation*} \end{example}

\paragraph{Dirac Spinor.} Recall that the Dirac matrices\index{Dirac matrix}\index{Lorentz group!Dirac representation} $\gamma^\mu$'s are defined by means of the Pauli matrices $\sigma^\mu$'s as follows:
\begin{equation*} \gamma^\mu = \begin{pmatrix} 0 & \sigma^\mu \\ \bar{\sigma}^\mu & 0 \end{pmatrix} \quad \text{and} \quad \gamma^5 = \gamma^0\gamma^1\gamma^2\gamma^3 = \begin{pmatrix} \mathrm{id}_{2 \times 2} & 0 \\ 0 & - \mathrm{id}_{2 \times 2} \end{pmatrix}. \end{equation*}
Let $(m, \, \bar{\psi}_D, \, \psi_D)$ be a Dirac mass triplet. We introduce the Dirac conjugate $\psi_D^+$ as
\begin{equation*} \bar{\psi}_D = \psi_D^+ \gamma_0, \end{equation*}
and we notice that
\begin{equation*} m \bar{\psi}_D \psi_D = \psi_R^c \psi_L + hc. \end{equation*}
The Dirac spinor representation is \textbf{not} an irreducible representation of the Lorentz group $\mathfrak{T}_+$, but it is the direct sum of two irreducible representations. More precisely, we have that
\begin{equation*} \psi_D \sim ( \frac{1}{2}, \, 0 ) \oplus (0, \, \frac{1}{2}), \end{equation*}
while the Weyl spinor representation is obviously irreducible and given by
\begin{equation*} \psi_M \sim ( \frac{1}{2}, \, 0 ). \end{equation*}
The tensor product representation $\psi_M \otimes \psi_M$ is clearly equal to the trivial representation $\underline{0}$ since it contains the singlet only. For example, the kinetic term 
\begin{equation*}\mathscr{L}_c = \bar{\psi} \imath \bar{\gamma}^\mu \partial_\mu \psi + \psi \leftrightarrow \chi, \end{equation*}
where $\bar{\psi} \equiv \psi^+$, is given by a $(1/2, \, 1/2)$ vector. Indeed, one can easily prove that
\begin{equation*} \bar{\psi} \sim (0, \, \frac{1}{2}) \quad \text{and} \quad \bar{\gamma}^\mu \psi \sim ( \frac{1}{2},\, 0) \quad \text{and} \quad \partial_\mu \sim (\frac{1}{2}, \, \frac{1}{2}) \implies \mathscr{L}_c \sim (\frac{1}{2}, \, \frac{1}{2}). \end{equation*}
In the neutrino theory ($m = 0$) the equation of motion is the well-known Weyl equation\index{Weyl equation}, which asserts that
\begin{equation} \label{eq.weyl} \imath \bar{\sigma}^\mu \partial_\mu \psi = 0. \end{equation}
On the other hand, in the Dirac theory ($m_D \neq 0$) the energy is given by
\begin{equation*} \mathscr{L}_c + m_D(\bar{\chi} \psi + \bar{\psi} \chi), \end{equation*}
and therefore the equation of motion is given by
\begin{equation*}\underbrace{\imath \bar{\sigma}^\mu \partial_\mu \psi}_{\text{Left (L) spinor}} + \underbrace{m_D \chi}_{\text{Right (R) spinor}} = 0. \end{equation*}
In any case, if we write explicitly the Weyl equation \eqref{eq.weyl}, we find that
\begin{equation*} (- \imath \partial_0 - \imath \sigma^i \partial_i ) \psi = 0, \end{equation*}
and therefore, if we set $p_i := - \imath \partial_i$ for $i = 1, \, 2, \, 3$, then
\begin{equation*} E :=  \imath \partial_0 \implies E = \vec{p} \cdot \vec{\sigma}.  \end{equation*}
In conclusion, since the Einstein's equation asserts that $E$ is equal to the modulus of $\vec{p}$, i.e. $\sqrt{p^2}$, we infer that
\begin{equation*} \frac{\vec{p} \cdot \vec{\sigma}}{|p|} = 1.  \end{equation*}

\section{Irreducible Representations of $\mathfrak{T}_+$: Part II}

In this finals section, the primary goal is to conclude the discussion of the possible set of generators of the Lorentz group and to find the finite-dimensional representations similarly to what we have already done for the groups introduced earlier.

\subsection{Generators of $\mathfrak{T}_+$}

Recall that the generators are defined by
\begin{equation*} (J_{\lambda \sigma})_\mu^\nu := - \imath \left( \delta_\lambda^\nu g_{\sigma \mu} - \delta_\sigma^\nu g_{\lambda \mu} \right) \quad \text{for $\lambda, \, \sigma = 0, \, \dots, \, 3$}. \end{equation*}
If we consider the Dirac matrices
\begin{equation*} \gamma_\mu = \begin{pmatrix} 0 & \sigma_\mu \\ \bar{\sigma}_\mu & 0 \end{pmatrix} \quad \text{and} \quad \gamma_5 = \gamma_0\gamma_1\gamma_2\gamma_3 , \end{equation*}
then it is easy to see that
\begin{equation*} J_{\mu \nu} = \frac{\imath}{4} [\gamma_\mu, \, \gamma_\nu] = \frac{\imath}{4} \begin{pmatrix} \sigma_\mu \bar{\sigma}_\nu - \sigma_\nu \bar{\sigma}_\mu & 0 \\ 0 &  \bar{\sigma}_\mu \sigma_\nu - \bar{\sigma}_\nu \sigma_\mu \end{pmatrix}, \end{equation*}
and, hence, we refer to $\{\gamma_\mu\}_{\mu = 0, \, \dots, \, 3}$ as \textit{chiral basis}, and to $\gamma_5$ as \textit{chiral operator}\index{chiral basis}\index{chiral operator}. Recall that
\begin{equation*}  \gamma_5= \begin{pmatrix} \mathrm{id}_{2 \times 2} & 0 \\ 0 & - \mathrm{id}_{2 \times 2} \end{pmatrix}, \end{equation*}
and this implies that the chiral operator commutes with each generator $J_{\mu \nu}$, that is,
\begin{equation*} [J_{\mu \nu}, \, \gamma_5] = 0 \implies \text{invariant chirality of the Lorentz group}, \end{equation*}
and it does not depend on the reference system.

\subsection{Finite Irreducible Representations of $\mathfrak{T}_+$}

In the first half of the chapter, we proved that the Lie algebra associated to $\mathfrak{T}_+$ is isomorphic to the Lie algebra $\mathrm{su}(2, \, \C) \times \mathrm{su}(2, \, \C)$. We also introduced the operators
\begin{equation*} \begin{cases} K_m := J_{m, \, 0}  & \text{for $m = 1, \, 2, \, 3$}, \\[0.8em] J_k := \frac{1}{2} \epsilon^{kmn} J_{m, \, n} & \text{for $k = 1, \, 2, \, 3$}, \end{cases} \quad \text{and} \qquad \begin{cases} M_n := \frac{1}{2}(J_n + \imath K_n) & \text{for $n = 1, \, 2, \, 3$}, \\[0.8em] M_n := \frac{1}{2} (J_n - \imath K_n) & \text{for $n = 1, \, 2, \, 3$}, \end{cases} \end{equation*}
in such a way that the following commutator relations hold true:
\begin{equation*}\begin{aligned} & [M_m, \, M_n] = \imath \epsilon_{mn \ell} M_\ell, \\[1em] & [N_m, \, N_n] =- \imath \epsilon_{mn \ell} N_\ell, \\[1em] & [M_m, \, N_n] =0. \end{aligned}\end{equation*}
It follows that the (second) sets of generators form two invariant subalgebra isomorphic to $\mathrm{su}(2, \, \C)$, which means that the Lie algebra of the Lorentz group $\mathfrak{T}_+$ is semisimple, but not simple.

Let $j \in \frac{1}{2}\Z$ be a representative of a finite-dimensional representation of $\mathrm{su}(2, \, \C)$, and let $\mathbb{J}^2 = j(j + 1)$ denote the Casimir operator. Consider the usual basis of eigenstates
\begin{equation*} \{ |\, j, \, m \rangle \}_{m = -j, \, \dots, \, j} \end{equation*}
of the irreducible representation of $\mathrm{su}(2, \, \C)$. Furthermore, recall that the irreducible finite-dimensional representations of the Lorentz group $\mathfrak{T}_+$ are in correspondence with the couples $(j_1, \, j_2)$, where $j_1, \, j_2 \in 1/2 \Z$ denote the irreducible representations of $\mathrm{SU}(2, \, \C)$.

We can define, as we have already done in the previous chapters, the up/down operators as follows (obtaining the six generators of the Lorentz group):
\begin{equation*} \begin{cases} K_\pm := \imath( \mathbbm{1} \otimes N_{\pm} - M_{\pm} \otimes \mathbbm{1}), \\[0.8em] K_3 = \imath(N_3 - M_3), \end{cases} \quad \text{and} \qquad  \begin{cases} J_\pm := M_{\pm} \otimes \mathbbm{1} + \mathbbm{1} \otimes N_{\pm}, \\[0.8em] J_3 = M_3 + N_3. \end{cases} \end{equation*}
Consider now a general representation $(j_1, \, j_2)$, and let us denote by $|\, j_1, \, m_1\rangle |\, j_2, \, m_2\rangle$ the eigenstates of the two irreducible representations of $\mathrm{SU}(2, \, \C)$. Then
\begin{equation*} (J_3)_{(m_1^\prime, \, m_2^\prime); \; (m_1, \, m_2)} = \delta_{m_1^\prime, \, m_1} \delta_{m_2^\prime, \, m_2} (m_1 + m_2), \end{equation*}
and (see \hyperref[sec:oqweofdks]{Section \ref{sec:oqweofdks}})
\begin{equation*} \begin{aligned} (J_+)_{(m_1^\prime, \, m_2^\prime); \; (m_1, \, m_2)} & = \delta_{m_1^\prime, \, m_1+1} \delta_{m_2^\prime, \, m_2} \sqrt{(j_1 - m_1)(j_1 + m_1 + 1)} + \dots
\\[1em] & \dots + \delta_{m_1^\prime, \, m_1} \delta_{m_2^\prime, \, m_2 + 1} \sqrt{(j_2 - m_2)(j_2 + m_2 + 1)}. \end{aligned} \end{equation*}
In a similar fashion, it turns out that
\begin{equation*} (K_3)_{(m_1^\prime, \, m_2^\prime); \; (m_1, \, m_2)} = \imath \delta_{m_1^\prime, \, m_1} \delta_{m_2^\prime, \, m_2} (m_2 - m_1), \end{equation*}
and
\begin{equation*} \begin{aligned} (K_+)_{(m_1^\prime, \, m_2^\prime); \; (m_1, \, m_2)} & = \imath \Big[ \delta_{m_1^\prime, \, m_1} \delta_{m_2^\prime, \, m_2 + 1} \sqrt{(j_2 - m_2)(j_2 + m_2 + 1)}  - \dots
\\[1em] & \dots - \delta_{m_1^\prime, \, m_1+1} \delta_{m_2^\prime, \, m_2} \sqrt{(j_1 - m_1)(j_1 + m_1 + 1)} \Big]. \end{aligned} \end{equation*}

\begin{remark}The representation $(j_1, \, j_2)$ is finite-dimensional, but it is not unitary. In fact, the Lorentz group $\mathfrak{T}_+$ is not compact, and hence unitary representations need to be infinite-dimensional. \end{remark}

\section{Chirality}

In this final section, we briefly discuss the physic part concerning the Lorentz group, and we show that the theory developed so far is strongly correlated with the Higgs mechanism.

\subsection{Standard Model}

The Gauge group in the \textit{standard model}\index{standard model} is given by
\begin{equation*} \mathfrak{G} := \mathrm{SU}_{QED}(3, \, \C) \times \mathrm{SU}_L(2, \, \C) \times \mathrm{U}_Y(1, \, \C). \end{equation*}
The second term $\mathrm{SU}_L(2, \, \C)$ is profoundly related to Higgs mechanism\index{Higgs mechanism}. Indeed, the group acts on the bosons $W^\mu$, and the couple $(W^3, \, Y)$ is transformed in the couple $(Z, \, Y)$, where $Z$ is a different boson. The Higgs mechanism breaks the symmetry of $\mathrm{U}_{em}(1, \, \C)$ since the boson $Z$ acquires a whole lot of mass. It follows that the radius is small, and the potential has an exponential growth, that is,
\begin{equation*} V_Y = g \frac{ \mathrm{e}^{- m \tau}}{\tau} \qquad \text{"\textit{Yukawa Potential}"}, \end{equation*}
and it is thus a non-Coulombian type of potential. We also notice that
\begin{equation*} \mathrm{SU}_L(2, \, \C) \implies \text{the symmetry is broken (left only).} \end{equation*}
There are different possible quarks, and, more precisely, we have
\begin{equation*} \begin{aligned} & \begin{pmatrix} u \\ d \end{pmatrix}_L, \quad \begin{pmatrix} c \\ s \end{pmatrix}_L, \quad \begin{pmatrix} t \\ b \end{pmatrix}_L \sim \text{$\underline{3} \times ( \underline{2}, \, \frac{1}{3} ) \times \begin{pmatrix} \frac{2}{3} \\[0.8em] - \frac{1}{3} \end{pmatrix}$ in $\mathfrak{G}$},
\\[1em] & \begin{pmatrix} u \\ d \end{pmatrix}_R \sim \text{$\underline{3} \times ( \begin{pmatrix} \underline{1} \\ \underline{1} \end{pmatrix}, \, \begin{pmatrix} \frac{4}{3} \\[0.8em] - \frac{2}{3} \end{pmatrix} ) \times \begin{pmatrix} \frac{2}{3} \\[0.8em] - \frac{1}{3} \end{pmatrix}$ in $\mathfrak{G}$}, \end{aligned} \end{equation*}
and, clearly, there are some differences between left and right. It also turns out that
\begin{equation*} \begin{aligned} & \begin{pmatrix} \nu_e \\ e^- \end{pmatrix}_L, \quad \begin{pmatrix} \nu_\mu \\ \mu^- \end{pmatrix}_L, \quad \begin{pmatrix} \nu_\tau \\ \tau^- \end{pmatrix}_L \sim \text{$\underline{3} \times \underline{2} \times (-1) \implies Q_{em} = -1$},
\\[1em] & e_R^-, \quad \mu_R^-, \quad\tau_R^- \sim \text{$\underline{1} \times \underline{0} \times (-2) \implies Q_{em} = -1$},
\\[1em] & (\nu_e)_R, \quad (\nu_\mu)_R, \quad (\nu_\tau)_R \sim \text{$\underline{1} \times \underline{1} \times 0 \implies Q_{em} = 0$}. \end{aligned} \end{equation*}
Notice that we thought that neutrino did not have any mass, but later (neutrino oscillations) we found that the mass is different from zero.

The fermions' mass mixes up the left (L) and right (R), and it is not separable. The mass of an elementary particle given by condensation is called \textit{Yukawa mass}, and we have that
\begin{equation*} \mathscr{L}_Y = g_Y (\bar{\psi}_L)(\phi) \psi_R + \dots \end{equation*}
Notice that the masses of the elementary particles are not all given as a result of the Higgs mechanism; for example, we do not know it yet for the neutrino. Even if the (R) neutrino $\nu_R$ does not exist, the (L) neutrino does, and it turns out that
\begin{equation*}m_L ( \psi^\prime \psi^2 - \psi^2 \psi^\prime) = 2 m_L \psi^\prime \psi^2 \neq 0, \end{equation*}
which means that the mass $m_L$ of the (L) neutrino should be different from zero. A similar argument works for the (R) neutrino $\nu_R$, and the Higgs mechanics (may..) gives us the second addendum of
\begin{equation*}m_R\nu_R \nu _R + m_D \bar{\nu}_R \nu_L + hc, \end{equation*}
while the first comes out from the fact that $\nu_R$ is unbiased with respect to everything else.