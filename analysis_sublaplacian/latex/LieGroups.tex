\chapter{Lie Groups and Lie Algebras} \thispagestyle{empty}
\label{chapter:Liegroups}

In this chapter, we introduce the notion of {\em Lie group} and set the ground for Fourier analysis on this class of spaces generalizing what we did on $\R^n$.

\section{Introduction to Lie groups}

\bd[Lie group] \index{Lie group}
A {\em Lie group} $\G$ is an abstract group and a smooth manifold in which the two structures are compatible. In other words, the map
\[
\G \times \G \ni (x, \, y) \longmapsto y^{-1} \cdot x \in \G
\]
is smooth.
\ed

\brmk
Notice that the compatibility between the two structures is also equivalent to requiring that
\[
\G \times \G \ni (x, \, y) \longmapsto x \cdot y \in \G \quad \text{and} \quad \ni \G y \longmapsto y^{-1} \in \G
\]
are both smooth maps.
\ermk

\brmk
In some books, a Lie group is defined as an analytic manifold rather than a smooth one. However, it can be proved that in this context smooth implies analytic - see, for example, \cite{serre1}.
\ermk

From now on, the symbol $\G$ will always indicate a Lie group equipped with a multiplication law $\cdot$ compatible with the smooth manifold structure.

\bd\index{Lie group!left-translation}\index{Lie group!right-translation}
Fix $a \in \G$. The {\em left-translation} operator $\ell_a : \G \to \G$ is defined by setting
\[
\ell_a(x) := a \cdot x
\]
and, similarly, the {\em right-translation} operator $r_a : \G \to \G$ is defined by
\[
r_a(x) := x \cdot a^{-1}.
\]
Notice that we use $a$ for $\ell_a$ and $a^{-1}$ for $r_a$ in order to make the composition laws coherent with each other, that is,
\[
\ell_a \ell_b = \ell_{a \cdot b} \qquad \text{and} \qquad r_a r_b = r_{a \cdot b}.
\]
\ed

\brmk
It is easy to verify that both $\ell_a$ and $r_a$ are diffeomorphisms of $\G$ with inverses $\ell_{a^{-1}}$ and $r_{a^{-1}}$ respectively.
\ermk

\begin{notation}
Let $f \in C^\infty(\G)$. We denote by $\check{f}$ the function
\[
\check{f}(x) := f(x^{-1}) \in C^\infty(\G).
\]
In a similar fashion, we introduce the translation operators $L_a, \, R_a : C^\infty(\G) \to C^\infty(\G)$, for $a \in \G$, as follows:
\[ \begin{aligned}
& L_a f(x) := f(a^{-1} \cdot x) = f \circ \ell_a^{-1} (x),
\\[1em] & R_a f(x) := f(x \cdot a) = f \circ r_a^{-1} (x).
\end{aligned}\]
\end{notation}

Recall that Lie group $\G$ is a smooth manifold and therefore the notion of tangent space at some $p \in \G$ is well-defined, for example, through derivations. Recall that
\[
v : C^\infty(\G) \longrightarrow \R
\]
is a {\em derivation} at $p \in \G$ if the following properties hold: \mbox{}
\begin{enumerate}[label=\textbf{(\alph*)}]
\item If $f \equiv g$ in a neighbourhood of $p$, then $v(f) = v(g)$.
\item The {\bf Leibniz rule} holds, that is,\index{Leibniz rule!derivation}
\[
v(fg) = f(p) v(g) + g(p) v(f).
\]
\end{enumerate}

\bpr
Let $\varphi : A \subseteq \R^{q} \to \G$ be a coordinate system around $p_0 \in \G$ and assume that $\varphi(0) = p_0$. Then $v \in T_{p_0}\G$ if and only if there exists $(a_1, \, \dots, \, a_q) \in \R^q$ such that
\[
v(f) = \sum_{j = 1}^q a_j \partial_{x_j}(f \circ \varphi)(0).
\]
In other words, a derivation belongs to the tangent space if and only if it is a directional derivative.
\epr

\bd[Vector field] \index{vector field}
A {\em vector field} $X$ defined on a Lie group $\G$ is a linear operator
\[
X : C^\infty(\G) \longrightarrow C^\infty(\G)
\]
that satisfies the Leibniz rule, that is,
\[
X(fg) = X(g) f + X(f) g.
\]
\ed

We denote by $\X(\G)$ the space of all vector fields defined on $\G$ which is easily seen to be infinite-dimensional. Furthermore, for $p \in \G$ we have
\[
f \longmapsto Xf(p) \in T_p \G,
\]
which means that a vector field $X$ can be used to associate functions to tangent vectors.

\bd \index{vector field!left-invariant}
The vector field $X \in \X(\G)$ is {\em left-invariant} if
\[
X(L_a f) = L_a (X f) \quad \text{for all $f \in C^\infty(\G)$ and all $p \in \G$}.
\]
\ed

\bex
In $\R^n$, the vector field defined by setting
\[
X= \sum_{j = 1}^n a_j(x) \frac{\partial}{\partial x_j}
\]
is left-invariant if and only if $a_j(x) \equiv a_j$ is a constant function for all $j \in \{1, \, \dots, \, n\}$.
\eex

\bpr
Let $X \in \X(\G)$ be a left-invariant vector field and let "$v = X(e)$" be the derivation defined by setting
\[
v(f) = Xf(e).
\]
Then, for all $x \in \G$ and for all $f \in C^\infty(\G)$, the vector field satisfies
\[
X f(x) = v( L_x^{-1} f).
\]
Conversely, given $v \in T_e \G$, the vector field defined by setting
\[
X_v f(x) := v(L_{x}^{-1} f)
\]
is left-invariant.
\epr

\begin{proof}
The first identity is trivial since
\[
X f(x) = L_{x^{-1}} (Xf)(e) = X (L_{x^{-1}}f)(e) = v(L_x^{-1} f).
\]
To prove that $X_v$ is a vector field, we need to check\footnote{It is clearly sufficient to prove that $g$ is smooth in a neighbourhood of $x$.} that $g := X_v f \in C^\infty(\G)$. A straightforward computation shows that
\[
X_v f(x) = v(L_{x}^{-1} f) = \sum_{j = 1}^n a_j \partial_{x_j}( L_x^{-1} f \circ \varphi) (0) = \sum_{j = 1}^n a_j \partial_{x_j} f(x \cdot \varphi(t)),
\]
which means that $g$ is smooth since we can write it as the composition of smooth functions. We now show that $X_v$ is left-invariant. Let $a \in \G$ and notice that
\[ \begin{aligned}
X_v(L_a f)(x) & = v( L_x^{-1} L_a f) =
\\[1em] & = v(L_{x \cdot a^{-1}} f) =
\\[1em] & = X_v f(a^{-1} \cdot x) =
\\[1em] & = L_a X_v(f)(x).
\end{aligned} \]
\end{proof}

It follows from the proposition that $v \mapsto X_v$ is a bijection between $T_e \G$ and the subset of $\X(\G)$ that consists of left-invariant vector fields.

\bd[Lie Algebra] \index{Lie algebra}
The {\em Lie algebra} $\cg$ associated to a Lie group $\G$ is the algebra generated by left-invariant vector fields,
\[
\cg := \{ X \in \X(\G) \: : \: \text{$X$ left-invariant}, \},
\]
and it is a vector space of dimension $q$.
\ed

\brmk
If $X, \, Y \in \cg$, then $[X, \, Y] \in \cg$.
\ermk

The algebra isomorphism between $\cg$ and $T_e \G$ is given by the map $v \mapsto X_v$, where $X_v$ can also be defined via the push-forward
\[
X_v(x) = \ell(x)_\ast v.
\]

\section{Exponential map and one-parameter groups}

Let $X \in \cg$ be a left-invariant vector field and let $\gamma_x$ be the integral curve originating at some point $x \in \G$, namely the solution of the problem
\[\begin{cases}
\gamma_x'(t) = X(\gamma_x(t)),
\\[.6em] \gamma_x(0) = x.
\end{cases}\]
We now claim that for all $x\in \G$ the following identity holds:
\begin{equation} \label{eq.z.1}
\gamma_x(t) = x \cdot \gamma_e(t).
\end{equation}
One way to see this is to use the exponential map. Recall that by definition
\[
\exp(tX) f(e) = f(\gamma_e(t)),
\]
so for any $x \in \G$ it turns out that
\[ \begin{aligned}
\exp(tX) L_{x^{-1}} f(e) & = L_{x^{-1}} \exp(tX) f(e)
\\[1em] & =  L_{x^{-1}} \left[ f(\gamma_e(t)) \right]
\\[1em] & = f(x \cdot \gamma_e(t)).
\end{aligned} \]
The left-hand side is equal to $[L_{x^{-1}} f](\gamma_e(t))$, and thus \eqref{eq.z.1} holds if we can prove that
\[
[L_{x^{-1}} f](\gamma_e(t)) = f(\gamma_x(t)).
\]
This is left as an exercise to the reader to get acquainted with the notions introduced so far.

\brmk
Another way to prove \eqref{eq.z.1} is to show that $x \cdot \gamma_e$ is the integral curve of $X$ starting from $x$. This is an immediate consequence of $X$ left-invariant vector field:
\[ \begin{aligned}
\frac{\mathrm{d}}{\mathrm{d}t} f(x \cdot \gamma_e(t)) & = \frac{\mathrm{d}}{\mathrm{d}t} L_{x^{-1}} f(\gamma_e(t))
\\[1em] & =X(L_{x^{-1}} f)(\gamma_e(t))
\\[1em] & = (Xf)(x \cdot \gamma_e(t)).
\end{aligned} \]
\ermk

The identity \eqref{eq.z.1} is extremely important and carries several consequences, one of which is the existence of one-parameter groups. First, notice that
\[
\text{$\gamma_e(\cdot)$ defined for $t \in (- \epsilon, \, \epsilon) \implies \gamma_x(\cdot)$ defined for $t \in (- \epsilon, \, \epsilon)$}.
\]
In particular, we have $\gamma_{\gamma_e(s)}(t) = \gamma_e(s) \cdot \gamma_e(t)$ and, since we know already (via the composition of the corresponding flows) that
\[
\gamma_{\gamma_e(s)}(t)= \gamma(t + s),
\]
we can infer that $\gamma_e(t + s) = \gamma_e(t) \cdot \gamma_e(s)$. This means that we can extend the domain of $\gamma_e$ to coincide with the real line, and hence
\[
\gamma_e \in C^\infty(\R, \, \G)
\]
is a {\em smooth groups homomorphism} from $\R$ to $\G$. In this case we say that $\gamma_e$\index{one-parameter group} is a {\em one-parameter group} and we can easily define a one-to-one correspondence
\[
\cg \longleftrightarrow \{ \text{one-parameter groups} \}
\]
where $\gamma$, a one-parameter group, gives a left-invariant vector field by setting
\[
Xf(x) = \frac{\mathrm{d}}{\mathrm{d}t} f(x \cdot \gamma(t)).
\]

\bex
The one-parameter groups of $(\R^n, \, +)$ are all of the form
\[
\gamma_v(t) := tv,
\]
where $v \in \R^n$. Notice that from a topological point of view $\gamma$ might not be a {\bf closed} subgroup (e.g., the irrational line in $\mathbb{T}$ which is dense)
\eex

Let $\G$ be a Lie group and identify its Lie algebra $\cg$ with the tangent space. 

\bd\index{Lie group!exponential map}
Fix $v \in T_e \G$ and let $\gamma_v$ be the one-parameter group satisfying the initial condition $\gamma_v^\prime(0) = v$. We call {\em exponential of $\G$} the mapping
\[
\exp_\G(v) := \gamma_v(1).
\]
\ed

\bthm
The exponential map $\exp_\G$ is a local diffeomorphism from a neighbourhood $U$ of $0 \in T_e \G$ to a neighbourhood $V$ of the identity element $e\in \G$.
\ethm

\begin{proof}
Simply notice that $\exp_\G$ is smooth (since it is the restriction of $(t, \, v) \mapsto \gamma_v(t)$ which is smooth) and
\[
\mathrm{d}(\exp_\G)_0 : T_e \G \longrightarrow T_e \G
\]
is the identity map since $\gamma_v^\prime(0)=v$ by definition.
\end{proof}

One might wonder whether or not $\exp_\G$ is a {\bf global} diffeomorphism. However, it is easy to verify that any compact Lie group gives a counterexample since
\[
T_e \G \cong \R^q
\]
is only locally compact so it cannot be diffeomorphic to a compact manifold such as $\G$. For example,
\[
\mathrm{SU}(2, \, \C) \cong S^3 \subset \C^2
\]
is a compact Lie group and $\exp_\G$ is not a global diffeomorphism as one can easily see from the picture below.

\begin{figure}[h!]
\centering
\includegraphics[width = 12.5cm, height = 2.5cm]{images/AA2.pdf}
\caption{Counterexamples to $\exp_\G$ global diffeomorphism.}
\label{fig:z1}
\end{figure}

Furthermore, notice that for $v$ and $w$ in a small neighbourhood of $0 \in T_e\G$ we find that there exists $u$ in the same neighbourhood such that
\[
(\exp_\G v)\cdot(\exp_\G w) = \exp_\G u,
\]
and $u$ is given by the \textsc{BCH} formula. In particular, the series
\[
v + w + \sum_{k_1 + k_2 \geq 1} c_{k_1, \, k_2}, \quad  c_{k_1, \, k_2}\in \mathcal{F}^{k_1, \, k_2}
\]
is convergent in a small neighbourhood of the origin and this is what (more or less) leads to the definition of Lie groups with "analytic" in place of "smooth".

\bex
Let $\G = \R$ and let $\G^\prime = \mathbb{T}$. Then it is easy to verify that
\[
\cg \cong \cg^\prime,
\]
and both are isomorphic to $\R$.
\eex

\brmk
This shows that two Lie groups with the same (up to isomorphism) Lie algebra are not necessarily isomorphic themselves; however, they are locally isomorphic.
\ermk

\bthm
Two Lie groups are locally isomorphic if and only if their Lie algebras are isomorphic.
\ethm

\brmk
What we did until now only cares about the connected component of $\G$ and ignores all the others as if they did not exist. Therefore, assuming that $\G$ is connected is not restrictive in any sense.
\ermk

\section{Lie algebras and connection with Lie groups}

We now introduce the notion of Lie algebra and show how is it related with the notion of Lie groups via left-invariant vector fields.

\bd \index{Lie algebra}
A Lie algebra $\cg$ is a vector space endowed with a bilinear operator $[ \cdot, \, - ] : \cg \times \cg \to \cg$ which is antisymmetric and satisfies the Jacobi identity\index{Jacobi identity}
\begin{equation} \index{Jacobi identity}
[x, \, [y,\,z]] + [z, \, [x,\,y]] + [y, \, [z,\,x]] = 0 \quad \text{for all $x, \, y, \, z \in \cg$}.
\end{equation}
\ed

\bex
The vector space of $n \times n$ real-valued matrices endowed with the bracket operator $[A,\, B] := AB - BA$ is a Lie algebra which is usually denoted by $\mathfrak{gl}(n, \, \R)$. Similarly,
\[
\mathfrak{sl}(n, \, \R) = \left\{ A \in \mathfrak{gl}(n, \, \R) \: : \: \mathrm{Tr}(A) = 0 \right\}
\]
and
\[
\mathfrak{so}(n, \, \R) = \left\{ A \in \mathfrak{gl}(n, \, \R) \: : \: \text{$A$ is skew-symmetric} \right\}
\]
are also Lie algebras with the same bracket operator. Notice that, in this case, the space of symmetric matrices in $\mathfrak{gl}(n, \, \R)$ is {\bf not} a Lie algebra.
\eex

\bex
The Euclidean space $\R^3$ with the multiplication law given by the wedge product $\wedge$ is a Lie algebra which is isomorphic to $\mathfrak{so}(3, \, \R)$ via
\[
\R^3 \ni (v_1, \, v_2, \, v_3) \longmapsto \begin{pmatrix} 0 & v_3 & v_2 \\ - v_3 & 0 & - v_1 \\ - v_2 & v_1 & 0 \end{pmatrix} \in \mathfrak{so}(3, \, \R).
\]
\eex

\bd[Homomorphism] \index{Lie algebra!homomorphism}
A Lie algebras homomorphism $\varphi$ is a linear mapping that preserves the bracket, that is,
\[
[x,\,y] = [\varphi(x), \, \varphi(y)].
\]
\ed

\bd\index{Lie algebra!commutative}
We say that $\cg$ is a {\em commutative} Lie algebra if $[x, \, y] = 0$ for all $x, \, y \in \cg$.
\ed

We now introduce an useful notation for the commutator of linear spaces. More precisely, let $\mathfrak{h}$ be a linear subspace of a Lie algebra $\cg$. Then
\[
[\cg, \, \mathfrak{h}] := \mathrm{Span} \langle [x, \, y] \: : \: x \in \cg, \, y \in \mathfrak{h} \rangle.
\]

\bd\index{Lie algebra!subalgebra}
Let $\cg$ be a Lie algebra. A linear subspace $\mathfrak{h}$ is a {\em Lie subalgebra} if it is closed under the bracket, that is,
\[
[\mathfrak{h}, \, \mathfrak{h}] \subseteq \mathfrak{h}.
\]
Moreover, we say that a Lie subalgebra $\mathfrak{h}$ is an {\em ideal}\index{Lie algebra!ideal} if
\[
[\mathfrak{h}, \, \cg] \subseteq \mathfrak{h}.
\]
\ed

\brmk
If $\mathfrak{h}$ is an ideal of $\cg$, then the quotient can be given a Lie algebra structure by setting
\[
[x + \mathfrak{h}, \, y + \mathfrak{h}] := [x, \, y] + \mathfrak{h}.
\]
Notice that this is not well-defined (i.e., it depends on the choice of the representatives) if $\mathfrak{h}$ is a subalgebra which is not an ideal.
\ermk

\bd \index{Lie algebra!centre}
Let $\cg$ be a Lie algebra. The {\em centre} of $\cg$ is defined as
\[
Z_\cg := \{ x \in \cg \: : \: [x, \, \cg] = 0 \}.
\]
\ed

\brmk
The linear subspace $Z_\cg$ is an ideal of $\cg$.
\ermk

\bd \index{Lie algebra!derived algebra}
Let $\cg$ be a Lie algebra. The {\em derived algebra} of $\cg$ is defined as
\[
\cg' := [\cg, \, \cg].
\]
\ed

The derived algebra of $\cg$ does not always coincide with $\cg$ itself. For example, if $\cg = \mathfrak{gl}(n, \, \R)$ it is easy to verify that
\[
[\cg, \, \cg] \subseteq \mathfrak{sl}(n, \, \R),
\]
and thus it is a proper ideal of $\cg$.

\bex
Let $\mathfrak{t}(n,\,\R)$ be the space of $n\times n$ real-valued upper-triangular matrices. Then $\mathfrak{t}(n,\,\R)$ is a Lie algebra because $AB, \, BA \in \mathfrak{t}(n,\,\R)$, but
\[
[\mathfrak{t}(n,\,\R), \, \mathfrak{t}(n,\,\R)] \subset \mathfrak{t}(n,\,\R)
\]
is a proper ideal because all its elements have the super-diagonal identically equal to zero.
\eex

The previous example suggests that we can iterate this construction a finite number of times. We obtain a sequence of derived algebras which are proper ideals of one another until
\[
[ \mathfrak{t}(n,\,\R), \, [\mathfrak{t}(n,\,\R), \, [\mathfrak{t}(n,\,\R), \, \dots, \, [\mathfrak{t}(n,\,\R), \, \mathfrak{t}(n,\,\R)]]\dots] = 0.
\]
This construction is what is usually called {\em descending central series} and can be done with any other Lie algebra, although there is no guarantee that it will eventually be zero.\index{descending central series}

\bd[Nilpotent] \index{Lie algebra!nilpotent}
Let $\cg$ be a finite-dimensional Lie algebra. If the descending central series stabilises at $0$, then we say that $\cg$ is {\em nilpotent}.
\ed

\bex
Let $\cg$ be the Heisenberg Lie algebra, namely the one generated as
\[
\cg = \mathrm{Span} \langle \partial_x - \frac{y}{2} \partial_z, \, \partial_y + \frac{x}{2} \partial_z, \, \partial_z \rangle.
\]
Then the derived algebra is $\cg' = \mathrm{Span} \langle \partial_z \rangle = \R$ and it is easy to verify that the $\cg$ is nilpotent of step two.
\eex

\bthm[Ado] \index{Ado's theorem}
Every finite-dimensional Lie algebra over $\R$ (or $\C$) is isomorphic to a matrices Lie algebra.
\ethm

The original proof of this result due to Ado can be found in \cite{ados}. There are several others with new methods, but they are all reasonably complicated so we will just skip it.

\bthm[Engel] \index{Engel's theorem}
Every finite-dimensional nilpotent Lie algebra over $\R$ (or $\C$) is isomorphic to a upper-triangular matrices Lie algebra.
\ethm

\bex[Heisenberg]
Let $\cg$ be the Heisenberg space, namely
\[
\cg = \mathrm{Span} \langle \partial_x - \frac{y}{2} \partial_z, \, \partial_y + \frac{x}{2} \partial_z, \, \partial_z \rangle.
\]
It is easy to see that this is isomorphic to the matrices algebra generated by
\[
\begin{pmatrix} 0 & x & z \\ 0 & 0 & y \\ 0 & 0 & 0 \end{pmatrix},
\]
where
\[
X = \begin{pmatrix} 0 & 1 & 0 \\ 0 & 0 & 0 \\ 0 & 0 & 0 \end{pmatrix}, \quad Y = \begin{pmatrix} 0 & 0 & 0 \\ 0 & 0 & 1 \\ 0 & 0 & 0 \end{pmatrix}, \quad Z = \begin{pmatrix} 0 & 0 & 1 \\ 0 & 0 & 0 \\ 0 & 0 & 0 \end{pmatrix}.
\]
\eex

The next theorem asserts that finite-dimensional Lie algebras are always given by $\mathrm{Lie}(\G)$, where $\G$ is a Lie group which is uniquely determined if we require connected and simply connected.

\bthm[Lie] \index{Lie's theorem}
Let $\cg$ be a finite-dimensional real Lie algebra. Then there exists a Lie group $\G$ such that $\cg$ i its Lie algebra. Furthermore, there exists a unique $\G$ which is connected and simply connected.
\ethm

One might wonder if the proof of this result is easier if we use Ado's theorem to restrict ourselves to a smaller class of Lie algebras. However, this is not the case. Let $\cg$ be a Lie algebra contained in $\mathfrak{gl}(n, \, \R)$ and notice that
\[
\mathfrak{gl}(n, \, \R) = \mathrm{Lie}( \mathrm{GL}(n, \, \R) ), 
\]
the set of all matrices with nonzero determinant. Now $\exp_\G(\cg)$ is a "nice" submanifold close to the identity, but we get no information about the behaviour of faraway points.

\bthm
Let $\cg$ be a Lie algebra and let $\G$ be the unique connected and simply connected Lie group with $\mathrm{Lie}(\G) \cong \cg$. If $\mathbb{H}$ is a connected Lie group with $\mathrm{Lie}(\mathbb{H}) \cong \cg$, then $\mathbb{H}$ is isomorphic to a quotient of $\G$ modulo a discrete central subgroup.
\ethm

\bex
For example, $\R$ is the unique connected and simply connected Lie group with Lie algebra isomorphic to $\R$. The torus $\mathbb{T}$ is, in fact, isomorphic to a quotient of $\R$:
\[
\mathbb{T} = \faktor{\R}{2 \pi \Z}.
\]
\eex

\bex
Let us consider
\[
\mathrm{SU}(2,\, \C) = \left\{ \begin{pmatrix} z_1 & - \bar{z_2} \\ z_2 & \bar{z_1} \end{pmatrix} \: : \: |z_1|^2 + |z_2|^2 = 1 \right\}
\]
and
\[
\mathrm{SO}(3,\, \R) = \left\{ R_{\theta, \, v} \: : \: \theta \in [0, \, 2 \pi), \, v \in \R^3, \, |v|=1 \right\}.
\]
It is easy to verify that $\mathrm{SU}(2,\, \C)$ is diffeomorphic to $S^3 \subset \C^2$ so it is connected and simply connected, while $\mathrm{SO}(3,\, \R)$ is connected but not simply connected. It can be proved that
\[
\mathfrak{su}(2, \, \C) \cong \mathfrak{so}(3, \, \R),
\]
and since the center of $\mathrm{SU}(2,\, \C)$ is $\{ \pm \mathrm{Id} \}$, then the unique possibility is that
\[
\mathrm{SO}(3, \, \R) = \faktor{\mathrm{SU}(2,\, \C)}{ \{ \pm \mathrm{Id} \}},
\]
which means that $\mathrm{SU}(2,\, \C)$ is a double covering of $\mathrm{SO}(3,\, \R)$.
\eex

\paragraph{Notation.} So far we introduced an exponential map which sends a vector field to its flow,
\[
\exp(tX) f(x) = f \circ \varphi_t(x),
\]
and an exponential map associated to a Lie group $\G$ that sends $\cg$ to $\G$. The connection between them follows easily if we identify $X_v \in \cg$ with $X_v \in T_e \G$ since
\[
\exp(t X_v) f(x) = f(x \exp_\G(tv)).
\]
Since Ado's theorem asserts that every finite-dimensional Lie algebra $\cg$ over $\R$ is isomorphic to a matrices Lie algebra, it only makes sense to study properties of $\mathrm{GL}(n, \ \R)$.

\bex
Recall that $\gln$ is a dense open set of $\R^{n \times n}$ so it quite clearly a Lie group. The tangent space is
\[
T_e \gln = \mathrm{M}(n, \, \R),
\]
the set of all $n \times n$ matrices, isomorphic to $\R^{n^2}$. Recall that for any $A \in \mathrm{M}(n, \, \R)$ the exponential
\[
\e^A = \sum_{n \in \N} \frac{A^n}{n!}
\]
is well-defined and converges absolutely since
\[
\left\| \sum_{n \in \N} \frac{A^n}{n!} \right\| \leq \sum_{n \in \N} \frac{\|A^n\|}{n!} \leq \sum_{n \in \N} \frac{\|A\|^n}{n!} = \e^{\|A\|}
\]
and the exponential of a real number is well-defined. Now let $\gamma_A(t) := \e^{At}$ and notice that this map satisfies the relations
\[
\gamma_A(s+t) = \gamma_A(t) \cdot \gamma_A(s) \quad \text{and} \quad \gamma_A(t) \cdot \gamma_A(-t) = \mathrm{id},
\]
so $\gamma_A(t)$ belongs to $\gln$ for all $A \in \mathrm{M}(n, \, \R)$. We can be more precise and prove that
\[
\mathrm{det} \, \e^A = \e^{\mathrm{Tr}(A)},
\]
which follows from the Jordan's normal form in $\C$ together with the obvious identity $\e^{PAP^{-1}} = P \e^A P^{-1}$, but we do not need to go as far. Anyway, $\gamma_A$ is a one-parameter group and its derivative is given by
\[
\gamma_A^\prime(t) = A \e^{At}.
\]
Now we want to prove that, given left-invariant vector fields $X_A$ and $X_B$ via the exponential identification, it turns out that
\[
[X_A, \, X_B] = X_{[A, \, B]},
\]
where $[A, \, B] = AB - BA$. Start by observing that
\[ \begin{aligned}
X_A f(x) & = \frac{\mathrm{d}}{\mathrm{d}t} \, \Big|_{t = 0} f(x \exp_\G(A)) = 
\\[1em] & =  \frac{\mathrm{d}}{\mathrm{d}t} \, \Big|_{t = 0} f(x \left( \mathrm{id} + tA + \mathcal{O}(t^2) \right) ) =
\\[1em] & =  \frac{\mathrm{d}}{\mathrm{d}t} \, \Big|_{t = 0} f(x + x tA) =
\\[1em] & =  \frac{\mathrm{d}}{\mathrm{d}t} \, \Big|_{t = 0} f(x_1 + t (xA)_1, \, \dots, \, x_n + t(xA)_n) =
\\[1em] & = \sum_{i, \, j = 1}^n (xA)_{i, \, j} \partial_{x_{i, \, j}} f(x)
\end{aligned} \]
from which it follows that
\[
X_A(e) = \sum_{i, \, j = 1}^n A_{i, \, j} \partial_{x_{i, \, j}}.
\]
If $B$ is another element, it is easy to see that
\[ \begin{aligned}
[X_A, \, X_B] f(e) & = X_A(X_B f) - X_B(X_A f) =
\\[1em] & = \sum_{i, \, j} A_{i, \, j} \partial_{i, \, j} \left( \sum_{k, \, \ell} (xB)_{k, \, \ell} \partial_{k,\, \ell}\right) f(e) - \sum_{i, \, j} B_{i, \, j} \partial_{i, \, j} \left( \sum_{k, \, \ell} (xA)_{k, \, \ell} \partial_{k,\, \ell}\right) f(e)
\\[1em] & = \sum_{i, \, j} A_{i,\, j} \sum_{k, \, \ell} \left[ \partial_{i, \, j} ( xB)_{k,\, \ell} \partial_{k, \, \ell} \right] f(e)
\\[1em] &  = \sum_{i, \, j} \sum_\ell \left[ B_{j, \, \ell} \partial_k f \right] (e) 
\\[1em] & = \sum_{i, \, \ell} (AB)_{i, \, \ell} \partial_{i, \, \ell} f(e) = X_{[A, \, B]}f(e)\end{aligned} \]
The Lie algebra $\mathfrak{gl}(n, \, \R)$ can be identified with $\mathrm{M}(n, \, \R)$ and the exponential map $A \mapsto \e^A$, while nice close to zero, is far from being injective everywhere. Indeed,
\[
\exp \left( t \begin{pmatrix} 0 & 1 \\ -1 & 0 \end{pmatrix} \right) = R_t,
\]
the rotation of angle $t$, is periodic with respect to $t$.
\eex

\section{Nilpotent Lie algebras}

In this section, we take a closer look to nilpotent Lie algebras and Lie groups since in the final part of the course we will study the sublaplacian operator on a special subclass of nilpotent groups.

\bthm \label{thm.algebraassocia}
Let $\cg$ be a nilpotent Lie algebra of step $m$ and define
\[
S(a, \, b) = a + b + \frac{1}{2}[a, \, b] + \sum_{k = 3}^m c_k(a, \, b).
\]
Then $S$ is a multiplication law defining a Lie group structure on $\cg$ which Lie algebra is isomorphic to $\cg$.
\ethm

\begin{proof}
First, notice that $S(a, \, 0) = a$ and $S(0, \, b) = b$ so the linear space $0$ plays the role of the identity in $\G$. Moreover,
\[
S(a, \, - a) = 0,
\]
so each element $a \in \G$ has an inverse which is given by $-a$. The associative property,
\[
S(S(a, \, b),\, c) = S(a, \, S(b,\,c),
\]
on the other hand, is difficult to prove. We use Ado's theorem and reduce to a simpler case since working on a manifold $\cong \R^N$ allows one to assume the simple connectedness.

\paragraph{Associativity.} The following result is given for granted:

\bl
There exists a neighbourhood of $0$ in $\mathfrak{gl}(n, \, \R)$ such that for all $A, \, B \in U$ the Baker-Campbell-Hausdorff formula converges.
\el

Let $U$ be a neighbourhood of $0$ in $\mathfrak{gl}(n, \, \R)$ such that the restriction of the exponential map is a diffeomorphism with an open neighbourhood of the identity matrix $V$. Select a smaller neighbourhood $V'$ in such a way that
\[
V' \cdot V' \subset V,
\]
and, similarly, select an even smaller one such that
\[
V'' \cdot V'' \subset V'.
\]
Let $U'$ and $U''$ be the preimages of $V'$ and $V''$. We can assume that $U'$ satisfies the assumption of the lemma above (otherwise we can take a smaller one). It follows that
\[
\forall a, \, b \in U', \, \exp(S(a, \, b)) \in V.
\]
Now let $a, \, b, \, c \in U^{''}$. It is easy to see that $S(a,\, b), \, S(a, \, c), \, S(b, \, c) \in U'$ and thus
\[
\e^{S(S(a, \, b), \, c)} = \e^{S(a, \, b)} \e^c = \e^a \e^b \e^c = \e^a \e^{S(b, \, c)} = \e^{S(a, \, S(b, \, c))}.
\]
This shows that associativity rule holds in $U''$ and since this is a polynomial identity that holds for small parameters we can extend it everywhere by analytic continuation (since there is a finite number of addenda).
\end{proof}

\section{Homomorphisms of Lie algebras and dilations}

Let $\Psi : \G \to \G'$, $\G$ connected, be a {\em homomorphism of Lie groups}\index{Lie group!homomorphism}, that is, a smooth\footnote{Requiring measurable here leads, surprisingly, to an equivalent definition.} map which is also a group homomorphism. Then
\[
\mathrm{d}\Psi_e : T_e \G \to T_e \G'
\]
induces a map $\Psi_\ast : \cg \to \cg'$ composing with the isomorphisms described earlier in the chapter that identify the Lie algebra with the tangent at the identity element. Let $v \in T_e \G$ and let $\gamma_v(t)$ be the relative one-parameter group ($\gamma_v^\prime(0) = v$). Then
\[
\gamma_{v'}(t) := \Psi \circ \gamma_v(t)
\]
is a one-parameter group in $\G'$ and hence turns out that $\Psi_\ast(v) = v'$.

\bpr
The map $\Psi_\ast$ is a Lie-algebra isomorphism, namely
\[
\Psi_\ast [u,\, v] = [\Psi_\ast u,\, \Psi_\ast v].
\]
\epr

\begin{proof}
Let $v \in \cg$. Then $\gamma_v(t) = \exp_\G(tv)$ and
\[
X_v f(x) = \frac{\rmd}{\rmd t} \, \Big|_{t = 0} f(x \cdot \exp_\G(tv)).
\]
is the corresponding left-invariant vector field. Let $h \in C^\infty(\G')$, $f := h \circ \Psi \in C^\infty(\G)$, and notice that
\[ \begin{aligned}
X_v f(x) & = \frac{\rmd}{\rmd t} \, \Big|_{t = 0} h \left( \Psi(x \cdot \exp_\G(tv)) \right)
\\[1em] & = \frac{\rmd}{\rmd t} \, \Big|_{t = 0} h \left( \Psi(x) \cdot \exp_{\G'}(tv') \right)
\\[1em] & = Y_{v'} h(\psi(x)) = Y_{v'} f(x),
\end{aligned} \]
which means that $X_v = Y_{\Psi_\ast(v)}$, where $Y_w$ is the left-invariant vector field relative to $w$ in the Lie algebra associated to $\G'$.
\end{proof}

\bpr
If $\G$ is connected, then $\Psi_\ast$ uniquely determines $\Psi$.
\epr

\begin{proof}
If $\Phi$ and $\Psi$ are two Lie groups homomorphisms with $\Phi_\ast = \Psi_\ast$, then
\[
\Phi(\exp_\G(tv)) = \exp_{\G'}( t \Phi_\ast v) = \exp_{\G'}( t \Psi_\ast v) = \Psi(\exp_\G(tv)).
\]
This means that $\Phi$ and $\Psi$ coincide in a neighbourhood $U$ of the identity element $e \in \G$, which we can assume to be symmetric. Then
\[
\Phi \equiv \Psi
\]
on each power $U^k$ of $U$, and hence we can consider $\mathcal{U} := \bigcup_{n \in \N} U^n$ subgroup of $\G$ since it is closed under multiplication and inverse. By definition, $\mathcal{U}$ is open and, using the identity
\[
\G \setminus \mathcal{U} = \bigcup_{p \notin \mathcal{U}} p \cdot \mathcal{U},
\]
we also infer that $\mathcal{U}$ is closed. Since $\G$ is compact the only possibility is $\mathcal{U} = \G$.
\end{proof}

\bcor
The map $\ast : \homs(\G,\, \G') \to \homs(\cg, \, \cg')$ is injective.
\ecor

\brmk
In general, it is not surjective. If $\G = \mathbb{T}^2$ and $\G' = \R$, then the unique Lie algebra homomorphism $\mathrm{id} : \R \to \R$ cannot be lifted to a map from $\mathbb{T}^2$ to $\R$.
\ermk

\bthm
Let $\G$ be a connected and simply connected Lie group. If $\eta \in \homs(\cg,\, \cg')$, then there exists $\Psi \in \homs(\G,\, \G')$ such that
\[
\Psi_\ast = \eta.
\]
\ethm

\brmk
If $\eta \in \homs(\cg,\, \cg')$ and $\cg$ is nilpotent, then the subalgebra $\eta(\cg) \subset \cg'$ is also nilpotent.
\ermk

\begin{proof}
We prove the result for $\G$ nilpotent. We proved that we can always assume $\G = \cg$ endowed with the Baker-Campbell-Hausdorff formula in place of the multiplication law. Notice that
\[
\G' = \faktor{\widetilde{\G'}}{D},
\]
so we can assume without loss of generality that $\G'$ is also connected and simply connected (up to replacing it with its universal covering). If $u,\, v \in \cg$, then
\[
u \cdot v = u + v + \frac{1}{2}[u,\,v] + \sum_{k \geq 3}^\iota c_k(u,\,v) =: S(u,\,v),
\]
and we can define
\[
\Psi(u) := \exp_{\G'}(\eta(u)).
\]
Then
\[ \begin{aligned}
\Psi(u) \cdot \Psi(v) & = \exp_{\G'}(\eta(u)) \exp_{\G'}(\eta(v))
\\[1em] & = \exp_{\G'}(S(\eta(u),\, \eta(v)))
\\[1em] & = \exp_{\G'}(\eta(S(u,\,v))) = \Psi(u\cdot v),
\end{aligned} \]
which means that $\Psi_\ast = \eta$.\end{proof}

\bd[Dilations] \index{dilation}
A one-parameter family of automorphisms $\{\eta_t\}_{t > 0} \subset \mathrm{Aut}(\cg)$ is a family of {\em dilations} if there exists a basis $\{e_1,\, \dots,\, e_q\}$ of $\cg$ and positive numbers $\lambda_1,\, \dots,\, \lambda_q \in \R_+$ such that
\[
\eta_t(e_j) = t^{\lambda_j} e_j.
\]
\ed

\bex
Consider $e_1,\, e_2$ orthonormal basis of $\R^2$ and define
\[
\eta_t(e_j) = t^{\lambda_j} e_j,
\]
where $\lambda_1 = 1$ and $\lambda_2 = 2$. The trajectories are half-parabolas as in the figure below:
\eex

\brmk
We can decompose $\cg$ in such a way that $\lambda_1 < \cdots < \lambda_k$ and
\[
\cg = \mathfrak{w}_1 \oplus \cdots \oplus \mathfrak{w}_k,
\]
where $\eta_t \, \big|_{\mathfrak{w}_j} = t^{\lambda_j} \mathrm{id}$ for each $j = 1,\, \dots, \, k$. Moreover
\[
\eta_{t \cdot s} = \eta_t \circ \eta_s,
\]
and since $\eta_t$ is an automorphism of the Lie algebra, we also have that
\[
\eta_t([u,\,v]) = [\eta_t u,\, \eta_t v].
\]
This property has an immediate consequence, namely if $u \in \mathfrak{w}_j$ and $v \in \mathfrak{w}_\ell$, then the commutator $[u,\,v]$ belongs to $\mathfrak{w}_{j + \ell}$ (which may also coincide with $\{0\}$).
\ermk

Clearly, $\lambda_j + \lambda_\ell > \max\{\lambda_j,\, \lambda_\ell\}$ so the existence of a family of dilations gives an upper bound on the number of commutators that one can take before it gives zero.

\bpr
A Lie algebra with a family of dilations is nilpotent.
\epr

\brmk
The opposite assertion is not true, but finding a counterexample requires a big effort since one needs commutators sufficiently complicated.
\ermk

\bex \label{ex.1.232.3}
Let $e_1,\, e_2,\, e_3$ be vectors in $\R^3$ such that $[e_1,\,e_2] = e_3$. Then
\[
\eta_t(e_1) = t^\lambda e_1 \quad \text{and} \quad \eta_t(e_2) = t^\mu e_2
\]
implies $\eta_t(e_3) = t^{\lambda + \mu} e_3$. These are the unique dilations for which the $e_i$'s are all eigenvectors.
\eex

\bd
If $\cg$ is a nilpotent Lie algebra with dilations $\delta_t$, then $\G = \cg$ endowed with the BCH formula has a family of automorphisms such that
\[
(\Psi_t)_\ast = \delta_t,
\]
which is usually referred to as {\em dilations of a Lie group}.
\ed

\section{Graded and stratified Lie algebras}

We start with the definition of a particular class of Lie algebras which is rather important for the consequences the decomposition bears.

\bd
Let $\cg$ be a nilpotent group. If there exists a vector space decomposition
\[
\cg = \cw_1 \oplus \cdots \oplus \cw_k
\]
such that
\[
[\cw_i, \cw_j] \subseteq \cw_{i+j},
\]
then we say that $\cg$ is a {\em graded Lie algebra}\index{Lie algebra!graded}. 
\ed

\brmk
Notice that a graded algebra is nilpotent, but it does not determine the step since, for example,
\[
\R^3 = \R \oplus \R \oplus \R = \R^2 \oplus \R.
\]
\ermk

\bl
If $\cg$ is a graded Lie algebra, the maps $\eta_t$ defined by setting
\[
\eta_t \, \big|_{\cw_j} := t^j \, \mathrm{Id}_{\cw_j}
\]
for $j = 1,\dots,k$ give a well-defined family of dilations on $\cg$.
\el

\bd\index{Lie algebra!stratified}
If $\cg$ is a graded Lie algebra generated by $\cw_1$, we say that $\cg$ is {\em stratified}.
\ed

\bex
The Heisenberg algebra described in \hyperref[ex.1.232.3]{Example \ref{ex.1.232.3}} is stratified if and only if $\lambda = \mu$ since we need both vectors $e_1$ and $e_2$ in $\cw_1$.
\eex

\bex
Let $\cg$ be the Lie algebra given by upper-triangular matrices. Then we can take $\mathfrak{w}_j$ to be the subalgebra generated by the matrices with {\bf only} the $j$th over-diagonal nonzero, and this makes $\cg$ a stratified algebra.
\eex

\bd[Sublaplacian] \index{sublaplacian}
Let $\cg$ be a stratified Lie algebra. If
\[
\mathfrak{w}_1 = \mathrm{Span} \langle e_1,\, \dots,\, e_m \rangle,
\]
then we can consider the differential operator
\[
- \Delta := - \sum_{j = 1}^m X_j^2.
\]
We say that $- \Delta$ is the {\em sublaplacian operator} on $\G$ since it is homogeneous of degree two.
\ed

\section{Homogeneous norms on Lie groups}

Let $\cg$ be a nilpotent Lie algebra, $\{ \delta_t \}_{t > 0}$ a family of dilations and $\G$ the connected and simply connected Lie group given by $\cg$ equipped with the multiplication law \eqref{bchformula}.

\bd \index{homogeneous norm}
A {\em homogeneous norm} on $\G$ is a map $|\cdot| : \G \to [0,\,\infty)$ satisfying the following properties: \mbox{}
\begin{enumerate}[label=(\roman*)]
\item $x\mapsto |x|$ is continuous with respect to the topology on $\G$;
\item $|x| = 0$ if and only if $x = 0$, the zero vector of $\cg$;
\item if $\delta_t$ also denotes the dilations on $\G$, then $|\delta_t x| = t |x|$ for all $t > 0$.
\end{enumerate}
\ed

We always refer to such a map as {\em homogeneous norm} because often, as we will see shortly, it does not satisfy the triangular inequality and thus it is not a norm.

\brmk
Let $\cg = \cv_1 \oplus \cdots \oplus \cv_k$ be the decomposition induces by the dilations in such a way that
\[
\delta_t \, \big|_{\mathfrak{v}_i} = t^{\lambda_i} \mathrm{Id}_{\cv_i}.
\]
If $\|\cdot \|$ is the norm inducing the topology on $\cg$, then it is easy to verify that
\[
|x| := \sum_{i = 1}^k \| x_i \|^{\frac{1}{\lambda_i}}
\]
is a homogeneous norm. This shows that in ({\romannumeral 1}) continuity is the ``right'' requirement because $|\cdot|$ is not even $C^1$ as soon as $\lambda_i \geq 2$ for some $i$.
\ermk

\bpr
A homogeneous norm satisfies a quasi-triangular inequality\index{quasi-triangular inequality}. In other words, there exists a constant $C \geq 1$ such that
\begin{equation} \label{quasitr}
|x \cdot y| \leq C(|x| + |y|) \quad \text{for all $x,y\in\G$}.
\end{equation}
\epr

\begin{proof}
The set $B := \{ x \in \G \: : \: |x| \leq 1\}$ is compact and so is $B^2 := B\cdot B$ because the multiplication law is continuous. Consequently, there exists a constant $C \geq 1$ such that
\[
B^2 \subseteq \{ x \in \G \: : \: |x| \leq C \},
\]
which can be easily rewritten as follows:
\[
|x|,\,|y| \leq 1 \implies |x \cdot y| \leq C.
\]
Now take $x,\, y \neq 0$ and notice that, if $t := \frac{1}{|x|+|y|}$, then $|\delta_t x| \leq 1$ and $|\delta_t y| \leq 1$. Applying the inequality above shows that
\[
|\delta_t x \cdot \delta_t y| \leq C \implies |x \cdot y| \leq \frac{C}{t} = C(|x| + |y|).
\]
\end{proof}

\brmk
It follows from the definition and \eqref{quasitr} that the map $d(x,y) := |x^{-1}\cdot y|$ is a left-invariant homogeneous quasi-distance\index{quasi-distance} (distance if $C = 1$):
\[
d(x,y) = d(zx, zy) \quad \text{and} \quad d(\delta_t x, \delta_t y) = t d(x,y).
\]
\ermk

Notice that a quasi-distance does not induce another topology on $\G$ in general, and this is the reason why we require $|\cdot|$ to be continuous with respect to the topology of $\G$

\brmk
The quasi-distance $d$ is, in general, not smooth. However, it is possible to obtain smoothness outside of the origin by taking the Euclidean sphere $S_\cg$ in $\G$ and setting
\[
|x| := \frac{1}{t} \quad \text{if $\delta_t x \in S_\cg$.}
\]
The reader should check that this is a homogeneous norm which is also smooth by the implicit function theorem (except at $x = 0$).
\ermk

In {\bf stratified} groups, it is easy to construct a homogeneous norm that satisfies \eqref{quasitr} with constant $C = 1$. For this, let $\cv_1 \oplus \cdots \oplus \cv_k$ be a stratification and write
\[
\cv_1 = \mathrm{Span} \langle e_1,\, \dots,\, e_q \rangle.
\]
The corresponding vector fields $X_1,\,\dots,\, X_q$ satisfy the H\"{o}rmander condition on $\G$, and hence we can consider the Carnot-Carathéodory distance
\[
d(x, y) := \left\{ L_\X(\gamma) \: : \: \text{$\gamma$ horizontal curve joining $x$ and $y$} \right\},
\]
where $L_\X$ is the length defined in \hyperref[nonisolength]{Section \ref{nonisolength}}. It is easy to verify that this is a homogeneous distance which is left-invariant and satisfies \eqref{quasitr} with $C=1$.

\bthm[\cite{sik}]
Every nilpotent connected and simply connected group with dilations has a homogeneous norm which satisfies the triangular inequality (i.e., $C = 1$) and is smooth outside of the origin.
\ethm

\bpr
If $|\cdot|$ and $|\cdot|'$ are two homogeneous norm on a Lie group $\G$ equipped with the same family of dilations, then they are equivalent.
\epr