\title{Fourier Analysis on Lie Groups - Lecture Notes}
\author{Francesco Paolo Maiale
        }
        
\documentclass[a4paper,10 pt, twoside]{report}

%Graphics/Geometry Packages
\usepackage{graphicx}
\usepackage[dvipsnames]{xcolor}
\usepackage[labelfont=bf]{caption}
\usepackage[pass]{geometry}
\usepackage{setspace}
\usepackage{fancyhdr}
%\usepackage{refcheck}

%Language Packages
\usepackage[english]{babel}
\usepackage[utf8]{inputenc}
\usepackage{etoolbox} 

\patchcmd{\part}{plain}{empty}{}{}

%Mathematics Packages
\usepackage{stmaryrd}

\usepackage{amsfonts}
\usepackage{amsthm}
\usepackage{amsmath, amssymb}
\usepackage{setspace}
%\usepackage{mathabx}
\usepackage{faktor}
\usepackage{mathrsfs}  
\usepackage{enumitem}
\usepackage{mathtools}
%\usepackage{accents}
%\usepackage{pifont}
%\usepackage{mwe}
\usepackage{bbm}
\usepackage{PTSansNarrow}
\usepackage[T1]{fontenc}
\usepackage{epigraph}

%Misc Packages
\usepackage{scalerel}[2014/03/10]
\usepackage[usestackEOL]{stackengine}
\usepackage{ragged2e}
\usepackage[framemethod=tikz]{mdframed}
\usepackage{marginnote}
\usepackage{xparse}
\usepackage{hyperref}
\usepackage{comment}

%TIKZ Packages
\usepackage{tikz-cd}
\usepackage{tikzpagenodes}
\usetikzlibrary{calc}
\usetikzlibrary{matrix}
\usetikzlibrary{plotmarks}

\usepackage{makeidx}

\makeindex %DDD

\hypersetup{
    colorlinks=true,
    linkcolor=cyan,
    filecolor=magenta,      
    urlcolor=cyan,
}


\DeclareRobustCommand\longtwoheadrightarrow
     {\relbar\joinrel\twoheadrightarrow}
     
     %% code from mathabx.sty and mathabx.dcl
\DeclareFontFamily{U}{mathx}{\hyphenchar\font45}
\DeclareFontShape{U}{mathx}{m}{n}{
      <5> <6> <7> <8> <9> <10>
      <10.95> <12> <14.4> <17.28> <20.74> <24.88>
      mathx10
      }{}
\DeclareSymbolFont{mathx}{U}{mathx}{m}{n}
\DeclareFontSubstitution{U}{mathx}{m}{n}
\DeclareMathAccent{\widecheck}{0}{mathx}{"71}
\DeclareMathAccent{\wideparen}{0}{mathx}{"75}

\def\cs#1{\texttt{\char`\\#1}}

\newcommand{\notimplies}{%
  \mathrel{{\ooalign{\hidewidth$\not\phantom{=}$\hidewidth\cr$\implies$}}}}
  
\pagestyle{plain}
\setlength{\topmargin}{0.0in}
\setlength{\headheight}{0.2in}
\setlength{\headsep}{0.2in}
\setlength{\footskip}{0.5in}
\setlength{\footnotesep}{0.15in}
\setlength{\textheight}{8.3in}
\setlength{\textwidth}{5.5in} % 6
\setlength{\oddsidemargin}{0.5in}
\setlength{\evensidemargin}{0.5in}
\setlength{\parindent}{0.2 in}
\setlength{\parskip}{0.1 in}
\setlength{\marginparwidth}{1.2 in}
\linespread{1} 


\newtheorem{theorem}{Theorem}[chapter]
\newtheorem{lemma}[theorem]{Lemma}
\newtheorem{proposition}[theorem]{Proposition}
\newtheorem{corollary}[theorem]{Corollary}
\theoremstyle{definition}
\newtheorem{definition}[theorem]{Definition}

\newtheorem{remark}[theorem]{Remark}
\newtheorem{example}[theorem]{Example}
\newtheorem*{notation}{Notation}
\newtheorem*{claim}{Claim}
\newtheorem*{problem}{Problem}
\newtheorem{exercise}{Exercise}[chapter]

\newtheorem{innercustomthm}{Theorem}
\newenvironment{customthm}[1]
  {\renewcommand\theinnercustomthm{#1}\innercustomthm}
  {\endinnercustomthm}


\newcommand{\smallO}[1]{\scriptstyle\mathcal{O}}
\DeclarePairedDelimiter\floor{\lfloor}{\rfloor}
\newcommand*\conj[1]{\overline{#1}}
\newcommand{\R}{\mathbb R}

\newcommand{\C}{\mathbb C}
\newcommand{\N}{\mathbb N}
\newcommand{\G}{\mathbb G}
\newcommand{\Z}{\mathbb Z}
\newcommand{\con}{\mathcal{C}\left(x, \, V, \, \alpha \right)}
\newcommand{\Q}{\mathbb Q}
\newcommand{\sD}{\mathscr D}
\newcommand{\cM}{\mathcal M}
\newcommand{\cB}{\mathcal B}
\newcommand{\cL}{\mathcal L}
\newcommand{\m}{\mathrm{m}}
\newcommand{\cD}{\mathcal D}
\newcommand{\e}{\mathrm e}
\newcommand{\ord}{\mathrm{ord}}
\newcommand{\gln}{\mathrm{GL}(n, \, \R)}
\newcommand{\Ad}{\mathfrak{Ad}}
\newcommand{\ad}{\mathfrak{ad}}
\newcommand{\cU}{\mathcal U}
\newcommand{\cA}{\mathcal A}
\newcommand{\cP}{\mathcal P}
\newcommand{\cE}{\mathcal E}
\newcommand{\cY}{\mathcal Y}
\newcommand{\p}{\mathbb P}
\newcommand{\X}{\mathfrak X}
\newcommand{\rmd}{\mathrm d}
\newcommand{\Tan}{\mathrm{Tan}}
\newcommand{\Y}{\mathfrak Y}
\newcommand{\F}{\mathscr F}
\newcommand{\cF}{\mathcal F}
\newcommand{\D}{\mathcal D}
\newcommand{\Le}{\mathcal{L}}
\newcommand{\homs}{\mathrm{Hom}}
\newcommand{\can}{\symbol{35}}
\newcommand{\dr}{\mathrm{d}}
\newcommand{\cS}{\mathscr{S}}
\newcommand{\scA}{\mathscr{A}}
\newcommand{\inj}{\hookrightarrow}
\newcommand{\spt}{\mathrm{spt}}
\newcommand{\cg}{\mathfrak{g}}
\newcommand{\cv}{\mathfrak v}
\newcommand{\cw}{\mathfrak w}
\newcommand{\Sc}{\mathscr{S}(\R^n)}
\newcommand{\Scp}{\mathscr{S}^\prime(\R^n)}
\newcommand{\End}{\mathrm{End}}
\newcommand*{\double}[2][.1ex]{%
  \mathrel{\vcenter{\offinterlineskip%
  \hbox{$#2$}\vskip#1\hbox{$#2$}}}}
\newcommand*{\doublerightarrow}{\double{\longrightarrow}}

\makeatletter
\newcommand{\proofstep}[1]{%
  \par% ensure starting on a new paragraph
  \addvspace{\medskipamount}% some vertical space
  \textbf{#1\@addpunct{.}}\enspace\ignorespaces
}
\makeatother

\newcommand{\brmk}{\begin{remark}}
\newcommand{\ermk}{\end{remark}}
\newcommand{\bl}{\begin{lemma}}
\newcommand{\el}{\end{lemma}}
\newcommand{\bthm}{\begin{theorem}}
\newcommand{\ethm}{\end{theorem}}
\newcommand{\bpr}{\begin{proposition}}
\newcommand{\epr}{\end{proposition}}
\newcommand{\bd}{\begin{definition}}
\newcommand{\ed}{\end{definition}}
\newcommand{\bex}{\begin{example}}
\newcommand{\eex}{\end{example}}
\newcommand{\bexe}{\begin{exercise}}
\newcommand{\eexe}{\end{exercise}}
\newcommand{\bcor}{\begin{corollary}}
\newcommand{\ecor}{\end{corollary}}

% Caution Form Code START

\newcounter{mycaution}
\newcommand\pointeranchor{}
\newcommand\boxanchor{}
\newlength\boxvshift
\newlength\uppertrianglecorner

\newcommand\tikzmark[1]{%
  \tikz[remember picture,overlay]\node[inner xsep=0pt,outer sep=0pt] (#1) {};}

\NewDocumentCommand{\caution}{O{c}O{BrickRed}O{Caution!}m}{%
\stepcounter{mycaution}%
\tikzmark{\themycaution}%
\if#1b\relax
\renewcommand\pointeranchor{mybox\themycaution.south east}%
\renewcommand\boxanchor{south east}%
\setlength\boxvshift{-10pt}%
\setlength\uppertrianglecorner{13pt}%
\else
\if#1t\relax
\renewcommand\pointeranchor{mybox\themycaution.north east}%
\renewcommand\boxanchor{north east}%
\setlength\boxvshift{10pt}%
\setlength\uppertrianglecorner{-7pt}%
\else
\if#1c\relax
\renewcommand\pointeranchor{mybox\themycaution.east}%
\renewcommand\boxanchor{east}%
\setlength\boxvshift{0pt}%
\setlength\uppertrianglecorner{3pt}%
\fi\fi\fi%
\begin{tikzpicture}[remember picture,overlay]
\node[draw=#2,anchor=\boxanchor,xshift=-\marginparsep,yshift=\boxvshift]   
  (mybox\themycaution)
  at ([yshift=3pt]current page text area.west|-\themycaution) 
  {\parbox{\marginparwidth}{\vskip10pt\RaggedRight\small#4}};
\node[fill=white,font=\color{#2}\sffamily,anchor=west,xshift=7pt]
  at (mybox\themycaution.north west) {\ #3\ };
\fill[#2]
  ([yshift=\uppertrianglecorner]\pointeranchor) --
  ([yshift=\uppertrianglecorner-3pt,xshift=3pt]\pointeranchor) --
  ([yshift=\uppertrianglecorner-6pt]\pointeranchor) -- cycle;
\end{tikzpicture}%
}

%% Caution Form Code END

\newcommand{\restr}{%
  \,\raisebox{-.127ex}{\reflectbox{\rotatebox[origin=br]{-90}{$\lnot$}}}\,%
}

\makeatletter
\renewcommand*\env@matrix[1][*\c@MaxMatrixCols c]{%
  \hskip -\arraycolsep
  \let\@ifnextchar\new@ifnextchar
  \array{#1}}
\makeatother

\newcommand{\bsquare}{\item[\color{magenta}\ding{110}]} 
\newcommand{\barrow}{\item[\color{blue}\ding{228}]}
\newcommand{\bwarrow}{\item[\color{gray}\ding{227}]}

\def\dashint{\,\ThisStyle{\ensurestackMath{%
  \stackinset{c}{.2\LMpt}{c}{.5\LMpt}{\SavedStyle-}{\SavedStyle\phantom{\int}}}%
  \setbox0=\hbox{$\SavedStyle\int\,$}\kern-\wd0}\int}
\def\ddashint{\,\ThisStyle{\ensurestackMath{%
  \stackinset{c}{.2\LMpt}{c}{.5\LMpt+.2\LMex}{\SavedStyle-}{%
    \stackinset{c}{.2\LMpt}{c}{.5\LMpt-.2\LMex}{\SavedStyle-}{%
      \SavedStyle\phantom{\int}}}}\setbox0=\hbox{$\SavedStyle\int\,$}\kern-\wd0}\int}

\newcommand*{\transp}[2][-3mu]{\ensuremath{\mskip1mu\prescript{\smash{\mathrm t\mkern#1}}{}{\mathstrut#2}}}%


\fancyhf{}
% Put the page number at the right edge of odd pages, and left edge of even pages.
\fancyhead[LO,RE]{\textbf \thepage}
% Custom text at the left edge of odd pages, and right edge of odd pages.
\fancyhead[RO]{ \slshape\nouppercase{\rightmark}}
\fancyhead[LE]{ \slshape\nouppercase{\leftmark}}

% Repeat for \fancyfoot if needed, e.g.
% Some decorative symbol at the centre of both odd and even pages
\fancyfoot[C]{ }

% Set this length to 0pt if you don't want any lines!
\renewcommand{\headrulewidth}{1pt}

% Apply the fancy header style
\pagestyle{fancy}

\begin{document}
\newpage \thispagestyle{empty}

\begin{center}

\begin{spacing}{1.5}
{\huge  \sf Lecture Notes}\\
\vspace*{\fill}
\end{spacing}
\begin{spacing}{2.5}
\textbf{\huge Analysis of Sublaplacians on Lie Groups}\\[0.5cm]
\vspace*{\fill}
\begin{minipage}{5cm}
\centering {\textit{Course held by}}
\end{minipage}
\hspace*{\fill}
\begin{minipage}{5cm}
\centering {\textit{Notes written by}} \\
\end{minipage}
\end{spacing}

\begin{spacing}{1.3}

\begin{minipage}{5cm}
\centering {\textbf{\large Prof. Fulvio Ricci}}
\end{minipage}
\hspace*{\fill}
\begin{minipage}{5cm}
\centering {\textbf{\large Francesco Paolo Maiale}}
\end{minipage}

\vspace*{\fill}

\textnormal{\large Scuola Normale Superiore, Pisa \\[0.4em] {\small\underline{Last Update}: \today}}

\end{spacing}
\end{center}

\newpage \thispagestyle{empty}
\begin{center}
\vspace*{1cm}
\begin{spacing}{2.5}
 {
\textbf{\huge Disclaimer}
}
\end{spacing} \end{center}

\begin{spacing}{1.2}
I wrote these notes to summarise the content of the course Analysis of Sublaplacians on Lie Groups, held by Professor Fulvio Ricci at SNS.

I tried to include all the topics that were discussed in class and combine it with some additional information from several other courses to produce a self-contained document.

I will try to review them periodically, but I am sure that at the end there will be a large number of mistakes and oversights. To report them, feel free to send me an email at \textbf{francesco (dot) maiale (at) sns (dot) it}.
\end{spacing}


{ \setlength{\parskip}{0.05 in}

\clearpage                       % Otherwise \pagestyle affects the previous page.
{                                % Enclosed in braces so that re-definition is temporary.
  \pagestyle{empty}              % Removes numbers from middle pages.
  \fancypagestyle{plain}         % Re-definition removes numbers from first page.
  {
    \fancyhf{}%                       % Clear all header and footer fields.
    \renewcommand{\headrulewidth}{0pt}% Clear rules (remove these two lines if not desired).
    \renewcommand{\footrulewidth}{0pt}%
  }
  \tableofcontents
  \thispagestyle{empty}          % Removes numbers from last page.
}}


\part{Fourier Analysis on Euclidean Spaces}

\chapter{Introduction}

In this chapter, we introduce the main topics of the course and give a brief overview of what we will see and what we will be able to prove by the end of the course.

\section{Plateau's Problem}

The primary goal and the motivating example of this course is the \textbf{Plateau's problem}, that is, the problem to find the $d$-dimensional surface $\Sigma$ of the minimal area with prescribed $(d-1)$-dimensional boundary $\Gamma$.

By the end, we will be able to prove that a solution indeed exists, but we will not find it explicitly since it is a $NP$ (hard) numerical problem.

As of now, the problem is not well defined. In fact, the notions of \textit{surface}, \textit{area}, and \textit{boundary} make sense in the smooth setting but, as the examples below show, we need to work in a less regular setting.

More precisely, requiring the surface to be smooth is not enough for modeling reasons (e.g., dip a wire frame into a soap solution, form a soap film, and look for the minimal surface whose boundary is the wire frame), and also for existence reasons.

\begin{example} Here we give a list of Plateau's problems with prescribed boundary conditions, and we write down the correct solutions, without proving anything. \mbox{}
\begin{enumerate}[label=\textbf{(\alph*)}]
\item Let us identify $\R^4 \cong \C \times \C$ and, if $d = 2$, let us consider the smooth boundary given by
\begin{equation*} \Gamma_1 := \left( S^1 \times \{0\} \right) \cup \left( \{0\} \times S^1 \right). \end{equation*}
Surprisingly, every minimizing sequence of smooth surfaces converges to a surface which is not smooth at all. Indeed, the solution of the problem is
\begin{equation*} \Sigma_1 := \left( D^2 \times \{0\} \right) \cup \left( \{0\} \times D^2 \right). \end{equation*}
The surface $\Sigma_1$ is clearly singular at the origin, but the singularity may be removed (by factorizing it into two nonsingular surfaces).
\item Let us identify $\R^4 \cong \C \times \C$ and, if $d = 2$, let us consider the smooth boundary given by
\begin{equation*} \Gamma_2 := \left\{ (z^2, \, z^3) \: : \: z \in S^1 \right\}. \end{equation*}
The solution to the Plateau's problem is
\begin{equation*} \Sigma_2 := \left\{ (z^2, \, z^3) \: : \: z \in D^2 \right\}, \end{equation*}
which is a non-smooth surface, whose singularity cannot be removed (since the polynomial $z_1^3 = z_2^2$ cannot be factorized).
\item Let us identify $\R^8 \cong \R^4 \times \R^4$ and, if $d = 7$, let us consider the smooth boundary given by
\begin{equation*} \Gamma_3 := S^3 \times S^3. \end{equation*}
The minimal surface of prescribed boundary $\Gamma_3$ is
\begin{equation*} \Sigma_3 := \left\{ (x_1, \, x_2) \in \R^4 \times \R^4 \: : \: |x_1| = |x_2| \leq 1 \right\}. \end{equation*}
\end{enumerate}
\end{example}

To conclude this introductive chapter, we give a brief overview of the main approaches (studied in this course) to the Plateau's problem, as $d$ ranges between $1$ and $\infty$.

\section{Geodesics problem ($d=1$)}

The geodesics problem (that is, find the shortest curve connecting two points) is, surprisingly, still an open in the non-Riemannian setting. However, in the Riemannian setting, the geodesics problem is completely solved.

Indeed, if we consider the curves parametrized by paths, the \textit{length} is a well-defined notion, and the associated functional is lower semi-continuous and coercive; hence the compactness is easy to prove.

There are many possible approaches to the geodesics problem, e.g., the Steiner approach and the set theoretical approach, which we describe briefly in the remainder of the section.

\paragraph{Steiner Problem.} It is also called networks approach, and it is used to prove the existence of the geodesics and find the explicit expression for it. The reader may consult \cite{steinerpro} for a detailed dissertation on the topic.

\paragraph{Set Theoretical Approach.} The main idea is to find a closed and connected set $\Sigma$ of minimum \textit{length}, containing a given finite set $\Gamma$. As we shall see later in the course, in this case the length is a well defined concept: the \textit{Hausdorff distance}.

In fact, if $X$ is a suitable space (metric, endowed with Hausdorff distance, etc...), then the class defined by
\begin{equation*}\mathcal{X} := \left\{ K \subseteq X \: : \: K \, \, \text{compact and connected} \right\} \end{equation*}
is compact and, by Gotab theorem\footnote{\cite{falconer} Let $\mathscr{C}$ be an infinite collection of non-empty compact sets all lying in a bounded portion $B$ of $\R^n$. Then there exists a sequence $\{E_j\}$ of distinct sets of $\mathfrak{C}$ convergent in the Hausdorff metric to a non-empty compact set $E$.}, $\mathcal{H}^1$ is lower semi-continuous on $X$. 

\section{"Surface" Problem ($d>1$)}

\paragraph{3.} The Plateau's problem is much harder when $d = 2$, but there are still many approaches possible some of which relying, in a certain sense, on the work already done in the geodesics case.

\paragraph{Set Theoretical Approach.} This approach is highly nontrivial. For example, one may ask what does it mean that a compact set $\Sigma$ spans a boundary $\Gamma$? Moreover, there is another problem one should deal with: the $2$-dimensional Hausdorff measure $\mathcal{H}^2$ is, generally, not lower semi-continuous. The reader may consult \cite{Reifenberg1960} for a complete treatise of the topic.

\begin{remark}Suppose that $d = 2$, $n = 3$ and that $\Sigma$ is a surface with boundary $\Gamma$. If $\gamma$ is another closed curve, linked to $\Gamma$ (by a nonzero linking number), then $\gamma \cap \Sigma \neq \emptyset$. \end{remark}

\paragraph{Parametric Approach.} This method is essentially due to Douglas \cite{douglas}. The main idea is the following: since a parametrization $\phi : D^2 \to \R^n$ defines surfaces in $\R^n$, the area functional is well-defined and given by the formula
\begin{equation*}A(\phi) := \int_{D^2} \left| \frac{\partial \, \phi}{\partial \, s_1} \wedge \frac{\partial \, \phi}{\partial \, s_2} \right| \, \mathrm{d}s_1 \, \mathrm{d}s_2. \end{equation*}
On the other hand, the existence through lower semi-continuity and the compactness are a delicate matter, since coercivity is not an easy property to obtain (the integrand is similar to a determinant).

There is a trick which is similar to the one we can use to find geodesics in the differential geometry setting. More precisely, we consider the functional
\begin{equation*}E(\phi) := \frac{1}{2} \, \int_{D^2} \left| \nabla \phi \right|^2 \, \mathrm{d}s_1 \, \mathrm{d}s_2. \end{equation*}
If we find a minimal point $\phi$ for $E$, then $\phi$ will be a \textbf{conformal parametrized} minimum for $A$. This trick, on the other hand, heavily depends on a nontrivial theorem: every such $\Sigma$ admits a conformal re-parametrization.

The lack of conformal parametrization, though, is what stop us from extending the same trick to dimension $d$ strictly bigger than $2$.

\paragraph{Higher Dimension.} If the codimension of $\Sigma$ is equal to $1$ (that is, $n = d+1$), then finite perimeter sets generalize the notion of open $(d+1)$-dimensional sets with smooth boundary in $\R^n$.

The class of finite perimeter sets has excellent compactness properties and a notion of area lower semi-continuous. 

This approach is called "weak" surfaces approach, and it is essentially due to Caccioppoli \cite{cacio} and De Giorgi \cite{degi}. A different approach, working for any $d$ and $n$, referred to as \textit{integral currents}, was introduced by Federer and Flaming in their joint paper \cite{fed}.
\chapter{Linear Differential Operators} \thispagestyle{empty}

In the first half of this chapter, we will focus on linear operators defined on $\Sc$ and taking values in the space of tempered distributions. More precisely, given
\[
T : \Sc \longrightarrow \Scp,
\]
we will show that the following characterisation holds:

\begin{customthm}{A}
Suppose that $T$ is a continuous and translation-invariant. Then there exists a unique $\Psi \in \Scp$ such that
\[
Tf = \Psi \ast f \quad \text{for all $f \in \Sc$}.
\]
\end{customthm}

Next, we consider translation-invariant operators between $L^p$ spaces, show that they identify a particular class $C_{p, \, q}$, which can be connected with known spaces for specific values of $p$ and $q$. For example, we shall prove the following result:

\begin{customthm}{B}
The space $C_{2, \, 2} $ is isometrically isomorphic to $\F (L^\infty)$.
\end{customthm}

In the last part of the chapter, we discuss a completely different topic: linear differential operators and fundamental solutions. Recall that
\[
Lf = \sum_{|\alpha| \leq m} c_\alpha \partial^\alpha f,
\] 
where $c_\alpha$ are functions, eventually constant, is a linear differential operator. Our goal is to prove a result of the utmost importance, which is due to Malgrange and Ehrenpreis.

\begin{customthm}{C}
Let $L$ be a linear differential operator with constant coefficients. Then there exists $\psi \in \D^\prime(\R^n)$ such that
\[
L \psi = \delta_0.
\] \end{customthm}

\section{Distributional kernels}

Recall that an operator $T : \Sc \to \Scp$ is continuous if the duality coupling
\begin{equation} \label{extra.3.1}
f \longmapsto \langle Tf, g \rangle
\end{equation}
is a continuous linear functional for all $g \in \Sc$. We already mentioned that \eqref{extra.3.1} can be identified to the following bilinear form
\[
(f, g) \longmapsto \langle Tf, g \rangle,
\]
which is separately continuous, provided that $T$ is continuous. By {\bf Banach-Alaoglu}, it is also jointly continuous and thus we can find $N \in \N$ and $C > 0$ such that
\[
\left| \langle Tf, g \rangle \right| \leq C \|f\|_{(N)} \|g\|_{(N)}.
\]

\bex
Let us consider an integral operator with kernel $k$, namely
\begin{equation} \label{extra.3.2}
Tf(x) := \int_{\R^n} k(x, \, y) f(y) \, \mathrm{d}y,
\end{equation}
and assume that $k$ is continuous and supported on a compact set $K$. The associated linear functional \eqref{extra.3.1} is given by
\[
\langle Tf, g \rangle = \int_{\R^n} g(x) \int_{\R^n} k(x, y) f(y) \, \mathrm{d}y \mathrm{d}x.
\]
Now introduce the symbol $g \otimes f(x, y)$ to indicate the product $f(y) g(x)$. Then we can rewrite the linear functional as follows:
\[
\langle Tf, \, g \rangle = \iint_{\R^n \times \R^n} k(x, \, y) (g \otimes f)(x, \, y) \, \mathrm{d}y \mathrm{d}x.
\]
If $f$ and $g$ are Schwartz functions we easily infer that $T$ is a continuous operator since
\[
\left| \langle Tf, \, g \rangle \right| \leq \|k \|_{L^1(\Omega)} \|f\|_{(0)} \|g\|_{(0)},
\]
and $\|k \|_{L^1(\Omega)}$ is finite because we assumed $k$ to be (more than) summable.
\eex

Now let $\Phi \in \mathcal{S}^\prime(\R_x^m \times \R_y^n)$ be a tempered distribution, $f \in \mathcal{S}(\R^m)$ and $g \in \mathcal{S}(\R^n)$ in such a way that $g \otimes f \in \mathcal{S}(\R_x^m \times \R_y^n)$. We can define the equivalent of \eqref{extra.3.2}, a \textit{distributional kernel}\index{distributional kernel}, by setting
\[
\langle T_\Phi f, \, g \rangle := \langle \Phi, \, g \otimes f \rangle.
\]
Since $\Phi$ is continuous, we can find $N \in \N$ and $C > 0$ such that
\[
\left| \langle T_\Phi f, \, g \rangle \right| \lesssim \|g \otimes f \|_{(N)} \lesssim \|f\|_{(N)} \|g\|_{(N)}.
\]
This shows that the operator $T_\Phi$ is continuous from $\mathcal{S}(\R^m)$ to $\mathcal{S}^\prime(\R^n)$.

\bthm \label{thm.2.1.1}
Let $T : \Sc \to \cS^\prime(\R^m)$ be a linear continuous operator. Then there exists a unique $\Phi \in \cS^\prime(\R^m \times \R^n)$ such that
\[
T = T_\Phi.\]
\ethm

We will not give full proof of this result, but instead, we wish to discuss the main ideas behind it. The first step is to find a tensorial decomposition for Schwartz functions.

\bl \label{lemma.2.1.1}
For every $n \in \N$ there exists $m \in \N$ such that for every function $h \in \cS(\R^m \times \R^n)$ we can decompose $h$ as follows:
\[
h = \sum_{j = 0}^\infty g_j \otimes f_j,
\]
where $f_j$ and $g_j$ are functions satisfying the estimate
\[
\sum_{j = 0}^\infty \|g_j\|_{(N)} \|f_j\|_{(N)} \lesssim \|h\|_{(M)}.
\]
\el

Now suppose that there exists $\Phi$ such that $T = T_\Phi$. By definition, given an arbitrary function $h$, we can write
\[
\langle \Phi, \, h \rangle = \sum_{j = 0}^\infty \langle \Phi, \, g_j \otimes f_j \rangle = \sum_{j = 0}^\infty \langle T f_j, \, g_j \rangle
\]
as a consequence of the previous technical lemma. The idea is to start from this identity and construct a $\Phi$ that satisfies it for every function $h \in \mathcal{S}(\R_x^m \times \R_y^n)$.

\bex
Consider the non-linear differential operator
\[
L f(x) := \sum_{|\alpha| \leq d} c_\alpha(x) \partial^\alpha f(x), 
\]
where $c_\alpha$ are smooth and bounded functions. It is easy to verify that
\[
L(\Sc) \subseteq \Sc \subset \Scp,
\]
so we can try to write explicitly the associated distributional kernel. If $\Phi$ exists, it must satisfy
\[
\langle \Phi, \, g \otimes f \rangle = \langle L f, \, g \rangle = \int_{\R^n} \left( \sum_{|\alpha| \leq d} c_\alpha(x) \partial^\alpha f(x) \right) g(x) \, \mathrm{d}x.
\]
The reader might prove, as an exercise, that
\[
\langle \Phi, \, h \rangle = \int_{\R^n} c_\alpha(x) ( \partial_2^\alpha h)(x, \, x) \, \mathrm{d}x,
\]
where $\partial_2^\alpha$ indicates the derivative with respect to the second variable. Finally, notice that this formula defines a distribution since it is linear and there results:
\[
\left| \langle \Phi, \, h \rangle \right| \leq \sum_{|\alpha| \leq d} \| c_\alpha \|_\infty \int_{\R^n} \frac{c \|h\|_{(N+d)}}{(1 + |x|)^N} \, \mathrm{d}x,
\]
which for $N$ big enough (so that the integral converges) leads to
\[
\left| \langle \Phi, \, h \rangle \right| \lesssim_{\alpha} \|h\|_{(N + d)}. 
\]
\eex

\section{Translation-invariant operators}

The goal of this section is to characterise linear operators $T : \Sc \to \Scp$ that are continuous and {\em translation-invariant}.

\bd \index{operator!translation-invariant}\index{distribution!translation}
We say that the operator $T : \Sc \to \Scp$ is {\em translation-invariant} if
\[
\langle \tau_h (Tf), \, g \rangle = \langle T(\tau_h f), \, g \rangle
\]
holds for all $h \in \R^n$ and all $g \in \Sc$. The translation of a distribution is given by
\[
\langle \tau_h \Phi, \, f \rangle := \langle \Phi, \, \tau_{-h} f \rangle.
\]
\ed

Let $\Phi$ be the distributional kernel given by \hyperref[thm.2.1.1]{Theorem \ref{thm.2.1.1}}. The operator $T = T_\Phi$ is translation-invariant if and only if
\[
 \langle \Phi, \, g \otimes (\tau_h f) \rangle = \langle \Phi, \, (\tau_{-h} g) \otimes f \rangle,
\]
and, if we replace $g$ with $\tau_h g$, it is also equivalent to
\begin{equation} \label{extra.4.1}
\langle \Phi, \, (\tau_h g) \otimes (\tau_h f) \rangle = \langle \Phi, \, g \otimes f \rangle.
\end{equation}
Note that, if $\varphi \in \mathcal{S}(\R^m \times \R^n)$ is not given by the elementary product $g \otimes f$, we can exploit the decomposition given by \hyperref[lemma.2.1.1]{Lemma \ref{lemma.2.1.1}}. Then \eqref{extra.4.1} implies that
\[ \begin{aligned}
\langle \Phi, \, \varphi \rangle & = \sum_{j=0}^\infty \langle \Phi, \, g_j \otimes f_j \rangle =
\\[1em] & = \sum_{j = 0}^\infty \langle \Phi, \, (\tau_h g_j) \otimes (\tau_h f_j) \rangle =
\\[1em] & = \langle \Phi, \, \tau_h \varphi \rangle,
\end{aligned} \]
which means that \eqref{extra.4.1} completely characterise the property of being translation-invariant for the operator $T_\Phi$.

\bex
Let $\Psi \in \Scp$. The {\em convolution operator} associated to $\Psi$ is defined by setting\index{operator!convolution}
\[ 
\Sc \ni f \longmapsto Tf := \Psi \ast f.
\]
A straightforward computation, which follows from the definitions, shows that
\[ \begin{aligned}
\langle \Psi \ast (\tau_h f), \, g \rangle & = \langle \Psi, \, \widecheck{\tau_h f} \ast g \rangle =
\\[1em] & = \langle \Psi, \, \check{f} \ast (\tau_{-h} g) \rangle =
\\[1em] & = \langle \Psi \ast f, \, \tau_{-h} g \rangle =
\\[1em] & = \langle \tau_h(\Psi \ast f), \, g \rangle.
\end{aligned} \]
From this chain of equalities we easily deduce that
\[
T(\tau_h f) = \Psi \ast (\tau_h f) = \tau_h(\Psi \ast f) = \tau_h(Tf), 
\]
which means that $T$ is translation-invariant as claimed.
\eex

The following result asserts that all linear continuous translation-invariant operators are necessarily equal to a {\em convolution operator}.

\bthm \label{thm.2.1.2}
Let $T : \Sc \to \Scp$ be a linear continuous translation-invariant operator. Then there exists a unique $\Psi \in \Scp$ such that
\[ Tf = \Psi \ast f. \]
\ethm

We can now investigate the relation between $\Psi$ and the kernel $\Phi$ associated to $T$. More precisely, we will show that in a special case the following formula holds:
\[
\Psi(x-y) := \Phi(x-y, \, 0),
\]
and then we will try to generalise it via an approximation argument.

\bex
Let $T$ be a linear continuous operator and let $\Phi$ be its kernel, that is,
\[
\langle Tf, \, g \rangle = \langle \Phi, \, g \otimes f \rangle.
\]
Suppose that $\Phi$ is a continuous {\bf bounded function}. We can rewrite the above expression via integrals as follows:
\[ 
\langle Tf, \, g \rangle = \iint \Phi(x, \, y) g(x) f(y) \, \mathrm{d}x \mathrm{d}y.
\]
If $T$ is also translation-invariant, then
\[
\iint \Phi(x, \, y) g(x) f(y-h) \, \mathrm{d}x \mathrm{d}y =  \iint \Phi(x, \, y) g(x+h) f(y) \, \mathrm{d}x \mathrm{d}y,
\]
which is equivalent, via two change of variables, to the condition
\[
\iint \Phi(x, \, y + h) g(x) f(y) \, \mathrm{d}x \mathrm{d}y =  \iint \Phi(x-h, \, y) g(x) f(y) \, \mathrm{d}x \mathrm{d}y.
\]
Since $\Phi$ is a continuous function, it is easy to see that it must satisfy
\[
\Phi(x + h, \, y) = \Phi(x, \, y-h),
\]
for all $x, \, y, \, h \in \R^n$. We can evaluate it at the point $y = y^\prime + h$ and find that
\[
\Phi(x + h, \, y+h) = \Phi(x, \, y).
\]
In particular, the function $\Phi(x, \, y) = \Phi(x - y, \, 0)$ depends on a single variable, the difference between $x$ and $y$, and therefore the following function is well-defined:
\[
\Psi(x-y) := \Phi(x-y, \, 0).
\]
It is now easy to check that
\[
Tf = \Psi \ast f,
\]
and this concludes the proof in the special case.
\eex

In the general case, the computation above is not valid. Nevertheless, we can approximate $\Phi$ via a family of mollifiers $\{\eta_\epsilon \otimes \eta_\epsilon \}_{\epsilon > 0}$ and use the fact that
\[
(\eta_\epsilon \otimes \eta_\epsilon) \ast \Phi
\]
is a smooth function that converges to $\Phi$ when $\epsilon$ tends to zero.

\bcor
If $T$ is a linear continuous translation-invariant operator, then
\[
T(f \ast g) = (Tf) \ast g \quad \text{for all $f, \, g \in \Sc$.}
\]
\ecor

\begin{proof}
By \autoref{thm.2.1.2}, we know that there exists a unique tempered distribution $\Psi$ such that $Tf = \Psi \ast f$, so it suffices to show that convolution is associative, that is,
\[
\Psi \ast (f \ast g) \stackrel{?}{=} (\Psi \ast f) \ast g.
\]
Let $h \in \Sc$ be a test function. Then
\[ \begin{aligned}
\langle \Phi \ast(f \ast g), \, h \rangle & = \langle \Phi, \, \widecheck{(f \ast g)} \ast h \rangle =
\\[1em] & = \langle \Phi, \, \check{f} \ast \check{g} \ast h \rangle =
\\[1em] & = \langle \Phi \ast f, \, \check{g} \ast h \rangle = \langle (\Phi \ast f) \ast g, \, h \rangle,
\end{aligned} \]
and this concludes the proof since $h$ is arbitrary.
\end{proof}

\section{Translation-invariant operators in $L^p$}

Let $1 \leq p < q \leq \infty$. If we can show that a tempered distribution $\Phi \in \Scp$ satisfies an inequality of the type
\begin{equation} \label{eq.13.1}
\| \Phi \ast f \|_{L^q(\R^n)}  \lesssim_{p, \, q} \|f\|_{L^p(\R^n)}
\end{equation}
for all $f \in \Sc$, then the associated linear operator
\[
T_\Phi : \Sc \subset L^p(\R^n) \longrightarrow L^q(\R^n) \subset \Scp
\]
is continuous. In particular, it extends uniquely to the closure of the domain,
\[
\widetilde{T_\Phi} : \overline{\Sc}^{\| \cdot \|_p} \longrightarrow L^q(\R^n).
\]

\brmk
The closure of $\Sc$ with respect to $\| \cdot \|_{L^p(\R^n)}$ coincides with $L^p(\R^n)$ if $p \geq 1$ is finite and is given by
\[ C_0(\R^n) := \{ f \in C(\R^n) \: : \: \lim_{|x| \to \infty} f(x) = 0 \}
\]
if $p = \infty$.
\ermk

Consequently, it makes sense to wonder whether or not a linear continuous translation-invariant operator $T$ from $L^p(\R^n)$ to $L^q(\R^n)$, $p < \infty$, is necessarily a convolution operator,
\[
T(f) = \Phi \ast f,
\]
where $\Phi$ belongs to an appropriate subspace of $\D^\prime(\R^n)$. Notice that
\[
T \, \big|_{\Sc} : \Sc \longrightarrow \Scp \supset L^q(\R^n),
\]
so by \autoref{thm.2.1.2} we can always find a tempered distribution $\Phi$ such that
\[
T(f) = \Phi \ast f \quad \text{for all $f \in \Sc$}.
\]
Set $p = 1$ and let $\{ \varphi_n \}_{n \in \N} \subset \Sc$ be an approximation identity (see \autoref{sec:appid}) such that whenever $f \in \Sc$
\[
\| \varphi_n \ast f - f \|_{L^1(\R^n)} \xrightarrow{n \to + \infty} 0.
\]
The operator $T$ is continuous, which means that $Tf$ is the limit (w.r.t the $L^q$ strong topology) of $T(\varphi_n \ast f)$; in other words, we have
\[
\| Tf - (T \varphi_n) \ast f \|_{L^q(\R^n)} \xrightarrow{n \to +\ infty} 0.
\]
Moreover, it is not hard to see that the family $\{ T \varphi_n \}_{n \in \N}$ is bounded in $L^q(\R^n)$. In fact, there exists a uniform constant $C(q) := C > 0$ such that
\[
\| T \varphi_n \|_{L^q(\R^n)} \leq C < \infty \quad \text{for all $n \in \N$}.
\]
By Banach-Alaoglu, we can find a subsequence $(n_k)_{k \in \N}$ for which $T \varphi_{n_k}$ converges to $\Phi \in L^q(\R^n)$ with respect to the {\bf weak topology}. It follows that, given $g \in \Sc$, we have
\[ \begin{aligned}
\langle Tf, \, g \rangle & = \langle T, \, \check{f} \ast g \rangle =
\\[1em] & = \langle T, \, \widecheck{(f \ast \varphi_n)} \ast g \rangle =
\\[1em] & = \langle T \varphi_n, \, \check{f} \ast g \rangle =
\\[1em] & = \langle T \varphi_n \ast f, \, g \rangle,
\end{aligned} \]
which means that, up to subsequences, we have
\[
\langle Tf, \, g \rangle = \lim_{n \to + \infty} \langle T \varphi_n \ast f, \, g \rangle = \langle \Phi \ast f, \, g\rangle,
\]
and this concludes the proof that $T$ is a convolution operator in the case $p = 1$.

\brmk
The computation above requires the additional assumption $q \neq 1, \, \infty$ since otherwise we end up with $L^\infty(\R^n)$ and $L^1(\R^n)$, which are not reflexive and hence Banach-Alaoglu theorem does not apply. 
\ermk

We now summarise what we have obtained above in the next proposition, which asserts that translation-invariant operators from $L^1$ to $L^q$ are convolution operators.

\bpr \label{proposition:exatra213}
Let $T : L^1(\R^n) \to L^q(\R^n)$, $1 < q < \infty$, be a linear continuous translation-invariant operator. Then there exists $\Psi \in L^q(\R^n)$ such that
\[
Tf = \Psi \ast f  \quad \text{for all $f \in L^1(\R^n)$.}
\]
Moreover, for all translations $h\in \R^n$ there results
\[
\Psi \ast (\tau_h f) = \tau_h (\Psi \ast f).
\]
\epr


\bd[$C_{p, \, q}$-spaces]
The $(p, \, q)$-convolution space is defined as
\[
C_{p, \, q} := \left\{ T : L^p(\R^n) \to L^q(\R^n) \: : \: \text{$T$ linear, continuous and translation-invariant} \right\}
\]
if $p$ is finite, and
\[
C_{\infty, \, q} := \left\{ T : C_0(\R^n) \to L^q(\R^n) \: : \: \text{$T$ linear, continuous and translation-invariant} \right\}
\]
otherwise.
\ed

\brmk
We proved earlier that we can associate to each element $T$ of $C_{p, \, q}$ a unique tempered distribution $\Psi_T$ in such a way that
\[
Tf = \Psi_T \ast f.
\]
This induces an isomorphism between $C_{p, \, q}$ and a subset of $\Scp$; more precisely,
\[
C_{p, \, q} = \left\{ \Phi \in \Scp \: : \: \text{$\|\Phi \ast f \|_{L^q(\R^n)} \lesssim_{p, \, q} \|f\|_{L^p(\R^n)}$ for all $f \in \Sc$} \right\}. 
\]
\ermk

Notice that \hyperref[proposition:exatra213]{Proposition \ref{proposition:exatra213}} can be reformulated by saying that the convolution space $C_{1, \, q}$ is isomorphic to $L^q(\R^n)$ for all $1 < q < \infty$. Indeed, by Young's inequality
\[
\| \Psi \ast f \|_{L^q(\R^n)} \leq \|f\|_{L^1(\R^n)} \|\Psi\|_{L^q(\R^n)},
\]
which, in turn, implies that the operator $\Psi \ast f$ is continuous from $L^1(\R^n)$ to $L^q(\R^n)$ as long as $\Psi$ belongs to $L^q(\R^n)$.

Unfortunately, {\bf Banach-Alaoglu's theorem} does not apply when $q = 1$, but it seems plausible to expect that measures will take the place of $L^q(\R^n)$.

\bpr \label{extra.propositoasdjsakde}
Let $\cM(\R^n)$ be the set of finite complex-valued Borel measures. Then
\[
C_{1, \, 1} \cong \cM(\R^n).
\]
\epr

\begin{exercise}
Prove \hyperref[extra.propositoasdjsakde]{Proposition \ref{extra.propositoasdjsakde}}.
\end{exercise}

\begin{proof}[Hint]
First, show that $L^1(\R^n)$ is isometrically contained in $\cM(\R^n)$, which is the dual space of $C_0(\R^n)$. Next, use it to find a subsequence converging to some $\mu \in \cM(\R^n)$ such that
\[
\|\mu\|_1 \leq \|T\|.
\]
Finally, the same argument used above shows that $C_{1, \, 1} \cong \cM(\R^n)$.
\end{proof}

To investigate further properties of the $C_{p, \, q}$-spaces we need a technical lemma, which asserts that $L^p$-functions behave in a "nice" way with respect to infinite translations.

\bl
Let $f \in L^p(\R^n)$, $1 \leq p < \infty$. Then
\begin{equation} \label{eq.2.2.1}
\lim_{|h| \to + \infty} \| \tau_h f - f \|_{L^p(\R^n)} = 2^{\frac{1}{p}} \|f\|_{L^p(\R^n)}. \end{equation}
Similarly, if $f \in C_0(\R^n)$, then
\begin{equation} \label{eq.2.2.2}
\lim_{|h| \to + \infty} \| \tau_h f - f \|_\infty =  \|f\|_\infty.
\end{equation}
\el

\begin{proof}
First, assume that $f \in C_c(\R^n)$. If $|h|$ is large enough, the support of $f$ and $\tau_h f$ are necessarily disjoint, and thus
\[
\|\tau_h f - f \|_{L^p(\R^n)}^p = \int_{\mathrm{spt}(f)} |f(x)|^p \, \mathrm{d}x + \int_{\mathrm{spt}(f) + h} |f(x-h)|^p \, \mathrm{d}x = 2 \|f\|_{L^p(\R^n)}^p. 
\]
Taking the $p$th square root, we obtain \eqref{eq.2.2.1}. For a generic $f \in L^p(\R^n)$, we consider the restrictions $f_r := f \chi_{B_r}$ and, given $\epsilon > 0$, we choose $r$ in such a way that
\[
\|f_r - f \|_{L^p(\R^n)} \leq \epsilon. 
\]
Finally, if $|h|$ is big enough (e.g., $|h| > 2r$ is enough), we infer that
\[ \begin{aligned}
\left| \|\tau_h f - f \|_{L^p(\R^n)} - 2^{\frac{1}{p}} \|f\|_{L^p(\R^n)} \right| & \leq \left| \| \tau_h f - f \|_{L^p(\R^n)}- \|\tau_h f_r - f_r\|_{L^p(\R^n)} \right| + \dots
\\[1em] & \dots + 2^{\frac{1}{p}} \left| \|f_r\|_{L^p(\R^n)} - \|f\|_{L^p(\R^n)} \right| \leq
\\[1em] & \leq \| \tau_h(f - f_r) - (f - f_r) \|_{L^p(\R^n)} + 2^{\frac{1}{p}} \|f - f_r\|_{L^p(\R^n)} \leq
\\[1em] & \leq (2 + 2^{\frac{1}{p}}) \|f_r\|_{L^p(\R^n)} \leq 4 \epsilon.
\end{aligned} \]
\end{proof}

\bthm \label{thm.cpqvuoti}
For all $q < p$ there results $C_{p, \, q} = \{0\}$.
\ethm

\begin{proof}
Suppose that there exists $T \in C_{p, \, q}$ such that $T \neq 0$. Then
\[
\| Tf \|_{L^q(\R^n)} \leq \|T\| \|f\|_{L^p(\R^n)}
\]
for all $f \in L^p(\R^n)$. Take $g := \tau_h f - f$ and exploit the linearity of $T$ to obtain the estimate
\[
\| T (\tau_h f) - Tf \|_{L^q(\R^n)} \leq \|T\| \|\tau_h f - f \|_{L^p(\R^n)}.
\]
Since $T$ is translation-invariant, the left-hand side can be rewritten as
\[
\| \tau_h (Tf) - Tf \|_{L^q(\R^n)}.
\]
Now let $|h|$ go to $\infty$ and apply \eqref{eq.2.2.1}. It turns out that
\begin{equation} \label{extraoekoakd,}
2^{\frac{1}{q}} \|Tf\|_{L^q(\R^n)} \leq 2^{\frac{1}{p}} \|T\| \|f\|_{L^p(\R^n)}.
\end{equation}
The operator norm $\| T \|$ is defined as the supremum of $\| T f \|_{L^q(\R^n)}$ as $f$ ranges in the closed ball of radius one in $L^p(\R^n)$. Thus taking the supremum in \eqref{extraoekoakd,} leads to
\[
 \|T\| \leq 2^{\frac{1}{p} - \frac{1}{q}} \|T\|, 
 \]
and this is absurd because $2^{\frac{1}{p} - \frac{1}{q}}$ is strictly less than one.
\end{proof}

\bthm
The space $C_{p, \, q} $ is isometrically isomorphic to $C_{q^\prime, \, p^\prime}$, where
\[
\frac{1}{p} + \frac{1}{p^\prime} = \frac{1}{q} + \frac{1}{q^\prime} = 1.
\]
\ethm

\begin{proof}
Let $T \in C_{p, \, q}$ and denote its dual operator by
\[
T^\prime : L^{q^\prime}(\R^n) \longrightarrow L^{p^\prime}(\R^n).
\]
At this point, it is not clear whether or not $T \, \big|_{\Sc}$ coincides with $T^\prime \, \big|_{\Scp}$. Consider the bilinear form $B_p : L^p(\R^n) \times L^{p^\prime}(\R^n) \to \C$ given by
\[
B_p(f, \, g) := \int_{\R^n} f(x) g(-x) \, \mathrm{d}x.
\]
This characterise, for all $1 \leq p \leq \infty$, the $L^p$-norm of $f$ via duality as follows:
\[
\|f\|_{L^p(\R^n)} = \sup_{\|g\|_{p^\prime} \leq 1} B_p(f, \, g).
\]
In particular, the dual operator $T^\prime$ can be defined via this bilinear form in such a way that $T^\prime g$ is the unique element satisfying
\[
B_p(f, \, T^\prime g) = B_q(Tf, \, g) \quad \text{for all $f \in L^p(\R^n)$}.
\]
Thanks to \autoref{thm.cpqvuoti}, we can assume $p \leq q$, $q > 1$ and $p < \infty$. Then
\[ \begin{aligned}
\|T^\prime\| & = \sup_{\substack{g \in \Sc\\[.3em] \|g\|_{L^{q^\prime}(\R^n)} \leq 1}} \|T^\prime g \|_{L^{p^\prime}(\R^n)} =
\\[1em] & =  \sup_{\substack{f, \, g \in \Sc \\[.3em] \|f\|_{L^p(\R^n)}, \|g\|_{L^{q^\prime}(\R^n)} \leq 1}} |B_p(f, \, T^\prime g)| =
\\[1em] & = \sup_{\substack{f, \, g \in \Sc \\[.3em] \|f\|_{L^p(\R^n)}, \|g\|_{L^{q^\prime}(\R^n)} \leq 1}} |B_q(Tf, \, g)| = \|T\|.
\end{aligned} \]
Now, if $Tf = K \ast f$, for $f, \, g \in \Sc$ we have
\[ \begin{aligned}
B_p(f, \, T^\prime g) & = B_q(Tf, \, g) =
\\[1em] & = \int_{\R^n} (K \ast f)(x) g(-x) \, \mathrm{d}x =
\\[1em] & = \langle K \ast f, \, \check{g} \rangle =
\\[1em] & = \langle f, \, \check{K} \ast \check{g} \rangle = B_p(f, \, Tg).
\end{aligned}
\]
This shows that $T = T^\prime$ and concludes the proof. The case $p = q = \infty$ and $p = q = 1$ are left to the reader as an exercise.
\end{proof}

\bthm
The space $C_{2, \, 2} $ is isometrically isomorphic to $\F(L^\infty(\R^n))$.
\ethm

\begin{proof}
Let $T \in C_{2, \, 2}$. Then there exists $\Phi \in \Scp$ such that $T$ is the convolution operator associated to $\Phi$. In particular,
\[
Tf = \Phi \ast f,
\]
which, taking the Fourier transform, leads to
\[ 
\F(Tf) = \F(\Phi) \F(f).
\]
Denote $\F(\Phi)$ by $M$ (which stands for {\em multiplicative operator}) and apply \autoref{planchh} to infer that $M$ is bounded on $L^2(\R^n)$ with
\[
\| M \| = \|T \|.
\]
Therefore, given $f \in \Sc$ it is easy to verify that $M f \in L^2(\R^n)$ and
\[
\| M f \|_{L^2(\Omega)} \leq \|T\| \|f\|_{L^2(\Omega)}.
\]
Let $\psi$ be a cutoff function with compact support, $\chi_{B_1} \leq \psi \leq \chi_{B_{\frac{3}{2}}}$, and call $\psi_j(\xi)$ the rescaling $\psi(2^{-j} \xi)$. The functions $m_j := M \psi_j$ belong to $L^2(\R^n)$ and
\[
m_j m_{j+1} = m_j 
\]
holds for all $j \in \N$. We can easily construct $m \in L_{\mathrm{loc}}^2(\R^n)$ which coincides with each $m_j$ on $B_{2^j}$ so $Mf = mf$ for all $f \in C_c^\infty(\R^n)$. We now claim that
\[
m \in L^\infty(\R^n) \quad \text{and} \quad \|m\|_\infty \leq \|M\|.
\]
We argue by contradiction. Suppose that there is $\delta > 0$ such that $|m| > \|M\|+\delta$ on a set $E$ of positive measure (which we may assume to be bounded). If we take
\[
f_j \in C_c^\infty \: : \: \| f_j - \chi_E \|_{L^2(\R^n)} \xrightarrow{j \to \infty} 0, \]
then
\[
M \chi_E = \lim_{j \to + \infty} M f_j = m \chi_E.
\]
It follows that $\|m \chi_E\|_{L^2(\R^n)} > (\|M\| + \delta) \|\chi_E\|_{L^2(\R^n)}$, which is absurd. Conversely, given $m \in L^\infty(\R^n)$, the associated operator is given by
\[
Tf = (\F^{-1} m) \ast f,
\]
and it is bounded on $L^2(\R^n)$ by Plancherel's identity, with $\|T\| \leq \|m\|_\infty$.
\end{proof}

\bcor \mbox{}
\begin{enumerate}[label=\textbf{(\alph*)}]
\item If $1 < p < q < 2$, then $C_{1, \, 1} \subset C_{p, \, p} \subset C_{q, \, q} \subset C_{2, \, 2}$.
\item If $\frac{1}{p} - \frac{1}{q} = \frac{1}{r^\prime}$, then $L^(\Omega) \subset C_{p, \, q}$.
\end{enumerate}
\ecor

\section{Differential operators and fundamental solutions}

In this section, we focus on a special class of operators, which are usually known as {\em differential operators}\index{differential operators}. The main goal is to study existence of solutions to the equation
\[
Lu = f
\]
passing via {\em fundamental solutions}, which is a natural notion arising from the fact that we can study the same problem in the phases space via Fourier transform,
\[
\F(Lu) = \F(f).
\]
Thus, if we can find a solution of the transformed equation, we can obtain a solution of the initial problem via the inverse Fourier transform (when it is invertible).

 
\bd[Differential operator] \index{differential operator}
Let $\alpha \in \N^n$ be a multi-index. The $\alpha$-derivative of a function $f$ is defined by setting
\[
\partial^\alpha f(x) = \partial_{x_1}^{\alpha_1} \dots \partial_{x_n}^{\alpha_n} f(x).
\]
A {\em linear differential operator} $L$ is a linear combination of derivatives, that is,
\[
(Lf)(x) = \sum_{|\alpha| \leq m} c_\alpha(x) \partial^\alpha f(x),
\]
where $c_\alpha : \R^n \to \R$ are real-valued functions. If all $c_\alpha$ are constant, we can associate a {\em polynomial} to $L$ which is given by
\[
P(\xi) := \sum_{|\alpha| \leq m} c_\alpha (\imath \xi)^\alpha,
\]
and the corresponding principal part
\[
P_m(\xi) := \sum_{|\alpha| = m} c_\alpha (\imath \xi)^\alpha.
\]
\ed

Notice that $L$ is translation-invariant and, if the $c_\alpha$ are all constant, it sends $\Sc$ into $\Sc$. Therefore, there exists a tempered distribution $\Phi$ such that
\begin{equation} \label{eq.a.1}
Lf = f \ast \Phi.
\end{equation}
A simple computation shows that $\Phi$ must be a linear combination of derivatives of the Dirac delta centred at the origin; more precisely, we have
\[
\Phi = \sum_{|\alpha| \leq m} c_\alpha \partial^\alpha \delta_0.
\]

\bd[Fundamental solution] \index{fundamental solution}
Let $L$ be a linear differential operator. We say that a distribution $\Psi \in \D^\prime(\R^n)$ is a {\em fundamental solution} for $L$ if
\begin{equation} \label{eq.a.2} L\psi = \delta_0. \end{equation}
\ed

The existence of a fundamental solution is one of the critical features in the theory of differential operators. In the next proposition, we explain why this is the case.

\bpr Let $L$ be a linear differential operator and let $\Psi$ be a fundamental solution. Then for any given $f \in C_c^\infty(\R^n)$ it turns out that
\[
L ( \Psi \ast f) = f \implies \text{$u = \Psi \ast f$ is a solution}.
\]
\epr

\begin{proof}
It is clearly sufficient to show that
\begin{equation} \label{eq.a.3}
L(\Psi \ast f) = (L\Psi) \ast f.
\end{equation}
For a test function $g \in \D(\R^n)$ it turns out that
\[
\begin{aligned} \langle L(\Psi \ast f), \, g \rangle & = \langle \Psi \ast f, \, \transp{L} g \rangle =
\\[1em] & = \langle \Psi, \, \transp{L}(\check{f} \ast g) \rangle =
\\[1em]& = \langle L \Psi, \, \check{f} \ast g \rangle = \langle (L \Psi) \ast f, \, g \rangle,
\end{aligned}\]
where $\transp{L}$ denotes the transpose operator. This shows that \eqref{eq.a.3} holds and concludes the proof.
\end{proof}

However, finding a fundamental solution for an arbitrary operator $L$ is no easy task. Let us consider a linear differential operator of the form
\[
(Lf)(x) = \sum_{|\alpha| \leq m} c_\alpha \partial^\alpha f(x),
\]
and suppose that we have a fundamental solution $\Psi \in \Scp$. Since its Fourier transform is well-defined, it makes sense to compute
\[
\langle \F(L\Psi), \, f \rangle =  \langle L \Psi \, \F f \rangle =  \langle\Psi, \, \transp{L}\F f \rangle.
\]
One can check that the transpose operator $\transp{L}$ is given by
\[ \transp{L}f(x) = \sum_{|\alpha| \leq m} (-1)^{|\alpha|}c_\alpha \partial^\alpha f(x).
\]
We can apply $\transp{L}$ to the Fourier transform of $f$ and find the following chain of equalities:
\[ \begin{aligned}
\transp{L} \F f(x) & = \sum_{|\alpha| \leq m} (-1)^{|\alpha|}c_\alpha \partial_x^\alpha \int_{\R^n} f(\xi) \e^{-\imath \xi \cdot x} \, \mathrm{d}\xi =
\\[1em] & = \sum_{|\alpha| \leq m} (-1)^{|\alpha|}c_\alpha \int_{\R^n} f(\xi) (-\imath \xi)^\alpha \e^{-\imath \xi \cdot x} \, \mathrm{d}\xi =
\\[1em]& = \int_{\R^n} f(\xi) P(\xi) \e^{\imath \xi \cdot x} \, \mathrm{d}\xi = \F (fP)(x),
\end{aligned} \]
where $P$ is the polynomial associated to the operator $L$, that is,
\[
P(\xi) := P_L(\xi) =\sum_{|\alpha| \leq m} c_\alpha( \imath \xi)^\alpha.
\]
It follows that
\[
\langle \F(L\Psi), \, f \rangle = \langle \Psi, \, \F(fP) \rangle = \langle P \F \Psi, \, f \rangle,
\]
which (using the fact that $\Psi$ is a fundamental solution) immediately leads to
\begin{equation} \label{eq.a.5}
P \F (\Psi) = 1.
\end{equation}
The naïve approach would suggest to simply take $\F \Psi$ as $\frac{1}{P}$ but, as we will see in the next examples, this is not always possible (because $P$ might fail to be entire - see \autoref{pwtheroem}).

\bex[Laplace operator] \index{Laplace operator}
Let $L := \Delta$ be the Laplace operator on $\R^n$, i.e.,
\[
Lf(x) = \left( \partial_{x_1}^2 + \dots + \partial_{x_n}^2 \right) f(x).
\]
The associated polynomial is $P(\xi) = - |\xi|^2$ and, clearly, we cannot define $\F \Psi$ as above because
\[
\left\langle - \frac{1}{|\cdot|^2}, \, f \right\rangle = - \int_{\R^n} \frac{f(x)}{|x|^2} \, \mathrm{d}x
\]
is not always a well-defined distribution. Indeed, if $n \geq 3$ then everything works fine, but for $n = 2$ the functional is not continuous.
\eex

\bex[Heat operator] \index{heat operator}
Let $L := \partial_t - \Delta_{\R^n}$ be the {\em heat operator} in $\R_t \times \R_x^n$. Then the associated polynomial is
\[
P(\xi) = \imath \xi_0^2 + |\xi|^2 = \imath \xi_0^2 + \sum_{j = 1}^n \xi_j^2.
\]
In this case, the reciprocal of $P$ is well-defined and it allows us to define $\F \Psi$ as above, obtaining a distribution that is a fundamental solution for $L$.
\eex

\bex[Wave operator] \index{wave operator}
Let $L := \partial_t^2 - \Delta_{\R^2}$ be the {\em wave operator} in $\R_t \times \R_x^2$. Then the associated polynomial is
\[
P(\xi) = -\xi_0^2 + \xi_1^2 + \xi_2^2.
\]
This polynomial vanishes on the whole cone
\[
\{ \xi \in \R^3 \: : \: \xi_0^2 = \xi_1^2 + \xi_2^2 \},
\]
and hence we cannot expect, in general, that the reciprocal defines the Fourier transform of a distribution.
\end{example}

That said, it is possible to show the existence of a fundamental solution for a special class of linear operators; more precisely, the ones with constant coefficients.

\bthm[Malgrange–Ehrenpreis] \index{Malgrange–Ehrenpreis theorem}
Let $L$ be a linear differential operator with constant coefficients. Then there exists $\Phi \in \D^\prime(\R^n)$ satisfying \eqref{eq.a.2}.
\ethm

%Suppose that $\Psi \in \Scp$ and $P \neq 0$ so that $\hat{\Psi} = \frac{1}{P}$ makes sense. Then
%\begin{equation*}\langle \Psi, \, f \rangle = \Psi \ast \check{f}(0) = \frac{1}{(2\pi)^n} \int_{\R^n} \widehat{\Psi \ast \check{f}}(\xi) \, \mathrm{d}\xi =\frac{1}{(2\pi)^n} \int_{\R^n}\frac{\hat{f}(-\xi)}{P(\xi)} \, \mathrm{d}\xi \end{equation*}
%and, if we take$f \in \mathcal{D}(\R^n)$, then we can use \hyperref[pwtheroem]{Theorem \ref{pwtheroem}} to infer that $f$ is entire on the complex extension $\C^n \supset \R^n$.

\begin{proof}[Sketch of the proof]
Let $P$ be the polynomial associated to the linear differential operator $L$. Then we can rewrite it as
\[ P(\xi) = \sum_{|\alpha| \leq m} a_\alpha \xi^\alpha,
\]
where $a_\alpha = \imath^\alpha c_\alpha$. In a similar fashion, we can define
\[ 
P_m(\xi) = \sum_{|\alpha| = m} a_\alpha \xi^\alpha,
\]
to be the principal part of $P$.

\paragraph{Step 1.} In particular, there must be a point $\bar{\xi} \in \R^n$ such that $P_m(\bar{\xi}) \neq 0$ and, up to a change of variables, we can always assume that
\[
\text{$\bar{\xi} = (1, \, \mathbf{0})$ and $P_m(\bar{\xi}) = 1$}.
\]
It follows that
\[
P_m(\xi) = \xi_1^m + \sum_{j=1}^{m-1} \xi_1^j q_{m - j}(\xi^\prime),
\]
where $\xi^\prime = (\xi_2, \, \dots, \, \xi_n) \in \R^{n-1}$ and $q_{m-j}$ is a $(m-j)$-homogeneous multinomial.

\paragraph{Step 2.} Now fix $\xi^\prime$ and denote by $z_1(\xi^\prime), \, \dots, \, z_p(\xi^\prime)$ the distinct roots, with multiplicities $m_j(\xi^\prime)$, of the equation
\[
\xi_1^m + \sum_{j=1}^{m-1} \xi_1^j q_{m - j}(\xi^\prime) = 0.
\]
A further change of variables, namely $\Gamma(\xi) := (\xi_1, \, \xi^\prime) + (\imath \varphi(\xi^\prime), \, 0)$, is necessary to avoid points where $P_m$ vanishes. The naïve idea would be to choose $\varphi$ in such a way that
\[
|\varphi(\xi^\prime) - z_j(\xi^\prime)| > 1 \quad \text{for all $j \in \{1, \, \dots, \, n\}$}.
\]
Let $B_j(\xi^\prime)$ be the ball centred at $z_j(\xi^\prime)$ with radius $\epsilon_j > 0$ in such a way that
\[
B_j(\xi^\prime) \cap B_\ell(\xi^\prime) = \varnothing \quad \text{for all $j \neq \ell$}.
\]
Applying {\bf Rouché's theorem}, we conclude that for a given $\xi^{\prime \prime}$, close enough to $\xi^\prime$, the roots of $P_m(\cdot, \, \xi^{\prime \prime})$ remain confined inside the collection of balls $\{ B_j \}$. Now define
\[
\text{$\varphi$ constant function satisfying $\varphi(\xi^\prime) := a \in [1, \, m+1]$},
\]
so that the estimate
\[
|a - \mathfrak{Im}(z_j(\xi^{\prime \prime}))| > 1
\]
holds for all $\xi^{\prime \prime} \in V$, where $V$ is the small neighbourhood of $\xi^\prime$ mentioned before. Now $\Gamma$ allows us to avoid all the zeros of the polynomial, and thus we can write
\begin{equation} \label{eq.c.1}
\langle \Phi, \, g \rangle = \int_{\R^n} \frac{ \F g(- \xi_1 - \imath \varphi(\xi^\prime), \, - \xi^\prime)}{P(\xi_1 + \imath \varphi(\xi^\prime), \, \xi^\prime)} \, \mathrm{d}\xi
\end{equation}
for any compactly supported function $g \in \D(\R^n)$. But now
\[
 \left| P(\xi_1 + \imath \varphi(\xi^\prime), \, \xi^\prime) \right| = \prod_{j = 1}^m \left| \xi_1 + \imath \varphi(\xi^\prime) - z_j(\xi^\prime)\right| \geq \prod_{j = 1}^m \left| \varphi(\xi^\prime) - \mathfrak{Im}(z_j(\xi^\prime)) \right| > 1,
 \]
so the denominator on the right-hand side of \eqref{eq.c.1} is smaller than the constant $1$. In a similar fashion, we can estimate the numerator as follows:
\[ \begin{aligned}
\left| \F g(- \xi_1 - \imath \varphi(\xi^\prime), \, - \xi^\prime) \right| & = \left| \int_{\R^n} g(x_1, \, x^\prime) \e^{\imath (x_1 (\xi_1 + \imath \varphi(\xi^\prime)) + x^\prime \xi^\prime} \, \mathrm{d}x \right| \leq
\\[1em] & \leq \int_{\R^n} |g(x_1, \, x^\prime)| \e^{x_1 \varphi(\xi^\prime)} \, \mathrm{d}x_1 \mathrm{d}x^\prime \leq
\\[1em] & \leq \|g\|_\infty r^n \e^{r |\varphi(\xi^\prime)|}.
\end{aligned} \]
Notice that the last inequality follows from \autoref{pwtheroem} because $g$ belongs to $\D(\R^n)$ and thus we can find $r > 0$ such that $\mathrm{spt}(g) \subset B_r$. More in general, there results
\[
|z^\alpha \F g (z)| \lesssim_\alpha \| \partial^\alpha g \|_\infty r^n \e^{r |\mathfrak{Im}(z)|}.
\]
Combining this with the estimate above leads to
\[
\left| \F g (- \xi_1 - \imath \varphi(\xi^\prime), \, - \xi^\prime) \right| \leq \|g\|_{(N)} (1+|\xi|)^{-N} r^n \e^{r |\varphi(\xi^\prime)|}
\]
and, if we choose $N \in \N$ big enough, we readily find that the right-hand side of \eqref{eq.c.1} is an absolutely convergent integral. More precisely, we have
\[
|\langle \Phi, \, g \rangle| \lesssim_r \|g\|_{(N)},
\]
which means that $\Phi$ is continuous (hence belongs to $\D^\prime(\R^n)$), and this concludes the proof.
\end{proof}

To conclude this section, we state a stronger result concerning the existence of fundamental solutions which is due to Łojasiewicz-H\"{o}rmander.

\bthm
[Łojasiewicz-H\"{o}rmander] \index{Lojasiewicz-Hormander theorem} 
Let $L$ be a linear differential operator with constant coefficients. Then there exists $\Phi \in \Scp$ satisfying \eqref{eq.a.2}.
\ethm

The proof of this result is rather involved. The reader may refer to \cite{krpa}, where a simple proof is presented, and the reference therein of the original works by Łojasiewicz and H\"{o}rmander.

\subsection{Applications of Malgrange–Ehrenpreis theorem}

Let $L$ be a linear differential operator with constant coefficients and $\Omega \subset \R^n$ an open set. We are interested in finding a solution of the problem
\begin{equation} \label{eq.c.2}
Lu = f
\end{equation}
when $f$ belongs to $\D^\prime(\Omega)$. However, all we can do is to find a solution in any bounded subsets $\Omega^\prime$ which is relatively compact in $\Omega$. To do it, consider a cutoff function
\[
\varphi \in \D(\Omega) \: : \:  \varphi \, \big|_{\Omega^\prime} \equiv 1
\]
and let $f_0 := \varphi f$. Clearly, $f_0$ is still a distribution that coincides with $f$ on $\Omega^\prime$, but it has a big advantage; namely, its support is compact. Thus
\[
L(\Phi \ast f_0) = f_0 \implies \text{$u = \Phi \ast f_0$ is a solution of \eqref{eq.c.2} in $\Omega^\prime$}.
\]

\bcor
Linear differential operators with constant coefficients are locally solvable\footnote{In this set of notes, locally solvable means that we can find a solution in any relative compact subset. However, this notation is not standard and the reader may find a completely different notion of locally solvable in most books.}.
\ecor

We will now show that if we drop the assumption on the coefficients, even first-order operators with complex coefficients may not be locally solvable.

\bthm
[Lewy] Let $f \in C^1(\R, \, \R)$ and consider the problem
\begin{equation}\label{lewy}
A u(z, \, t) = f^\prime(t),
\end{equation}
where $A$ is the first-order linear differential operator defined by
\[
A = 2 \partial_{\bar{z}} - 2 \imath z \partial_t.
\]
Suppose that \eqref{lewy} has a solution of class $C^1$ defined in a neighbourhood of the form
\[
D(0, \, \delta) \times (-\epsilon, \, \epsilon).
\]
Then $f$ must be analytic in the interval $(- \epsilon, \, \epsilon)$. 
\ethm

\brmk
There are no solutions even in the distributional sense for some initial data $\psi$. In particular, the operator $A$ is not locally solvable in a neighbourhood of the origin. 
\ermk

\begin{proof}
We will try to follow the original argument given in \cite{hanlevi}, although the complex variable $z$ will usually be replaced by $(x, \, y)$. Using polar coordinates, we have
\[
x + \imath y = r^{\frac{1}{2}} \e^{\imath \theta},
\]
so that we can rewrite $f$ as
\[
f(x, \, y) = f(r^{\frac{1}{2}}\cos\theta, \, r^{\frac{1}{2}} \sin \theta) =: g_f(r,\, \theta).
\]
A simple computation shows that the partial derivatives in this new coordinate system are given by
\[ \begin{aligned}
& \partial_r g_f(r, \, \theta) = \frac{1}{2} r^{-\frac{1}{2}} \left(\cos\theta \partial_x f + \sin \theta \partial_y f \right),
\\[1em] & \partial_\theta g_f(r, \, \theta) = r^{\frac{1}{2}} \left(-\sin\theta \partial_x f + \cos \theta \partial_y f \right),
\end{aligned}\]
and this readily leads to the Jacobian matrix of the change of variables:
\[ \begin{aligned}
& r^{\frac{1}{2}} \partial_x f = (2r \cos \theta \partial_r - \sin \theta \partial_\theta) g_f,
\\[1em] & r^{\frac{1}{2}} \partial_y f = (2r \sin \theta \partial_r + \cos\theta \partial_\theta) g_f.
\end{aligned}\]
The complex derivative $\partial_{\bar{z}}$ is thus given by
\[
2\partial_{\bar{z}} = \partial_x +\imath \partial_y = 2 r^{\frac{1}{2}} \e^{\imath \theta} \partial_r + \imath r^{-\frac{1}{2}} \e^{\imath \theta} \partial_\theta,
\]
except at the point $(0, \, 0)$ where this is not well-defined since $r^{-\frac{1}{2}}$ is singular. We now exploit the polar coordinates to infer that
\[ \begin{aligned}
\int_0^{2\pi} \left[(\partial_x + \imath \partial_y)f\right](r^{\frac{1}{2}} \e^{\imath \theta}) \, \mathrm{d}\theta & = \int_0^{2\pi} \left[  r^{\frac{1}{2}} \e^{\imath \theta} \partial_r + \imath r^{-\frac{1}{2}} \e^{\imath \theta} \partial_\theta \right] g(r, \, \theta) \, \mathrm{d}\theta \, {\color{orange}=}
\\[1em] & \, {\color{orange}=} \int_0^{2 \pi} [ r^{\frac{1}{2}} \e^{\imath \theta} \partial_r g ](r, \, \theta) \, \mathrm{d}\theta + r^{-\frac{1}{2}} \int_0^{2 \pi} g(r, \, \theta) \e^{\imath \theta} \, \mathrm{d}\theta =
\\[1em] & = \int_0^{2\pi} \e^{\imath \theta} \left[ 2 r^{\frac{1}{2}} \partial_r +  r^{-\frac{1}{2}} \right] g(r, \,\theta) \, \mathrm{d}\theta =
\\[1em] & = 2 \int_0^{2\pi} \e^{\imath \theta} \partial_r(  r^{\frac{1}{2}} g )(r, \, \theta) \, \mathrm{d}\theta =
\\[1em] & 2 \partial_r \int_0^{2\pi} r^{\frac{1}{2}} g(r, \, \theta) \e^{\imath \theta} \, \mathrm{d}\theta
\end{aligned} \]
where the {\color{orange}orange} identity follows from integration by parts of $\e^{\imath \theta} \partial_\theta g$. Now set
\[
U(t, \, r) := \int_0^{2\pi} r^{\frac{1}{2}} \e^{\imath \theta} u(r^{\frac{1}{2}} \e^{\imath \theta}, \, t) \, \mathrm{d}\theta,
\]
and notice that
\[ \begin{aligned}
\partial_t U + \imath \partial_r U & = \int_0^{2\pi} r^{\frac{1}{2}} \e^{\imath \theta} \partial_t u(r^{\frac{1}{2}} \e^{\imath \theta}, \, t) \, \mathrm{d}\theta + \frac{1}{2} \imath \int_0^{2\pi} \left[(\partial_x + \imath \partial_y)f\right](r^{\frac{1}{2}} \e^{\imath \theta}) \, \mathrm{d}\theta =
\\[1em] & = \imath \int_0^{2 \pi} \left( \partial_{\bar{z}} - \imath z \partial_t \right) u ( r^{\frac{1}{2}} \e^{\imath \theta}, \, t) \, \mathrm{d}\theta.
\end{aligned} \]
Let $F$ be a primitive of $f$. Then the function defined by setting
\[
\tilde{U}(t + \imath r) := U(t, \, r) - 2 \pi \imath F(t)
\]
is holomorphic (since $\partial_{\bar{z}} \tilde{U} = 0$) in $(- \epsilon, \, \epsilon) \times (0, \, \delta^2)$, which means that it extends holomorphically to $z=0$ by setting
\[
\tilde{U}(t) := - 2 \pi \imath F(t).
\]
It follows that $F$ is a real analytic function, and by definition so is its derivative $f$. This concludes the proof of the theorem.
\end{proof}

\brmk
A linear differential operator with constant real-valued coefficient of order one is necessarily locally solvable so we must go to order at least $2$ to find a counterexample.
\ermk
\chapter{Hypoelliptic Operators and H\"{o}rmander Theorem}

In this chapter, we introduce a notion that generalises the one of elliptic operators. Namely, we say that a linear differential operator $L$ with smooth coefficients in $\Omega$ is {\bf hypoelliptic}\index{hypoelliptic operator} if
\[
Lu \in C^\infty(\Omega^\prime) \implies u \in C^\infty(\Omega^\prime)
\]
for all $\Omega^\prime \subset \Omega$. The main goal of this chapter is to state and prove the well-known H\"{o}rmander theorem which gives a sufficient condition for an operator of the form
\[
L = \sum_{j =1}^k X_j^2
\]
to be hypoelliptic, where the $X_j$ are vector fields.

\begin{customthm}{A} Assume that, at each $x \in \Omega$, the vector fields
\[
\{X_j\}_{j = 1, \, \dots, \, k}, \, \{ [X_j, \, X_i] \}_{1 \leq i < j \leq k}, \, \{ [X_j, \, [X_i, \, X_\ell]]\}_{i, \,j, \, \ell}, \, \dots
\]
span $\R^n$. Then the operator $L = \sum_{j = 1}^k X_j^2$ is hypoelliptic. \end{customthm}

\section{Local solvability for constant coefficients operators}

We can characterise hypoelliptic operators via the {\em singular support}, which is well-defined for all distributions in $\D^\prime(\R^n)$.

\bd \index{distribution!singular support}
The {\em singular support} of a distribution $\phi \in \D^\prime(\Omega)$ is defined as the complement of
\[
\left\{ \Omega^\prime \subset \Omega \: : \: \text{$\Omega$ open and $\phi \in C^\infty(\Omega^\prime)$} \right\},
\]
and it will be denoted by $\mathrm{spt}_{\mathrm{sing}}(\phi)$. \end{definition}

\bpr
A linear differential operator $L$ is hypoelliptic if and only if for all $u \in \D^\prime(\Omega)$ the following inclusion holds:
\[
\mathrm{spt}_{\mathrm{sing}}(u) \subseteq \mathrm{spt}_{\mathrm{sing}}(Lu).
\]
\epr

The next result describes the local behaviour of hypoelliptic operators and allows one to estimate the $C^k$-norm of compactly supported functions. Recall that
\[
\|f\|_k := \sum_{|\alpha| \leq k} \| \partial^\alpha f \|_\infty.
\]

\bthm \label{thm.d.1}
Let $L$ be a hypoelliptic operator and fix $x \in \Omega$ and $k \in \N$. There exist a compact neighbourhood $U_k$ of $x$ and $k^\prime = k^\prime(k) \in \N$ such that
\begin{equation} \label{eq.d.1}
\|f\|_{k} \lesssim_k \| Lf \|_{k^\prime} \quad \text{for all $f \in \D_{U_k}$.}
\end{equation}
\ethm

We first need to introduce some notations and state a couple of technical results. For a compact subset $K$ of $\Omega$ and $\nu \in \N$ define
\[
V^\nu(K) := \{ f \in C^\nu(K) \: : \: Lf \in \D_K \},
\]
equipped with the family of norms
\[
\|f\|_{V^\nu, \, k} := \|f\|_\nu + \|Lf \|_k.
\]

\bl
The couple $(V^\nu(K), \, \|\cdot\|_{V^\nu, \, k})$ is a Fréchet space. If in addition $L$ is hypoelliptic, then it coincides with $\D_K$.
\el

\begin{proof}
The first assertion is left to the reader as an exercise. If $L$ is hypoelliptic, then
\[
Lf \in \D_K \implies f \in \D_U \quad \text{for all $U \subset K$ open}.
\]
By compactness, we can find $U_1, \, \dots, \, U_n \subset K$ such that $K = \cup_{i = 1}^n U_i$ and, applying the implication above, yields
\[
Lf \in \D_K \implies f \in \D_K \implies V^\nu(K) = \D_K.
\]
\end{proof}

\bl
Let $f \in C^\infty(\Omega)$ with $\mathrm{spt}(f) \subset B(r)$. Then
\begin{equation} \label{eq.1.11.1}
\|f\|_k \leq 2r \|f\|_{k+1} \quad \text{for all $k \in \N$}.
\end{equation}
\el

\begin{proof}
To simplify the notations, we can use the equivalent norm
\[
\|f\|_k := \|f\|_\infty + \max_{|\alpha| = k} \| \partial^\alpha f \|_\infty.
\]
We can assume that the support of $f$ is contained in the cube $Q := [-r, \, r]^n$. For any multi-index $\alpha$ of length less than or equal to $k$, let $\alpha^\prime = \alpha + (1, \, 0, \, \dots, \, 0)$. If $x \in Q$,
\[\begin{aligned}
|\partial^\alpha f(x)| & = \left| \int_{-r}^{x_1} \partial^{\alpha^\prime} f(t, \, x_2, \, \dots, \, x_n) \, \mathrm{d}t \right| \leq
\\[1em] & \leq 2r \| \partial^{\alpha^\prime} f\|_\infty,
\end{aligned}\]
so that passing to the supremum yields \eqref{eq.1.11.1}.
\end{proof}

\begin{proof}[Proof of Theorem \ref{thm.d.1}]
Let $K$ be a compact neighbourhood of $x$ and fix $k \in \N$. The identity map
\[
\iota : \D_K \longrightarrow V^{k-1}(K) 
\]
is continuous and, as a consequence of what we have proved above, also surjective. By the {\bf open mapping theorem} $\iota$ is a homeomorphism. Therefore, the inclusion
\[
j : V^{k-1}(K) \hookrightarrow C^k(K)
\]
is continuous. Consequently, we can find $k^\prime \in \N$ and $C_k > 0$ such that
\[
\|f\|_k \lesssim_k  \|f\|_{k-1} + \|Lf\|_{k^\prime} \quad \text{for every $f \in \D_K$}.
\]
Finally, let $U_k$ be a ball centered at $x$ of radius $\frac{1}{4 C_k}$; it follows from the estimate \eqref{eq.1.11.1} that
\[
\|f\|_k \lesssim_k \|Lf\|_{k^\prime},
\]
and this concludes the proof.
\end{proof}

\bcor
If $L$ is hypoelliptic, then $L$ is injective on $\mathcal{D}(U_k)$.
\ecor

\bd[Locally solvable] \index{locally solvable}
Let $L$ be a linear differential operator with smooth coefficients on $\Omega$. We say that $L$ is \textit{locally solvable} at $x \in \Omega$ if
\[
\forall k \in \N, \, \exists V_k \in \mathcal{N}(x) \: : \: \forall \psi \in \mathcal{D}_k^\prime(\Omega), \, \exists u \in \mathcal{D}^\prime(V_k) \: : \: \text{$Lu = \psi$ on $V_k$}.
\]
In other words, for each $k \in \N$ we can find a neighbourhood of $x$, $V_k$, such that for each compactly supported distribution $\psi$ of order $k$ the equation
\[
Lu = \psi
\]
admits a solution $u \in \D^\prime(V_k)$.
\ed

\bthm
Let $L$ be a hypoelliptic operator. Then $\transp{L}$ is locally solvable. 
\ethm

\begin{proof}
Given $x \in \Omega$, let $U_0$ be a compact neighbourhood of $x$. Since $\psi \in \D^\prime(\Omega)$ is continuous, we know that there are $k \in \N$ and $C >0$ such that
\[
| \langle \psi, \, f \rangle | \leq C \|f\|_{(k)} \quad \text{for all $f \in \mathcal{D}(U_0)$.}
\]
Let $U := U_k$ be given as in \autoref{thm.d.1} and assume that $U \subset U_0$. Let
\[
\X := \{ Lg \: : \: g \in \D_U \} \subseteq \D_U,
\]
and define the linear functional $\lambda : \X \to \C$ by setting
\[
\lambda(Lg) := \langle \psi, \, g \rangle.
\]
This is well-defined because $L$ is injective and, using \autoref{thm.d.1}, we can find $k^\prime \in \N$ which provides a bound to the functional $\lambda$:
\[
\left| \lambda(Lg) \right| \leq C \|g\|_{(k)} \lesssim_k \|Lg\|_{(k^\prime)}. 
\]
Applying Hahn-Banach we can extend $\lambda$ to a continuous linear functional on $\D_{k^\prime, \, U}$. For $g \in \D_U$ there is a distribution $u$ on $V$ of order $k^\prime$ such that
\[
\langle u, \, Lg \rangle = \lambda(Lg) = \langle \psi, \, g \rangle, 
\]
which means that $\transp{L}u = \psi$ on $V := U$.
\end{proof}

Let $L$ be a constant coefficient differential operator with symbol $\sigma$. The following characterisation of hypoelliptic operators on $\R^n$ is due to Hörmander in \cite{hormhyp}.

\bthm
A linear differential operator $L$ with smooth coefficients is hypoelliptic on $\R^n$ if and only if, for some $\delta > 0$, the polynomial $p$ satisfies the inequality
\[
\left| \frac{ \partial^\alpha p(\xi) }{p(\xi)} \right| \leq C |\xi|^{- \delta |\alpha|} \quad \text{for $|\xi|$ sufficiently big,}
\]
for all $\alpha \in \N^n$ with length less than or equal to the degree of $p$ as a polynomial.
\ethm

This condition is satisfied by two fundamental classes of operators which also contains the Laplace operator (first one) and the heat operator (second one): \mbox{}
\begin{enumerate}[label={\color{cyan}(\roman*)}, leftmargin=2.5\parindent]
\item Elliptic operators whose principal polynomial $p_0$ only vanishes at the origin, e.g., the Laplace operator with
\[
p_0(\xi) = \xi_1^2 + \dots + \xi_n^2.
\]
\item Operators with polynomial only vanishes at the origin and is homogeneous with respect to some non-isotropic
dilations, that is,
\[
p(\xi) = \sum_{\alpha} c_\alpha \xi^\alpha,
\]
where the sum is restricted to multi-indices $\alpha$ satisfying the affine relation $b \cdot \alpha = m$.
\end{enumerate}

\section{H\"{o}rmander theorem}

The primary goal of this section is to characterise hypoelliptic operators of the form $X_j X_j$, where $X_1, \, \dots, \, X_k$ are smooth real vector fields defined on $\Omega$.

\caution{The reader who is not familiar with the notions of vector field, commutator, flow, etc. is encouraged to read \autoref{sec:vff} before going any further.}

\paragraph{Commutators.} Let $X$ denote the vector field on $\R^n$ defined by
\[
X = \sum_{j = 1}^n a_j(x) \partial_{x_j},
\]
where $a_j \in C^\infty(\Omega)$ for all $j$. Let $Y$ be another vector field, given by
\[
Y = \sum_{j = 1}^n b_j(x) \partial_{x_j},
\]
with $b_j \in C^\infty(\Omega)$. We can now compute $XY$ and $YX$ explicitly:
\[ \begin{aligned}
& XYf = \sum_{j, \, k = 1}^n a_j(x) b_k(x) \partial_{x_j} \partial_{x_k} f(x) + \sum_{j, \, k = 1}^n a_j(x) \partial_{x_j} b_k(x) \partial_{x_k} f(x),
\\[1em] & YX f = \sum_{j, \, k = 1}^n a_j(x) b_k(x) \partial_{x_j} \partial_{x_k} f(x) + \sum_{j, \, k = 1}^n b_j(x) \partial_{x_j} a_k(x) \partial_{x_k} f(x).
\end{aligned} \]
This shows that $XY \neq YX$ or, in other words, $X$ and $Y$ do not commute. On the other hand, the commutator between $X$ and $Y$ is given by
\[
[X, \, Y] f(x) = \sum_{k=1}^n \left( \sum_{j=1}^n (a_j(x) \partial_{x_j} b_k(x) - b_j(x) \partial_{x_j} a_k(x)) \right)\partial_{x_k} f(x),
\]
and it is easy to see that it only depends on the value of $a_j$, $b_j$ and $f$ at the point $x$. If $L = \sum_{j = 1}^k X_j^2$, then $L$ is an {\em elliptic} operator on $\Omega$ if and only if
\[
\{ X_j(x) \}_{j = 1, \, \dots, \, k}
\]
spans all of $\R^n$ at all points $x \in \Omega$. In particular, $k$ must be greater than or equal to $n$.

\bthm[H\"{o}rmander] \index{H\"{o}rmander theorem} \label{thm.hormander1}
Assume that, at each $x \in \Omega$, the vector fields
\[
\{X_j\}_{j = 1, \, \dots, \, k}, \, \{ [X_j, \, X_i] \}_{1 \leq i < j \leq k}, \, \{ [X_j, \, [X_i, \, X_\ell]]\}_{i, \,j, \, \ell}, \, \dots
\]
span $\R^n$. Then the operator $L = \sum_{j = 1}^k X_j^2$ is hypoelliptic.
\ethm

\brmk
Under the same assumptions, the operator $\sum_{j = 1}^{k-1} X_j^2 + X_k$ is also hypoelliptic.
\ermk

\bex[Grushin plane] \index{Grushin plane}
Let $X = \partial_x$ and $Y = x \partial_y$ defined on $\R^2$. The operator
\[
L = \partial_x^2 + x^2 \partial_y^2
\]
is {\em elliptic} away from $\{(x,\,y) \in \R^2 \: : \: x = 0\}$. However, the commutator is given by
\[
[X, \, Y] \, \big|_{x = 0} = \partial_y,
\]
and hence $L$ is hypoelliptic on the whole plane $\R^2$, as a consequence of \autoref{thm.hormander1}.
\eex

\bex[Heisenberg space] \index{Heisenberg group}
The operator defined in $\R^3$ by
\[
L = \underbrace{(\partial_x - 2y \partial_z)^2}_{:=X^2} +  \underbrace{(\partial_y + 2x \partial_z)^2}_{:=Y^2}
\]
is hypoelliptic and it is usually referred to as sublaplacian\index{sublaplacian}. Notice that $X$ and $Y$ are always linear independent and
\[
[X, \, Y] = 4 \partial_z \implies \text{$\{ X(x), \, Y(x), \, [X, \, Y](x) \}$ basis of $\R^3$}.
\]
\eex

\section{Besov potential spaces}

In this section, we introduce a few technical tools that are needed to prove the H\"{o}rmander hypoellipticity theorem. To be more precise, given $X$ vector field on $\Omega$, we would like to exploit the flow to define a normed space with
\[
\|f \|_{X, \, \alpha, \, \delta},
\]
which supposedly generalises the usual Lipschitz spaces, and investigate them.

\subsection{Besov spaces of functions}

\bd[Lipschitz] \index{Lipschitz function}
Let $f$ be a function defined on $\R^n$. We say that $f$ is $\alpha$-Lipschitz\footnote{{\bf N.B.} In the literature, functions satisfying this property with $\alpha \in (0,\, 1)$ are called $\alpha$-H\"{o}lder while the terminology Lipschitz is reserved to the special case $\alpha = 1$.}, where $\alpha \in (0, \, 1]$, if
\begin{equation} \label{eq.g.3}
|f(x+h) - f(x)| \lesssim |h|^\alpha \quad \text{for all $x, \, h \in \R^n$.}
\end{equation}
Moreover, we say that $f$ is {\em locally} $\alpha$-Lipschitz if
\[
|f(x+h) - f(x)| \lesssim |h|^\alpha
\]
holds for all $x \in \R^n$ and for all $h$ in a ball which radius depends on $x$, i.e. $B(x, \, r(x))$.
\ed

\brmk
A function which is $\alpha$-Lipschitz with $\alpha > 1$ is constant; this is why we require $\alpha$ to be in $(0, \, 1]$.
\ermk

\brmk
We can also consider functions satisfying the inequality
\begin{equation} \label{eq.g.4}
|f(x+2h) - 2f(x + h) + f(x)| \lesssim |h|^\alpha.
\end{equation}
In this case, $\alpha\in (0, \, 2]$ does not lead to a trivial definition and, for all $\alpha \in (0, \, 1)$ - note that $\alpha = 1$ is excluded! - it is equivalent to \eqref{eq.g.4}. It is easy to verify that
\[
\alpha > 2 \implies \text{$f$ is affine,}
\]
which makes sense if we think about $f(x+2h) - 2f(x + h) + f(x)$ as a {\em good} approximation of the second derivative of $f$.
\ermk

In any case, it is not hard to verify that the condition \eqref{eq.g.3} can be also rewritten using translations as follows:
\[
\sup_{h \in \R^n \setminus \{0\}} |h|^{-\alpha} \| \tau_h f - f \|_\infty < \infty
\]
Starting from this observation, we can define the {\em $\alpha$-Besov space}\index{Besov space} as follows:
\[
\Lambda_\alpha^\infty = \{ f \in L^\infty(\R^n) \: : \: \sup_{h \in \R^n \setminus \{0\}} |h|^{-\alpha} \| \tau_h f - f \|_\infty < \infty \},
\]
endowed with the norm
\[
\|f\|_{\Lambda_\alpha^\infty} := \|f\|_{\infty} +  \sup_{h \in \R^n \setminus \{0\}} |h|^{-\alpha} \| \tau_h f - f \|_\infty.
\]

\bpr
Let $\alpha \in (0, \, 1]$. A function $f$ belongs to $\Lambda_\alpha^\infty$ if and only if $f$ is continuous, bounded and $\alpha$-Lipschitz.
\epr

\begin{proof}
Let $(\varphi_\epsilon)_{\epsilon > 0}$ be an approximate identity with $\mathrm{spt}(\varphi) \subset B_1$. Then
\[
\varphi_\epsilon \ast f(x) - f(x) = \int_{\R^n} [f(x-y) - f(x)] \varphi_\epsilon(y) \, \mathrm{d}y.
\]
Taking the $\| \cdot \|_\infty$ norm immediately leads to the following chain of inequalities:
\begin{equation*} \begin{aligned} \| \varphi_\epsilon \ast f(x) - f(x) \|_\infty & \leq \int_{\R^n} \| \tau_y f - f \|_\infty \varphi_\epsilon(y) \, \mathrm{d}y \leq
\\[1em] & \leq \|f\|_{\Lambda_\alpha^\infty} \int_{B_\epsilon} |y|^\alpha \varphi_\epsilon(y) \, \mathrm{d}y = \epsilon^\alpha \|f\|_{\Lambda_\alpha^\infty}. \end{aligned} \end{equation*}
\end{proof}

We can now introduce a slightly more refined version of Besov spaces, where $L^\infty$ is replaced by $L^p$ and the supremum norm by the $\| \cdot \|_p$ one.

\bd \index{Besov space!$p$-power}
For $1 \leq p \leq \infty$ we define the {\em $p$-Besov space} as
\[
\Lambda_\alpha^p = \{ f \in L^p(\R^n) \: : \: \sup_{h \in \R^n \setminus \{0\}} |h|^{-\alpha} \| \tau_h f - f \|_{L^p(\R^n)} < \infty \},
\]
endowed with the norm
\[
\|f\|_{\Lambda_\alpha^p} := \|f\|_{L^p(\R^n)} +  \sup_{h \in \R^n \setminus \{0\}} |h|^{-\alpha} \| \tau_h f - f \|_{L^p(\R^n)}.
\]
\ed

\brmk
We proved earlier that $f \in L^p(\R^n)$ is enough to conclude that
\[
\| \tau_h f - f \|_{L^p(\R^n)} \xrightarrow{|h| \to 0} 0,
\]
but here we are asking for something more: that the convergence happens with "$|h|^\alpha$-speed".
\ermk

\brmk
The $p$-Besov space is nonempty since we always have that
\[
f_s(x) := |x|^{-s} \chi_{B(0, \, 1)}(x)
\]
is an element of $\Lambda_{\frac{n}{p} - s}^p$ for $s < \frac{n}{p}$.
\ermk

\bd \index{Besov space!$(p, \, q)$-power}
For $1 \leq p \leq \infty$ and $1 \leq q < \infty$ we define the {\em $(p, \, q)$-Besov space} as
\[
\Lambda_\alpha^{p, \, q} = \left\{ f \in L^p(\R^n) \: : \:  \left[ \int_{\R^n} \left( |h|^{-\alpha} \| \tau_h f - f \|_{L^p(\R^n)} \right)^q \, \frac{\mathrm{d}h}{|h|^n} \right]^{\frac{1}{q}} < \infty \right\}, 
\]
endowed with the norm
\[
\|f\|_{\Lambda_\alpha^{p, \, q}} := \|f\|_{L^p(\R^n)} + \left[ \int_{\R^n} \left( |h|^{-\alpha} \| \tau_h f - f \|_{L^p(\R^n)} \right)^q \, \frac{\mathrm{d}h}{|h|^n} \right]^{\frac{1}{q}}.
\]
The $(p, \, \infty)$-Besov space $\Lambda_\alpha^{p, \, \infty}$ is defined in such a way that it corresponds with the $\Lambda_\alpha^p$ defined above.
\ed

\bl
Let $1 \leq p < \infty$. For all $h \in \R^n$ there results
\begin{equation} \label{eq.h.1}
\lim_{|h| \to \infty} \| \tau_h f - f \|_{L^p(\R^n)} =2^{\frac{1}{p}} \|f\|_{L^p(\R^n)}.
\end{equation}
\el

\begin{proof}
First, assume that $f$ is compactly supported in a ball of radius $r$. Then, if $|h| > 2r$, we have
\[
\mathrm{spt}(f) \cap \mathrm{spt} (\tau_h f) = \varnothing.
\]
The $L^p$-norm is now easy to estimate since
\[
\| \tau_h f - f \|_{L^p(\R^n)}^p = \int_{B_r} |f|^p \, \mathrm{d}x + \int_{B_r + h} |\tau_h f|^p \, \mathrm{d}x = 2 \|f\|_{L^p(\R^n)}^p, 
\]
and this concludes the proof since we can always approximate $f \in L^p(\R^n)$ by a sequence of compactly supported functions.
\end{proof}

Now let $a > 0$ be fixed and suppose that $p < \infty$. We would like to estimate the seminorm associated to $\| \cdot \|_{\Lambda_\alpha^{p, \, q}}$ in the range $|h| > a$. From \eqref{eq.h.1} it follows that
\[ \begin{aligned}
\left[ \int_{|h|>a} \left( |h|^{-\alpha} \| \tau_h f - f \|_{L^p(\R^n)} \right)^q \, \frac{\mathrm{d}h}{|h|^n} \right]^{\frac{1}{q}} & \leq \left(2 \|f\|_{L^p(\R^n)} \right)^q \int_{|h| > a} \frac{\mathrm{d}h}{|h|^{n + \alpha q}} \leq
\\[1em] & \leq C_{q, \, \alpha} \|f\|_{L^p(\R^n)}^q.
\end{aligned} \]

\bd[Sobolev space] \index{fractional Sobolev space}
Let $s > 0$. We define the {\em $s$-fractional Sobolev space} as
\[
H^s(\R^n) := \left\{ f \in L^2(\R^n) \: : \: (1 + |\cdot|^2)^s \F(f)(\cdot) \in L^2(\R^n) \right\}.
\]
\ed

We shall now study the connection between the fractional space $H^s(\R^n)$ and the Besov potential space $\Lambda_s^{2, \, 2}$, when $0 <  s = \alpha < 1$. First, recall that
\begin{equation} \label{eq.plan}
\| \F(g) \|_{L^2(\R^n)} = \|g\|_{L^2(\R^n)}
\end{equation}
as a consequence of Plancherel's theorem. It follows that
\[
\| \tau_h f - f \|_{L^2(\R^n)} = \| \F(\tau_h f - f) \|_{L^2(\R^n)},
\]
and we can easily compute the first term via a simple change of variables:
\[
\F(\tau_h f)(\xi) = \mathrm{e}^{- \imath h \cdot \xi} \F f(\xi). 
\]
Now let $f \in \Lambda_\alpha^{2, \, 2}$. It follows from the properties above that
\[ \begin{aligned}
\|f\|_{\Lambda_\alpha^{2, \, 2}} & = \| \F f \|_{L^2(\R^n)} +  \left[ |h|^{- 2\alpha}  \int_{\R^n} \| \mathrm{e}^{-\imath h \cdot \xi} \F f - \F f \|_{L^2(\R^n)}^2 \, \frac{\mathrm{d}h}{|h|^n} \right]^{\frac{1}{2}} =
\\[1em] & = \|\F f \|_{L^2(\R^n)} + \left[ \int_{\R^n} |\F f(\xi)|^2 \, \mathrm{d}\xi \int_{\R^n} | \mathrm{e}^{- \imath h \cdot \xi} - 1|^2 \, \frac{\mathrm{d}h}{|h|^{n+2\alpha}} \right]^{\frac{1}{2}} \stackrel{\star}{=}
\\[1em] & \stackrel{\star}{=} \| \F f \|_{L^2(\R^n)} + \left[ 2 \int_{\R^n} \frac{1 - \cos(h^\prime)}{|h^\prime|^{n + 2 \alpha}} \, \mathrm{d}h^\prime \int_{\R^n} |\xi|^{2 \alpha} |\F f(\xi)|^2 \, \mathrm{d}\xi\right]^{\frac{1}{2}}.
\end{aligned} \]
The identity $\star$ follow from the change of variables $h^\prime := |\xi| h$ which Jacobian is $|\xi|^n$. It is easy to verify that
\[
2 \int_{\R^n} \frac{1 - \cos(h^\prime)}{|h^\prime|^{n + 2 \alpha}} \, \mathrm{d}h^\prime \leq C(n, \, \alpha)
\]
since the integrand is asymptotically equivalent respectively to $|h^\prime|^{2 - n - 2\alpha}$ for $|h^\prime|\to 0$ and to $|h^\prime|^{-n-2 \alpha}$ for $|h^\prime| \to \infty$. It turns out that
\[
\|f\|_{\Lambda_\alpha^{2, \, 2}} \leq \| \F f \|_2 + C(n, \, \alpha) \| |\cdot|^\alpha \F f(\cdot) \|_{L^2(\R^n)} \leq \| f \|_{\Lambda_\alpha^{2, \, 2}}, 
\]
so we can finally conclude that
\begin{equation*} \Lambda_\alpha^{2, \, 2} = H^\alpha(\R^n). \end{equation*}

\bpr \label{prop2oqwdqwple}
Let $0 < \beta < \alpha < 1$. Then the following inclusions hold:
\[
\Lambda_\alpha^{2, \, 2} \subset \Lambda_\alpha^{2, \, \infty} \subset \Lambda_\beta^{2, \, 2}. 
\]
\epr

\begin{proof}
The second inclusion is left as an exercise for the reader. As for the first one, let $f \in \Lambda_\alpha^{2, \, 2}$ be an arbitrary function. We claim that for each $h \in \R^n$ we have
\[
|h|^{-2 \alpha} \| \tau_h f - f \|_{L^2(\R^n)}^2 \leq C |h|^{2 - 2 \alpha} \|f\|_{\Lambda_\alpha^{2, \, 2}}.
\]
However, this follows immediately from the following inequalities:
\[ \begin{aligned}
\| \tau_h f - f \|_{L^2(\R^n)}^2 & \leq c \int_{\R^n} |\F f(\xi)|^2 | \mathrm{e}^{- \imath \xi \cdot h} - 1|^2 \, \mathrm{d}\xi \leq
\\[1em] & \leq c^\prime |h|^2 \int_{\R^n} |\F f(\xi)|^2 |\xi|^2 \, \mathrm{d}\xi,
\end{aligned}\]
where the last one is obtained as before using this time the real part of $\mathrm{e}^{- \imath \xi \cdot h} - 1$. By assumption, the quantity $2 - 2 \alpha$ is always strictly positive so, taking into account that
\[ \begin{aligned}
& \| \tau_h f - f \|_{L^2(\R^n)}^2 \simeq \|f\|_{L^2(\R^n)}^2 \quad \text{as $|h| \to \infty$},
\\[0.8em] &|h|^{-2\alpha}\| \tau_h f - f \|_{L^2(\R^n)}^2 \xrightarrow{|h| \to \infty} 0,
\end{aligned} \]
we readily infer that
\[
\sup_{h \neq 0} |h|^{-2 \alpha} \| \tau_h f - f \|_{L^2(\R^n)}^2 \leq \widetilde{C} \|f\|_{\Lambda_\alpha^{2, \, 2}}.
\]
\end{proof}

\subsection{Besov spaces of vector fields}

Let $\Omega \subset \R^n$ be an open set, $X$ a smooth vector field on $\Omega$ and denote by $\Phi(x, \, t)$ the {\em flow} generated by $X$. For $\alpha \in (0, \, 1]$ and $\delta > 0$ we define
\[
\|f\|_{X,\, \alpha, \, \delta} := \|f\|_{L^2(\Omega)} +  \sup_{|t|\leq \delta} |t|^{-\alpha} \| \exp(tX) f - f \|_{L^2(\Omega)}. 
\]
Notice that $\e^{tX} f(x) = f \circ \Phi_t(x)$, so the quantity above is well-defined provided that $\delta > 0$ is small enough for the flow to be defined {\bf at all $x \in \Omega$}. Since it might happen that
\[
\inf_{x \in \Omega} \left\{ t \in \R \: : \: (x, \, t) \in \mathrm{dom}(\Phi) \right\} = 0,
\]
we usually restrict ourselves to compact subsets of $\Omega$, as the following remark illustrates.

\brmk
If $K \subset \Omega$ is a compact set, then the flow is defined at all $x \in K$ up to a uniform time $\delta_K > 0$. Thus, the quantity
\[
\|f\|_{X,\, \alpha} = \|f\|_{L^2(\Omega)} +  \sup_{|t| \leq \delta_K} |t|^{-\alpha} \| \exp (tX) f - f \|_{L^2(\Omega)} 
\]
is well-defined for all functions $f \in L^2(\Omega)$ with compact support in $K$.
\ermk

\bl
Let $K \subset \Omega$ and let $0 < \delta^\prime < \delta_K$. Then
\[
\|f\|_{X,\, \alpha, \, \delta^\prime} \leq  \|f\|_{X,\, \alpha, \, \delta_K} \quad \text{for all $f \in L^2(\Omega)$ with $\mathrm{spt}(f) \subset K$}.
\]
\el

\begin{proof}
This is obvious since on the right-hand side we are simply taking the supremum over a larger value of possible $t$'s.
\end{proof}

\bpr \label{prop.i.1}
Let $K \subset \Omega$ be a compact subset and let
\[
\varphi : \Omega \times [0, \, \delta) \longrightarrow \Omega
\]
be a function satisfying the following properties: \mbox{}
\begin{enumerate}[label=\textbf{\color{magenta}(\alph*)}, leftmargin=2.5\parindent]
\item For all $t \in [0, \, \delta)$ the function $\varphi_t$ is $C^\infty(\Omega)$ with respect to the variable $x$.
\item All partial derivatives of any order with respect to $x$ are continuous in $t$.
\item There exists $\mu > 0$ such that $\varphi(x, \, t) = x + \mathcal{O}(|t|^\mu)$ when $t \to 0^+$.
\end{enumerate}
Then there exists a constant $C = C(K,\, \varphi)$ such that for all $\alpha \in (0, \, 1]$ and $|t|$ small enough we have
\begin{equation} \label{eq.i.1}
\| f \circ \varphi_t - f \|_{L^2(\Omega)} \leq C|t|^{\mu \alpha} \|f\|_{\Lambda_\alpha^{2, \, \infty}} \quad \text{for all $f \in L^2(\Omega)$ with $\mathrm{spt}(f) \subset K$}.
\end{equation}
\epr

\begin{proof}
Start by considering all $h$ such that $|h| < |t|^\mu$ and apply the triangular inequality to the left-hand side of \eqref{eq.i.1}. Then
\[
\| f \circ \varphi_t - f \|_{L^2(\Omega)} \leq \| f \circ \varphi_t - \tau_h f\|_{L^2(\Omega)} + \| \tau_h f - f \|_{L^2(\Omega)},
\]
where the second addendum is already known to be estimates by $|h|^\alpha \|f\|_{\Lambda_\alpha^{2,\, \infty}}$. As for the first addendum, take the average with respect to $|t|^\mu$,
\[
\frac{1}{|t|^{n \mu}} \int_{|h| < |t|^\mu} \| f \circ \varphi_t - \tau_h f \|_{L^2(\Omega)}^2 \, \mathrm{d}h = \frac{1}{|t|^{n \mu}} \int_{|h| < |t|^\mu}  \int_\Omega | f(\varphi_t(x)) - f(x - h)|^2 \, \mathrm{d}x \mathrm{d}h,
\]
and apply the following change of variables:
\[
(y, \, u) = \Psi_t(x, \, h) := (x-h, \, \varphi_t(x) - (x-h)).
\]
Notice that $\Psi_0(x, \, h) = (x-h, \, h)$ is a {\bf diffeomorphism} and $\Psi_t$, for $|t|$ small enough, is a perturbation, small in the $C^1$-norm, of $\Psi_0$, and thus a diffeomorphism. Thus
\[
| f(\varphi_t(x)) - f(x - h)|^2= |f(u+y) - f(y)|^2,
\]
and this concludes the proof using the estimate above and the fact that $|h|$ and $|u|$ are not so different.
\end{proof}

\bpr
Let $K \subset \Omega$ be a compact subset and let $X$ be a vector field on $\Omega$. There exists a constant $C = C(K,\, X) > 0$ such that, for all $\alpha \in (0, \, 1]$ and $|t|$ small enough,
\begin{equation} \label{eq.i.2}
\| f \|_{X, \, \alpha} \leq C \|f\|_{\Lambda_\alpha^{2, \, \infty}} \quad \text{for all $f \in L^2(\Omega)$ with $\mathrm{spt}(f) \subset K$}.
\end{equation}
\epr

\begin{exercise}
Prove \eqref{eq.i.2} exploiting the same methods proposed for the proof of \eqref{eq.i.1}.
\end{exercise}

Recall that, if $X$ is a vector field and $f$ a function, then there is a notion of product $fX$ which is also a vector field. In particular, there is a map
\[
C_c^\infty(\Omega) \ni \eta \longmapsto X_\eta := \eta X \in \mathfrak{X}(\Omega),
\]
which sends a smooth compactly supported function to a smooth vector field that is defined by setting
\[
X_\eta(p) := \eta(p) X(p).
\]
Our next goal is to understand the connection between $\| \cdot \|_{X, \, \alpha, \, \delta}$ and $\| \cdot \|_{X_\eta, \, \alpha, \, \delta}$. First, it is useful to see how integral curves are related to each other. Let $\gamma_x^\eta(t)$ and $\gamma_x(t)$ be the respective integral curves originating from the same point $x \in \Omega$, and notice that
\begin{equation} \label{eq.i.3}
\gamma_x^\eta(t) = \gamma_x(\tau(t, \, x)),
\end{equation}
where $\tau(t, \, x)$ is a function describing the time-discrepancy between the two curves. We differentiate \eqref{eq.i.3} with respect to $t$ and obtain
\[
X_\eta(\gamma_x^\eta(t)) = \partial_t \tau(t, \, x) X(\gamma_x^\eta(t)),
\]
which easily leads to
\[
\partial_t \tau(x, \, t) = \eta( \gamma_x^\eta(t)) = \eta( \gamma_x( \tau(x, \, t))).
\]
We thus obtain, for each $x \in \Omega$, a ODE problem with fixed initial value
\[ \begin{cases}
\partial_t \tau_x(t) = \eta(\gamma_x(\tau_x(t))), \\[0.8em] \tau_x(0) = 0,
\end{cases} \]
which gives local existence and also a uniform lower bound on the time of existence if we restrict $x$ to $K$.

\bpr
Let $K \subset \Omega$ be compact. There exists a constant $C = C(K,\, \eta)$ such that for all $\alpha \in (0, \, 1]$ we have
\begin{equation} \label{eq.i.4}
\| f \|_{X_\eta, \, \alpha} \leq C \|f\|_{X, \, \alpha} \quad \text{for all $f \in L^2(\Omega)$ with $\mathrm{spt}(f) \subset K$}.
\end{equation}
\epr

\begin{proof}Let $\Phi$ and $\Phi^\eta$ be the flows of, respectively, $X$ and $X_\eta$. As before, we first apply the triangular inequality to the left-hand side of \eqref{eq.i.4} to obtain
\[
\| \exp(t X_\eta)f - f \|_{L^2(\Omega)} \leq \| \exp(t X_\eta) f  - \exp(sX)f\|_{L^2(\Omega)} + \| \exp(sX) f - f\|_{L^2(\Omega)}.
\]
By definition, we can estimate the second addendum $\| \exp(sX) f - f\|_{L^2(\Omega)}$ with $|s|^\alpha \|f\|_{X, \, \alpha}$, so we choose $s$ with $|s| \leq |t|$. As before, take the average over $t$,
\[\begin{aligned}
\frac{1}{2t} \int_{-|t|}^{|t|} \| \e^{t X_\eta}f  - \e^{s X}f\|_{L^2(\Omega)}^2 \, \mathrm{d}s & = \frac{1}{2t} \int_{-|t|}^{|t|} \int_\Omega | f \circ \Phi^\eta(x, \, t) - f \circ \Phi(x, \, s) |^2 \, \mathrm{d}x \mathrm{d}s =
\\[1em] & = \frac{1}{2t} \int_{-|t|}^{|t|} \int_\Omega | f \circ\Phi(x, \, \tau(x, \, t)) - f \circ \Phi(x, \, s) |^2 \, \mathrm{d}x \mathrm{d}s.
\end{aligned}\]
Now apply the change of variables
\[
(y, \, u) = \Psi_t(x, \, s) := (\Phi(x, \, s), \, \tau(x, \, t) - s),
\]
and notice that $\Psi_0(x, \, s) = (\Phi(x, \, s), \, - s)$ is a diffeomorphism since its Jacobian has determinant equal to
\[
- \nabla_x \Phi(x, \, s).
\]
Therefore, since we can choose $|t|$ to be as small as we want, we can apply the same argument given in \eqref{eq.i.3} to conclude that $\Psi_t$ is a diffeomorphism and
\[ 
\| \e^{t X_\eta }f  - \e^{s X}f\|_2 \leq C |u|^{\alpha} \|f\|_{X, \, \alpha}.
\]
Finally, from the definition of $u$ and the fact that $|t|$ is small we find that
\[
|u| \leq |s| + |\tau(x,\, t)| \leq |t| + |\tau(x, \, t)| \stackrel{|t| \ll 1}{\leq} C^\prime |t|,
\]
and this concludes the proof. \end{proof}

\subsection{Besov spaces of combinations of vector fields}

The goal of this section is to investigate the relation between $\| \cdot \|_{X+Y, \, \alpha}$ and $\| \cdot \|_{X, \, \alpha} + \| \cdot \|_{Y, \, \alpha}$, where $X$ and $Y$ are two smooth vector fields. If $[X, \, Y] = 0$, then
\[
\| f \|_{X + Y, \, \alpha, \, \delta} = \|f\|_{L^2(\Omega)} + \sup_{|t| < \delta}\left\{ |t|^{-\alpha} \| \e^{tX + tY} f - f \|_{L^2(\Omega)} \right\}
\]
may be estimated using the triangular inequality with either $\e^{tX}$ or $\e^{tY}$:
\[ \begin{aligned}
\| \e^{tX + tY} f - f \|_{L^2(\Omega)} & \leq \| \e^{tX}(\e^{tY} f - f) \|_{L^2(\Omega)} + \| \e^{tX} f - f \|_{L^2(\Omega)} =
\\[1em] & = \| (\e^{tY} f - f) \circ \varphi_{X, \, t} \|_{L^2(\Omega)} + \underbrace{ \| \e^{tX} f - f \|_{L^2(\Omega)} }_{\leq |t|^\alpha \|f\|_{X, \, \alpha}} {\color{blue}{}\leq}
\\[1em] & {\color{blue}\leq{}} C|t|^\alpha \left( \|f\|_{Y, \, \alpha} + \|f\|_{X, \, \alpha} \right).
\end{aligned} \]
The {\color{blue}blue} inequality follows from the fact that, if $|t|$ is small enough, the flow $\varphi_{X,\, t}$ is a diffeomorphism and hence preserves the $L^2$-norm:
\[
\| (\e^{tY} f - f) \circ \varphi_{X, \, t} \|_{L^2(\Omega)} = \| \e^{tY} f - f \|_{L^2(\Omega)}.
\]

\bpr
Let $K \subset \Omega$ be compact and let $X$ and $Y$ be smooth vector fields on $\Omega$ such that
\[
[X, \, Y] = 0.
\]
Then there exists a constant $C = C(K,\, X, \, Y) > 0$ such that for all $\alpha \in (0, \, 1]$ we have
\begin{equation} \label{eq.i.5}
\| f \|_{X+Y, \, \alpha} \leq C (\|f\|_{X, \, \alpha} + \|f\|_{Y, \, \alpha}) \quad \text{for all $f \in L^2(\Omega)$ with $\mathrm{spt}(f) \subset K$}.
\end{equation}
\epr

If $[X, \, Y] \neq 0$, then things do not work out in the same way. We now state a lemma which strengthens the conclusions reached in \hyperref[prop.diff.2.1]{Proposition \ref{prop.diff.2.1}}.

\bl
Let $X_1, \, \dots, \, X_p$ be smooth vector fields on $\Omega$. \mbox{}
\begin{enumerate}[label=\textbf{(\roman*)}]
\item For all $m \geq 1$ there exists $N(m) = N$ such that
\begin{equation} \label{eq.i.6}
\exp \left( t(X_1 + \dots + X_p)\right) f(x) = \e^{\pm t X_{i_1}} \dots  \mathrm{e}^{\pm t X_{i_N}}f(x) + \mathcal{O}(|t|^m),
\end{equation}
where $i_k \in \{1, \, \dots, \, p\}$ for all $k = 1, \, \dots, \, N$ and $\mathcal{O}(|t|^m)$ is uniform once we fix a compact subset $K \subset \Omega$.
\item Let $C_q$ be a elementary iterated commutator of multidegree $q$. Then for all $m \geq 1$ we can find $N(m) = N$ such that
\begin{equation} \label{eq.i.7}
\exp \left( t C_q\right) f(x) = \e^{\pm t^{\frac{1}{q}} X_{i_1}} \cdots  \e^{\pm t^{\frac{1}{q}} X_{i_N}}f(x) + \mathcal{O}(|t|^m),
\end{equation}
where $i_k \in \{1, \, \dots, \, p\}$ for all $k = 1, \, \dots, \, N$ and $\mathcal{O}(|t|^m)$ is uniform once we fix a compact subset $K \subset \Omega$.
\end{enumerate} 
\el

\brmk
Multiplying by the inverses of the exponentials in the right-hand side in \eqref{eq.i.6} yields
\begin{equation} \label{eq.i.8}
\e^{\mp t X_{i_N}} \cdots  \e^{\mp t X_{i_1}} \e^{ t(X_1 + \dots + X_p} f(x) - f(x) = \mathcal{O}(|t|^m),
\end{equation}
which means that the flow $\Psi$ associated to this composition of these exponentials satisfies
\[
f \circ \Psi(x, \, t) = f(x) + \mathcal{O}(|t|^m).
\]
If we choose $f$ to be the coordinate function $e_j$, $j = 1, \, \dots, \, n$, we get
\[
\Psi(x, \, t) =x + \mathcal{O}(|t|^m), 
\]
and hence $\Psi$ {\bf satisfies} the assumptions of \hyperref[prop.i.1]{Proposition \ref{prop.i.1}}.
\ermk

\bthm
Let $X_1, \, \dots, \, X_p$ be smooth vector fields on $\Omega$, $K \subset \Omega$ compact and $\sigma \in (0, \, 1)$. Then for all $\alpha \geq 1$ and all $f \in L^2(\Omega)$ with support contained in $K$ we have: \mbox{}
\begin{enumerate}[label=\textbf{(\roman*)}]
\item There exists $C(X,\, p, \, \sigma) = C$ such that
\begin{equation} \label{eq.i.9}
\|f\|_{X_1 + \dots + X_p, \, \alpha} \leq C \sum_{i = 1}^p \|f\|_{X_i, \, \alpha} + C^\prime \|f\|_{\Lambda_\sigma^{2, \, \infty}}.
\end{equation}
\item Let $q \in \N$. There exists $C(p, \, q) = C$ such that
\begin{equation} \label{eq.i.10}
\|f\|_{C_q, \, \frac{\alpha}{q}} \leq C \sum_{i = 1}^p \|f\|_{X_i, \, \alpha} + C^\prime \| f \|_{\Lambda_\sigma^{2, \, \infty}}.
\end{equation}
\end{enumerate}
\ethm

\begin{proof}
We only prove $\mathbf{(ii)}$. Use \eqref{eq.i.7} to rewrite the left-hand side as
\[
 \| \e^{tC_q} f - f \|_{L^2(\Omega)} = \| \e^{\pm t^{\frac{1}{q}} X_{i_1}} \cdots \e^{\pm t^{\frac{1}{q}} X_{i_N}} (f \circ \Psi_t) - f \|_{L^2(\Omega)}.
\]
This implies the following chain of inequalities
\[ \begin{aligned}
 \mathbf{(LHS)} & \leq \| \e^{\pm t^{\frac{1}{q}} X_{i_1}} \cdots  \e^{\pm t^{\frac{1}{q}} X_{i_N}} (f \circ \Psi_t - f)  \|_{L^2(\Omega)} + \| \e^{\pm t^{\frac{1}{q}} X_{i_1}} \cdots  \e^{\pm t^{\frac{1}{q}} X_{i_N}} f - f  \|_{L^2(\Omega)} {{}\color{orange}\leq}
\\[1em] & {\color{orange}\leq{}} C \| f \circ \Psi_t - f \|_{L^2(\Omega)} + \| \e^{\pm t^{\frac{1}{q}} X_{i_1}} \dots  \e^{\pm t^{\frac{1}{q}} X_{i_N}} f - f  \|_{L^2(\Omega)}  {{}\color{blue}\leq}
\\[1em] & {\color{blue}\leq{}} C \| f \circ \Psi_t - f \|_{L^2(\Omega)} + C^\prime \sum_{i = 1}^p \| \e^{t^{\frac{1}{q}} X_i} f - f  \|_{L^2(\Omega)} \leq
\\[1em] & \leq C |t|^{m\sigma} \|f\|_{\Lambda_\sigma^{2, \, \infty}} + C^\prime \sum_{i = 1}^p |t|^{\frac{\alpha}{q}} \|f\|_{X_j, \, \alpha},
\end{aligned}\]
and we conclude by choosing $m \sigma \geq \frac{\alpha}{q}$.

The {\color{orange}orange} inequality follows from the fact that each $\varphi_{X_j, \, t}$ is a diffeomorphism for $|t|$ small enough, while the {\color{blue}blue} one follows from the fact that the second term can be rewritten as a telescopic sum (removing one exponential at a time.)
\end{proof}

\bl \label{lemma.j.1}
Let $0 < \beta < \alpha \leq 1$ and let $\epsilon > 0$. Then there exists a positive constant $C_\epsilon$ such that
\begin{equation} \label{eq.j.1}
\|f\|_{\Lambda_\beta^{2, \, \infty}} \leq C_\epsilon \|f\|_{L^2(\Omega)} + \epsilon \|f\|_{\Lambda_\alpha^{2, \, \infty}} \quad \text{for all $f \in L^2(\R^n)$}.
\end{equation}
\el

Notice that the constant $C_\epsilon$ does not depend on $\epsilon$ only, but on the length of the interval over which the supremum is taken. Indeed, recall that the Besov norm is
\[
\|f\|_{\Lambda_\alpha^2} = \|f\|_{L^2(\Omega)} + \sup_{0 < |t| < a} |h|^{-\alpha} \| \tau_h f - f \|_2,
\]
so it depends on the choice of $a$ - although taking $a \neq a^\prime$ only leads to equivalent norms.

\begin{proof}
Fix $a > 0$ and take $0 < \delta < a$. Then
\[ \begin{aligned} 
\|f\|_{\Lambda_\beta^{2, \, \infty}} & \leq \|f\|_{L^2(\Omega)} + \sup_{0 < |h| \leq \delta} |h|^{-\beta} \| \tau_h f - f \|_{L^2(\Omega)} + \sup_{ \delta \leq |h| < a} |h|^{-\beta} \| \tau_h f - f \|_{L^2(\Omega)} \leq
\\[1em] & \leq \|f\|_{L^2(\Omega)} + \delta^{\alpha - \beta} \sup_{0 < |h| \leq \delta} |h|^{-\alpha} \| \tau_h f - f \|_{L^2(\Omega)} + 2 \delta^{-\beta} \|f\|_{L^2(\Omega)} =
\\[1em] & \leq (1 + 2 \delta^{-\beta}) \|f\|_{L^2(\Omega)} + \delta^{\alpha - \beta} \sup_{0 < |h| < a} |h|^{-\alpha} \| \tau_h f - f \|_{L^2(\Omega)}.
\end{aligned}\]
Now choose $\delta$ in such a way that $\delta^{\alpha - \beta} = \epsilon$ and notice that \eqref{eq.j.1} holds with $C_\epsilon$ given by
\[
C_\epsilon = (1 + 2\delta^{-\beta} - \delta^{\alpha - \beta}).
\]
\end{proof}

\bthm
Let $X_1, \, \dots, \, X_k$ be smooth vector fields on $\Omega$ satisfying the condition
\[
\mathrm{Span} \left\langle X_1(x), \, \dots, \, X_k(x), \, \dots, \, \text{commutators up to order $m$ at $x$} \right\rangle = \R^n
\]
at all $x \in K \subset \Omega$, $K$ compact subset. Then for all $\alpha \in (0, \, 1]$ we have
\begin{equation} \label{eq.j.2}
\|f\|_{\Lambda_{\frac{\alpha}{m}}^{2, \, \infty}} \leq C_{\alpha, \, m, \, K} \sum_{j = 1}^k \|f\|_{X_j, \, \alpha} \quad \text{for all $f \in L^2(\Omega)$ with $\mathrm{spt}(f) \subset K$}.
\end{equation}
\ethm


\begin{proof}
First, notice that
\[
\| f \|_{\Lambda_\beta^{2, \, \infty}} \simeq \sum_{j = 1}^n \|f \|_{\partial_{x_j}, \, \beta},
\]
where $\partial_{x_j}$ denote the coordinate vector fields. Fix $x \in K$ and let
\[
\{ Y_1, \, \dots, \, Y_n\} \subset \{ X_1, \, \dots, \, X_k, \, \dots, \, \text{commutators up to order $m$} \}
\]
be the basis of $\R^n$ at $x$ which exists by assumption. Then there exists a small neighbourhood $U_x$ of $x$ such that
\[
\mathrm{Span} \langle Y_1(y), \, \dots, \, Y_n(y) \rangle = \R^n \quad \text{for all $y \in U_x$}.
 \]
Using the coordinate vector fields, we can write in a unique way
\[
Y_j(y) = \sum_{i = 1}^n \lambda_{i, \, j}(y) \partial_{x_i}
\]
for some smooth functions $\lambda_{i, \, j}$ defined on $U_x$. Invert the metric to find smooth functions $\eta_{i, \, j}$ such that the following holds:
\[
\partial_{x_j} = \sum_{i = 1}^n \eta_{i, \, j}(y) Y_i(y).
\]
The collection of open sets $\{ U_x \: : \: x \in K\}$ covers $K$ so, by compactness, we can select a finite family of point $x_1, \, \dots, \, x_N$ such that
\[
K \subseteq \bigcup_{i = 1}^N U_{x_i}.
\]
Let us consider a partition of unity\index{partition of unity} relative to $\cU$,
\[
\{ \varphi_i : U_{x_i} \longrightarrow \R \}_{1 \leq i \leq N},
\]
and glue together the vector fields $\partial_{x_j}$ to find a global representation that holds at all points of $K$; namely, we have
\[ 
\partial_{x_j} = \sum_{\ell = 1}^N \varphi_\ell(x) \sum_{i = 1}^n \eta_{i, \, j}^\ell(x) Y_i^\ell(x) \quad \text{for all $x \in K$}.
\]
We added the superscript $\ell$ to $\eta$ and $Y$ because they depend also on the choice of the point $x_i$. Using \eqref{eq.i.9}, followed by \eqref{eq.i.4}, we can easily infer that
\[ \begin{aligned}
\|f\|_{\partial_{x_j}, \, \frac{\alpha}{m}} & \leq C_\sigma \sum_{\ell=1}^N \sum_{i = 1}^n \|f\|_{\varphi_\ell \eta_{i, \, j} Y_i, \, \frac{\alpha}{m}} \leq
\\[1em] & \leq C_\sigma^\prime \sum_{i = 1}^n \|f\|_{Y_i, \, \frac{\alpha}{m}} + C^\prime \|f\|_{\Lambda_\sigma^{2, \, \infty}}.\end{aligned} \]
Now apply \eqref{eq.i.10} with $q = m$ - since $Y_k$ is at most a commutator of order $m$ - and notice that
\[ 
\|f\|_{\partial_{x_j}, \, \frac{\alpha}{m}} \leq C_{m, \, \sigma}^{\prime\prime} \sum_{i = 1}^k \|f\|_{X_i, \, \alpha} + C^\prime \|f\|_{\Lambda_\sigma^{2, \, \infty}}.
\]
It turns out that
\[
\| f \|_{\Lambda_\beta^{2,\, \infty}} \simeq \sum_{j = 1}^n \|f \|_{\partial_{x_j}, \, \beta} \leq \widetilde{C}_{m, \, n, \, \sigma} \sum_{i = 1}^k \|f\|_{X_i, \, \alpha} + C_n^{\prime\prime} \|f\|_{\Lambda_\sigma^{2, \, \infty}}.
\]
To estimate the second term in the right-hand side, take $\sigma$ smaller than $\frac{\alpha}{m}$ and notice that by \eqref{eq.j.1} we can find a constant $C_\epsilon > 0$ such that
\[
\|f\|_{\Lambda_\sigma^{2, \, \infty}} \leq \epsilon \|f\|_{\Lambda_{\frac{\alpha}{m}}^{2, \, \infty}} + C_\epsilon \|f\|_{L^2(\Omega)}. \]
The conclusion follows immediately if we choose $\epsilon$ in such a way that
\[
C_n^{\prime\prime} \epsilon < \frac{1}{2}.
\]
\end{proof}

\bcor
Let $f \in C_c^\infty(\Omega)$ with support contained in some compact set $K \subset \Omega$. Then there exists a constant $C = C(K)>0$ such that
\begin{equation} \label{eq.j.3}
\|f\|_{\Lambda_{\frac{1}{m}}^{2, \, \infty}} \leq C \left( \sum_{j = 1}^2 \|X_j f\|_{L^2(\Omega)} + \|f\|_{L^2(\Omega)} \right).
\end{equation}
\ecor

\begin{proof}
It suffices to estimate $\| f\|_{X_j, \, \alpha}$ when $\alpha = 1$. We have
\[ \begin{aligned}
\left|  \int_\Omega \e^{t X} \, \mathrm{d}x \right|^2 & = \left| \int_\Omega \int_0^t \frac{\mathrm{d}}{\mathrm{d}s} \left( \e^{sX} \right)f(x) \, \mathrm{d}s \mathrm{d}x \right|^2 \leq
\\[1em] & \leq a \int_\Omega \int_0^t |\e^{sX}X f(x)|^2 \, \mathrm{d}s \mathrm{d}x =
\\[1em] & = a \int_0^t \| \e^{sX}X f \|{L^2(\Omega)}^2 \, \mathrm{d}s = a^2 \|Xf\|{L^2(\Omega)}^2,
\end{aligned} \] 
where $a$ is the parameter determining the Besov norm. This shows that
\[
\|f\|_{X_j, \, 1} \simeq \|X_j f \|_{L^2(\Omega)} + \|f\|_{L^2(\Omega)},
\]
and thus the conclusion follows from a straightforward application of \eqref{eq.j.2}.
\end{proof}

\bcor
For every $s < \frac{1}{m}$ there results
\begin{equation} \label{eq.j.4}
\|f\|_{H^s(\Omega)} \leq C \left( \sum_{j = 1}^2 \|X_j f\|_{L^2(\Omega)} + \|f\|_{L^2(\Omega)} \right).
\end{equation}
\ecor

\begin{proof}
This is a simple consequence of the inclusions
\[ \Lambda_\beta^{2, \, \infty} \subset \Lambda_\alpha^{2, \, 2} = H^\alpha (\Omega) \subset \Lambda_\alpha^{2, \, \infty}, \]
which holds true as soon as $\beta > \alpha$ - see \hyperref[prop2oqwdqwple]{Proposition \ref{prop2oqwdqwple}}.
\end{proof}

\section{A step in H\"{o}rmander's theorem}

In this section, we will show how to apply all these estimates proved so far to obtain a small step towards H\"{o}rmander's theorem.

\bthm[H\"{o}rmander] 
Assume that, at each $x \in \Omega$, the vector fields
\[
\{X_j\}_{j = 1, \, \dots, \, k}, \, \{ [X_j, \, X_i] \}_{1 \leq i < j \leq k}, \, \{ [X_j, \, [X_i, \, X_\ell]]\}_{i, \,j, \, \ell}, \, \dots
\]
span $\R^n$. Then the operator $L = \sum_{j = 1}^k X_j^2$ is hypoelliptic.
\ethm

\bl[A priori estimate]
Let $f \in C_c^\infty(\Omega)$ with support contained in a compact set $K \subset \Omega$. Then there exists a constant $C_s(K) = C_s > 0$ such that
\[
\|f\|_{H^s(\Omega)} \leq C_s\left( \|f\|_{L^2(\Omega)} + \|Lf\|_{\X}^\prime \right)
\]
for every $s < \frac{1}{m}$, where $m$ is the order of commutators that are necessary to generate $\R^n$.
\el

\begin{proof}
Let $K \subset \Omega^\prime \Subset \Omega$. Define the norm
\[
\|f\|_\X := \left[ \|f\|_{L^2(\Omega)}^2 + \sum_{j = 1}^k \|X_j f \|_{L^2(\Omega)}^2\right]^{\frac{1}{2}}
\]
and, for $u \in \D^\prime(\Omega^\prime)$, introduce the dual norm
\[
\|u\|_\X^\prime = \sup_{\substack{f \in \D(\Omega^\prime) \\[.2em] \|f\|_\X \leq 1}} | \langle u, \, f \rangle |.
\]
If $f \in \D(\Omega^\prime)$, then it is easy to check that
\[
\|f\|_\X^\prime \leq \|f \|_{L^2(\Omega)} \leq \|f\|_\X.
\]
The scalar product $\langle Lf, \, f \rangle$, in terms of the vector fields $X_j$, is given by
\[
\langle Lf, \, f \rangle = - \left\langle \sum_{j = 1}^k X_j^2 f, \, f \right\rangle = - \sum_{j =1 }^k \left\langle X_j f, \, X_j^\ast f \right\rangle,
\]
so we first need to understand how the duals $X_j^\ast$ behave. A simple computation shows that
\[ \begin{aligned}
\left\langle X^\ast f, \, g \right\rangle & := \left\langle f, \, X g \right\rangle = \int_\Omega f(x) \left[ \sum_{i = 1}^n a_i(x) \partial_{x_i} f(x) \right] \, \mathrm{d}x =
\\[1em] & = - \int_\Omega  \sum_{i = 1}^n \partial_{x_i}(a_i f)(x) g(x) \, \mathrm{d}x =
\\[1em] & = - \int_\Omega g(x) \sum_{i = 1}^n a_i(x) \partial_{x_i} f(x) \, \mathrm{d}x - \int_\Omega g(x) f(x) \underbrace{\sum_{i = 1}^n \partial_{x_i} a_i(x)}_{:= b(x)} \, \mathrm{d}x
\end{aligned} \]
so that
\[
Xf(x) = \sum_{i=1}^n a_i(x) \partial_{x_i} f(x) \implies X^\ast f(x) = - \sum_{i = 1}^n a_i(x) \partial_{x_i} f(x) - b(x)f(x).
\]
It follows that
\[
\left\langle Lf, \, f \right\rangle = \sum_{j = 1}^k \left\langle X_j f, \, (X_j + b_j)f \right\rangle = \sum_{j = 1}^k \|X_j f\|_{L^2(\Omega)}^2 + \sum_{j = 1}^k \left\langle X_j f, \, b_j f \right\rangle, \]
and the second term in the right-hand side can easily be dealt with using once more the formula for the dual $X_j^\ast$:
\[ \begin{aligned}
\sum_{j = 1}^k \left\langle X_j f, \, b_j f \right\rangle & = - \sum_{j = 1}^k \left\langle f, \, (X_j + b_j) b_j f \right\rangle =
\\[1em] & = - \sum_{j = 1}^k \left\langle f, \, X_j(b_j f) \right\rangle - \left\langle f, \, b_j^2 f \right\rangle =
\\[1em] & = - \sum_{j = 1}^k\left[ \left\langle f, \, X_j(b_j) f \right\rangle - \left\langle f, \, b_j X_j(f) \right\rangle - \left\langle f, \, b_j^2 f \right\rangle \right].
\end{aligned} \]
Since $\sum_j \langle f, \, b_j X_j(f) \rangle$ is equal to the quantity on the left-hand side, we find that
\[
\sum_{j = 1}^k \left\langle X_j f, \, b_j f \right\rangle  = - \frac{1}{2} \sum_{j =1}^k \left[ \left\langle f, \, X_j(b_j) f \right\rangle + \left\langle f, \, b_j^2 f \right\rangle\right].
\]
Therefore, the scalar product between $Lf$ and $f$ is equal to
\[
\left\langle Lf, \, f \right\rangle = \sum_{j = 1}^k \|X_j f\|_{L^2(\Omega)}^2 - \frac{1}{2} \sum_{j =1}^k \left[ \left\langle f, \, X_j(b_j) f \right\rangle + \left\langle f, \, b_j^2 f \right\rangle\right], \]
from which we immediately deduce that
\[ \begin{aligned}
\|f\|_\X & = \|f\|_{L^2(\Omega)} +\langle Lf, \, f \rangle + \frac{1}{2} \sum_{j =1}^k \left[ \left\langle f, \, X_j(b_j) f \right\rangle + \left\langle f, \, b_j^2 f \right\rangle\right] =
\\[1em] & = \langle Lf, \, f \rangle + \frac{1}{2}  \sum_{j = 1}^k \left\langle f, \, (X_j b_j + b_j^2 + 2) f \right\rangle.
\end{aligned} \]
Now $(X_j b_j + b_j^2 + 2)$ is bounded on $\Omega^\prime$ so there exists a positive constant $C^\prime$ such that
\[
\frac{1}{2}  \sum_{j = 1}^k \left\langle f, \, (X_j b_j + b_j^2 + 2) f \right\rangle \leq C^\prime \|f\|_{L^2(\Omega)}^2.
\]
Consequently,
\[ \begin{aligned}
\|f\|_\X & \leq \| Lf \|_\X^\prime \|f\|_\X + C^\prime \|f\|_{L^2(\Omega)}^2 \leq
\\[1em] & \leq \| Lf \|_\X^\prime \|f\|_\X + C^\prime \|f\|_{L^2(\Omega)} \|f\|_\X,
\end{aligned} \]
which implies that
\[
\|f\|_\X \leq \widetilde{C}(\|f\|_{L^2(\Omega)}^2 + \|Lf\|_\X^\prime).
\]
Using \eqref{eq.j.4} we conclude the proof of this lemma.
\end{proof}

\section{A few consequences of H\"{o}rmander's theorem}

In this conclusive section, we show a few consequences of H\"{o}rmander's theorem such as the behaviour of integral curves of commutators and {\em nonisotropic lengths}. For example,

\begin{customthm}{A}
Let $\Omega$ be a connected subset of $\R^n$ and let $X_1, \, \dots, \, X_k$ be a H\"{o}rmander system on $\Omega$. Then any two points in $\Omega$ can be joined by a {\em horizontal curve}.
\end{customthm}

\subsection{Integral curves of commutators}

Let $X_1, \, \dots, \, X_k$ be smooth vector fields on $\Omega$ satisfying the H\"{o}rmander condition with order $p$; in other words, for all $x \in \Omega$ we have
\[
\R^n = \mathrm{Span} \left\langle X_1(x), \, \dots, \, X_k(x), \, \text{commutators of order $\leq p$ at $x$} \right\rangle.
\]
Fix $x \in \Omega$ and denote by
\[
\cY_x := \{ Y_1, \, \dots, \, Y_n \}
\]
the collection of vector fields that gives a basis of $\R^n$ at $x$ among the ones above. By continuity of the determinant map, the set
\[
\{Y_1(y), \, \dots, \, Y_n(y) \}
\]
is still a basis of $\R^n$ at all $y$ in an neighbourhood $U_x$ of $x$ small enough. Now consider the map
\[
\Phi_{x}(t_1, \, \dots, \, t_n) := \gamma_{x}^{\sum_{j = 1}^n t_j Y_j}(1) = \varphi_{ \sum_{j=1}^n t_j Y_j,\, 1}(x),
\]
where $\gamma_{x}^{\sum_{j = 1}^n t_j Y_j}(1)$ is the integral curve relative to the vector field
\[
t_1 Y_1 + \cdots + t_n Y_n
\]
starting from the point $x \in \Omega$ and evaluated at time $\tau = 1$. This map is well-defined provided that we take $|t_i|$ sufficiently small for each $i = 1,\, \dots, \, n$. The Jacobian at the origin is given by
\[
J_{\Phi_{x}}(0, \, \dots, \, 0) = (Y_1(x), \, \dots, \, Y_n(x))
\]
since
\[
\frac{\partial}{\partial t_j} \, \Big|_{\vec{t} = 0} \Phi_{x}= \frac{\rmd}{\rmd s} \, \Big|_{s = 0} \Phi_{x}(0, \, \dots, \, s, \, \dots, \, 0) = Y_j(x)
\]
follows immediately by exploiting the equation defining the integral curve, that is,
\[
\gamma_x^\prime(\tau) = Y_j(\gamma_x(\tau)).
\]
The vector fields $Y_i$ are linearly independent at $x$ so the determinant of the Jacobian is nonzero, namely
\[
\mathrm{det}\left(J_{\Phi_{x}}(0, \, \dots, \, 0) \right)\neq 0,
\]
and hence $\Phi_{x}$, restricted to a small ball of $\R^n$ centred at the origin, is a diffeomorphism onto a neighbourhood of $x$. In a similar fashion, the map
\[
\Psi_{x}(t_1, \, \dots, \, t_n) :=  \varphi_{Y_1, \, t_1} \circ \dots \circ \varphi_{Y_n, \, t_n}(x)
\]
is the map that follows the integral curve of $Y_1$ starting from $x \in \Omega$ for a time $t_1$ and then iteratively following the integral curve of $Y_{j+1}$ starting from $Y_j(t_j)$ for a time $t_{j+1}$. But
\[
J_{\Psi_{x}}(0, \, \dots, \, 0) = (Y_1(x), \, \dots, \, Y_n(x))
\]
so $\Psi_{x}$ is also a diffeomorphism restricted to a small ball of $\R^n$ centred at the origin (and its differential coincides with the one given by the map $\Phi_{x}$.)

\brmk
Let $Y_1, \, \dots, \, Y_N$ denote the vector fields $X_1, \, \dots, \, X_k$ and their elementary commutators up to order $p$. The corresponding map
\[
\Phi_x(t_1, \, \dots, \, t_N) := \varphi_{Y_1, \, t_1} \circ \dots \circ \varphi_{Y_N, \, t_n}(x)
\]
has Jacobian with maximal rank, and hence it maps a small ball of $\R^n$ centred at the origin onto a neighbourhood of $x$. It is worth remarking that the choice of the vector fields does not depend on $x \in \Omega$ since the H\"{o}rmander condition holds.
\ermk

\bd[Horizontal Curve] \index{horizontal curve}
We say that a piecewise $C^1$ function $\gamma :[a, \,b] \to \Omega$ is a {\em horizontal curve} if
\[
\gamma^\prime(t) \in \mathrm{Span} \langle X_1(\gamma(t)), \, \dots, \, X_k(\gamma(t)) \rangle \quad \text{for almost every $t \in [a, \, b]$}.
\]
\ed

\begin{problem}
Can we join any two points of $\Omega$ via a horizontal curve? If $\Omega$ is connected, it is enough to show that each $x \in \Omega$ has a neighbourhood $U_x$ such that
\[
y \in U_x \implies \text{$\exists \, \gamma :[a, \, b] \to \Omega$ horizontal with $\gamma(0) = x$, $\gamma(1) = y$}.
\]
\end{problem}

If $\Psi$ is the map defined above, we might try to move along integral curves to connect points. However, there is no guarantee that
\[ Y_j \in \{ X_1, \, \dots, \, X_k \} \quad \text{for all $j = 1, \, \dots, \, n$}, \]
and this is, in fact, impossible when $k < n$. Therefore, our goal is to exploit the theory developed in the previous sections to "approximate" the flow when $Y_j$ is a commutator of order $\geq 1$. We start with the simplest case possible:

\bl
\label{lemma:fsd} Suppose that $Y = [X_1, \, X_2]$. Then the map
\[
\widetilde{\varphi}_{Y, \, t} (x) := \begin{cases}
\varphi_{X_2, \, \sqrt{t}} \circ \varphi_{X_1, \, \sqrt{t}} \circ \varphi_{X_1, \, -\sqrt{t}} \circ \varphi_{X_2, \,-\sqrt{t}}(x), & \text{for $t \geq 0$},
\\[.8em] \varphi_{X_2, \, \sqrt{|t|}} \circ \varphi_{X_1, \, \sqrt{|t|}} \circ \varphi_{X_1, \, -\sqrt{|t|}} \circ \varphi_{X_2, \, -\sqrt{|t|}}(x) & \text{for $t < 0$} \end{cases} \]
is $C^1$ with respect to $t$ and satisfies the identity
\[
\frac{\rmd }{\rmd t} \, \Big|_{t = 0} \widetilde{\varphi}_{Y, \, t} (x) = Y(x).
\]
\el

\begin{proof}
Denote by $\widetilde{\exp}$ the exponential map related to the modified flow $\widetilde{\varphi}_{Y, \, t}$ so that
\[
\widetilde{\exp}(t Y)f(x) =\exp(\sqrt{t} X_2)\exp(\sqrt{t} X_1)\exp(-\sqrt{t} X_1)\exp(-\sqrt{t} X_2)f(x).
\]
We can easily compute the derivative for $t > 0$ as follows:
\[ \begin{aligned}
\frac{\rmd}{\rmd t} \widetilde{\varphi}_{Y, \, t} (x) & = \frac{1}{2 \sqrt{t}} \left( \e^{\sqrt{t} X_2} \left[ (X_1 + X_2), \,  \e^{\sqrt{t} X_1} \e^{-\sqrt{t} X_1} \right] \e^{-\sqrt{t} X_2} \right) f(x) = 
\\[1em] & = \frac{1}{2 \sqrt{t}} \left( \e^{\sqrt{t} X_2} \left[ (X_1 + X_2), \,  \mathrm{Id} + \sqrt{t}(X_2 - X_1) + \mathcal{O}(t) \right] \e^{-\sqrt{t} X_2} \right) f(x) =
\\[1em] & = ([X_1, \, X_2] + \mathcal{O}(t))f(x).
\end{aligned} \]
We can do a similar computation for $t < 0$, and this gives the continuity of the $t$-derivative at $0$ which concludes the proof since at any other point it is trivial.
\end{proof}

We can now find a way to replace higher-order commutators iteratively. More precisely, suppose that $Y = [X_1, \, [X_2, \, X_3]] =: [X_1, \, X^\prime]$ and write for $t > 0$
\[
\widetilde{\varphi}_{Y, \, t}(x) := \varphi_{X^\prime,\, t^{\frac{2}{3}}} \circ \varphi_{X_1, \, t^{\frac{1}{3}}}\circ \varphi_{X_1, \, -t^{\frac{1}{3}}} \circ  \varphi_{X^\prime, \, - t^{\frac{2}{3}}} (x).
\]
Now apply the formula given in \hyperref[lemma:fsd]{Lemma \ref{lemma:fsd}} to $X^\prime$ to rewrite the right-hand side as the composition of $\varphi_{X_j, \, t^{\frac{1}{3}}}$, $j = 1,\, 2, \, 3$, and iterate it to elementary commutators of any order.

\bthm \label{thm:1231231}
Suppose that $\Omega$ is connected and let $X_1, \, \dots, \, X_k$ be a H\"{o}rmander system on $\Omega$. Then any two points in $\Omega$ can be joined by a horizontal curve.
\ethm

\subsection{Nonisotropic lengths}
\index{nonisotropic lengths}\label{nonisolength}

Let $\gamma$ be a horizontal curve associated to the H\"{o}rmander system $\X := \{X_1, \, \dots, \, X_k\}$. By definition
\[
\gamma^\prime(t) \in \mathrm{Span}\langle X_1(\gamma(t)), \, \dots, \, X_k(\gamma(t)) \rangle,
\]
so we can find coefficients $a_1,\, \dots, \, a_k$ such that
\[
\gamma^\prime(t) = \sum_{j = 1}^k a_j X_j(\gamma(t)),
\]
but the representation is not unique (because, in general, $\X$ is not made up of linearly independent vectors). Therefore, to define the velocity of $\gamma$ at $t$, we consider all possible representation and write it as the smallest possible, namely\index{horizontal curve!velocity}
\[
|\gamma^\prime(t)|_\X := \inf\left\{ \sum_{j = 1}^k |a_j| \: : \: \gamma^\prime(t) = \sum_{j = 1}^k a_j X_j(\gamma(t)) \right\}.
\]
The {\em length}\index{horizontal curve!length} of $\gamma$ is defined through the velocity as usual,
\[
L_\X(\gamma) := \int_a^b |\gamma^\prime(t)|_\X \, \rmd t,
\]
and since it is not restrictive to assume $\Omega$ connected, we can apply \autoref{thm:1231231} and infer that the function defined by setting
\[
d_\X(x, \, y) := \inf \left\{ L_\X(\gamma)\: : \: \text{$\gamma$ horizontal curve joining $x$ and $y$} \right\}
\]
is a well-defined distance between the points of $\Omega$. This is usually referred to in the literature as {\em control distance}\index{control distance} associated to the vector fields $X_j$.

Now let $\gamma : [a, \, b] \to \Omega$ be a piecewise $C^1$ curve and $\Y := \{Y_1, \, \dots, \, Y_N\}$ the collection of all the $X_j$'s and commutators up to order $p$ given by the H\"{o}rmander condition. Then
\[
\gamma^\prime(t) = \sum_{j = 1}^N a_j Y_j(\gamma(t)),
\]
but this representation is also non-unique since $\{Y_1, \, \dots, \, Y_N\}$ is in general not a minimal set of generators. If $d_j$ is the degree of $Y_j$, we define
\[
|\gamma^\prime(t)|_\Y := \inf\left\{ \sum_{j = 1}^k |a_j|^{\frac{1}{d_j}} \: : \: \gamma^\prime(t) = \sum_{j = 1}^N a_j Y_j(\gamma(t)) \right\}
\]
and, as above, the corresponding length
\[ L_\Y(\gamma) := \int_a^b |\gamma^\prime(t)|_\Y \, \mathrm{d}t, \]
and the distance
\[
d_\Y(x, \, y) := \inf \left\{ L_\Y(\gamma)\: : \: \text{$\gamma$ piecewise $C^1$ curve joining $x$ and $y$} \right\}.
\]
We will now make a similar construction, which is highly local but gives accurate information about the magnitude of the distance in a small neighbourhood. Fix $x \in \Omega$ and pick a basis of $\R^n$ following these "rules": \mbox{}
\begin{enumerate}[label={\color{orange}\textbf{(\alph*)}}]
\item Up to a relabeling, pick $X_1, \, \dots X_{\ell_1}$ in such a way that $X_1(y), \, \dots, \, X_{\ell_1}(y)$ are linearly independent for all $y \in U_x$.
\item Pick $[X_{i_1},\, X_{j_1}], \, \dots [X_{i_{\ell_2}},\, X_{j_{\ell_2}}]$ in such a way that $[X_{i_1},\, X_{j_1}](y), \, \dots [X_{i_{\ell_2}},\, X_{j_{\ell_2}}](y)$ are linearly independent for all $y \in U_x$.
\item Similar process up to commutators of order $p$. We get a basis, which we denote by
\[ \{ Y_1, \, \dots, \, Y_{\ell_1}, \, Y_{\ell_1 + 1}, \, \dots, \, Y_{\ell_2}, \, \dots, \, Y_{\ell_{p+1}} \} \]
satisfying the following dimensional relation
\[
\sum_{j = 1}^{p+1} \ell_j = n, \quad \ell_j \geq 0.
\]
\end{enumerate}
We proved that this basis induces a diffeomorphism $\Psi \, \big|_{B(0, \, \epsilon)}$ onto a neighbourhood $U_x$ of $x$ for some $\epsilon > 0$. If $y \in U_x$ and $\gamma$ joins $y$ with $x$, then there exists a unique way to write
\[
\gamma^\prime(t) = \sum_{j = 1}^n a_j Y_j(\gamma(t)),
\]
and, consequently,
\[
|\gamma^\prime(t)|_\Y := \sum_{j = 1}^n |a_j|^{\frac{1}{d_j}}
\]
is well-defined and leads to a distance, which is obviously {\color{red}local}. Furthermore, if we write the $n$-uple $(t_1, \, \dots, \, t_n)$ as
\[
(\vec{t}_0, \, \dots, \, \vec{t}_p), \quad \text{where $\vec{t}_j = (t_{\ell_j + 1},\\, \dots, \, t_{\ell_{j+1}})$},
\]
then we can easily prove that
\[ \begin{aligned}
& y = \Psi(\vec{t}_0, \, 0, \, \dots, \, 0) \implies d(x, \, y) \simeq \epsilon,
\\[1em] & y = \Psi(\vec{t}_0, \, \vec{t}_1, \, 0,\, \dots, \, 0) \implies \epsilon \leq d(x, \, y) \leq \epsilon^{\frac{1}{2}},
\\[1em] & y = \Psi(\vec{t}_0, \, \vec{t}_1, \, \vec{t}_2, \, 0,\, \dots, \, 0) \implies \epsilon^{\frac{1}{2}} \leq d(x, \, y) \leq \epsilon^{\frac{1}{3}},
\end{aligned} \]
and, more in general, the distance between $x$ and $y \in U_x$ is always $\leq \epsilon^{\frac{1}{p}}$. It follows that the corresponding ball $B_d(x, \, r)$ satisfies the following:
\[ 
\text{$y \in B_d(x, \, r)$ in the direction generated by commutators of order $m$} \implies d(x, \, y) \leq r^m.
\]

\subsection{Applications to distributions} %% QUI

Before going through this section, the reader is encouraged to refresh some concepts in differential geometry such as distributions, foliations, etc. In \hyperref[sec:foldist]{Section \ref{sec:foldist}}, the bare minimum necessary to understand this section is presented without any proof.

\paragraph{Framework.} Let $M \subset \C^n$ be a hypersurface and let $\varphi : \C^n \to \R$ be the function with nonzero gradient $\nabla \varphi \neq 0$ such that
\[
M = \{ z \in \C^n \: : \: \varphi(z) = 0 \} = \varphi^{-1}(0),
\]
and, for $z \in M$, consider the real tangent space\index{tangent space!real}
\[
T_z M = ( \nabla \varphi(z) )^\perp.
\]
The symbol $\perp$ denotes the orthogonal with respect to the real scalar product $\langle \cdot, \, - \rangle$ and it is easy to verify that $T_z M$ is a $(2n-1)$-dimensional real vector space. We can also define the complex tangent\index{tangent space!complex} space
\[
T_z^\C M := ( \nabla \varphi(z) )^\perp,
\]
where the orthogonal is now taken with respect to the Hermitian inner product, which we will always denote by $\langle \cdot, \,  - \rangle_\C$. Observe that
\[
T_z^\C M = ( T_z M) \cap (\imath T_z M),
\]
so the complex one is, in general, different from the real one and actually its dimension is lower than $2n - 1$. For each $a \in \R$ we can analogously define a hypersurface by setting
\[
M_a := \{ z \in \C^n \: : \: \varphi(z) = a \} = \varphi^{-1}(a).
\]
Then the corresponding family of {\em real tangent spaces}
\[
\left\{ T_z M_{\varphi(z)} \: : \: z \in \C^n \right\}
\]
is an integrable distribution since it is tangent to the foliation of $\C^n$ made up of the hypersurfaces $\{M_a\}_{a\in \R}$. On the other hand, it is easy to see that
\[
\left\{ T_z^\C M_{\varphi(z)} \: : \: z \in \C^n \right\}
\]
might fail to be integrable.

\bex
If $\varphi(z) := \mathfrak{Im}(z_n)$, then
\[
T_z^\C M_a = \xi + \{ (z^\prime, \, 0) \: : \: z^\prime \in \C^{n-1} \},
\]
where $\xi \in \C^n$ is such that $\mathfrak{Im}(\xi) = a$. Then the distribution of complex tangent spaces defined by $\varphi$ is integrable.
\eex

\bex
If we consider the sphere of radius one in $\C^n$, it is easy to see that the real tangent space cannot coincide with the complex one because
\[
\eta \perp T_p \mathbb{S}^{2n - 1} \implies \imath \eta \parallel T_p \mathbb{S}^{2n - 1}.
\]
\eex

\brmk
Let $\mathbb{S}^{n-1} \subset \R^n$ be the real unit sphere. Then a vector field $X$ is tangent to the sphere at $p$ if and only if
\[
X(x_1^2 + \dots + x_n^2) = 0.
\]
For example, the angular derivative $\Omega_{i,\,j} := x_i \partial_j - x_j \partial_i$ is always tangent to the sphere (in any dimension and for all $i \neq j$).
\ermk

\bex
In $\C^2$, consider the vector fields
\[ \begin{aligned}
& X = \bar{z}_2 \partial_{z_1} -  \bar{z}_1 \partial_{z_2},
\\[1em] & Y = z_2 \partial_{\bar{z}_1} -  z_1 \partial_{\bar{z}_2}.
\end{aligned} \]
It is rather easy to prove that $X$ and $Y$ are tangent to the complex tangent space of the complex sphere $S_\C^1$. However, the commutator is equal to
\[
[X,\,Y] = z_1 \partial_{z_1} + z_2 \partial_{z_2} - \bar{z}_1 \partial_{\bar{z}_1} - \bar{z}_2 \partial_{\bar{z}_2},
\]
and this is not tangent to the complex tangent space; in particular, the distribution
\[
\left\{ T_p^\C S_\C^1 \: : \: p \in \C^2 \right\}
\]
is not integrable.
\eex

\bex[Grushin plane] \index{Grushin plane}
Consider on $\R^2$ the vector fields $X = \partial_x$ and $Y = x \partial_y$. It is easy to verify that $X$ and $Y$ span $\R^2$ everywhere except for the line $\{ x = 0\}$. However,
\[ 
[X, \, Y] = \partial_y,
\]
so they satisfy the H\"{o}rmander condition at order one. We now want to estimate the distance between points on $\R^2$, which we have defined in the previous section as
\[
d_\X(x, \, y) := \inf \left\{ L_\X(\gamma)\: : \: \text{$\gamma$ horizontal curve joining $x$ and $y$} \right\}.
\]
We recall that $\gamma$ is a horizontal curve if and only if $\gamma^\prime(t)$ belongs to the vector space spanned by $X(\gamma(t))$ and $Y(\gamma(t))$. Observe that
\[
p \in \R^2 \: : \: p_x \neq 0 \implies d_\X(p, \, q) \propto d(p, \, q)
\]
since outside of $\{x = 0\}$, the vector field $Y$ in a small neighbourhood of each point behaves similarly to $\partial_y$ multiplied by some constant.

On the other hand, if both $p$ and $q$ belong to the $y$-axis, then we cannot move vertically so a horizontal curve joining them must be as in \autoref{figure:t.1}. Moreover,
\[
L_\X(\gamma) =\inf_{\delta > 0} \left\{ 2 \delta + \frac{h}{\delta}  \right\},
\]
which is achieved when $\delta = \frac{1}{\sqrt{2}} \sqrt{h}$. It follows that
\[
d_\X(p, \, q) = 2 \sqrt{2} \sqrt{h} \simeq \sqrt{h},
\]
so it is not comparable with the Euclidean distance but, at the same time, is coherent with the abstract theory developed in the previous sections.
\eex

\begin{figure}[h!]
\centering
\includegraphics[width = 12cm, height = 8cm]{images/AA3.pdf}
\caption{A picture of the curve joining two point on the $y$-axis in the Grushin plane.}
\label{figure:t.1}
\end{figure}

\bex[Heisenberg space] \index{Heisenberg group}
Consider on $\R^3$ the vector fields
\[
X = \partial_x - \frac{y}{2} \partial_z \quad \text{and} \quad Y = \partial_y + \frac{x}{2} \partial_z.
\]
As before, an easy computation shows that the commutator $[X, \, Y]$ equal to $\partial_z$ and thus the system $\X$ satisfies the H\"{o}rmander condition at order one. However, describing horizontal curves is not as easy as in the Grushin plane, and it can be done via a kind of constructive process which is briefly explained in \autoref{figure:t.2}.
\eex

\begin{figure}[h!]
\centering
\includegraphics[width = 14cm, height = 7cm]{images/AA4.pdf}
\caption{How to construct a horizontal curve in the Heisenberg group.}
\label{figure:t.2}
\end{figure}

\brmk
We can estimate the distance $d_\X$ in these two cases because they both enjoy a very special property. Namely, it is easy to verify that
\[
[X, \, [X, \, Y]] = [Y, \, [Y, \, X]] = 0
\]
so the Baker-Campbell-Hausdorff formula (see \autoref{thm.bchthm}) only consists of three terms:
\[
x \times y = x + y + \frac{1}{2}[x, \, y].
\]
Notice that even a minor modification of the first example, namely $X = \partial_x$ and $Y = e^x \partial_y$, leads to a Lie algebra which is infinite-dimensional since
\[
[X, \, X, \, \dots, \, [X, \, Y] \dots ] = Y
\]
for any number of the element $X$.
\ermk

In the general case, it makes sense to put some restrictions on $\X := \{X_1, \, \dots, \, X_k\}$ which are strictly related to the above examples. \mbox{}
\begin{enumerate}[label={\color{orange}\textbf{(\alph*)}}]
\item There is $N \in \N$ such that commutators of order $\geq N$ are zero. In particular,
\[
\mathrm{Span} \langle X_1, \, \dots, \, X_k, \, \dots \rangle
\]
is a finite-dimensional vector space.
\item The dimension of the span is exactly equal to $n$, i.e., nonzero commutators form a H\"{o}rmander system.
\end{enumerate}

The first assumption is easily justified. Indeed, let $\{X_1, \, \dots, \, X_n\}$ generate $\R^n$ at each point $p \in \R^n$ and consider the elliptic operator
\[
L = \sum_{j = 1}^n X_j^2 = \sum_{j = 1}^n a_j(x) \partial_{x_j}\left( \sum_{i =1}^n a_i(x) \partial_{x_i} \right).
\]
Solving the equation $Lu = 0$ is not trivial, but one could try to build an approximate solution by freezing the coefficients at some $x_0 \in \R^n$ so that $L$ becomes an elliptic operator with constant coefficient. However, freezing
\[
L = \partial_x^2 + x^2 \partial_y^2
\]
at any point on the $y$-axis leads to the equation
\[
\partial_x^2 u(x, \, y) = 0,
\]
so there is a loss of information which results in a loss of regularity with respect to the actual solution.

One idea that allows performing a better solution analysis is to modify the vector fields $X$ and $Y$ slightly in such a way that they satisfy condition {\color{orange}\textbf{(1)}}.

In any case, these two properties are equivalent to the fact that $\R^n$ can be equipped with a multiplication law $\cdot$ which is strictly related to the vector fields. We will come back to this in the next chapter, after a short introduction on Lie groups.

\part{Fourier Analysis on Lie Groups}

\chapter{Lie Groups and Lie Algebras} \thispagestyle{empty}
\label{chapter:Liegroups}

In this chapter, we introduce the notion of {\em Lie group} and set the ground for Fourier analysis on this class of spaces generalizing what we did on $\R^n$.

\section{Introduction to Lie groups}

\bd[Lie group] \index{Lie group}
A {\em Lie group} $\G$ is an abstract group and a smooth manifold in which the two structures are compatible. In other words, the map
\[
\G \times \G \ni (x, \, y) \longmapsto y^{-1} \cdot x \in \G
\]
is smooth.
\ed

\brmk
Notice that the compatibility between the two structures is also equivalent to requiring that
\[
\G \times \G \ni (x, \, y) \longmapsto x \cdot y \in \G \quad \text{and} \quad \ni \G y \longmapsto y^{-1} \in \G
\]
are both smooth maps.
\ermk

\brmk
In some books, a Lie group is defined as an analytic manifold rather than a smooth one. However, it can be proved that in this context smooth implies analytic - see, for example, \cite{serre1}.
\ermk

From now on, the symbol $\G$ will always indicate a Lie group equipped with a multiplication law $\cdot$ compatible with the smooth manifold structure.

\bd\index{Lie group!left-translation}\index{Lie group!right-translation}
Fix $a \in \G$. The {\em left-translation} operator $\ell_a : \G \to \G$ is defined by setting
\[
\ell_a(x) := a \cdot x
\]
and, similarly, the {\em right-translation} operator $r_a : \G \to \G$ is defined by
\[
r_a(x) := x \cdot a^{-1}.
\]
Notice that we use $a$ for $\ell_a$ and $a^{-1}$ for $r_a$ in order to make the composition laws coherent with each other, that is,
\[
\ell_a \ell_b = \ell_{a \cdot b} \qquad \text{and} \qquad r_a r_b = r_{a \cdot b}.
\]
\ed

\brmk
It is easy to verify that both $\ell_a$ and $r_a$ are diffeomorphisms of $\G$ with inverses $\ell_{a^{-1}}$ and $r_{a^{-1}}$ respectively.
\ermk

\begin{notation}
Let $f \in C^\infty(\G)$. We denote by $\check{f}$ the function
\[
\check{f}(x) := f(x^{-1}) \in C^\infty(\G).
\]
In a similar fashion, we introduce the translation operators $L_a, \, R_a : C^\infty(\G) \to C^\infty(\G)$, for $a \in \G$, as follows:
\[ \begin{aligned}
& L_a f(x) := f(a^{-1} \cdot x) = f \circ \ell_a^{-1} (x),
\\[1em] & R_a f(x) := f(x \cdot a) = f \circ r_a^{-1} (x).
\end{aligned}\]
\end{notation}

Recall that Lie group $\G$ is a smooth manifold and therefore the notion of tangent space at some $p \in \G$ is well-defined, for example, through derivations. Recall that
\[
v : C^\infty(\G) \longrightarrow \R
\]
is a {\em derivation} at $p \in \G$ if the following properties hold: \mbox{}
\begin{enumerate}[label=\textbf{(\alph*)}]
\item If $f \equiv g$ in a neighbourhood of $p$, then $v(f) = v(g)$.
\item The {\bf Leibniz rule} holds, that is,\index{Leibniz rule!derivation}
\[
v(fg) = f(p) v(g) + g(p) v(f).
\]
\end{enumerate}

\bpr
Let $\varphi : A \subseteq \R^{q} \to \G$ be a coordinate system around $p_0 \in \G$ and assume that $\varphi(0) = p_0$. Then $v \in T_{p_0}\G$ if and only if there exists $(a_1, \, \dots, \, a_q) \in \R^q$ such that
\[
v(f) = \sum_{j = 1}^q a_j \partial_{x_j}(f \circ \varphi)(0).
\]
In other words, a derivation belongs to the tangent space if and only if it is a directional derivative.
\epr

\bd[Vector field] \index{vector field}
A {\em vector field} $X$ defined on a Lie group $\G$ is a linear operator
\[
X : C^\infty(\G) \longrightarrow C^\infty(\G)
\]
that satisfies the Leibniz rule, that is,
\[
X(fg) = X(g) f + X(f) g.
\]
\ed

We denote by $\X(\G)$ the space of all vector fields defined on $\G$ which is easily seen to be infinite-dimensional. Furthermore, for $p \in \G$ we have
\[
f \longmapsto Xf(p) \in T_p \G,
\]
which means that a vector field $X$ can be used to associate functions to tangent vectors.

\bd \index{vector field!left-invariant}
The vector field $X \in \X(\G)$ is {\em left-invariant} if
\[
X(L_a f) = L_a (X f) \quad \text{for all $f \in C^\infty(\G)$ and all $p \in \G$}.
\]
\ed

\bex
In $\R^n$, the vector field defined by setting
\[
X= \sum_{j = 1}^n a_j(x) \frac{\partial}{\partial x_j}
\]
is left-invariant if and only if $a_j(x) \equiv a_j$ is a constant function for all $j \in \{1, \, \dots, \, n\}$.
\eex

\bpr
Let $X \in \X(\G)$ be a left-invariant vector field and let "$v = X(e)$" be the derivation defined by setting
\[
v(f) = Xf(e).
\]
Then, for all $x \in \G$ and for all $f \in C^\infty(\G)$, the vector field satisfies
\[
X f(x) = v( L_x^{-1} f).
\]
Conversely, given $v \in T_e \G$, the vector field defined by setting
\[
X_v f(x) := v(L_{x}^{-1} f)
\]
is left-invariant.
\epr

\begin{proof}
The first identity is trivial since
\[
X f(x) = L_{x^{-1}} (Xf)(e) = X (L_{x^{-1}}f)(e) = v(L_x^{-1} f).
\]
To prove that $X_v$ is a vector field, we need to check\footnote{It is clearly sufficient to prove that $g$ is smooth in a neighbourhood of $x$.} that $g := X_v f \in C^\infty(\G)$. A straightforward computation shows that
\[
X_v f(x) = v(L_{x}^{-1} f) = \sum_{j = 1}^n a_j \partial_{x_j}( L_x^{-1} f \circ \varphi) (0) = \sum_{j = 1}^n a_j \partial_{x_j} f(x \cdot \varphi(t)),
\]
which means that $g$ is smooth since we can write it as the composition of smooth functions. We now show that $X_v$ is left-invariant. Let $a \in \G$ and notice that
\[ \begin{aligned}
X_v(L_a f)(x) & = v( L_x^{-1} L_a f) =
\\[1em] & = v(L_{x \cdot a^{-1}} f) =
\\[1em] & = X_v f(a^{-1} \cdot x) =
\\[1em] & = L_a X_v(f)(x).
\end{aligned} \]
\end{proof}

It follows from the proposition that $v \mapsto X_v$ is a bijection between $T_e \G$ and the subset of $\X(\G)$ that consists of left-invariant vector fields.

\bd[Lie Algebra] \index{Lie algebra}
The {\em Lie algebra} $\cg$ associated to a Lie group $\G$ is the algebra generated by left-invariant vector fields,
\[
\cg := \{ X \in \X(\G) \: : \: \text{$X$ left-invariant}, \},
\]
and it is a vector space of dimension $q$.
\ed

\brmk
If $X, \, Y \in \cg$, then $[X, \, Y] \in \cg$.
\ermk

The algebra isomorphism between $\cg$ and $T_e \G$ is given by the map $v \mapsto X_v$, where $X_v$ can also be defined via the push-forward
\[
X_v(x) = \ell(x)_\ast v.
\]

\section{Exponential map and one-parameter groups}

Let $X \in \cg$ be a left-invariant vector field and let $\gamma_x$ be the integral curve originating at some point $x \in \G$, namely the solution of the problem
\[\begin{cases}
\gamma_x'(t) = X(\gamma_x(t)),
\\[.6em] \gamma_x(0) = x.
\end{cases}\]
We now claim that for all $x\in \G$ the following identity holds:
\begin{equation} \label{eq.z.1}
\gamma_x(t) = x \cdot \gamma_e(t).
\end{equation}
One way to see this is to use the exponential map. Recall that by definition
\[
\exp(tX) f(e) = f(\gamma_e(t)),
\]
so for any $x \in \G$ it turns out that
\[ \begin{aligned}
\exp(tX) L_{x^{-1}} f(e) & = L_{x^{-1}} \exp(tX) f(e)
\\[1em] & =  L_{x^{-1}} \left[ f(\gamma_e(t)) \right]
\\[1em] & = f(x \cdot \gamma_e(t)).
\end{aligned} \]
The left-hand side is equal to $[L_{x^{-1}} f](\gamma_e(t))$, and thus \eqref{eq.z.1} holds if we can prove that
\[
[L_{x^{-1}} f](\gamma_e(t)) = f(\gamma_x(t)).
\]
This is left as an exercise to the reader to get acquainted with the notions introduced so far.

\brmk
Another way to prove \eqref{eq.z.1} is to show that $x \cdot \gamma_e$ is the integral curve of $X$ starting from $x$. This is an immediate consequence of $X$ left-invariant vector field:
\[ \begin{aligned}
\frac{\mathrm{d}}{\mathrm{d}t} f(x \cdot \gamma_e(t)) & = \frac{\mathrm{d}}{\mathrm{d}t} L_{x^{-1}} f(\gamma_e(t))
\\[1em] & =X(L_{x^{-1}} f)(\gamma_e(t))
\\[1em] & = (Xf)(x \cdot \gamma_e(t)).
\end{aligned} \]
\ermk

The identity \eqref{eq.z.1} is extremely important and carries several consequences, one of which is the existence of one-parameter groups. First, notice that
\[
\text{$\gamma_e(\cdot)$ defined for $t \in (- \epsilon, \, \epsilon) \implies \gamma_x(\cdot)$ defined for $t \in (- \epsilon, \, \epsilon)$}.
\]
In particular, we have $\gamma_{\gamma_e(s)}(t) = \gamma_e(s) \cdot \gamma_e(t)$ and, since we know already (via the composition of the corresponding flows) that
\[
\gamma_{\gamma_e(s)}(t)= \gamma(t + s),
\]
we can infer that $\gamma_e(t + s) = \gamma_e(t) \cdot \gamma_e(s)$. This means that we can extend the domain of $\gamma_e$ to coincide with the real line, and hence
\[
\gamma_e \in C^\infty(\R, \, \G)
\]
is a {\em smooth groups homomorphism} from $\R$ to $\G$. In this case we say that $\gamma_e$\index{one-parameter group} is a {\em one-parameter group} and we can easily define a one-to-one correspondence
\[
\cg \longleftrightarrow \{ \text{one-parameter groups} \}
\]
where $\gamma$, a one-parameter group, gives a left-invariant vector field by setting
\[
Xf(x) = \frac{\mathrm{d}}{\mathrm{d}t} f(x \cdot \gamma(t)).
\]

\bex
The one-parameter groups of $(\R^n, \, +)$ are all of the form
\[
\gamma_v(t) := tv,
\]
where $v \in \R^n$. Notice that from a topological point of view $\gamma$ might not be a {\bf closed} subgroup (e.g., the irrational line in $\mathbb{T}$ which is dense)
\eex

Let $\G$ be a Lie group and identify its Lie algebra $\cg$ with the tangent space. 

\bd\index{Lie group!exponential map}
Fix $v \in T_e \G$ and let $\gamma_v$ be the one-parameter group satisfying the initial condition $\gamma_v^\prime(0) = v$. We call {\em exponential of $\G$} the mapping
\[
\exp_\G(v) := \gamma_v(1).
\]
\ed

\bthm
The exponential map $\exp_\G$ is a local diffeomorphism from a neighbourhood $U$ of $0 \in T_e \G$ to a neighbourhood $V$ of the identity element $e\in \G$.
\ethm

\begin{proof}
Simply notice that $\exp_\G$ is smooth (since it is the restriction of $(t, \, v) \mapsto \gamma_v(t)$ which is smooth) and
\[
\mathrm{d}(\exp_\G)_0 : T_e \G \longrightarrow T_e \G
\]
is the identity map since $\gamma_v^\prime(0)=v$ by definition.
\end{proof}

One might wonder whether or not $\exp_\G$ is a {\bf global} diffeomorphism. However, it is easy to verify that any compact Lie group gives a counterexample since
\[
T_e \G \cong \R^q
\]
is only locally compact so it cannot be diffeomorphic to a compact manifold such as $\G$. For example,
\[
\mathrm{SU}(2, \, \C) \cong S^3 \subset \C^2
\]
is a compact Lie group and $\exp_\G$ is not a global diffeomorphism as one can easily see from the picture below.

\begin{figure}[h!]
\centering
\includegraphics[width = 12.5cm, height = 2.5cm]{images/AA2.pdf}
\caption{Counterexamples to $\exp_\G$ global diffeomorphism.}
\label{fig:z1}
\end{figure}

Furthermore, notice that for $v$ and $w$ in a small neighbourhood of $0 \in T_e\G$ we find that there exists $u$ in the same neighbourhood such that
\[
(\exp_\G v)\cdot(\exp_\G w) = \exp_\G u,
\]
and $u$ is given by the \textsc{BCH} formula. In particular, the series
\[
v + w + \sum_{k_1 + k_2 \geq 1} c_{k_1, \, k_2}, \quad  c_{k_1, \, k_2}\in \mathcal{F}^{k_1, \, k_2}
\]
is convergent in a small neighbourhood of the origin and this is what (more or less) leads to the definition of Lie groups with "analytic" in place of "smooth".

\bex
Let $\G = \R$ and let $\G^\prime = \mathbb{T}$. Then it is easy to verify that
\[
\cg \cong \cg^\prime,
\]
and both are isomorphic to $\R$.
\eex

\brmk
This shows that two Lie groups with the same (up to isomorphism) Lie algebra are not necessarily isomorphic themselves; however, they are locally isomorphic.
\ermk

\bthm
Two Lie groups are locally isomorphic if and only if their Lie algebras are isomorphic.
\ethm

\brmk
What we did until now only cares about the connected component of $\G$ and ignores all the others as if they did not exist. Therefore, assuming that $\G$ is connected is not restrictive in any sense.
\ermk

\section{Lie algebras and connection with Lie groups}

We now introduce the notion of Lie algebra and show how is it related with the notion of Lie groups via left-invariant vector fields.

\bd \index{Lie algebra}
A Lie algebra $\cg$ is a vector space endowed with a bilinear operator $[ \cdot, \, - ] : \cg \times \cg \to \cg$ which is antisymmetric and satisfies the Jacobi identity\index{Jacobi identity}
\begin{equation} \index{Jacobi identity}
[x, \, [y,\,z]] + [z, \, [x,\,y]] + [y, \, [z,\,x]] = 0 \quad \text{for all $x, \, y, \, z \in \cg$}.
\end{equation}
\ed

\bex
The vector space of $n \times n$ real-valued matrices endowed with the bracket operator $[A,\, B] := AB - BA$ is a Lie algebra which is usually denoted by $\mathfrak{gl}(n, \, \R)$. Similarly,
\[
\mathfrak{sl}(n, \, \R) = \left\{ A \in \mathfrak{gl}(n, \, \R) \: : \: \mathrm{Tr}(A) = 0 \right\}
\]
and
\[
\mathfrak{so}(n, \, \R) = \left\{ A \in \mathfrak{gl}(n, \, \R) \: : \: \text{$A$ is skew-symmetric} \right\}
\]
are also Lie algebras with the same bracket operator. Notice that, in this case, the space of symmetric matrices in $\mathfrak{gl}(n, \, \R)$ is {\bf not} a Lie algebra.
\eex

\bex
The Euclidean space $\R^3$ with the multiplication law given by the wedge product $\wedge$ is a Lie algebra which is isomorphic to $\mathfrak{so}(3, \, \R)$ via
\[
\R^3 \ni (v_1, \, v_2, \, v_3) \longmapsto \begin{pmatrix} 0 & v_3 & v_2 \\ - v_3 & 0 & - v_1 \\ - v_2 & v_1 & 0 \end{pmatrix} \in \mathfrak{so}(3, \, \R).
\]
\eex

\bd[Homomorphism] \index{Lie algebra!homomorphism}
A Lie algebras homomorphism $\varphi$ is a linear mapping that preserves the bracket, that is,
\[
[x,\,y] = [\varphi(x), \, \varphi(y)].
\]
\ed

\bd\index{Lie algebra!commutative}
We say that $\cg$ is a {\em commutative} Lie algebra if $[x, \, y] = 0$ for all $x, \, y \in \cg$.
\ed

We now introduce an useful notation for the commutator of linear spaces. More precisely, let $\mathfrak{h}$ be a linear subspace of a Lie algebra $\cg$. Then
\[
[\cg, \, \mathfrak{h}] := \mathrm{Span} \langle [x, \, y] \: : \: x \in \cg, \, y \in \mathfrak{h} \rangle.
\]

\bd\index{Lie algebra!subalgebra}
Let $\cg$ be a Lie algebra. A linear subspace $\mathfrak{h}$ is a {\em Lie subalgebra} if it is closed under the bracket, that is,
\[
[\mathfrak{h}, \, \mathfrak{h}] \subseteq \mathfrak{h}.
\]
Moreover, we say that a Lie subalgebra $\mathfrak{h}$ is an {\em ideal}\index{Lie algebra!ideal} if
\[
[\mathfrak{h}, \, \cg] \subseteq \mathfrak{h}.
\]
\ed

\brmk
If $\mathfrak{h}$ is an ideal of $\cg$, then the quotient can be given a Lie algebra structure by setting
\[
[x + \mathfrak{h}, \, y + \mathfrak{h}] := [x, \, y] + \mathfrak{h}.
\]
Notice that this is not well-defined (i.e., it depends on the choice of the representatives) if $\mathfrak{h}$ is a subalgebra which is not an ideal.
\ermk

\bd \index{Lie algebra!centre}
Let $\cg$ be a Lie algebra. The {\em centre} of $\cg$ is defined as
\[
Z_\cg := \{ x \in \cg \: : \: [x, \, \cg] = 0 \}.
\]
\ed

\brmk
The linear subspace $Z_\cg$ is an ideal of $\cg$.
\ermk

\bd \index{Lie algebra!derived algebra}
Let $\cg$ be a Lie algebra. The {\em derived algebra} of $\cg$ is defined as
\[
\cg' := [\cg, \, \cg].
\]
\ed

The derived algebra of $\cg$ does not always coincide with $\cg$ itself. For example, if $\cg = \mathfrak{gl}(n, \, \R)$ it is easy to verify that
\[
[\cg, \, \cg] \subseteq \mathfrak{sl}(n, \, \R),
\]
and thus it is a proper ideal of $\cg$.

\bex
Let $\mathfrak{t}(n,\,\R)$ be the space of $n\times n$ real-valued upper-triangular matrices. Then $\mathfrak{t}(n,\,\R)$ is a Lie algebra because $AB, \, BA \in \mathfrak{t}(n,\,\R)$, but
\[
[\mathfrak{t}(n,\,\R), \, \mathfrak{t}(n,\,\R)] \subset \mathfrak{t}(n,\,\R)
\]
is a proper ideal because all its elements have the super-diagonal identically equal to zero.
\eex

The previous example suggests that we can iterate this construction a finite number of times. We obtain a sequence of derived algebras which are proper ideals of one another until
\[
[ \mathfrak{t}(n,\,\R), \, [\mathfrak{t}(n,\,\R), \, [\mathfrak{t}(n,\,\R), \, \dots, \, [\mathfrak{t}(n,\,\R), \, \mathfrak{t}(n,\,\R)]]\dots] = 0.
\]
This construction is what is usually called {\em descending central series} and can be done with any other Lie algebra, although there is no guarantee that it will eventually be zero.\index{descending central series}

\bd[Nilpotent] \index{Lie algebra!nilpotent}
Let $\cg$ be a finite-dimensional Lie algebra. If the descending central series stabilises at $0$, then we say that $\cg$ is {\em nilpotent}.
\ed

\bex
Let $\cg$ be the Heisenberg Lie algebra, namely the one generated as
\[
\cg = \mathrm{Span} \langle \partial_x - \frac{y}{2} \partial_z, \, \partial_y + \frac{x}{2} \partial_z, \, \partial_z \rangle.
\]
Then the derived algebra is $\cg' = \mathrm{Span} \langle \partial_z \rangle = \R$ and it is easy to verify that the $\cg$ is nilpotent of step two.
\eex

\bthm[Ado] \index{Ado's theorem}
Every finite-dimensional Lie algebra over $\R$ (or $\C$) is isomorphic to a matrices Lie algebra.
\ethm

The original proof of this result due to Ado can be found in \cite{ados}. There are several others with new methods, but they are all reasonably complicated so we will just skip it.

\bthm[Engel] \index{Engel's theorem}
Every finite-dimensional nilpotent Lie algebra over $\R$ (or $\C$) is isomorphic to a upper-triangular matrices Lie algebra.
\ethm

\bex[Heisenberg]
Let $\cg$ be the Heisenberg space, namely
\[
\cg = \mathrm{Span} \langle \partial_x - \frac{y}{2} \partial_z, \, \partial_y + \frac{x}{2} \partial_z, \, \partial_z \rangle.
\]
It is easy to see that this is isomorphic to the matrices algebra generated by
\[
\begin{pmatrix} 0 & x & z \\ 0 & 0 & y \\ 0 & 0 & 0 \end{pmatrix},
\]
where
\[
X = \begin{pmatrix} 0 & 1 & 0 \\ 0 & 0 & 0 \\ 0 & 0 & 0 \end{pmatrix}, \quad Y = \begin{pmatrix} 0 & 0 & 0 \\ 0 & 0 & 1 \\ 0 & 0 & 0 \end{pmatrix}, \quad Z = \begin{pmatrix} 0 & 0 & 1 \\ 0 & 0 & 0 \\ 0 & 0 & 0 \end{pmatrix}.
\]
\eex

The next theorem asserts that finite-dimensional Lie algebras are always given by $\mathrm{Lie}(\G)$, where $\G$ is a Lie group which is uniquely determined if we require connected and simply connected.

\bthm[Lie] \index{Lie's theorem}
Let $\cg$ be a finite-dimensional real Lie algebra. Then there exists a Lie group $\G$ such that $\cg$ i its Lie algebra. Furthermore, there exists a unique $\G$ which is connected and simply connected.
\ethm

One might wonder if the proof of this result is easier if we use Ado's theorem to restrict ourselves to a smaller class of Lie algebras. However, this is not the case. Let $\cg$ be a Lie algebra contained in $\mathfrak{gl}(n, \, \R)$ and notice that
\[
\mathfrak{gl}(n, \, \R) = \mathrm{Lie}( \mathrm{GL}(n, \, \R) ), 
\]
the set of all matrices with nonzero determinant. Now $\exp_\G(\cg)$ is a "nice" submanifold close to the identity, but we get no information about the behaviour of faraway points.

\bthm
Let $\cg$ be a Lie algebra and let $\G$ be the unique connected and simply connected Lie group with $\mathrm{Lie}(\G) \cong \cg$. If $\mathbb{H}$ is a connected Lie group with $\mathrm{Lie}(\mathbb{H}) \cong \cg$, then $\mathbb{H}$ is isomorphic to a quotient of $\G$ modulo a discrete central subgroup.
\ethm

\bex
For example, $\R$ is the unique connected and simply connected Lie group with Lie algebra isomorphic to $\R$. The torus $\mathbb{T}$ is, in fact, isomorphic to a quotient of $\R$:
\[
\mathbb{T} = \faktor{\R}{2 \pi \Z}.
\]
\eex

\bex
Let us consider
\[
\mathrm{SU}(2,\, \C) = \left\{ \begin{pmatrix} z_1 & - \bar{z_2} \\ z_2 & \bar{z_1} \end{pmatrix} \: : \: |z_1|^2 + |z_2|^2 = 1 \right\}
\]
and
\[
\mathrm{SO}(3,\, \R) = \left\{ R_{\theta, \, v} \: : \: \theta \in [0, \, 2 \pi), \, v \in \R^3, \, |v|=1 \right\}.
\]
It is easy to verify that $\mathrm{SU}(2,\, \C)$ is diffeomorphic to $S^3 \subset \C^2$ so it is connected and simply connected, while $\mathrm{SO}(3,\, \R)$ is connected but not simply connected. It can be proved that
\[
\mathfrak{su}(2, \, \C) \cong \mathfrak{so}(3, \, \R),
\]
and since the center of $\mathrm{SU}(2,\, \C)$ is $\{ \pm \mathrm{Id} \}$, then the unique possibility is that
\[
\mathrm{SO}(3, \, \R) = \faktor{\mathrm{SU}(2,\, \C)}{ \{ \pm \mathrm{Id} \}},
\]
which means that $\mathrm{SU}(2,\, \C)$ is a double covering of $\mathrm{SO}(3,\, \R)$.
\eex

\paragraph{Notation.} So far we introduced an exponential map which sends a vector field to its flow,
\[
\exp(tX) f(x) = f \circ \varphi_t(x),
\]
and an exponential map associated to a Lie group $\G$ that sends $\cg$ to $\G$. The connection between them follows easily if we identify $X_v \in \cg$ with $X_v \in T_e \G$ since
\[
\exp(t X_v) f(x) = f(x \exp_\G(tv)).
\]
Since Ado's theorem asserts that every finite-dimensional Lie algebra $\cg$ over $\R$ is isomorphic to a matrices Lie algebra, it only makes sense to study properties of $\mathrm{GL}(n, \ \R)$.

\bex
Recall that $\gln$ is a dense open set of $\R^{n \times n}$ so it quite clearly a Lie group. The tangent space is
\[
T_e \gln = \mathrm{M}(n, \, \R),
\]
the set of all $n \times n$ matrices, isomorphic to $\R^{n^2}$. Recall that for any $A \in \mathrm{M}(n, \, \R)$ the exponential
\[
\e^A = \sum_{n \in \N} \frac{A^n}{n!}
\]
is well-defined and converges absolutely since
\[
\left\| \sum_{n \in \N} \frac{A^n}{n!} \right\| \leq \sum_{n \in \N} \frac{\|A^n\|}{n!} \leq \sum_{n \in \N} \frac{\|A\|^n}{n!} = \e^{\|A\|}
\]
and the exponential of a real number is well-defined. Now let $\gamma_A(t) := \e^{At}$ and notice that this map satisfies the relations
\[
\gamma_A(s+t) = \gamma_A(t) \cdot \gamma_A(s) \quad \text{and} \quad \gamma_A(t) \cdot \gamma_A(-t) = \mathrm{id},
\]
so $\gamma_A(t)$ belongs to $\gln$ for all $A \in \mathrm{M}(n, \, \R)$. We can be more precise and prove that
\[
\mathrm{det} \, \e^A = \e^{\mathrm{Tr}(A)},
\]
which follows from the Jordan's normal form in $\C$ together with the obvious identity $\e^{PAP^{-1}} = P \e^A P^{-1}$, but we do not need to go as far. Anyway, $\gamma_A$ is a one-parameter group and its derivative is given by
\[
\gamma_A^\prime(t) = A \e^{At}.
\]
Now we want to prove that, given left-invariant vector fields $X_A$ and $X_B$ via the exponential identification, it turns out that
\[
[X_A, \, X_B] = X_{[A, \, B]},
\]
where $[A, \, B] = AB - BA$. Start by observing that
\[ \begin{aligned}
X_A f(x) & = \frac{\mathrm{d}}{\mathrm{d}t} \, \Big|_{t = 0} f(x \exp_\G(A)) = 
\\[1em] & =  \frac{\mathrm{d}}{\mathrm{d}t} \, \Big|_{t = 0} f(x \left( \mathrm{id} + tA + \mathcal{O}(t^2) \right) ) =
\\[1em] & =  \frac{\mathrm{d}}{\mathrm{d}t} \, \Big|_{t = 0} f(x + x tA) =
\\[1em] & =  \frac{\mathrm{d}}{\mathrm{d}t} \, \Big|_{t = 0} f(x_1 + t (xA)_1, \, \dots, \, x_n + t(xA)_n) =
\\[1em] & = \sum_{i, \, j = 1}^n (xA)_{i, \, j} \partial_{x_{i, \, j}} f(x)
\end{aligned} \]
from which it follows that
\[
X_A(e) = \sum_{i, \, j = 1}^n A_{i, \, j} \partial_{x_{i, \, j}}.
\]
If $B$ is another element, it is easy to see that
\[ \begin{aligned}
[X_A, \, X_B] f(e) & = X_A(X_B f) - X_B(X_A f) =
\\[1em] & = \sum_{i, \, j} A_{i, \, j} \partial_{i, \, j} \left( \sum_{k, \, \ell} (xB)_{k, \, \ell} \partial_{k,\, \ell}\right) f(e) - \sum_{i, \, j} B_{i, \, j} \partial_{i, \, j} \left( \sum_{k, \, \ell} (xA)_{k, \, \ell} \partial_{k,\, \ell}\right) f(e)
\\[1em] & = \sum_{i, \, j} A_{i,\, j} \sum_{k, \, \ell} \left[ \partial_{i, \, j} ( xB)_{k,\, \ell} \partial_{k, \, \ell} \right] f(e)
\\[1em] &  = \sum_{i, \, j} \sum_\ell \left[ B_{j, \, \ell} \partial_k f \right] (e) 
\\[1em] & = \sum_{i, \, \ell} (AB)_{i, \, \ell} \partial_{i, \, \ell} f(e) = X_{[A, \, B]}f(e)\end{aligned} \]
The Lie algebra $\mathfrak{gl}(n, \, \R)$ can be identified with $\mathrm{M}(n, \, \R)$ and the exponential map $A \mapsto \e^A$, while nice close to zero, is far from being injective everywhere. Indeed,
\[
\exp \left( t \begin{pmatrix} 0 & 1 \\ -1 & 0 \end{pmatrix} \right) = R_t,
\]
the rotation of angle $t$, is periodic with respect to $t$.
\eex

\section{Nilpotent Lie algebras}

In this section, we take a closer look to nilpotent Lie algebras and Lie groups since in the final part of the course we will study the sublaplacian operator on a special subclass of nilpotent groups.

\bthm \label{thm.algebraassocia}
Let $\cg$ be a nilpotent Lie algebra of step $m$ and define
\[
S(a, \, b) = a + b + \frac{1}{2}[a, \, b] + \sum_{k = 3}^m c_k(a, \, b).
\]
Then $S$ is a multiplication law defining a Lie group structure on $\cg$ which Lie algebra is isomorphic to $\cg$.
\ethm

\begin{proof}
First, notice that $S(a, \, 0) = a$ and $S(0, \, b) = b$ so the linear space $0$ plays the role of the identity in $\G$. Moreover,
\[
S(a, \, - a) = 0,
\]
so each element $a \in \G$ has an inverse which is given by $-a$. The associative property,
\[
S(S(a, \, b),\, c) = S(a, \, S(b,\,c),
\]
on the other hand, is difficult to prove. We use Ado's theorem and reduce to a simpler case since working on a manifold $\cong \R^N$ allows one to assume the simple connectedness.

\paragraph{Associativity.} The following result is given for granted:

\bl
There exists a neighbourhood of $0$ in $\mathfrak{gl}(n, \, \R)$ such that for all $A, \, B \in U$ the Baker-Campbell-Hausdorff formula converges.
\el

Let $U$ be a neighbourhood of $0$ in $\mathfrak{gl}(n, \, \R)$ such that the restriction of the exponential map is a diffeomorphism with an open neighbourhood of the identity matrix $V$. Select a smaller neighbourhood $V'$ in such a way that
\[
V' \cdot V' \subset V,
\]
and, similarly, select an even smaller one such that
\[
V'' \cdot V'' \subset V'.
\]
Let $U'$ and $U''$ be the preimages of $V'$ and $V''$. We can assume that $U'$ satisfies the assumption of the lemma above (otherwise we can take a smaller one). It follows that
\[
\forall a, \, b \in U', \, \exp(S(a, \, b)) \in V.
\]
Now let $a, \, b, \, c \in U^{''}$. It is easy to see that $S(a,\, b), \, S(a, \, c), \, S(b, \, c) \in U'$ and thus
\[
\e^{S(S(a, \, b), \, c)} = \e^{S(a, \, b)} \e^c = \e^a \e^b \e^c = \e^a \e^{S(b, \, c)} = \e^{S(a, \, S(b, \, c))}.
\]
This shows that associativity rule holds in $U''$ and since this is a polynomial identity that holds for small parameters we can extend it everywhere by analytic continuation (since there is a finite number of addenda).
\end{proof}

\section{Homomorphisms of Lie algebras and dilations}

Let $\Psi : \G \to \G'$, $\G$ connected, be a {\em homomorphism of Lie groups}\index{Lie group!homomorphism}, that is, a smooth\footnote{Requiring measurable here leads, surprisingly, to an equivalent definition.} map which is also a group homomorphism. Then
\[
\mathrm{d}\Psi_e : T_e \G \to T_e \G'
\]
induces a map $\Psi_\ast : \cg \to \cg'$ composing with the isomorphisms described earlier in the chapter that identify the Lie algebra with the tangent at the identity element. Let $v \in T_e \G$ and let $\gamma_v(t)$ be the relative one-parameter group ($\gamma_v^\prime(0) = v$). Then
\[
\gamma_{v'}(t) := \Psi \circ \gamma_v(t)
\]
is a one-parameter group in $\G'$ and hence turns out that $\Psi_\ast(v) = v'$.

\bpr
The map $\Psi_\ast$ is a Lie-algebra isomorphism, namely
\[
\Psi_\ast [u,\, v] = [\Psi_\ast u,\, \Psi_\ast v].
\]
\epr

\begin{proof}
Let $v \in \cg$. Then $\gamma_v(t) = \exp_\G(tv)$ and
\[
X_v f(x) = \frac{\rmd}{\rmd t} \, \Big|_{t = 0} f(x \cdot \exp_\G(tv)).
\]
is the corresponding left-invariant vector field. Let $h \in C^\infty(\G')$, $f := h \circ \Psi \in C^\infty(\G)$, and notice that
\[ \begin{aligned}
X_v f(x) & = \frac{\rmd}{\rmd t} \, \Big|_{t = 0} h \left( \Psi(x \cdot \exp_\G(tv)) \right)
\\[1em] & = \frac{\rmd}{\rmd t} \, \Big|_{t = 0} h \left( \Psi(x) \cdot \exp_{\G'}(tv') \right)
\\[1em] & = Y_{v'} h(\psi(x)) = Y_{v'} f(x),
\end{aligned} \]
which means that $X_v = Y_{\Psi_\ast(v)}$, where $Y_w$ is the left-invariant vector field relative to $w$ in the Lie algebra associated to $\G'$.
\end{proof}

\bpr
If $\G$ is connected, then $\Psi_\ast$ uniquely determines $\Psi$.
\epr

\begin{proof}
If $\Phi$ and $\Psi$ are two Lie groups homomorphisms with $\Phi_\ast = \Psi_\ast$, then
\[
\Phi(\exp_\G(tv)) = \exp_{\G'}( t \Phi_\ast v) = \exp_{\G'}( t \Psi_\ast v) = \Psi(\exp_\G(tv)).
\]
This means that $\Phi$ and $\Psi$ coincide in a neighbourhood $U$ of the identity element $e \in \G$, which we can assume to be symmetric. Then
\[
\Phi \equiv \Psi
\]
on each power $U^k$ of $U$, and hence we can consider $\mathcal{U} := \bigcup_{n \in \N} U^n$ subgroup of $\G$ since it is closed under multiplication and inverse. By definition, $\mathcal{U}$ is open and, using the identity
\[
\G \setminus \mathcal{U} = \bigcup_{p \notin \mathcal{U}} p \cdot \mathcal{U},
\]
we also infer that $\mathcal{U}$ is closed. Since $\G$ is compact the only possibility is $\mathcal{U} = \G$.
\end{proof}

\bcor
The map $\ast : \homs(\G,\, \G') \to \homs(\cg, \, \cg')$ is injective.
\ecor

\brmk
In general, it is not surjective. If $\G = \mathbb{T}^2$ and $\G' = \R$, then the unique Lie algebra homomorphism $\mathrm{id} : \R \to \R$ cannot be lifted to a map from $\mathbb{T}^2$ to $\R$.
\ermk

\bthm
Let $\G$ be a connected and simply connected Lie group. If $\eta \in \homs(\cg,\, \cg')$, then there exists $\Psi \in \homs(\G,\, \G')$ such that
\[
\Psi_\ast = \eta.
\]
\ethm

\brmk
If $\eta \in \homs(\cg,\, \cg')$ and $\cg$ is nilpotent, then the subalgebra $\eta(\cg) \subset \cg'$ is also nilpotent.
\ermk

\begin{proof}
We prove the result for $\G$ nilpotent. We proved that we can always assume $\G = \cg$ endowed with the Baker-Campbell-Hausdorff formula in place of the multiplication law. Notice that
\[
\G' = \faktor{\widetilde{\G'}}{D},
\]
so we can assume without loss of generality that $\G'$ is also connected and simply connected (up to replacing it with its universal covering). If $u,\, v \in \cg$, then
\[
u \cdot v = u + v + \frac{1}{2}[u,\,v] + \sum_{k \geq 3}^\iota c_k(u,\,v) =: S(u,\,v),
\]
and we can define
\[
\Psi(u) := \exp_{\G'}(\eta(u)).
\]
Then
\[ \begin{aligned}
\Psi(u) \cdot \Psi(v) & = \exp_{\G'}(\eta(u)) \exp_{\G'}(\eta(v))
\\[1em] & = \exp_{\G'}(S(\eta(u),\, \eta(v)))
\\[1em] & = \exp_{\G'}(\eta(S(u,\,v))) = \Psi(u\cdot v),
\end{aligned} \]
which means that $\Psi_\ast = \eta$.\end{proof}

\bd[Dilations] \index{dilation}
A one-parameter family of automorphisms $\{\eta_t\}_{t > 0} \subset \mathrm{Aut}(\cg)$ is a family of {\em dilations} if there exists a basis $\{e_1,\, \dots,\, e_q\}$ of $\cg$ and positive numbers $\lambda_1,\, \dots,\, \lambda_q \in \R_+$ such that
\[
\eta_t(e_j) = t^{\lambda_j} e_j.
\]
\ed

\bex
Consider $e_1,\, e_2$ orthonormal basis of $\R^2$ and define
\[
\eta_t(e_j) = t^{\lambda_j} e_j,
\]
where $\lambda_1 = 1$ and $\lambda_2 = 2$. The trajectories are half-parabolas as in the figure below:
\eex

\brmk
We can decompose $\cg$ in such a way that $\lambda_1 < \cdots < \lambda_k$ and
\[
\cg = \mathfrak{w}_1 \oplus \cdots \oplus \mathfrak{w}_k,
\]
where $\eta_t \, \big|_{\mathfrak{w}_j} = t^{\lambda_j} \mathrm{id}$ for each $j = 1,\, \dots, \, k$. Moreover
\[
\eta_{t \cdot s} = \eta_t \circ \eta_s,
\]
and since $\eta_t$ is an automorphism of the Lie algebra, we also have that
\[
\eta_t([u,\,v]) = [\eta_t u,\, \eta_t v].
\]
This property has an immediate consequence, namely if $u \in \mathfrak{w}_j$ and $v \in \mathfrak{w}_\ell$, then the commutator $[u,\,v]$ belongs to $\mathfrak{w}_{j + \ell}$ (which may also coincide with $\{0\}$).
\ermk

Clearly, $\lambda_j + \lambda_\ell > \max\{\lambda_j,\, \lambda_\ell\}$ so the existence of a family of dilations gives an upper bound on the number of commutators that one can take before it gives zero.

\bpr
A Lie algebra with a family of dilations is nilpotent.
\epr

\brmk
The opposite assertion is not true, but finding a counterexample requires a big effort since one needs commutators sufficiently complicated.
\ermk

\bex \label{ex.1.232.3}
Let $e_1,\, e_2,\, e_3$ be vectors in $\R^3$ such that $[e_1,\,e_2] = e_3$. Then
\[
\eta_t(e_1) = t^\lambda e_1 \quad \text{and} \quad \eta_t(e_2) = t^\mu e_2
\]
implies $\eta_t(e_3) = t^{\lambda + \mu} e_3$. These are the unique dilations for which the $e_i$'s are all eigenvectors.
\eex

\bd
If $\cg$ is a nilpotent Lie algebra with dilations $\delta_t$, then $\G = \cg$ endowed with the BCH formula has a family of automorphisms such that
\[
(\Psi_t)_\ast = \delta_t,
\]
which is usually referred to as {\em dilations of a Lie group}.
\ed

\section{Graded and stratified Lie algebras}

We start with the definition of a particular class of Lie algebras which is rather important for the consequences the decomposition bears.

\bd
Let $\cg$ be a nilpotent group. If there exists a vector space decomposition
\[
\cg = \cw_1 \oplus \cdots \oplus \cw_k
\]
such that
\[
[\cw_i, \cw_j] \subseteq \cw_{i+j},
\]
then we say that $\cg$ is a {\em graded Lie algebra}\index{Lie algebra!graded}. 
\ed

\brmk
Notice that a graded algebra is nilpotent, but it does not determine the step since, for example,
\[
\R^3 = \R \oplus \R \oplus \R = \R^2 \oplus \R.
\]
\ermk

\bl
If $\cg$ is a graded Lie algebra, the maps $\eta_t$ defined by setting
\[
\eta_t \, \big|_{\cw_j} := t^j \, \mathrm{Id}_{\cw_j}
\]
for $j = 1,\dots,k$ give a well-defined family of dilations on $\cg$.
\el

\bd\index{Lie algebra!stratified}
If $\cg$ is a graded Lie algebra generated by $\cw_1$, we say that $\cg$ is {\em stratified}.
\ed

\bex
The Heisenberg algebra described in \hyperref[ex.1.232.3]{Example \ref{ex.1.232.3}} is stratified if and only if $\lambda = \mu$ since we need both vectors $e_1$ and $e_2$ in $\cw_1$.
\eex

\bex
Let $\cg$ be the Lie algebra given by upper-triangular matrices. Then we can take $\mathfrak{w}_j$ to be the subalgebra generated by the matrices with {\bf only} the $j$th over-diagonal nonzero, and this makes $\cg$ a stratified algebra.
\eex

\bd[Sublaplacian] \index{sublaplacian}
Let $\cg$ be a stratified Lie algebra. If
\[
\mathfrak{w}_1 = \mathrm{Span} \langle e_1,\, \dots,\, e_m \rangle,
\]
then we can consider the differential operator
\[
- \Delta := - \sum_{j = 1}^m X_j^2.
\]
We say that $- \Delta$ is the {\em sublaplacian operator} on $\G$ since it is homogeneous of degree two.
\ed

\section{Homogeneous norms on Lie groups}

Let $\cg$ be a nilpotent Lie algebra, $\{ \delta_t \}_{t > 0}$ a family of dilations and $\G$ the connected and simply connected Lie group given by $\cg$ equipped with the multiplication law \eqref{bchformula}.

\bd \index{homogeneous norm}
A {\em homogeneous norm} on $\G$ is a map $|\cdot| : \G \to [0,\,\infty)$ satisfying the following properties: \mbox{}
\begin{enumerate}[label=(\roman*)]
\item $x\mapsto |x|$ is continuous with respect to the topology on $\G$;
\item $|x| = 0$ if and only if $x = 0$, the zero vector of $\cg$;
\item if $\delta_t$ also denotes the dilations on $\G$, then $|\delta_t x| = t |x|$ for all $t > 0$.
\end{enumerate}
\ed

We always refer to such a map as {\em homogeneous norm} because often, as we will see shortly, it does not satisfy the triangular inequality and thus it is not a norm.

\brmk
Let $\cg = \cv_1 \oplus \cdots \oplus \cv_k$ be the decomposition induces by the dilations in such a way that
\[
\delta_t \, \big|_{\mathfrak{v}_i} = t^{\lambda_i} \mathrm{Id}_{\cv_i}.
\]
If $\|\cdot \|$ is the norm inducing the topology on $\cg$, then it is easy to verify that
\[
|x| := \sum_{i = 1}^k \| x_i \|^{\frac{1}{\lambda_i}}
\]
is a homogeneous norm. This shows that in ({\romannumeral 1}) continuity is the ``right'' requirement because $|\cdot|$ is not even $C^1$ as soon as $\lambda_i \geq 2$ for some $i$.
\ermk

\bpr
A homogeneous norm satisfies a quasi-triangular inequality\index{quasi-triangular inequality}. In other words, there exists a constant $C \geq 1$ such that
\begin{equation} \label{quasitr}
|x \cdot y| \leq C(|x| + |y|) \quad \text{for all $x,y\in\G$}.
\end{equation}
\epr

\begin{proof}
The set $B := \{ x \in \G \: : \: |x| \leq 1\}$ is compact and so is $B^2 := B\cdot B$ because the multiplication law is continuous. Consequently, there exists a constant $C \geq 1$ such that
\[
B^2 \subseteq \{ x \in \G \: : \: |x| \leq C \},
\]
which can be easily rewritten as follows:
\[
|x|,\,|y| \leq 1 \implies |x \cdot y| \leq C.
\]
Now take $x,\, y \neq 0$ and notice that, if $t := \frac{1}{|x|+|y|}$, then $|\delta_t x| \leq 1$ and $|\delta_t y| \leq 1$. Applying the inequality above shows that
\[
|\delta_t x \cdot \delta_t y| \leq C \implies |x \cdot y| \leq \frac{C}{t} = C(|x| + |y|).
\]
\end{proof}

\brmk
It follows from the definition and \eqref{quasitr} that the map $d(x,y) := |x^{-1}\cdot y|$ is a left-invariant homogeneous quasi-distance\index{quasi-distance} (distance if $C = 1$):
\[
d(x,y) = d(zx, zy) \quad \text{and} \quad d(\delta_t x, \delta_t y) = t d(x,y).
\]
\ermk

Notice that a quasi-distance does not induce another topology on $\G$ in general, and this is the reason why we require $|\cdot|$ to be continuous with respect to the topology of $\G$

\brmk
The quasi-distance $d$ is, in general, not smooth. However, it is possible to obtain smoothness outside of the origin by taking the Euclidean sphere $S_\cg$ in $\G$ and setting
\[
|x| := \frac{1}{t} \quad \text{if $\delta_t x \in S_\cg$.}
\]
The reader should check that this is a homogeneous norm which is also smooth by the implicit function theorem (except at $x = 0$).
\ermk

In {\bf stratified} groups, it is easy to construct a homogeneous norm that satisfies \eqref{quasitr} with constant $C = 1$. For this, let $\cv_1 \oplus \cdots \oplus \cv_k$ be a stratification and write
\[
\cv_1 = \mathrm{Span} \langle e_1,\, \dots,\, e_q \rangle.
\]
The corresponding vector fields $X_1,\,\dots,\, X_q$ satisfy the H\"{o}rmander condition on $\G$, and hence we can consider the Carnot-Carathéodory distance
\[
d(x, y) := \left\{ L_\X(\gamma) \: : \: \text{$\gamma$ horizontal curve joining $x$ and $y$} \right\},
\]
where $L_\X$ is the length defined in \hyperref[nonisolength]{Section \ref{nonisolength}}. It is easy to verify that this is a homogeneous distance which is left-invariant and satisfies \eqref{quasitr} with $C=1$.

\bthm[\cite{sik}]
Every nilpotent connected and simply connected group with dilations has a homogeneous norm which satisfies the triangular inequality (i.e., $C = 1$) and is smooth outside of the origin.
\ethm

\bpr
If $|\cdot|$ and $|\cdot|'$ are two homogeneous norm on a Lie group $\G$ equipped with the same family of dilations, then they are equivalent.
\epr
\chapter{Fourier Analysis on Lie Groups} \thispagestyle{empty}
\label{chapter:FouLiegroups}

The main goal of this chapter is to develop Fourier analysis on unimodular Lie groups and investigate fundamental solutions and local solvability of differential operators.

\begin{customthm}{A}
Let $\G$ be a locally compact and separable group. Then there exists a locally finite left-invariant measure on $\G$, which is unique up to a multiplicative constant.
\end{customthm}

If $\G$ is a unimodular group, then the convolution of two functions (or a function and a distribution) can easily be defined as in the Euclidean setting replacing $\dr x$ with the Haar measure. Namely, if $\Phi \in \cD'(\G)$ and $f \in C_c(\G)$, one can set
\[
\langle \Phi \ast f,\, g \rangle := \langle \Phi, g \ast \check{f} \rangle.
\]
We show that several properties valid in the Euclidean setting can be extended to unimodular Lie groups with little effort. Next, we introduce left-invariant differential operators
\[
D : C_c^\infty(\cM) \to C^\infty(\cM)
\]
and prove two characterizations in terms of polynomials defined on the corresponding Lie algebra. In conclusion, we show that some operators have fundamental solutions:

\begin{customthm}{B}
Let $Q$ be the homogeneous dimension of $\G$ and let $D$ be a left-invariant operator of homogeneity order $\beta \leq Q$, that is,
\[
D(f \circ \delta_t) = t^\beta (Tf) \circ \delta_t.
\]
Assume also that $D$ and $\transp{D}$ are hypoelliptic. Then $D$ has a fundamental solution with homogeneity order $-Q + \beta$.
\end{customthm}

\section{Existence of Haar measures}

In this section, we recall a few properties concerning the existence and uniqueness of left-invariant measures defined on a {\bf topological} group $\G$.

\bd[Radon measure] \index{Radon measure}
Let $X$ be a Hausdorff topological space. A {\em Radon measure} is a measure on $\cB(X)$ that satisfies the following properties:
\begin{enumerate}[label=(\roman*), topsep=.05em]
\item It is finite on all compact sets;
\item outer regular on all Borel sets; and
\item inner regular on all open sets.
\end{enumerate}
\ed

\bd[Push-forward] \index{measure!push-forward} Let $\mu$ be a positive Radon measure on $\X$, and let $f : \X \to \Y$ be a Borel function. The {\em push-forward} measure of $\mu$ via $f$ is defined by setting
\[
f_{\can}\mu(E) := \mu \left( f^{-1}(E) \right) \quad \text{for all $E \in \cB(\Y)$}. 
\]
\ed

\brmk
Let $\left(\X, \, \cB(\X) \right)$ and $\left(\Y, \, \cB(\Y) \right)$ be two measure spaces, and let $\mu$ be a positive Radon measure on $\X$. Then the push-forward $f_\can \mu$ is a well-defined measure on $\cB(\Y)$.
\ermk

Let $\G$ be a topological group. For any $y \in \G$, we denote by $\ell_y$ the left-multiplication map ($x \mapsto y \cdot x$) and by $r_y$ the right-multiplication map ($x \mapsto x \cdot y^{-1}$).

\bd[Invariant Measure] \index{invariant measure}\index{Haar measure}
Let $\mu$ be a measure defined on $\G$. We say that $\mu$ is a {\em left-invariant} measure if
\[
\left( \ell_y \right)_{\can} \mu = \mu \quad \text{for all $y \in \G$},
\]
and a {\em right-invariant} one if
\[
\left( r_y \right)_{\can} \mu = \mu \quad \text{for all $y \in \G$}.
\]
A measure $\mu$ which is both left-invariant and right-invariant is simply called \textit{invariant}.
\end{definition}

\bthm \label{theorem:1das}
Let $\G$ be a compact group. Then there exists a unique left-invariant (or right-invariant) probability measure on $\G$, called {\bf Haar measure}.\index{Haar measure}
\ethm

We present the proof of this result under the additional assumption $\G$ commutative. The reader will find in \href{http://aurora.asc.tuwien.ac.at/~funkana/downloads_general/sem_kiesenhofer.pdf}{this paper} the proof of the result in the general case.

\begin{proof}[Idea of the proof]
Let $\G$ be a commutative group, and let $\cP(\G)$ be the space of probability measures defined on $\G$. For a given $g \in \G$, consider
\[
\cP_g := \left\{ \mu \in \cP(\G) \: \left| \: \left(\ell_g\right)_{\can} \mu = \mu \right. \right\},
\]
the subset of $\cP(\G)$ that contains all the $g$-invariant probability measures.

\proofstep{Step 1} We claim that, for any $g \in \G$, the set $\cP_g$ is nonempty. Fix $\mu_0 \in \cP$ and define, for every $n \in \N$, the probability measure
\[
\mu_n := \frac{\mu_0 + \left(\ell_{g} \right)_{\can} \mu_0 + \cdots + \left(\ell_{g^n} \right)_{\can} \mu_0}{n+1} \in \cP(\G),
\]
where $g^{n} := g \cdots g$ is the product of $n$ copies of $g$. By compactness, there exists a subsequence $\mu_{n_k}$ weakly$\star$ converging to a measure $\mu_\infty$. It is easy to see that
\[
\left(\ell_g \right)_{\can} \mu_{n_k} \to \mu_\infty \implies \left(\ell_g\right)_{\can} \mu_\infty = \mu_\infty,
\]
which means that $\mu_\infty$ is $g$-invariant or, in other words, $\mu_\infty \in \cP_g$.

\proofstep{Step 2} Let $g, \, h \in \G$ be two elements, let $\mu_0 \in \cP_g$ be an invariant measure, and let $\mu_\infty$ be the weak$\star$ limit of the sequence
\[
\mu_n := \frac{\mu_0 + \left(\ell_{h} \right)_{\can} \mu_0 + \dots + \left(\ell_{h^n} \right)_{\can} \mu_0}{n+1} \in \cP(\G).
\]
Since $\cP_g$ is weakly$\star$ closed, we conclude that $\mu_\infty \in \cP_g \cap \cP_h$. An inductive argument proves that the family $\{\cP_g\}_{g \in \G}$ has the finite intersection property which, by compactness, implies
\[
\bigcap_{g \in \G} \cP_g \neq \varnothing.
\]

\proofstep{Step 3} We claim that the intersection above only consists of one element. In order to prove that, we define the ``convolution of two measures'' by setting
\[
\mu_1 \ast \mu_2 (E) := \left(\mu_1 \times \mu_2 \right) \left( \{ (x_1, \, x_2) \: \left| \: x_1 + x_2 \in E \right. \} \right).
\]
The reader may prove that the convolution is commutative, and also that
\begin{equation} \label{eq.zz.1}
\mu_1 \ast \mu_2 = \mu_1,
\end{equation}
whenever $\mu_1$ is a left-invariant measure. Now, if $\lambda, \, \mu \in \cap_{g \in \G} \cP_g$ are two invariant measures, then property \eqref{eq.zz.1} implies the uniqueness:
\[
\mu = \mu \ast \lambda = \lambda \ast \mu = \lambda \implies \mu = \lambda.
\]
\end{proof}

\bthm \label{thm.invmes}
Let $\G$ be a locally compact and separable group. Then there exists a locally finite left-invariant measure on $\G$, unique up to a multiplicative constant.
\ethm

A proof of this theorem can be found in most geometric measure theory books, but for a quick overview the reader may consult \href{http://simonrs.com/HaarMeasure.pdf}{this paper}.

\brmk
Notice that \hyperref[thm.invmes]{Theorem \ref{thm.invmes}} gives the existence of a left-invariant measure $\mu$ and, with a similar proof, of a right-invariant measure $\lambda$. However, in general $\lambda \neq \mu$.
\ermk

\bd[Unimodular] \index{unimodular group}
A locally compact and separable group $\G$ whose left-invariant measure is right-invariant is called {\em unimodular}.
\ed

We can immediately give a characterization of unimodular Lie groups in terms of the determinant of the adjoint representation; we first start off with a few definitions.

\bd[Adjoint] \index{adjoint representation!of a Lie group}
Let $\G$ be a Lie group and let
\[
\Psi : \G \to \mathrm{Aut}(\G)
\]
be the mapping that sends $g$ to $\Psi_g$, where $\Psi_g$ is the inner automorphism $h \mapsto g h g^{-1}$. The {\em adjoint map} is defined as
\[
\Ad_g = \mathrm{d} (\Psi_g)_e : \cg \to \cg
\]
using the identification $T_e \G \cong \cg$. The corresponding mapping
\[
\Ad : \G \ni g \longmapsto \Ad_g \in \mathrm{Aut}(\cg)
\]
is called {\em adjoint representation} of the group $\G$.
\ed

\bd[Adjoint] \index{adjoint representation!of a Lie algebra}
Let $\cg$ be a Lie algebra over some field. Given $x \in \cg$ we can define the {\em adjoint action}\index{adjoint action} as
\[
\ad_x : \cg \ni y\longmapsto [x,y] \in \cg.
\]
Since the bracket is bilinear, this determines a linear mapping
\[
\ad : \cg \ni x \longmapsto \ad_x \in \End(\cg),
\]
which is usually called {\em adjoint representation} of the Lie algebra $\cg$.
\ed

\brmk
A Lie group $\G$ is unimodular if and only if
\[
\left| \mathrm{det}(\Ad_g) \right| = 1 \quad \text{for all $g \in \G$}.
\]
For a connected Lie group $\G$, this is equivalent to requiring that the trace of $\ad_g$ is zero for all $g \in \cg$, as the reader might show to practice.
\ermk

\brmk
The following classes of groups are unimodular: \mbox{}
\begin{enumerate}[label=(\alph*), itemsep=.1em]
\item compact groups;
\item discrete groups;
\item commutative locally compact groups;
\item connected reductive Lie groups;
\item locally compact nilpotent groups (in particular, nilpotent Lie groups).
\end{enumerate}
\ermk

\bcor
If $\G$ is unimodular and $\dr x$ is the invariant measure, then
\[
\int_\G f(x) \, \dr x = \int_\G f(x^{-1}) \, \dr x.
\]
\ecor

Now let $\G$ be a nilpotent, connected and simply connected Lie group endowed with the BCH-coordinates (also known as {\em canonical coordinates of the first kind})\index{canonical coordinates of the first kind} and let
\[
\cg = \cg_0 \supset \cg_1 \supset \cdots \supset \cg_m = \{0\}
\]
be the {\em central descent sequence} given by the derivate groups. Let $\cv_1,\,\dots,\,\cv_m$ be linear spaces such that the following decomposition holds for all $0 \leq k \leq m-1$:
\[
\cg_k = \cg_{k+1} \oplus \cv_{k+1}.
\]
Then $\cg = \cv_1 \oplus \cdots \oplus \cv_m$ and $x \in \cg$ can be written as a $m$-tuple $(x_1,\dots,x_m)$ with $x_k \in \cv_k$ for all $k = 1,\dots,m$. Consequently, the product between any two elements is given by
\[ \begin{aligned}
x \cdot y = \Bigg(x_1 + y_1, x_2 + y_2 &+ \frac{1}{2}[x_1,y_1], \dots, x_k + y_k + P_k(x_1,\dots,x_{k-1},\,y_1,\dots,y_{k-1}),
\\[1em]& \dots, x_m + y_m + P_m(x_1,\dots,x_{m-1},y_1,\dots,y_{m-1}) \Bigg),
\end{aligned}\]
where each $P_j$ is a polynomial of order $j$. Let $\dr x$ be the Lebesgue measure on $\cg$ (which is a linear space) and notice that the change of variables formula implies
\[
\int_\G f(a \cdot x) \, \dr x = \int_\G f(x) \, \dr x.
\]
This is easy to check because we can change one variable at a time starting from the last one ($x_m$). The reason is that, if we look at the formula above we have
\[
x_m + y_m + P_m(x_1,\,\dots,\,x_{m-1},\,y_1,\,\dots,\,y_{m-1}),
\]
and this depends {\bf linearly} on $x_m$ (that is, equal to $x_m + c$). We now summarize the result we obtained here in the following proposition: 

\bpr
Let $\G$ be a nilpotent, connected and simply connected Lie group. Then the Lebesgue measure is invariant (both left and right).
\epr

\brmk
We can use the exponential map to introduce the so-called {\em canonical coordinates of the second kind}\index{canonical coordinates of the second kind}. Let $\cg = \cv_1 \oplus \cdots \oplus \cv_m$ and consider the mapping
\[
\Phi(x_1,\dots,x_m) := \exp_\G (x_1) \cdots \exp_\G (x_m).
\]
Then $\Phi$ is a local diffeomorphism, which can be used to parametrize $\G$ via coordinates which are the ones we just mentioned.
\ermk

\subsection{Homogeneous dimension}

Let $\G$ be a homogeneous group, let $\cg = \cv_1 \oplus \cdots \oplus \cv_m$ be the decomposition induced by the family of dilations $\{\delta_t\}_{t > 0}$ and $\dr x$ the Haar measure. Then it is easy to verify that
\begin{equation} \label{eq.zz.12}
\int_\G f(\delta_t x) \, \mathrm{d}x = t^{-Q} \int_\G f(x) \, \mathrm{d}x,
\end{equation}
where $Q$ is the so-called {\em homogeneous dimension of $\G$}\index{homogeneous dimension} given by the sum
\[
Q = \sum_{i = 1}^m \lambda_i \, \dim (\cv_i),
\]
where $\delta_t \, \big|_{\cv_i} \equiv t^{\lambda_i} \, \mathrm{id}_{\cv_i}$. Using $f = \chi_A$ as a test function in \eqref{eq.zz.12} gives the identity
\[
\mu(\delta_t A) = t^Q \mu(A) \quad \text{for all $A \subseteq \G$}.
\]
Furthermore, the integral
\[
\int_{|x|<1} |x|^{-\alpha} \, \dr x
\]
converges if and only if $\alpha < Q$. This ``shows'' that the homogeneous dimension plays the role of the topological dimension in Lie groups equipped with a Haar measure.

\section{Convolution on Lie groups}

Let $\G$ be a {\bf unimodular} Lie group, $\dr y$ the invariant measure and $f,\,g \in C_c(\G)$. The {\em convolution}\index{Lie group!convolution} of $f$ and $g$ is the function defined by setting
\[
f \ast g(x) := \int_\G f(x \cdot y^{-1}) g(y) \, \dr y.
\]
A simple change of variables shows that
\[
f \ast g(x) = \int_\G f(y) g(y^{-1} \cdot x) \, \dr y,
\]
but this is different from the convolution
\[
g \ast f(x) = \int_\G f(y) g(x\cdot y^{-1}) \, \dr y.
 \]
The reason is that the product $y^{-1}\cdot x$ is not equal anymore to $x \cdot y^{-1}$ for all $x,\,y \in \G$ unless the group is commutative. We can actually prove a stronger characterization:

\bpr
The convolution is commutative on $C_c(\G)$ if and only if $\G$ is abelian.
\epr

\begin{proof}
One implication is trivial. Therefore, assume that that $\G$ is not abelian and take any two points $x,y \in \G$ such that
\[
x\cdot y \neq y\cdot x.
\]
Let $U_1$ and $U_2$ be neighborhoods of $x$ and $y$ respectively which are disjoint. By continuity of the product, we can always find $V_1 \ni x$ and $V_2 \ni y$ such that
\[
V_1 \cdot V_2 \subseteq U_1 \quad \text{and} \quad V_2 \cdot V_1 \subseteq U_2.
\]
If $f \in C_c(V_1)$ and $g \in C_c(V_2)$, the well-known supports inclusions
\[
\mathrm{spt}(f \ast g) \subseteq \mathrm{spt}(f) \cdot \mathrm{spt}(g) \subseteq U_1 \quad \text{and} \quad \mathrm{spt}(g \ast f) \subseteq \mathrm{spt}(g) \cdot \mathrm{spt}(f) \subseteq U_2
\]
show that $f \ast g \neq g \ast f$ as they are supported on disjoint sets.
\end{proof}

\brmk
If $f \in C_c^\infty(\G)$ and $g \in C_c(\G)$, then the convolution belongs to $C_c^\infty(\G)$ and
\[
X_j(f \ast g) = (X_j f) \ast g.
\]
\ermk

\bd
Let $\mu$ be a Radon measure on $\G$ and let $f \in C_c(\G)$. The convolution between these two objects is defined as follows:
\[
f \ast \mu(x) := \int_\G f(x \cdot y^{-1}) \, \dr \mu(y).
\]
In a similar fashion, one can define
\[
\mu \ast f(x) := \int_\G f(y^{-1} \cdot x) \, \dr \mu(y).
\]
\ed

\brmk
If $\mu := \delta_a$ is the Dirac delta, then it is easy to verify that
\[
f \ast \delta_a(x) = \int_\G f(x \cdot y^{-1}) \delta_a(y) \, \dr y = f(x\cdot a^{-1}) = R_{a^{-1}} f(x),
\]
and, similarly,
\[
\delta_a \ast f(x) = \int_\G f(y^{-1} \cdot x) \delta_a(y) \, \dr y = f( a^{-1} \cdot x) = L_a f(x).
\]
This means that for finite Radon measures $\mu$ and $\nu$ we can define the convolution as
\[
\int_\G f(x) \, \dr(\mu\ast\nu)(x) = \int_\G \int_\G f(x\cdot y) \, \dr \mu(x) \dr \nu(y)
\]
in such a way that the identity $\delta_a \ast \delta_b = \delta_{a \cdot b}$ holds.
\ermk

\bd
Let $\Phi \in \D'(\G)$ be a distribution. Then we can define the left and right translations by setting
\[\begin{aligned}
& \langle L_a \Phi,\, f \rangle := \langle \Phi, L_{a^{-1}} f \rangle,
\\[1em] & \langle R_a \Phi,\, f \rangle := \langle \Phi, R_{a^{-1}} f \rangle.
\end{aligned}\]
\ed

The convolution between a distribution $\Phi \in \cD'(\G)$ and a function $f \in C_c(\G)$ is defined in the usual way by setting
\[
\langle \Phi \ast f,\, g \rangle := \langle \Phi, g \ast \check{f} \rangle,
\]
where $\check{f}(x) := f(x^{-1})$. Notice that the order of $g \ast \check{f}$ is important since $\G$ might not be commutative. It is easy to verify that $\Phi \ast f$ is actually a function defined by
\[
\Phi \ast f(x) = \langle \Phi, L_x \check{f} \rangle
\]
and, similarly,
\[
f \ast \Phi(x) = \langle \Phi, R_{x^{-1}} \check{f} \rangle.
\]

\brmk We can define the convolution between two distributions in a similar fashion, but there is something to take into account now. Since $\Phi \ast f$ is a function, to define
\[
\langle \Psi, \Phi \ast f \rangle,
\]
we need $\Phi \ast f$ to be {\bf compactly supported}. Therefore,
\[
\langle \Psi \ast \Phi, f \rangle := \langle \Psi, \check{\Phi} \ast f \rangle
\]
is well-defined provided that, for example, $\Psi \in \D'(\G)$, $f \in C_c(\G)$ and $\Phi$ is a compactly supported distribution.
\ermk

\subsection{Schwartz spaces}

Defining $\cS(\G)$ for a general Lie group $\G$ is hard, but if we require nilpotent (connected and simply connected), then we can identify it with $\cg$ via \eqref{bchformula} and write
\[
\cS(\G) := \cS(\cg),
\]
where $\cg \cong \R^N$ for some $N \in \N$. That said, there are two problems we need to take care of: \mbox{}
\begin{enumerate}[label=(\roman*)]
\item The derivatives are rapidly decreasing, but what can we say about $X_j f$?
\item The estimate asserts that 
\[
|\partial^\alpha f(x)| = o(|x|^{-M}),
\]
however, the norm $|\cdot|$ is the Euclidean one. What is the relation with the homogeneous norm on $\G$, if $\G$ is homogeneous?
\end{enumerate}
The problem ({\romannumeral 1}) is easy to solve since we can always write
\[
X_j f(x) = \partial_{x_j} f(x) + \sum a_{jk}(x) \partial_{x_k} f(x),
\]
where $a_{jk}$ is a {\bf polynomial function} that vanishes at the origin for each $k$. Since $a_{jk}$ is a polynomial, we easily infer that
\[
|\partial^\alpha f(x)| = o(|x|^{-M}) \implies |X_{j_1} \cdots X_{j_k} f(x)| = o(|x|^{-M}).
\]
The problem ({\romannumeral 2}), on the other hand, is harder to deal with and outside the scopes of this course.



\section{Left-invariant differential operators}

The goal of this section is to characterize left-invariant differential operators on Lie groups, but we first need to recollect some definitions on manifolds.

\bd \index{differential operator!on a manifold}
Let $\cM$ be a manifold. A {\em differential operator} on $\cM$ is a map
\[
D : C_c^\infty(\cM) \to C^\infty(\cM)
\]
such that we can write
\[
Df ( \varphi(t) ) = \sum_{|\alpha| \leq m} a_\alpha(t) \partial_t^\alpha (f \circ \varphi)(t)
\]
for all $f \in C_c^\infty(\cM)$ and all local coordinates $\varphi$, where $a_\alpha$ are $C^\infty(\cM)$ coefficients.
\ed

\bd
Let $D$ be a differential operator on a manifold $\cM$ and fix $x_0 = \varphi(t_0)$ for some local coordinate $\varphi$. Then the {\em order of $D$ at $x_0$} is
\[
\ord(D,x_0) := \inf \left\{ k \in \N \: : \: \text{$a_\alpha \equiv 0$ for all $\alpha \: : \: |\alpha|=k+1$} \right\}.
\]
\ed

\bexe
Show that the order $\ord(D,x_0)$ does not depend on the choice of the local coordinate around $x_0$.
\eexe

\bthm[Peetre] \index{Peetre theorem}\label{petreetheorem}
Let $\cM$ be a manifold. A linear operator $D : C_c^\infty(\cM) \to C^\infty(\cM)$ is a differential operator if and only if
\[
\spt(Df) \subseteq \spt(f) \quad \text{for all $f \in C_c^\infty(\cM)$}.
\]
\ethm

The proof of this result is far beyond the purpose of this course. For more information, see the original articles \cite{pe1}, \cite{pe2} and the related works.

\bd \index{differential operator!left-invariant}
Let $\G$ be a Lie group and $D$ a differential operator. We say that $D$ is {\em left-invariant}, and we write $D \in \sD(\G)$, if
\[
D(L_x f) = L_x (Df).
\]
\ed

\bthm\label{thm.ppp.2}
Let $\cg$ be a Lie algebra and let $p(t)$ be a polynomial on $\cg$. Define
\[
D_p f(x) := p(\partial_t) \, \Big|_{t=0} \, f(x \exp_\G(t))
\]
for $x \in \G$. Then $D_p$ is a left-invariant differential operator and, conversely, given $D \in \sD(\G)$ there exists a unique polynomial $p$ such that $D = D_p$.
\ethm

\begin{proof}
First, we notice that $D_p$ is linear so by \hyperref[petreetheorem]{Theorem \ref{petreetheorem}} to prove that $D_p$ is a differential operator it suffices to show that
\[
x \notin \spt(f) \implies x \notin \spt(D_p f).
\]
This is trivial because there exists $\delta > 0$ such that for all $|t| < \delta$ one has $f(x \exp_\G(t)) = 0$ and, given that $p(\partial_t) \, \Big|_{t=0} \, f(x \exp_\G(t))$ depends only on small values of $|t|$, we conclude that
\[
D_p f(x') = 0 \quad \text{for all $x'$ in a neighbourhood of $x$}.
\]
To check that $D_p$ is left invariant we simply notice that
\[ \begin{aligned}
D_p(L_y f)(x) & = p(\partial_t) \, \Big|_{t=0} \left( L_{(y^{-1}\cdot x)^{-1}} f \circ \exp_\G(t) \right)(0)
\\[1em] & = L_y (D_p f)(x),
\end{aligned} \]
and this concludes the proof of the first part. Now let $D \in \sD(\G)$ and take the local chart around the origin $\varphi = \exp_\G$ so that the identity
\[
Df (\varphi(t)) = \sum_{|\alpha| \leq k} a_\alpha(t) \partial^\alpha(f \circ \varphi)(t)
\]
implies, by setting $t = 0$, that
\[
Df (e) = D_p f (e) \quad \text{where $p(t) = \sum_{|\alpha| \leq k} a_\alpha(0) t^\alpha$}.
\]
This is enough to conclude because both $D$ and $D_p$ are left-invariant and hence
\[
Df (x) = D_p f (x)\quad \text{for all $x \in \G$.}
\]
\end{proof}

\brmk
This shows that the order of $D$ does not depend on the point $x \in \G$ and is equal to the degree of the polynomial $p$ such that $D = D_p$.
\ermk

There is a different characterisation of these differential operators that requires the introduction of left-invariant vector fields. Let $(X_1,\,\dots,\,X_n)$ be a basis of $\cg$ and set
\[
X^\alpha := X_1^{\alpha_1} \cdots X_n^{\alpha_n}.
\]

\bthm[Poincaré-Birkhoff-Witt] \label{theorempbw}
Let $D \in \sD(\G)$. Then $D$ admits a unique decomposition of the form
\[
D = \sum_{|\alpha| \leq k} c_\alpha X^\alpha =: \widetilde{D}_p,
\]
where $p(t) = \sum_{|\alpha| \leq k} c_\alpha t^\alpha$.
\ethm

\begin{proof}The reader may consult \cite{aas1} or the references therein. \end{proof}

\brmk
Let $p = \sum_{|\alpha| \leq k} c_\alpha t^\alpha$. The operator $\widetilde{D}_p$ introduced above can also be written as follows:
\[
\widetilde{D}_p = p(\partial_{t'}) \, \Big|_{t' = 0} \, f(x \exp_\G(t_1' X_1)  \cdots \exp_\G( t_n' X_n)).
\]
Therefore, using the chart $\psi(t') := \exp_\G(t_1' X_1) \cdots \exp_\G(t_n' X_n)$, we have
\[
\widetilde{D}_p = p(\partial_{t'}) \, \Big|_{t' = 0} \, (L_{x^{-1}}f) \circ \psi(t').
\]
Similarly, if we  consider the chart $\varphi(t) = \exp_\G(t_1 X_1 + \cdots + t_n X_n)$, we can write
\[
D_p f(x) = p(\partial_t) \, \Big|_{t = 0} \, (L_{x^{-1}}f) \circ \varphi(t).
\]
The change of coordinates is $t = \varphi^{-1} \circ \psi(t') = t' + \mathcal{O}(|t'|^2)$, and thus we can write
\[
\partial_{t'}^\alpha (g \circ u)(0) = \partial_{t}^\alpha (g \circ u)(0) + \cdots,
\]
which means that $\widetilde{D}_p$ and $D_p$ coincide up to lower order terms.
\ermk

\begin{proof}[Proof of Theorem \ref{theorempbw}]
By induction on the degree of $p$. The base case is trivial, so we can assume that it holds for all $j < k$. By \hyperref[thm.ppp.2]{Theorem \ref{thm.ppp.2}} we can write
\[
D = D_p
\]
for some polynomial of order $k$. Then
\[
D_p - \widetilde{D}_p = D_{< k},
\]
where $D_{< k}$ is a differential operator of order strictly less than $k$. By inductive hypothesis, there exists a polynomial $q$ of degree $<k$ such that $D_{<k}' = \widetilde{D}_q$. It follows that
\[
D = D_p = \widetilde{D}_p + \widetilde{D}_q = \widetilde{D}_{p+q},
\]
and this concludes because $\deg(p+q) = k$.
\end{proof}

\bd \index{universal enveloping algebra}
Let $\cg$ be a Lie algebra and $T(\cg)$ be the tensor algebra. The {\em universal enveloping algebra} $U(\cg)$ is defined by the quotient
\[
\faktor{T(\cg)}{\langle x \otimes y - y \otimes x - [x,\,y]\rangle}.
\]
\ed

\bthm
Let $\cg$ be a Lie algebra. Then $U(\cg)$ is isomorphic to $\sD(\G)$, where $\G$ is the unique connected and simply connected Lie group.
\ethm

\begin{proof} See \cite{aas1} and the references therein. \end{proof}

\section{Solvability of left-invariant differential operators}

We first recall the definition of locally solvable operator (at some point) in a manifold and then specialise it to Lie groups.

\bd \index{locally solvable!on a manifold}
Let $D$ be a differential operator on a manifold $\cM$. We say that $D$ is {\em locally solvable} at $x \in \cM$ if for all $k \in \N$ there exists $V_k \ni x$ neighbourhood such that
\[
\forall \psi \in \cD'(\cM), \, \exists \, u \in D_k'(V_k) \: : \: Du = \psi \quad \text{on $V_k$}.
\]
\ed

We now recall a critical result that holds when $\cM = \R^n$ is the Euclidean space, which asserts that fundamental solutions exist for certain operators.

\bthm[Malgrange–Ehrenpreis]
Let $L$ be a linear differential operator with constant coefficients. Then there exists $\Phi \in \D^\prime(\R^n)$ satisfying
\[
L\psi = \delta_0.
\]
\ethm

\bcor
Linear differential operators with constant coefficients are locally solvable.
\ecor

Notice that linear differential operators with constant coefficients can be replaced with $\sD(\R^n)$, which means that one might expect the following result to be true:

{\color{red}\bpr
Every $D \in \sD(\G)$ is locally solvable.
\epr}

Unfortunately, this statement is {\bf false} for a general Lie group $\G$. Consider, for example, the Haus Levy's operator in $\R^3$ with coordinates $(x,y,t)$ given by
\[
L =\left( \frac{1}{2} \partial_x + y \partial_t \right) + \imath \left( \frac{1}{2} \partial_y - x \partial_t \right) =: X + \imath Y
\]
Notice that $[X,\,Y] = - \partial_t =: T$ and $\{ X,\, Y,\, T \}$ span $\R^3$ at all points $(x,y,t)$. Then $L \in \sD(H)$, where $H$ is the Heisenberg group, but we know that $L$ is not locally solvable.

\bd
Let $\G$ be a Lie group and $D$ a differential operator. A fundamental solution is $K \in \D'(\G)$ such that
\[
DK = \delta_e,
\]
where $e$ is the identity element of $\G$.
\ed

\bpr
If there is a fundamental solution $K$ for $D \in \sD(\G)$, then it is locally solvable.
\epr

\begin{proof} Let $\varphi \in \D'(\G)$ with compact support. Let $u = \varphi \ast K$ (the order here is important) and notice that
\[ \begin{aligned}
Du(x) & = D(\varphi \ast K)(x)
\\[1em] & = \int \varphi(y) D_x K(y^{-1}\cdot x) \, \mathrm{d} y
\\[1em] & = \int \varphi(y) D_x (L_y K)(x) \, \mathrm{d} y
\\[1em] & = \int \varphi(y) L_y(D_x K)(x) \, \mathrm{d} y = \varphi \ast D K(x).
\end{aligned} \] \end{proof}

\bpr
A differential operator $D \in \sD(\G)$ is locally solvable if and only if there exists a local fundamental solution, namely there exists $V_0 \ni e$ neighbourhood and $K \in D_0'(V_0)$ such that
\[
DK = \delta_e \quad\text{on $V_0$}.
\]
\epr

\bpr
If $D \in \sD(\G)$ is hypoelliptic, then $\transp{D}\in\sD(\G)$ and it admits a local fundamental solution.
\epr

Now let $\G$ be a homogeneous Lie group with a family of dilations $\{\delta_s\}_{s > 0}$ and write $\cg = \sum_\lambda \cg_\lambda$ in such a way that
\[
\delta_s \, \big|_{\cg_\lambda} \equiv s^\lambda \mathrm{Id}.
\]
We would like to find sufficient conditions on a left-invariant differential operator $D$ for a fundamental solution to exist.

\bd\index{distribution!homogeneous}
A distribution $\phi \in \cD'(\G)$ is {\em homogeneous} of order $\alpha$ if
\[
\phi \circ \delta_s = s^\alpha \phi,
\]
where the left-hand side is defined by
\[
\langle \phi \circ \delta_s, f \rangle = \langle \phi, s^{-Q} f \circ \delta_{s^{-1}} \rangle.
\]
\ed

\brmk \mbox{}
\begin{enumerate}[label=(\roman*)]
\item The Dirac delta at the origin $0 \in \G$ (using the BCH coordinates), $\delta_0$, is homogeneous of order $- Q$.
\item If $\phi$ is homogeneous of order $\alpha$ and $X \in \cg_\lambda$, then $X \phi$ has homogeneity order $\alpha - \lambda$.
\end{enumerate}
\ermk

\bd
Let $Tf := f \ast K$ be the convolution operator. We say that $T$ is homogeneous of order $\beta$ if
\[
T(f \circ \delta_s) = s^\beta (Tf) \circ \delta_s
\]
so that, if $T = \mathrm{Id}$, then the order is zero.
\ed

\brmk \mbox{}
\begin{enumerate}[label=(\roman*)]
\item If $T = X \in \cg_\lambda$, then the homogeneity order of $T$ is equal to $\lambda$.
\item $T$ is homogeneous of order $\beta$ if and only if $K$ is homogeneous of order $-Q-\beta$.
\item If $T$ is homogeneous of order $\beta$ and $U$ is homogeneous of order $\alpha$, the composition $TU$ has homogeneity order $\alpha + \beta$.
\end{enumerate}
\ermk

\bd\index{operator!homogeneous}
A differential operator $D$ is homogeneous of order $\beta$ if
\[
D(f \circ \delta_s) = s^\beta (Df) \circ \delta_s.
\]
\ed

\bex
Let $D$ be a differential operator homogeneous of order $\beta$ and let $K$ be a homogeneous fundamental solution for $D$. Since
\[
DK = \delta_0,
\]
we can consider the convolution operator $Tf = f \ast K$ and find that
\[
D(Tf) = f \implies DT = \mathrm{Id}.
\]
It is easy to verify that, at this point, the unique possibility is that $K$ is homogeneous of order $-Q+\beta$.
\eex

\bex
If $\G = \R^n$ and $D = \Delta$, then a fundamental solution must have homogeneity order $- n + 2$ and hence it must be
\[
\frac{c}{|x|^{n-2}}
\]
for $n \geq 3$. If $n = 2$, then there is no homogeneity and it can be proved that a fundamental solution for the Laplace operator is
\[
c \log |x|.
\]
\eex

\bex
Let $D = \partial_t - \Delta_x$ be the heat kernel on $\R \times \R^n$ and consider the dilations 
\[
\delta_s(t,x) = (s^2t,sx).
\]
Then the operator has homogeneity order equal to two and
\[
K(t,x) := \begin{cases} \frac{1}{(2 \pi t)^{\frac{n}{2}}} \e^{- \frac{|x|^2}{4t}} & \text{if $t > 0$}, \\0 & \text{otherwise}, \end{cases}
\]
is a fundamental solution. Since $\delta_s(t,x) = (s^2 t,sx)$ we have that
\[
K(s^2 t,sx) = s^{-n} K(t,x) \implies \text{the homogeneity order of $K$ is $n$.}
\]
\eex

\bthm \label{thm.fshl2}
Let $Q$ be the homogeneous dimension of $\G$ and suppose that \mbox{}
\begin{enumerate}[label=(\roman*)]
\item $D$ is a left-invariant operator of homogeneity order $\beta \leq Q$;
\item $D$ and $\transp{D}$ are hypoelliptic.
\end{enumerate}
Then $D$ has a fundamental solution with homogeneity order $-Q + \beta$.
\ethm

\begin{proof}
The operator $D$ has a fundamental solution $K$ defined on some neighborhood $V$ of the origin $0 \in \G$. Since $D$ is hypoelliptic, from
\[
DK = 0 \quad \text{on $V \setminus\{0\}$}
\]
we infer that $K$ must be smooth away from the origin. Let $\eta \in \cD(V)$ be a cutoff function taking values in $[0,1]$ and set
\[
K_1 := \eta K.
\]
Then $K_1$ is defined on all $\G$, smooth away from $0$, and satisfies the equation
\[
DK_1 = \delta_0 + \Phi,
\]
where $\Phi$ is compactly supported in a set that does not contain the origin. Notice that, once again, the hypoellipticity of $D$ implies $\Phi \in \cD(\G)$.

\proofstep{Step 1} Fix a homogeneous norm $|\cdot|$ on $\G$. Then
\[
\spt( \Phi ) \subset \{x \: : \: a < |x| < b\}
\]
and, if we define $K_t := t^{\beta-Q} K_1 \circ \delta_{t^{-1}}$ and $\Phi_t := t^{-Q} \Phi \circ \delta_{t^{-1}}$, we also find that
\[
D K_t = t^{-Q} (D K_1) \circ \delta_{t^{-1}} = \delta_0 + \Phi_t
\]
because $\delta_0$ has degree of homogeneity equal to $-Q$ (as verified above). Since
\[
\spt(\Phi_t) = \delta_t \left( \spt (\Phi) \right) \subset \{ x \: : \: ta < |x| < tb \}
\]
we easily infer that $K_t$ is another local fundamental solution of $D$. Moreover, given that
\[
\lim_{t \to \infty} \Phi_t = 0
\]
in the sense of distributions, it is sufficient to prove that $\lim_{t \to \infty}K_t = K$ exists as a distribution to conclude that $K$ is a fundamental solution for $D$. If such $K$ exists, then
\[
s^{Q-\beta} K \circ \delta_s= \lim_{t \to \infty} s^{Q-\beta} K_t \circ \delta_s = \lim_{t \to \infty} K_{t/s} = K
\]
shows that it is homogeneous of degree $-Q+\beta$ as claimed.

\proofstep{Step 2} Let $t > 1$. We would like to write $K_t$ as
\[
K_t = K_1 + \int_1^t \frac{\dr K_s}{\dr s} \, \dr s,
\]
but, a priori, the integrand might not be a well-defined distribution. To prove it, we start with noticing that for $\varphi \in \cD(\G)$ we have the identity
\[ \begin{aligned}
\lim_{s \to 1} \frac{1}{s-1} \langle K_s - K_1,\varphi\rangle & = \lim_{s \to 1} \frac{1}{s-1} \langle K_1, s^\beta \varphi \circ \delta_s - \varphi \rangle
\\[1em] & = \left\langle K_1, \frac{\dr}{\dr s} \, \bigg|_{s=1} (s^\beta \varphi \circ \delta_s) \right\rangle.
\end{aligned} \]
Let $(x_1,\dots,x_n)$ be coordinates of $\G$ such that $\delta_s x = (s^{\lambda_1} x_1,\dots,s^{\lambda_n}x_n)$. Then
\[
\frac{\dr}{\dr s} \, \bigg|_{s=1} \varphi(\delta_s x) = \sum_{j=1}^n \left( \lambda_j \partial_{x_j} \varphi(x) \right) x_j = E \varphi(x),
\]
where $E$ is called {\em modified Euler operator}. It follows that
\[
\frac{\dr }{\dr s} \, \bigg|_{s=1} \langle K_s, \varphi \rangle = \langle K_1, \beta \varphi + E \varphi \rangle,
\]
which immediately gives the identity
\[
K_1' = \beta K_1 + \transp{E}K_1 = (\beta-Q)K_1 - E K_1.
\]
To prove that $K_1'$ is smooth (i.e., it belongs to $\cD(\G)$) simply notice that
\[
D K_1' = \frac{\dr}{\dr s} \, \bigg|_{s=1} (D K_s) = - Q \Phi - E \Phi,
\]
which is a smooth function. Since $D$ is hypoelliptic, $K_1'$ is smooth on $\G$ and it has compact support.

\proofstep{Step 3} For any $s > 1$ we have
\[ \begin{aligned}
 \frac{\dr K_s}{\dr s} & = s^{-1}  \frac{\dr}{\dr u} \, \bigg|_{u=1} K_{su}
 \\[.6em] & = s^{-1} s^{\beta-Q}  \frac{\dr}{\dr u} \, \bigg|_{u=1} K_{u} \circ \delta_{s^{-1}}
 \\[.6em] & = s^{\beta-Q-1} K_1' \circ \delta_{s^{-1}}.
\end{aligned} \]
If $\varphi \in \cD(\G)$, then
\[
\int_1^t \langle s^{\beta-Q-1} K_1' \circ  \delta_{s^{-1}}, \varphi \rangle = \int_1^t s^{\beta-Q-1} \int_\G K_1'(\delta_{s^{-1}} x) \varphi(x) \, \dr x \dr s.
\]
Since $\beta < Q$, we have
\[
\int_1^t s^{\beta-Q-1} \int_\G |K_1'(\delta_{s^{-1}} x)| |\varphi(x)| \, \dr x \dr s \leq C \|\varphi\|_{L^1(\G)},
\]
showing that the integral is actually well-defined in the sense of distributions as $t \to \infty$.

\proofstep{Step 4} The smoothness of $K$ away from zero follows from the hypoellipticity of $D$, or alternatively, from the fact that for all $x \neq 0$ we have
\[
K(x) = K_1(x) + \int_1^\infty s^{\beta-Q-1} K_1'(\delta_{s^{-1}} x) \, \dr s.
\]

\proofstep{Step 5} For the uniqueness, let $K$ and $H$ be two $(-Q+\beta)$-homogeneous fundamental solutions for $D$. Then
\[
D(K - H) = 0
\]
gives, by hypoellipticity, that $K - H \in C^\infty(\G)$, which is impossible because it is homogeneous of order $- Q + \beta$ which is always negative.
\end{proof}

\brmk
If $D = X + \imath Y$ is the Levi's operator on the Heisenberg group, it is easy to see that $D$ is not hypoelliptic and hence it is not a counterexample to \hyperref[thm.fshl2]{Theorem \ref{thm.fshl2}}.
\ermk

\bex
If $\G$ is stratified and $\cg = \cg_1 \oplus \cdots \oplus \cg_k$ with $\{X_1,\dots,X_k\}$ basis, then
\[
D = \sum_{j=1}^k X_j^2
\]
is homogeneous of order two. Therefore, for all $Q \geq 3$ we can apply the previous theorem to infer the existence of a fundamental solution for $D$.
\eex

\brmk
In general $K(x) = |x|^{-Q+2}$, but $|\cdot|$ is the homogeneous norm on $\G$ which is in most cases a rather mysterious object.
\ermk

\bex
In the Heisenberg group with $X_j = \partial_{x_j} - \frac{y_j}{2} \partial_t$ and $Y_j = \partial_{y_j} - \frac{x_j}{2} \partial_t$ we have the fundamental solution
\[
\Phi_k(\mathbf{z}, \, t) = \frac{2^k \Gamma(k/2)^2}{\pi^{k+1}} \left( |\mathbf{z}|^4 + t^2 \right)^{-\frac{k}{2}},
\]
which is, accordingly to a previous result, smooth away from the origin and homogeneous of degree $2 - Q$.
\eex

\appendix
\part{Appendix}
\chapter{Topics in Differential Geometry} \thispagestyle{empty}

The goal of this chapter is to recollect some useful and well-known facts in differential geometry which are needed throughout the course. The reader who is totally unfamiliar with the topics discussed here may refer to \cite{manifolds}. 

\section{Introduction}

Recall that a {\em smooth manifold}\index{smooth manifold} is a topological manifold which is equipped with an equivalence class of atlases whose transition maps are all smooth. Namely, a manifold is a couple
\[
\left(M, \, \{ \varphi_i : U_i \to V_i \subset \R^q \}_{i \in I} \right),
\]
where $U_i$ and $V_i$ are open sets and the transition maps
\[
\varphi_{i, j} : \varphi_i(U_i \cap U_j) \to \varphi_j(U_i \cap U_j),
\]
defined by setting $\varphi_{i, j} = \varphi_j^{-1} \circ \varphi_i$, are smooth.

\bd
We say that a map between smooth manifolds $f : M \to N$ is smooth if it is so along some charts, that is,
\[
\psi_j \circ f \circ \varphi_i^{-1} : V_i \subset \R^q \to Z_i \subset \R^p \in C^\infty.
\]
\ed

The tangent space at $p$ to a smooth manifold $M$ is easy to define when $M$ is embedded in $\R^q$, but it requires an abstract definition if $M$ is not a submanifold of the Euclidean space.

\bd
Let $M$ be a smooth manifold and fix $p \in M$. A {\em derivation}\index{derivation} $v$ at $p$ is an operator that assigns a number $v(f)$ to any smooth real-valued function $f$ defined in a neighbourhood of $p$, that satisfies the following assumptions: \mbox{}
\begin{enumerate}[label=\textbf{(\alph*)}]
\item If there exists $U \ni p$ such that $f \, \big|_U \equiv g \, \big|_U$, then $v(f) = v(g)$.
\item The operator $v$ is linear.
\item The operator $v$ satisfies the Leibniz rule, that is,
\[
v(fg) = f(p) v(g) + g(p) v(f).
\]
\end{enumerate}
The set of all derivations at $p$ is called {\em tangent space at $p$ to $M$} and it will always be denoted with the symbol $\Tan_p M$.
\ed

\brmk
The tangent space $\Tan_p M$ is a vector space.
\ermk

\subsubsection*{Immersion, embeddings and submanifolds}

Let $f : M \to N$ be a smooth map between two smooth manifolds. We say that $f$ is a {\em immersion}\index{immersion} at $p \in M$ if the differential
\[
\mathrm{d}f_p : \Tan_p M \longrightarrow \Tan_{f(p)} N
\]
is injective as a linear map between vector spaces. It is worth remarking that an immersion is locally injective, but it might fail to be so globally: for example, the $8$-knot in $\R^2$ has a self-intersection at the origin, but it is an immersion. The natural "upgrade" of this notion is that of {\em embedding}\index{embedding}.

\bd We say that a smooth map $f : M \to N$ is an embedding if it is an immersion and a homeomorphism onto its image. \ed

The homeomorphism condition implies that an embedding is globally injective, but the vice versa is not always\footnote{A proper injective immersion is an embedding, so if $M$ is compact globally injective together with immersion is enough to get an embedding. } true although the counterexample is a little bit more subtle than the previous one. In any case, for us embeddings will take on a fundamental role thanks to the following property:

\bl
If $f : M \to N$ is an embedding, then $f(M)$ is a smooth submanifold of $N$.
\el

\section{Bundles}

{\color{red}Work in progress...}

\section{Vector fields and flows}
\label{sec:vff}

In this section, we introduce the notion of a {\em smooth vector field} and we investigate the properties of the associated flow.

\bl
Let $\Omega$ be an open connected subset of $\R^n$ and let $X : C^\infty(\Omega) \to C^\infty(\Omega)$ be a linear operator. Then the following assertions are equivalent: \mbox{}
\begin{enumerate}[label=\textbf{(\alph*)}]
\item The operator $X$ is a smooth vector field satisfying
\begin{equation*} Xf(x) = \sum_{j = 1}^n a_j(x) \partial_j, \end{equation*}
where $a_j \in C^\infty(\Omega)$ and $\partial_j := \partial_{x_j}$.
\item The operator $X$ satisfies the Leibniz rule.
\end{enumerate}
\el

\begin{proof}
Suppose that $X$ satisfies the Leibniz rule and let $f \in C^\infty(\Omega)$. For a fixed point $\bar{x} \in \Omega$ we can exploit the smoothness of $f$ to write
\[ \begin{aligned}
f(x) & = f(\bar{x}) + \frac{\mathrm{d}}{\mathrm{d}t} \int_0^1 f(tx + (1-t)\bar{x}) \, \mathrm{d}t =
\\[1em] & = f(\bar{x}) + \sum_{j = 1}^n (x_j - \bar{x}_j) \int_0^1 \partial_j f(tx + (1-t)\bar{x}) \, \mathrm{d}t. 
\end{aligned} \]
Observe\footnote{Use the Leibniz rule with $f = g$ constant.} that $X$ vanishes on constant functions, i.e. $X(c) = 0$ for any $c \in \R$ and thus
\[
X(f)(x) = \underbracket{X(f(\bar{x}))}_{= 0 } + \sum_{j = 1}^n X \left( (x_j - \bar{x}_j) h_j(x) \right).
\]
We evaluate the expression at $x = \bar{x}$ and conclude that
\[
X(f)(\bar{x}) =\sum_{j = 1}^n X (x_j)(\bar{x}) h_j(\bar{x}).
\]
Set $a_j(\bar{x}) := X(x_j)(\bar{x})$ and observe that $h_j(\bar{x})$ is nothing but the partial derivative $\partial_j$. This concludes the proof of the lemma.
\end{proof}

To define the flow, we first need the notion of {\em integral curve}\index{integral curves} of a vector field $X$. The idea is to look for solutions $\gamma : (-\delta, \, \delta) \to \Omega$ of the following Cauchy problem:
\[ \begin{cases}
\gamma^\prime(t) = X(\gamma(t)),
\\[.8em] \gamma(0) = x_0.
\end{cases}\]
The local existence theorem tells us that we can find a unique maximal solution, say $\gamma_{x_0}$, that is defined in the (maximal) interval $I_{x_0}$. We say that $\gamma_{x_0}$ is the integral curve of $X$ originating at $x_0$.

\brmk
If $K \subset \Omega$ is compact, then there exists $\epsilon_K > 0$ such that
\[
\text{$(- \epsilon_K, \, \epsilon_K) \subset I_x$ for all $x \in K$}.
\]
In particular, integral curves are always defined up to a time $\delta > 0$ which is uniform with respect to the choice of the originating point.
\ermk

The {\em flow}\index{vector field flow} of a vector field $X$ is defined by setting
\[
\Phi_X(x, \, t) := \gamma_x(t) \quad \text{(or $\varphi_{X, \, t}(x)$)},
\]
where $\gamma_x$ is the maximal solution of the Cauchy problem above. It is easy to verify that $\Phi_X$ is smooth ($C^\infty$) with respect to the couple $(x, \, t)$.

\brmk
If $\{X_y\}_{y \in \Theta}$ is a family of smooth vector fields, then the map
\begin{equation*}(x, \, y, \, t) \longmapsto \Phi_{X_y}(x, \, t) \end{equation*}
is smooth with respect to the triplet $(x, \, y,\, t)$.
\ermk

Now, let us denote by $D_X$ the domain of $\Phi_X$ (which happens to be vertically connected) and define $\varphi_{X, \, t}(\cdot)$ as the flow at a fixed time $t$, that is,
\[
\varphi_{X, \, t}(x) := \gamma_x(t).
\]
It is easy to verify that $\varphi_{X, \, 0}$ is the identity map and also that
\[
\varphi_{X, \, t} \circ \varphi_{X, \, s} = \varphi_{X, \, s + t}
\]
by the local uniqueness of the solution of the Cauchy problem. This immediately translates to the analogous property for the flow:
\begin{equation} \label{eq.f.1}
\Phi_X(x, \, t+s) = \Phi(\Phi(x, \, t), \, s).
\end{equation}

\bpr
Let $A \subseteq \Omega \times \R$ be an open neighbourhood of $\Omega \times \{0\}$ that is vertically connected. Let $\Phi \in C^\infty(A,\, \Omega)$ satisfy \eqref{eq.f.1} and $\Phi(x, \, 0) = x$. Then for any $f \in C^\infty(\Omega)$ it turns out that
\begin{equation*}Xf(x) = \frac{\mathrm{d}}{\mathrm{d}t} \, \Big|_{t = 0} f(\Phi(x, \, t)) \end{equation*}
is a vector field and $\Phi = \Phi_X \, \Big|_A$.
\epr

\section{Exponential map}

Fix a positive time $t$. We can consider\footnote{Whenever it makes sense, of course, but in this section we will purposely ignore this issue.} the operator
\begin{equation*} f \longmapsto f \circ \varphi_{X, \, t}, \end{equation*}
which sends $C^\infty(\Omega)$ into $C^\infty(\Omega_t)$, where $\Omega_t$ is the set of all $x$ such that the flow survives until (at least) time $t$. We denote it by
\begin{equation*} \mathrm{exp}(tX)f  := f \circ \varphi_{X, \, t}, \end{equation*}
and we call it exponential\index{exponential map}. It can easily be proved via the Cauchy problem above that
\begin{equation*} f \circ \varphi_{X, \, t} = f \circ \varphi_{tX, \, 1} \end{equation*}
since, roughly speaking, we can interpret $X$ as the speed at which we are running across the curve.

\begin{lemma}[Properties of the exponential] \mbox{}
\begin{enumerate}[label={\color{magenta}(\arabic*)}]
\item The exponential $\mathrm{exp}(0 X)$ coincides with the identity map.
\item The composition is given by $\mathrm{exp}((s+t)X) = \mathrm{exp}(sX)\mathrm{exp}(tX)$.
\item The inverse is given by $\mathrm{exp}(-tX) = \left[ \mathrm{exp}(tX) \right]^{-1}$.
\item The derivative is given by
\begin{equation*} \frac{\mathrm{d}}{\mathrm{d}t} \mathrm{exp}(tX) = X \mathrm{exp}(tX) =  \mathrm{exp}(tX)X, \end{equation*}
and therefore for all $k \in \N$ and $f \in C^\infty$ we have the following Taylor formula:
\begin{equation}\label{taylor.e}  \mathrm{exp}(tX) f(x) = \sum_{j = 0}^k \frac{t^j}{j!} X^j f(x) + \mathcal{O}(t^{k+1}). \end{equation}
If $x \in K$, then this identity makes sense for all time in $(-\epsilon_K, \, \epsilon_K)$ and the big-$\mathcal{O}$ is uniform.
\item If $u : \Omega \to \Omega^\prime$ is a diffeomorphism, then
\begin{equation*} u \circ \Phi_X(\cdot, \, t) = \Psi(\cdot, \, t) \end{equation*}
is the flow of the pushforward vector field $X^\prime = u_\ast X$.
\end{enumerate}
\end{lemma}

We are now interested in composition of exponential maps generated by different vector fields that may also not commute. In particular, we will compare the following three terms:
\begin{equation*} \mathrm{e}^{tX} \mathrm{e}^{sY}, \quad \mathrm{e}^{sY}\mathrm{e}^{tX} \quad \text{and} \quad \mathrm{e}^{tX + sY} . \end{equation*}

\begin{example}In $\R^2$ consider the vector fields $X = \partial_x$ and $Y = \partial_y$. Then
\begin{equation*}\varphi_{X, \, t}(x, \, y) = (x+t, \, y)  \quad \text{and} \quad \varphi_{Y, \, s}(x, \, y) = (x, \, y + s), \end{equation*}
so it is immediate to verify that these two flows commute, that is,
\begin{equation*} \varphi_{X, \, t} \circ \varphi_{Y, \, s} \equiv  \varphi_{Y, \, s} \circ \varphi_{X, \, t}. \end{equation*}
Still in $\R^2$, we can consider $X$ as above and $Y = x \partial_y$. In \hyperref[fig.g.1]{Figure \ref{fig.g.1}} we show that
\begin{equation*} \varphi_{X, \, t} \circ \varphi_{Y, \, s}(x, \, y) \neq  \varphi_{Y, \, s} \circ \varphi_{X, \, t} (x, \, y) \end{equation*}
for all points $(x, \, y) \in \R^2$, which means that the flows do not commute and something bad happens (as we will see later on). 
\end{example}

%%FIG. IPAD

Now apply the Taylor expansion (up to the order two) to both $\mathrm{e}^{tX}$ and $\mathrm{e}^{sY}$. Assuming that $s$ and $t$ are admissible, we can write them as
\begin{equation*} \begin{aligned} & \mathrm{e}^{tX} f(x) = f(x) + t X f(x) + \frac{t^2}{2} X^2 f(x) + \mathcal{O}(t^3),
\\[1em] & \mathrm{e}^{sY} g(x) = g(x) + sY g(x) + \frac{s^2}{2} Y^2 g(x) + \mathcal{O}(s^3). \end{aligned} \end{equation*}
We can now use these formulas to compute a second-order approximation of $ \mathrm{e}^{tX} \mathrm{e}^{sY}$ and $\mathrm{e}^{sY}\mathrm{e}^{tX}$ to better understand how the difference behaves. Namely, we have
\begin{equation*}\mathrm{e}^{sY}\mathrm{e}^{tX} f(x) = f(x) + \left[ t X + s Y \right] f(x) + \left[ \frac{t^2}{2} X^2 + st YX + \frac{s^2}{2} Y^2 \right] f(x) + \dots, \end{equation*}
and, by symmetry, also
\begin{equation*}\mathrm{e}^{tX} \mathrm{e}^{sY}f(x) = f(x) + \left[ t X + s Y \right] f(x) + \left[ \frac{t^2}{2} X^2 + st XY + \frac{s^2}{2} Y^2 \right] f(x) + \dots. \end{equation*}
It turns out that the difference at the second order is given by
\begin{equation*}\left[ \mathrm{e}^{sY}, \, \mathrm{e}^{tX} \right] f(x) = st [Y, \, X] f(x), \end{equation*}
which means that $[Y, \, X] = 0$ and having the flows commute are strictly related problems. As a matter of fact, even the exponential $\mathrm{e}^{tX + sY}$ differs from the previous ones (at the second-order approximation) by something multiplied by $[Y, \, X]$.

\begin{theorem} Let $X$ and $Y$ be smooth vector fields on $\Omega$. The following assertions are equivalent: \mbox{}
\begin{enumerate}[label=\textbf{(\arabic*)}]
\item There exists $\delta > 0$ such that for all $s, \, t$ admissible with $\max\{|t|, \, |s|\} < \delta$ we have
\begin{equation*}\mathrm{e}^{tX} \mathrm{e}^{sY} = \mathrm{e}^{sY}\mathrm{e}^{tX}. \end{equation*}
\item There exists $\delta > 0$ such that for all $t$ admissible with $|t| < \delta$ we have
\begin{equation*}\mathrm{e}^{tX}Y = Y \mathrm{e}^{tX}. \end{equation*}
\item There exists $\delta > 0$ such that for all $s$ admissible with $|s| < \delta$ we have
\begin{equation*}\mathrm{e}^{sY}X = X\mathrm{e}^{sY}. \end{equation*}
\item The vector fields $X$ and $Y$ commute, that is, $[X, \, Y] = 0$.
\item There exists $\delta > 0$ such that for all $s, \, t$ admissible with $\max\{|t|, \, |s|\} < \delta$ we have
\begin{equation*}\mathrm{e}^{tX} \mathrm{e}^{sY} = \mathrm{e}^{tX + sY}. \end{equation*}
\end{enumerate}
Furthermore, if one of these assertions hold, we can extend $\delta$ in such a way that $t$ and $s$ can be chosen among all the admissible ones.
\end{theorem}

\begin{proof}Suppose that $\mathbf{(1)}$ holds and notice that we can write
\begin{equation*} \mathrm{e}^{tX}\mathrm{e}^{sY}\mathrm{e}^{-tX} f = f \circ \varphi_{X, \, -t} \circ \varphi_{Y, \, s} \circ \varphi_{X, \, t}. \end{equation*}
Denote by $\Phi_t(x, \, s)$ the composition on the right-hand side except for $f$ and notice that it has the property of a flow, which means that there exists $X_t$ smooth vector field on $\Omega$ such that its flow coincide, that is,
\begin{equation*} \Phi_t(x, \, s) = \varphi_{X_t, \, s}(x). \end{equation*}
It is now easy to verify that
\begin{equation*} \mathrm{e}^{tX}\mathrm{e}^{sY}\mathrm{e}^{-tX} f(x) = \mathrm{e}^{s X_t} f(x),\end{equation*}
and therefore we can compute $X_t f(x)$ by differentiating the formula above at $s = 0$ with respect to $s$. More precisely, we have that
\begin{equation*} X_t = \frac{\mathrm{d}}{\mathrm{d}s} \, \big|_{s = 0} \mathrm{e}^{s X_t} = \mathrm{e}^{tX} Y \mathrm{e}^{-tX}.\end{equation*}
Since $\mathbf{(1)}$ holds, we also have that $X_t$ is identically (w.r.t. $t$) equal to $Y$. The formula above now tells us that $\mathbf{(2)}$ (and, equivalently, $\mathbf{(3)}$) holds. Now notice that
\begin{equation*} \frac{\mathrm{d}}{\mathrm{d}t} \, \big|_{t = 0}  X_t = \mathrm{e}^{tX} [Y, \, X] \mathrm{e}^{-sX}, \end{equation*}
and therefore if we assume that $\mathbf{(2)}$ (and $\mathbf{(3)}$) holds, we find that
\begin{equation*} \frac{\mathrm{d}}{\mathrm{d}t} \, \big|_{t = 0}  X_t = [Y, \, X].  \end{equation*}
But $Y_t$ is identically $Y$ so the derivative must be equal to zero and hence the commutator $[Y, \, X]$ must be zero well. This proves that $\mathbf{(2)}$ implies $\mathbf{(4)}$, but the reverse argument proves the reverse implication (although we will not need it.) Now introduce the function
\begin{equation*}T(s, \, t) := \mathrm{e}^{-tX}\mathrm{e}^{tX + sY}\mathrm{e}^{-sY}. \end{equation*}
If we can prove that $T$ is constant (in both variables) and equal to the identity, then we would be able to conclude that $\mathbf{(5)}$ holds. For this, let
\begin{equation*}T_\alpha(s) := \mathrm{e}^{- \alpha s X}\mathrm{e}^{\alpha s X + sY}\mathrm{e}^{-sY},\end{equation*}
where $\alpha$ is an arbitrary parameter which allows us to reduce the derivative problem to a single-variable function. Assume that $\mathbf{(4)}$ holds and notice that
\begin{equation*} [\alpha X + Y, \, Y] = 0. \end{equation*}
An easy computation, together with this commutator relation, shows that $T_\alpha^\prime(s)$ is identically equal to zero, and therefore we can infer $\mathbf{(5)}$. Finally $\mathbf{(5)}$ trivially implies $\mathbf{(1)}$, so the chain of implications is complete.\end{proof}

\section{Foliations and distributions}
\label{sec:foldist}

In this section, we introduce some higher-dimensional analogues of vector fields and integral curves, replacing vectors with $k$-dimensional subspaces and integral curves with $k$-dimensional submanifolds.

\bd\index{submanifold!immersed}
Let $M$ be a smooth manifold. An {\em immersed submanifold} in $M$ is the image of an immersion $S \hookrightarrow M$.
\ed

\bd[Foliation] \index{foliation} Let $M$ be a smooth manifold. A {\em $k$-dimensional foliation} is a partition of $M$,
\[
M = \bigcup_{F \in \F} F,
\]
where each $F$ is an injectively immersed connected submanifold, such that the following holds: for every $p\in M$ there is a chart
\[
\Phi : U \longrightarrow \R^n,
\]
with $p \in U$, that sends the intersection of every $F \in \F$ with $U$ into a collection of countably many parallel affine $k$-planes of the type
\[
\{x_{k+1} = c_{k+1}, \, \dots, \, x_n = c_n\}.
\]
\ed

The immersed submanifolds in the definition are usually referred to as {\em leaves}\index{foliation!leaves} of the foliation $\F$. Moreover, any chart $\varphi$ that satisfies the property above is said to be {\em compatible} with the foliation.

\brmk
Any foliation is made up of uncountably many leaves. Indeed, the countable union of immersed manifolds of dimension strictly smaller than $M$ has measure zero.
\ermk

There is an equivalent definition of foliation which translates everything to local conditions on the transition maps. More precisely, we have:

\bd[Foliation] \index{foliation} Let $M$ be a smooth manifold. A {\em $k$-dimensional foliation} is an atlas $\{ \varphi_i : U_i \to \R^n\}$, compatible with the smooth structure of $M$, which transition maps are locally given by
\[
\varphi_{i, j}(x, \, y) = (\varphi_{i, j}^1(x, \, y), \, \varphi_{i, j}^2(y)),
\]
where $(x, \, y) \in \R^{k} \times \R^{\m - k}$.
\ed


\begin{definition}[Distribution] \index{distribution} Let $M$ be a smooth manifold. We say that $D$ is a $k$-distribution in $M$ if $D$ is a rank-$k$ subbundle $D$ of the tangent bundle $TM$, where
\[ TM = \bigcup_{p \in M} T_p M. \]\end{definition}

In other words, a distribution is a collection of $k$-subspaces $D_p \subset T_p M$ that varies smoothly (in a smooth manifold!) with $p \in M$.

\begin{definition}[Integrable] \index{distribution!integrable} We say that a distribution $D$ on $M$ is {\em integrable} if it is tangent to some foliation $\mathscr{F}$. Namely, for each $p \in M$ there is $F \in \mathscr{F}$ such that
\[ D_p = T_p F. \]\end{definition}

We say that a distribution $D$ is {\em involutive} if whenever $X$ and $Y$ are tangent vector fields to $D$ (i.e., $X(p),\, Y(p) \in D_p$), then also $[X, \, Y]$ is a tangent vector field.

\begin{theorem}[Frobenius] \index{Frobenius theorem}A distribution $D$ is integrable if and only if it is involutive. \end{theorem}
\chapter{Baker–Campbell–Hausdorff Formula}

Let $X$ and $Y$ be vector fields. If $[X, \, Y] = 0$, then it is easy to verify that the exponential map of the sum is given by
\[
\e^{tX + sY} = \e^{tX} \e^{sY},
\]
but this is not the case if $X$ and $Y$ do not commute. The main goal of this chapter is to prove the \textbf{Baker–Campbell–Hausdorff formula}, which gives us a way to express
\[
\e^{tX+ sY}
\]
in terms of the so-called elementary commutators. The central role taken on by this formula will be more clear when we will talk about Lie groups and, more specifically, Lie algebras.

\section{Free associative algebra}

Let $R$ be a commutative ring. The free (associative, unital) algebra\index{free algebra} on $n$ variables, say $\{a_1, \, \dots, \, a_n\}$ is the free $R$-module with a basis consisting of all words over the alphabet
\[
\scA := \{ a_1, \, \dots, \, a_n \},
\]
including the empty word, which is the unit of the free algebra. It is easy to see that endowed with the concatenation of words as multiplication, we obtain an $R$-algebra

Let $R := \R$ and consider the {\em associative free algebra generated} generated by two elements, say $x$ and $y$, over $R$. We recall that an element (or word) of length $L$ can be written as
\[
u = x^{i_1} y^{j_1} \dots x^{i_\ell} y^{j_\ell},
\]
where $i_k, \, j_k \geq 0$ and $\sum_{k = 1}^\ell (i_k + j_k) = L$. If $[x, \, y] = 0$, then $L$-words are of the form
\[
u = x^{\ell} y^{L - \ell},
\]
but, in general, we do not expect to be in such a simple situation. It makes sense to introduce the notation $a[x, \, y]$ for the free algebra generated by $x$ and $y$. Recall that
\[
a \llbracket x, \, y \rrbracket
\]
indicates the set of formal power series with elements in $a[x, \, y]$. Namely, $u \in a \llbracket x, \, y \rrbracket$ if
\begin{equation*} u = \sum_{k = 0}^\infty c_k, \end{equation*}
where $c_k$ is a linear combination of words of length $k$. The exponential of $u$,
\[
\e^u = \sum_{k = 0}^\infty \frac{1}{k!} u^k,
\]
is well-defined provided that $c_0 = 0$ (so we can apply Taylor's theorem) so we define 
\[
a_0 \llbracket x, \, y \rrbracket := \{ u \in a_0 \llbracket x, \, y \rrbracket \: : \: c_0 = 0 \}.
\]

\brmk
Let $\cA$ be an associative algebra. The bracket operator
\[
[u, \, v] := uv - vu
\]
satisfies the following properties: \mbox{}
\begin{enumerate}[label={\color{red}\textbf{(\alph*)}}]
\item $[ \cdot, \, \ast ]$ is bilinear;
\item $[ \cdot, \, \ast ]$ is skew-symmetric;
\item $[ \cdot, \, \ast ]$ satisfies the Jacobi identity\index{Jacobi identity}, that is,
\begin{equation}\label{jacobi1000}\left[ [u,\,v], \, w \right] +  \left[ [w,\, u], \, v \right] +  \left[ [v,\,w], \, u \right] = 0. \end{equation}
\end{enumerate}
In particular, any associative algebra $\cA$ endowed with this bracket operator is a Lie algebra\footnote{Recall that a Lie algebra only requires the existence of a bracket operator\index{bracket operator}, but it does not have to be associative. In any case, we say that when $[\cdot, \, \ast] \equiv 0$ the associated Lie algebra is abelian.}. The interested reader can find a more precise definition in \hyperref[chapter:Liegroups]{Chapter \ref{chapter:Liegroups}}, but we still recommend to read it after this one.
\ermk

\section{Baker–Campbell–Hausdorff formula}

Let $\cF$ be the Lie sub-algebra generated by $x$ and $y$ in $a \llbracket x, \, y \rrbracket$. Then $\cF$ is closed under the bracket operator and it is easy to verify that
\[
x, \, y \in \cF, \quad [x, \, y] \in \cF, \quad [[x, \, y], \, x] \in \cF.
\]
One might also check that elements such as $x^2$ or $y^2$ are not in $\cF$ proving that it is not possible to obtain them via the bracket operator starting from $x$ and $y$ only.

\bd
A commutator of the form
\[
\left[ a_1, \, \dots [a_{k-2}, \, [a_{k-1}, \, a_k]] \dots \right]
\]
is called {\em elementary iterated commutator}\index{elementary iterated commutator}.
\ed

\brmk
The Lie sub-algebra $\cF$ is not made up entirely of elementary iterated commutators. For example,
\[
[ [[x, \, y], \, x], \, [x, \, y]] \in \cF
\]
is not a elementary commutator. However, using \eqref{jacobi1000}, one can prove that it can also be written as the sum of elementary commutators.
\ermk

\bpr
The linear space $\cF$ is spanned by $x, \, y$ and elementary commutators.
\epr

As a corollary of this result, we can always write $\cF$ as the sum of linear subspaces in the following form:
\[
\cF = \sum_{k_1 + k_2 \geq 1} \cF^{k_1, \, k_2},
\]
where $(k_1, \, k_2)$, called {\em bidegree}, indicates the number of times $x$ and $y$ appear in the elementary iterated commutator.

\bthm[Baker–Campbell–Hausdorff] \label{thm.bchthm} \index{Baker–Campbell–Hausdorff formula}
Let $x$ and $y$ be as above. Then
\[
\e^{x} \e^{y} = \e^u, 
\]
and we can express $u$ explicitly with the following formula:
\[
u = x + y + \sum_{k_1 + k_2 \geq 1} c_{k_1, \, k_2},
\]
where $c_{k_1, \, k_2}\in \mathcal{F}^{k_1, \, k_2}$ for all $(k_1, \, k_2)$ admissible. In particular, if $X$ and $Y$ are smooth vector fields on $\Omega$ that do not commute, we have that
\begin{equation}\label{bchformula}
\e^{tX} \e^{sY} = \e^{tX + sY + \sum_{k_1 + k_2 \geq 1} c_{k_1, \, k_2}(tX, \, sY)}.
\end{equation}
\ethm

\brmk
There are iterative methods to determine the coefficients $c_{k_1, \, k_2}$, but we usually care only about the abstract formula. In any case, the first two terms - which is always useful to remember - are given by
\[
X + Y + \frac{1}{2}[X, \, Y] + \frac{1}{12} \left( [X, \, [X, \, Y]] - [Y,\, [X, \, Y]] \right).
\]
\ermk

\brmk
The \textsc{BCH} formula allows one to write
\[
\e^{u} \e^{v} = \e^w
\]
even if $u$ and $v$ are elements of $\cF$ rather than $x$ or $y$. Hence $w$ is not, in general, in $\cF$ but rather a formal sum of such elements.
\ermk

\subsection{Consequences of the BCH formula} %% Qui

\begin{proposition} \label{prop.diff.2.1}\mbox{}
\begin{enumerate}[label=\textbf{(\roman*)}]
\item For each $n \geq 1$ there exists a unique decomposition
\begin{equation*} \mathrm{e}^{x+y} = \mathrm{e}^x \mathrm{e}^y \mathrm{e}^{w_2} \dots \mathrm{e}^{w_n} \mathrm{e}^{r_{n+1}}, \end{equation*}
where $w_j \in \mathcal{F}^j$ and $r_{n+1} \in \sum_{k \geq n + 1} \mathcal{F}^k$.
\item If $c_p \in \mathcal{F}^p$, then there exists $q(p) := q > 0$ such that
\begin{equation} \label{eq.g.1} \mathrm{e}^{c_p} = \mathrm{e}^{a_1}\dots \mathrm{e}^{a_q} \mathrm{e}^{r_{p+1}}, \end{equation}
where $a_j \in \{ \pm x, \, \pm y\}$ and $r_{p+1} \in \sum_{k \geq p + 1} \mathcal{F}^k$.
\end{enumerate} \end{proposition}

\begin{proof} We can prove both assertions with $n = 1$, $p = 2$ and then apply the induction principle to conclude that it holds for all $n$ and all $p$. \mbox{}
\begin{enumerate}[label=\textbf{(\roman*)}]
\item Set
\begin{equation*} \star := \mathrm{e}^{-y}\mathrm{e}^{-x}\mathrm{e}^{x + y}, \end{equation*}
and use the \textsc{BCH} formula to rewrite the first two terms as follows:
\begin{equation*} \star = \mathrm{e}^{-(x+y) + r_2(x, \, y)}\mathrm{e}^{x + y}. \end{equation*}
Now use the formula again to obtain the identity
\begin{equation*} \star = \mathrm{e}^{r_2(x, \, y) + \frac{1}{2}[-(x+y) + r_2(x, \, y), \, x + y]} = \mathrm{e}^{\widetilde{r_2}(x, \, y)}, \end{equation*}
and this proves the formula for $n = 1$ since
\begin{equation*} \mathrm{e}^{x+y} = \mathrm{e}^x \mathrm{e}^y \mathrm{e}^{\widetilde{r_2}(x, \, y)}. \end{equation*}
\item Let $p = 2$. Then
\begin{equation*} \begin{aligned} \mathrm{e}^{x}\mathrm{e}^{y}\mathrm{e}^{-x}\mathrm{e}^{-y} & = \mathrm{e}^{x + y + \frac{1}{2}[x, \, y] + r_3(x, \, y)} \mathrm{e}^{-(x + y) + \frac{1}{2}[x, \, y] + r_3(-x, \, -y)} =
\\[1em] & = \mathrm{e}^{[x, \, y] + \frac{1}{2} \left[ x + y + \frac{1}{2}[x, \, y] + r_3(x, \, y), \, -(x + y) + \frac{1}{2}[x, \, y] + r_3(-x, \, -y) \right] + r_3^\prime } =
\\[1em] & = \mathrm{e}^{[x, \, y] + r_3^{\prime \prime}} \stackrel{\star}{=}
\\[1em] & \stackrel{\star}{=} \mathrm{e}^{[x, \, y]} \mathrm{e}^{r_3^{\prime \prime}} \mathrm{e}^{r_2([x, \, y], \, r_3^{\prime \prime})} =
\\[1em] & = \mathrm{e}^{[x, \, y]} \mathrm{e}^{\widetilde{r_3}(x, \, y)},\end{aligned}\end{equation*}
\end{enumerate}
where the identity $\star$ follows from the previous point and all the others from applications of the \textsc{BCH} formula. \end{proof}

% \begin{corollary}  \end{corollary} VEDI BENE

\begin{remark} The formula \eqref{eq.g.1} for $\mathrm{e}^{x+y}$ can be generalized to more than two elements. Namely, one can prove that
\begin{equation} \label{eq.g.1.a} \mathrm{e}^{x_1 + \dots + x_n} = \mathrm{e}^{a_1}\dots \mathrm{e}^{a_q} \mathrm{e}^{r_{p+1}}, \end{equation}
where $a_j \in \{ \pm x_1, \, \dots, \, \pm x_n\}$ and $r_{p+1} \in \sum_{k \geq p + 1} \mathcal{F}^k$ and $\mathcal{F}^k$ is the space of elementary commutators with $n$-degree equal to $k$.\end{remark}

\begin{remark}If $X$ and $Y$ are smooth vector fields, then
\begin{equation*} \mathrm{e}^{t(X + Y)} = \mathrm{e}^{tX} \mathrm{e}^{tY} \mathrm{e}^{\frac{t^2}{2}[X, \, Y]} (1 + \mathcal{O}(t^3)). \end{equation*}
Therefore
\begin{equation*} \mathrm{e}^{-\frac{t^2}{2}[X, \, Y]} \mathrm{e}^{-tY}\mathrm{e}^{-tX}\mathrm{e}^{t(X + Y)} f(x) - f(x) = \mathcal{O}(t^3), \end{equation*}
so we find that the derivatives of first and second order of the left-hand side with respect to $t$ is zero at $t = 0$.
A straightforward computation shows that
\begin{equation*}0 = \frac{\mathrm{d}}{\mathrm{d}t} \, \big|_{t = 0} f(x) = - Yf(x) - Xf(x) + (X + Y)f(x) = 0, \end{equation*}
while, with a little bit more efforts, the second derivative equal to zero gives us
\begin{equation*}0 = \frac{\mathrm{d}}{\mathrm{d}t^2} \, \big|_{t = 0} f(x) = [...]. \end{equation*}
\end{remark}


\printindex

\bibliographystyle{siam}
\addcontentsline{toc}{chapter}{Bibliography} \markboth{Bibl}{}
\bibliography{bibs}{} % BIBLIOGRAFIA
\end{document}
