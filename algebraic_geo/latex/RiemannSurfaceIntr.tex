\chapter{Introduction to Riemann Surfaces} \thispagestyle{empty}

In this chapter, we introduce the notion of \textit{Riemann surface}, and we profoundly analyze the fundamental example: \textit{smooth algebraic projective curves}.

In the final part, we discuss two examples - both of which will appear many times in the remainder of this course: the Riemann sphere, and the complex torus.

\section{Main Definitions and Basic Properties}

In this section, we give the definition of Riemann surface. The reader should pay attention to the fact that a Riemann surface does not need to be \textit{connected}, but we will ask for it in the definition because we will mostly be dealing with connected compact Riemann surfaces.

\begin{definition}[Riemann Surface] \index{Riemann Surface} Let $X$ be a topological manifold. We say that $X$ is a \textit{Riemann surface} if the following properties are satisfied: \mbox{}
\begin{enumerate}[label=\textbf{(\alph*)}]
\item $\mathrm{dim}_{\mathbb{R}} \, X = 2$.
\item $X$ is Hausdorff, second-countable (i.e., there exists a countable basis) and connected.
\item $X$ has a \index{Complex Atlas}complex structure. Namely, there exists an atlas
\begin{equation*} \mathcal{U} = \left\{ \varphi_i : U_i \to V_i \subseteq \C \right\}_{i \in I}\end{equation*}
such that $\varphi : U_i \to V_i$ is a homeomorphism of open sets, and the transition maps $\varphi_{i, \, j} := \varphi_j \circ \varphi_{i}^{-1}$ are biholomorphic functions. In particular, the map
\begin{equation*}\varphi_{i, \, j} : \varphi_i \left(U_i \cap U_j\right) \subset \C \to \varphi_j \left(U_i \cap U_j\right) \subset \C \end{equation*}
is holomorphic with respect to the complex variable $z$.
\end{enumerate}
\end{definition}

\begin{definition}[Biholomorphic] \index{Biholomorphic functions} Let $f : U \subset \C^n \to V \subset \C^n$ be a complex function. We say that $f$ is \textit{biholomorphic} if $f$ is holomorphic, bijective and its inverse $f^{-1} : V \to U$ is also holomorphic.
\end{definition}

\begin{remark}A Riemann surface $X$ is always orientable. In fact, the Jacobian of the transition maps is always strictly positive (since $\varphi_{i, \, j}$ is holomorphic), hence the atlas is orientated. The interested reader may find a complete proof of this fact \href{https://math.stackexchange.com/a/174365}{here}. \end{remark}

\begin{theorem}[Structure] \index{Structure Theorem} Let $X$ be a compact Riemann surface. Then $X$ is either homeomorphic to a sphere ($g(X) = 0$), a torus ($g(X) = 1$) or a $n$-torus ($g(X) = n$). \end{theorem}

\begin{figure}[h]
\centering
\includegraphics[width=16cm, height=10cm]{Images/GAC001.png}
\label{fig:classth}
\caption{Structure Theorem for Compact Riemann Surfaces.}
\end{figure} 

\section{Projective Curves}

Let $\p^2(\C)$ be the complex projective space of dimension $2$, and let $[z_1 : z_2 : z_0]$ be the coordinates so that $\{ z_0 = 0\}$ is the line at infinity. An algebraic curve is defined as
\begin{equation*} X := \left\{ F(z_1, \, z_2, \, z_0) = 0 \right\}, \end{equation*}
where $F$ is a homogeneous\footnote{For every $\lambda \in \C$, it turns out that $F(\lambda \, z_1, \, \lambda \, z_2, \, \lambda \, z_0) = \lambda^d \, F(z_1, \, z_2, \, z_0)$.} polynomial of degree $d$. Recall that \index{Reduced and Irreducible, Curves} \mbox{}
\begin{enumerate}[label=\textbf{(\alph*)}]
\item $X$ is irreducible $\iff$ $F$ is irreducible;
\item $X$ is reduced $\iff$ $\mathcal{I}(F) = \sqrt{ \mathcal{I}(F)}$,
\end{enumerate}
where $\mathcal{I}(F)$ is the ideal associated to $F$ (in this case, simply $\mathcal{I}(F) = (F)$).

\paragraph{N.B.} From now on we will assume that $X$ is an \textit{irreducible} and \textit{reduced} (projective) algebraic curve.

\begin{definition}[Singular points \index{Singular Points!of a curve}] A point $p \in X$ is \textit{singular} for $X$ if
\begin{equation*}F(p) = \frac{\partial \, F}{\partial \, z_i}(p) = 0, \qquad \forall \, i = 0, \, 1, \, 2. \end{equation*}
\end{definition}

\begin{definition}[Smooth] \index{Algebraic curve!Smoothness} The (projective) algebraic curve $X \subset \p^2(\C)$ is \textit{smooth} if and only if no point $p \in X$ is singular.
\end{definition}

\begin{remark}The projective complex plane $\p^2(\C)$ admits a standard atlas $\mathcal{U} := \left\{ (U_i, \, \varphi_i) \right\}_{i = 0, \, 1, \, 2}$, which is defined by setting
\begin{equation*}U_i := \{ z_i \neq 0\} \qquad \text{and} \qquad \varphi_i : U_i \longrightarrow \C^2, \quad [z_1 : z_2 : z_0] \longmapsto \left( \frac{z_j}{z_i}, \, \frac{z_k}{z_i} \right). \end{equation*}
Then $\mathcal{U}$ induces, by restriction, an atlas on $X$, which is given by
\begin{equation*}\mathcal{X} := \left\{ (U_i \cap X, \, \varphi_i \, \big|_{X}) \right\}_{i = 0, \, 1, \, 2}. \end{equation*} \end{remark}

\begin{proposition}The projective algebraic curve $X$ is smooth if and only if the affine algebraic curve $X_i := X \cap U_i \subset U_i \cong \C^2$ is a smooth for every $i = 0, \, 1, \, 2$. \end{proposition}

\begin{theorem}[Implicit function] \label{implfunc}Let $F \in \C[z_1, \, z_2]$ be any polynomial, and denote by $X := \{ F = 0 \} \subset \C^2$ the associated algebraic variety.

Let $p \in X$ be a point such that $\partial_{z_2} \, F(p) \neq 0$. There are a neighborhood $U_p$ of $p$ and a holomorphic function $G : U \to V$ such that
\begin{equation*} X \cap U = \left\{ (z_1, \, G(z_1)) \, \left| \, z_1 \in V \right. \right\}. \end{equation*}
 \end{theorem}

\begin{corollary}Let $p \in X$ be a smooth point. There exists a neighborhood $U_p$ of $p$ (which is exactly the one given by \hyperref[implfunc]{Theorem \ref{implfunc}}), such that $X$ has a local complex structure, that is,
\begin{equation*} U \cap X = \left\{ (z_1, \, G(z_1)) \right\} \stackrel{\simeq}{\longrightarrow} \left\{ z_1 \, \left| \, z_1 \in V \right. \right\}.\end{equation*}
\end{corollary}

\begin{theorem}\index{Algebraic curve!Classification theorem} Let $X \subset \p^2(\C)$ be a smooth algebraic curve of degree $d$. Then $X$ is a compact Riemann surface, and its genus is equal to
\begin{equation*} g(X) = \frac{(d-1)(d-2)}{2}. \end{equation*} \end{theorem}

\begin{proof}We will prove this result later, but the reader which is already interested in a simple proof of this fact, may jump and take at look at \href{https://math.la.asu.edu/~paupert/degreegenus.pdf}{this paper}. \end{proof}

\subsection{Multiplicities}

\paragraph{Recall.} Let $X \subseteq \p^2(\C)$ be any irreducible and reduced algebraic curve, and let us denote it by
\begin{equation*}X := \{ F(z_1, \, z_2, \, z_0) = 0\},\end{equation*}
where $F$ is an homogeneous irreducible polynomial of degree $d$ such that the associated ideal $(F)$ is radical. Let $p \in X$ be a singular point of $X$, that is, a point where $F$ and all its derivatives vanish:
\begin{equation*}F(p) = \frac{\partial \, F}{\partial \, z_1}(p) = \frac{\partial \, F}{\partial \, z_2}(p) = \frac{\partial \, F}{\partial \, z_0}(p) = 0. \end{equation*}

\begin{definition}[Multiplicity] \index{Singular Points!multiplicities} The \textit{multiplicity} of a point $p$ in $X$ is the least integer $k$ among all the multiplicities of $p$ with respect to the intersection between $X$ and the lines passing through $p$. \end{definition}

\noindent More precisely, we define
\begin{equation*}\mathrm{molt}_p \left(X\right) := \min \left\{ \mathrm{molt}_p \left(r, \, X\right) \: : \: r \, \, \text{line passing through $p$} \right\}. \end{equation*}
It is straightforward to prove that a point $p \in X$ is smooth if and only if its multiplicity $\mathrm{molt}_p \left(X\right)$ is equal to $1$. In particular, any singular point has multiplicity greater or equal than $2$.

\begin{proposition} Assume that $p := (0, \, 0, \, 1)$ belongs to $X$. The dehomogenization\index{Dehomogenization} of $F$ with respect to the coordinate $z_0$ (i.e., in $X_0 = X \cap \mathcal{U}_0$) is given by
\begin{equation*}F(z_1, \, z_2, \, 1) = \sum_{k \geq m} F_k(z_1, \, z_2), \end{equation*}
where the $F_k$ are homogeneous polynomials of degree $k$. Then the multiplicity of $X$ at $p$ is the minimum degree of the dehomogenization, that is,
\begin{equation*}\mathrm{molt}_p \left(X\right) = m. \end{equation*}
 \end{proposition}

\begin{definition}[Ordinary point] \index{Singular Points!ordinary}Let $p \in X$ be any singular point, and suppose that its multiplicity is equal to $m$. We say that $p$ is an \textit{ordinary multiple point} if, locally,
\begin{equation*}F = \prod_{j = 1}^{m} H_j, \end{equation*}
where the $H_j$ are linear forms such that $H_j = H_i \iff i = j$.\end{definition}

\begin{example} \mbox{}
\begin{enumerate}[label=\textbf{(\arabic*)}]
\item An ordinary double point is locally (in a neighborhood of $p:=(0, \, 0, \, 1)$) given by the equation $z_1 \cdot z_2 = 0$ (see e.g. \hyperref[fig:punti]{Figure \ref{fig:punti}}, left).
\item A non-ordinary double point is locally given by an equation of the form $z_1^3 = z_2^2$, and it corresponds to a singular cuspid kind of point (see e.g. \hyperref[fig:punti]{Figure \ref{fig:punti}}, right). \end{enumerate}
\end{example}

\begin{figure}[h]
\centering
\includegraphics[width=16cm, height=8cm]{Images/GAC002.png}
\caption{Examples of singular points: left ordinary, right non-ordinary.}
\label{fig:punti}
\end{figure} 

\subsection{Resolution of the Singularities}
\label{sec:blow}

\paragraph{Introduction.} Let $X$ be a singular projective algebraic curve. The primary goal of this subsection is to give to the reader two methods which are useful to resolve the singularities of $X$, that is, to find a smooth algebraic curve $\widetilde{X}$ and a surjective map 
\begin{equation*}\Phi : \widetilde{X} \longtwoheadrightarrow X. \end{equation*}

\paragraph{Topological Approach.} Let $p \in X$ be an ordinary multiple point, and let $\mathcal{U}$ be a neighborhood of $p$ that does not contain any other singular point of $X$. By definition, the polynomial $F$ is locally (i.e., in $\mathcal{U}$) given by the product of $m$ linear forms:
\begin{equation*}F = \prod_{j = 1}^{m} H_j. \end{equation*}
If we denote by $\Delta$ the Poincaré disk\index{Poincaré disk}, that is,
\begin{equation*} \Delta := \left\{ z \in \C \: \left| \: |z| < 1 \right. \right\} \qquad \text{and} \qquad \Delta^\ast := \Delta \setminus \{0\}, \end{equation*}
then the algebraic curve - without the singular point $p$ - is locally homeomorphic to the union of $m$ copies of $\Delta^\ast$; more precisely, it turns out that
\begin{equation*} \left( \mathcal{U} \cap \left\{ H_j = 0 \right\} \right) \setminus \{p\} \cong \Delta^\ast, \qquad \forall \, j \in \{1, \, \dots, \, m\}. \end{equation*}
Consequently, the algebraic curve $X$ is \textit{locally} homeomorphic to the wedge of $m$ disks ($\Delta$) of center $p$, that is, there is a homeomorphism
\begin{equation*} \mathcal{U} \cap X \cong \bigwedge_{j = 1}^{m} \Delta. \end{equation*}
If we remove the singular point $p$, the wedge is clearly homeomorphic to the disjoint union of the $m$ disks deprived of their centers, i.e.,
\begin{equation*} \left( \mathcal{U} \cap X \right) \setminus \{p\} \cong \bigsqcup_{j = 1}^{m} \Delta^\ast. \end{equation*}
At this point, the resolution of the singularity $p$ is entirely straightforward: we add a center to each disk $\Delta^\ast$. Formally, we define the smooth manifold
\begin{equation*} \widetilde{X} := \left( X \setminus \{p\} \right) \cup \left\{ q_1, \, \dots, \, q_m \right\}, \end{equation*}
and we prove that it is homeomorphic to a disjoint union of balls, i.e.,
\begin{equation*} \widetilde{X} \cong \bigsqcup_{j = 1}^{m} \left( \Delta^\ast \cup \{q_j\}\right) \cong \bigsqcup_{j = 1}^{m} \Delta. \end{equation*}
In the general case, we give a brief sketch of what the ideas behind are. Take any $p \in X$ and any chart which sends $p$ to the origin of $\C$ in such a way that there exists (locally) a function $f:X \to \C$ with the additional property that
\begin{equation*} f : f^{-1} \left(\Delta^\ast\right) \to \Delta^\ast \end{equation*}
is a covering of order $m$.

On the other hand, we know that the connected coverings of $\Delta^\ast \cong S^1$ are all and only of the form $z \mapsto z^m$. Thus we only need to add $m$ points \textit{over} the point $(0, \, 0)$.

\begin{figure}[h]
\centering
\includegraphics[width=18cm, height=10cm]{Images/GAC003.png}
\caption{Idea behind the topological approach.}
\label{fig:ta}
\end{figure} 

\paragraph{Blowup Approach.} \index{Blowup}Let $p = (0, \, 0) \in \C^2$. The blowup of $\C^2$ at the point $p$ is defined as follows:
\begin{equation*} \mathrm{Bl}_{p} \left( \C^2 \right) := \left\{ \left(z_1, \, z_2; \; [a : b]\right) \: \left| \: z_1 \, b = z_2 \, a  \right.\right\} \subseteq \C^2 \times \p^1(\C). \end{equation*}
There exists a map
\begin{equation*} \pi : \mathrm{Bl}_{p} \left( \C^2 \right) \longrightarrow \C^2 \end{equation*}
such that $\pi^{-1}(p) \cong \p^1(\C)$. We denote by $E$ the fiber $\pi^{-1}(p)$, and, from now on, we will refer to it as the \textit{exceptional line}\index{Blowup!Exceptional line} (since it contains the directions of the lines passing through $p$).

Consequently, the complement of $E$ in the blowup is homeomorphic to the complement of the fiber, that is,
\begin{equation*} \mathrm{Bl}_{p} \left( \C^2 \right) \setminus E \cong \pi^{-1} \left( \C^2 \setminus \{(0, \, 0)\} \right).  \end{equation*}

Assume that $p = (0, \, 0) \in X_0$ is a singular point of the affine algebraic curve $X_0 := \left\{ F(z_1, \, z_2) = 0 \right\} \subseteq \C^2$. Then $F$ is a sum of homogeneous polynomials of order $\geq m$, that is,
\begin{equation*}F(z_1, \, z_2, \, 1) = \sum_{k \geq m} F_k(z_1, \, z_2), \end{equation*}
and we may assume, without loss of generality, that $\{z_1 = 0 \}$ is not tangent to $X_0$ (i.e., $a \neq 0$).

If we set $v := b/a$, then we can define the \textbf{strict transform}\index{Strict transform} of $F$ with respect to the coordinates $(z_1, \, v)$ as follows:
\begin{equation} \label{eq:stricttansfor} \widetilde{F}(z_1, \, v) := F(z_1, \, z_1 \cdot v) \cdot z_1^{-m}. \end{equation}
In a neighborhood $\mathcal{U}$ of $p$ the map 
\begin{equation*}\widetilde{X_0} := \left\{ \widetilde{F} = 0 \right\} \implies \widetilde{X_0} \cap \mathcal{U} \longtwoheadrightarrow X_0 \cap \mathcal{U}\end{equation*}
is surjective, and it is straightforward to prove that $u$ is a slope coefficient of a tangent line at $p$ to $X_0$ if and only if it belongs to $E$.

\begin{remark}If $X$ is a singular algebraic curve, then there is no guarantee that $\widetilde{X}$ will be smooth after a single application of the blowup approach.

The next result states that a finite sequence of blowups is enough to obtain a smooth algebraic curve, and in \hyperref[example_3]{Example \ref{example_3}} we describe a case where two steps are necessary.\end{remark}

\begin{theorem}Let $X$ be a singular algebraic curve. There exists a finite sequence of blowups 
\begin{equation*} \widetilde{X}^n \longtwoheadrightarrow \widetilde{X}^{n-1} \longtwoheadrightarrow \dots \longtwoheadrightarrow \widetilde{X}^1 \longtwoheadrightarrow X \end{equation*}
such that $\widetilde{X}^n$ is a compact Riemann surface (thus a smooth algebraic curve). \end{theorem}

\begin{proof}[Idea] The proof of this result is divided into three steps. The reader may try to prove it as an exercise.

\paragraph{Step 1.} There are only finitely many singular points in $X$. \textit{Hint: apply Bezout's theorem applied to the polynomials $(F, \, F^\prime)$, where $F^\prime$ is the usual derivative of $F$}.

\paragraph{Step 2.} Local resolution of the singularities.

\paragraph{Step 3.} For every singular point $p \in X$, the quantity $\mathrm{molt}_p \left(X\right)$ eventually decreases to $0$ as the second step is repeated. \end{proof}

\begin{definition}[Infinitely Near]\index{Singular Points!Infinitley near} Let $p \in X$ be a singular point. A point $q$ is \textit{infinitely near} to $p \in X$, and we denote it by $q \in E(p)$, if $q$ belongs to the exceptional line $E$.\end{definition}

\begin{theorem}\index{Algebraic curve!Resolution theorem}Let $X \subset \p^2(\C)$ be an irreducible and reduced algebraic curve. \mbox{}
\begin{enumerate}[label=\textbf{(\alph*)}]
\item If $X$ is smooth, then $X$ is a compact Riemann surface of genus
\begin{equation*} g(X) = \frac{(d-1)(d-2)}{2}. \end{equation*}
\item If $X$ is singular, then there exist a compact Riemann surface $\widetilde{X}$ and a birational morphism $\pi : \widetilde{X} \longtwoheadrightarrow X$ (that is, an isomorphism outside of singular points) such that
\begin{equation*} g\left(\widetilde{X}\right) = \frac{(d-1)(d-2)}{2} - \sum_{p \in \mathrm{Sing}(X)} \delta_p, \end{equation*}
where $\delta_p$ is equal to
\begin{equation*} \delta_p = \frac{m_p \, (m_p - 1)}{2} + \sum_{q \in E(p)} \frac{m_q \, (m_q - 1)}{2}.  \end{equation*}\end{enumerate}\end{theorem}

\subsection{Examples}

In this brief subsection, we describe the resolution of singularities in very few simple cases, and we come back to the first example of the course.

\begin{example} \label{example_1} Let $X$ be, locally, the algebraic curve defined by the polynomial
\begin{equation*}F(z_1, \, z_2) := z_1^2 - z_2^2 = 0.\end{equation*}
The reader may check that $p = (0, \, 0)$ is singular, and its multiplicity is equal to $2$. If we set $z_2 := v \cdot z_1$, then the strict transform of $F$ is given by
\begin{equation*} \widetilde{F}(z_1, \, v) = \left[ z_1^2 - v^2 \, z_1^2 \right] z_1^{-2} = 1 - v^2. \end{equation*}
Consequently, the intersection $E \cap \{ \widetilde{F}(z_1, \, v) = 0\}$ is made of two points ($v = \pm 1$), corresponding to the singular cross of lines in $X$ (see \hyperref[fig:esdl]{Figure \ref{fig:esdl}}, left).  \end{example}

\begin{example} \label{example_2}Let $X$ be, locally, the algebraic curve defined by the polynomial
\begin{equation*}F(z_1, \, z_2) := z_1^3 - z_2^2 = 0. \end{equation*}
The point $p = (0, \, 0)$ is singular, and its multiplicity is also equal to $2$. If we set $z_2 := v \cdot z_1$, then the strict transform of $F$ is given by
\begin{equation*} \widetilde{F}(z_1, \, v) = \left[ z_1^3 - v^2 \, z_1^2 \right] \, z_1^{-2} = z_1 - v^2. \end{equation*}
Clearly the point $q \in E(p)$ is smooth, but it is tangent to the exceptional line $E$, thus the contribution of $\delta_p$ is nonzero (see \hyperref[fig:esdl]{Figure \ref{fig:esdl}}, center).  \end{example}

\begin{example} \label{example_3} Let $X$ be, locally, the algebraic curve defined by the polynomial
\begin{equation*}F(z_1, \, z_2) := z_1^4 - z_2^2 = 0.\end{equation*}
The point $p = (0, \, 0)$ is singular, and its multiplicity is also $2$. If we set $z_2 := v \cdot z_1$, then the strict transform of $F$ is given by
\begin{equation*} \widetilde{F}(z_1, \, v) = \left[ z_1^4 - v^2 \, z_1^2 \right] \, z_1^{-2} = z_1^2 - v^2. \end{equation*}
The algebraic curve $\widetilde{X}^1$ is singular and, actually, it is the same algebraic curve of \hyperref[example_1]{Example \ref{example_1}}. If we set $z_1 := h \cdot v$, then the second strict transform of $F$ is
\begin{equation*} \widetilde{F}^{(2)}(h, \, v) = 1 - h^2, \end{equation*}
thus $\widetilde{X}^2 \longtwoheadrightarrow \widetilde{X}^1 \longtwoheadrightarrow X$ is the resolution of singularities $X$ (see \hyperref[fig:esdl]{Figure \ref{fig:esdl}}, right).\end{example}

\begin{figure}[h]
\centering
\includegraphics[width=16cm, height=10cm]{Images/GCBUE.png}
\caption{From left to right: Examples \ref{example_1}, \ref{example_2} and \ref{example_3}}
\label{fig:esdl}
\end{figure} 

\begin{example}[Global] Let $\underline{X} \subset \C^2$ be the affine curve defined by the equation
\begin{equation*}z_2^2 - (z_1^2 - 1)(z_1^2 - 4) = 0.\end{equation*}
Its projective closure is obtained via homogenization\index{Homogenization} of the equation, that is,
\begin{equation*} X := \left\{ F(z_1, \, z_2, \, z_0 ) = 0 \right\} \subset \p^2(\C), \qquad F(z_1, \, z_2, \, z_0) = (z_1^2 - z_0^2)(z_1^2 - 4 \, z_0^2) - z_0^2 \, z_2^2. \end{equation*}
It is a simple exercise to prove that the only singular point is $p_\infty = (0, \, 1, \, 0)$ and that it belongs to the line at infinity.

Consequently, about $p_\infty$ it makes sense to use the chart with coordinates $(z_1, \, 1, \, z_0)$ in such a way that, at least locally, the algebraic curve is given by the equation
\begin{equation*}X = \left\{ z_0^2 - (z_1^2 - z_0^2)(z_1^2 - 4 \, z_0^2) = 0 \right\} \stackrel{\sim p_\infty}{=} \left\{ z_0^2 = (z^\prime)^4 \right\}. \end{equation*}
As a consequence, around $p_\infty$ the situation is similar to the one studied in \hyperref[example_3]{Example \ref{example_3}}. In particular, there is a compact Riemann surface $\widetilde{X}$, whose genus is given by\footnote{Indeed, it is straightforward to prove that $m_{p_\infty} = 2$, $m_{q_\infty} = 2$ and $m_{\widetilde{q_\infty}} = 1$.}
\begin{equation*}g\left(\widetilde{X} \right) = 3 - \delta_{p_\infty} = 1, \end{equation*}
and a surjective map $\widetilde{X} \longtwoheadrightarrow X$. We conclude that $\widetilde{X} \cong \mathbb{T}$, that is, $\widetilde{X}$ is homeomorphic to the complex torus (see \hyperref[ex:13]{Example \ref{ex:13}}).

The result is coherent with the fact that the algebraic curve $X$ may be obtained from a $3$-torus by gluing together the two extremal holes and throttling them in such a way to get a $1$-torus with a weird point $p_\infty$. 
\end{example}

\begin{remark}The genus that we have introduced in this subsection is called \textit{arithmetic genus} of a planar algebraic curve, and it is equal to the leading coefficient of the Hilbert polynomial (associated to the local coordinate ring).

This notion of genus is equivalent to the topological one if $X$ is a smooth algebraic curve, where
\begin{equation*}g_{top} \left(X \right) := \frac{h^1\left(X, \, \mathbb{Z} \right)}{2}, \qquad h^1 := \mathrm{dim} \, H^1\left(X, \, \mathbb{Z}\right). \end{equation*}
Recall that, equivalently, we have
\begin{equation*}g_{top}(X) := \frac{\chi_{top}(X) + 2}{2}, \end{equation*}
where $\chi_{top}(X)$ is the topological \textit{Euler characteristic}.
\end{remark}

\section{First Examples Riemann Surfaces}

Recall that a Riemann surface $X$ is a Hausdorff, second-countable, connected manifold endowed with a complex structure.

\paragraph{Riemann Sphere $C_\infty$.}\index{Riemann Surface!Riemann Sphere} Let $S^2 := \{ (x_1, \, x_2, \, x_3) \in \R^3 \: : \: x_1^2 + x_2^2 + x_3^2 = 1 \}$ be the two-dimensional sphere, and let us consider the atlas given by the stereographic projections:
\begin{equation*} \begin{aligned} & \varphi_0 : U_0 := S^2 - \{N\} \to \C, \qquad \varphi_0(x_1, \, x_2, \, x_3) := \frac{x_1}{1 - x_3} + \imath \, \frac{x_2}{1-x_3},\\[1em] & \varphi_1 : U_1 := S^2 - \{S\} \to \C, \qquad \varphi_1(x_1, \, x_2, \, x_3) := \frac{x_1}{1 + x_3} + \imath \, \frac{x_2}{1+x_3}. \end{aligned} \end{equation*}
If we set $\mathcal{A}^\prime := \left\{ (U_0, \, \varphi_0), \, (U_1, \, \varphi_1) \right\}$, then one can easily check that it is not a holomorphic atlas since the transition map
\begin{equation*}\varphi_{0, \, 1} = \varphi_1 \circ \varphi_0^{-1}(x, \, y) = \frac{x}{x^2 + y^2} + \imath \, \frac{y}{x^2 + y^2} \end{equation*}
does not satisfy the Riemann-Cauchy equations\footnote{A function $f = u + \imath \, v : A \subseteq \C \longrightarrow \C$ is holomorphic if and only if
\begin{equation*}\partial_x \, u = \partial_y \, v \qquad \text{and} \qquad \partial_y \, u = - \partial_x \ v. \end{equation*}}. Therefore, we try to modify one of the charts above, e.g. we take the conjugate of $\varphi_1$:
\begin{equation*} \overline{\varphi_1} : U_1 := S^2 - \{S\} \to \C, \qquad \varphi_1(x_1, \, x_2, \, x_3) := \frac{x_1}{1 + x_3} - \imath \, \frac{x_2}{1+x_3}. \end{equation*}
The transition map becomes
\begin{equation*}\varphi_{0, \, 1} = \overline{\varphi_1} \circ \varphi_0^{-1}(x, \, y) = \frac{x}{x^2 + y^2} - \imath \, \frac{y}{x^2 + y^2} = \frac{1}{z}, \end{equation*}
and it is an easy exercise to check that it is holomorphic, that is,
\begin{equation*}\mathcal{A} := \left\{ (U_0, \, \varphi_0), \, (U_1, \, \overline{\varphi_1}) \right\}\end{equation*}
is a holomorphic atlas of $C_\infty$. In particular, the atlas $\mathcal{A}$ induces a holomorphic structure on the Riemann sphere $C_\infty$ and, as a consequence, it turns out that $\C_\infty$ is biholomorphic to $\p^1(\C)$.

In fact, if we consider the complex projective space with coordinates $[z_0 : z_1]$, together with the atlas
\begin{equation*} U_0 := \left\{ z_0 \neq 0 \right\}, \quad U_1 := \left\{ z_1 \neq 0 \right\}, \end{equation*}
then the coordinate in $U_0$ is $z = \frac{z_1}{z_0}$, the coordinate in $U_1$ is $w = \frac{z_0}{z_1}$ and the transition map above is the change of coordinates from $U_0$ to $U_1$
\begin{equation*} z \in U_0 \mapsto w = \frac{1}{z} \in U_1. \end{equation*}

\paragraph{Complex Tori $\mathbb{T}$.}\index{Riemann Surface!Complex Torus} Let $\Lambda \subset \C$ be a lattice of the form $\mathbb{Z} \, \omega_1 + \mathbb{Z} \, \omega_2$, where $\{\omega_1, \, \omega_2\}$ are linearly independent over $\R$, that is,
\begin{equation*}\tau := \frac{\omega_1}{\omega_2} \in \C \setminus \R. \end{equation*}
The \textit{complex torus} is defined as the quotient space $\mathbb{T} = \faktor{\C}{\Lambda}$ which is topologically homeomorphic to the product $S^1 \times S^1$.

Let $\pi : \C \to \faktor{\C}{\Lambda}$ be the standard projection. The topology on $\T$ is the quotient topology, thus we only need to equip it with a complex structure.

For any point $z \in \C \setminus \Lambda$ there is a neighborhood $U_z \subset \C$ such that $U_z \cap \Lambda = \emptyset$. If we set
\begin{equation*} \delta := \min \left\{ d(\xi_1, \, \xi_2) \: : \: \xi_1, \, \xi_2 \in \Lambda \right\}, \end{equation*}
then, given $z \in \C \setminus \Lambda$ and $p = \pi(z) \in \T$, the key idea is to find a neighborhood of $p$, starting from the image of $U_z$ via the map $\pi$. Indeed, the set
\begin{equation*}\Delta(z, \, \epsilon) := \left\{ \xi \in \C \: : \: |\xi - z| < \epsilon, \, \, \epsilon < \delta \right\}, \end{equation*}
is strictly contained in $U_z$ for a suitable choice of $\epsilon > 0$, thus we can find a local holomorphic structure at each point $p$ by using the covering
\begin{equation*} \mathcal{F} := \left\{ \left( v(p, \, \epsilon), \, \pi^{-1} \, \big|_{v(p, \, \epsilon)} \right) \right\}_{p \in \T}, \end{equation*}
where $v(p, \, \epsilon)$ is homeomorphic to $\Delta(z, \, \epsilon)$ via $\pi$.

For any $p_0, \, p_1 \in \T$ there are charts $\varphi_0 : U_0 := v(p_0, \, \epsilon) \to \Delta(z_0, \, \epsilon)$ and $\varphi_1 : U_1 := v(p_1, \, \epsilon) \to \Delta(z_1, \, \epsilon)$; thus the transition map is given by the composition
\begin{equation*} \varphi_1 \circ \varphi_0^{-1} : \C \to \C. \end{equation*}
If $\varphi_0(U_0)$ and $\varphi_1(U_1)$ belong to the same fundamental parallelogram of $\C \setminus \Lambda$, then there is nothing to prove since we can define the transition map as the identity on the intersection. 

On the other hand, if $\varphi_0(U_0)$ and $\varphi_1(U_1)$ belong to different fundamentals parallelograms, then it turns out that
\begin{equation*} \pi \left( \varphi_1 \circ \varphi_0^{-1} (z) \right) = \pi (z), \qquad \forall z \in \varphi_0(U_0) \cap \varphi_1(U_1). \end{equation*}
Consequently, the function $\eta(z) :=  \varphi_1 \circ \varphi_0^{-1} (z) - z$ is continuous and with values in $\Lambda$, a discrete set, thus it needs to be locally constant. In particular, the transition map is locally given by
\begin{equation*} \varphi_1 \circ \varphi_0^{-1} = z + c,\end{equation*}
which is clearly holomorphic. Therefore the atlas $\mathcal{F}$ induces a complex structure on $\T$.

\begin{remark} If $\T_1$ and $\T_2$ are two complex tori, then we will prove in \hyperref[sec:42]{Section \ref{sec:42}} that $\T_1$ is not necessarily biholomorphic to $\T_2$.

On the other hand, the first example shows that every Riemann surface, whose genus is $g = 0$, is biholomorphic to $\p^1(\C)$. \end{remark}

\begin{figure}[h]
\centering
\includegraphics[width=14cm, height=8cm]{Images/GACNEW.png}
\label{fig:CT}
\caption{Complex Tori}
\end{figure} 