\title{Geometria Algebrica C - Lecture Notes}
\author{Francesco Paolo Maiale}
\date{\today}

\documentclass[a4paper,10 pt, titlepage]{report}

\usepackage{graphicx, wrapfig}
\usepackage{amsfonts}
\usepackage[pass]{geometry}
\usepackage{setspace}
\usepackage{amsthm, amssymb}
\usepackage{amsmath}
\usepackage[english]{babel}
\usepackage{tikz-cd}
\usepackage[utf8]{inputenc}
\usepackage{mathtools}
\usepackage{enumitem}
\usepackage{fancyhdr}
\usepackage{accents}
\usepackage{blindtext}
\usepackage{mathrsfs}
\usepackage{makeidx}         % permette di generare l'indice
\usepackage{pifont}
\usepackage{xparse}
\usepackage{xfrac}
\usepackage{faktor}
\usepackage{hyperref}
\usepackage{xcolor}
\usetikzlibrary{decorations.markings,arrows}

\definecolor{dogwoodrose}{rgb}{0.84, 0.09, 0.41}

\hypersetup{
    colorlinks=true,
    linkcolor=dogwoodrose,
    filecolor=magenta,      
    urlcolor=cyan,
}


\makeindex %DDD

\pagestyle{plain}
\setlength{\topmargin}{0.0in}
\setlength{\headheight}{0.2in}
\setlength{\headsep}{0.2in}
\setlength{\footskip}{0.5in}
\setlength{\footnotesep}{0.15in}
\setlength{\textheight}{8.3in}
\setlength{\textwidth}{5.5in} % 6
\setlength{\oddsidemargin}{0.5in}
\setlength{\evensidemargin}{0.5in}
\setlength{\parindent}{0.2 in}
\setlength{\parskip}{0.1 in}
\setlength{\marginparwidth}{1.2 in}


\newtheorem{theorem}{Theorem}[chapter]
\newtheorem{lemma}[theorem]{Lemma}
\newtheorem{proposition}[theorem]{Proposition}
\newtheorem{corollary}[theorem]{Corollary}
\theoremstyle{definition}
\newtheorem{definition}[theorem]{Definition}

\newtheorem{remark}{Remark}[chapter]
\newtheorem{example}{Example}[chapter]
\newtheorem*{notation}{Notation}
\newtheorem*{scolium}{Scolium}
\newtheorem{exercise}{Exercise}[chapter]

\newcommand{\smallO}[1]{\scriptstyle\mathcal{O}}
\DeclarePairedDelimiter\floor{\lfloor}{\rfloor}
\newcommand*\conj[1]{\overline{#1}}
\newcommand{\R}{\mathbb R}
\newcommand{\C}{\mathbb C}
\newcommand{\N}{\mathbb N}
\newcommand{\Z}{\mathbb Z}
\newcommand{\Q}{\mathbb Q}
\newcommand{\p}{\mathbb P}
\newcommand{\T}{\mathbb T}
\newcommand{\F}{\mathcal F}
\newcommand{\G}{\mathcal G}
\newcommand{\M}{\mathcal M}
\newcommand{\K}{\mathcal K}
\newcommand{\h}{\mathcal H}
\newcommand{\U}{\mathcal U}
\newcommand{\Div}{\mathrm {Div}}
\newcommand{\Pic}{\mathrm {Pic}}
\newcommand{\Jac}{\mathrm {Jac}}
\newcommand{\phiD}{\varphi_{|D|}}
\newcommand*{\double}[2][.1ex]{%
  \mathrel{\vcenter{\offinterlineskip%
  \hbox{$#2$}\vskip#1\hbox{$#2$}}}}
\newcommand*{\doublerightarrow}{\double{\longrightarrow}}

\numberwithin{equation}{chapter}
\numberwithin{figure}{chapter}

\newcommand{\bsquare}{\item[\color{magenta}\ding{110}]} 
\newcommand{\barrow}{\item[\color{blue}\ding{228}]}
\newcommand{\bwarrow}{\item[\color{blue}\ding{227}]}
\newcommand{\bman}{\item[\ding{43}]}

\DeclareRobustCommand\longtwoheadrightarrow
     {\relbar\joinrel\twoheadrightarrow}
     
\DeclareFontFamily{U}{mathx}{\hyphenchar\font45}
\DeclareFontShape{U}{mathx}{m}{n}{
      <5> <6> <7> <8> <9> <10>
      <10.95> <12> <14.4> <17.28> <20.74> <24.88>
      mathx10
      }{}
\DeclareSymbolFont{mathx}{U}{mathx}{m}{n}
\DeclareFontSubstitution{U}{mathx}{m}{n}
\DeclareMathAccent{\widecheck}{0}{mathx}{"71}
\DeclareMathAccent{\wideparen}{0}{mathx}{"75}

\def\cs#1{\texttt{\char`\\#1}}

\DeclareDocumentEnvironment{steps}%
{O{Step}}% If no argument is given the label defaults to 'Step'
{\begin{enumerate}[label=#1 \arabic*]}%
{\end{enumerate}}

\makeatletter% http://tex.stackexchange.com/questions/29517/forcing-new-line-after-item-number-in-enumerate-environment/29518#29518
\def\step{%
   \@ifnextchar[ \@step{\@noitemargtrue\@step[\@itemlabel]}}
\def\@step[#1]{\item[#1]\mbox{}\\\hspace*{\dimexpr-\labelwidth-\labelsep}}
\makeatother
\fancyhf{}
% Put the page number at the right edge of odd pages, and left edge of even pages.
\fancyhead[LO,RE]{\textbf \thepage}
% Custom text at the left edge of odd pages, and right edge of odd pages.
\fancyhead[RO]{ \rightmark}
\fancyhead[LE]{ \leftmark}

% Repeat for \fancyfoot if needed, e.g.
% Some decorative symbol at the centre of both odd and even pages
\fancyfoot[C]{ }

% Set this length to 0pt if you don't want any lines!
\renewcommand{\headrulewidth}{1pt}

% Apply the fancy header style
\pagestyle{fancy}

\begin{document}
\newpage \thispagestyle{empty}

\begin{center}

\begin{spacing}{1.5}
{\huge  \sf Lecture Notes}\\
\vspace*{\fill}
\end{spacing}
\begin{spacing}{2.5}
\textbf{\huge Algebraic Geometry C}\\[0.5cm]
\vspace*{\fill}
\begin{minipage}{5cm}
\centering {\textit{Course held by}}
\end{minipage}
\hspace*{\fill}
\begin{minipage}{5cm}
\centering {\textit{Notes written by}} \\
\end{minipage}
\end{spacing}

\begin{spacing}{1.3}

\begin{minipage}{5cm}
\centering {\textbf{\large Prof. Marco Franciosi}}
\end{minipage}
\hspace*{\fill}
\begin{minipage}{5cm}
\centering {\textbf{\large Francesco Paolo Maiale}}
\end{minipage}

\vspace*{\fill}

\textnormal{\large Department of Mathematics \\[0.4em] Pisa University \\[0.4em] \today}

\end{spacing}
\end{center}

\newpage \thispagestyle{empty}
\begin{center}
\vspace*{1cm}
\begin{spacing}{2.5}
 {
\textbf{\huge Disclaimer}
}
\end{spacing} \end{center}

\begin{spacing}{1.2}
These notes came out of the \textit{Algebraic Geometry C} course, held by Professor Marco Franciosi in the second semester of the academic year 2016/2017.

They include all the topics that were discussed in class; I added some remarks, simple proof, etc.. for my convenience.

I have used them to study for the exam; hence they have been reviewed patiently and carefully. Unfortunately, there may still be many mistakes and oversights; to report them, send me an email at \textbf{francescopaolo (dot) maiale (at) gmail (dot) com}.
\end{spacing}


{ \setlength{\parskip}{0.05 in}

\clearpage                       % Otherwise \pagestyle affects the previous page.
{                                % Enclosed in braces so that re-definition is temporary.
  \pagestyle{empty}              % Removes numbers from middle pages.
  \fancypagestyle{plain}         % Re-definition removes numbers from first page.
  {
    \fancyhf{}%                       % Clear all header and footer fields.
    \renewcommand{\headrulewidth}{0pt}% Clear rules (remove these two lines if not desired).
    \renewcommand{\footrulewidth}{0pt}%
  }
  \tableofcontents
  \thispagestyle{empty}          % Removes numbers from last page.
}}

\part{Riemann Surfaces}
\chapter{Introduction}

In this chapter, we introduce the main topics of the course and give a brief overview of what we will see and what we will be able to prove by the end of the course.

\section{Plateau's Problem}

The primary goal and the motivating example of this course is the \textbf{Plateau's problem}, that is, the problem to find the $d$-dimensional surface $\Sigma$ of the minimal area with prescribed $(d-1)$-dimensional boundary $\Gamma$.

By the end, we will be able to prove that a solution indeed exists, but we will not find it explicitly since it is a $NP$ (hard) numerical problem.

As of now, the problem is not well defined. In fact, the notions of \textit{surface}, \textit{area}, and \textit{boundary} make sense in the smooth setting but, as the examples below show, we need to work in a less regular setting.

More precisely, requiring the surface to be smooth is not enough for modeling reasons (e.g., dip a wire frame into a soap solution, form a soap film, and look for the minimal surface whose boundary is the wire frame), and also for existence reasons.

\begin{example} Here we give a list of Plateau's problems with prescribed boundary conditions, and we write down the correct solutions, without proving anything. \mbox{}
\begin{enumerate}[label=\textbf{(\alph*)}]
\item Let us identify $\R^4 \cong \C \times \C$ and, if $d = 2$, let us consider the smooth boundary given by
\begin{equation*} \Gamma_1 := \left( S^1 \times \{0\} \right) \cup \left( \{0\} \times S^1 \right). \end{equation*}
Surprisingly, every minimizing sequence of smooth surfaces converges to a surface which is not smooth at all. Indeed, the solution of the problem is
\begin{equation*} \Sigma_1 := \left( D^2 \times \{0\} \right) \cup \left( \{0\} \times D^2 \right). \end{equation*}
The surface $\Sigma_1$ is clearly singular at the origin, but the singularity may be removed (by factorizing it into two nonsingular surfaces).
\item Let us identify $\R^4 \cong \C \times \C$ and, if $d = 2$, let us consider the smooth boundary given by
\begin{equation*} \Gamma_2 := \left\{ (z^2, \, z^3) \: : \: z \in S^1 \right\}. \end{equation*}
The solution to the Plateau's problem is
\begin{equation*} \Sigma_2 := \left\{ (z^2, \, z^3) \: : \: z \in D^2 \right\}, \end{equation*}
which is a non-smooth surface, whose singularity cannot be removed (since the polynomial $z_1^3 = z_2^2$ cannot be factorized).
\item Let us identify $\R^8 \cong \R^4 \times \R^4$ and, if $d = 7$, let us consider the smooth boundary given by
\begin{equation*} \Gamma_3 := S^3 \times S^3. \end{equation*}
The minimal surface of prescribed boundary $\Gamma_3$ is
\begin{equation*} \Sigma_3 := \left\{ (x_1, \, x_2) \in \R^4 \times \R^4 \: : \: |x_1| = |x_2| \leq 1 \right\}. \end{equation*}
\end{enumerate}
\end{example}

To conclude this introductive chapter, we give a brief overview of the main approaches (studied in this course) to the Plateau's problem, as $d$ ranges between $1$ and $\infty$.

\section{Geodesics problem ($d=1$)}

The geodesics problem (that is, find the shortest curve connecting two points) is, surprisingly, still an open in the non-Riemannian setting. However, in the Riemannian setting, the geodesics problem is completely solved.

Indeed, if we consider the curves parametrized by paths, the \textit{length} is a well-defined notion, and the associated functional is lower semi-continuous and coercive; hence the compactness is easy to prove.

There are many possible approaches to the geodesics problem, e.g., the Steiner approach and the set theoretical approach, which we describe briefly in the remainder of the section.

\paragraph{Steiner Problem.} It is also called networks approach, and it is used to prove the existence of the geodesics and find the explicit expression for it. The reader may consult \cite{steinerpro} for a detailed dissertation on the topic.

\paragraph{Set Theoretical Approach.} The main idea is to find a closed and connected set $\Sigma$ of minimum \textit{length}, containing a given finite set $\Gamma$. As we shall see later in the course, in this case the length is a well defined concept: the \textit{Hausdorff distance}.

In fact, if $X$ is a suitable space (metric, endowed with Hausdorff distance, etc...), then the class defined by
\begin{equation*}\mathcal{X} := \left\{ K \subseteq X \: : \: K \, \, \text{compact and connected} \right\} \end{equation*}
is compact and, by Gotab theorem\footnote{\cite{falconer} Let $\mathscr{C}$ be an infinite collection of non-empty compact sets all lying in a bounded portion $B$ of $\R^n$. Then there exists a sequence $\{E_j\}$ of distinct sets of $\mathfrak{C}$ convergent in the Hausdorff metric to a non-empty compact set $E$.}, $\mathcal{H}^1$ is lower semi-continuous on $X$. 

\section{"Surface" Problem ($d>1$)}

\paragraph{3.} The Plateau's problem is much harder when $d = 2$, but there are still many approaches possible some of which relying, in a certain sense, on the work already done in the geodesics case.

\paragraph{Set Theoretical Approach.} This approach is highly nontrivial. For example, one may ask what does it mean that a compact set $\Sigma$ spans a boundary $\Gamma$? Moreover, there is another problem one should deal with: the $2$-dimensional Hausdorff measure $\mathcal{H}^2$ is, generally, not lower semi-continuous. The reader may consult \cite{Reifenberg1960} for a complete treatise of the topic.

\begin{remark}Suppose that $d = 2$, $n = 3$ and that $\Sigma$ is a surface with boundary $\Gamma$. If $\gamma$ is another closed curve, linked to $\Gamma$ (by a nonzero linking number), then $\gamma \cap \Sigma \neq \emptyset$. \end{remark}

\paragraph{Parametric Approach.} This method is essentially due to Douglas \cite{douglas}. The main idea is the following: since a parametrization $\phi : D^2 \to \R^n$ defines surfaces in $\R^n$, the area functional is well-defined and given by the formula
\begin{equation*}A(\phi) := \int_{D^2} \left| \frac{\partial \, \phi}{\partial \, s_1} \wedge \frac{\partial \, \phi}{\partial \, s_2} \right| \, \mathrm{d}s_1 \, \mathrm{d}s_2. \end{equation*}
On the other hand, the existence through lower semi-continuity and the compactness are a delicate matter, since coercivity is not an easy property to obtain (the integrand is similar to a determinant).

There is a trick which is similar to the one we can use to find geodesics in the differential geometry setting. More precisely, we consider the functional
\begin{equation*}E(\phi) := \frac{1}{2} \, \int_{D^2} \left| \nabla \phi \right|^2 \, \mathrm{d}s_1 \, \mathrm{d}s_2. \end{equation*}
If we find a minimal point $\phi$ for $E$, then $\phi$ will be a \textbf{conformal parametrized} minimum for $A$. This trick, on the other hand, heavily depends on a nontrivial theorem: every such $\Sigma$ admits a conformal re-parametrization.

The lack of conformal parametrization, though, is what stop us from extending the same trick to dimension $d$ strictly bigger than $2$.

\paragraph{Higher Dimension.} If the codimension of $\Sigma$ is equal to $1$ (that is, $n = d+1$), then finite perimeter sets generalize the notion of open $(d+1)$-dimensional sets with smooth boundary in $\R^n$.

The class of finite perimeter sets has excellent compactness properties and a notion of area lower semi-continuous. 

This approach is called "weak" surfaces approach, and it is essentially due to Caccioppoli \cite{cacio} and De Giorgi \cite{degi}. A different approach, working for any $d$ and $n$, referred to as \textit{integral currents}, was introduced by Federer and Flaming in their joint paper \cite{fed}.
\chapter{Introduction to Riemann Surfaces} \thispagestyle{empty}

In this chapter, we introduce the notion of \textit{Riemann surface}, and we profoundly analyze the fundamental example: \textit{smooth algebraic projective curves}.

In the final part, we discuss two examples - both of which will appear many times in the remainder of this course: the Riemann sphere, and the complex torus.

\section{Main Definitions and Basic Properties}

In this section, we give the definition of Riemann surface. The reader should pay attention to the fact that a Riemann surface does not need to be \textit{connected}, but we will ask for it in the definition because we will mostly be dealing with connected compact Riemann surfaces.

\begin{definition}[Riemann Surface] \index{Riemann Surface} Let $X$ be a topological manifold. We say that $X$ is a \textit{Riemann surface} if the following properties are satisfied: \mbox{}
\begin{enumerate}[label=\textbf{(\alph*)}]
\item $\mathrm{dim}_{\mathbb{R}} \, X = 2$.
\item $X$ is Hausdorff, second-countable (i.e., there exists a countable basis) and connected.
\item $X$ has a \index{Complex Atlas}complex structure. Namely, there exists an atlas
\begin{equation*} \mathcal{U} = \left\{ \varphi_i : U_i \to V_i \subseteq \C \right\}_{i \in I}\end{equation*}
such that $\varphi : U_i \to V_i$ is a homeomorphism of open sets, and the transition maps $\varphi_{i, \, j} := \varphi_j \circ \varphi_{i}^{-1}$ are biholomorphic functions. In particular, the map
\begin{equation*}\varphi_{i, \, j} : \varphi_i \left(U_i \cap U_j\right) \subset \C \to \varphi_j \left(U_i \cap U_j\right) \subset \C \end{equation*}
is holomorphic with respect to the complex variable $z$.
\end{enumerate}
\end{definition}

\begin{definition}[Biholomorphic] \index{Biholomorphic functions} Let $f : U \subset \C^n \to V \subset \C^n$ be a complex function. We say that $f$ is \textit{biholomorphic} if $f$ is holomorphic, bijective and its inverse $f^{-1} : V \to U$ is also holomorphic.
\end{definition}

\begin{remark}A Riemann surface $X$ is always orientable. In fact, the Jacobian of the transition maps is always strictly positive (since $\varphi_{i, \, j}$ is holomorphic), hence the atlas is orientated. The interested reader may find a complete proof of this fact \href{https://math.stackexchange.com/a/174365}{here}. \end{remark}

\begin{theorem}[Structure] \index{Structure Theorem} Let $X$ be a compact Riemann surface. Then $X$ is either homeomorphic to a sphere ($g(X) = 0$), a torus ($g(X) = 1$) or a $n$-torus ($g(X) = n$). \end{theorem}

\begin{figure}[h]
\centering
\includegraphics[width=16cm, height=10cm]{Images/GAC001.png}
\label{fig:classth}
\caption{Structure Theorem for Compact Riemann Surfaces.}
\end{figure} 

\section{Projective Curves}

Let $\p^2(\C)$ be the complex projective space of dimension $2$, and let $[z_1 : z_2 : z_0]$ be the coordinates so that $\{ z_0 = 0\}$ is the line at infinity. An algebraic curve is defined as
\begin{equation*} X := \left\{ F(z_1, \, z_2, \, z_0) = 0 \right\}, \end{equation*}
where $F$ is a homogeneous\footnote{For every $\lambda \in \C$, it turns out that $F(\lambda \, z_1, \, \lambda \, z_2, \, \lambda \, z_0) = \lambda^d \, F(z_1, \, z_2, \, z_0)$.} polynomial of degree $d$. Recall that \index{Reduced and Irreducible, Curves} \mbox{}
\begin{enumerate}[label=\textbf{(\alph*)}]
\item $X$ is irreducible $\iff$ $F$ is irreducible;
\item $X$ is reduced $\iff$ $\mathcal{I}(F) = \sqrt{ \mathcal{I}(F)}$,
\end{enumerate}
where $\mathcal{I}(F)$ is the ideal associated to $F$ (in this case, simply $\mathcal{I}(F) = (F)$).

\paragraph{N.B.} From now on we will assume that $X$ is an \textit{irreducible} and \textit{reduced} (projective) algebraic curve.

\begin{definition}[Singular points \index{Singular Points!of a curve}] A point $p \in X$ is \textit{singular} for $X$ if
\begin{equation*}F(p) = \frac{\partial \, F}{\partial \, z_i}(p) = 0, \qquad \forall \, i = 0, \, 1, \, 2. \end{equation*}
\end{definition}

\begin{definition}[Smooth] \index{Algebraic curve!Smoothness} The (projective) algebraic curve $X \subset \p^2(\C)$ is \textit{smooth} if and only if no point $p \in X$ is singular.
\end{definition}

\begin{remark}The projective complex plane $\p^2(\C)$ admits a standard atlas $\mathcal{U} := \left\{ (U_i, \, \varphi_i) \right\}_{i = 0, \, 1, \, 2}$, which is defined by setting
\begin{equation*}U_i := \{ z_i \neq 0\} \qquad \text{and} \qquad \varphi_i : U_i \longrightarrow \C^2, \quad [z_1 : z_2 : z_0] \longmapsto \left( \frac{z_j}{z_i}, \, \frac{z_k}{z_i} \right). \end{equation*}
Then $\mathcal{U}$ induces, by restriction, an atlas on $X$, which is given by
\begin{equation*}\mathcal{X} := \left\{ (U_i \cap X, \, \varphi_i \, \big|_{X}) \right\}_{i = 0, \, 1, \, 2}. \end{equation*} \end{remark}

\begin{proposition}The projective algebraic curve $X$ is smooth if and only if the affine algebraic curve $X_i := X \cap U_i \subset U_i \cong \C^2$ is a smooth for every $i = 0, \, 1, \, 2$. \end{proposition}

\begin{theorem}[Implicit function] \label{implfunc}Let $F \in \C[z_1, \, z_2]$ be any polynomial, and denote by $X := \{ F = 0 \} \subset \C^2$ the associated algebraic variety.

Let $p \in X$ be a point such that $\partial_{z_2} \, F(p) \neq 0$. There are a neighborhood $U_p$ of $p$ and a holomorphic function $G : U \to V$ such that
\begin{equation*} X \cap U = \left\{ (z_1, \, G(z_1)) \, \left| \, z_1 \in V \right. \right\}. \end{equation*}
 \end{theorem}

\begin{corollary}Let $p \in X$ be a smooth point. There exists a neighborhood $U_p$ of $p$ (which is exactly the one given by \hyperref[implfunc]{Theorem \ref{implfunc}}), such that $X$ has a local complex structure, that is,
\begin{equation*} U \cap X = \left\{ (z_1, \, G(z_1)) \right\} \stackrel{\simeq}{\longrightarrow} \left\{ z_1 \, \left| \, z_1 \in V \right. \right\}.\end{equation*}
\end{corollary}

\begin{theorem}\index{Algebraic curve!Classification theorem} Let $X \subset \p^2(\C)$ be a smooth algebraic curve of degree $d$. Then $X$ is a compact Riemann surface, and its genus is equal to
\begin{equation*} g(X) = \frac{(d-1)(d-2)}{2}. \end{equation*} \end{theorem}

\begin{proof}We will prove this result later, but the reader which is already interested in a simple proof of this fact, may jump and take at look at \href{https://math.la.asu.edu/~paupert/degreegenus.pdf}{this paper}. \end{proof}

\subsection{Multiplicities}

\paragraph{Recall.} Let $X \subseteq \p^2(\C)$ be any irreducible and reduced algebraic curve, and let us denote it by
\begin{equation*}X := \{ F(z_1, \, z_2, \, z_0) = 0\},\end{equation*}
where $F$ is an homogeneous irreducible polynomial of degree $d$ such that the associated ideal $(F)$ is radical. Let $p \in X$ be a singular point of $X$, that is, a point where $F$ and all its derivatives vanish:
\begin{equation*}F(p) = \frac{\partial \, F}{\partial \, z_1}(p) = \frac{\partial \, F}{\partial \, z_2}(p) = \frac{\partial \, F}{\partial \, z_0}(p) = 0. \end{equation*}

\begin{definition}[Multiplicity] \index{Singular Points!multiplicities} The \textit{multiplicity} of a point $p$ in $X$ is the least integer $k$ among all the multiplicities of $p$ with respect to the intersection between $X$ and the lines passing through $p$. \end{definition}

\noindent More precisely, we define
\begin{equation*}\mathrm{molt}_p \left(X\right) := \min \left\{ \mathrm{molt}_p \left(r, \, X\right) \: : \: r \, \, \text{line passing through $p$} \right\}. \end{equation*}
It is straightforward to prove that a point $p \in X$ is smooth if and only if its multiplicity $\mathrm{molt}_p \left(X\right)$ is equal to $1$. In particular, any singular point has multiplicity greater or equal than $2$.

\begin{proposition} Assume that $p := (0, \, 0, \, 1)$ belongs to $X$. The dehomogenization\index{Dehomogenization} of $F$ with respect to the coordinate $z_0$ (i.e., in $X_0 = X \cap \mathcal{U}_0$) is given by
\begin{equation*}F(z_1, \, z_2, \, 1) = \sum_{k \geq m} F_k(z_1, \, z_2), \end{equation*}
where the $F_k$ are homogeneous polynomials of degree $k$. Then the multiplicity of $X$ at $p$ is the minimum degree of the dehomogenization, that is,
\begin{equation*}\mathrm{molt}_p \left(X\right) = m. \end{equation*}
 \end{proposition}

\begin{definition}[Ordinary point] \index{Singular Points!ordinary}Let $p \in X$ be any singular point, and suppose that its multiplicity is equal to $m$. We say that $p$ is an \textit{ordinary multiple point} if, locally,
\begin{equation*}F = \prod_{j = 1}^{m} H_j, \end{equation*}
where the $H_j$ are linear forms such that $H_j = H_i \iff i = j$.\end{definition}

\begin{example} \mbox{}
\begin{enumerate}[label=\textbf{(\arabic*)}]
\item An ordinary double point is locally (in a neighborhood of $p:=(0, \, 0, \, 1)$) given by the equation $z_1 \cdot z_2 = 0$ (see e.g. \hyperref[fig:punti]{Figure \ref{fig:punti}}, left).
\item A non-ordinary double point is locally given by an equation of the form $z_1^3 = z_2^2$, and it corresponds to a singular cuspid kind of point (see e.g. \hyperref[fig:punti]{Figure \ref{fig:punti}}, right). \end{enumerate}
\end{example}

\begin{figure}[h]
\centering
\includegraphics[width=16cm, height=8cm]{Images/GAC002.png}
\caption{Examples of singular points: left ordinary, right non-ordinary.}
\label{fig:punti}
\end{figure} 

\subsection{Resolution of the Singularities}
\label{sec:blow}

\paragraph{Introduction.} Let $X$ be a singular projective algebraic curve. The primary goal of this subsection is to give to the reader two methods which are useful to resolve the singularities of $X$, that is, to find a smooth algebraic curve $\widetilde{X}$ and a surjective map 
\begin{equation*}\Phi : \widetilde{X} \longtwoheadrightarrow X. \end{equation*}

\paragraph{Topological Approach.} Let $p \in X$ be an ordinary multiple point, and let $\mathcal{U}$ be a neighborhood of $p$ that does not contain any other singular point of $X$. By definition, the polynomial $F$ is locally (i.e., in $\mathcal{U}$) given by the product of $m$ linear forms:
\begin{equation*}F = \prod_{j = 1}^{m} H_j. \end{equation*}
If we denote by $\Delta$ the Poincaré disk\index{Poincaré disk}, that is,
\begin{equation*} \Delta := \left\{ z \in \C \: \left| \: |z| < 1 \right. \right\} \qquad \text{and} \qquad \Delta^\ast := \Delta \setminus \{0\}, \end{equation*}
then the algebraic curve - without the singular point $p$ - is locally homeomorphic to the union of $m$ copies of $\Delta^\ast$; more precisely, it turns out that
\begin{equation*} \left( \mathcal{U} \cap \left\{ H_j = 0 \right\} \right) \setminus \{p\} \cong \Delta^\ast, \qquad \forall \, j \in \{1, \, \dots, \, m\}. \end{equation*}
Consequently, the algebraic curve $X$ is \textit{locally} homeomorphic to the wedge of $m$ disks ($\Delta$) of center $p$, that is, there is a homeomorphism
\begin{equation*} \mathcal{U} \cap X \cong \bigwedge_{j = 1}^{m} \Delta. \end{equation*}
If we remove the singular point $p$, the wedge is clearly homeomorphic to the disjoint union of the $m$ disks deprived of their centers, i.e.,
\begin{equation*} \left( \mathcal{U} \cap X \right) \setminus \{p\} \cong \bigsqcup_{j = 1}^{m} \Delta^\ast. \end{equation*}
At this point, the resolution of the singularity $p$ is entirely straightforward: we add a center to each disk $\Delta^\ast$. Formally, we define the smooth manifold
\begin{equation*} \widetilde{X} := \left( X \setminus \{p\} \right) \cup \left\{ q_1, \, \dots, \, q_m \right\}, \end{equation*}
and we prove that it is homeomorphic to a disjoint union of balls, i.e.,
\begin{equation*} \widetilde{X} \cong \bigsqcup_{j = 1}^{m} \left( \Delta^\ast \cup \{q_j\}\right) \cong \bigsqcup_{j = 1}^{m} \Delta. \end{equation*}
In the general case, we give a brief sketch of what the ideas behind are. Take any $p \in X$ and any chart which sends $p$ to the origin of $\C$ in such a way that there exists (locally) a function $f:X \to \C$ with the additional property that
\begin{equation*} f : f^{-1} \left(\Delta^\ast\right) \to \Delta^\ast \end{equation*}
is a covering of order $m$.

On the other hand, we know that the connected coverings of $\Delta^\ast \cong S^1$ are all and only of the form $z \mapsto z^m$. Thus we only need to add $m$ points \textit{over} the point $(0, \, 0)$.

\begin{figure}[h]
\centering
\includegraphics[width=18cm, height=10cm]{Images/GAC003.png}
\caption{Idea behind the topological approach.}
\label{fig:ta}
\end{figure} 

\paragraph{Blowup Approach.} \index{Blowup}Let $p = (0, \, 0) \in \C^2$. The blowup of $\C^2$ at the point $p$ is defined as follows:
\begin{equation*} \mathrm{Bl}_{p} \left( \C^2 \right) := \left\{ \left(z_1, \, z_2; \; [a : b]\right) \: \left| \: z_1 \, b = z_2 \, a  \right.\right\} \subseteq \C^2 \times \p^1(\C). \end{equation*}
There exists a map
\begin{equation*} \pi : \mathrm{Bl}_{p} \left( \C^2 \right) \longrightarrow \C^2 \end{equation*}
such that $\pi^{-1}(p) \cong \p^1(\C)$. We denote by $E$ the fiber $\pi^{-1}(p)$, and, from now on, we will refer to it as the \textit{exceptional line}\index{Blowup!Exceptional line} (since it contains the directions of the lines passing through $p$).

Consequently, the complement of $E$ in the blowup is homeomorphic to the complement of the fiber, that is,
\begin{equation*} \mathrm{Bl}_{p} \left( \C^2 \right) \setminus E \cong \pi^{-1} \left( \C^2 \setminus \{(0, \, 0)\} \right).  \end{equation*}

Assume that $p = (0, \, 0) \in X_0$ is a singular point of the affine algebraic curve $X_0 := \left\{ F(z_1, \, z_2) = 0 \right\} \subseteq \C^2$. Then $F$ is a sum of homogeneous polynomials of order $\geq m$, that is,
\begin{equation*}F(z_1, \, z_2, \, 1) = \sum_{k \geq m} F_k(z_1, \, z_2), \end{equation*}
and we may assume, without loss of generality, that $\{z_1 = 0 \}$ is not tangent to $X_0$ (i.e., $a \neq 0$).

If we set $v := b/a$, then we can define the \textbf{strict transform}\index{Strict transform} of $F$ with respect to the coordinates $(z_1, \, v)$ as follows:
\begin{equation} \label{eq:stricttansfor} \widetilde{F}(z_1, \, v) := F(z_1, \, z_1 \cdot v) \cdot z_1^{-m}. \end{equation}
In a neighborhood $\mathcal{U}$ of $p$ the map 
\begin{equation*}\widetilde{X_0} := \left\{ \widetilde{F} = 0 \right\} \implies \widetilde{X_0} \cap \mathcal{U} \longtwoheadrightarrow X_0 \cap \mathcal{U}\end{equation*}
is surjective, and it is straightforward to prove that $u$ is a slope coefficient of a tangent line at $p$ to $X_0$ if and only if it belongs to $E$.

\begin{remark}If $X$ is a singular algebraic curve, then there is no guarantee that $\widetilde{X}$ will be smooth after a single application of the blowup approach.

The next result states that a finite sequence of blowups is enough to obtain a smooth algebraic curve, and in \hyperref[example_3]{Example \ref{example_3}} we describe a case where two steps are necessary.\end{remark}

\begin{theorem}Let $X$ be a singular algebraic curve. There exists a finite sequence of blowups 
\begin{equation*} \widetilde{X}^n \longtwoheadrightarrow \widetilde{X}^{n-1} \longtwoheadrightarrow \dots \longtwoheadrightarrow \widetilde{X}^1 \longtwoheadrightarrow X \end{equation*}
such that $\widetilde{X}^n$ is a compact Riemann surface (thus a smooth algebraic curve). \end{theorem}

\begin{proof}[Idea] The proof of this result is divided into three steps. The reader may try to prove it as an exercise.

\paragraph{Step 1.} There are only finitely many singular points in $X$. \textit{Hint: apply Bezout's theorem applied to the polynomials $(F, \, F^\prime)$, where $F^\prime$ is the usual derivative of $F$}.

\paragraph{Step 2.} Local resolution of the singularities.

\paragraph{Step 3.} For every singular point $p \in X$, the quantity $\mathrm{molt}_p \left(X\right)$ eventually decreases to $0$ as the second step is repeated. \end{proof}

\begin{definition}[Infinitely Near]\index{Singular Points!Infinitley near} Let $p \in X$ be a singular point. A point $q$ is \textit{infinitely near} to $p \in X$, and we denote it by $q \in E(p)$, if $q$ belongs to the exceptional line $E$.\end{definition}

\begin{theorem}\index{Algebraic curve!Resolution theorem}Let $X \subset \p^2(\C)$ be an irreducible and reduced algebraic curve. \mbox{}
\begin{enumerate}[label=\textbf{(\alph*)}]
\item If $X$ is smooth, then $X$ is a compact Riemann surface of genus
\begin{equation*} g(X) = \frac{(d-1)(d-2)}{2}. \end{equation*}
\item If $X$ is singular, then there exist a compact Riemann surface $\widetilde{X}$ and a birational morphism $\pi : \widetilde{X} \longtwoheadrightarrow X$ (that is, an isomorphism outside of singular points) such that
\begin{equation*} g\left(\widetilde{X}\right) = \frac{(d-1)(d-2)}{2} - \sum_{p \in \mathrm{Sing}(X)} \delta_p, \end{equation*}
where $\delta_p$ is equal to
\begin{equation*} \delta_p = \frac{m_p \, (m_p - 1)}{2} + \sum_{q \in E(p)} \frac{m_q \, (m_q - 1)}{2}.  \end{equation*}\end{enumerate}\end{theorem}

\subsection{Examples}

In this brief subsection, we describe the resolution of singularities in very few simple cases, and we come back to the first example of the course.

\begin{example} \label{example_1} Let $X$ be, locally, the algebraic curve defined by the polynomial
\begin{equation*}F(z_1, \, z_2) := z_1^2 - z_2^2 = 0.\end{equation*}
The reader may check that $p = (0, \, 0)$ is singular, and its multiplicity is equal to $2$. If we set $z_2 := v \cdot z_1$, then the strict transform of $F$ is given by
\begin{equation*} \widetilde{F}(z_1, \, v) = \left[ z_1^2 - v^2 \, z_1^2 \right] z_1^{-2} = 1 - v^2. \end{equation*}
Consequently, the intersection $E \cap \{ \widetilde{F}(z_1, \, v) = 0\}$ is made of two points ($v = \pm 1$), corresponding to the singular cross of lines in $X$ (see \hyperref[fig:esdl]{Figure \ref{fig:esdl}}, left).  \end{example}

\begin{example} \label{example_2}Let $X$ be, locally, the algebraic curve defined by the polynomial
\begin{equation*}F(z_1, \, z_2) := z_1^3 - z_2^2 = 0. \end{equation*}
The point $p = (0, \, 0)$ is singular, and its multiplicity is also equal to $2$. If we set $z_2 := v \cdot z_1$, then the strict transform of $F$ is given by
\begin{equation*} \widetilde{F}(z_1, \, v) = \left[ z_1^3 - v^2 \, z_1^2 \right] \, z_1^{-2} = z_1 - v^2. \end{equation*}
Clearly the point $q \in E(p)$ is smooth, but it is tangent to the exceptional line $E$, thus the contribution of $\delta_p$ is nonzero (see \hyperref[fig:esdl]{Figure \ref{fig:esdl}}, center).  \end{example}

\begin{example} \label{example_3} Let $X$ be, locally, the algebraic curve defined by the polynomial
\begin{equation*}F(z_1, \, z_2) := z_1^4 - z_2^2 = 0.\end{equation*}
The point $p = (0, \, 0)$ is singular, and its multiplicity is also $2$. If we set $z_2 := v \cdot z_1$, then the strict transform of $F$ is given by
\begin{equation*} \widetilde{F}(z_1, \, v) = \left[ z_1^4 - v^2 \, z_1^2 \right] \, z_1^{-2} = z_1^2 - v^2. \end{equation*}
The algebraic curve $\widetilde{X}^1$ is singular and, actually, it is the same algebraic curve of \hyperref[example_1]{Example \ref{example_1}}. If we set $z_1 := h \cdot v$, then the second strict transform of $F$ is
\begin{equation*} \widetilde{F}^{(2)}(h, \, v) = 1 - h^2, \end{equation*}
thus $\widetilde{X}^2 \longtwoheadrightarrow \widetilde{X}^1 \longtwoheadrightarrow X$ is the resolution of singularities $X$ (see \hyperref[fig:esdl]{Figure \ref{fig:esdl}}, right).\end{example}

\begin{figure}[h]
\centering
\includegraphics[width=16cm, height=10cm]{Images/GCBUE.png}
\caption{From left to right: Examples \ref{example_1}, \ref{example_2} and \ref{example_3}}
\label{fig:esdl}
\end{figure} 

\begin{example}[Global] Let $\underline{X} \subset \C^2$ be the affine curve defined by the equation
\begin{equation*}z_2^2 - (z_1^2 - 1)(z_1^2 - 4) = 0.\end{equation*}
Its projective closure is obtained via homogenization\index{Homogenization} of the equation, that is,
\begin{equation*} X := \left\{ F(z_1, \, z_2, \, z_0 ) = 0 \right\} \subset \p^2(\C), \qquad F(z_1, \, z_2, \, z_0) = (z_1^2 - z_0^2)(z_1^2 - 4 \, z_0^2) - z_0^2 \, z_2^2. \end{equation*}
It is a simple exercise to prove that the only singular point is $p_\infty = (0, \, 1, \, 0)$ and that it belongs to the line at infinity.

Consequently, about $p_\infty$ it makes sense to use the chart with coordinates $(z_1, \, 1, \, z_0)$ in such a way that, at least locally, the algebraic curve is given by the equation
\begin{equation*}X = \left\{ z_0^2 - (z_1^2 - z_0^2)(z_1^2 - 4 \, z_0^2) = 0 \right\} \stackrel{\sim p_\infty}{=} \left\{ z_0^2 = (z^\prime)^4 \right\}. \end{equation*}
As a consequence, around $p_\infty$ the situation is similar to the one studied in \hyperref[example_3]{Example \ref{example_3}}. In particular, there is a compact Riemann surface $\widetilde{X}$, whose genus is given by\footnote{Indeed, it is straightforward to prove that $m_{p_\infty} = 2$, $m_{q_\infty} = 2$ and $m_{\widetilde{q_\infty}} = 1$.}
\begin{equation*}g\left(\widetilde{X} \right) = 3 - \delta_{p_\infty} = 1, \end{equation*}
and a surjective map $\widetilde{X} \longtwoheadrightarrow X$. We conclude that $\widetilde{X} \cong \mathbb{T}$, that is, $\widetilde{X}$ is homeomorphic to the complex torus (see \hyperref[ex:13]{Example \ref{ex:13}}).

The result is coherent with the fact that the algebraic curve $X$ may be obtained from a $3$-torus by gluing together the two extremal holes and throttling them in such a way to get a $1$-torus with a weird point $p_\infty$. 
\end{example}

\begin{remark}The genus that we have introduced in this subsection is called \textit{arithmetic genus} of a planar algebraic curve, and it is equal to the leading coefficient of the Hilbert polynomial (associated to the local coordinate ring).

This notion of genus is equivalent to the topological one if $X$ is a smooth algebraic curve, where
\begin{equation*}g_{top} \left(X \right) := \frac{h^1\left(X, \, \mathbb{Z} \right)}{2}, \qquad h^1 := \mathrm{dim} \, H^1\left(X, \, \mathbb{Z}\right). \end{equation*}
Recall that, equivalently, we have
\begin{equation*}g_{top}(X) := \frac{\chi_{top}(X) + 2}{2}, \end{equation*}
where $\chi_{top}(X)$ is the topological \textit{Euler characteristic}.
\end{remark}

\section{First Examples Riemann Surfaces}

Recall that a Riemann surface $X$ is a Hausdorff, second-countable, connected manifold endowed with a complex structure.

\paragraph{Riemann Sphere $C_\infty$.}\index{Riemann Surface!Riemann Sphere} Let $S^2 := \{ (x_1, \, x_2, \, x_3) \in \R^3 \: : \: x_1^2 + x_2^2 + x_3^2 = 1 \}$ be the two-dimensional sphere, and let us consider the atlas given by the stereographic projections:
\begin{equation*} \begin{aligned} & \varphi_0 : U_0 := S^2 - \{N\} \to \C, \qquad \varphi_0(x_1, \, x_2, \, x_3) := \frac{x_1}{1 - x_3} + \imath \, \frac{x_2}{1-x_3},\\[1em] & \varphi_1 : U_1 := S^2 - \{S\} \to \C, \qquad \varphi_1(x_1, \, x_2, \, x_3) := \frac{x_1}{1 + x_3} + \imath \, \frac{x_2}{1+x_3}. \end{aligned} \end{equation*}
If we set $\mathcal{A}^\prime := \left\{ (U_0, \, \varphi_0), \, (U_1, \, \varphi_1) \right\}$, then one can easily check that it is not a holomorphic atlas since the transition map
\begin{equation*}\varphi_{0, \, 1} = \varphi_1 \circ \varphi_0^{-1}(x, \, y) = \frac{x}{x^2 + y^2} + \imath \, \frac{y}{x^2 + y^2} \end{equation*}
does not satisfy the Riemann-Cauchy equations\footnote{A function $f = u + \imath \, v : A \subseteq \C \longrightarrow \C$ is holomorphic if and only if
\begin{equation*}\partial_x \, u = \partial_y \, v \qquad \text{and} \qquad \partial_y \, u = - \partial_x \ v. \end{equation*}}. Therefore, we try to modify one of the charts above, e.g. we take the conjugate of $\varphi_1$:
\begin{equation*} \overline{\varphi_1} : U_1 := S^2 - \{S\} \to \C, \qquad \varphi_1(x_1, \, x_2, \, x_3) := \frac{x_1}{1 + x_3} - \imath \, \frac{x_2}{1+x_3}. \end{equation*}
The transition map becomes
\begin{equation*}\varphi_{0, \, 1} = \overline{\varphi_1} \circ \varphi_0^{-1}(x, \, y) = \frac{x}{x^2 + y^2} - \imath \, \frac{y}{x^2 + y^2} = \frac{1}{z}, \end{equation*}
and it is an easy exercise to check that it is holomorphic, that is,
\begin{equation*}\mathcal{A} := \left\{ (U_0, \, \varphi_0), \, (U_1, \, \overline{\varphi_1}) \right\}\end{equation*}
is a holomorphic atlas of $C_\infty$. In particular, the atlas $\mathcal{A}$ induces a holomorphic structure on the Riemann sphere $C_\infty$ and, as a consequence, it turns out that $\C_\infty$ is biholomorphic to $\p^1(\C)$.

In fact, if we consider the complex projective space with coordinates $[z_0 : z_1]$, together with the atlas
\begin{equation*} U_0 := \left\{ z_0 \neq 0 \right\}, \quad U_1 := \left\{ z_1 \neq 0 \right\}, \end{equation*}
then the coordinate in $U_0$ is $z = \frac{z_1}{z_0}$, the coordinate in $U_1$ is $w = \frac{z_0}{z_1}$ and the transition map above is the change of coordinates from $U_0$ to $U_1$
\begin{equation*} z \in U_0 \mapsto w = \frac{1}{z} \in U_1. \end{equation*}

\paragraph{Complex Tori $\mathbb{T}$.}\index{Riemann Surface!Complex Torus} Let $\Lambda \subset \C$ be a lattice of the form $\mathbb{Z} \, \omega_1 + \mathbb{Z} \, \omega_2$, where $\{\omega_1, \, \omega_2\}$ are linearly independent over $\R$, that is,
\begin{equation*}\tau := \frac{\omega_1}{\omega_2} \in \C \setminus \R. \end{equation*}
The \textit{complex torus} is defined as the quotient space $\mathbb{T} = \faktor{\C}{\Lambda}$ which is topologically homeomorphic to the product $S^1 \times S^1$.

Let $\pi : \C \to \faktor{\C}{\Lambda}$ be the standard projection. The topology on $\T$ is the quotient topology, thus we only need to equip it with a complex structure.

For any point $z \in \C \setminus \Lambda$ there is a neighborhood $U_z \subset \C$ such that $U_z \cap \Lambda = \emptyset$. If we set
\begin{equation*} \delta := \min \left\{ d(\xi_1, \, \xi_2) \: : \: \xi_1, \, \xi_2 \in \Lambda \right\}, \end{equation*}
then, given $z \in \C \setminus \Lambda$ and $p = \pi(z) \in \T$, the key idea is to find a neighborhood of $p$, starting from the image of $U_z$ via the map $\pi$. Indeed, the set
\begin{equation*}\Delta(z, \, \epsilon) := \left\{ \xi \in \C \: : \: |\xi - z| < \epsilon, \, \, \epsilon < \delta \right\}, \end{equation*}
is strictly contained in $U_z$ for a suitable choice of $\epsilon > 0$, thus we can find a local holomorphic structure at each point $p$ by using the covering
\begin{equation*} \mathcal{F} := \left\{ \left( v(p, \, \epsilon), \, \pi^{-1} \, \big|_{v(p, \, \epsilon)} \right) \right\}_{p \in \T}, \end{equation*}
where $v(p, \, \epsilon)$ is homeomorphic to $\Delta(z, \, \epsilon)$ via $\pi$.

For any $p_0, \, p_1 \in \T$ there are charts $\varphi_0 : U_0 := v(p_0, \, \epsilon) \to \Delta(z_0, \, \epsilon)$ and $\varphi_1 : U_1 := v(p_1, \, \epsilon) \to \Delta(z_1, \, \epsilon)$; thus the transition map is given by the composition
\begin{equation*} \varphi_1 \circ \varphi_0^{-1} : \C \to \C. \end{equation*}
If $\varphi_0(U_0)$ and $\varphi_1(U_1)$ belong to the same fundamental parallelogram of $\C \setminus \Lambda$, then there is nothing to prove since we can define the transition map as the identity on the intersection. 

On the other hand, if $\varphi_0(U_0)$ and $\varphi_1(U_1)$ belong to different fundamentals parallelograms, then it turns out that
\begin{equation*} \pi \left( \varphi_1 \circ \varphi_0^{-1} (z) \right) = \pi (z), \qquad \forall z \in \varphi_0(U_0) \cap \varphi_1(U_1). \end{equation*}
Consequently, the function $\eta(z) :=  \varphi_1 \circ \varphi_0^{-1} (z) - z$ is continuous and with values in $\Lambda$, a discrete set, thus it needs to be locally constant. In particular, the transition map is locally given by
\begin{equation*} \varphi_1 \circ \varphi_0^{-1} = z + c,\end{equation*}
which is clearly holomorphic. Therefore the atlas $\mathcal{F}$ induces a complex structure on $\T$.

\begin{remark} If $\T_1$ and $\T_2$ are two complex tori, then we will prove in \hyperref[sec:42]{Section \ref{sec:42}} that $\T_1$ is not necessarily biholomorphic to $\T_2$.

On the other hand, the first example shows that every Riemann surface, whose genus is $g = 0$, is biholomorphic to $\p^1(\C)$. \end{remark}

\begin{figure}[h]
\centering
\includegraphics[width=14cm, height=8cm]{Images/GACNEW.png}
\label{fig:CT}
\caption{Complex Tori}
\end{figure} 
\chapter{Functions and Maps} \thispagestyle{empty}

\section{Functions on Riemann Surfaces}

\begin{definition}[Holomorphic function]\index{Holomorphic!function} Let $X$ be a Riemann surface. A function $f : X \to \C$ is \textit{holomorphic} at $p \in X$ if there exists a chart around $p$
\begin{equation*} \varphi : U_p \xrightarrow{\quad \sim \quad} \Delta \subseteq \C\end{equation*}
such that the composition $f \circ \varphi^{-1} : \Delta \longrightarrow \C$ is holomorphic at $\varphi(p)$ (or, equivalently, if it is holomorphic in an appropriate open subset of $\Delta$ containing $\varphi(p)$).\end{definition}

\begin{example} Let $X = \C_\infty$ be the Riemann sphere and let $p_\infty$ be the point at infinity. The reader may  easily prove as an exercise that
\begin{equation*} \text{$f(z)$ is holomorphic at $p_\infty$} \iff \text{$f \left(\frac{1}{z} \right)$ is holomorphic at $0$}. \end{equation*}
Hence, if $f(z) = p(z)/q(z)$ is a holomorphic function and a quotient of polynomials, then the degree of $p$ needs to be less or equal than the degree of $q$. Notice that this is not a sufficient condition. \end{example}

\begin{example} Let $X \subseteq \p^2(\C)$ be a smooth algebraic curve and let $p = [z_1 : z_2 : z_0] \in X$ such that $z_0 \neq 0$. Then $z_1/z_0$ and $z_2/z_0$ are locally holomorphic functions in $\p^2(\C)$, that is,
\begin{equation*}\frac{z_1}{z_0} \, \big|_{X} \quad \text{and} \quad \frac{z_2}{z_0} \, \big|_{X} \end{equation*}
are holomorphic functions at $p$.\end{example}

\begin{notation} Let $X$ be a Riemann surface and let $U \subseteq X$ be an open subset. We denote the set of all the holomorphic functions from $U$ to $\C$ by
\begin{equation*} \mathcal{O}_X(U) := \left\{ f:U\to \C \: \left| \: f \, \, \text{holomorphic} \right. \right\}. \end{equation*} \end{notation}

\begin{definition}[Singularities Type]\index{Holomorphic!singularity types} Let $f : U \setminus \{p\} \to \C$ be a holomorphic function and let $\varphi : U_p \longrightarrow \Delta$ be a chart around $p$. We say that the singularity at $p$ is \mbox{}
\begin{enumerate}[label=\textbf{(\arabic*)}]
\item \textit{removable} if $\varphi(p)$ is a removable singularity for $f \circ \varphi^{-1} : \Delta^\ast \subset \C \to \C$;
\item a \textit{pole} if $\varphi(p)$ is a pole for $f \circ \varphi^{-1} : \Delta^\ast \subset \C \to \C$;
\item \textit{essential} if $\varphi(p)$ is an essential singularity for $f \circ \varphi^{-1} : \Delta^\ast \subset \C \to \C$.
\end{enumerate} \end{definition}

\begin{definition}[Meromorphic function]\index{Meromorphic!function} Let $X$ be a Riemann surface. The function $f : X \to \C$ is \textit{meromorphic} at $p \in X$ if \mbox{}
\begin{enumerate}[label=\textbf{(\alph*)}]
\item there exists an open neighborhood $U \subseteq X$ of $p$ such that $f$ is holomorphic in $U \setminus \{p\}$;
\item $p$ is either a removable singularity, or a pole.
\end{enumerate} \end{definition}

\begin{proposition}[Characterization] Let $X$ be a Riemann surface, and let $f : X \to \C$ be a meromorphic function. Then $f$ locally is the sum of a Laurent series
\begin{equation*} f(z) = \sum_{n \geq k} c_n \, z^n, \end{equation*}
and vice versa. \end{proposition}

\begin{definition}[Order]\index{Meromorphic!order} Let $f : X \to \C$ be a meromorphic function. The \textit{order} of $f$ at $p \in X$ is the minimum integer $k$ in the Laurent series expansion with nonzero coefficient, that is,
\begin{equation*} \mathrm{ord}_p(f) = \min \left\{k \: : \: c_k \neq 0 \right\}.  \end{equation*} \end{definition}

\begin{remark} Any holomorphic function is also a harmonic function (by Riemann-Cauchy); hence holomorphic functions satisfy the maximum principle. \end{remark}

\begin{theorem}[Maximum Modulus] Let $f : U \subseteq X \to \C$ be a holomorphic function defined on any open connected subset $U$ of $X$. If there exists $p \in U$ such that
\begin{equation*} |f(x)| \leq |f(p)|, \qquad \forall \,  x \in U,\end{equation*}
then $f$ is constant. \end{theorem}

An important consequence of this theorem is that introducing only the holomorphic functions on compact Riemann surfaces would be a great restriction.

\begin{corollary} \label{Liov}Let $f : X \to \C$ be a holomorphic function and let $X$ be a compact Riemann surface. Then the function $f$ is constant. \end{corollary}

\begin{theorem} Let $f : \p^1(\C) \to \C$ be a meromorphic function. Then $f$ is a rational function, that is there exist $p, \, q$ homogeneous polynomial of the same degree such that
\begin{equation*} f(z) = \frac{p(z)}{q(z)}. \end{equation*} \end{theorem}

\begin{proof}Let us set $U_0 := \{ z_0 \neq 0\} \subseteq \p^1(\C)$, and recall that $U_0 \cong \C$ with coordinate $z = z_1/z_0$. Consider the dehomogenization
\begin{equation*} \widetilde{f}(z) := f(z, \, 1), \end{equation*}
and let
\begin{equation*} \{\lambda_j\}_{j \in J} := \left\{ \text{zeros and poles of $\widetilde{f}$ in $U_0$} \right\}\qquad \text{and} \qquad e_j := \mathrm{ord}_{\widetilde{f}}(\lambda_j). \end{equation*}
The function
\begin{equation*} \widetilde{R}(z) := \prod_{j \in J} \left(z - \lambda_j\right)^{e_j} \end{equation*}
comes with the same poles and zeros of $\widetilde{f}$ in $U_0$, so we may extend it to a function $R$ defined on the whole projective space $\p^1(\C)$ as follows:
\begin{equation*} R(z) := z_0^n \, \prod_{j \in J} \left(b_j \, z_1 - a_j \, z_0 \right)^{e_j}, \end{equation*}
where $\lambda_j = [a_j : b_j]$ and $n = - \sum_{j} e_j$. In conclusion, we notice that the function
\begin{equation*} g(z) := \frac{f(z)}{R(z)} \end{equation*}
has no zeros or poles in $U_0$, hence we only need to check the infinity point $p_\infty = [1 : 0]$.

\paragraph{Useful Trick.} If $p_\infty$ is not a pole for $g$, then $g$ has no poles, and it is hence constant. If on the other hand, $g$ has a pole in $p_\infty$, the reciprocal $1/g$ is holomorphic and consequently constant.\end{proof}

\section{Holomorphic Maps Between Riemann Surfaces}

In this section, $X$ and $Y$ will denote Riemann surfaces unless stated otherwise.

\begin{definition}[Holomorphic Map]\index{Holomorphic!map} A mapping $F : X \longrightarrow Y$ is \textit{holomorphic} at $p \in X$ if and only if there exist charts $\varphi : U_p \to \Delta_1$ on $X$ and $\psi : V_{F(p)} \to \Delta_2$ on $Y$ such that the composition $\psi \circ F \circ \varphi^{-1}$ is holomorphic at $\varphi(p)$. \end{definition}

It is particularly useful to visualize the definition above through the following commutative diagram:
\begin{equation*}\begin{tikzcd}
U_p \arrow{r}{F} \arrow{d}{\varphi} & V_{F(p)} \arrow{d}{\psi} \\
\Delta_1 \subseteq \C \arrow{r}{f} & \Delta_2 \subseteq \C
\end{tikzcd}  \end{equation*}

\begin{definition}A mapping $F : X \to Y$ is \textit{holomorphic} if and only if it is holomorphic at each point $p \in X$. \end{definition}

\begin{lemma} Let $F : X \to Y$ be a map between Riemann surfaces. \mbox{}
\begin{enumerate}[label=\textbf{(\alph*)}]
\item The identity $\mathrm{id}_X : X \to X$ is holomorphic.
\item The composition between a holomorphic map and a holomorphic function is a holomorphic function.

More precisely, if $F : X \to Y$ and $g : W \subseteq Y \to \C$ are holomorphic and $W$ is an open subset of $Y$, the composition $g \circ F$ is a holomorphic function on $F^{-1}(W)$.
\item The composition between holomorphic maps is still a holomorphic map.

More precisely, if $F : X \to Y$ and $G : W \subseteq Y \to Z$ are holomorphic maps and $W$ is an open subset of $Y$, the composition $G \circ F$ is a holomorphic map from $F^{-1}(W)$ to $Z$.
\item The composition between a holomorphic map and a meromorphic function is a meromorphic function.

More precisely, let $F : X \to Y$ be a holomorphic map, let $g : W \subseteq Y \to \C$ be a meromorphic function and let $W \subseteq Y$ be an open subset. If $F(X)$ is not contained in the set of poles of $g$, then $g \circ F$ is a meromorphic function on $F^{-1}(W)$.
\item The composition between a holomorphic map and a meromorphic map is a meromorphic map.

More precisely, let $F : X \to Y$ be a holomorphic map, let $G : W \subseteq Y \to \C$ be a meromorphic map and let $W \subseteq Y$ be an open subset. If $F(X)$ is not contained in the set of poles of $G$, then $G \circ F$ is a meromorphic function from $F^{-1}(W)$ to $Z$.
\end{enumerate}\end{lemma}

\begin{proof}These are all trivial facts, thus the proof is left to the reader as a simple exercise. \end{proof}

Let $F : X \to Y$ be a nonconstant holomorphic map between Riemann surfaces. For every subset $W \subseteq Y$, $F$ induces a $\C$-algebra homomorphism
\begin{equation*} F^\ast : \mathcal{O}_Y (W)  \xrightarrow{g \mapsto g \circ F} \mathcal{O}_X \left( F^{-1}(W) \right). \end{equation*}
Similarly, $F$ induces a $\C$-algebra homomorphism between meromorphic functions on $W$ and meromorphic function on $F^{-1}(W)$ via composition:
\begin{equation*} F^\ast : \mathcal{M}_Y (W)  \xrightarrow{g \mapsto g \circ F} \mathcal{M}_X \left( F^{-1}(W) \right). \end{equation*}
If $F: X \to Y$ and $G : Y \to Z$ are holomorphic maps, then it is trivial to prove that the operator $^\ast$ reverses the composition order, i.e.,
\begin{equation*}\left(G \circ F \right)^\ast = F^\ast \circ G^\ast. \end{equation*}

\begin{corollary}\index{Morphism}Riemann surfaces equipped with holomorphic mappings form a \textbf{category}. \end{corollary}

\begin{proposition}[Open Mapping Theorem] \label{prop:omt} Let $F : X \to Y$ be a nonconstant holomorphic map between Riemann surfaces. Then $F$ is open. \end{proposition}

\begin{corollary} Let $F : X \to Y$ be a nonconstant holomorphic map. Assume that \mbox{}
\begin{enumerate}[label=\textbf{(\alph*)}]
\item $X$ is a connected and compact Riemann surface;
\item $Y$ is a connected Riemann surface. 
\end{enumerate}
Then the map $F$ is surjective, and $Y$ is compact. \end{corollary}

\begin{proof}From the \hyperref[prop:omt]{Open Mapping Theorem \ref{prop:omt}} it follows that $F$ is an open map. Consequently the image of $X$, $F(X)$, is open in $Y$.

On the other hand, $X$ is compact and hence $F(X)$ is a compact subset of a Hausdorff space $Y$, which means that $F(X)$ is closed. Finally, since $Y$ is connected and $F$ is nonconstant, we infer that $F(X) = Y$.\end{proof}

\section{Global Properties of Holomorphic Maps}

Let $f$ be a holomorphic function defined on a Riemann surface $X$. The complex plane $\C =: Y$ is a Riemann surface; hence we may always identify $f$ with the holomorphic map $f : X \to Y$.

\paragraph{Meromorphic Map Identification.} Let $f$ be a meromorphic function defined on $X$. By definition, $f$ is holomorphic away from its poles, and thus it assumes as values complex numbers. Therefore, it is natural to define a map $F : X \to \C_\infty$ by setting
\begin{equation} \label{Fgrande} F(x) := \begin{cases} f(x) & \text{if $x$ is not a pole of $f$,} \\ \infty & \text{if $x$ is a pole of $f$}. \end{cases} \end{equation}

\begin{theorem}\label{correspondencemeromorm}There exists a $1-1$ correspondence between
\begin{equation*} \left\{ \begin{gathered} \text{Meromorphic functions $f$} \\ \text{defined on $X$} \end{gathered} \right\} \longleftrightarrow \left\{ \begin{gathered} \text{Holomorphic maps} \\ F : X \to \C_\infty \\ \text{which are not identically $\infty$} \end{gathered} \right\} \end{equation*} \end{theorem}

\begin{proof}[Sketch of the Proof] First, we observe that the function defined by \eqref{Fgrande} is holomorphic at every point of $X$. The proof of this simple fact is left to the reader as an exercise.

Observe also that, since $\C_\infty \cong \p^1(\C)$, holomorphic maps $F : X \to \C_\infty$ are in correspondence with holomorphic maps $F : X \to \p^1(\C)$, hence it suffices to prove that there is a $1$-$1$ correspondence
\begin{equation*} \left\{ \begin{gathered} \text{Meromorphic functions $f$} \\ \text{defined on $X$} \end{gathered} \right\} \longleftrightarrow \left\{ \begin{gathered} \text{Holomorphic maps} \\ F : X \to \p^1(\C) \\ \text{which are not identically $\infty$} \end{gathered} \right\}. \end{equation*} 
Let $p \in X$ be any point. Locally - in a neighborhood $U_p \ni p$ - the function $f$ is the ratio of two holomorphic functions, i.e.
\begin{equation*} f(x) = \frac{g(x)}{h(x)} \qquad \forall \, x \in U_p \subset X. \end{equation*}
The corresponding map to $\p^1(\C)$, in this neighborhood of $p$, is given by
\begin{equation*} U_p \ni x \mapsto [g(x) : h(x)] \in \p^1(\C). \end{equation*}
A meromorphic function is not globally the ratio of holomorphic functions; thus this representation is possible only locally, in a neighborhood of each point.
\end{proof}

\paragraph{Normal Form \cite{miranda}.}\index{Holomorphic!normal form} In this paragraph, we want to briefly introduce the so-called \textit{normal form} of a holomorphic map between Riemann surfaces.

\begin{proposition}\label{nf} Let $F: X \to Y$ be a nonconstant holomorphic map, and let $p \in X$ be a point of the domain. There exists a unique integer $m \geq 1$ which satisfies the following property: for every chart $\psi : U_{F(p)} \subset Y \to \Delta^\prime$ centered\footnote{A chart $\varphi$ is centered at a point $q \in X$ if $\varphi(q) = 0$.} at $F(p)$, there exists a chart $\varphi : U_p \subset X \to \Delta$ centered at $p$ such that
\begin{equation*} \psi \circ F \circ \varphi^{-1}(z) = z^m. \end{equation*}
\end{proposition}

\begin{definition}[Multiplicity]\index{Holomorphic!normal form, multiplicity} The \textit{multiplicity} of $F$ at $p$, denoted by $\mathrm{mult}_p \, F$, is the unique integer $m$ such that there are local coordinates near $p$ and $F(p)$ with $F$ having the form $z \mapsto z^m$. \end{definition}

\begin{figure}[h]
\centering
\includegraphics[width=15cm, height=6cm]{Images/GAC40.png}
\label{fig:multiplicity}
\caption{Idea of Normal Form and Multiplicity}
\end{figure} 

There is an easy way to compute the multiplicity that does not require finding charts realizing the normal form. Take any local coordinates $z$ near $p$ and $w$ near $F(p)$, and let
\begin{equation*} z_0 \longleftrightarrow p \qquad \text{and} \qquad w_0 \longleftrightarrow F(p). \end{equation*}
There exists a holomorphic function $h$ such that $w = h(z)$ in such a way that $w_0 = h(z_0)$; hence the multiplicity $\mathrm{mult}_p \, F$ of $F$ at $p$ is one more than the order of vanishing of the derivative $h^\prime(z_0)$ of $h$ at $z_0$, that is,
\begin{equation*} \mathrm{mult}_p \, F = 1 + \mathrm{ord}_{z_0} \left( \frac{\mathrm{d} \, h}{\mathrm{d} \, z} \right). \end{equation*}
In particular, the multiplicity is the exponent of the lowest strictly positive term of the power series for $h$. Namely, we have that
\begin{equation*} h(z) = h(z_0) + \sum_{i = m}^{\infty} c_i \, (z - z_0)^i \implies \mathrm{mult_p} \, F = \min \left\{ i \in \mathbb{Z} \: \left| \: c_i \neq 0 \right. \right\} . \end{equation*}

\section{The Degree of a Holomorphic Map}

In this section, we introduce the notion of degree of a holomorphic map, and we set the ground for the main result of this chapter: \textit{the Hurwitz formula}.

\begin{theorem}Let $F : X \to Y$ be a nonconstant holomorphic map between connected and compact Riemann surfaces. For each $y \in Y$, the quantity
\begin{equation*} d_y(F) := \sum_{p \in F^{-1}(y)} \mathrm{mult}_p \, F \end{equation*}
is constant, independent of $y \in Y$. \end{theorem}

\begin{proof} The idea of the proof is to show that the map $y \mapsto d_y(F)$ is a locally constant function from $Y$ to $\mathbb{Z}$. Since $Y$ is connected, a locally constant function must be constant.

\paragraph{Step 1.} Let $y \in Y$ and let $F^{-1}(y) = \{ x_1, \, \dots, \, x_n\}$ be the fiber. Set $\mathrm{molt}_{x_j}(F) := m_j$ to be the multiplicity at $x_j$, for each $j = 1, \, \dots, \, n$.

By \hyperref[nf]{Proposition \ref{nf}} (normal form) there are neighborhoods $U_i$ of $x_i$ such that $U_i \cap U_j = \emptyset$ for $i \neq j$, and $F \, \big|_{U_i}$ sends $z_i$ to $w_i = z_i^{m_i}$.

\paragraph{Step 2.} The thesis is equivalent to the existence of a neighborhood $U$ of $y$ with the additional property that, for any $y^\prime \in U$,
\begin{equation*} F^{-1}(y^\prime) \subset \bigcup_{j = 1}^{n} U_j. \end{equation*}
We argue by contradiction. Suppose that there exists a sequence of points $(p_k)_{k \in \mathbb{N}} \subset X$ such that
\begin{equation*}p_k \notin \bigcup_{j = 1}^n U_j,\end{equation*}
but $F(p_k)$ converges to $y$. Since $X$ is compact and $F$ is continuous, there exists a subsequence $(p_{k_h})_{h \in \mathbb{N}}$ such that $p_{k_h} \xrightarrow{h \to + \infty} \bar{x}$ and $F(\bar{x}) = y$.

Hence $\bar{x}$ must be equal to $x_j$ for some $j \in \{1, \, \dots, \, n\}$, but this is absurd since no point of the sequence $(p_k)_{k \in \N}$ lies in the neighborhoods $U_i$ of the $x_i$'s.
 \end{proof}
 
\begin{definition}[Degree]\index{Holomorphic!degree} Let $F : X \to Y$ be a nonconstant holomorphic map between connected and compact Riemann surfaces. The \textit{degree} of $F$, denoted by $\mathrm{deg} \, F$, is the quantity $d_y(F)$ computed at any possible $y \in Y$. \end{definition}
 
\begin{corollary} Let $F : X \to Y$ be a holomorphic map of connected compact Riemann surfaces. The map $F$ is locally biholomorphic to $\psi \circ F \circ \varphi^{-1}$ (sending $z$ to $z$) and $\mathrm{deg} \, F = 1$ if and only if $X \cong Y$. \end{corollary}

\begin{notation}Let $ F :X \to Y$ be a mapping of Riemann surfaces. \mbox{}
\begin{enumerate}[label=\textbf{(\alph*)}]
\item A point $p \in X$ is called a \textit{ramification point}\index{Ramification point} if $\mathrm{molt}_p(F) \geq 2$.
\item A point $q \in Y$ is called a \textit{branch point}\index{Branch point} if it is the image of a ramification point.
\item The \textit{ramification index} in $p \in X$ is defined as $\mathrm{molt}_p(F) - 1$.
\end{enumerate}\end{notation}

\section{Hurwitz's Formula}

\paragraph{Introduction.} Let $X$ be a compact connected Riemann surface of genus $g$. The \index{Euler-Poincaré characteristic!topological} Euler-Poincaré topological characteristic is defined as
\begin{equation*} \chi_{\mathrm{top}} := b_0 - b_1 + b_2, \end{equation*}
where $b_i := \mathrm{dim}\left(H_i(X, \, \R) \right) = \mathrm{rank}\left(H_i(X, \, \mathbb{Z}) \right)$ is the $i$-th \index{Betti's number}Betti's number. In a similar fashion, if $X$ is a manifold, then
\begin{equation*} \mathrm{dim}_\R (X) = n \implies \chi_{\mathrm{top}} = \sum_{j = 0}^{n} (-1)^j \cdot b_j(X). \end{equation*}

\begin{lemma} Let $X$ be a compact connected Riemann surface of genus $g(X)$. Then
\begin{equation*} b_0 = b_2 = \symbol{35} \, \text{connected components} = 1, \end{equation*}
while $H_1\left(X, \, \mathbb{Z} \right) = \mathrm{Ab} \left(\pi_1(X) \right)$. In particular, the following identity holds:
\begin{equation} \label{genusidentity} \chi_{\mathrm{top}} = 2 - 2 \, g(X).\end{equation}
\end{lemma}
 
\begin{proposition}The Euler-Poincaré characteristic does not depend on the triangulation of $X$, that is,
\begin{equation*}\chi_{\mathrm{top}} = v - e + f, \end{equation*}
where $v$ is the number of vertexes, $e$ is the number of edges and $f$ is the number of faces. \end{proposition}

\begin{proof} A sketch of the argument may be found in \cite[Page 51]{miranda}.  \end{proof}

\begin{theorem}[Hurwitz's Formula]\index{Hurwitz's Formula} \label{th:hf}Let $F  :X \to Y$ be a nonconstant holomorphic map between compact Riemann surfaces of genus $g(X)$ and $g(Y)$ respectively. Then
\begin{equation} \label{eq:hur} 2 \, \left(g(X) - 1 \right) = 2\, \mathrm{deg} \, F \cdot \left( g(Y) - 1 \right) + \sum_{p \in X} \left[ \mathrm{mult}_p \, F - 1 \right] \end{equation} \end{theorem}

\begin{proof} The Riemann surface $X$ is compact, thus the set of ramification point is finite and the sum on the right-hand side is finite.

\paragraph{Step 1.} Let us take any triangulation $\tau$ of $Y$, such that each branch point of $F$ is a vertex. Denote by $v$ the number of vertexes, $e$ the number of edges and $t$ the number of triangles (faces). 

Assume that, if $q \in Y$ is a branch point and $T \ni q$ a triangle, then $T$ is contained in a neighborhood $U_q$ of $q$ such that 
\begin{equation*}F : \bigsqcup_{j = 1}^{m_q} U_j \to U_q \end{equation*}
is in normal form. Lift this triangulation to $X$ via the map $F$, i.e. $\tau^\prime = F^{-1}(\tau)$, and notice that any ramification point is a vertex of a triangle.

\paragraph{Step 2.} Since there are no ramification point over the general point of any triangle, each one lifts to $\mathrm{deg}(F)$ triangles in $X$. Let $q \in Y$ be any point; then
\begin{equation*} \left| F^{-1}(q) \right| = \sum_{p \in F^{-1}(q)} 1 = \mathrm{deg} \, F + \sum_{p \in F^{-1}(q)} \left[1 - \mathrm{mult}_p \, F \right]. \end{equation*}
The number of edges of $\tau^\prime$ is $e^\prime = \mathrm{deg} \, F \cdot e$, the number of triangles is $t^\prime = \mathrm{deg} \, F \cdot t$ and the number of vertexes is
\begin{equation*}\begin{aligned} v^\prime &  = \sum_{q \in v(Y)} \left[ \mathrm{deg} \, F + \sum_{p \in F^{-1}(q)} \left(1 - \mathrm{mult}_p \, F \right) \right] = \\[1em] &= \mathrm{deg}\, F \cdot v - \sum_{q \in v(Y)} \, \sum_{p \in F^{-1}(q)} \left[ \mathrm{mult}_p \, F - 1 \right]=  \\[1em] & = \mathrm{deg}\, F \cdot v - \sum_{p \in v(X)} \left[ \mathrm{mult}_p \, F - 1 \right]. \end{aligned}\end{equation*}
Therefore we have that
\begin{equation*} \begin{aligned} 2 \, g(X) - 2 & = - v^\prime + e^\prime - t^\prime = \\[1em] & = - \mathrm{deg}\, F \cdot v + \sum_{p \in v(X)} [\mathrm{mult}_p \, F - 1] + \mathrm{deg}\, F  \cdot e - \mathrm{deg} \, F \cdot t = \\[1em] & = 2\, \mathrm{deg} \, F \cdot \left( g(Y) - 1 \right) + \sum_{p \in X} \left[ \mathrm{mult}_p \, F - 1 \right], \end{aligned} \end{equation*}
since every ramification point in $X$ is, actually, contained in the set $v(X)$ of vertexes of $X$ by construction.
\end{proof}

\begin{remark} Let $X$ be a compact Riemann surface. Then there are only finitely many points $p \in X$ with multiplicity greater or equal than $2$. \end{remark}

\begin{remark}The Hurwitz's formula \eqref{eq:hur} gives us more information than the value of the genus. In fact, if we divide it by $2$, it turns out that
\begin{equation*} g(X) = \mathrm{deg}\, F \cdot \left( g(Y) - 1 \right) + \frac{1}{2} \, \sum_{p \in X} \left[ \mathrm{mult}_p\, F - 1 \right] + 1, \end{equation*}
and thus\mbox{} 
\begin{enumerate}[label=\textbf{(\alph*)}]
\item $g(X) \geq g(Y)$;
\item the sum $\sum_{p \in X} \left(\mathrm{mult}_p \, F - 1\right)$ is even.
\end{enumerate}  \end{remark}

\begin{example}Let $F : \p^1(\C) \to \p^1(\C)$ be a function \textit{induced} by a homogeneous polynomial $p$ of degree $d$. More precisely, if $z = z_1/z_0$ is the coordinate associated to the chart $U_0 := \left\{ z_0 \neq 0 \right\} \cong \C$, then the restriction of $F$ to $U_0$ is given by
\begin{equation*} \widetilde{F} : U_0 \cong \C \longrightarrow \C, \qquad z \longmapsto p(z). \end{equation*}
We are interested in finding the ramification points and computing the multiplicities (to check the validity of the Hurwitz's formula).

\paragraph{Step 1.} The ramification points, away from infinity, are the discrete set given by
\begin{equation*} \left\{ \text{ramification points of $F$} \right\} \cap U_0 = \left\{ z \in \C \: \left| \: p^\prime(z) = 0 \right. \right\}, \end{equation*}
and hence there are $d - 1$ (not necessarily distinct) ramification points.

\paragraph{Step 2.} On the other hand, at the infinity point $p_\infty$, we simply pass to the second chart $U_1 := \left\{ z_1 \neq 0 \right\} \cong \C$, with coordinate $w = z_0/z_1$, and we notice that
\begin{equation*} F(w) = w^d. \end{equation*}
Therefore, we can infer that
\begin{equation*}  \mathrm{mult}_\infty(F) = d - 1, \end{equation*}
and, recalling that $g(\p^1(\C)) = 0$, the Hurwitz's formula \eqref{eq:hur} yields to
\begin{equation*} - 2 = d \cdot (-2) + R \iff R = 2 \, (d - 1),\end{equation*}
which is coherent with the computation above.
\end{example}
\chapter{More Examples of Riemann Surfaces} \thispagestyle{empty}

In the first part of this chapter, we show a simple application of the Hurwitz's formula: we compute the genus for a smooth algebraic curve $X \subset \p^2(\C)$, finally proving what we have mentioned several times, that is,
\begin{equation*}g(X) = \frac{(d - 1)(d -2)}{2}. \end{equation*}
Next, we study the group of automorphisms for compact Riemann surfaces of genus $0$ and $1$; in the final part, we investigate the action of finite groups $\mathcal{G}$, and we prove an estimate on $|\mathcal{G}|$ which follows from the Hurwitz's formula.

\section{Application of Hurwitz's Formula}

\paragraph{Genus of Algebraic Curves.} Let $X = \{ f(z_1, \, z_2, \, z_0) = 0 \} \subset \p^2(\C)$ be an algebraic curve, defined by a homogeneous polynomial $f$ of degree equal to $d$.

Assume that $X$ is smooth (so that $X$ is a Riemann surface, by Dini's theorem\footnote{See \hyperref[implfunc]{Theorem \ref{implfunc}}.}) and assume also that, up to a change of coordinates, the following properties are satisfied: \mbox{}
\begin{enumerate}[label=\textbf{(\alph*)}]
\item $p = [1 : 0 : 0] \notin X$;
\item $\{z_2 \neq 0\}$ is not tangent at any point of $X$.
\end{enumerate}
Let us consider the projective line $L = \{ z_0 = 0 \} \cong \p^1(\C)$, and let us denote by $\pi : X \to L$ the associated projection. As usual, we can work in the chart $U_0 = \{ z_0 \neq 0\}$ with coordinates $z = z_1/z_0$ and $w = z_2/z_0$ so that
\begin{equation*}\begin{tikzcd}[contains/.style = {draw=none,"\in" description,sloped}]
 & X \cap U_0 \ar[r, "f"] & L \cap U_0 \\
  & (z, \, w)  \ar[u,contains] \ar[r,mapsto] & z \ar[u,contains]  \\
\end{tikzcd} \end{equation*}
First, we observe that by assumption \textbf{(a)}, the degree of the map associated to $f$ is exactly equal to $d$ on the intersection $X \cap \{ z_0 \neq 0 \}$.

On the other hand, at infinity there are no ramifications, and thus $R = R \, \big|_{U_0}$. More precisely, the ramification points of $X \cap U_0$ may be explicitly found as a solution of the system
\begin{equation*} \begin{cases} \widetilde{f}(z, \, w) = 0 \\[0.8em] \frac{\partial \, \widetilde{f}}{\partial \, w}(z, \, w) = 0, \end{cases} \end{equation*}
or, equivalently, of the system
\begin{equation*} \begin{cases} f(z_1, \, z_2, \, z_0) = 0 \\[0.8em] \frac{\partial \, f}{\partial \, z_0}(z_1, \, z_2, \, z_0) = 0. \end{cases} \end{equation*}
Hence by Bezout theorem the sum of the multiplicities of the ramification points is equal to the product of the degrees, that is,
\begin{equation*}R = d \cdot (d-1). \end{equation*}
Finally, from the Hurwitz's formula \eqref{eq:hur}, it turns out that
\begin{equation*} 2 \, g(X) - 2 = -2 \, d + R \implies g(X) = \frac{(d - 1)(d-2)}{2},\end{equation*}
as we suggested many times in the previous chapters.

The reader may consult \cite[pp. 144-145]{miranda} for a different approach to the problem, which still results in a simple application of the Hurwitz's formula.

\section{Automorphism of Riemann Surfaces}
\label{sec:42}
\index{Automorphisms!of genus $0$ surfaces}

\paragraph{Genus $0$ Automorphisms.} Let $X$ be a compact Riemann surface of genus zero. If $f : \p^1(\C) \to \p^1(\C)$ is an automorphism, then its degree is necessarily $\mathrm{deg} \, f = 1$, and thus
\begin{equation*} f(z_0, \, z_1) = (a \, z_1 + b \, z_0, \, c \, z_1+d \,z_0) .\end{equation*}
A necessary condition for $f$ to be an automorphism is that the two linear polynomials have no zero in common, that is,
\begin{equation*} \mathrm{det} \, \begin{pmatrix} a & b \\ c & d \end{pmatrix} = ad - bc \neq 0. \end{equation*}
In the affine setting (e.g. in $U_0 = \{z_0 \neq 0\}$), it turns out that, with respect to the coordinate $z = z_1/z_0$, the mapping is given by
\begin{equation*} \bar{f}: \C \to \C, \qquad z \longmapsto\frac{a\,z + b}{c \, z + d}.\end{equation*}
In particular, the group of automorphism of $\p^1(\C)$ can be completely characterized as
\begin{equation*}\p \, \mathrm{GL}(2; \; \C) = \faktor{\mathrm{GL}(2; \; \C)}{\C^\ast} \cong \mathrm{Aut} \left( \p^1(\C) \right),\end{equation*}
and the isomorphism is explicitly given by
\begin{equation*} \faktor{\mathrm{GL}(2; \; \C)}{\C^\ast} \ni A = \begin{pmatrix} a & b \\ c & d \end{pmatrix} \longmapsto  f(z_0, \, z_1) = (a \, z_1 + b \, z_0, \, c \, z_1+d \,z_0) \in \mathrm{Aut}\left( \p^1(\C) \right).\end{equation*}

\index{Automorphisms!of genus $1$ surfaces}
\paragraph{Genus $1$ Automorphisms.} In this paragraph, we characterize the holomorphic mappings $f$ between complex tori and give a criterion to decide if $f$ is an isomorphism or not.

\begin{proposition} \mbox{}
\begin{enumerate}[label=\textbf{(\arabic*)}]
\item Let $X$ be a compact Riemann surface of genus $g(X) = 1$.

Then $X$ is isomorphic to a complex torus $\faktor{\C}{\Lambda}$, with $\Lambda = \Z \, \omega_1 + \Z \, \omega_2$ lattice generated by $\R$-linearly independent elements $\omega_1$ and $\omega_2$.
\item Let $\omega_1, \, \omega_2 \in \C$ be two elements such that
\begin{equation*} \tau := \frac{\omega_1}{\omega_2} \in \C \setminus \R, \end{equation*}
and denote by $\Lambda$ the associated lattice. Then the quotient $\faktor{\C}{\Lambda}$ is a compact Riemann surface of genus one.
\end{enumerate}
\end{proposition}

We are now ready to prove the main theorem about holomorphic maps between complex tori. In particular, by the end of the section, we shall show that there are non-isomorphic complex tori (and hence non-isomorphic Riemann surfaces of genus one).

\begin{theorem}[\cite{miranda}] \label{theorem:124}Let $X = \faktor{\C}{\Lambda}$ and let $Y = \faktor{\C}{\Gamma}$ be compact Riemann surfaces of genus $1$. A holomorphic map $f : X \to Y$ is induced by a function $G : \C \to \C$ defined by $G : z \mapsto \gamma \, z + \alpha$, where $\alpha$ and $\gamma$ are fixed complex numbers. Moreover, the following properties hold true: \mbox{}
\begin{enumerate}[label=\textbf{(\alph*)}]
\item If $0 \mapsto 0$, then $\alpha = 0$ and $f$ is a group homomorphism.
\item The mapping $f$ is an isomorphism if and only if $\gamma \cdot \Lambda = \Gamma$.
\end{enumerate}
\end{theorem}
 
\begin{proof} By composing $f$ with a suitable translation on $Y$ we may always assume that $f(0) = 0$.

\paragraph{Step 1.} Since $g(X) = g(Y) = 1$, the Hurwitz's formula \eqref{eq:hur} proves that $f$ is an unramified map. In particular, it is a topological covering, and hence so is the composition $f \circ \pi_X : \C \to Y$.

Since the domain is simply connected, this must be isomorphic - as a covering - to the universal covering $p : \C \to Y$. Therefore there is a map $G : \C \to \C$ and a commutative diagram
\begin{equation*}\begin{tikzcd}
\C \arrow{rr}{G} \arrow{dd}{\pi_X} && \C \arrow{dd}{\pi_Y} \\
\\
X \arrow{rr}{f} && Y
\end{tikzcd} \end{equation*}

\paragraph{Step 2.} The map $G$ is induced on the universal coverings by the commutativity of the diagram above and sends $0$ to a lattice point; we may assume in fact that $G(0) = 0$, since composing with translation by a lattice point does not affect the projection map $\pi$. Moreover $f$ is a well-defined map of quotients, thus
\begin{equation*} f \left( X \right) \subseteq Y \implies G(\Lambda) \subseteq \Gamma, \end{equation*}
that is,
\begin{equation*} G(z + \ell) \equiv_{\Gamma} G(z), \qquad \forall \, \ell \in \Lambda. \end{equation*}
Therefore there exists a lattice point $\omega(z, \, \ell) \in \Gamma$ such that
\begin{equation*} \omega(z, \, \ell) = G(z + \ell)- G(z) \in \Gamma. \end{equation*}
But $\Gamma$ is a discrete subset and $\C$ is connected, thus we infer that $\omega(z, \, \ell)$ is locally constant in the variable $z$. In particular, it turns out that
\begin{equation*} \partial_z \, \left[G(z + \ell)- G(z) \right] = 0, \qquad \forall \, \ell \in \Lambda, \end{equation*}
and thus $G^\prime$ is invariant for $\Lambda$ (i.e., up to translations for elements of the lattice). As a consequence, $G^\prime$ is uniquely determined by its values on a fundamental parallelogram $P_{\Lambda}$.

\paragraph{Step 3.} Therefore $G^\prime : \C \to \C$ is a holomorphic function, whose value is determined by $G^\prime \, \big|_{P_{\Lambda}}$, and thus it is bounded (since $P$ is compact). By Liouville's Theorem\footnote{See \hyperref[Liov]{Corollary \ref{Liov}}.}, there exists $\gamma \in \C$ such that $G^\prime(z) = \gamma$ and this concludes the proof of the point \textbf{(a)}.

\paragraph{Step 4.} Finally, if $\gamma \cdot \Lambda = \Gamma$, then $\gamma^{-1} \cdot \Gamma = \Lambda$, and so the map $H(z) = \gamma^{-1} \left(z - \alpha \right)$ induces a holomorphic map from $Y$ to $X$ which is an inverse for $G$.\end{proof}

\begin{remark} The degree of $f$ is also given by the index of $\gamma \cdot \Lambda$ in $\Gamma$, that is,
\begin{equation*} \mathrm{deg}\, f = \left| \faktor{\Gamma}{\gamma \Lambda} \right|. \end{equation*}
In particular, if $f$ is an isomorphism, its degree is equal to $1$ and $\gamma \cdot \Lambda = \Gamma$. \end{remark}

\begin{proposition}Let $X = \faktor{\C}{\Lambda}$ be a compact Riemann surface of genus $1$. The holomorphic map
\begin{equation*} f : X \to X, \qquad z \longmapsto \gamma \cdot z \end{equation*}
is an automorphism of $X$ - sending $0$ to $0$ - if and only if either
\begin{enumerate}[label=\textbf{(\arabic*)}]
\item $\Lambda$ is a squared lattice, and $\gamma$ is a $4^{th}$-root of unity; or
\item $\Lambda$ is a hexagonal lattice, and $\gamma$ is a $6^{th}$-root of unity; or
\item $\Lambda$ is neither squared or hexagonal, and $\gamma = \pm 1$.
\end{enumerate}
\end{proposition}

\begin{proof} By \hyperref[theorem:124]{Theorem \ref{theorem:124}} a necessary condition for $f : X \to X$ to be an automorphism (sending $0$ to $0$) it that $\gamma \cdot \Lambda = \Lambda$, and thus $\|\gamma\| = 1$. If $\gamma = \pm 1$ there is nothing else to prove.

Assume that $\gamma \notin \R$ and let $\ell \in \Lambda \setminus \{0\}$ be an element of minimal length. Then so is $\gamma \cdot \ell$ and it belongs to $\Lambda$. Clearly $\gamma \cdot (\gamma \ell) \in \Lambda$, thus there exists $m, \, n \in \Z$ such that
\begin{equation*}\gamma^2 \, \ell = n \, \ell + m \, \gamma \ell, \end{equation*}
therefore $\gamma$ is a root (of norm equal to $1$) of the polynomial
\begin{equation*}p(\lambda) = \lambda^2 - m \, \lambda + n, \qquad m, \, n \in \Z. \end{equation*}
The proof is now complete since
\begin{equation*} \begin{aligned}&p(\lambda) = \lambda^2 \pm 1 \leadsto \text{$\Lambda$ is a square}, \\[1em] &
p(\lambda) = \lambda^2  \pm \lambda \pm 1 \leadsto \text{$\Lambda$ is a hexagonal}. \end{aligned}\end{equation*}
\end{proof}

\begin{corollary} Let $X$ be a compact Riemann surface of genus $1$. Then $X \cong \faktor{\C}{\Lambda}$, where $\Lambda = \langle 1, \, \tau \rangle$ and $\tau = \xi + \imath \, \eta$ is a complex number such that $\eta > 0$. \end{corollary}

\begin{proof}If $\widetilde{\Lambda} =  \left< \omega_1, \, \omega_2 \right>$, then one can consider the isomorphism of lattices given by
\begin{equation*}G : \widetilde{\Lambda} \longrightarrow \Lambda, \qquad \omega_1 \mapsto 1, \quad \omega_2 \mapsto \frac{\omega_2}{\omega_1}. \end{equation*}
If the imaginary part is not positive, then we may consider the isomorphism of lattices given by
\begin{equation*}H : \widetilde{\Lambda} \longrightarrow \Lambda, \qquad \omega_1 \mapsto \frac{\omega_1}{\omega_2} \quad \omega_2 \mapsto 1.\end{equation*}\end{proof}

\begin{corollary} Let $\Lambda = \left< 1, \, \tau \right>$ and let $\Lambda^\prime = \left< 1, \, \tau^\prime \right>$ be two lattices defined over $\C$. Set $X = \faktor{\C}{\Lambda}$ and $X^\prime = \faktor{\C}{\Lambda^\prime}$. Then $X \cong X^\prime$ if and only if there exists
\begin{equation*} \begin{pmatrix} a & b \\ c & d \end{pmatrix} \in \mathrm{SL}(2, \, \Z), \end{equation*}
such that
 \begin{equation*}\tau = \frac{a + b \, \tau^\prime}{c + d \, \tau^\prime}. \end{equation*}\end{corollary}
 
\begin{proof}By \hyperref[theorem:124]{Theorem \ref{theorem:124}} a sufficient condition for $X$ to be isomorphic to $X^\prime$ is the existence of a complex number $\gamma \in \C$ such that $\gamma \cdot \Lambda = \Lambda^\prime$. 

Equivalently, we only need to prove that there is $\gamma$ such that $\left< \gamma, \,  \gamma \, \tau \right>$ generates the lattice $\Lambda^\prime$. The inclusion $\subseteq$ is satisfied if there are integers $a, \, b, \, c, \, d$ such that 
\begin{equation*} \gamma = c \, \tau^\prime + d, \qquad \gamma \, \tau = a \, \tau^\prime + b. \end{equation*}
Eliminating $\gamma$ from these equations gives a relation between $\tau$ and $\tau^\prime$, that is,
\begin{equation*}\tau = \frac{a + b \, \tau^\prime}{c + d \, \tau^\prime}. \end{equation*}
Finally, for $\gamma$ and $\gamma \, \tau$ to generate $\Lambda^\prime$, the determinant of the matrix (i.e., $ad - bc$) must be equal to $\pm 1$. But it is easy to see that it is exactly equal to $1$, since both $\tau$ and $\tau^\prime$ lie in the upper half-plane.\end{proof}
 
\paragraph{Conclusion.} We proved that the group of automorphisms of a compact Riemann surface $X$ fixing $0$, denoted by $\mathrm{Aut}_0(X)$, is isomorphic to
\begin{equation*} \begin{aligned}& \mathrm{Aut}_0(X) \cong \faktor{\Z}{4} & \text{if $\Lambda$ is square}; \\[1em] & \mathrm{Aut}_0(X) \cong \faktor{\Z}{6} & \text{if $\Lambda$ is hexagonal}; \\[1em]
& \mathrm{Aut}_0(X) \cong \faktor{\Z}{2} & \text{otherwise}. \end{aligned}\end{equation*}

This simple result yields to a surprising fact: the complex torus defined using a square lattice is not isomorphic to a complex torus defined using a hexagonal lattice.

Thus there are non-isomorphic complex tori, i.e., if $g \geq 1$ then there exist surfaces of the same genus which are not isomorphic (this does not happen for $g = 0$ since the only surface of genus zero is the projective space).

\section{Group Actions on Riemann Surfaces}

\paragraph{Finite Group Actions.} In this section, $\G$ will denote a finite group and $X$ a Riemann surface. In the last paragraph, we will briefly talk about the case of $\G$ infinite.

\begin{definition}[Action]\index{Group action} An \textit{action} of a group $\G$ on a set $X$ is a map $\mu : \G \times X \to X$, denoted by $\mu(g, \, p) := g \cdot p$, which satisfies the following properties:
\begin{enumerate}[label=\textbf{(\arabic*)}]
\item $(g \, h) \cdot p = g \cdot (h \cdot p)$ for any $g, \, h \in \G$ and $p \in X$;
\item $e \cdot p = p$ for $p \in X$, where $e \in \G$ is the identity.
\end{enumerate} \end{definition}

\vspace{2mm}
The reader who is already familiar with the basic definitions may skip this paragraph. The \textit{orbit} of a point $p \in X$ is the set $\G \cdot p := \left\{ g \cdot p \: \left| \: g \in \G \right. \right\}$. If $A \subset X$ is any subset, we denote by $\G \cdot A$ the set of orbits of points in $A$, that is,
\begin{equation*}\G \cdot A = \bigcup_{p \in A} \G \cdot p. \end{equation*}
The \textit{stabilizer} of a point $p \in X$ is the set of the elements of the group $\G$ not moving $p$, i.e., $\G_p = \left\{ g \in \G \: \left| \: g \cdot p = p \right. \right\}$.

\begin{theorem}[Class Formula]\index{Group action!class formula} Let $\G$ be a finite group acting on a set $X$. For any $p \in X$ it turns out that
\begin{equation} \left| \G \cdot p \right| \cdot \left| \G_p \right| = \left| \G \right|.\label{classformula}\end{equation}
\end{theorem}

\begin{definition}[Effective Action]\index{Group action!effective} Let $\G$ be a finite group acting on a set $X$. The action is said to be \textit{effective} if the associated kernel is trivial. \end{definition}

More precisely, the kernel $K$ associated to an action is the intersection of all stabilizer subgroups
\begin{equation*} K = \bigcap_{p \in X} \G_p. \end{equation*}
Therefore it is a normal subgroup of $\G$, and thus the quotient group $\faktor{\G}{K}$ acts on $X$ with trivial kernel and identical orbits to the action of $G$. In particular, we can always assume without loss of generality that an action is effective.

\begin{definition}[Holomorphic Action]\index{Group action!holomorphic} Let $\G$ be a finite group acting on a set $X$. The action is said to be \textit{holomorphic} if for every $g \in \G$, the bijection 
\begin{equation*}\phi_g : X \ni p \longmapsto g \cdot p \in X\end{equation*}
 is a holomorphic map from $X$ to itself, i.e., $\phi_g$ belongs to $\mathrm{Aut}(X)$. \end{definition}

\begin{remark}The quotient space associated to an action, denoted by $\faktor{X}{\G}$, is the set of the orbits. Recall that the topology on $\faktor{X}{\G}$ is easily defined via the natural projection, i.e.,
\begin{equation*}\text{$U \subset \faktor{X}{\G}$ is open} \iff \text{$\pi^{-1}(U)$ is open in $X$}.\end{equation*}
Recall also that $\pi$ is an open map when the action is continuous (or, even better, holomorphic). \end{remark}

\paragraph{Stabilizer Subgroups.} In this short paragraph, we list some facts about the stabilizer subgroup of a finite group $\G$, acting holomorphically and effectively on a Riemann surface.

\begin{proposition}Let $\G$ be a finite group acting holomorphically and effectively on a Riemann surface $X$, and let $p \in X$ be a fixed point. \mbox{}
\begin{enumerate}[label=\textbf{(\arabic*)}]
\item The stabilizer subgroup $\G_p$ is a finite cyclic group.
\item If $\G$ is not finite, then $\G_p$ is still a cyclic group if it is finite.
\item The points of $X$ with nontrivial stabilizers are a discrete subset.
\end{enumerate} \end{proposition}

\paragraph{The Quotient Riemann Surface.}\index{Riemann Surface!quotient} In order to put a complex structure on the quotient surface $\faktor{X}{G}$, we must find complex charts.

\begin{proposition}\label{prop:acts}Let $\G$ be a finite group acting holomorphically and effectively on a Riemann surface $X$, and let $p \in X$ be a fixed point. Then there exists an open neighborhood $U$ of $p$ such that: \mbox{}
\begin{enumerate}[label=\textbf{(\alph*)}]
\item $U$ is invariant under the action of the stabilizer subgroup $\G_p$;
\item $U \cap \left(g \cdot U\right) = \emptyset$ for every $g \notin \G_p$;
\item the map $\faktor{U}{\G_p} \to \faktor{X}{\G}$, which sends a point $q \in U$ to its orbit $[q]$, is a homeomorphism onto an open subset of the quotient $\faktor{X}{\G}$;
\item no point of $U$, except $p$, is fixed by an element of $\G_p$.
\end{enumerate}\end{proposition}

\begin{proof}[Sketch of the Proof] Suppose that $\G \setminus \G_p = \{g_1, \, \dots, \, g_n\}$ are the elements of $\G$ not fixing $p$. A Riemann surface is, in particular, Hausdorff, thus, for each $i =1, \, \dots, \, n$, we can find open neighborhoods $V_i$ of $p$ and $W_i$ of $g_i \cdot p$ such that $V_i \cap W_i = \emptyset$.

In particular, $g_i^{-1} \cdot W_i$ is an open neighborhood of $p$ as $i$ ranges in $\{1, \, \dots, \, n\}$. Let us consider $R_i = V_i \cap \left(g_i^{-1} \cdot W_i \right)$, and let us set
\begin{equation*} U := \bigcap_{g \in \G_p} g \cdot R, \quad \text{where $R = \bigcup_{i = 1}^{n} R_i$}. \end{equation*}
This is exactly the sought open neighborhood of $p$, and it is now easy to check that it satisfies the properties \textbf{(a)}-\textbf{(d)}.
\end{proof}

\begin{theorem}Let $\G$ be a finite group acting holomorphically and effectively on a Riemann surface $X$. The quotient $\faktor{X}{\G}$ is a Riemann surface, whose complex charts are given by \hyperref[prop:acts]{Proposition \ref{prop:acts}}.

Moreover, $\pi : X \to \faktor{X}{\G}$ is a holomorphic map, whose degree is equal to $|\G|$, such that $\mathrm{mult}_p(\pi) = |\G_p|$ for any point $p \in X$. \end{theorem}

\paragraph{Ramification of the Quotient Map.} Let $\G$ be a finite group acting holomorphically and effectively on a Riemann surface $X$, and denote by $Y = \faktor{X}{\G}$ the quotient space.

\vspace{1.7mm}
Suppose that $y \in Y$ is a branch point, and let $x_1, \, \dots, \, x_s$ be the points of $X$ lying above $y$, i.e., $\pi^{-1}(y) = \{x_1, \, \dots, \, x_s\}$. Clearly the $x_i$'s are all in the same orbit by definition, thus they all have conjugate stabilizer subgroups, and each one of them is of the same order $r$.

Moreover, the number $s$ of points in this orbit is the index of the stabilizer, and so is equal to $|\G|/r$. This argument proves the following lemma:

\begin{lemma} Let $\G$ be a finite group acting holomorphically and effectively on a Riemann surface $X$, and let $Y = \faktor{X}{\G}$ be the quotient.

For every branch point $y \in Y$, there is an integer $r \geq 2$ such that $\pi^{-1}(y)$ consists of exactly $|\G|/r$ points of $X$, and at each of these preimage points $\pi$ has multiplicity exactly equal to $r$.  \end{lemma}

\begin{corollary}  \label{cor:123} Let $\G$ be a finite group acting holomorphically and effectively on a Riemann surface $X$, and let $Y = \faktor{X}{\G}$ be the quotient. Suppose that there are $k$ branch points $y_1, \, \dots, \, y_k \in Y$ such that, for each $i = 1, \, \dots, \, n$, $\pi$ has multiplicity $r_i$ at the $|G|/r_i$ points above $y_i$. Then
\begin{equation*} \begin{aligned} 2 \, g(X) - 2 & = \left|G\right| \, \left( 2 \, g(Y) - 2 \right) + \sum_{i=1}^{k} \frac{|G|}{r_i} \, (r_i - 1) = \\[1em] & = \left|G\right| \, \left[ 2 \, g(Y) - 2 + \sum_{i=1}^{k} \left( 1 - \frac{1}{r_i} \right) \right]. \end{aligned} \end{equation*}\end{corollary}

\begin{lemma}\label{lemma:231923} Let $r_1, \, \dots, \, r_k$ be given integers such that $r_i \geq 2$ for each $i$. Let
\begin{equation*}R := \sum_{i=1}^{k} \left(1 - \frac{1}{r_i}\right). \end{equation*}
Then it turns out that\mbox{}
\begin{enumerate}[label=\textbf{(\alph*)}]
\item \begin{equation*} R < 2  \iff \left(k, \, \{ r_i \}\right) = \begin{cases} k = 1 & \text{any $r_1$}; \\[0.4em] k = 2 & \text{any $r_1, \, r_2$}; \\ k = 3 & \text{$\{r_i\} = \{2, \, 2, \, r_3\}$; or} \\ k = 3 & \text{$\{r_i\} = \{2, \, 3, \, 3\}, \, \{2, \, 3, \, 4\}$ or $\{2, \, 3, \, 5\}$.} \end{cases} \end{equation*}
\item \begin{equation*} R = 2 \iff \left(k, \, \{ r_i \}\right) = \begin{cases} k = 3 & \text{$\{r_i\} = \{2, \, 3, \, 6\}$, $\{2, \, 4, \, 4\}$ or $\{3, \, 3, \, 3\}$; or} \\ k = 4 & \text{$\{r_i\} = \{2, \, 2, \, 2, \, 2\}$}. \end{cases} \end{equation*}
\item If $R > 2$ then $R \geq 2 + \frac{1}{42}$.
\end{enumerate} \end{lemma}

\paragraph{Hurwitz's Theorems on Automorphism} For compact Riemann surfaces of genus bigger or equal to $2$, \hyperref[cor:123]{Corollary \ref{cor:123}} leads to a bound on the order of the group $\G$ acting holomorphically and effectively.

\begin{theorem}[Hurwitz's Theorem]\index{Hurwitz's Theorem} \label{htf} Let $\G$ be a finite group acting holomorphically and effectively on a compact Riemann surface $X$ of genus $g(X) \geq 2$. Then
\begin{equation*} |\G| \leq 84 \cdot \left(g(X) - 1 \right). \end{equation*} \end{theorem}

\begin{proof} By \hyperref[cor:123]{Corollary \ref{cor:123}} it turns out that
\begin{equation} \label{111} 2\,g(X) - 2 = \left| \mathcal{G} \right| \, \left[ 2\,g \left(\faktor{X}{\G} \right) - 2 + R \right], \end{equation}
where $R$ is defined as in the Lemma above. \mbox{}
\begin{enumerate}[label=\textbf{(\arabic*)}]
\item Suppose that $g \left(\faktor{X}{\G} \right) \geq 1$. If there is no ramification, i.e., $R=0$, then $g \left(\faktor{X}{\G} \right) \geq 2$ (since $g(X) - 2 > 0$), and this implies immediately that
\begin{equation*}|\G| \leq g(X) - 1.\end{equation*}
If the ramification is nonzero, i.e., $R \neq 0$, then $R \geq 1/2$. Therefore $2 \, g \left(\faktor{X}{\G} \right) - 2 + R \geq 1/2$, and from \eqref{111} it follows that
\begin{equation*} |\G| \leq 4 \cdot \left(g(X) - 1 \right).\end{equation*}
\item Assume then that $g\left(\faktor{X}{\G} \right) = 0$. Then \eqref{111} reduces to
\begin{equation*} 2\,g(X) - 2 = \left| \mathcal{G} \right| \, \left[ R - 2 \right], \end{equation*}
which forces $R > 2$. Therefore Lemma \ref{lemma:231923} implies that $R - 2 \geq 1/42$, i.e,
\begin{equation*} \left| \mathcal{G} \right| \leq 84 \cdot (g - 1), \end{equation*}
as claimed.
\end{enumerate}
\end{proof}

In fact, the group of all automorphisms of a compact Riemann surface of genus at least two is a finite group. It implies that for such a Riemann surface, we have
\begin{equation*} \left| \mathrm{Aut}(X) \right| \leq 84 \cdot \left( g(X) - 1 \right), \end{equation*}
since the full group of the automorphisms certainly acts holomorphically and effectively on $X$; we shall prove this later on in the course.

\part{Differential Forms}
\chapter{Differential Forms} \thispagestyle{empty}

The main result of this chapter is the \textit{residue theorem} for compact Riemann surfaces, which will be extremely useful in some of the most significant results of this course (e.g., \textit{Serre duality theorem}).

\section{Holomorphic $1$-forms}

\begin{definition}[Holomorphic $1$-Forms]\index{Holomorphic!$1$-form} Let $V$ be an open subset of $\C$. A \textit{holomorphic $1$-form} (in the coordinate $z$) on $V$ is an expression $\omega$ of the form
\begin{equation*} \omega = f(z) \, \mathrm{d}z, \end{equation*}
where $f : V \to \C$ is a holomorphic function. \end{definition}

Let $\omega_1 = f(z) \, \mathrm{d}z, \, \omega_2 = g(w) \, \mathrm{d}w$ be holomorphic $1$-forms, respectively in the coordinates $z$ and $w$, defined on open subsets $V_1, \, V_2 \subset \C$.

\begin{definition}[Transformation] Let $T : V_2 \to V_1$ be a holomorphic map such that $z = T(w)$. We say that $\omega_1$ \textit{transforms} to $\omega_2$ under $T$ if and only if
\begin{equation*} g(w) = f \left( T(w) \right) \cdot T^\prime(w), \qquad \forall \, w \in V_2. \end{equation*} \end{definition}

\begin{remark}If $T$ is an invertible transformation and $S$ is its inverse, then $\omega_1$ transforms to $\omega_2$ under $T$ if and only if $\omega_2$ transforms to $\omega_1$ under $S$. \end{remark}

\begin{definition}[Meromorphic $1$-Forms]\index{Meromorphic!$1$-form} Let $V$ be an open subset of $\C$. A \textit{meromorphic $1$-form} (in the coordinate $z$) on $V$ is an expression $\omega$ of the form
\begin{equation*} \omega = f(z) \, \mathrm{d}z, \end{equation*}
where $f : V \to \C$ is a meromorphic function. \end{definition}

Let $\omega_1 = f(z) \, \mathrm{d}z, \, \omega_2 = g(w) \, \mathrm{d}w$ be meromorphic $1$-forms, respectively in the coordinates $z$ and $w$, defined on open subsets $V_1, \, V_2 \subset \C$.

\begin{definition}[Transformation] Let $T : V_2 \to V_1$ be a holomorphic map such that $z = T(w)$. We say that $\omega_1$ \textit{transforms} to $\omega_2$ under $T$ if and only if
\begin{equation*} g(w) = f \left( T(w) \right) \cdot T^\prime(w), \qquad \forall \, w \in V_2. \end{equation*}\end{definition}

\paragraph{Differential forms on Riemann surfaces.} In this paragraph, we extend, in a natural way, the definition of holomorphic $1$-forms on Riemann surfaces.

We denote by $X$ a Riemann surface, and we let $\mathcal{A} = \{ \varphi_\alpha : U_\alpha \to V_\alpha \subseteq \C \}_{\alpha \in I}$ be a complex atlas associated to $X$.

\begin{definition}[Holomorphic Form, \cite{miranda}]\index{Holomorphic!$1$-form on Riemann surface} A \textit{holomorphic $1$-form} on $X$ is a collection of holomorphic $1$-forms $\{ \omega_\phi \}$, one for each chart $\phi : U \to V$ in the coordinate of the codomain $V$, such that if two charts have overlapping domains, then the associated holomorphic $1$-form $\omega_{\phi_1}$ transforms to $\omega_{\phi_2}$ under the change of coordinate $T := \phi_1 \circ \phi_2^{-1}$.\end{definition}

On the other hand, to define a holomorphic $1$-form on a Riemann surface one does not need to give a holomorphic $1$-form on each chart, but only the charts of some atlas.

\begin{lemma} Let $\{\omega_\alpha\}$ be a given collection of $1$-forms, one for each chart of the atlas $\mathcal{A}$, which transform to each other on their overlapping domains. Then there exists a unique holomorphic $1$-form on $X$ extending this collection on any of the charts of $X$. \end{lemma}

\begin{definition}[Meromorphic Form, \cite{miranda}]\index{Meromorphic!$1$-form on Riemann surface} A \textit{meromorphic $1$-form} on $X$ is a collection of meromorphic $1$-forms $\{ \omega_\phi \}$, one for each chart $\phi : U \to V$ in the coordinate of the codomain $V$, such that if two charts have overlapping domains, then the associated holomorphic $1$-form $\omega_{\phi_1}$ transforms to $\omega_{\phi_2}$ under the change of coordinate $T := \phi_1 \circ \phi_2^{-1}$.\end{definition}

\paragraph{Order.}\index{Meromorphic!$1$-form order} Let $p \in X$ be a point and let $\omega$ be a meromorphic $1$-form, defined in a neighborhood $U \subset X$ of a point $p$.

Let $z$ be a local coordinate centered at $p$, in such a way that $\omega = f(z) \, \mathrm{d}z$ for some function $f : U \to \C$, meromorphic at the point $z = 0$.

\begin{definition}The \textbf{order} of $\omega$ at $p$, denoted by $\mathrm{ord}_p(\omega)$, is the order of the function $f$ at the origin $z = 0$, i.e.,
\begin{equation*}\mathrm{ord}_p(\omega) = \mathrm{ord}_0(f). \end{equation*} \end{definition}

It is a simple exercise to prove that this definition does not depend on the particular local representation $f$ of $\omega$, nor on the neighborhood $U$ of $p$.

\paragraph{$C^\infty$ Forms.} Let $X$ be a Riemann surface and let $p \in X$ be a point. If we take a chart
\begin{equation*} \varphi : U_p \xrightarrow{\sim} V \subseteq \C \end{equation*}
centered at $p$, with local coordinate $z$, then a straightforward computation yields to the following result:
\begin{equation*} \frac{\partial}{\partial \, z} = \frac{1}{2} \left( \frac{\partial}{\partial \, x} - \imath \, \frac{\partial}{\partial \, y} \right), \qquad  \frac{\partial}{\partial \, \bar{z}} = \frac{1}{2} \left( \frac{\partial}{\partial \, x} + \imath \, \frac{\partial}{\partial \, y} \right). \end{equation*}
In particular, a function $f : V \subset \C \to \C$ is holomorphic at $p$ if and only if
\begin{equation*}\frac{\partial \, f}{\partial \, \bar{z}}(p) = 0. \end{equation*}
Therefore the differential of a $C^\infty$ function $f : U_p \subset X \to \C$ is given by
\begin{equation*} \mathrm{d}f = \frac{\partial \, f}{\partial \, z} \, \mathrm{d}z + \frac{\partial \, f}{\partial \, \bar{z}} \, \mathrm{d}\bar{z}. \end{equation*}
A $1$-form $\omega$ of class $C^\infty$ is an expression of the form
\begin{equation*}\omega = g_1(z, \, \bar{z}) \, \mathrm{d}z + g_2(z, \, \bar{z}) \, \mathrm{d}\bar{z}, \end{equation*}
and it is holomorphic if and only if $g_1$ is a holomorphic function not depending on $\bar{z}$, and $g_2$ is identically equal to $0$, i.e., a holomorphic $1$-form of class $C^\infty$ is an expression of the form
\begin{equation*}\omega = g_1(z) \, \mathrm{d}z. \end{equation*}
In conclusion, if $\omega$ is a $1$-form of class $C^\infty$, then its differential $\mathrm{d}\omega$ is a $2$-form and it is given by the formula
\begin{equation*}\mathrm{d} \omega = \left( \frac{ \partial \, g_2 }{\partial \, z} - \frac{ \partial \, g_1 }{\partial \, \bar{z}} \right) \, \mathrm{d}z \wedge \mathrm{d}\bar{z}.\end{equation*}

\section{Integration of a $1$-form along paths}

\paragraph{Path Integration.}\index{Path integration} Let $X$ be a Riemann surface and let
\begin{equation*} \varphi : U \xrightarrow{\sim} V \subseteq \C\end{equation*}
be any chart. If $\gamma : [a, \, b] \to X$ is a piece-wise differentiable path such that $\gamma \left([a, \, b] \right) \subset U$, then the composition $\varphi \circ \gamma : [a, \, b] \to V \subset \C$ is also a path, sending $t$ to $z(t)$.

If we identify $U \simeq V$, then the $1$-form can be locally written in the form $\omega =  g_1(z, \, \bar{z}) \, \mathrm{d}z + g_2(z, \, \bar{z}) \, \mathrm{d}\bar{z}$ and thus we can define the integral along $\gamma$ as follows:
\begin{equation} \label{eq:pathss} \int_{\gamma} \omega := \int_{a}^{b} \left[ g_1\left(z(t), \, \bar{z}(t) \right) \cdot z^\prime(t) + g_2\left(z(t), \, \bar{z}(t) \right) \cdot \bar{z}^\prime(t) \right] \, \mathrm{d}t. \end{equation}

If $\gamma : [a, \, b] \to X$ is a generic path, then $\gamma \left([a, \, b] \right)$ is a compact set in $X$ and thus there exist a finite number of charts $\varphi_1 : U_1 \to V_1, \, \dots, \, \varphi_n : U_n \to V_n$ such that
\begin{equation*} \gamma([a, \, b]) \subseteq \bigcup_{i = 1}^{n} U_i. \end{equation*}
If we let $\gamma_i := \gamma \, \big|_{U_i}$, we can define the integral of $\omega$ along $\gamma$ as
\begin{equation} \label{eq:pathss2} \int_{\gamma} \omega := \sum_{i=1}^{n} \int_{a_{i-1}}^{a_{i}} \left[ g_{1, \, i}\left(z(t), \, \bar{z}(t) \right) \, z^\prime(t) + g_{2, \, i}\left(z(t), \cdot \bar{z}(t) \right) \cdot \bar{z}^\prime(t) \right] \, \mathrm{d}t. \end{equation}

\begin{figure}[ht]
\centering
\includegraphics[width = 14cm, height = 8cm]{images/GAC051.png}
\label{fig:ex1}
\caption{Covering of the path $\gamma$}
\end{figure} 

\paragraph{Winding Number.}\index{Winding number} Let $\gamma : [a, \, b] \to \C$ be a closed path around the origin. The integral
\begin{equation*} I_\gamma(0) := \frac{1}{2 \, \pi \, \imath} \int_{\gamma} \frac{1}{z} \, \mathrm{d}z \end{equation*}
is called \textit{winding number} of $\gamma$ and, intuitively, it counts the number of complete rotations around the origin.

More precisely, it depends only on the class of homotopy of $\gamma$, and it is easy to prove that, if $\gamma$ is homotopic to $S^1$ counterclockwise oriented, then $I_\gamma(0) = 1$.

\begin{lemma} Let $X$ be a Riemann surface, and let $p \in X$ be any point. If $\omega$ is a meromorphic $1$-form locally defined on a chart $U_p$ around $p$ and $\gamma$ is a simple path, contained in $U_p$, not enclosing any other pole of $f$, then
\begin{equation*} \mathrm{Res}_{p} ( \omega ) = \frac{1}{2 \pi \imath} \int_\gamma \omega. \end{equation*}
\end{lemma}

Recall that the residue of $\omega$ at a certain point $p \in X$ is defined by looking at $\omega$ locally. More precisely, if $\omega = f(z) \, \mathrm{d}z$ in a neighborhood of $p$, then
\begin{equation*} f(z) = \sum_{n \geq m} c_n \, z^n, \end{equation*}
and the residue is exactly the coefficient of $1/z$, i.e., $c_{-1}$.

\begin{theorem}[Stokes]\index{Stokes Theorem} Let $X$ be a Riemann surface and let $D \subset X$ be a triangulable domain, whose border is piece-wise differentiable. If $\omega$ is a $C^\infty$ $1$-form on $X$, then
\begin{equation} \label{stokes} \int_{\partial \, D} \omega = \iint_{D} \mathrm{d} \omega. \end{equation} \end{theorem}

\begin{theorem}[Residues Theorem]\index{Residues Theorem} \label{residui} Let $X$ be a compact Riemann surface and let $\omega$ be a meromorphic $1$-form on $X$. Then the sum of the residues is zero, i.e.,
\begin{equation*} \sum_{p \in X} \mathrm{Res}_p(\omega) = 0. \end{equation*} \end{theorem}

\begin{proof}The set of poles of $\omega$ is a discrete subset of $X$, thus it is finite by compactness of $X$. Assume that $p_1, \, \dots, \, p_n$ are the poles of $\omega$, and let $\gamma_1, \, \dots, \, \gamma_n$ be simple paths enclosing only the corresponding pole $p_i$. Let $D_i$ be closed sets such that $\partial \, D_i = \gamma_i$, and let $D = X \setminus \cup_{i = 1}^{n} D_i$; then
\begin{equation*} \int_{\gamma_i} \omega = 2 \, \pi \, \imath \, \mathrm{Res}_{p_i}(\omega), \qquad \forall \, i = 1, \, \dots, \, n.\end{equation*}
Formally $\partial \, D = - \sum_{i = 1}^{n} \gamma_i$ (see \hyperref[fig:reisdui]{Figure \ref{fig:reisdui}}), therefore
\begin{equation*} 2 \, \pi \, \imath \, \sum_{i = 1}^{n} \mathrm{Res}_{p_i}(\omega) = \sum_{i = 1}^{n} \int_{\gamma_i} \omega = - \int_{\partial \, D} \omega = - \iint_{D} \mathrm{d}\omega = 0,\end{equation*}
since $\omega$ is holomorphic on $D$.\end{proof}

\begin{corollary} Let $X$ be a compact Riemann surface and let $f : X \to \C$ a nonconstant meromorphic function. Then the sum of the orders is zero, i.e.,
\begin{equation*} \sum_{p \in X} \mathrm{ord}_p(f) = 0. \end{equation*} \end{corollary}

\begin{proof} First, we observe that $\mathrm{ord}_p(f) = n$ if and only if - locally - it turns out that
\begin{equation*} f(z) = c_n \, z^n + \mathcal{O}(z^{n+1}). \end{equation*}
Consider the logarithmic differential\index{Logarithmic differential}
\begin{equation*} \omega = \frac{1}{f} \, \mathrm{d}f, \end{equation*}
and notice that the differential of $f$ is (locally) defined by
\begin{equation*} \mathrm{d}f = f^\prime(z) \, \mathrm{d}z = n \, c_n \, z^{n - 1} \, \mathrm{d}z + \dots. \end{equation*}
In particular, multiplying by $1/f$ we find that
\begin{equation*} \frac{1}{f} \, \mathrm{d}f = \left( \frac{n}{z} + \dots \right),\end{equation*}
and thus the key identity
\begin{equation*} \mathrm{ord}_p(f) = \mathrm{Res}_{p} \left( \frac{1}{f} \, \mathrm{d}f \right), \end{equation*}
which concludes the proof since the sum of the residues is zero by \hyperref[residui]{Theorem \ref{residui}}. \end{proof}

\newpage

\begin{figure}[h]
\centering
\includegraphics[width = 14cm, height = 8cm]{images/GC11p.png}
\caption{Idea of the Residues Theorem proof}
\label{fig:reisdui}
\end{figure} 

\part{Sheaf Cohomology}
\chapter{Sheaf Theory} \thispagestyle{empty}

In this chapter, we want to introduce and develop the sheaf language to simplify the comprehension of the next topics.

In particular, we present the cohomology groups (which will ease the discussion of the divisor vector space $L(D)$), and we also prove the long exact sequence in cohomology theorem (which will be used extensively in the next chapters.)

The reader may consult \href{https://www2.bc.edu/dawei-chen/Lecture-2013.pdf}{these notes} for a more in-depth dissertation.

\section{Definitions and First Properties}

\begin{definition}[Sheaf]\index{Sheaf} Let $X$ be a topological space. A \textit{sheaf} $\F$ on $X$ associates to each open set $U \subset X$ an abelian group $\F(U)$, along with a \textit{restriction map} $\rho_{V}^{U} : \F(V) \to \F(U)$ for any open sets $U \subset V$, satisfying the following conditions: \mbox{}
\begin{enumerate}[label=\textbf{(\arabic*)}]
\item \textbf{Compatibility Conditions.} 
\begin{enumerate}[label=\textit{(1.\arabic*)}]
\item  $\F(\emptyset) = 0$.
\item  $\rho_{U}^{U} = \mathrm{id}_U$.
\item If $W \subset V \subset U$, then $\rho_{W}^{U} = \rho_{W}^{V} \circ \rho_{V}^{U}$.
\end{enumerate}
\item \textbf{Locality Conditions.} 
\begin{enumerate}[label=\textit{(2.\arabic*)}]
\item If $\mathcal{U} := \{U_i\}_i$ is a covering of $U$, then
\begin{equation*} \sigma \in \F(U) \: : \: \sigma \, \big|_{U_i} = 0 \quad \forall \, i \implies \sigma \equiv 0. \end{equation*}
\item For any covering $\mathcal{U} := \{U_i\}_{i \in I}$ of $U$ and any collection $\{\sigma_i \}_{i \in I}$ of sections $\sigma_i \in \F(U_i)$ it turns out that, if
\begin{equation*}\sigma_i \, \big|_{U_i \cap U_j} = \sigma_j \, \big|_{U_i \cap U_j}\end{equation*}
for any $i, \, j$, then there exists $\sigma \in \F(U)$ such that $\sigma \, \big|_{U_i} = \sigma_i$.
\end{enumerate}
\end{enumerate}
If $\F$ satisfies only the compatibility conditions, then we say that $\F$ is a \textit{presheaf}. \end{definition}

\begin{definition}[Stalks]\index{Sheaf!stalks} Let $\F$ be a sheaf on a topological space $X$, and let $p \in X$. The \textit{stalk} at $p$, denoted by $\F_p$, is the direct limit of all sections containing $p$. \end{definition}

More precisely, suppose that $U$ and $V$ are two open subsets, both containing $p$, with two section $\sigma_U$ and $\sigma_V$; define an equivalence relation
\begin{equation*}\sigma_U \sim \sigma_V \iff \exists \, W \subset U \cap V \: : \: \sigma_U \, \big|_{W} = \sigma_V \, \big|_{W}. \end{equation*}
Then the stalk at $p$ is defined by setting
\begin{equation*} \F_p := \varinjlim_{U \ni p} \F(U) = \left( \bigsqcup_{U \ni p} \F(U) \right) \, / \, \sim. \end{equation*}
There is a group homomorphism $\rho_U : \F(U) \to \F_p$ mapping a section $\sigma_\alpha$ to its equivalence class, and the image is called the \textit{germ} of $\sigma_\alpha$.

\begin{example} To get accustomed with the definitions, the reader may try to check that the following are all sheaves. \mbox{}
\begin{enumerate}[label=\textbf{(\alph*)}]
\item The (locally) constant sheaf $U \mapsto \underline{\C}(U) = \C$,  denoted by $\underline{\C}$, whose restriction maps are the identities between the copies of $\C$.
\item The sheaf of holomorphic functions $U \mapsto \mathcal{O}_X(U) := \left\{ f : U \to \C \: \left| \: \text{$f$ is holomorphic in $U$} \right. \right\}$. For any $p \in X$ the stalk is given by
\begin{equation*}\mathcal{O}_{X, \, p} := \left\{ f : X \to \C \: \left| \: \text{$f$ is meromorphic outside of $p$} \right. \right\}. \end{equation*}
\item The sheaf of meromorphic functions $U \mapsto \mathcal{M}_X(U) := \left\{ f : U \to \C \: \left| \: \text{$f$ is meromorphic in $U$} \right. \right\}$.
\item The sheaf of the holomorphic differentials $U \mapsto \Omega_X^1(U) := \left\{ f(z) \, \mathrm{d}z \: \left| \: \text{$f$ is holomorphic in $U$} \right. \right\}$.
\end{enumerate}
\end{example}

\paragraph{Morphisms of Sheaves.}\index{Sheaf!morphism} Let $\mathcal{E}$ and $\F$ be sheaves on $X$. A morphism $f : \mathcal{E} \to \F$ is a collection of group homomorphisms $f_U : \mathcal{E}(U) \to \F(U)$ such that the following diagram is commutative
\begin{equation*}\begin{tikzcd}
 \mathcal{E}(U) \arrow{rr}{f_U} \arrow{dd}{\rho_{U}^{V}} &&  \F(U) \arrow{dd}{\rho_{U}^{V}} \\ \\
 \mathcal{E}(V)  \arrow{rr}{f_V} & & \F(V)
\end{tikzcd}  \end{equation*}
that is, $\left(f_U(\sigma) \right) \, \big|_{V} = f_V \left( \sigma \, \big|_V \right)$.

\begin{example}[Inclusion Maps] The first kind of morphisms between sheaves we study are the inclusion maps. Indeed, they come up whenever, for any $U \subset X$, the group $\F(U)$ is a subgroup of the group $\G(U)$. \mbox{}
\begin{enumerate}[label=\textbf{(\arabic*)}]
\item \textit{Constant sheaves}: $\underline{\Z} \subset \underline{\R} \subset \underline{\C}$.
\item \textit{Holomorphic/Meromorphic sheaves}: $\underline{\C} \subset \mathcal{O}_X \subset \mathcal{M}_X$.
\item \textit{Nonzero holomorphic/meromorphic sheaves}: $\mathcal{O}_X^\ast \subset \mathcal{M}_X^\ast$.
\item \textit{Sheaves of functions with bounded poles}: if $D_1 \leq D_2$ are divisors on $X$, then $\mathcal{O}_X[D_1] \subset \mathcal{O}_X[D_2]$.
\end{enumerate}\end{example}

\paragraph{Kernel.}\index{Sheaf!kernel} Suppose that $\phi : \F \to \G$ is sheaf map between two group sheaves on $X$. Define a subsheaf $\K \subset \F$, called the \textit{kernel} of $\phi$, by setting
\begin{equation*} \K(U) := \mathrm{ker}(\phi_U) \end{equation*}
for any $U \subseteq X$, that is, the group associated to the open set $U$ is exactly the kernel of the group homomorphism $\phi_U : \F(U) \to \G(U)$.

\begin{proposition} Let $\phi : \F \to \G$ be a sheaf map between two group sheaves on $X$. \mbox{}
\begin{enumerate}[label=\textbf{(\alph*)}]
\item The kernel $\K$ is a sheaf.
\item The cokernel (which is defined in the same way) is a presheaf, but it is generally not a sheaf.\end{enumerate}\end{proposition}

\paragraph{Associated Sheaf.}\index{Sheaf!associated sheaf} If $\F$ is a presheaf, then it is always possible to extend it to a sheaf $\widetilde{F}$, which is usually called associated sheaf.

For example, we define $\mathrm{coker}(f) (U)$ to be a collection of sections $\sigma_\alpha \in \G(U_\alpha)$ for an open covering $\{U_\alpha\}_{\alpha}$ of $U$, such that for all $\alpha$ and $\beta$ it turns out that
\begin{equation*} \sigma_\alpha \, \big|_{U_\alpha \cap U_\beta} - \sigma_{\beta} \, \big|_{U_\alpha \cap U_\beta} \in f_{U_\alpha \cap U_\beta} \left( \mathcal{F}(U_\alpha \cap U_\beta) \right). \end{equation*}
The definition depends on the choice of an open covering. Thus we need to find a way to get rid of this obstacle.

The idea, as we shall see also later in the course, is to use the direct limit. More precisely, we identify two collections $\left\{ (U_\alpha, \, \sigma_\alpha) \right\}$ and $\left\{ (V_\beta, \, \sigma_\beta) \right\}$ if for all $p \in U_\alpha \cap V_\beta$ there exists an open set $W$ satisfying $p \in W \subset U_\alpha \cap V_\beta$, such that
\begin{equation*} \sigma_\alpha \, \big|_{W} - \sigma_{\beta} \, \big|_{W} \in f_{W} \left( \mathcal{F}(W) \right). \end{equation*}
This identification yields an equivalence relation and correspondingly we define the coker sheaf as the group of equivalence classes of the above sections.

\section{Exact Sequences}

In this section, we introduce a fundamental notion that is used vastly in algebra and geometry: short exact sequences, and the long exact sequence in cohomology theorem.

\paragraph{Short Exact Sequences of Sheaves.}\index{Sheaf!short exact sequence} We say that a sequence of sheaf maps
\begin{equation*} 0 \xrightarrow{} \K \xrightarrow{} \F \xrightarrow{\phi} \G \xrightarrow{} 0\end{equation*}
is a \textit{short exact sequence} if the sheaf map $\phi$ is surjective and $\K$ is the kernel sheaf associated to $\phi$.

\begin{remark}There is an equivalent - and maybe more useful - definition of a short exact sequence of sheaf maps, which relies on the notion of a short exact sequence of abelian groups. More precisely, the sequence of sheaf maps
\begin{equation*} 0 \xrightarrow{} \F \xrightarrow{\phi} \G \xrightarrow{\psi} \mathcal{H} \xrightarrow{} 0\end{equation*}
is an exact short sequence if and only if
\begin{equation*} 0 \xrightarrow{} \F_p \xrightarrow{\phi_p} \G_p \xrightarrow{\psi_p} \mathcal{H}_p \xrightarrow{} 0\end{equation*}
is an exact sequence of abelian groups, for every $p \in X$. \end{remark}

\begin{example} Here is a brief list of short exact sequences. \mbox{}
\begin{enumerate}[label=\textbf{(\alph*)}]
\item On a Riemann surface, the sequence
\begin{equation*} 0 \xrightarrow{} \underline{\C} \xrightarrow{} \mathcal{O}_X \xrightarrow{\mathrm{d} = \partial} \Omega_X^1 \xrightarrow{} 0\end{equation*}
is exact, since the kernel sheaf of the differential map is exactly the (locally) constant sheaf.
\item The sequence
\begin{equation*} 0 \xrightarrow{} \underline{\Z} \xrightarrow{} \mathcal{O}_X \xrightarrow{\mathrm{e}^{2 \pi \, i \, \cdot}} \mathcal{O}_X^\ast \xrightarrow{} 0\end{equation*}
is exact, since the kernel sheaf of the exponential map is exactly the integer-valued (locally) constant sheaf.
\item The sequence
\begin{equation*} 0 \xrightarrow{} \Omega_X^1 \xrightarrow{} \mathcal{E}_X^{1, \, 0} \xrightarrow{\bar{\partial}} \mathcal{E}_X^{2} \xrightarrow{} 0\end{equation*}
is exact.
\item The sequence
\begin{equation*} 0 \xrightarrow{} \underline{\C} \xrightarrow{} C^\infty \xrightarrow{d} \mathrm{ker} \left( d : \mathcal{E}_X^1 \to \mathcal{E}_X^2 \right) \xrightarrow{} 0\end{equation*}
is exact, since the kernel of $d$ in this setting is exactly the constant functions space.
\end{enumerate}
\end{example}

\paragraph{Sheaves on Riemann surfaces.}\index{Sheaf!on Riemann surfaces} Let $X$ be a compact Riemann surface and let $p \in X$. The \textit{skyscraper sheaf} centered at $p$ is defined as
\begin{equation*} \left( \C_p \right)_x = \begin{cases} 0 & \text{if $x \neq p$} \\ \C & \text{if $x = p$}. \end{cases} \end{equation*}
We can also define the sheaf of holomorphic functions such that $p$ is a zero, that is,
\begin{equation*} \mathcal{J}_{X, \, p} = \left\{ \text{$f$ is holomorphic and $f(p) = 0$}   \right\}, \end{equation*}
which can be easily denoted using the divisors. In fact, if we let $f \in \mathcal{J}_p$, then it is easy to prove that $\mathrm{div}(f) \geq p$, and thus we may denote it by
\begin{equation*} \mathcal{J}_{X, \, p} = \mathcal{O}_X[-p]. \end{equation*}

\begin{proposition} Let $X$ be a compact Riemann surface. There exists an exact sequence of sheaf maps\begin{equation*} 0 \xrightarrow{} \mathcal{O}_X[-p] \xrightarrow{} \mathcal{O}_X \xrightarrow{f \mapsto f(p)} \C_p \xrightarrow{} 0. \end{equation*} \end{proposition}

\begin{proof}[Idea of the Proof]Notice that, if $x \neq p$, then
\begin{equation*} (\C_p)_x = 0 \qquad \text{and} \qquad (\mathcal{O}_X[-p])_x \cong \mathcal{O}_{X, \, x} \end{equation*}
and that
\begin{equation*} (\mathcal{O}_X[-p])_p \cong \left\{ \text{maximal ideals in $\mathcal{O}_{X, \, p}$} \right\} \qquad \text{and} \qquad \faktor{\mathcal{O}_{X, \, p}}{\mathcal{M}_X }\cong \C_p. \end{equation*} \end{proof}

\section{Čech Cohomology of Sheaves}

In this section will assume that every covering is \textit{locally finite}. This assumption is by no means necessary at this point, but it will come in handy (and, actually, necessary) soon.

\paragraph{Čech Cochains.}\index{Cech!Cochains} Let $\F$ be a sheaf of abelian groups on a topological space $X$. Let $\mathcal{U} := \{U_i\}_i$ be an open covering of $X$, and fix an integer $n \geq 0$. For every collection of indices $(i_0, \, \dots, \, i_n)$, we denote the intersection of the corresponding open sets by
\begin{equation*} U_{i_0, \, \dots, \, i_n} := U_{i_0} \cap \dots \cap U_{i_n}. \end{equation*}
The deletion of the $k$-th index is denoted by the symbol $\hat{i_k}$, and it is clear by the definition that
\begin{equation*} U_{i_0, \, \dots, \, i_n} \subset  U_{i_0, \, \dots, \widehat{i_k}, \, \dots \, i_n}. \end{equation*}

\begin{definition} A \textit{Čech $n$-cochain} for the sheaf $\F$ over the open cover $\mathcal{U}$ is a collection of sections of $\F$, one over each $U_{i_0, \, \dots, \, i_n}$.
\end{definition}

The space of Čech $n$-cochains for $\F$ over $\mathcal{U}$ is denoted by $\widecheck{C}^n(\mathcal{U}, \, \F)$. In particular, a Čech $0$-cochain is simply a collection of sections, that is, one gives a section of $\F$ over each open set in the cover. Similarly, a $1$-cochain is a collection of sections of $\F$ over every intersection of two open sets of the cover; the typical notation for a $1$-cochain is $(f_{i, \, j}) \in \F  \left(U_i \cap U_j \right)$.

\begin{remark}If $\phi : \F \to \G$ is a sheaf map, then it induces map on the cochains space
\begin{equation*} \phi : \widecheck{C}^n(\U, \, \F) \to \widecheck{C}^n(\U, \, \G) \end{equation*}
for any open covering $\U$, defined by
\begin{equation*}(f_{i_0, \, \dots, \, i_n}) \longmapsto \left( \phi(f_{i_0, \, \dots, \, i_n}) \right). \end{equation*} \end{remark}

\paragraph{Čech Cochains Complexes.}\index{Cech!complexes} Define a co-boundary operator
\begin{equation*} d : \widecheck{C}^n(\U, \, \F) \to \widecheck{C}^{n+1}(\U, \, \F) \end{equation*}
by setting
\begin{equation*} d \left( (f_{i_0, \, \dots, \, i_n}) \right) := (g_{i_0, \, \dots, \, i_{n+1}}),\end{equation*}
where
\begin{equation*} g_{i_0, \, \dots, \, i_{n+1}} = \sum_{k = 0}^{n+1} (-1)^k \, \rho\left(f_{i_0, \, \dots, \, \widehat{i_k}, \, \dots, \, i_{n+1}} \right). \end{equation*}
In the above formula $\rho$ denotes the restriction map for the sheaf $\F$ corresponding to the inclusion $U_{i_0, \, \dots, \, i_n} \subset  U_{i_0, \, \dots, \widehat{i_k}, \, \dots \, i_n}$.

Any $n$-cochain $c$ with $d(c) = 0$ is called a \textit{$n$-cocycle}\index{Cech!cocycle}; the space of $n$-cocycles is denoted by $ \widecheck{Z}^n(\U, \, \F)$ and it is simply the kernel of $d$ at the $n$-th level.

Any $n$-cochain $c$ with $c = d(c^\prime)$ for some $(n-1)$-cochain $c^\prime$ is called a \textit{$n$-coboundary}\index{Cech!coboundary}; the space of $n$-coboundaries is denoted by $\widecheck{B}^n(\U, \, \F)$.

It is straightforward, but tedious, to prove that $d \circ d = 0$. Thus we have a \textit{Čech cochain complex}
\begin{equation*} 0 \xrightarrow{0} \widecheck{C}^0(\U, \, \F) \xrightarrow{d} \widecheck{C}^1(\U, \, \F) \xrightarrow{d} \widecheck{C}^2(\U, \, \F) \xrightarrow{d} \dots. \end{equation*}

\paragraph{Cohomology with respect to a Cover.}\index{Cech!cohomology} The fact that $d^2 = 0$ implies that for every $n \in \N$
\begin{equation*} \widecheck{B}^n(\U, \, \F) \subset \widecheck{Z}^n(\U, \, \F). \end{equation*}

\begin{definition}[Cohomology] The $n^{th}$ Čech cohomology group $\widecheck{H}^n(\U, \, \F)$ of $\F$ with respect to the open cover $\mathcal{U}$ is the quotient group
\begin{equation*} \widecheck{H}^n(\U, \, \F) = \faktor{\widecheck{Z}^n(\U, \, \F)}{\widecheck{B}^n(\U, \, \F)}. \end{equation*}\end{definition}

There is a lot of work behind the following definition (see \cite[pp. 295--297]{miranda}), but the main point is that we can define a Čech cohomology group \textbf{independent} of the cover $\mathcal{U}$. In fact, one introduces a refinement of $\mathcal{U}$ and proves that the cohomologies can be compared and they depend only on the particular coverings.

\begin{definition}[Čech Cohomology] Fix a sheaf $\F$ and a integer $n \geq 0$. The $n^{th}$ Čech cohomology group of $\F$ on $X$ is the group
\begin{equation*} H^n(X, \, \F) = \varinjlim_{\mathcal{U}} \widecheck{H}^n(\U, \, \F).\end{equation*}\end{definition}

\begin{proposition} Let $X$ be a complex manifold, paracompact and smooth. If $\underline{\R}$ is the constant sheaf on $X$, then
\begin{equation*} H^n(X, \, \underline{\R}) = H_{\mathrm{dr}}^n(X, \, \underline{\R}) = H_{\mathrm{sing}}^n(X, \, \underline{\R}).\end{equation*}\end{proposition}

\begin{remark} \label{rmk:sdksdkskdksd} There is an isomorphism
\begin{equation*} H^0(X, \, \F) \cong \F(X). \end{equation*}
In fact, the $0$-coboundary is given by $\{0\}$, while the $0$-cycle is given by
\begin{equation*} \widecheck{Z}^0(X, \, \F) = \left\{ \{f_\gamma\}_{\gamma \in \Gamma} \: \left| \: f_\alpha - f_\beta = 0\, \, \, \text{in $U_\alpha \cap U_\beta$ for any $\alpha, \, \beta \in \Gamma$} \right. \right\}.\end{equation*}
Therefore $f_\alpha = f_\beta$ in the intersection $U_\alpha \cap U_\beta$ easily implies that it is possible to extend both $f_\alpha$ and $f_\beta$ to a function $f$ in $X$. \end{remark}

In particular, if $X$ is a compact Riemann surface, then the $0$ cohomology group of the holomorphic sheaf is given by
\begin{equation*} H^0(X, \, \mathcal{O}_X) = \mathcal{O}_X(X) = \C, \end{equation*}
since a holomorphic function defined on the whole compact $X$ is bounded and thus constant.

\section{Sheaves of $\mathcal{O}_X$-modules}

\paragraph{Sheaves of $\mathcal{O}_X$-modules.} Let $X$ be a complex manifold (i.e., a manifold with a holomorphic structure).

\begin{definition}[Coherent]\index{Sheaf!coherent} A sheaf $\F$ of $\mathcal{O}_X$-modules is \textit{coherent} if and only if for any $p \in X$ there exist an open neighborhood $U \subset X$ and an exact sequence
\begin{equation*} \mathcal{O}_X^s(U) \xrightarrow{} \mathcal{O}_X^r(U) \xrightarrow{} \F(U) \xrightarrow{} 0, \end{equation*}
i.e, if and only if it is finitely presented. \end{definition}

\begin{example} Let $X$ be a Riemann surface. Then the holomorphic $1$-form sheaf $\Omega_X^1$ is invertible and the isomorphism is given by
\begin{equation*} \Omega_X^1(U) \ni f(z) \, \mathrm{d}z \longmapsto f(z) \in \mathcal{O}_X(U). \end{equation*}
\end{example}

\begin{example}If $\mathrm{dim}_\C X = n$, then
\begin{equation*} \mathcal{O}(X)^n \twoheadrightarrow \Omega_X^1(U), \qquad (f_1, \, \dots, \, f_n) \longmapsto f_1 \, \mathrm{d}z_1 + \dots + f_n \, \mathrm{d}z_n, \end{equation*}
where $z_1, \, \dots, \, z_n$ are local coordinates. Clearly the sheaf is coherent but it is not invertible.
\end{example}

\paragraph{Fundamental Properties.} In this paragraph we briefly discuss some of the most important properties of the particular class of sheaves we just introduced.

\begin{definition}[Support]\index{Sheaf!support} Let $\F$ be a sheaf on $X$. The \textit{support} of $\F$ is defined as the set of the nontrivial stalks, that is,
\begin{equation*} \mathrm{spt}\left(\F \right) = \left\{ x \in X \: \left| \: \F_x \neq 0 \right. \right\}. \end{equation*}
\end{definition}

\begin{proposition} \label{prop:sptttt} Let $X$ be a compact complex manifold (with holomorphic structure) and let $\F$ be a coherent sheaf over $X$. \mbox{}
\begin{enumerate}[label=\textbf{(\alph*)}]
\item The $p$-th cohomology group $H^p(X, \, \F)$ is a finite-dimensional $\C$-vector space.
\item The $p$-th cohomology group is zero for any $p > \mathrm{dim}_\C X$.
\end{enumerate} \end{proposition}

\begin{corollary} Let $X$ be a compact Riemann surface and let $\F$ be a coherent sheaf over $X$. Then the $p$-th cohomology group is given by
\begin{equation*} H^p(X, \, \F) \, \,  \begin{cases} = 0 & \text{if $p \geq 2$} \\ \neq 0 & \text{if $p = 0, \, 1$}. \end{cases} \end{equation*}
\end{corollary}

\begin{theorem} \index{Sheaf!long exact sequence}\label{longcomo} Let $X$ be a compact complex manifold (with holomorphic structure). If
\begin{equation*} 0 \xrightarrow{} \F \xrightarrow{\varphi}\G \xrightarrow{\psi} \h \xrightarrow{} 0 \end{equation*}
is an exact sequence of coherent $\mathcal{O}_X$-modules, then there is a long exact sequence in cohomology, that is, there exists $\partial$ such that
\begin{equation*} 0 \xrightarrow{} H^0(\F) \xrightarrow{H^0(\varphi)}H^0(\G) \xrightarrow{H^0(\psi)} H^0(\h) \xrightarrow{\partial} H^1(\F) \xrightarrow{} \dots \end{equation*}
is an exact long sequence.\end{theorem}

\begin{proof}[Idea of the Proof] We can always choose a covering $\U := \{U_\alpha\}_{\alpha \in I}$ such that, for any $U_\alpha$, it turns out that
\begin{equation*} 0 \xrightarrow{} \F(U_\alpha) \xrightarrow{\varphi_\alpha} \G(U_\alpha) \xrightarrow{\psi_\alpha} \h(U_\alpha) \xrightarrow{} 0 \end{equation*}
is a short exact sequence. Consequently, the sequence
\begin{equation*} 0 \xrightarrow{} H^0(\F) \xrightarrow{H^0(\varphi)}H^0(\G) \xrightarrow{H^0(\psi)} H^0(\h) \end{equation*}
is exact, but the latter map may not be surjective (in general, it will not be). We want to define a map $\partial : H^0(\h) \to H^1(\F)$ that makes the sequence exact at $H^0(\h)$.

Let $\sigma \in H^0(\h)$ and let us consider two open sets of the covering, e.g., $U_\alpha$ and $U_\beta$. We already know that there exists $s_\alpha \in \G(U_\alpha)$ such that $s_\alpha \mapsto \sigma \, \big|_{U_\alpha}$ and there exists $s_\beta \in \G(U_\beta)$ such that $s_\beta \mapsto \sigma \, \big|_{U_\beta}$. Let us set $\tilde{\delta} (\sigma) := s_\alpha - s_\beta = g_{\alpha, \, \beta}$.

Clearly $g_{\alpha, \, \beta} \in Z^1 \left( U_\alpha \cap U_\beta \right) \subseteq \G\left( U_\alpha \cap U_\beta \right)$ and by the exactness of the local sequence, it turns out that $\psi(g_{\alpha, \, \beta}) = 0$ in $\h \left( U_\alpha \cap U_\beta \right)$ and therefore there exists $f_{\alpha, \, \beta} \in \F \left( U_\alpha \cap U_\beta \right) $ such that
\begin{equation*} \varphi( f_{\alpha, \, \beta} ) = g_{\alpha, \, \beta}. \end{equation*}
We can set
\begin{equation*} \partial (\sigma) := \left\{ f_{\alpha, \, \beta} \right\}_{\alpha, \, \beta} \in H^1(\F) \end{equation*}
and, by using the induction principle, we conclude the proof.
\end{proof}

\begin{corollary} Let $X$, $\F$, $\G$ and $\h$ be as above. Then
\begin{equation*} \chi(\G) = \chi(\F) + \chi(\h), \qquad \text{where} \quad \chi(\F) := \sum_{i = 0}^{\mathrm{dim}_\C\left(\mathrm{spt}(\F) \right)} (-1)^i \, \mathrm{dim}_\C \, \left(H^i(\F) \right). \end{equation*} \end{corollary}

\section{GAGA Principle}
\index{GAGA Principle}

Let $X$ be a projective manifold, equipped with the Zariski topology, and let
\begin{equation*}\mathcal{O}_X^{\mathrm{alg}} := \left\{ f : X \to \C \: \left| \: \text{$f$ algebraic} \right. \right\}. \end{equation*}
On the other side, let $X$ be a compact holomorphic manifold, equipped with the Hausdorff topology, and let
\begin{equation*}\mathcal{O}_X^{\mathrm{h}} := \left\{ f : X \to \C \: \left| \: \text{$f$ holomorphic} \right. \right\}. \end{equation*}

\begin{theorem} Let $X$ be a smooth projective manifold. Then there exists an application
\begin{equation*} \F \longmapsto \F^h := \F \bigotimes_{\mathcal{O}_X^{\mathrm{alg}}} \mathcal{O}_X^{\mathrm{h}}, \end{equation*}
where $\F$ is a coherent sheaf of $\mathcal{O}_X^{\mathrm{alg}}$-modules and $\F^h$ a coherent sheaf of $\mathcal{O}_X^{\mathrm{h}}$-modules, such that
\begin{equation*} H_{\mathrm{Zar}}^i(X, \, \F) \cong H_{\mathrm{Hau}}^i(X, \, \F^h). \end{equation*} \end{theorem}


\section{Invertible $\mathcal{O}_X$-module Sheaves}

\begin{definition}[Invertible]\index{Sheaf!invertible} A sheaf $\mathcal{L}$ of $\mathcal{O}_X$-modules is \textit{invertible} if and only if there exists a covering $\mathcal{U} := \{ U_i \}_{i \in I}$ of $X$ such that: \mbox{}
\begin{enumerate}[label=\textbf{(\alph*)}]
\item For any $i \in I$, there is an isomorphism $\phi_i : \mathcal{L}(U_i) \xrightarrow{\sim} \mathcal{O}_X(U_i)$.
\item For any $i, \, j \in I$, there is an invertible function $f_{i, \, j}$, defined on $U_i \cap U_j$, such that $\phi_i = f_{i, \, j} \cdot \phi_j$.
\end{enumerate} \end{definition}

\begin{remark} Let $\mathcal{L}$ be an invertible sheaf of $\mathcal{O}_X$-modules and let $U_i, \, U_j, \, U_k \in \mathcal{U}$. It follows from the definition that, in the triple intersection, the following relation holds true:
\begin{equation*} f_{i, \, k} = f_{i, \, j} \cdot f_{j, \, k}. \end{equation*} \end{remark}

\begin{remark} Equivalently, a sheaf $\F$ of $\mathcal{O}_X$-modules is \textit{invertible} if and only if for every $p \in X$ there is an open neighborhood $U$ of $p$, such that $\mathcal{O} \, \big|_U \cong \F \, \big|_U$ as sheaves of $\mathcal{O} \, \big|_U$-modules on the space $U$.

The invertible sheaves are locally free rank one $\mathcal{O}$-modules. An isomorphism $\phi_U : \mathcal{O} \, \big|_U \to \F \, \big|_U$ is called a \textit{trivialization} of $\F$ over $U$. \end{remark}

\begin{remark}By definition, if $U$ is an open neighborhood of $p$ such that $\mathcal{O} \, \big|_U \cong \F \, \big|_U$, then there is an isomorphism $\phi_U : \mathcal{O}(U) \to \F(U)$.\end{remark}

Let us consider the complex projective line $\p^1(\C)$ endowed with the usual atlas given by $U_0 = \left\{ [1 : z_1] \right\}$, with local coordinate $z := z_1/z_0$, and $U_1 = \left\{ [z_0 : 1] \right\}$, with local coordinate $w := z_0/z_1$.

\begin{example}\label{ex:p23psadapdppewpw}For any fixed $m \in \Z$, we consider the invertible sheaf $\mathcal{O}_{\p^1}[m]$, which is defined as follows: \begin{enumerate}[label=\textbf{(\alph*)}]
\item for $i = 1$ and $i=2$, there are isomorphisms $\mathcal{O}_{\p^1}[m](U_i) \cong \mathcal{O}(U_i) \cong \mathcal{O}(\C)$;
\item the transition map is given by
\begin{equation*} f_{1, \, 0} = \left( \frac{z_1}{z_0} \right)^m. \end{equation*}
\end{enumerate}
We are interested in computing the $0$-cohomology group of $\mathcal{O}_{\p^1}[m]$ for $m = 0$, $m = 1$ and $m \in \Z$.
\begin{enumerate}[label=\textbf{(\arabic*)}]
\item If $m = 0$, then it is straightforward to prove that
\begin{equation*} \mathcal{O}_{\p^1}[0] = \mathcal{O}_{\p^1} \qquad \text{and} \qquad H^0 \left( \p^1, \, \mathcal{O}_{\p^1} \right) = \C. \end{equation*}
\item Let $m = 1$. By definition of the sheaf $\mathcal{O}_{\p^1}[1]$, it turns out that
\begin{equation*} \begin{aligned} & \exists \, f(z) \in \mathcal{O}_{\p^1}(U_0) \quad \text{such that $f$ is holomorphic in $U_0$,} \\[1em] & \exists \, g(z) \in \mathcal{O}_{\p^1}(U_1) \quad \text{such that $g$ is holomorphic in $U_1$}. \end{aligned} \end{equation*}
To compute the group $H^0 \left( \p^1, \, \mathcal{O}_{\p^1} \right)$ we need to check how $f$ and $g$ glue in the intersection $U_0 \cap U_1$. The assumption \textbf{(b)} on the transition map easily implies that
\begin{equation*} f(z) = z \, g \left( \frac{1}{z} \right), \qquad \forall \, z \in U_0 \cap U_1. \end{equation*}
If we use the Laurent develop, it turns out that
\begin{equation*} f(z) = \sum_{i \geq 0} a_i \, z^i \qquad \text{and} \qquad g\left(z\right) = \sum_{i \geq 0} b_i \, z^{-i}, \end{equation*}
hence $f(z) = z \cdot g(1/z)$ in the intersection if and only if
\begin{equation*} f(z) = a_0 + a_1 \, z \qquad \text{and} \qquad g(z) = \frac{b_{-1}}{z} + b_0. \end{equation*}
Therefore we can easily conclude that
\begin{equation*}H^0 \left( \p^1, \, \mathcal{O}_{\p^1}[1] \right) \cong \left\{ \begin{gathered} \text{homogeneous polynomials} \\ \text{of degree $1$ in $z_0$ and $z_1$} \end{gathered} \right\}.  \end{equation*}
\item Let $m > 0$ be any positive integer. There is an isomorphism
\begin{equation*}\left\{ p(z_0, \, z_1) \: \left| \: \begin{gathered} \text{homogeneous polynomials} \\ \text{of degree $m$ in $z_0$ and $z_1$} \end{gathered} \right. \right\} \cong H^0 \left( \p^1, \, \mathcal{O}_{\p^1}[m] \right).  \end{equation*}
which is defined by
\begin{equation*} p(z_0, \, z_1) \longmapsto \begin{cases} \frac{p(z_0, \, z_1)}{z_0^m} & \text{in $U_0$} \\[0.8em] \frac{p(z_0, \, z_1)}{z_1^m} & \text{in $U_1$}. \end{cases} \end{equation*}
The same result holds true for $m < 0$, but there are no polynomials of degree less than zero. In particular, the $H^0$ cohomology group is the trivial one, that is,
\begin{equation*} H^0 \left( \p^1, \, \mathcal{O}_{\p^1}[m] \right) = 0, \qquad \forall \, m \in \Z^{-}.  \end{equation*}
\end{enumerate}
\end{example}

\section{Operations on Sheaves}

\paragraph{Tensor Product.}\index{Sheaf!tensor product} Let $\mathcal{L}$ and $\F$ be invertible sheaves of $\mathcal{O}_X$-modules. Their tensor product is the sheaf denoted by $\mathcal{L} \otimes \F$ which is defined, in terms of co-cycles, as follows:
\begin{equation*} \left( \mathcal{L}, \, \F \right) \ni \left(\ell_{i, \, j}, \, f_{i, \, j} \right) \longmapsto \ell_{i, \, j} \cdot f_{i, \, j} \in \mathcal{L} \otimes \F. \end{equation*}
For example, if $m, \, k \in \Z$, it is relatively easy to prove that the following isomorphism exists:
\begin{equation*} \mathcal{O}_{\p^1}[m] \otimes \mathcal{O}_{\p^1}[k] \cong \mathcal{O}_{\p^1}[m + k]. \end{equation*}

\paragraph{Inverse.}\index{Sheaf!inverse} Let $\mathcal{L}$ be an invertible sheaf. The inverse is denoted by $\mathcal{L}^{-1}$ and it is the unique sheaf such that
\begin{equation*} \mathcal{L} \otimes \mathcal{L}^{-1} = \mathcal{O}_X. \end{equation*}
In particular if $X$ is a Riemann surface or, more generally, a complete holomorphic manifold
\begin{equation*} \left( \left\{ \text{invertible sheaves} \right\}, \, \bigotimes \right) \quad \text{is a group}. \end{equation*}
For a more precise formulation of the above argument, the reader may jump directly to \hyperref[sec:pic]{Section \ref{sec:pic}}.

\paragraph{Line Bundle.}\index{Sheaf!line bundles equivalence} There is a correspondence between invertible sheaf on a smooth manifold $X$ and line bundles, that is,
\begin{equation*}\left\{ \begin{gathered} \text{Invertible sheaves} \\ \text{on $X$ smooth} \end{gathered}  \right\} \stackrel{\sim}{\longleftrightarrow} \left\{ \text{Line bundles} \right\}.  \end{equation*}
If $X$ is an invertible Riemann surface, then
\begin{equation*} \left\{ \text{$F \to X$ Line bundle} \right\}  \stackrel{\sim}{\longleftrightarrow} \begin{gathered} \text{$\exists \, \mathcal{U} := \{U_i\}_{i \in I}$ covering such that} \\ \text{$F_i : F \, \big|_{U_i} \to U_i \times \C$ which sends $z$ to $(z, \, f_i(z))$} \\ \text{and $f_i = g_{i, \, j} \cdot f_j$ in the intersection $U_i \cap U_j$}. \end{gathered} \end{equation*}
A holomorphic section of $F$ is simply the holomorphic mapping $f_i : U_i \to \C$ such that
\begin{equation*} f_i = g_{i, \, j} \cdot f_j \qquad \text{in $U_i \cap U_j$}. \end{equation*}
In particular, there is a correspondence
\begin{equation*}\text{$\F$ sheaf of the sections of $F$} \stackrel{\sim}{\longleftrightarrow} \text{$F \to X$ Line bundles}, \end{equation*}
given by
\begin{equation*} \text{$f_{i}$ (and $g_{i, \, j}$)} \longmapsto F \to U_i \times \C \: : \: z \mapsto \left(z, \, f_i(z) \right). \end{equation*}

\paragraph{Algebraic Curves.}\index{Sheaf!and algebraic curves} Let $\F$ be an invertible sheaf. Clearly it corresponds locally to $\phi_i = z^{n_i}$ which is $0$ if and only if $z = 0$ is a point of multiplicity $n_i$ (an analogous argument works for $\infty$).

If $X \subset \p^2(\C)$ is a smooth algebraic curve of degree $d$, then we can always consider the vector subspace made up of fixed degree smooth algebraic curves, that is,
\begin{equation*} W := \left\{ \text{smooth algebraic curves of fixed degree $h$} \right\} \subset \p^2(\C). \end{equation*}
By Bezout's theorem\footnote{Let $X$ and $Y$ be plane algebraic curves of degree $n$ and $m$ respectively. Then there are $m \cdot n$ points in the intersection $X \cap Y$, counted with the respective multiplicities, provided that $X$ and $Y$ have no common components.}, it follows that for any $Y \in W$, the intersection $Y \cap X$ is given by $h \cdot d$ points with the right multiplicities (this assertion is not very precise, but we only want to give a rough idea of this construction).

Let $Y_1, \, Y_2 \in W$ and consider the points $p_i \in X \cap Y_1$ and $q_i \in Y_2$; since $W$ is a vector subspace, also the linear combinations of these two elements will individuate $Y_3 \in W$, such that the points $r_i \in X \cap Y_3$ are linear combinations of the previous points (see the \hyperref[fig:ksad2]{Figure \ref{fig:ksad2}}).

\begin{figure}[t]
\centering
\includegraphics[width = 15cm, height = 9cm]{images/GC12.png}
\caption{Idea of the construction}
\label{fig:ksad2}
\end{figure} 

\part{Divisor Groups}
\chapter{Divisors} \thispagestyle{empty}

\section{Divisors on Riemann Surfaces}

\begin{definition}[Divisor]\index{Divisor} Let $X$ be a Riemann surface. A \textit{divisor} $D$ on $X$ is a discretely supported function $D : X \to \Z$, that is, a formal sum
\begin{equation*}D = \sum_{p \in X} D(p) \cdot p, \end{equation*}
where $D(p) \in \Z$ is equal to the multiplicity of $D$ at $p$, and $D(p) \neq 0$ for only finitely many $p \in X$. \end{definition}

\paragraph{Divisors Group.}\index{Divisor!group} Given $D_1$ and $D_2$ divisors on $X$, there is a sum operation which is defined by setting
\begin{equation*} D_1 + D_2 = \sum_{p \in X} \left[ D_1(p) + D_2(p) \right] \cdot p, \end{equation*}
and it is easy to prove that $D_1 + D_2$ is still a divisor on $X$. In particular, if we denote the divisors on $X$ by $\Div(X)$, it turns out that $\left( \Div(X), \, + \right)$ is a \textit{commutative group}.

\begin{definition}[Degree]\index{Divisor!degree} Let $X$ be a Riemann surface. The \textit{degree} is the mapping
\begin{equation*}\mathrm{deg} : \Div(X) \to \Z, \qquad D = \sum_{p \in X} D(p) \cdot p \longmapsto \sum_{p \in X} D(p). \end{equation*} \end{definition}

\paragraph{Principal Divisors.}\index{Divisor!principal} Let $X$ be a compact Riemann surface and let $f : X \to \C \cup \{ \infty \}$ be a meromorphic function. There is a mapping $\mathrm{div} : \mathcal{M}\left(X; \; \C \cup \{\infty\} \right) \to \Div(X)$ defined by setting
\begin{equation*} \mathrm{div}(f) := \sum_{p \in X} \mathrm{ord}_p(f) \cdot p. \end{equation*}
The divisor associated to a function is called \textit{principal} and, by the \hyperref[residui]{Residues Theorem \ref{residui}}, it turns out that the degree is always equal to $0$, that is,
\begin{equation*} \mathrm{deg} \left( \mathrm{div}(f) \right) = \sum_{p \in X} \mathrm{ord}_p(f) = 0. \end{equation*}

\begin{example} Let $X = \p^1(\C)$ and let $f(z_0, \, z_1) = z_0 \, (z_0 - z_1) \, z_1^{-2}$. The principal divisor associated to $f$ is given by
\begin{equation*} \mathrm{div}(f) = 1 \cdot [0 : 1] + 1 \cdot [1 : 1] - 2 \cdot [1 : 0], \end{equation*}
coherently with the properties already discussed above. \end{example}

\begin{definition}[Poles/Zeros divisor of $f$]\index{Divisor!poles/zeros} Let $X$ be a compact Riemann surface and let $f : X \to \C \cup \{\infty\}$ be a meromorphic function. The divisor of the zeros is defined as
\begin{equation*}\mathrm{div}_0(f) := \sum_{p \: : \: \mathrm{ord}_p(f) \geq 0} \mathrm{ord}_p(f) \cdot p, \end{equation*}
while the divisor of the poles is defined as
\begin{equation*}\mathrm{div}_\infty(f) := \sum_{p \: : \: \mathrm{ord}_p(f) \leq 0} \left( - \mathrm{ord}_p(f) \right) \cdot P. \end{equation*} \end{definition}

\begin{definition}[Effective Divisor]\index{Divisor!effective} Let $X$ be a compact Riemann surface. A divisor $D \in \Div(X)$ is \textit{effective} if $D(p) \geq 0$ for every $p \in X$.\end{definition}

Consequently, any divisor $D \in \Div(X)$ may be written as a difference between two effective divisors, that is,
\begin{equation*} D = D_0 - D_\infty. \end{equation*}
Finally, there is a partial order on the set of all divisors which is defined by
\begin{equation*} D_1 \geq D_2 \iff  D_1 - D_2 \geq 0 \iff D_1(p) - D_2(p) \geq 0 \quad \forall \, p \in X. \end{equation*}

\section{Invertible Sheaf of $\mathcal{O}_X$-modules associated to a divisor $D$}
\index{Divisor!invertible sheaf}

Let $X$ be a compact Riemann surface. The sheaf of $\mathcal{O}_X$-modules associated to a divisor $D \in \Div(X)$ is denoted by $\mathcal{O}_X[D]$, and it is defined by setting
\begin{equation*} X \supseteq U \longmapsto \mathcal{O}_X[D](U) := \left\{ f : U \to \C \: \left| \: \text{$f$ is meromorphic and $\mathrm{div}(f) \geq - D$} \right. \right\}. \end{equation*}

\begin{proposition}Let $X$ be a compact Riemann surface and let $D \in \Div(X)$. Then $\mathcal{O}_X[D]$ is an invertible sheaf. \end{proposition}

\begin{proof} Let $\mathrm{spt}\left(D \right) = \{p_1, \, \dots, \, p_n\}$. If we set $U_0 := X \setminus \{p_1, \, \dots, \, p_n\}$, then $\mathcal{O}_X\left[D\right]\left(U_0 \right)$ consists of meromorphic functions with no poles, that is, it is isomorphic to the group of holomorphic function defined on $U_0$:
\begin{equation*} \mathcal{O}_X\left[D\right]\left(U_0 \right) \cong \mathcal{O}_X(U_0). \end{equation*}
For any $i = 1, \, \dots, \, n$ we may choose a neighborhood $U_i \ni p_i$ such that $U_i \cap U_j = \emptyset$ whenever $i \neq j$. If the multiplicity of $p_i$ is equal to $n_i$, then we may locally (in $U_i$) choose $\varphi_i = z^{n_i}$ in such a way that $D \, \big|_{U_i} = \mathrm{div}(\varphi_i)$ for any $i = 1, \, \dots, \, n$. Thus there is an isomorphism
\begin{equation*} \mathcal{O}_X(U_i) \xrightarrow{\sim} \mathcal{O}_X[D](U_i), \qquad f \longmapsto \frac{1}{\varphi_i} \cdot f. \end{equation*}
More precisely, there is an equivalence
\begin{equation*} \mathcal{O}_X[D](U_i) = \left\{ f : U_i \to \C \: \left| \: \text{$f$ is meromorphic and $\mathrm{ord}_{p_i}(f) \geq - n_i$} \right. \right\} \xrightarrow{\sim} \left\{ f = \frac{g}{\varphi_i} \right\}, \end{equation*}
where $g : U_i \to \C$ is holomorphic. Indeed, in a more general setting than $X$ Riemann surface, it turns out that the transition maps are given by
\begin{equation*}f_{i, \, j} = \frac{\varphi_j}{\varphi_i}. \end{equation*} \end{proof}

\begin{definition}Let $X$ be a compact Riemann surface and let $D \in \Div(X)$. The $0$th cohomology group is the vector space of the global sections of $\mathcal{O}_X[D]$, and it is denoted by $L(D)$. \end{definition}

\noindent More precisely, it turns out that
\begin{equation*} L(D) := H^0 \left( X, \, \mathcal{O}_X[D] \right) = \left\{ f : X \to \C \: \left| \: \text{$f$ meromorphic and $\mathrm{div}(f) \geq - D$} \right. \right\}. \end{equation*}

\begin{example} Let $X = \p^1(\C)$, $p = [1, \, 0]$ and $D = 1 \cdot p$. If we denote by $z = z_1/z_0$ the local coordinate in $U_0 = \left\{ [1 \: : \: z_1] \right\} \cong \C$, then $p = 0$ in $U_0$. By definition
\begin{equation*} \mathcal{O}_{\p^1}[D] \left(U_0 \right) = \left\{ f : U_0 \to \C \: \left| \: \text{$f$ meromorphic and $\mathrm{ord}_0(f) \geq - 1$} \right. \right\}, \end{equation*}
hence the following equality also holds true:
\begin{equation*} \mathcal{O}_{\p^1}[D] \left(U_0 \right) = \left\{ \frac{g(z)}{z} \: \left| \: \text{$g : U_0 \to \C$ holomorphic in $U_0$} \right. \right\}. \end{equation*}
Assume that $w$ is the local coordinate of $U_1 = \left\{ [z_0 \: : \: 1] \right\} \cong \C$; it remains to study how the functions behaves in the intersection  $U_0 \cap U_1$. If we use the Laurent develop, it turns out that
\begin{equation*} f(z) = \sum_{i \geq -1} a_i \, z^i \qquad \text{and} \qquad f(w) = \sum_{i \geq -1} a_i \, w^{i} = \sum_{i \geq -1} a_i \, z^{-i}, \end{equation*}
hence $f(z) = f(w)$ in the intersection if and only if
\begin{equation*} f(z) = \frac{a_{-1}}{z} + a_0\qquad \text{and} \qquad f(w) = a_{0} + a_{-1} \, w. \end{equation*}
Therefore we can easily conclude that
\begin{equation*} \mathcal{O}_{\p^1}[D] \left(U_0 \right) = \left\{ \frac{a_{-1}}{z} + a_0 \right\} \qquad \text{and} \qquad \mathcal{O}_{\p^1}[D] \left(U_1 \right) = \left\{ a_{-1} \, w + a_0 \right\} , \end{equation*}
that is, there is an isomorphism (see \hyperref[ex:p23psadapdppewpw]{Example \ref{ex:p23psadapdppewpw}})
\begin{equation*} \mathcal{O}_{\p^1}[D] \cong \mathcal{O}_{\p^1}[1]. \end{equation*}
\end{example}

We conclude this section with a brief discussion of $L(D)$, as $D$ ranges in the divisor group of a Riemann surface $X$. First notice that if $D_1 \leq D_2$, then there is a natural inclusion $L(D_1) \subseteq L(D_2)$.

\paragraph{Empty $L(D)$.} Recall that a meromorphic function $f$ is holomorphic if and only if $\mathrm{div}(f) \geq 0$; therefore
\begin{equation*} L(0) = \mathcal{O}(X) :=  \left\{ f : X \to \C \: \left| \: \text{$f$ holomorphic on $X$} \right. \right\}. \end{equation*}
In particular, if $X$ is compact the only holomorphic functions on the whole $X$ are the constants; thus $L(0) = \mathcal{O}(X) \cong \C$.

\begin{lemma} Let $X$ be a compact Riemann surface. If $D$ is a divisor on $X$ with degree strictly less than zero, then $L(D) = \{0\}$. \end{lemma}

\begin{proof}Let $f \in L(D)$ be a nonzero function. The divisor
\begin{equation*} E := D + \mathrm{div}(f) \end{equation*}
is positive ($E \geq 0$), by definition of $L(D)$. Therefore $\mathrm{deg}(E) \geq 0$ and we conclude that there is a contradiction by taking the degree of the defining formula of $E$:
\begin{equation*} \mathrm{deg}(E) > 0 > \mathrm{deg}(D) = \mathrm{deg}(E). \end{equation*} \end{proof}

\begin{proposition}Let $X = \p^1(\C)$ be the complex projective space and let $D \in \Div(X)$ be a positive divisor such that $\mathrm{deg}(D) = d$. Then
\begin{equation*} L(D) = H^0 \left( \p^1, \, \mathcal{O}_{\p^1}[D] \right) \cong \left\{ \text{homogeneous polynomials in $z_0$, $z_1$ of degree $d$} \right\}. \end{equation*} \end{proposition}

\begin{proof}[Proof \cite{miranda}]Let us write the divisor as
\begin{equation*} D = \sum_{i = 1}^{n} e_i \cdot \lambda_i + e_\infty \cdot \infty \end{equation*}
with $\lambda_i \in \C$ distinct, such that $e_1 + \dots + e_n + e_\infty = d \geq 0$, and let us consider the function
\begin{equation*} f_D(z) = \prod_{i = 1}^{n} (z - \lambda_i)^{-e_i}. \end{equation*}
With the above notation, it turns out that the thesis is equivalent to proving that
\begin{equation*}L(D) = \left\{ f_D(z) \cdot g(z) \: \left| \: \text{$g(z)$ is a polynomial of degree at most $\mathrm{deg}(D)$} \right. \right\}. \end{equation*} 

\paragraph{Step 1.} Fix a polynomial $g(z)$ of degree $e$ and notice that $\infty$ is a pole of $g$, whose degree is equal to $e$. The divisor of $f_D(z)$ is exactly
\begin{equation*} \sum_{i = 1}^{n} - e_i \cdot \lambda_i + \left( \sum_{i = 1}^{n} - e_i \right) \cdot \infty, \end{equation*}
therefore
\begin{equation*}\begin{aligned}\mathrm{div} \left( f_D(z) \cdot g(z) \right) + D & = \mathrm{div}(g) + \mathrm{div}(f_D) + D \geq \\[1em] & \geq \left( \sum_i e_i + e_\infty - e \right) \cdot \infty = \left( \mathrm{deg}(D) - e \right) \cdot \infty, \end{aligned} \end{equation*}
which proves that $e \leq \mathrm{deg}(D)$. This proves that the given space is a subspace of $L(D)$.

\paragraph{Step 2.} Vice versa, let us take any nonzero $h(z) \in L(D)$ and let us set $g := \frac{h}{f_D}$. We have
\begin{equation*}\begin{aligned}\mathrm{div} \left( g \right) & = \mathrm{div}(h) - \mathrm{div}(f_D) \geq - D - \mathrm{div}(f_D) \geq \\ & \geq \left( - \sum_i e_i - e_\infty \right) \cdot \infty = - \mathrm{deg}(D) \cdot \infty, \end{aligned} \end{equation*}
which shows that $g$ can have no poles in the finite part $\C$, and can have a pole of order at most $\mathrm{deg}(D)$ at $\infty$. This forces $g$ to be a polynomial of degree at most $\mathrm{deg}(D)$. \end{proof}

\section{Linear Systems of Divisors}

\paragraph{Linear Equivalence.} In this paragraph, we introduce the notion of equivalence between divisors (on a compact Riemann surface) and prove that it is an equivalence relation in the space $\Div(X)$.

\begin{definition}[Linear Equivalence]\index{Divisor!linear equivalence} Let $X$ be a (compact) Riemann surface and let $D_1, \, D_2 \in \Div(X)$. The divisors are said to be \textit{linearly equivalent} if there exists a meromorphic function $f : X \to \C$ such that
\begin{equation*} \mathrm{div}(f) = D_1 - D_2. \end{equation*} \end{definition}

\begin{notation} If $D_1$ and $D_2$ are equivalent divisors, we shall write $D_1 \sim D_2$ (or $D_1 \equiv D_2$). \end{notation}

\begin{proposition} Let $X$ be a compact Riemann surface. \mbox{}
\begin{enumerate}[label=\textbf{(\arabic*)}]
\item $\sim$ is an equivalence relation in $\Div(X)$.
\item $D \sim 0$ if and only if there exists a meromorphic function $f : X \to \C$ such that $D = \mathrm{div}(f)$.
\item If $D_1 \sim D_2$, then $\mathrm{deg}(D_1) = \mathrm{deg}(D_2)$.
\end{enumerate}
\end{proposition}

\begin{remark}If $X = \p^1(\C)$, then
\begin{equation*} \mathcal{O}_{\p^1}[D] = \mathcal{O}_{\p^1}\left[ \mathrm{deg}(D) \right] \end{equation*}
immediately implies that $D_1 \sim D_2$ if and only if $\mathrm{deg}(D_1) = \mathrm{deg}(D_2)$.
\end{remark}

\paragraph{Linear Systems.} We are finally ready to introduce the notion of linear system associated with a divisor $D \in \Div(X)$.

\begin{definition}[Complete Linear System]\index{Divisor!linear system}Let $X$ be a Riemann surface and let $D \in \Div(X)$ be any divisor. The \textit{complete linear system of $D$}, denoted by $|D|$, is the set of all nonnegative divisors $E \geq 0$ which are equivalent to $D$, i.e.
\begin{equation*} \left| D \right| = \left\{ E \in \Div(X) \: \left| \: \text{$E \sim D$ and $E \geq 0$} \right. \right\}. \end{equation*} \end{definition}

There is a geometric/algebraic structure to a complete linear system $|D|$ which is related to the vector space $L(D)$. Let $\p (L(D))$ be the projective space associated to the vector space $L(D)$; we may define a function
\begin{equation*} S :  \p \left( L(D) \right) \to |D| \end{equation*}
by sending the span of a function $f \in L(D)$ to the divisor $\mathrm{div}(f) + D$. Note that this map is well defined, since the divisor of a multiple $\lambda \cdot f$ is equal to the divisor of $f$.

\begin{lemma}\label{lemma:consdodfkf} If $X$ is a compact Riemann surface, the map $S$ defined above is a $1$-$1$ correspondence. \end{lemma}

\begin{proof} Suppose that there are functions $f, \, g : X \to \C$ such that $S(f) = S(g)$. If we cancel the $D$'s, it turns out that $\mathrm{div}(f) = \mathrm{div}(g)$ or, equivalently, that
\begin{equation*} \mathrm{div} \left( \frac{f}{g} \right) = 0. \end{equation*}
The function $f/g$ has no zeros or poles on $X$, thus (by compactness of $X$) it must be a identically equal to a nonzero constant $\lambda \in \C$, i.e., they are the same element in the domain of $S$.

Let $E \in |D|$ be any divisor. By definition $E \sim D$ and $E \geq 0$, therefore there exists $f \in L(D)$ such that
\begin{equation*} E = \mathrm{div}(f) + D, \end{equation*}
which is equivalent to $S(f) = E$, i.e., $S$ is surjective. \end{proof}

Thus for a compact Riemann surface, complete linear systems have a natural projective space structure.

\begin{example} Let $X = \p^1(\C)$, let $p = [0: 1] \in X$ and set $D := 1 \cdot p$. We have already proved that
\begin{equation*} H^0 \left( \p^1, \, \mathcal{O}_{\p^1}[D] \right) \cong \left\{ \frac{a \, z + b}{z} \right\} \end{equation*}
in $U_0 = \{z_0 \neq 0\}$ with local coordinate $z = z_1/z_0$. The linear system $|D|$ is given by the set of positive divisors $E$, such that
\begin{equation*} E - D = \mathrm{div}(f). \end{equation*}
The meromorphic function $f$ has a unique zero of order $1$, and hence
\begin{equation*} \mathrm{div}(f) = q - p \qquad q \in \p^1(\C), \end{equation*}
implies that
\begin{equation*} |D| = \left\{ E = 1 \cdot q \: \left| \: q \in \p^1(\C) \right. \right\}. \end{equation*}
\end{example}

\begin{proposition} Let $X$ be a compact Riemann surface and let $D \in \Div(X)$. Then
\begin{equation*} \mathrm{deg}(D) < 0 \iff H^0\left(X, \, \mathcal{O}_X[D] \right) = 0 \iff |D| = \emptyset. \end{equation*}\end{proposition}

\begin{proof}The former equivalence is easy and it has already been proved above. For any $E \in |D|$, it turns out that
\begin{equation*} \begin{cases} E \geq 0 \\ E = D + \mathrm{div}(f) \end{cases} \implies \begin{cases} E \geq 0 \\ \mathrm{deg}(E) = \mathrm{deg}(D) + 0 < 0\end{cases} \implies \text{absurd, $E$ is positive}. \end{equation*}\end{proof}

%A \textit{general} linear system is a subset of a complete linear system $|D|$, which corresponds (via the map $S$) to a \textit{linear subspace} of $\p(L(D))$. The dimension of a linear system is the dimension of the linear subspace of $|D|$, considered as a projective space.

%\paragraph{Isomorphisms between $L(D)$'s under Linear Equivalence} If two divisors are linearly equivalent, then the associated spaces of meromorphic functions are naturally isomorphic.

%\begin{proposition}Suppose that $D_1$ and $D_2$ are linearly equivalent divisors on a Riemann surface $X$. Let $h$ be a meromorphic function such that $D_1 - D_2 = \mathrm{div}(h)$. Then there is an isomorphism of complex vector space, given by
%\begin{equation*} \mu_h : L(D_1) \xrightarrow{\simeq} L(D_2), \qquad f \longmapsto f \cdot h. \end{equation*}
%In particular, if $D_1 \equiv D_2$, then $\mathrm{dim} \, L(D_1) = \mathrm{dim} \, L(D_2)$. \end{proposition}

%\begin{proof}It's a straightforward consequence of the fact that $\mu_{1/h}$ is the inverse map of $\mu_h$, and both are well defined. \end{proof}

%The same construction used above in defining the space of functions with poles bounded by a divisor can be used to defined spaces of meromorphic $1$-forms.

%\begin{definition} The space of \textit{meromorphic} $1$-forms with poles bounded by a divisor $D$, denoted by $L^{(1)}(D)$, is the set of meromorphic $1$-forms
%\begin{equation*}L^{(1)}(D) := \left\{ \omega \in \mathcal{M}^{(1)}(X) \: \left| \: \text{$\mathrm{div}(\omega) \geq - D$} \right. \right\}. \end{equation*} \end{definition}

%\paragraph{Computation of $L(D)$ on the Riemann Sphere.} Suppose that $D$ is a divisor on the Riemann Sphere with $\mathrm{deg}(D) \geq 0$. 

%\begin{corollary} Let $D$ be a divisor on the Riemann Sphere. Then 
%\begin{equation*} \mathrm{dim} \, L(D) = \begin{cases} 0 & \text{if $\mathrm{deg}(D) < 0$, and} \\ 1 + \mathrm{deg}(D) & \text{otherwise}. \end{cases} \end{equation*}\end{corollary}

%\paragraph{Computation of $L(D)$ for a Complex Torus.} Let $X = \faktor{\C}{\Lambda}$ be a complex torus.

%\begin{proposition} Let $D \in \Div(X)$. \mbox{}
%egin{enumerate}[label=\textbf{(\alph*)}]
%item If $\mathrm{deg}(D) < 0$, then $L(D) = \{0\}$.
%\item If $\mathrm{deg}(D) = 0$ and $D \sim 0$, then $\mathrm{dim} \, L(D) = 1$.
%item If $\mathrm{deg}(D) = 0$ and $D \not \sim 0$, then $L(D) = \{0\}$.
%\item If $\mathrm{deg}(D) > 0$, then $\mathrm{dim} \,L(D) = \mathrm{deg}(D)$.
%\end{enumerate}\end{proposition}
 
% \begin{proof} \mbox{} %FARE
%\begin{enumerate}[label=\textbf{(\alph*)}]
%\item 
%\item 
%\item 
%\item If $\mathrm{deg}(D) = 1$, then $D$ is linearly equivalent to a positive divisor, so we may assume that $D = p$ for some point $p \in X$.

%Clearly the constant functions belong to $L(D)$, thus its dimension is at least one. On the other hand, suppose that $L(D)$ contains a nonconstant meromorphic function $f$. This function $f$ must have a pole, bounded by $p$, hence $f$ has a single pole at $p$ and no other pole. Therefore the associated map $F: X \to \C_\infty$ has degree one (i.e. an isomorphism), but this is absurd.

%To finish the proof we may proceed by induction on $D$; assume then that $\mathrm{deg}(D) = d > 1$. Let $D = D_1 + p$ for some divisor $D_1$ of degree $d-1$ and some point $p \in X$. By induction, we know that $\mathrm{dim} \, L(D_1) = d -1 $.

%Find a positive divisor $E \sim D$, which does not have $p$ in its support; this is always possible. Let $f$ be a meromorphic function on $X$ with $\mathrm{div}(f) = E - D$; notice that $f \in L(D)$. Also we have that $\mathrm{div}(f) + D_1 = E - p$ which is not nonnegative; hence $f \notin L(D)$. This proves that $L(D_1) \neq L(D)$, and so $\mathrm{dim} \,L(D) > d - 1$ since $L(D_1) \subset L(D)$.

%To see that the dimension of $L(D)$ is exactly $d$, choose a local coordinate $z$ centered at $p$, and suppose that $D(p) = n$. Then every $f \in L(D)$ has a Laurent series in $z$ whose lowest possible term is $z^{-n}$. Consider the linear map $\tau : L(D) \to \C$ sending $f$ to the coefficient of $z^{-n}$. The kernel of $\tau$ is exactly $L(D - p) = L(D_1)$, hence $L(D)$ has dimension at most one more than the dimension of $L(D)$.
%\end{enumerate}\end{proof}

%\paragraph{A Bound on the Dimension of $L(D)$.} We are now ready to conclude this section with a bound on the dimension of $L(D)$ for a compact Riemann surfaces, proving that these spaces are finite-dimensional.

%\begin{lemma} Let $X$ be a Riemann surface, let $D$ be a divisor of $X$, and let $p$ be a point of $X$. Then either $L(D - p) = L(D)$ or $L(D - p)$ has co-dimension one in $L(D)$. \end{lemma}

%\begin{proposition} Let $X$ be a compact Riemann surface, and let $D$ be a divisor on $X$. Then the space of functions $L(D)$ is a finite-dimensional complex vector space. Indeed, if we write $D = P - N$, with $P$ and $N$ nonnegative divisors with disjoint support, then
%\begin{equation*} \mathrm{dim} \, L(D) \leq 1 + \mathrm{deg}(P). \end{equation*}
%In particular, if $D$ is a nonnegative divisor, then $\mathrm{dim} \, L(D) \leq 1 + \mathrm{deg}(D)$. \end{proposition}

\section{Divisors and Maps to Projective Space}

In this section, we shall be concerned with the possibility to embed a Riemann surface into a projective space holomorphically.

\paragraph{Holomorphic Maps to Projective Space.}\index{Holomorphic!map to projective space} The first step is to understand what is the meaning of a "holomorphic map to the complex projective space $\p^n$".

\begin{definition}[Holomorphic Map] Let $X$ be a Riemann surface. A map $\varphi : X \to \p^n$ is \textit{holomorphic} at the point $p \in X$ if there are a neighborhood $U_p$ of $p$ and holomorphic functions $\sigma_0, \, \dots, \, \sigma_n : U_p \to \C$, not all zero at $p$, such that - locally - $\varphi$ has the form
\begin{equation*} \varphi(x) = \left[ \sigma_0(x) : \: \dots \: : \sigma_n(x) \right]. \end{equation*}\end{definition}

Observe that, if one of the $\sigma_i$'s is nonzero at $p$, then it will be nonzero in a neighborhood of $p$; thus the map given by the $\sigma_i$'s is be well defined - at least locally.

\paragraph{Maps to Projective Space as Meromorphic Functions.} On a compact Riemann surface, the holomorphic maps are constant, and thus one cannot expect to use the same holomorphic function $\sigma_i$ at all points $p \in X$ to define a holomorphic map.

Let $X$ be a Riemann surface. Choose $n + 1$ meromorphic functions $\sigma = (\sigma_0, \, \dots, \, \sigma_n)$ on $X$, not all identically to zero. Define $\varphi_\sigma : X \to \p^n$ by setting
\begin{equation} \varphi_\sigma(p) = \left[ \sigma_0(p) : \: \dots \: : \sigma_n(p) \right]. \label{eq:mapproj} \end{equation}
A priori $\varphi_\sigma$ is defined at $p$ if \mbox{}
\begin{enumerate}[label=\textbf{(\arabic*)}]
\item $p$ is not a pole of any $\sigma_i$, and
\item $p$ is not a zero of every $\sigma_i$.
\end{enumerate}
The reader may check by herself that $\varphi_\sigma$ is holomorphic at all points $p$ satisfying both condition \textbf{(1)} and \textbf{(2)} (i.e., it is holomorphic at every definition point).

On the other hand, the function may also be defined on points which violate the first condition, as a consequence of the next result.

\begin{lemma} \label{lemms:fasfls} If the meromorphic functions $\sigma_0, \, \dots, \, \sigma_n$ are not all identically zero at $p$, then the map \eqref{eq:mapproj} given above can be extended to a holomorphic map defined at $p$. \end{lemma}

\begin{proof} Set
\begin{equation*} m := \min_{i = 0, \, \dots, \, n} \mathrm{ord}_p \, \sigma_i.\end{equation*}
By definition there is a neighborhood $U_p$ of $p$ such that
\begin{equation*} \sigma_i \, \big|_{U_p} \end{equation*}
has not other pole inside $U$, that is, using the local coordinate $z$ it turns out that
\begin{equation*} g_i(z) = z^{-m} \, \sigma_i(z) \end{equation*}
is a holomorphic map at all point of $U_p$, for each $i = 0, \, \dots, \, n$. Therefore, it suffices to define the function $\varphi_\sigma$ at the point $p$ as follows:
\begin{equation*} \varphi_\sigma(p) := \left[g_0(p) : \: \dots\: : g_n(p) \right] = z^{-m} \left[ \sigma_0(p) :  \: \dots \: : \sigma_n(p) \right]. \end{equation*}  \end{proof}

\begin{example}Let $X = \p^1(\C)$, $p = [1 : 0] \in X$, and let $D = 2 \cdot p \in \Div(X)$. In the previous section we have proved that the $0$-th cohomology group may be identified as follows:
\begin{equation*} H^0 \left( \p^1, \, \mathcal{O}_{\p^1}[D] \right) = \left\{\left. \frac{1}{z^2} \, P(z) \: \right| \: \mathrm{deg}(P) \leq 2 \right\}. \end{equation*}
It is easy to prove that in homogeneous coordinates we have the identity
\begin{equation*} H^0 \left( \p^1, \, \mathcal{O}_{\p^1}[D] \right) = \left\{ \frac{c_0 \, z_0^2 + c_1 \, z_0 \, z_1 + c_2 \, z_1^2}{z_1^2} \right\}, \end{equation*}
and hence a basis of the $0$-th cohomology group is given by
\begin{equation*} \sigma_0 = \left( \frac{z_0}{z_1} \right)^2, \qquad \sigma_1 = \frac{z_0 \, z_1}{z_1^2} \quad \text{and} \quad \sigma_2 = \left( \frac{z_1}{z_1} \right)^2. \end{equation*}
As a consequence, \hyperref[lemms:fasfls]{Lemma \ref{lemms:fasfls}} allows us to write the $\varphi_\sigma : \p^1 \to \p^2$ as
\begin{equation*} [z_0 : z_1] \longmapsto \left[ z_0^2 : z_0 \, z_1 : z_1^2 \right], \end{equation*}
which is the so-called \textit{Veronese embedding} (see \hyperref[fig:ve]{Figure \ref{fig:ve}}). Clearly the image of this map inside $\p^2$ is given by
\begin{equation*} \left\{ [y_0 \: : \: y_1 \: : \: y_2] \in \p^2(\C) \: \left| \: y_1^2 = y_0 \, y_2 \right. \right\}, \end{equation*}
and hence
\begin{equation*}|D| = \left\{E \in \Div(X) \: \left| \: \begin{gathered} \text{$E = \mathrm{div}(f) + D$, $E \geq 0$ such that} \\ \text{$f$ has a pole of order $2$ in $p$ and} \\ \text{$f$ has two zeros of order $1$ ($q_1$ and $q_2$)} \end{gathered} \right. \right\}. \end{equation*}
More precisely, in $\p^2$ the divisor $E = q_1 + q_2$ corresponds to the intersection of the line $\ell$ with the image of the map $\varphi_\sigma$.
\end{example}

\begin{figure}[ht]
\centering
\includegraphics[width = 12cm, height = 6cm]{images/GCp.png}
\label{fig:ve}
\caption{Veronese Embedding}
\end{figure} 

Formally, it turns out that, if $\varphi : X \to \p^n$ is holomorphic and $H$ is a hyperplane of $\p^n$, then the intersection $H \cap \varphi(X)$ may be regarded as a divisor of $X$.

\begin{definition}[Hyperplane Divisor]\index{Divisor!hyperplane} Let
\begin{equation*} H = \left\{h := \sum_{i=0}^n a_i \, y_i = 0 \right\} \subset \p^n(\C) \end{equation*}
be a hyperplane. If $q$ is a point of $\varphi(X) \cap H$, and $h_0 = \sum_{i = 0}^n b_i \, y_i$ defines a hyperplane such that $h_0(q) \neq 0$, then we define the \textit{pull-back of $H$} as
\begin{equation*} \varphi^\ast(H) = \sum_{q \in H \cap \varphi(X)} \left[ \sum_{p \in \varphi^{-1}(q)} \mathrm{ord}_p \left( \frac{h}{h_0} \circ \varphi \right) \cdot p \right]. \end{equation*}\end{definition}

\begin{remark}In the previous definition, we divide $h$ by $h_0$ since $h$ is a holomorphic map and, as we have already proved earlier, it would be constant on any compact Riemann surface. \end{remark}

%\begin{proposition}\label{prop:propr}Let $\varphi : X \to \p^n$ be a holomorphic map. Then there is a $(n + 1)$-tuple of meromorphic functions $f = (f_0, \, \dots, \, f_n)$ on $X$ such that $\varphi = \varphi_f$. Moreover if two $(n+1)$-tuples $f$ and $g$ induce the same map, so that $\varphi_f = \varphi_g$ as holomorphic maps to $\p^n$, then there is a meromorphic function $\lambda$ on $X$ such that $g_i = \lambda \cdot f_i$ for every $i$. \end{proposition}

%In particular, there is a $1$-$1$ correspondence between the set of holomorphic maps from $X$ to $\p^n$ and the projective space $\p_{\mathcal{M}(X)}^n$ (which is the set of $1$-dimensional subspaces of the vector space $\mathcal{M}(X)^{n + 1}$ defined over the field $\mathcal{M}(X)$).

%PARAGRAFO NON SVOLTO
%\paragraph{The Linear Systems of a Holomorphic Map.} Let $\varphi : X \to \p^n$ be a holomorphic map to a projective space. We can always write it as
%\begin{equation*} \varphi(x) = \left[ f_0(x) \: :  \: \dots \: : \: f_n(x) \right], \end{equation*}
%where each $f_i$ is a meromorphic function on $X$. Let $D = - \min_{i = 0, \, \dots, \, n} \{ \mathrm{div}(f_i) \}$ be the inverse of the minimum divisor of the divisors of the functions. Thus, for $p \in X$, we have that $-D(p) \leq \mathrm{ord}_p(f_i)$ for each $i$.

%Therefore $-D \leq \mathrm{div}(f_i)$ for each $i$, and we have that $f_i \in L(D)$ for every $i$. More precisely, if we let $V_f$ to be the $\C$-linear span of the functions $\{f_i\}$, that is, the set of all linear combinations $\sum_i a_i \, f_i$ with $a_i \in \C$, we have that $V_f$ is a linear subspace of $L(D)$.

%Therefore the set of divisors $|\varphi| = \left\{ \mathrm{div}(g) + D \: \left| \: g \in V_f \right. \right\}$ forms a linear system on $X$, a subsystem of the complete linear system $|D|$ of all positive divisors linearly equivalent to $D$.

%Clearly the construction of $D$ depends on the choice of the meromorphic functions used to define $\varphi$, but the linear system depends only on $\varphi$.

%\begin{lemma}The linear system $|\varphi|$ defined above is independent of the choice of the functions $\{f_i\}$ used to define $\varphi$.\end{lemma}

%\begin{proof}Suppose that $\varphi$ is also defined by
%\begin{equation*} \varphi(x) = \left[ g_0(x) \: :  \: \dots \: : \: g_n(x) \right]. \end{equation*}
%By Proposition \ref{prop:propr} there is a meromorphic function $\lambda: X \to \C$ such that $g_i = \lambda \, f_i$ for each $i$. Since $\mathrm{div}(g_i) = \mathrm{div}(\lambda) + \mathrm{div}(f_i)$, the minimum of the divisors of the $g_i$'s will differ from the minimum of the divisors of the $f_i$'s by the divisor of $\lambda$. Hence
%\begin{equation*} D_f - D_g = \mathrm{div}(\lambda) \implies D_f \sim D_g \implies |D_f| = |D_g|. \end{equation*}
%Finally, we notice that a typical member of $|\varphi_g|$ is also an element of $|\varphi_f|$ and vice versa:
%\begin{equation*} \begin{aligned} \mathrm{div} \left( \sum_i a_i \, g_i \right) + D_g & = \left( \sum_i a_i \, \lambda \, f_i \right) = \\ & = \left( \sum_i a_i \, f_i \right) + \mathrm{div}(\lambda) + D_g = \\ & = \left( \sum_i a_i \, f_i \right) + D. \end{aligned}\end{equation*}
%Hence the two linear system are the same and the definition of $|\varphi|$ doesn't depend on the representation.\end{proof}

%\begin{definition} Let $\varphi : X \to \p^n$ be an holomorphic map with non degenerate image. The linear system defined above is called the \textit{linear system of the map $\varphi$}. \end{definition}

%Note that in general the degree of the map $\varphi$, as it was defined previously in the course, is generally different from the degree of the divisor in the associated linear system $|\varphi|$.

%A linear system of dimension $n$ whose divisors have degree $d$ is often called a $g_d^n$.

%\begin{lemma} Let $\varphi : X \to \p^n$ be a holomorphic map. Then for every $p \in X$ there is a divisor $E \in |\varphi|$ which does not have $p$ in its support. In other words, there is no point of $X$ which is contained in every divisor of the linear system $|\varphi|$. \end{lemma}

\paragraph{Correspondence Linear Subsystems-Subspaces.} In this paragraph, we want to prove that the only restriction for $\varphi_{|D|}$ not being holomorphic, is that there exists a point $p \in X$ such that $p \in \mathrm{spt} \, E$ for every divisor $E \in |D|$.

\begin{definition}[Base Point Free]\index{b.p.f.} Let $V \subset |D|$ be a linear system on a Riemann surface $X$. A point $p \in X$ is a \textit{base point} of $V$ if and only if every divisor $E \in V$ contains $p$, that is,
\begin{equation*} E \geq p. \end{equation*}
A linear system is \textit{base point free} - b.p.f. from now on - if there are no base points. \end{definition}

\begin{remark} A point $p \in X$ is a base point for a linear system $V \subseteq \left|D\right|$ if and only if
\begin{equation*}f(p) = 0, \qquad \forall \, f \in \overline{V} \subseteq H^0 \left(X, \, \mathcal{O}_X[D]\right).\end{equation*}
Equivalently, if $\sigma_0, \, \dots, \, \sigma_n$ is a basis for $\overline{V}$, then $p$ is a base point if and only if
\begin{equation*} \sigma_i(p) = 0, \qquad \forall \, i = 0, \, \dots, \, n. \end{equation*}\end{remark}

Let $\overline{V} \subset H^0 \left(X, \, \mathcal{O}_X[D]\right)$ be a linear subsystem. Suppose that $V$ is b.p.f, and suppose also that $\overline{V}$ is a projective space of dimension $n + 1$. It induces a morphism
\begin{equation*} \varphi : X \to \p^n(\C) = \p(V^v) \qquad x \longmapsto \left(\sigma_0(x), \, \dots, \, \sigma_n(x) \right), \end{equation*}
where $V^v$ is the geometric dual of $V$ (i.e. the vector space of the hyperplanes of $V$), and the $\sigma_i$ are meromorphic functions not identically equal to zero.

%We have proved above that holomorphic maps to $\p^n$ with non degenerate image have an associated linear system $|\varphi|$ base-point-free.

%Suppose that $Q \subset |D|$ is a linear system, a subsystem of a complete linear system $|D|$, and let $V \subset L(D)$ be the vector space corresponding to $Q$. For every $f \in L(D)$ and every point $p \in X$, we have that $D(p) + \mathrm{ord}_p(f) \geq 0$. So $p$ is a base point of $Q$ if and only if for every $f \in V$, we have $D(p) + \mathrm{ord}_p(f) \geq 1$. Since $f \in L(D)$ already, this is equivalent to saying that $f \in L(D - p)$.

%\begin{lemma} A point $p \in X$ is a base point of $Q \subseteq |D|$ if and only if $V \subseteq L(D - p)$. In particular $p$ is a base point of $|D|$ if and only if $L(D - p) = L(D)$. \end{lemma}

%\begin{proposition} Let $D$ be a divisor on a compact Riemann surface $X$. Then a point $p \in X$ is a base point of $|D|$ if and only if $\mathrm{dim} \, L(D) = \mathrm{dim} \, L(D - p)$. Hence $|D|$ is base-point-free if and only if for every point $p \in X$, the dimension of $L(D - p)$ is equal to the dimension of $L(D)$ minus one. \end{proposition}

%\paragraph{Holomorphic Map via a Linear System.} In this paragraph we shall prove that the base-point-free property of the linear system of a holomorphic map in fact characterizes such systems.

%\begin{proposition} Let $Q \subset |D|$ be a base-point-free linear system of dimension $n$ (projective) on a compact Riemann surface $X$. Then there is a holomorphic map $\varphi : X \to \p^n$ such that $Q = |\varphi|$. Moreover $\varphi$ is unique up to the choice of coordinates in $\p^n$. \end{proposition}

%\begin{proof} \end{proof} %

\begin{theorem}There is a $1$-$1$ correspondence
\begin{equation*} \left\{ \begin{gathered} \text{b.p.f. linear systems $\overline{V} \subseteq H^0\left(X, \, \mathcal{O}_X[D]\right)$} \\ \text{of projective dimension $n + 1$, and $D$ a $d$-divisor} \end{gathered} \right\} \leftrightarrow \left\{ \begin{gathered} \text{holomorphic maps $\varphi : X \to \p^n(\C)$} \\ \text{with non degenerate image,} \\ \text{such that $\varphi^\ast(H)$ is a $d$-divisor} \end{gathered} \right\}. \end{equation*} \end{theorem}

The holomorphic map needs to have a \textit{non-degenerate} image, and this assumption cannot be relaxed. Indeed, if the image $\varphi(X)$ is contained in a hyperplane of $\p^n(\C)$, then the correspondence fails to be $1$-$1$.

The reader may prove this fact by herself; e.g., consider a map $X \to \p^2$ and look at it as immersed in a higher-dimension space $X \to \p^2(\C) \hookrightarrow \p^3(\C)$.

\begin{proof}Let $V$ be a b.p.f system as in the assumptions. The associated holomorphic map is the one we have already constructed above, i.e.,
\begin{equation*} \varphi : X \to \p^n(\C) = \p(V^v) \qquad x \longmapsto \left(\sigma_0(x), \, \dots, \, \sigma_n(x) \right). \end{equation*}
Suppose that $y_0, \, \dots, \, y_n$ are the coordinates of $\p^n$, and let $p \in \varphi(X) \cap H$ be any point,
\begin{equation*} H := \left\{ h := \sum_{i = 0}^n a_i \, y_i = 0 \right\} \quad \text{and} \quad H_0 := \left\{ h_0 := \sum_{i = 0}^n b_i \, y_i = 0 \right\} \end{equation*}
be hyperplanes such that $p \notin H_0$. By definition it turns out that
\begin{equation*} \varphi^\ast \left(H \right) = \sum_{p \in X} \mathrm{ord}_p \left( \frac{h}{h_0} \circ \varphi \right) \cdot p, \end{equation*}
and this divisor has clearly degree equal to $d$.

The opposite arrow is a direct consequence of \hyperref[lemma:prox]{Lemma \ref{lemma:prox}} which is stated and proved right below.
\end{proof}

\begin{lemma} \label{lemma:prox} Let $\varphi : X \to \p^n(\C)$ be a holomorphic map and assume that it is b.p.f., i.e., for any $p \in X$ there exists $i$ such that $\sigma_i(p) \neq 0$. Let
\begin{equation*} D := - \sum_{p \in X} \min_{i} \left( \mathrm{ord}_p(\sigma_i) \right) \cdot p \end{equation*}
and let $H := \left\{ h := \sum_{i = 0}^n a_i \, y_i = 0 \right\}$ be a hyperplane. Then
\begin{equation*} \mathrm{div} \left( \sum_{i = 0}^{n} a_i \, \sigma_i \right) + D = \varphi^\ast(H). \end{equation*} \end{lemma}

\begin{proof}Let $p \in X$, let $j$ be the index such that $\mathrm{ord}_p(D) = - \mathrm{ord}_p(\sigma_j)$, and let $h_0 := y_j$. It easily follows that
\begin{equation*} \frac{h}{h_0} \circ \varphi = \sum_{i = 0}^{n} a_i \, \frac{\sigma_i}{\sigma_j} \implies \mathrm{ord}_p \left( \varphi^\ast (H) \right) = \mathrm{ord}_p \left(\sum_{i = 0}^{n} a_i \, \sigma_i \right) \underbracket{- \mathrm{ord}_p(\sigma_j)}_{= \mathrm{ord}_p(D)}, \end{equation*}
therefore
\begin{equation*} \left\{ \varphi^\ast(H) \: \left| \: \text{$H$ projective hyperplane} \right. \right\} = \left\{ \mathrm{div}(f) + D \: \left| \: f \in \left< \sigma_0, \, \dots, \, \sigma_n \right> \right. \right\} . \end{equation*} \end{proof}

\begin{example}Let $\varphi : \p^1 \to \p^2$ be the Veronese embedding, i.e.,
\begin{equation*} (z_0, \, z_1) \longmapsto \left( \frac{z_0^2}{z_1^2}, \, \frac{z_0}{z_1}, \, 1 \right), \end{equation*}
locally in the chart $U_0$. The divisor is given by
\begin{equation*} - D = - 2 \cdot [1, \, 0] \implies D = 2\cdot p, \qquad p := [1, \, 0], \end{equation*}
and notice that $\varphi(p) = p_\infty \in \varphi(\p^1) \subset \p^2$.

Let $H$ be the hyperplane defined by the equation $h := a_0 \, y_0 + a_1 \, y_1 + a_2 \, y_2 = 0$ and let $H_0$ be the hyperplane defined by the equation $h_0 := y_0 = 0$. Then the pullback is given by
\begin{equation*} \varphi^\ast(H) = \mathrm{div} \left(a_0 \, z_0^2 + a_1 \, z_0 \, z_1 + a_2 \, z_2^2 \right) - \mathrm{div} \left(z_0^2 \right) + 2 \cdot p =  \mathrm{div} \left(a_0 \, z_0^2 + a_1 \, z_0 \, z_1 + a_2 \, z_2^2 \right), \end{equation*}
coherently with the fact that the Veronese embedding is globally given by
\begin{equation*} (z_0, \, z_1) \longmapsto \left( z_0^2, \, z_0 \, z_1, \, z_1^2 \right). \end{equation*} \end{example}

\section{Inverse Image of Divisors}

Let $X$ and $Y$ be compact Riemann surfaces and let $F : X \to Y$ be a holomorphic function. The \textit{pullback} via $F$ of $q \in Y$ is defined by
\begin{equation*}F^\ast(q) = \sum_{p \in F^{-1}(q)} \mathrm{molt}_p (F) \cdot p. \end{equation*}

\begin{definition}[Divisor Pullback]\index{Divisor!pullback} Let $D$ be a divisor on $Y$ of the form
\begin{equation*} D = \sum_{q \in Y} n(q) \cdot q. \end{equation*}
The \textit{pullback} of $D$ via $F$ is a divisor on $X$, defined by the formula
\begin{equation*} F^\ast(D) := \sum_{q \in Y} n(q) \cdot F^\ast(q). \end{equation*}
\end{definition}

\begin{proposition}Let $X$ and $Y$ be compact Riemann surfaces and let $F : X \to Y$ be a holomorphic function. \mbox{}
\begin{enumerate}[label=\textbf{(\arabic*)}]
\item The pull-back $F^\ast : \Div(Y) \to \Div(X)$ is a group homomorphism.
\item If $g : Y \to \C$ is a meromorphic function, then
\begin{equation*}F^\ast \left( \mathrm{div}(g) \right) = \mathrm{div} \left(F^\ast \, g \right) = \mathrm{div} \left(g \circ F\right). \end{equation*}
\item The degree is multiplicative, i.e.,
\begin{equation*}\mathrm{deg} \left( F^\ast(D) \right) = \mathrm{deg}(F) \cdot \mathrm{deg}(D).\end{equation*}
\item The pull-back commutes with the holomorphic function sheaf associated to a divisor, that is,
\begin{equation*} F^\ast \left( \mathcal{O}_Y[D] \right) = \mathcal{O}_X \left[F^\ast (D) \right]. \end{equation*}
\end{enumerate}
\end{proposition}

\begin{theorem}Let $X$ be a compact Riemann surface, and let $\varphi : X \to \p^n(\C)$ be a holomorphic map. Assume that $Y = \varphi(X) \subseteq \p^n$ is a smooth algebraic curve of degree $e := \mathrm{deg}(Y)$. Then
\begin{equation*} \mathrm{deg} \left( \varphi^\ast(H) \right) = \mathrm{deg}(Y) \cdot \mathrm{deg}\left( \varphi :X \to Y \right). \end{equation*}
\end{theorem}

\begin{remark}In general, the image $Y = \varphi(X)$ is an algebraic curve (not necessarily smooth) - and a Riemann surface -, whose degree is defined by the formula
\begin{equation*} \mathrm{deg}(Y) = \mathrm{deg}(Y \cap H) = \sum_{q \in Y \cap H} \mathrm{ord}_q(H). \end{equation*}
\end{remark}

\begin{proof} Let us consider a point $p \in X$, and let us consider the hyperplanes
\begin{equation*} H := \left\{ h := \sum_{i = 0}^n a_i \, y_i = 0 \right\} \quad \text{and} \quad H_0 := \left\{ h_0 := \sum_{i = 0}^n b_i \, y_i = 0 \right\} \end{equation*}
such that $\varphi(p) \in H \cap Y$ and $\varphi(p) \notin H_0$. By definition, it turns out that
\begin{equation*} \mathrm{ord}_p \left( \varphi^\ast(H) \right) = \mathrm{ord}_p \left( \frac{h}{h_0} \circ \varphi \right) = \mathrm{molt}_p(\varphi) \cdot \mathrm{ord}_{\varphi(p)} \left( \frac{h}{h_0} \right), \end{equation*}
therefore
\begin{equation*} \begin{aligned} \mathrm{deg} \left( \varphi^\ast(H) \right) & = \sum_{p \in X}  \mathrm{molt}_p(\varphi) \cdot \mathrm{ord}_{\varphi(p)} \left( \frac{h}{h_0} \right) = \\ & = \sum_{q \in Y} \left[ \sum_{p \in \varphi^{-1}(q)}  \mathrm{molt}_p(\varphi) \cdot \mathrm{ord}_{q} \left(H \right) \right] = \\ & = \mathrm{deg}\left( \varphi :X \to Y \right) \cdot \underbrace{\sum_{q \in Y} \mathrm{ord}_q(H)}_{= e}. \end{aligned} \end{equation*} \end{proof}

\section{Canonical Divisor}

The sheaf $\Omega_X^1$ consists of all the holomorphic $1$-forms defined on $X$. Recall that, locally, a $1$-form can be identified to a holomorphic function, i.e.,
\begin{equation*} \Omega_X^1 \ni \omega \longmapsto \omega = f(z) \, \mathrm{d}z. \end{equation*}
In particular, the reader may check by herself that $\Omega_X^1$ is an invertible sheaf. Let $\U = \{U_i\}_{i \in I}$ be a covering of $X$, and let $\varphi_i : U_i \xrightarrow{\sim} V_i \subset \C$ be a collection of charts such that
\begin{equation*} U_i \xrightarrow{\sim} V_i \cong \Delta, \qquad \omega \, \big|_{U_i} \longmapsto f_i(z) \, \mathrm{d}z. \end{equation*}
The transition maps are denoted, as usual, by $\varphi_{i, \, j} := \varphi_i \circ \varphi_j^{-1}$. It follows that, if the intersection $U_i \cap U_j$ is nonempty, the $1$-forms can be glued together by
\begin{equation} \label{int} \omega_j = \left( \omega_i \circ \varphi_{i, \, j} \right) \cdot \varphi_{i, \, j}^{\prime}.\end{equation}

\begin{proposition} \label{dasodosl}Let $\omega_1, \, \omega_2 \in \Omega_X^1$. There exists a unique meromorphic function $g : X \to \C$ such that $\omega_2 = g \, \omega_1$. In particular, locally
\begin{equation*} \omega_2 = f_2(z) \, \mathrm{d}z = g(z) \, f_1(z) \, \mathrm{d}z, \end{equation*}
where $\omega_i = f_i(z) \, \mathrm{d}z$. \end{proposition}

\begin{proof}Let $\U = \{U_i\}_{i \in I}$ be a (connected) covering of $X$. For each $i$ it turns out that
\begin{equation*} \omega_1 = f_i^{(1)}(z) \, \mathrm{d}z \quad \text{and} \quad \omega_2 = f_i^{(2)}(z) \, \mathrm{d}z \qquad \text{in $U_i \xrightarrow{\sim} V_i$}. \end{equation*}
Since $f_i^{(1)}(z)$ and $ f_i^{(2)}(z)$ are both holomorphic at every point of the open set $V_i$, the function
\begin{equation*} h_i(z) := \frac{ f_i^{(2)}(z)}{ f_i^{(1)}(z)} \end{equation*}
is meromorphic at every point of $V_i$. In the intersection $U_i \cap U_j$, it follows from \eqref{int} that
\begin{equation*}h_i(z) = \frac{ \left( f_i^{(2)} \circ \varphi_{i, \, j} \right) \cdot \varphi_{i, \, j}^{\prime} (z)}{ \left( f_i^{(1)} \circ \varphi_{i, \, j}\right) \cdot \varphi_{i, \, j}^{\prime} (z)} = \frac{f_i^{(2)}}{f_i^{(1)}} \circ \varphi_{i, \, j}(z), \end{equation*}
that is, the function
\begin{equation*} g(z) = h_i \circ \varphi_i(z) \qquad \text{for $x \in V_i$ and $x = \varphi(z)$},\end{equation*}
is a well-defined meromorphic function on the whole surface $X$.
\end{proof}

\begin{definition}[$1$-Forms Divisor]\index{Divisor!$1$-form} Let $X$ be a Riemann surface and let $\U = \{U_i\}_{i \in I}$ be a covering of $X$. The \textit{canonical divisor} associated to the holomorphic $1$-form $\omega \in \Omega_X^1$, denoted by $\mathrm{div}(\omega)$, is locally defined by
\begin{equation*} \mathrm{div}(\omega) \, \big|_{U_i} = \sum_{p \in U_i} \mathrm{ord}_p(f) \cdot p, \end{equation*}
where $f$ is the function such that $\omega = f(z) \, \mathrm{d}z$ in $U_i$. \end{definition}

\begin{proposition}Let $X$ be a Riemann surface and let $\omega_1, \, \omega_2 \in \Omega_X^1$ be holomorphic $1$-forms. There exists a meromorphic function $g$ such that
\begin{equation*} \mathrm{div}(\omega_1) = \mathrm{div}(\omega_1) + \mathrm{div}(g),\end{equation*}
that is, $\omega_1 \sim \omega_2$. \end{proposition}

\begin{proof}A simple corollary of \hyperref[dasodosl]{Proposition \ref{dasodosl}}.\end{proof}

\begin{definition}[Canonical Divisor]\index{Divisor!canonical} A divisor $K_X \in \Div(X)$ is a \textit{canonical divisor} if it is the divisor of a holomorphic $1$-form, that is,
\begin{equation*} \exists \, \omega \in \Omega_X^1 \: : \: K_X = \mathrm{div}(\omega). \end{equation*}\end{definition}

\begin{remark} The canonical divisor is \textbf{not} unique, but, for any $K_X^\prime = \mathrm{div}(\omega^\prime)$ and any $K_X^{\prime\prime} = \mathrm{div}(\omega^{\prime\prime})$, it turns out that $K_X^\prime \sim K_X^{\prime\prime}$. \end{remark}

\begin{remark}There is an isomorphism $\mathcal{O}_X[K_X] \cong \Omega_X^1$ as invertible sheaves, since the co-cycles are the same. \end{remark}

\begin{example} Let us consider the sheaf $\Omega_{\p^1}^1$, and suppose that $z$ is the local coordinate of $U_0$ and $w$ is the local coordinate of $U_1$. It turns out that
\begin{equation*} w = \frac{1}{z} \implies \mathrm{d}w = - \frac{1}{z^2} \, \mathrm{d}z \qquad \text{in $U_0 \cap U_1$}, \end{equation*}
therefore
\begin{equation*} f_0(z)\, \mathrm{d}z = f_1(w) \, \mathrm{d}w \iff f_0(z) = - \frac{1}{z^2} \, f_1 \left( \frac{1}{z} \right) \, \mathrm{d}z. \end{equation*}
In particular, the co-cycle is given by
\begin{equation*} f_{0, \, 1}(z) = - \frac{1}{z^2} = - \left( \frac{z_1}{z_0} \right)^{-2},\end{equation*}
and this immediately implies that $\Omega_{\p^1}^1 \cong \mathcal{O}_{\p^1}(-2)$.
\end{example}

\begin{example}Let $X = \faktor{\C}{\Lambda}$ be a Riemann surface of genus $1$. Then $\omega = 1 \cdot \mathrm{d}z$ is a holomorphic $1$-form, but it is zero at the quotient. In particular, $K_X = 0$ and thus $\Omega_X^1 \cong \mathcal{O}_X[K_X] = \mathcal{O}_X$. \end{example}

\section{Riemann-Hurwitz Theorem}

In this final section, we want to state and prove the \textit{Riemann-Hurwitz theorem}, which links the canonical divisor of two compact connected Riemann surfaces via a morphism.

\begin{theorem}[Riemann-Hurwitz]\index{Riemann-Hurwitz Theorem}\label{th:rh} Let $X$ and $Y$ be compact connected Riemann surfaces, and let $\pi : X \to Y$ be a morphism. Then
\begin{equation*} K_X \sim \pi^\ast(K_Y) + R, \end{equation*}
where $R$ is the ramification divisor, and it is defined as
\begin{equation*} R = \sum_{p \in X} \left(\mathrm{ord}_p(\pi) - 1 \right) \cdot p. \end{equation*}
 \end{theorem}

\begin{proof} Recall that there is an isomorphism between invertible sheaves $\Omega_Y^1 \cong \mathcal{O}_Y[K_Y]$ since the co-cycles coincide.

\paragraph{Step 1.} Let $p \in X$, and let $q \in Y$ be a point in the image (i.e. $\pi(p) = q$). Let $W_q$ be a neighborhood of $q$ in $Y$ such that
\begin{equation*} \omega_Y = f(w) \, \mathrm{d}w, \end{equation*}
where $K_Y = \mathrm{div}(\omega_Y)$. By \hyperref[nf]{Proposition \ref{nf}}, there exists a neighborhood $U_p$ of $p$ in $X$ such that
\begin{equation*} \pi \, \big|_{U_p} : U_p \longrightarrow W_q \subseteq Y, \qquad z \longmapsto w = z^m, \end{equation*}
where $p \longleftrightarrow 0$ and $q \longleftrightarrow 0$.

\paragraph{Step 2.} By definition, we have that
\begin{equation*} \pi^\ast \left(f(w) \, \mathrm{d} w\right) = f(z^m) \cdot \left(m \, z^{m-1} \right) \, \mathrm{d}z, \end{equation*}
where, intuitively, the first term corresponds to the differential $\omega_X$, and the second term corresponds to the ramification at $p$.

Therefore, by taking the divisors of both left-hand side and right-hand side, we have that
\begin{equation*} \mathrm{div} \left( \omega_X \, \big|_{U_p} \right) = \mathrm{div} \left( f(z^m) \right) = \pi^\ast \left( \mathrm{div}(f(w)) \right) + \mathrm{div} \left(z^{m-1} \right),\end{equation*}
and this concludes the proof, since
\begin{equation*} K_X = \mathrm{div} \left( \omega_X \right) =\pi^\ast \left( \mathrm{div}(\omega_Y) \right) + \sum_{p \in X} \left( \mathrm{ord}_p(\pi) - 1 \right) \cdot p.\end{equation*}
\end{proof}

\begin{corollary} \label{cor:dincan}Let $X$ be a connected Riemann surface of genus $g$, and let $K_X$ be a canonical divisor of $X$. Then $\mathrm{deg}(K_X) = 2g - 2 = - \chi_{top}(X)$. \end{corollary}

\begin{proof} Let $\pi : X \to \p^1$ be a morphism of degree $d$. Such a map always exists, but the result is highly nontrivial\footnote{In \hyperref[sec:merglo]{Section \ref{sec:merglo}} we have proved that there is a meromorphic function $f : X \longrightarrow \C$, and we know that this can be identified with a holomorphic map $F : X \longrightarrow \p^1(\C)$.}. It follows from the \hyperref[th:rh]{Riemann-Hurwitz Theorem \ref{th:rh}} that
\begin{equation} \label{prima} K_X = \pi^\ast(K_{\p^1}) + R \implies \mathrm{deg}(K_X) = \underbrace{d \cdot \mathrm{deg}(K_{\p^1})}_{= - 2d} + \sum_{p \in X} \left( \mathrm{ord}_p(\pi) - 1 \right). \end{equation}
On the other hand, the \hyperref[th:hf]{Hurwitz Theorem \ref{th:hf}} gives us the identity
\begin{equation} \label{seconda} \chi_{top}(X) = d \cdot \chi_{top}(\p^1) - \sum_{p \in X} \left( \mathrm{ord}_p(\pi) - 1 \right), \end{equation}
hence, if we combine \eqref{prima} and \eqref{seconda} together, then we can conclude that
\begin{equation*}\mathrm{deg}(K_X) = 2g - 2 = - \chi_{top}(X). \end{equation*}
\end{proof}

\begin{remark}As a consequence of the previous result, it turns out that there are three big families of algebraic curves, i.e., 
\begin{equation*} g = 0 \qquad \mathrm{deg}(K_X) < 0 \end{equation*}
\begin{equation*} g = 1 \qquad \mathrm{deg}(K_X) = 0 \end{equation*}
\begin{equation*} g \geq 2 \qquad \mathrm{deg}(K_X) > 0. \end{equation*}\end{remark}
\chapter{The Riemann-Roch Theorem and Serre Duality} \thispagestyle{empty}
\label{chap:8}

Let $X$ be a Riemann surface, and let be given a divisor $D \in \Div(X)$. The primary goals of \hyperref[chap:8]{Chapter \ref{chap:8}} and of \hyperref[chap:10]{Chapter \ref{chap:10}} are the following: \mbox{}
\begin{enumerate}[label=\textbf{\arabic*)}]
\item Find an isomorphism for the $0$th cohomology group $H^0\left(X, \, \mathcal{O}_X[D] \right)$, or, at least, an estimate of the dimension $h^0\left(X, \, \mathcal{O}_X[D] \right)$.
\item Study the map $\varphi_{|D|}$. More precisely, we would like to know if $|D|$ is a b.p.f. linear system (i.e., if $\varphi_{|D|}$ is a morphism) and, in that case, if $\varphi_{|D|}$ is injective (or, even better, an embedding).
\end{enumerate}

\section{Rough Estimate of $h^0 \left(X, \, \mathcal{O}_X[D] \right)$}

\begin{remark}Let $X$ be a compact Riemann surface, and let $p \in X$. There exists a short exact sequence of sheaves, given by
\begin{equation*} 0 \xrightarrow{} \mathcal{I}_p \xrightarrow{} \mathcal{O}_X \xrightarrow{} \C_p \xrightarrow{} 0,\end{equation*}
where $\mathcal{I}_p$ is the sheaf of the ideals of the function vanishing at $p$, $\mathcal{O}_X$ is the holomorphic function sheaf and $\C_p$ is the skyscraper sheaf.\end{remark}

\begin{proposition}Let $X$ be a compact Riemann surface, and let $p \in X$. There is an isomorphism of sheaves
\begin{equation*} \mathcal{I}_p \cong \mathcal{O}_X[-p]. \end{equation*}\end{proposition}

\begin{proof}Notice that, locally, $p$ is the divisor of a function $\varphi : U \to \Delta \subset \C$ that sends $p$ to $0$. The reader may fill in the details of the proof as an exercise, following one of the two possibilities below:\mbox{}
\begin{enumerate}[label=\textbf{(\arabic*)}]
\item Prove that $\mathcal{O}_X[-p]$ is locally generated by $\varphi$, as a consequence of the fact that the sheaf $\mathcal{O}_X[p]$ is (locally) generated by the function $\frac{1}{\varphi}$.
\item Prove that for any subset $U \subseteq X$
\begin{equation*} f \in \mathcal{O}_X[-p](U) \iff f = \varphi \cdot g, \end{equation*}
where $g$ is an holomorphic function defined on $U$. 
\end{enumerate}\end{proof} 

\begin{proposition} \label{exas}Let $D \in \Div(X)$ be a divisor, and let $p \in X$ be a point. Then there is a short exact sequence of sheaf maps
\begin{equation*} 0 \xrightarrow{} \mathcal{O}_X[D - p] \xrightarrow{} \mathcal{O}_X[D] \xrightarrow{} \C_p \xrightarrow{} 0.\end{equation*}
\end{proposition}

\begin{proof} It suffices to prove the exactness in each position.

\paragraph{Left Exactness.} There is a natural inclusion
\begin{equation*} \mathcal{O}_X[D - p] \hookrightarrow \mathcal{O}_X[D], \end{equation*}
as a consequence of the fact that
\begin{equation*}\mathrm{div} \, f + D - p \geq 0 \implies \mathrm{div} \, f + \underbrace{(D - p) + p}_{= D} \geq 0. \end{equation*}

\paragraph{Middle/Right Exactness.} Let $U_y \subseteq X$ be an open neighborhood of a point $y \neq p \in X$, and assume that $p \notin U_y$. In this case it is straightforward to prove that
\begin{equation*} \mathcal{O}_X[D - p](U_y) \cong \mathcal{O}_X[D](U_y) \end{equation*}
is an isomorphism, coherently with the fact that
\begin{equation*} \C_p(U_y) = 0. \end{equation*}
Let $U_p \subseteq X$ be an open neighborhood of $p$. If we set $m := \mathrm{ord}_p \, D^\prime$, then it turns out that there is a meromorphic function 
\begin{equation*} f(z) = z^{-(m+1)} \, h(z) \qquad \text{locally in $U_p$,} \end{equation*}
for some $h$ non-vanishing at $p$, such that
\begin{equation*}\mathrm{div} \, f \, \big|_{U_p} + (D - p) \, \big|_{U_p} = - p. \end{equation*}
It follows that the function $f$ generates the cokernel, and thus
\begin{equation*} \faktor{\mathcal{O}_X[D]}{\mathcal{O}_X[D - p} \cong \C \cdot \{ z^{m+1} \} \cong \C \cong \C_p, \end{equation*}
since $\C \cdot \{ z^{m+1} \}$ is supported at $p$.
\end{proof}

\begin{remark}More generally, there is a functor
\begin{equation*} F : \left( \Div(X), \, + \right) \longrightarrow \left( \{ \text{invertible sheaves} \}, \, \otimes \right), \end{equation*}
defined by
\begin{equation*} D \mapsto \mathcal{O}_X[D] \qquad \text{and} \qquad D_1 + D_2 \mapsto \mathcal{O}_X[D_1] \otimes \mathcal{O}_X[D_2] \cong \mathcal{O}_X[D_1 + D_2]. \end{equation*}
It is not hard to prove that this functor is exact, since
\begin{equation*} \left( 0 \xrightarrow{} \mathcal{O}_X[-p] \xrightarrow{} \mathcal{O}_X \xrightarrow{} \C_p \xrightarrow{} 0 \right) \otimes_{\mathcal{O}_X} \mathcal{O}_X[D]\end{equation*}
is isomorphic to
\begin{equation*} 0 \xrightarrow{} \mathcal{O}_X[D-p] \xrightarrow{} \mathcal{O}_X[D] \xrightarrow{} \C_p \xrightarrow{} 0 .\end{equation*}
This functor is analyzed more in-depth in \hyperref[sec:pic]{Section \ref{sec:pic}} using the language of the Picard group. \end{remark}

\begin{remark} Recall that
\begin{equation*} \mathrm{deg} \, D < 0 \implies \begin{cases} H^0\left(X, \, \mathcal{O}_X[D] \right) = 0, \\[0.5em] |D| = \emptyset. \end{cases} \end{equation*} \end{remark}

\begin{proposition}\label{prop:stimah0} Let $X$ be a compact connected Riemann surface, and let $D$ be a divisor of positive degree $d \geq 0$. Then the following estimate holds true:
\begin{equation} \label{eq:stimah0} h^0 \left(X, \, \mathcal{O}_X[D] \right) \leq d + 1. \end{equation} \end{proposition}

\begin{proof} We first distinguish between divisor with empty linear system and divisor with a nontrivial linear system, and then we proceed by induction on the degree of $D$.

\paragraph{Case $|D| = \emptyset$.} If $|D| = \emptyset$, then $\mathrm{deg} \, D = 0$ (since it is positive by assumption) and thus we infer by the previous remark that
\begin{equation*}h^0 \left(X, \, \mathcal{O}_X[D] \right) = 0. \end{equation*}

\paragraph{Case $|D| \neq \emptyset$, Base Step.} Let $D$ be a divisor of degree $0$, and let $E \in |D|$ be an effective divisor linearly equivalent to $D$.

Clearly $E$ is the null divisor $0$ (the coefficients are positive, and they sum to zero;) thus $|D| = \{0\}$ and the dimension satisfies the estimate \eqref{eq:stimah0} as expected:
\begin{equation*}h^0 \left(X, \, \mathcal{O}_X[D] \right) = 1. \end{equation*}

\paragraph{Case $|D| \neq \emptyset$, Inductive Step.}  Let $D$ be a divisor of degree $d$, let $E \in |D|$ be an effective divisor, and take $p \in \mathrm{spt}(E)$. By \hyperref[exas]{Proposition \ref{exas}} there is a short exact sequence
\begin{equation*} 0 \xrightarrow{} \mathcal{O}_X[E - p] \xrightarrow{} \mathcal{O}_X[E] \xrightarrow{} \C_p \xrightarrow{} 0,\end{equation*}
which induces a long sequence in cohomology (see \hyperref[longcomo]{Theorem \ref{longcomo}}):
\begin{equation*} 0 \xrightarrow{} H^0 \left(X, \, \mathcal{O}_X[E - p] \right) \xrightarrow{} H^0 \left(X, \, \mathcal{O}_X[E] \right) \xrightarrow{} H^0\left(X, \, \C_p \right) \xrightarrow{} \dots\end{equation*}
The map $H^0 \left(X, \, \mathcal{O}_X[E] \right) \xrightarrow{} H^0\left(X, \, \C_p \right)$ is not necessarily surjective, therefore we can only infer an inequality between the dimensions, i.e.,
\begin{equation*} h^0 \left(X, \, \mathcal{O}_X[E] \right) \leq h^0 \left(X, \, \mathcal{O}_X[E - p] \right) + h^0 \left(X, \, \C_p \right). \end{equation*}
In conclusion, recall that there is an isomorphism $H^0\left(X, \, \C_p \right) \cong \C$, and apply the inductive hypothesis to obtain the sought inequality:
\begin{equation*} h^0 \left(X, \, \mathcal{O}_X[E] \right) \leq (d - 1) + 1 + 1 = d+1. \end{equation*}
\end{proof}

\paragraph{Skyscraper Sheaf.}\index{Sheaf!skyscraper} Let $X$ be a Riemann surface, let $p \in X$ be a point and take any natural number $n \in \N$. Take $D \in \Div(X)$ and set $\Delta := n \cdot p$; by \hyperref[exas]{Proposition \ref{exas}} there is a short exact sequence
\begin{equation*} 0 \xrightarrow{} \mathcal{O}_X[D - \Delta] \xrightarrow{} \mathcal{O}_X[D] \xrightarrow{} \mathcal{O}_\Delta \xrightarrow{} 0.\end{equation*}
The sheaf $\mathcal{O}_\Delta$ is the skyscraper sheaf, supported in $\{p\}$, and such that
\begin{equation*}\left( \mathcal{O}_\Delta \right)_q = \begin{cases} 0 & \text{if $q \neq p$} \\[0.8em] \faktor{\mathcal{O}_{X, \, p}}{\mathcal{M}_{X, \, p}^n} \cong \C^n & \text{if $q = p$}, \end{cases} \end{equation*}
where $\mathcal{O}_{X, \, p}$ is the stalk of the holomorphic function sheaf at $p$, and $\mathcal{M}_{X, \, p}$ is the maximal ideal at $p$. More precisely, in the local coordinate we have $p = 0$ and $\mathrm{div}(z) = p$, which in turn implies that
\begin{equation*} \faktor{\mathcal{O}_{X, \, p}}{(x^n)} = \left\{ a_0 + \dots + a_{n-1} \, x^{n-1} \right\}. \end{equation*}

\begin{remark}The above argument can be easily generalized to a linear combination of points. \mbox{}
\begin{enumerate}[label=\textbf{(\alph*)}]
\item The isomorphism $H^0 \left( X, \, \mathcal{O}_\Delta \right) \cong \C^n$ proves that
\begin{equation*} h^0 \left( X, \, \mathcal{O}_\Delta \right) = n. \end{equation*}
\item By \hyperref[prop:sptttt]{Proposition \ref{prop:sptttt}} it turns out that
\begin{equation*} H^1 \left( X, \, \mathcal{O}_\Delta \right) = 0,  \end{equation*}
since the dimension of the support is zero - i.e., strictly less than the index of the cohomology group.
\item If $\Delta = n_1 \cdot p_1 + \dots + n_k \cdot p_k$, then $\mathcal{O}_\Delta$ is a skyscraper sheaf supported in $\{p_1, \, \dots, \, p_k\}$. In a similar fashion, the reader may prove that
\begin{equation*} h^0 \left( X, \, \mathcal{O}_\Delta \right) = \sum_{i = 1}^{k} n_i \qquad \text{and} \qquad h^1 \left( X, \, \mathcal{O}_\Delta \right) = 0.\end{equation*}
\end{enumerate}
\end{remark}

\section{Riemann-Roch Formula}

The major result we want to achieve in this section is the \textit{Riemann-Roch} formula for compact connected Riemann surfaces.

\begin{remark}Recall that the dimension of $X$ over $\C$ is, by definition, equal to $1$. Therefore Betti's numbers are all zero except for $h^0$ and $h^1$, that is,
\begin{equation*} h^j \left( X, \, \mathcal{O}_X[D] \right) = 0 \qquad \forall \, j \geq 2, \, \, \forall \, D \in \Div(X). \end{equation*} \end{remark}

\begin{definition}[Holomorphic Euler Characteristic]\index{Euler-Poincaré characteristic!holomorphic} The \textit{Euler (holomorphic) characteristic} of a surface $X$ with respect to a divisor $D \in \Div(X)$ is defined by
\begin{equation*} \chi \left(X, \,  \mathcal{O}_X[D] \right) = h^0\left(X, \, \mathcal{O}_X[D] \right) - h^1\left(X, \, \mathcal{O}_X[D] \right). \end{equation*} \end{definition}

\begin{definition}[Arithmetic Genus]\index{Arithmetic genus} The \textit{arithmetic genus} of a Riemann surface $X$ is defined by
\begin{equation*} p_a(X) := 1 - \chi \left( X, \, \mathcal{O}_X \right). \end{equation*}
\end{definition}

\begin{remark}If $X$ is a compact connected Riemann surface, then $h^0 \left( X, \, \mathcal{O}_X \right) = 1$ and hence
\begin{equation*} p_a(X) = h^1 \left(X, \, \mathcal{O}_X \right). \end{equation*} \end{remark}

\begin{theorem}[Riemann-Roch]\index{Riemann-Roch Theorem} \label{RiemannRoch} Let $X$ be a connected compact Riemann surface, and let $D \in \Div(X)$ be any divisor. Then it turns out that
\begin{equation}\label{r-r:eq1} \chi \left(X, \,  \mathcal{O}_X[D] \right) = \mathrm{deg} \, D + \chi \left( X, \, \mathcal{O}_X \right), \end{equation}
or, equivalently, that
\begin{equation}\label{r-r:eq2} h^0 \left( X, \, \mathcal{O}_X[D] \right) - h^1 \left( X, \, \mathcal{O}_X[D] \right) = \mathrm{deg} \, D +1 - p_a(X). \end{equation}\end{theorem}

\begin{proof} We first assume that the divisor $D$ is effective and we derive \eqref{r-r:eq1} by induction on the degree of $d = \mathrm{deg} \, D$; only then we solve the general case.

\paragraph{Effective Divisor, Base Step.} If $\mathrm{deg} \, D = 0$, then $D = 0$ (since the coefficients are positive and their sum is equal to zero). It follows that
\begin{equation*} \mathcal{O}_X[D] \cong \mathcal{O}_X \implies \eqref{r-r:eq1}. \end{equation*}

\paragraph{Effective Divisor, Inductive Step.} Suppose that $\mathrm{deg} \, D = d > 0$, and let $p \in \mathrm{spt}(D)$ be a point. By \hyperref[exas]{Proposition \ref{exas}} there is a short exact sequence
\begin{equation*} 0 \xrightarrow{} \mathcal{O}_X[D - p] \xrightarrow{} \mathcal{O}_X[D] \xrightarrow{} \C_p \xrightarrow{} 0,\end{equation*}
which induces an identity on the Euler characteristics, that is,
\begin{equation*}  \chi \left(X, \,  \mathcal{O}_X[D] \right) =  \chi \left(X, \,  \mathcal{O}_X[D - p] \right) + \chi \left(X, \,  \C_p \right). \end{equation*}
As we have already observed, the Euler characteristic of $\C_p$ is given by the difference between $h^0\left(X, \, \C_p \right)$ and $h^1\left(X, \, \C_p \right)$; since the dimension of the support is less than $1$, it turns out that
\begin{equation*} \chi \left(X, \,  \C_p \right) = h^0 \left( X, \, \C_p \right) - h^1 \left( X, \, \C_p \right) = 1 - 0 = 1. \end{equation*}
Using the induction hypothesis, we immediately obtain the thesis for an effective divisor:
\begin{equation*}  \chi \left(X, \,  \mathcal{O}_X[D] \right) =  \chi \left(X, \,  \mathcal{O}_X \right) + \mathrm{deg}(D) - \mathrm{deg}(p) + 1 =  \chi \left(X, \,  \mathcal{O}_X \right) + \mathrm{deg}(D). \end{equation*}

\paragraph{General Divisor.} Let $D$ be any divisor, and let
\begin{equation*}D = D^+ - D^-\end{equation*}
be the decomposition of $D$ in the positive part and the negative part (both of which are effective). As usual, by \hyperref[exas]{Proposition \ref{exas}} there is a short exact sequence
\begin{equation*} 0 \xrightarrow{} \mathcal{O}_X[D^+ - D^-] \xrightarrow{} \mathcal{O}_X[D^+] \xrightarrow{} \C_{D^-} \xrightarrow{} 0,\end{equation*}
which induces an identity of Euler characteristics, that is,
\begin{equation*} \begin{aligned} \chi \left(X, \,  \mathcal{O}_X[D] \right) & = \chi \left(X, \,  \mathcal{O}_X[D^+] \right) - \chi \left(X, \, \C_{D^-} \right) = \\[1em] & = \chi \left(X, \, \mathcal{O}_X \right) + \mathrm{deg} \, D^+ - \mathrm{deg} \, D^- = \\[1em] & = \chi \left( X, \, \mathcal{O}_X \right) + \mathrm{deg} \, D, \end{aligned} \end{equation*}
and this concludes the proof.\end{proof}

\section{Serre Duality}

The primary goal of this section is to use every tool we have introduced so far to prove the notorious \textit{Serre Duality Theorem}, which will come to handy to justify different results in the following sections.

\begin{theorem}[Serre] \label{serreduality1} Let $X$ be a compact connected Riemann surface, let $K_X$ be a canonical divisor and let $D$ be any divisor on $X$. Then there is an isomorphism
\begin{equation*} H^1 \left(X, \, \mathcal{O}_X[D] \right)^{v} \cong H^0 \left(X, \, \mathcal{O}_X \left[K_X - D \right] \right), \end{equation*}
where $^v$ denotes the dual vector space.\end{theorem}

\subsection{Mittag-Leffler Problem}
\index{Mittag-Leffler problem}

Let $X$ be a compact Riemann surfaces. Let $p_1, \, \dots, \, p_s \in X$ be given points, and suppose that for any $i = 1, \, \dots, \, s$ there is a \textit{polar polynomial}, that is,
\begin{equation*} h_i(z) = \sum_{k = - n_i}^{-1} a_k \, z^k, \qquad \text{in $U_{p_i} \cong \Delta$ neighborhood of $p_i$ with local coordinate $z$}. \end{equation*}
In this section, we investigate the Mittag-Leffler problem, that is, we want to determine if there exists a function meromorphic on $X$ such that: \mbox{}
\begin{enumerate}[label=\textbf{(\arabic*)}]
\item The function $f: X \to \C$ is holomorphic outside of the finite set $\{p_1, \, \dots, \, p_s\}$.
\item The principal part of $f$ in $U_{p_i}$ is given by the polar polynomial $h_i$.
\end{enumerate}
A meromorphic function $f: X \to \C$ satisfying these properties exists locally, but the problem is to find one globally defined. The answer, as we shall be able to prove soon, depends on
\begin{equation*} H^0 \left( X, \, \mathcal{O}_X[D] \right) \qquad \text{and} \qquad H^1 \left( X, \, \mathcal{O}_X[D] \right), \end{equation*}
where the divisor is simply defined by
\begin{equation*} D := \sum_{i = 1}^{s} n_i \cdot p_i. \end{equation*}

\paragraph{Laurent Tails.}\index{Laurent tails} Let $X$ be a Riemann surface, let $p \in X$ be a point, and let $U_p \ni p$ be an open neighborhood with coordinate $z_p$. A \textit{Laurent tail} with respect to $p$ is a function of the form
\begin{equation}\label{lttttt} r_p(z_p) = \sum_{i = - n_p}^{k_p} a_i \, z_p^i,  \end{equation}
where $a_i \in \C$ are complex coefficients.

\begin{definition}[Laurent Tail Divisor]\index{Divisor!Laurent tail} A \textit{Laurent tail divisor} on $X$ is a finite formal sum
\begin{equation*} \sum_{p \in X} r_p(z_p) \cdot p, \end{equation*}
where $r_p(-)$ is a Laurent polynomial in the local coordinate $z_p$, that is, a Laurent series of the form \eqref{lttttt} with a finite number of terms. \end{definition}

\begin{notation}Let $X$ be a Riemann surface. We denote by $\mathcal{T}_X$ the set of all the Laurent tail divisors defined on $X$. \end{notation}

\begin{definition}[Laurent Tail Sheaf]\index{Sheaf!Laurent tails} Let 
\begin{equation*} D = \sum_{p \in X} D(p) \cdot p \in \Div(X).\end{equation*}
The \textit{Laurent tail divisor sheaf} associated to $D$ is defined by setting
\begin{equation}\label{sjesd} U \longmapsto \mathcal{T}_X[D](U) := \left\{\left. \sum_{p \in X} r_p( - ) \cdot p \: \right| \: \forall \, p \in U \: : \: k_p < - D(p) \right\}, \end{equation}
where $k_p$ is the maximal order of the function $r_p$, as defined in \eqref{lttttt}. \end{definition}

The reader may check by herself that \eqref{sjesd} actually defines a sheaf. For every divisor $D \in \Div(X)$, there is a truncation map
\begin{equation*} t_D : \mathcal{T}_X(U) \longrightarrow \mathcal{T}_X[D](U), \end{equation*}
which is defined by
\begin{equation*} \sum_{p \in X} r_p(-) \cdot p \longmapsto \sum_{p \in X} t_D(r_p)(-) \cdot p, \end{equation*}
where
\begin{equation*} t_D(r_p)(z_p) = \sum_{i = - n_p}^{- D(p) - 1} a_i \, z_p^i.\end{equation*}

\paragraph{Meromorphic Field.} Let us consider the field
\begin{equation*} \M := \left\{ \text{field of meromorphic function on $X$} \right\}. \end{equation*}
The constant presheaf may be also defined by setting
\begin{equation*} \M_X(U) := \left\{ f : U \to \M \: \left| \: \text{$f$ continuous and $\M$ has the discrete topology} \right. \right\}, \end{equation*}
in such a way that
\begin{equation*} \text{$U$ connected} \implies \M_X(U) \cong \M, \end{equation*}
and the restriction maps are the identity maps. If we denote by $\M_X$ the associated sheaf, then one can prove that \mbox{}
\begin{enumerate}[label=\textbf{(\alph*)}]
\item $H^0 \left(X, \, \M_X \right) \cong \M$, and
\item $H^1 \left(X, \, \M_X \right) = 0$.
\end{enumerate}
In particular, for every divisor $D \in \Div(X)$ there exists a homomorphism of sheaves
\begin{equation*} \alpha_D : \M_X \longrightarrow \mathcal{T}_X[D], \end{equation*}
which can be easily defined \textit{locally} as
\begin{equation*} p \in U_p \implies f(z_p) = \sum_{i \geq - n_p} a_i \, z_p^i \longmapsto r_p(z_p) = \sum_{i = - n_p}^{- D(p) - 1} a_i \, z^i. \end{equation*}
By definition, the kernel of $\alpha_D$ is isomorphic to $\mathcal{O}_X[D]$; hence there is a short exact sequence of sheaf maps
\begin{equation*} 0 \xrightarrow{} \mathcal{O}_X[D] \xrightarrow{} \M_X \xrightarrow{\alpha_D} \mathcal{T}_X[D] \xrightarrow{} 0 \end{equation*}
inducing a long exact sequence in cohomology, that is,
\begin{equation*} 0 \xrightarrow{} H^0 \left(X, \, \mathcal{O}_X[D] \right) \xrightarrow{} H^0 \left(X, \, \M_X \right) \xrightarrow{\alpha_D} H^0 \left(X, \, \mathcal{T}_X[D] \right) \xrightarrow{} H^1 \left(\mathcal{O}_X[D] \right) \xrightarrow{} 0. \end{equation*}
By \hyperref[rmk:sdksdkskdksd]{Remark \ref{rmk:sdksdkskdksd}} we can infer that
\begin{equation*} H^0 \left(X, \, \M_X \right) \cong \M(X) \qquad \text{and} \qquad H^0 \left(X, \, \mathcal{T}_X[D] \right) \cong \mathcal{T}_X[D](X), \end{equation*}
and thus it follows that
\begin{equation*} L(D) := H^0 \left(X, \, \mathcal{O}_X[D] \right) \cong \mathrm{ker}(\alpha_D) \qquad \text{and} \qquad H^1 \left(X, \, \mathcal{O}_X[D] \right) \cong \mathrm{coker}(\alpha_D). \end{equation*}
By definition, there is an isomorphism
\begin{equation*} \mathcal{O}_X[K_x - D] \cong \Omega_X^1[-D],\end{equation*}
from which it follows that
\begin{equation*} \begin{aligned} H^0 \left(X, \, \mathcal{O}_X[K_x - D]\right) & \cong H^0 \left(X, \, \Omega_X^1[-D] \right) = \\[1em] & = \left\{ \omega = f(z) \, \mathrm{d}z \: \left| \: \text{$f$ meromorphic and $\mathrm{ord}_p(f) \geq D(p)$} \right. \right\}. \end{aligned} \end{equation*}

\subsection{Proof of Serre Duality Theorem}
\index{Serre Duality Theorem}

\paragraph{Road Map.} In this section, we finally demonstrate the \hyperref[serreduality1]{Serre Theorem \ref{serreduality1}} based on what we have proved so far. The road map of the proof is the following:  \mbox{}
\begin{enumerate}[label=\textbf{(\arabic*)}]
\item There exists a pairing
\begin{equation*} \mathrm{Res}(\cdot, \, \cdot) : H^0 \left( X, \, \Omega_X^1[-D] \right) \times \mathcal{T}_X[D](X) \to \C. \end{equation*}
\item The map defined above pass to the quotient. More precisely, it turns out that
\begin{equation*} \mathrm{Res}(\cdot, \, T) \equiv 0 \qquad \forall \, T \in \mathrm{Im}(\alpha_D), \end{equation*}
and hence there exists a pairing
\begin{equation*} \mathrm{Res}(\cdot, \, \cdot) : H^0 \left( X, \, \Omega_X^1[-D] \right) \times \mathrm{coker}(\alpha_D) \to \C. \end{equation*}
\item The pairing defined in the previous step is non degenerate.
\end{enumerate}

\begin{proof}[Proof of Theorem \ref{serreduality1}] The argument is rather involved. Hence we divide it into many different steps.

\paragraph{Step 1.} Let
\begin{equation*} \mathrm{Res}(\cdot, \, \cdot) : H^0 \left( X, \, \Omega_X^1[-D] \right) \times \mathcal{T}_X[D](X) \to \C, \qquad (\omega, \, T) \mapsto \mathrm{Res}(\omega, \, T) \end{equation*}
be the map defined by
\begin{equation*} \mathrm{Res}(\omega, \, T) := \sum_{p \in X} \mathrm{Res}_p(T_p \, \omega). \end{equation*}
More precisely, if $U_p$ is an open neighborhood of a point $p \in X$ with local coordinate $z_p$, then it turns out that
\begin{equation*}\omega = \sum_{i \geq D(p)} \left( c_i \, z_p^i \right) \, \mathrm{d}z_p \qquad \text{and} \qquad T_p = \sum_{i = - s_p}^{-D(p) - 1} a_i \, z_p^i, \end{equation*}
and thus
\begin{equation*}\mathrm{Res}_p(T_p \, \omega) = \sum_{i \geq D(p)} a_{-i - 1} \cdot c_i \end{equation*}
is exactly equal to the coefficient of $z_p^{-1}$. 

\paragraph{Step 2.} In this step, the primary goal is to prove that the map defined above descends to the quotient in the second variable, that is, to
\begin{equation*} \faktor{\mathcal{T}_X[D](X)}{\mathrm{Im}(\alpha_D)}. \end{equation*}
Let $f \in \mathcal{M}$ be a meromorphic function, let $p \in X$ be any point, and let $z_p$ be the associated local coordinate; then
\begin{equation*} f(z_p) = \sum_{i \geq - n_p} a_i \, z_p^i \longmapsto \alpha_D(f)(z_p) = \sum_{i \geq - n_p}^{- D(p) - 1} a_i \, z_p^i. \end{equation*}
The residue at $p$ is thus given by
\begin{equation*}\mathrm{Res}_p(f \cdot \omega) = \sum_{i \geq D(p)} a_{-i - 1} \cdot c_i = \mathrm{Res}_p(\alpha_D(f) \cdot \omega) \end{equation*}
since the terms whose index is $j \geq - D(p)$ of $\alpha_D(f)$ do not give any contribution to the sum above. The \hyperref[residui]{Residue Theorem \ref{residui}} immediately implies that
\begin{equation*}\sum_{p \in X} \mathrm{Res}_p(f \cdot \omega) = 0 \implies \mathrm{Res} \left( \alpha_D(f) \cdot \omega \right) = 0,\end{equation*}
which is exactly what we wanted to prove.

\paragraph{Step 3.} In this step, we want to prove that the functional
\begin{equation} \label{serre:fun} \mathrm{Res} : H^0 \left(X, \, \mathcal{O}_X[K_X - D] \right) \to \left( H^1 \left(X, \, \mathcal{O}_X[D] \right) \right)^v, \qquad \omega \mapsto \mathrm{Res}(\omega, \, -) \end{equation}
is an isomorphism (i.e., the pairing is non degenerate). \mbox{}
\begin{enumerate}[label=\textbf{(\alph*)}]
\item \textbf{Linear.} The linearity of the map \eqref{serre:fun} follows easily from the properties of the residue.
\item \textbf{Injective.} Let $D = \sum_{p \in X} D(p) \cdot p$, and let $\omega$ be such that
\begin{equation*} \mathrm{Res}\left(T, \, \omega \right) = 0, \qquad \forall \, T = \sum_{p \in X} T_p \cdot p. \end{equation*}
Let $p \in X$ be a point, let $z_p$ be the local coordinate, and let $k = \mathrm{ord}_p(\omega)$ (in particular, $- 1 - k < - D(p)$). It follows that
\begin{equation*} z_p^{-1-k} \cdot p \in \mathcal{T}_X[D](X), \end{equation*}
and also that, if we set $\omega := \sum_{i \geq k} \left( c_i \, z_p^i \right) \, \mathrm{d}z_p$, with the lowest coefficient $c_k$ different from $0$, then one can easily check that
\begin{equation*} \mathrm{Res}(\omega, \, z^{-1-k} \cdot p) = \mathrm{Res}_p (z_p^{-1-k} \cdot \sum_{i \geq k} c_i \, z_p^i \, \mathrm{d}z_p^i = c_k, \end{equation*}
which is not zero. This contradiction shows that $\mathrm{Res}(\omega, \, -)$ cannot be the identically zero map on $H^1\left(X, \, \mathcal{O}_X[D] \right)$, unless $\omega = 0$.
\item \textbf{Surjective.} Recall that
\begin{equation*} H^1\left(X, \, \mathcal{O}_X[D] \right) \cong \faktor{ \mathcal{T}_X[D](X) }{\mathrm{Im}(\alpha_D)}. \end{equation*}
Let us consider a functional $\Phi : \mathcal{T}_X[D](X) \to \C$ vanishing on the image of $\alpha_D$, that is, assume that
\begin{equation*} \Phi \, \big|_{\mathrm{Im}(\alpha_D)} \equiv 0. \end{equation*}
We want to construct a differential form $\omega \in H^0 \left(X, \, \Omega_X^1[-D] \right)$ such that
\begin{equation*} \Phi(-) = \mathrm{Res}(\omega, \, -). \end{equation*}
The proof of this property is a consequence of two technical lemmas; hence we interrupt the argument for a few pages and resume it when we are ready to conclude.
\end{enumerate}
\end{proof}

\paragraph{Truncation Maps.} Let $D_1, \, D_2 \in \Div(X)$ be two divisors, and assume that $D_1 \leq D_2$. There exists a truncation map
\begin{equation*} t_{D_2}^{D_1} : \mathcal{T}_X[D_1](X) \longrightarrow \mathcal{T}_X[D_2](X), \end{equation*}
which is defined by
\begin{equation*} \sum_{i \geq -n_p}^{- D_1(p) - 1} a_i \, z^i \longmapsto \sum_{i \geq -n_p}^{- D_2(p) - 1} a_i \, z^i. \end{equation*}
Let $D \sim D^\prime$ be linearly equivalent divisors (i.e., $D^\prime = D - \mathrm{div}(f)$). Let $p \in X$ be a point, and let $r_p \in \mathcal{T}_X[D](X)$ be the Laurent tail given by
\begin{equation*} r_p(z_p) = \sum_{i \geq - n_p}^{-D(p) - 1} a_i \, z_p^i. \end{equation*}
There is a unique integer $h$ such that $f(z_p) = z_p^h$ (i.e., it is a map of order $h$ at $p$, and $z_p$ is the local coordinate at $p$), and thus
\begin{equation*} \left(f \cdot r_p\right) (z_p) = \sum_{i \geq - n_p}^{-D(p) - 1} a_i \, z^{i + h} \qquad \text{and} \qquad \mathrm{deg}(f \cdot r_p) < - D(p) + \mathrm{ord}_p \, f = - D^\prime(p). \end{equation*}
We conclude that there exists an isomorphism
\begin{equation*} \mu_f : \mathcal{T}_X[D](X) \xrightarrow{ \quad \sim \quad } \mathcal{T}_X[D - \mathrm{div} \, f](X), \end{equation*}
which is defined by
\begin{equation*}\sum_{p \in X} r_p(-) \cdot p \longmapsto \sum_{p \in X} \left( f \cdot r_p \right)(-) \cdot p. \end{equation*}
To prove that $\mu_f$ is an actual isomorphism, it is enough to check that the map
\begin{equation*} \mu_{\frac{1}{f}} : \mathcal{T}_X[D - \mathrm{div}(f)](X) \longrightarrow \mathcal{T}_X[D](X) \end{equation*}
is the inverse.

\begin{remark} \mbox{}
\begin{enumerate}[label=\textbf{(\alph*)}]
\item It may be useful to rewrite the isomorphism as
\begin{equation*} \mu_f : \mathcal{T}_X[D + \mathrm{div} \, f](X) \xrightarrow{\quad \sim \quad} \mathcal{T}_X[D](X). \end{equation*}
\item Let $\Phi : \mathcal{T}_X[D](X) \longrightarrow \C$ be a linear functional. If $\Phi$ vanishes on $\mathrm{Im} \, \alpha_{D}$, then the composition $\Phi \circ \mu_f$ vanishes on the whole image of $\alpha_{D + \mathrm{div} \, f}$.
\end{enumerate} \end{remark}

\begin{lemma}[\cite{miranda}] \label{lemma:serre1} Let $\Phi_1$ and $\Phi_2$ be two linear functionals defined on $H^1\left(X, \, \mathcal{O}_X[A]\right)$ for some divisor $A \in \Div(X)$. There is a positive divisor $C$ and nonzero meromorphic functions $f_1, \, f_2 \in H^0\left(X, \, \mathcal{O}_X[C]\right)$ such that
\begin{equation*} \Phi_1 \circ t_{A}^{A - C - \mathrm{div} \, f_1} \circ \mu_{f_1} = \Phi_2 \circ t_A^{A - C - \mathrm{div} \, f_2} \circ \mu_{f_2} \end{equation*}
as functionals on $H^1\left(X, \, \mathcal{O}_X[A - C]\right)$. In other words, the two maps on $\mathcal{T}_X[A - C](X)$ in the diagram
\begin{equation*} \begin{tikzcd}& \mathcal{T}_X[A - C - \mathrm{div} \, f_1](X) \ar[r, "t"] & \mathcal{T}_X[A](X) \ar[dr, "\Phi_1"] & 
\\ \mathcal{T}_X[A - C](X) \ar[ur, "\mu_{f_1}"] \ar[dr, "\mu_{f_2}"]& & & \C \\
& \mathcal{T}_X[A - C - \mathrm{div}\, f_2](X) \ar[r, "t"] & \mathcal{T}_X[A](X) \ar[ur, "\Phi_2"] & \end{tikzcd} \end{equation*}
are equal for some $C$ and some $f_1, \, f_2 \in H^0\left(X, \, \mathcal{O}_X[C]\right) \setminus \{0\}$. \end{lemma}

\begin{proof}We argue by contradiction. Suppose that no such divisor $C$ and functions $f_i$ exist. Then for every positive divisor $C$ it turns out that the $\C$-linear map
\begin{equation*} H^0\left(X, \, \mathcal{O}_X[C]\right) \times H^0\left(X, \, \mathcal{O}_X[C]\right) \to \left(H^1\left(X, \, \mathcal{O}_X[A-C]\right)\right)^v \end{equation*} 
defined by sending a pair $(f_1, \, f_2)$ to
\begin{equation*} \Phi_1 \circ t_{A}^{A - C - \mathrm{div}(f_1)} \circ \mu_{f_1} - \Phi_2 \circ t_A^{A - C - \mathrm{div}\, f_2} \circ \mu_{f_2} \end{equation*}
is injective. In particular, for every such $C$ we must have
\begin{equation} \label{lemma:serre1:eq1}  h^1 \left(X, \, \mathcal{O}_X[A-C]\right) \geq 2 \cdot h^0\left(X, \, \mathcal{O}_X[C]\right), \end{equation}
and, as a consequence of the \hyperref[RiemannRoch]{Riemann-Roch Theorem \ref{RiemannRoch}}, we can also infer that
\begin{equation} \label{lemma:serre1:eq2} h^1 \left(X, \, \mathcal{O}_X[A-C]\right) = h^0 \left(X, \, \mathcal{O}_X[A-C]\right) + g(X) - 1 - \mathrm{deg}(A - C). \end{equation}
The divisor $C$ is positive; hence
\begin{equation*}h^0 \left(X, \, \mathcal{O}_X[A-C]\right) \leq h^0 \left(X, \, \mathcal{O}_X[A]\right) \qquad \text{and} \qquad \mathrm{deg}(A - C) \leq \mathrm{deg}(A). \end{equation*}
It follows from \eqref{lemma:serre1:eq1} and the \hyperref[RiemannRoch]{Riemann-Roch Theorem \ref{RiemannRoch}} that
\begin{equation*}h^1 \left(X, \, \mathcal{O}_X[A-C]\right) \geq 2 \cdot h^0\left(X, \, \mathcal{O}_X[C]\right) \geq 2 \left[ \mathrm{deg}(C) + 1 - g(X) \right] =2 \, \mathrm{deg}(C) + K_1, \end{equation*}
where $K_1$ is a constant, and it follows from  \eqref{lemma:serre1:eq2} that
\begin{equation*} h^1 \left(X, \, \mathcal{O}_X[A-C]\right) \leq \mathrm{deg}(C) + \left( h^0\left(X, \, \mathcal{O}_X[A] \right) + g(X) - 1 - \mathrm{deg}(A) \right) = \mathrm{deg}(C) + K_2, \end{equation*}
where $K_2$ is another constant. These growth rate are clearly incompatible for $\mathrm{deg}(C)$ sufficiently big, and this gives the sought contradiction.
\end{proof}

\begin{lemma}[\cite{miranda}] \label{lemma:serre2} Let $D_1 \in Div(X)$ be a divisor, and let $\omega \in H^0\left(X, \, \Omega_X^1[-D_1] \right)$ be a differential form. Suppose that there is another divisor $D_2 \geq D_1$ such that the residue map
\begin{equation*} \mathrm{Res}(\omega, \, -) : \mathcal{T}_X[D_1](X) \to \C \end{equation*}
vanishes on the kernel
\begin{equation*} \mathrm{Ker}\left( t_{D_2}^{D_1}  :  \mathcal{T}_X[D_1](X) \longrightarrow \mathcal{T}[D_2](X) \right).\end{equation*}
Then $\omega$ belong to $H^0 \left(X, \, \Omega_X^1[-D_2] \right)$. \end{lemma}

\begin{proof}We argue by contradiction. If $\omega \notin H^0 \left(X, \, \Omega_X^1[-D_2] \right)$, then there exists a point $p \in X$ with $k = \mathrm{ord}_p(\omega) < D_2(p)$. Let us consider the Laurent tail divisor
\begin{equation*} Z = z_p^{-k-1} \cdot p. \end{equation*}
Then $Z \in \mathrm{ker}(t_{D_2}^{D_1})$, but the residue map does not vanish; this contradiction proves the lemma.\end{proof}

\begin{proof}[Proof of Theorem \ref{serreduality1}, Part II] We are now ready to finish the proof of the Serre duality theorem. \mbox{}
\begin{enumerate}[label=\textbf{(\alph*)}]
\setcounter{enumi}{3}
\item \textbf{Surjective, Part II.} Let $\Phi : H^1 \left(X, \, \mathcal{O}_X[D] \right) \to \C$ be a functional, which we consider as a functional on $\mathcal{T}_X[D](X)$, vanishing on $\alpha_D(\mathcal{M}_X)$.

Let $\omega$ be a holomorphic $1$-form, and let $K = \mathrm{div}(\omega)$ be a canonical divisor so that
\begin{equation*} \omega \in H^0 \left(X, \, \mathcal{O}_X[K] \right) = H^0 \left(X, \, \Omega_X^1 \right). \end{equation*}
Let $A \in \Div(X)$ be a divisor such that $A \leq D$ and $A \leq K$, so that $\omega \in  H^0 \left(X, \, \Omega_X^1[-A] \right)$. Let us set $\Phi_A := \Phi \circ t_D^A : \mathcal{T}_X[A](X) \to \C$. By \hyperref[lemma:serre1]{Lemma \ref{lemma:serre1}} it turns out that there exists a divisor $C \geq 0$ and $f_1, \, f_2$ meromorphic functions such that
\begin{equation} \label{proofserre:eq1} \Phi_A \circ t_A^{A - C - \mathrm{div} \, f_1} \circ \mu_{f_1} = \mathrm{Res}(\omega, \, -) \circ t_A^{A - C - \mathrm{div} \, f_2} \circ \mu_{f_2}. \end{equation}
In the right-hand side of \eqref{proofserre:eq1} we have the map $\mathrm{Res}(\omega, \, -) \circ t_A^{A - C - \mathrm{div} \, f_2}$, which is nothing else than the residue map $\mathrm{Res}(\omega, \, -)$ acting on $\mathcal{T}_X[A - C - \mathrm{div} \, f_2](X)$; on the other hand, the composition $\mathrm{Res}(\omega, \, -) \circ \mu_{f_2}$ is exactly equal to
\begin{equation*} \mathrm{Res}(f_2 \cdot \omega, \, -) : \mathcal{T}_X[A-C](X) \to \C, \end{equation*}
and hence the identity \eqref{proofserre:eq1} becomes
\begin{equation*} \Phi_A \circ t_A^{A - C - \mathrm{div} \, f_1} \circ \mu_{f_1} = \mathrm{Res}(f_2 \cdot \omega, \, -). \end{equation*}
Composing with $\mu_{1/f_1}$ it turns out that
\begin{equation*} \Phi_A \circ t_A^{A - C - \mathrm{div}\, f_1} = \mathrm{Res}\left(\frac{f_2}{f_1} \cdot \omega, \, - \right) \end{equation*}
as functionals on $\mathcal{T}_X[A - C - \mathrm{div}\, f_1](X)$.

We observe that $(f_2/f_1) \, \omega$ belongs to $H^0\left(X, \, \Omega_X^1[C + \mathrm{div}\, f_1 - A] \right)$, and also that
\begin{equation*} \mathrm{Res}\left(\frac{f_2}{f_1} \cdot \omega, \, - \right) \equiv 0 \quad \text{on $\mathrm{Ker}\left( t_A^{A - C - \mathrm{div} \, f_1 } \right)$}. \end{equation*}
By \hyperref[lemma:serre2]{Lemma \ref{lemma:serre2}} we have that $(f_2/f_1) \, \omega \in H^0 \left( X, \, \Omega_X^1[-A] \right)$, and hence
\begin{equation*} \mathrm{Res}\left(\frac{f_2}{f_1} \cdot \omega, \, - \right) = \Phi_A. \end{equation*}
By definition, the map $\Phi_A$ is the composition between $\Phi$ and $t_D^A$; hence the residue map above vanishes on the kernel of $\mathrm{Ker}(t_D^A)$, which, in turn, implies that
\begin{equation*} \frac{f_2}{f_1} \cdot \omega \in H^0 \left( X, \, \Omega_X^1[- D] \right) \implies \Phi = \mathrm{Res} \left( \frac{f_2}{f_1} \cdot \omega, \, - \right) : H^1 \left(X, \, \mathcal{O}_X[D]\right) \longrightarrow \C, \end{equation*}
and this completes the proof of the theorem.
\end{enumerate} \end{proof}

\section{The Equality of the Three Genera}

\begin{corollary} \label{serredualitycor}Let $X$ be a compact connected Riemann surface and let $K_X$ be a canonical divisor. There is an isomorphism
\begin{equation*} H^1 \left(X, \, \mathcal{O}_X \right)^{v} \cong H^0 \left(X, \, \mathcal{O}_X \left[K_X \right] \right). \end{equation*}\end{corollary}

\begin{definition}[Geometric Genus]\index{Geometric genus} The \textit{geometric genus} of a Riemann surface $X$ is defined by
\begin{equation*} p_g(X) := h^0 \left( X, \, \mathcal{O}_X[K_X] \right). \end{equation*}
\end{definition}

\begin{corollary} Let $X$ be a compact connected Riemann surface. The three notions of genus are equivalent, that is,
\begin{equation*} p_a(X) = p_g(X) = g(X). \end{equation*}\end{corollary}

\begin{proof} The \hyperref[serreduality1]{Serre Duality Theorem \ref{serreduality1}} immediately implies that
\begin{equation*}p_a(X) = h^1 \left(X, \, \mathcal{O}_X \right) = h^0 \left(X, \, \mathcal{O}_X[K_X] \right) = p_g(X),\end{equation*}
therefore it remains to prove that one of them also coincides with the \textit{number of holes} $g(X)$.

The \hyperref[th:rh]{Riemann-Hurwitz formula \ref{th:rh}} asserts that $\mathrm{deg}(K_X) = 2 \left(g(X) - 1 \right)$, while the \hyperref[RiemannRoch]{Riemann-Roch Theorem \ref{RiemannRoch}} asserts that
\begin{equation*}h^0 \left( X, \, \mathcal{O}_X[D] \right) - h^1 \left( X, \, \mathcal{O}_X[D] \right) = \mathrm{deg} \, D +1 - p_a(X). \end{equation*}
The \hyperref[serreduality1]{Serre Duality Theorem \ref{serreduality1}} implies
\begin{equation*}2g - 1 = 2 \, p_a(X) - h^1 \left( X, \, \mathcal{O}_X[K_X] \right), \end{equation*}
hence it suffices to prove that $h^1 \left( X, \, \mathcal{O}_X[K_X] \right) = 1$. But this is, once again, a simple consequence of the Serre duality:
\begin{equation*} h^1 \left( X, \, \mathcal{O}_X[K_X] \right) = h^0 \left( X, \, \mathcal{O}_X \right)=  1. \end{equation*}\end{proof}

\section{Analytic Interpretation (Hodge)}
\index{Hodge analytic interpretation}

Let $\Omega_X^1$ be the sheaf of holomorphic $1$-form. There exists a short exact sequence of sheaves
\begin{equation*} 0 \xrightarrow{} \C \xrightarrow{} \mathcal{O}_X \xrightarrow{f \mapsto \mathrm{d}f} \Omega_X^1 \xrightarrow{} 0,\end{equation*}
where the middle map is locally defined as follows:
\begin{equation*}f(z) \longmapsto \mathrm{d}f(z) := f^\prime(z) \, \mathrm{d}z.\end{equation*}
The long exact sequence in cohomology is thus given by
\begin{equation*}\begin{aligned} 0 \xrightarrow{} H^0 \left(X, \, \C \right) & \xrightarrow{} H^0 \left(X, \, \mathcal{O}_X \right) \xrightarrow{} H^0 \left( X, \, \Omega_X^1 \right) \xrightarrow{} H^1 \left(X, \, \C \right) \xrightarrow{} \dots \\[1em] & \dots \xrightarrow{} H^1 \left(X, \, \mathcal{O}_X \right) \xrightarrow{} H^1 \left( X, \, \Omega_X^1 \right) \xrightarrow{} H^2 \left( X, \, \C \right) \xrightarrow{} 0. \end{aligned} \end{equation*}
Clearly $H^i \left(X, \, \C \right) \cong H^i \left(X, \, \R \right) \otimes_{\R} \C$ implies that
\begin{equation*} \begin{cases} H^0 \left(X, \, \C \right) \cong \C, \\[0.6em] H^1 \left(X, \, \C \right) \cong \C^{2g}, \\[0.6em] H^2 \left(X, \, \C \right) \cong \C, \end{cases} \end{equation*}
while
\begin{equation*} \begin{cases} H^0 \left(X, \, \mathcal{O}_X \right) \cong \C \implies H^0 \left(X, \, \mathcal{O}_X \right) \cong H^0 \left(X, \, \C \right), \\[0.6em] H^0 \left(X, \, \Omega_X^1 \right) = H^0 \left(X, \, \mathcal{O}_X[K_X] \right), \\[0.6em] H^1 \left(X, \, \Omega_X^1 \right) = H^1 \left(X, \, \mathcal{O}_X[K_X] \right) \cong H^0 \left(X, \, \mathcal{O}_X \right) \cong \C \cong H^2 \left(X, \, \C \right). \end{cases} \end{equation*}
We infer that there is a short exact sequence
\begin{equation*}0 \xrightarrow{} H^0 \left( X, \, \Omega_X^1 \right) \xrightarrow{} H^1 \left(X, \, \C \right) \xrightarrow{}  H^1 \left(X, \, \mathcal{O}_X \right) \xrightarrow{}  0, \end{equation*}
and it induces an equality on the dimensions of the cohomology groups, i.e.,
\begin{equation*} p_a(X) + p_g(X) = 2 \, g. \end{equation*}
\chapter{Applications of Riemann-Roch Theorem} \thispagestyle{empty}
\label{chap:10}

In this chapter, we fully exploit the Serre duality theorem and the Riemann-Roch theorem to show major results about both high-degree and low-degree divisors.

Moreover, we prove that, if $X$ is a compact Riemann surface and $D \in \Div(X)$ is a divisor such that $\mathrm{deg} \, D \geq 2 \, g(X) + 1$, then the analytic manifold
\begin{equation*} \varphi_{|D|} (X) := Y \subseteq \p^n \end{equation*}
is also an algebraic curve.

In the last sections, we investigate the \textit{canonical map}, and we set the ground for the notorious \textit{Clifford theorem}.

\section{Very Ample Divisors}

\begin{definition}[Very Ample Divisor]\index{Divisor!very ample} Let $X$ be a holomorphic manifold. A divisor $D \in \Div(X)$ is said \textit{very ample} if the associated map
\begin{equation*}\phiD : X \to \p^n(\C) = \p \left( H^0 \left(X, \, \mathcal{O}_X[D] \right)^v \right)\end{equation*}
is an \textit{embedding}. \end{definition}

\begin{remark}In particular, a divisor $D \in \Div(X)$ is very ample if and only if \mbox{}
\begin{enumerate}[label=\textbf{\arabic*})]
\item the linear system $|D|$ is b.p.f.;
\item the morphism $\phiD$ is injective;
\item the differential $\mathrm{d}\left(\phiD\right)_p$ is injective at all points $p \in X$.
\end{enumerate}\end{remark}

\begin{definition}[Ample]\index{Divisor!ample} Let $X$ be a holomorphic manifold. A divisor $D \in \Div(X)$ is \textit{ample} if there exists a natural number $k \in \N$ such that the divisor $k \cdot D$ is very ample. \end{definition}

\begin{remark}Let $D$ be a b.p.f. divisor. The morphism $\phiD : X \to \p^n(\C)$ is defined by
\begin{equation*} p \longmapsto \left( \sigma_0(p), \, \dots, \, \sigma_n(p) \right), \end{equation*}
where $\sigma_0, \, \dots, \, \sigma_n$ is a basis for $H^0 \left(X, \, \mathcal{O}_X[D] \right)$.\mbox{}
\begin{enumerate}[label=\textbf{\arabic*})]
\item The morphism $\phiD$ is injective if and only if, for any $p, \, q \in X$, there is a global section $f \in H^0 \left(X, \, \mathcal{O}_X[D] \right)$ such that $f(p) \neq f(q)$.

Equivalently, $\phiD$ is injective if and only if \mbox{}
\begin{enumerate}[label=\textbf{(\alph*)}]
\item there is $f \in H^0 \left(X, \, \mathcal{O}_X[D] \right)$ such that $f(p) = 0$ and $f(q) \neq 0$; 
\item there exists $E \in |D|$ such that $p \in \mathrm{spt}(E)$ and $q \notin \mathrm{spt}(E)$.
\end{enumerate}
\item The differential $\mathrm{d}\left(\phiD\right)_p$ sends the tangent space $T_P \, X$ to the tangent $T_{\phiD(P)} \, \p^n$, thus it is injective if and only if there exists $f \in H^0 \left(X, \, \mathcal{O}_X[D] \right)$ such that $\mathrm{ord}_p \left(f - f(p) \right) = 1$.
\end{enumerate}
\end{remark}

\begin{theorem}[Very Ample - Numerical Criterion]\index{Numerical Criterion}\label{doasodls} Let $X$ be a compact connected Riemann surface, and let $D \in \Div(X)$ be a divisor. \mbox{}
\begin{enumerate}[label=\textbf{(\alph*)}]
\item The linear system $|D|$ is b.p.f. if and only if, for any $p \in X$, it turns out that
\begin{equation*} h^0 \left( X, \, \mathcal{O}_X[D - p] \right) =  h^0 \left( X, \, \mathcal{O}_X[D] \right) - 1. \end{equation*}
\item The divisor $D$ is very ample if and only if, for any $p, \, q \in X$ (eventually $p=q$), it turns out that
\begin{equation*} h^0 \left( X, \, \mathcal{O}_X[D - p - q] \right) =  h^0 \left( X, \, \mathcal{O}_X[D] \right) - 2. \end{equation*}
\end{enumerate}
\end{theorem}

\begin{proof} \mbox{}
\begin{enumerate}[label=\textbf{(\alph*)}]
\item Let $p \in X$. By \hyperref[exas]{Proposition \ref{exas}} there is a short exact sequence of sheaf maps
\begin{equation*} 0 \xrightarrow{} \mathcal{O}_X[D - p] \xrightarrow{} \mathcal{O}_X[D] \xrightarrow{} \C_p \xrightarrow{} 0,\end{equation*}
which induces a long exact sequence in cohomology (see \hyperref[longcomo]{Theorem \ref{longcomo}}):
\begin{equation*} 0 \xrightarrow{} H^0 \left(X, \, \mathcal{O}_X[D - p] \right) \xrightarrow{} H^0 \left(X, \, \mathcal{O}_X[D] \right) \xrightarrow{f \mapsto f(p)} H^0\left(X, \, \C_p \right) \xrightarrow{} \dots.\end{equation*}
Observe that there exists a global section $f \in H^0 \left(X, \, \mathcal{O}_X[D] \right)$ such that $f(p) \neq 0$ if and only if $H^0 \left(X, \, \mathcal{O}_X[D] \right) \longtwoheadrightarrow \C_p$, which is equivalent to the identity
\begin{equation*} H^0 \left(X, \, \mathcal{O}_X[D - p] \right) = \mathrm{Ker} \left(H^0 \left(X, \, \mathcal{O}_X[D] \right) \xrightarrow{f(p)} H^0 \left(X, \, \C_p \right) \right). \end{equation*}
The dimension of $H^0(X, \, \C_p)$ is equal to $1$, thus we can infer that
\begin{equation*} h^0\left(X, \, \mathcal{O}_X[D - p] \right) = h^0 \left( X, \, \mathcal{O}_X[D] \right) - 1, \end{equation*}
which is exactly what we wanted to prove.
\item Assume that $|D|$ is a b.p.f. linear system.

\paragraph{Step 1.} Suppose that there are two points $p, \, q \in X$ such that $\phiD(p) = \phiD(q)$, that is, $\phiD$ is \textit{not} injective. It follows that, as vector spaces,
\begin{equation*} H^0 \left(X, \, \mathcal{O}_X[D - p] \right) = H^0 \left(X, \, \mathcal{O}_X[D-q] \right), \end{equation*}
which in turn implies that
\begin{equation*} h^0 \left(X, \, \mathcal{O}_X[D - p] \right) = h^0 \left(X, \, \mathcal{O}_X[D-q] \right) = h^0 \left(X, \, \mathcal{O}_X[D] \right) - 1. \end{equation*}
On the other hand, the assumption allows us to infer that
\begin{equation*} H^0 \left(X, \, \mathcal{O}_X[D - p] \right) = H^0 \left(X, \, \mathcal{O}_X[D - p - q] \right)  =  H^0 \left(X, \, \mathcal{O}_X[D-q] \right), \end{equation*}
as vector spaces, and hence
\begin{equation*} h^0 \left(X, \, \mathcal{O}_X[D - p - q] \right) =  h^0 \left(X, \, \mathcal{O}_X[D] \right) - 1 \neq h^0 \left(X, \, \mathcal{O}_X[D] \right) - 2.  \end{equation*}
Vice versa, suppose that the formula does not hold true, i.e., there are $p, \, q \in X$ such that
\begin{equation*} h^0 \left(X, \, \mathcal{O}_X[D - p - q] \right) =  h^0 \left(X, \, \mathcal{O}_X[D] \right) - 1. \end{equation*}
By \hyperref[exas]{Proposition \ref{exas}} there is a short exact sequence
\begin{equation*} 0 \xrightarrow{} \mathcal{O}_X[D - p - q] \xrightarrow{} \mathcal{O}_X[D - p] \xrightarrow{} \C_{q} \xrightarrow{} 0, \end{equation*}
which induces a long exact sequence in cohomology (see \hyperref[longcomo]{Theorem \ref{longcomo}}):
\begin{equation*} 0 \xrightarrow{} H^0 \left(X, \, \mathcal{O}_X[D - p - q] \right) \xrightarrow{} H^0 \left(X, \, \mathcal{O}_X[D - p] \right) \xrightarrow{} H^0\left(X, \, \C_q \right) \xrightarrow{} \dots.\end{equation*}
The assumption on the dimension proves that
\begin{equation*} H^0 \left(X, \, \mathcal{O}_X[D - p - q] \right) \cong H^0 \left(X, \, \mathcal{O}_X[D - p] \right), \end{equation*}
is an isomorphism, and hence
\begin{equation*}H^0 \left( X, \, \mathcal{O}_X[D - p] \right) = \mathrm{Ker} \left( H^0 \left(X, \, \mathcal{O}_X[D - p] \right) \xrightarrow{f \mapsto f(q)} H^0\left(X, \, \C_q \right) \right). \end{equation*}
In particular, for any $f \in H^0 \left(X, \, \mathcal{O}_X[D] \right)$ it follows that
\begin{equation*}f(p) = 0 \leadsto f(q) = 0, \end{equation*}
since 
\begin{equation*}f(p) = 0 \implies f \in H^0 \left(X, \, \mathcal{O}_X[D - p] \right) \implies f(q) = 0. \end{equation*}

\paragraph{Step 2.} We now want to prove that $\mathrm{d}\left(\phiD\right)_p$ is injective if and only if
\begin{equation*} h^0 \left( X, \, \mathcal{O}_X[D - 2\cdot p] \right) =  h^0 \left( X, \, \mathcal{O}_X[D] \right) - 2. \end{equation*}
First, we observe that
\begin{equation*} \begin{tikzcd}[ifandonly/.style = {draw=none,"\iff" description,sloped}]
 \text{$\mathrm{d}\left(\phiD\right)_p : T_p \, X \to T_{\phiD(p)} \, \p^n$ is injective} \arrow[rr, ifandonly] \arrow[d, ifandonly] & & H^0 \left(X, \, \mathcal{O}_X[D] \right) \twoheadrightarrow \faktor{\mathcal{M}_{X, \, p}}{\mathcal{M}_{X, \, p}^2} \arrow[lld, ifandonly] \\ \exists \, f \in H^0 \left(X, \, \mathcal{O}_X[D] \right) \: : \: \mathrm{ord}_p\left(f - f(p) \right) = 1. & \end{tikzcd} \end{equation*}
If we set $\Delta = 2 \cdot p$, then one can easily prove that
\begin{equation*} \begin{cases} H^0 \left(X, \, \mathcal{O}_X[D] \right) \longtwoheadrightarrow H^0\left(X, \, \C_p \right) \\ \\ H^0 \left(X, \, \mathcal{O}_X[D] \right) \longtwoheadrightarrow \faktor{\mathcal{M}_{X, \, p}}{\mathcal{M}_{X, \, p}^2} \end{cases} \iff H^0 \left(X, \, \mathcal{O}_X[D] \right) \longtwoheadrightarrow H^0 \left(X, \, \mathcal{O}_\Delta \right). \end{equation*}
Indeed, locally $p$ corresponds to $z = 0$ and $\Delta$ corresponds to $z^2 = 0$; hence $\mathcal{O}_\Delta$ is isomorphic to the quotient $\faktor{\C[[z]]}{z^2}$. Then we have that
\begin{equation*}  H^0 \left(X, \, \mathcal{O}_X[D] \right) \longtwoheadrightarrow H^0 \left(X, \, \mathcal{O}_\Delta \right)\end{equation*}
if and only if
\begin{equation*}0 \xrightarrow{} H^0 \left(X, \, \mathcal{O}_X[D - 2\cdot p] \right) \xrightarrow{} H^0 \left(X, \, \mathcal{O}_X[D] \right) \xrightarrow{} H^0\left(X, \, \mathcal{O}_\Delta \right) \xrightarrow{} 0 \end{equation*}
is a short exact sequence; thus
\begin{equation*} h^0 \left( X, \, \mathcal{O}_X[D - 2 \cdot p ] \right) =  h^0 \left( X, \, \mathcal{O}_X[D] \right) - 2, \end{equation*}
and this is exactly what we wanted to prove.
\end{enumerate}
\end{proof}

\begin{theorem}[High-Degree Divisors]\index{Divisor!of high degree} \label{hdd}Let $X$ be a compact connected Riemann surface of genus $g(X)$, and let $D \in \Div(X)$ be a divisor of degree $d$. \mbox{}
\begin{enumerate}[label=\textbf{(\alph*)}]
\item If $d \geq 2 \,g(X) - 1$, then
\begin{equation*}H^1 \left( X, \, \mathcal{O}_X[D] \right) = \{0\}. \end{equation*}
\item If $d \geq 2 \, g(X)$, then $|D|$ is a b.p.f. linear system.
\item If $d \geq 2 \, g(X) + 1$, then $D$ is very ample.
\end{enumerate}
\end{theorem}

\begin{proof}\mbox{}
\begin{enumerate}[label=\textbf{(\alph*)}]
\item Let $K_X$ be a canonical divisor. By \hyperref[cor:dincan]{Corollary \ref{cor:dincan}} it turns out that
\begin{equation*} \mathrm{deg} \left(K_X - D \right) = \mathrm{deg} \left(K_X \right) - \mathrm{deg}\left(D\right) < 0, \end{equation*}
and hence the \hyperref[serreduality1]{Serre Duality Theorem \ref{serreduality1}} implies that
\begin{equation*} H^1 \left( X, \, \mathcal{O}_X[D] \right)^v \cong H^0 \left( X, \, \mathcal{O}_X[K_X - D] \right) = \{0\}. \end{equation*}
\item Let $p \in X$ be any point. By \hyperref[exas]{Proposition \ref{exas}} there is a short exact sequence of sheaf maps
\begin{equation*} 0 \xrightarrow{} \mathcal{O}_X[D - p] \xrightarrow{} \mathcal{O}_X[D] \xrightarrow{} \C_{p} \xrightarrow{} 0,\end{equation*}
inducing a long exact sequence in cohomology (see \hyperref[longcomo]{Theorem \ref{longcomo}}):
\begin{equation*} \begin{aligned} 0 \xrightarrow{} H^0 \left(X, \, \mathcal{O}_X[D - p] \right) & \xrightarrow{} H^0 \left(X, \, \mathcal{O}_X[D] \right) \xrightarrow{} H^0\left(X, \, \C_p \right) \xrightarrow{} \dots \\[1em] & \dots \xrightarrow{} H^1 \left(X, \, \mathcal{O}_X[D - p] \right) \xrightarrow{} H^1 \left(X, \, \mathcal{O}_X[D] \right) \xrightarrow{} 0.\end{aligned}\end{equation*}
By assumption, the degree of the divisor $D - p$ is greater of equal than $2 \, g(X) - 1$, and hence by \textbf{(a)} it follows that
\begin{equation*} H^1 \left(X, \, \mathcal{O}_X[D] \right) = 0 \qquad \text{and} \qquad H^1 \left(X, \, \mathcal{O}_X[D - p] \right) = 0. \end{equation*}
Therefore $H^0 \left(X, \, \mathcal{O}_X[D] \right) \twoheadrightarrow H^0 \left(X, \, \C_p \right)$ is surjective, and this is enough to conclude that the linear system $|D|$ is b.p.f., as a consequence of the numerical criterion (i.e.,  \hyperref[doasodls]{Theorem \ref{doasodls}}).
\item Let $p, \, q \in X$ be points, and let us set $\Delta := p + q$. There is a short exact sequence, deriving from \hyperref[exas]{Proposition \ref{exas}},
\begin{equation*} 0 \xrightarrow{} \mathcal{O}_X[D - p - q] \xrightarrow{} \mathcal{O}_X[D] \xrightarrow{} \C_{\Delta} \xrightarrow{} 0,\end{equation*}
which induces a long sequence in cohomology (see \hyperref[longcomo]{Theorem \ref{longcomo}}):
\begin{equation*} \begin{aligned} 0 \xrightarrow{} H^0 \left(X, \, \mathcal{O}_X[D - p - q] \right) & \xrightarrow{} H^0 \left(X, \, \mathcal{O}_X[D] \right) \xrightarrow{} H^0\left(X, \, \mathcal{O}_\Delta \right) \xrightarrow{} \dots \\[1em] & \dots \xrightarrow{} H^1 \left(X, \, \mathcal{O}_X[D - p - q] \right) \xrightarrow{} H^1 \left(X, \, \mathcal{O}_X[D] \right) \xrightarrow{} 0.\end{aligned}\end{equation*}
By assumption, the divisor $D - p - q$ has degree greater or equal than $2 \, g(X) - 1$, and thus it follows from \textbf{(a)} that
\begin{equation*}H^1 \left(X, \, \mathcal{O}_X[D - p - q] \right) = 0. \end{equation*}
Therefore $H^0 \left(X, \, \mathcal{O}_X[D] \right) \twoheadrightarrow H^0\left(X, \, \mathcal{O}_\Delta \right)$ is surjective and, since $H^0\left(X, \, \mathcal{O}_\Delta \right)$ has dimension $2$, it follows that
\begin{equation*}h^0\left(X, \, \mathcal{O}_X[D] \right) - 2 = H^0\left(X, \, \mathcal{O}_X[D - p - q] \right). \end{equation*}
In conclusion, the numerical criterion (\hyperref[doasodls]{Theorem \ref{doasodls}}) implies that $D$ is a very ample divisor, which is exactly what we wanted to prove.
\end{enumerate}
\end{proof}

\begin{corollary} Let $X$ be a compact connected Riemann surface. If $D \in \Div(X)$ is an effective divisor, then $D$ is ample. \end{corollary}

\begin{corollary} Let $X$ be a compact connected Riemann surface. \mbox{}
\begin{enumerate}[label=\textbf{(\arabic*)}]
\item If $g(X) = 0$, then $X \cong \p^1(\C)$.
\item If $g(X) = 1$, then $X \cong \{F_3 = 0\} \subseteq \p^2$ is a cubic plane curve.
\end{enumerate}\end{corollary}

\begin{proof}\mbox{}
\begin{enumerate}[label=\textbf{(\arabic*)}]
\item Let $p \in X$ be a point, and let us consider the divisor $D := 1\cdot p$. Clearly $\mathrm{deg} \, D \geq 2 \, g(X) + 1 = 1$, thus $D$ is very ample and $h^1 \left(X, \, \mathcal{O}_X[D] \right) = 0$. By \hyperref[RiemannRoch]{Riemann-Roch \ref{RiemannRoch}} it follows that
\begin{equation*}h^0  \left(X, \, \mathcal{O}_X[D] \right) = \mathrm{deg} \, D + 1 - g(X) = 2,\end{equation*}
and hence $\phiD : X \to \p^1$ is an embedding of degree equal to $1$; since $X$ is compact and connected, we infer that $\phiD$ is the sought isomorphism.
\item Let $p \in X$ be a point, and let us consider the divisor $D := 3\cdot p$. Clearly $\mathrm{deg} \, D \geq 2 \,g(X) + 1 = 3$, thus $D$ is very ample and $h^1 \left(X, \, \mathcal{O}_X[D] \right) = 3$. By \hyperref[RiemannRoch]{Riemann-Roch \ref{RiemannRoch}} it follows that
\begin{equation*}h^0  \left(X, \, \mathcal{O}_X[D] \right) = \mathrm{deg} \, D + 1 - g(X) = 3,\end{equation*}
therefore $\phiD : X \to \p^2$ is an embedding, and $D$ is a divisor relative to a section of an hyperplane of dimension $3$, that is, $\phiD$ is cubic (see \hyperref[rdslds]{Theorem \ref{rdslds}}).
\end{enumerate}
\end{proof}

\section{Algebraic Curves and Riemann Surfaces}

In this section, the primary goal is to prove that there exists a $1$-$1$ correspondence of categories
\begin{equation*} \left\{ \begin{gathered} \text{Riemann surfaces} \\ \text{compact and connected} \end{gathered} \right\} \xleftrightarrow{\quad \sim \quad} \left\{ \begin{gathered} \text{Smooth algebraic} \\ \text{projective curves} \end{gathered} \right\}. \end{equation*}

\paragraph{Algebraic Curves.} In this paragraph, the primary goal is to prove the following statement: If $X$ is a compact Riemann surface and $D \in \Div(X)$ is a divisor of degree $\mathrm{deg} \, D \geq 2 \, g(X) + 1$, then the analytic manifold
\begin{equation*} \varphi_{|D|} (X) := Y \subseteq \p^N(\C) \end{equation*}
is also an algebraic manifold. More precisely, we will sketch the proof of the following theorem:

\begin{theorem} \label{rdslds} Let $X$ be a compact connected Riemann surface, and let $D \in \Div(X)$ be a divisor of degree $\mathrm{deg} \, D \geq 2 \, g(X) + 1$. Then
\begin{equation*} \varphi_{|D|} (X) := Y \subseteq \p^N( \C ) \end{equation*}
is an algebraic curve, that is,
\begin{equation*} Y = \mathcal{V}(g_1, \, \dots, \, g_r),\end{equation*}
where $g_1, \, \dots, \, g_r \in \C[x_0, \, \dots, \, x_N]$ are homogeneous polynomials.\end{theorem}

\begin{remark}There is a more general theorem, proved by Wei-Liang Chow, which asserts that $Y \subset \p^N(\C)$ analytic manifold is an algebraic manifold (see \cite{chow}). \end{remark}

\begin{proof} The argument is rather involved. Hence we divide the proof into many steps, and we state and prove everything we need in this environment.

\paragraph{Step 1.} Let us consider the graded algebra
\begin{equation} \label{rd} R(D) := \bigotimes_{n \geq 0} H^0 \left(X, \, \mathcal{O}_X[ n \cdot D ] \right), \end{equation}
along with the maps
\begin{equation*}H^0 \left(X, \, \mathcal{O}_X[n \cdot D] \right) \otimes H^0 \left(X, \, \mathcal{O}_X[m \cdot D] \right) \longrightarrow H^0 \left(X, \, \mathcal{O}_X[(n+m) \cdot D] \right) \end{equation*}
defined by sending the tensor product $s \otimes t$ to $s \cdot t$, as $n$ and $m$ range in the set of all natural numbers.

\begin{example}[Projective Space] Let $X = \p^1(\C)$ with coordinates $[x_0, \, x_1]$, and let $D := [0 : 1]$ be a divisor. We have already proved that
\begin{equation*} \begin{aligned} & H^0 \left( X, \, \mathcal{O}_X[D] \right) = \left< x_0, \, x_1 \right>, \\[1em] & H^0 \left( X, \, \mathcal{O}_X[2\cdot D] \right) \cong \left< x_0^2, \, x_0 \, x_1, \, x_1^2 \right>, \end{aligned} \end{equation*}
hence
\begin{equation*} H^0 \left( X, \, \mathcal{O}_X[D] \right) \otimes H^0 \left( X, \, \mathcal{O}_X[2\cdot D] \right) \to H^0 \left( X, \, \mathcal{O}_X[3\cdot D] \right)\end{equation*}
is a well-defined map, which sends $p \otimes q$ to $p \cdot q$, where $p$ and $q$ are homogeneous polynomials of degree respectively one and two (coherently with the definition of degree for polynomials).
\end{example}

\paragraph{Step 2.} Now, we state a result concerning the graded algebra $R(D)$ which will be essential in this proof; the reader may refer to \href{https://arxiv.org/pdf/math/0311534.pdf}{this paper}.

\vspace{2.5mm}
\noindent\fbox{
\parbox{\textwidth}{ \begin{theorem}[Castelnuovo-Mumford] \label{cmm} Let $X$ be a compact Riemann surface, and let $D \in \Div(X)$ be a divisor such that $\mathrm{deg} \, D \geq 2 \, g(X) + 1$. Then the graded algebra $R(D)$, defined in \eqref{rd}, is generated in degree one, that is, for every $n \in \N$ there is a surjective map
\begin{equation*} H^0 \left(X, \, \mathcal{O}_X[D] \right)^{\otimes n} \longtwoheadrightarrow H^0 \left(X, \, \mathcal{O}_X[n\cdot D] \right). \end{equation*} \end{theorem} }}

\vspace{2.5mm}
\noindent We observe that, for any $n \in \N$, the left-hand side may be written as the direct sum between the symmetric and the antisymmetric part, i.e.,
\begin{equation*} H^0 \left(X, \, \mathcal{O}_X[D] \right)^{\otimes n} = \Lambda^n \left( H^0 \left(X, \, \mathcal{O}_X[D] \right) \right) \oplus \mathrm{Sym}^n \left(H^0 \left(X, \, \mathcal{O}_X[D] \right) \right).   \end{equation*}
The product $\cdot$ is commutative; hence
\begin{equation*} \Lambda^n \left( H^0 \left(X, \, \mathcal{O}_X[D] \right) \right) \ni t \otimes s - s \otimes t \longmapsto t \cdot s - s \cdot t = 0 \in H^0 \left(X, \, \mathcal{O}_X[n \cdot D] \right),\end{equation*}
which means that for every $n \in \N$
\begin{equation*} H^0 \left(X, \, \mathcal{O}_X[D] \right)^{\otimes n} \twoheadrightarrow H^0 \left(X, \, \mathcal{O}_X[n\cdot D] \right) \iff   \mathrm{Sym}^n \left(H^0 \left(X, \, \mathcal{O}_X[D] \right) \right) \twoheadrightarrow H^0 \left(X, \, \mathcal{O}_X[n\cdot D] \right).\end{equation*}

\paragraph{Step 3.} There is a natural identification
\begin{equation*} H^0 \left(X, \, \mathcal{O}_X[D] \right) = \left\langle x_0, \, \dots, \, x_N \right\rangle \left( \cong \left(\C^{N + 1}\right)^v \right), \end{equation*}
from which it follows that for every $n \in \N$ there is an isomorphism
\begin{equation*} \mathrm{Sym}^n \left(H^0 \left(X, \, \mathcal{O}_X[D] \right) \right) \cong \C[x_0, \, \dots, \, x_N]_n,\end{equation*}
where $\C[x_0, \, \dots, \, x_N]_n$ is the set of all homogeneous polynomials in the variables $x_0, \, \dots, \, x_N$ of degree $n$. If we set
\begin{equation*} Y := \varphi_{|D|}(X) \subseteq \p^N(\C), \end{equation*}
where $x_0, \, \dots, \, x_N$ are the coordinates of $\p^N(\C)$, then the surjective map
\begin{equation*} \bigotimes_{n \geq 0} \mathrm{Sym}^n \left(H^0 \left(X, \, \mathcal{O}_X[D] \right) \right) \longtwoheadrightarrow \bigotimes_{n \geq 0} H^0 \left(X, \, \mathcal{O}_X[n\cdot D] \right),\end{equation*}
which exists as a consequence of \hyperref[cmm]{Theorem \ref{cmm}}, induces a different surjective map - by composing with the isomorphism above -, that is,
\begin{equation*} \alpha : \C[x_0, \, \dots, \, x_N] = \bigoplus_{n \geq 0} \C[x_0, \, \dots, \, x_N]_n \longtwoheadrightarrow R(D). \end{equation*}

\paragraph{Step 4.} Let us consider the ideal
\begin{equation*} I := \mathrm{Ker}(\alpha), \end{equation*}
and let us denote by $Z$ the \textit{algebraic variety} associated with $I$, that is, set
\begin{equation*}Z := \mathcal{V}(I). \end{equation*}
Every polynomial $p \in I$ vanishes on $Y$, as a consequence of the fact that the support of the graded algebra $R(D)$ is entirely contained in $Y$. More precisely, the following inclusion holds
\begin{equation*} Y \subseteq V(I) = Z, \end{equation*}
and therefore the thesis is equivalent to showing that the opposite containment also holds.

\begin{remark}[Irreducibility] \mbox{}
\begin{enumerate}[label=\textbf{(\arabic*)}]
\item The graded algebra $R(D)$ is an integral domain since $X$ is a connected surface; hence $X$ is an irreducible surface.

On the other hand, the surface $Y$ is the image via embedding of $X$, and it is thus irreducible as well. 
\item The reader may check, as a simple exercise, that the ideal $I$ is prime; consequently, the algebraic variety $Z$ is irreducible.
\end{enumerate} \end{remark}

\begin{remark} As a consequence of the previous \textit{Remark}, it is enough to prove that the dimension $\mathrm{dim}_\C (Z)$ is equal to $1$ to  infer that $Y = Z$. \end{remark}

\paragraph{Step 5.} In this final step, we briefly introduce the concept of \textit{Hilbert polynomial}, and we state a major result - due to Hilbert and Serre - concerning the relation between the dimension of $Z$ over $\C$ and the behavior of the polynomial at infinity.

\begin{definition}[Hilbert Polynomial] Let $Z \subseteq \p^N(\C)$ be an algebraic variety. The \textit{Hilbert polynomial} associated to $Z$ is the polynomial such that
\begin{equation*} t \gg 0 \implies p_Z(t) = \mathrm{dim}_\C \, S(Z)_t, \end{equation*}
where $S(Z)_t := \faktor{\C[x_0, \, \dots, \, x_N]_t}{I}$, that is, the $t$-degree part of $S(Z)$.\end{definition}

\begin{theorem}[Hilbert-Serre] \label{hss} Let $Z \subseteq \p^N(\C)$ be a $1$-dimensional algebraic variety. For every $t$, the Hilbert polynomial is given by
\begin{equation*} p_Z(t) = a_1 \, t + a_0. \end{equation*}
In particular, it turns out that
\begin{equation*} \mathrm{deg}(p_Z) = \mathrm{dim}_\C (Z). \end{equation*} \end{theorem}

\vspace{1.8mm}
\noindent In the previous steps we have proved that there is a commutative diagram
\begin{equation*} \begin{tikzcd}[contains/.style = {draw=none,"\simeq" description,sloped}]
 \C[x_0, \, \dots, \, x_N]_t \arrow[d, contains] \arrow[r, twoheadrightarrow] & S(Z)_t \arrow[d, contains] \\ \mathrm{Sym}^t \left( H^0 \left(X, \, \mathcal{O}_X[D] \right) \right) \arrow[r, twoheadrightarrow] & H^0 \left(X, \, \mathcal{O}_X[t \cdot D] \right) \end{tikzcd} \end{equation*}
hence
\begin{equation*} p_Z(t) = \mathrm{dim}_\C \, S(Z)_t = h^0 \left(X, \, \mathcal{O}_X[t \cdot D] \right). \end{equation*}
Finally, the \hyperref[RiemannRoch]{Riemann-Roch Theorem \ref{RiemannRoch}} allows us to find the Hilbert polynomial, which is given by
\begin{equation*} h^0 \left(X, \, \mathcal{O}_X[t \cdot D] \right) = t \cdot \mathrm{deg} \, D + 1 - g(X) \implies \mathrm{dim}_\C \, Z = 1,  \end{equation*}
and this concludes the proof of \hyperref[rdslds]{Theorem \ref{rdslds}}. \end{proof}

\paragraph{Equivalence of Categories.} In this final paragraph, we state and prove three relevant results which will allow us to demonstrate the equivalence theorem mentioned at the beginning.

\begin{proposition}\label{prop:peoro}Let $Y \subseteq \p^N(\C)$ be an algebraic curve of degree $d$, and assume $N \geq 4$. Then there exists a point $O \in \p^N(\C)$ such that the canonical projection, centered at $O$, given by
\begin{equation*}\pi_O : \p^N(\C) \to \p^{N-1}(\C) \end{equation*}
has the additional property that
\begin{equation*} \pi_O(Y) \cong Y. \end{equation*}\end{proposition}

\begin{proof}The projection centered at $O$ has the property $\pi_O(Y) \cong Y$ if and only if \mbox{}
\begin{enumerate}[label=\textbf{(\alph*)}]
\item $\pi_O \, \big|_Y$ is injective, if and only if $O \notin \{ \text{secant lines to $Y$} \}$;
\item $\mathrm{d} \left(\pi_O \, \big|_Y \right)$ is injective, if and only if $O \notin \{ \text{tangent lines to $Y$} \}$.
\end{enumerate}

\paragraph{Secant.} The set of all the secant lines to $Y$ is given by
\begin{equation*} \mathrm{sec}(Y) := \overline{ \left\{ \mathrm{Span} \langle p, \, q \rangle \: \left| \: p \neq q \in Y \right. \right\} }, \end{equation*}
and hence there exists a morphism
\begin{equation*} \Phi : \p^1 \times \left( Y \times Y \setminus \Delta_Y \right) \longrightarrow \mathrm{sec}(Y), \end{equation*}
where $\Delta_Y := \{(p, \, p) \: \left| \: p \in Y \right. \}$ is the diagonal of $Y$, which is defined by
\begin{equation*}\left( [\lambda_0 :\lambda_1], \, p, \, q \right) \longmapsto \left( \lambda_0 \, p + \lambda_1 \, q \right). \end{equation*}

\paragraph{Tangent.} The set of all the tangent lines to $Y$ may be \textit{locally} identified by the isomorphism
\begin{equation*} \mathrm{tan}(Y) \cong \overline{ \left\{ \left(p, \, [\lambda_0 : \lambda_1] \right) \: \left| \: p \in Y, \, \, [\lambda_0 \: : \: \lambda_1] \in \p^1(\C) \right. \right\} }, \end{equation*}
which is defined by
\begin{equation*} \Psi : Y \times \p^1(\C) \to \mathrm{tan}(Y), \qquad \left(p, \, [\lambda_0 : \lambda_1] \right) \longmapsto p + v \cdot \frac{\lambda_0}{\lambda_1}, \end{equation*}
where $v$ is a tangent vector.

\paragraph{Dimensional Argument.} The previous points allow us to infer that
\begin{equation*}\begin{aligned} & \mathrm{dim} \left( \mathrm{sec}(Y) \right) \leq 2 \cdot \mathrm{dim}(Y) + 1 = 3, \\[1em] & \mathrm{dim} \left( \mathrm{tan}(Y) \right) \leq \mathrm{dim}(Y) + 1 = 2, \end{aligned} \end{equation*}
which, in turn, imply the following estimate on the dimension:
\begin{equation*} \mathrm{dim} \left( \mathrm{sec}(Y) \cup \mathrm{tan}(Y) \right) \leq 3. \end{equation*}
In conclusion, the assumption $N \geq 4$ is sufficient to infer that such a point $O$ - satisfying \textbf{(a)} and \textbf{(b)} - must exist.
\end{proof}

\begin{figure}[h]
\centering
\includegraphics[width = 8cm, height = 8cm]{images/GEOC1.png}
\caption{Idea of \hyperref[prop:peoro]{Proposition \ref{prop:peoro}}}
\label{fig:12i}
\end{figure} 

\begin{corollary}\label{cor:pdsppdpsd}Let $X$ be a compact connected Riemann surface. There exists an isomorphism
\begin{equation*} \Phi : X \xrightarrow{\quad \sim \quad} Y \subseteq \p^3(\C), \end{equation*}
where $Y$ is an algebraic curve.\end{corollary}

\begin{proof}Let $D \in \Div(X)$ be a divisor such that $\mathrm{deg} \, D \geq 2 \, g(X) + 1$. The morphism
\begin{equation*} \varphi_{|D|} : X \hookrightarrow Y_0 \subseteq \p^N(\C) \end{equation*}
is an embedding; if we compose it with a sequence $\pi_1, \, \dots, \, \pi_{N-3}$ of adequate projections (whose existence is a consequence of \hyperref[prop:peoro]{Proposition \ref{prop:peoro}}), then we obtain that
\begin{equation*} X \xrightarrow{\varphi_{|D|}} Y_0 \xrightarrow{\pi_{N-3} \circ \dots \circ \pi_1} Y \subseteq \p^3(\C) \end{equation*} 
is an isomorphism, since it is composition of isomorphisms.
\end{proof}

\begin{proposition}\label{prop:oewoe} Let $Y$ be an algebraic curve of degree $d$ in $\p^3(\C)$. There exists a point $O \in \p^3(\C)$ such that the projection centered at $O$
\begin{equation*}\pi_O : \p^3(\C) \to \p^2(\C) \end{equation*}
has the property that $\pi_O(Y) \subseteq \p^2(\C)$ is an algebraic curve of degree $d$, with a finite number of simple knots.\end{proposition}

\begin{figure}[h]
\centering
\includegraphics[width = 6cm, height = 8cm]{images/GEOC2.png}
\caption{Idea of the proof of \hyperref[prop:oewoe]{Proposition \ref{prop:oewoe}}}
\label{fig:123i}
\end{figure} 

\paragraph{Equivalence of Categories.} We are now ready to state and prove the main result of the section.

\begin{theorem}\label{th:eccc}There is a $1$-$1$ correspondence
\begin{equation*} \left\{ \begin{gathered} \text{Riemann surfaces} \\ \text{compact and connected} \end{gathered} \right\} \xleftrightarrow{\quad \sim \quad}  \left\{ \begin{gathered} \text{Smooth algebraic} \\ \text{projective curves} \end{gathered} \right\}. \end{equation*} \end{theorem}

\begin{proof}Let $X$ be a compact connected Riemann surface, and let $p \in X$ be a point. The divisor
\begin{equation*} D = ( 2 \, g(X) + 1) \cdot p \in \Div(X)\end{equation*}
induces an embedding
\begin{equation*} \varphi_{|D|} : X \xrightarrow{\sim} Y \subseteq \p^N(\C), \end{equation*}
and this is exactly what we wanted to prove.

Vice versa, let $Y \subseteq \p^N(\C)$ be an algebraic curve. A finite number of applications of \hyperref[prop:peoro]{Proposition \ref{prop:peoro}} yields to a projection $\widetilde{\pi}$ such that
\begin{equation*}\widetilde{\pi}(Y) := \widetilde{Y} \subseteq \p^3(\C)\end{equation*}
has the property that, for every $p \in \widetilde{Y}$, there is an affine neighborhood $U_p \ni p$ such that
\begin{equation*} \widetilde{Y} \cap U_p = V(g_1, \, g_2) \cap U_p \qquad \text{and} \qquad \mathrm{rank} \left( \frac{\partial \, g_i}{\partial \, x_j} \right) = 2. \end{equation*}
The conclusion follows immediately if one applies the maximal rank theorem\footnote{\textbf{Maximal Rank Theorem:} If $F : M \to N$ has maximal rank near a point $p \in M$, then there exist a neighborhood $U$ of $p$ and $V$ of $F(p)$, and there are diffeomorphisms $u : T_p \, M \xrightarrow{\sim} U$ and $v : T_{F(p)} \, N \xrightarrow{\sim} V$ such that $F(U) \subseteq V$ and
\begin{equation*} \mathrm{d}F_p = v^{-1} \circ F \circ u. \end{equation*} }.

\paragraph{Alternative Approach.} Let $\pi : \p^N(\C) \to \p^2(\C)$ be the projection, and let $\widetilde{Y}$ be the algebraic curve with a finite number of simple knots (see \hyperref[prop:oewoe]{Proposition \ref{prop:oewoe}}).

The construction of the Riemann surfaces follows from the blowup method introduced in \hyperref[sec:blow]{Subsection \ref{sec:blow}}, but it is quite a lot harder.\end{proof}

\subsection{Equivalence Theorem}

In this subsection, we want to give a different proof of \hyperref[th:eccc]{Theorem \ref{th:eccc}} that allows us, via a third category, to be more precise about the category morphisms.

\begin{theorem}[Chow]\index{Chow Theorem}\label{chow} Let $Y \subseteq \p^N(\C)$ be an analytic manifold. \mbox{}
\begin{enumerate}[label=\textbf{(\arabic*)}]
\item $Y$ is an algebraic variety, that is, $Y = \mathcal{V}(I_Y)$.
\item Any meromorphic function on $Y$ is rational.
\item Any holomorphic map $f :  Y \to Y^\prime$ is given by rational functions.

More precisely, if we let
\begin{equation*} \C[Y] = \faktor{\C[x_0, \, \dots, \, x_N]}{I_Y} \qquad \text{and} \qquad \C(Y) = \mathrm{Frac}\left(\C[Y] \right), \end{equation*}
then the field of all the meromorphic functions $f : Y \to \C$, denoted by $\mathcal{M}(Y)$, is isomorphic to $\C(Y)$.
\end{enumerate}
\end{theorem}

\begin{remark} If $Y$ is an algebraic curve, i.e. $\mathrm{dim}_\C \, Y = 1$, then the field
\begin{equation*} \M(Y) \cong \C(Y) \end{equation*}
has transcendental degree one over $\C$. \end{remark}

\begin{theorem}\label{th:ecccgen}There is a $1$-$1$ correspondence between the following three categories:
\begin{equation*} \begin{tikzcd} \left\{ \begin{gathered} \text{Riemann surfaces} \\ \text{compact and connected} \end{gathered} \right\} \ar[rr, leftrightarrow] \ar[dr, leftrightarrow]& &\ar[dl, leftrightarrow] \left\{ \begin{gathered} \text{Smooth algebraic} \\ \text{projective curves} \end{gathered} \right\} \\ & \left\{ \begin{gathered} \text{Field of the form $\C(X)$} \\ \text{with transcendental degree} \\ \text{one over $\C$} \end{gathered} \right\} & .\end{tikzcd} \end{equation*}\end{theorem}

\begin{proof}We divide the argument into three steps.

\paragraph{Step 1.1.}Let us consider the functor
\begin{equation*} \Phi :  \left\{ \begin{gathered} \text{Riemann surfaces} \\ \text{compact and connected} \end{gathered} \right\} \longrightarrow \left\{ \begin{gathered} \text{Field of the form $\C(X)$} \\ \text{with transcendental degree} \\ \text{one over $\C$} \end{gathered} \right\}\end{equation*}
defined by
\begin{equation*} \Phi(X) := \mathcal{M}(X), \end{equation*}
that is, it sends a compact connected Riemann surfaces to the field of meromorphic functions defined on $X$, and also
\begin{equation*} \mathrm{Hom}(X, \, Y) \ni f \longmapsto f^\ast \in \mathrm{Hom} \left( \M(Y), \, \M(X) \right), \end{equation*}
where $f$ is a surjective (i.e., nonconstant) morphism, and
\begin{equation*} f^\ast (h) := h \circ f. \end{equation*}
This functor is \textit{essentially surjective}, as a consequence of the following result which we will not prove.

\vspace{2.5mm}
\noindent\fbox{
\parbox{14cm}{ \begin{theorem} Let $\mathcal{M}$ be a field with transcendental degree one over $\C$. Then $\M$ is isomorphic to the field of quotient of
\begin{equation*} \faktor{ \C[x, \, y] }{(F)}, \end{equation*}
where $F$ is an irreducible polynomial. \end{theorem}}}

\vspace{2.5mm}
\noindent In particular, every $\M$ in the codomain induces an affine algebraic curve given by
\begin{equation*} \widetilde{X} = \mathcal{V}(F) \subseteq \C^2. \end{equation*}
The Riemann surface $X$ such that $\Phi(X) = \M$ can be easily defined starting from $\widetilde{X}$: take the projectivization in $\p^2(\C)$, and then resolve the singularities (see \hyperref[sec:blow]{Subsection \ref{sec:blow}}).

\paragraph{Step 1.2.} The functor $\Phi$ is fully faithful\footnote{\textbf{Definition.} The map
\begin{equation*}\Phi_{X, \, Y} : \mathrm{Hom}(X, \, Y) \xrightarrow{\quad \sim \quad} \mathrm{Hom} \left( \Phi(X), \, \Phi(Y) \right) \end{equation*}
is an isomorphism for every $X, \, Y$ objects.}. Indeed, let us set $X_1 = \mathcal{V}(F)$ and $X_2 = \mathcal{V}(G)$, and let us consider the morphism
\begin{equation*} \varphi : \M_1 \longrightarrow \M_2. \end{equation*}
We may always consider for the maps
\begin{equation*} \alpha_i : \C(X_i) \xrightarrow{\quad \sim \quad} \M_i, \qquad i = 1, \, 2,\end{equation*}
sending the projections $\pi_x$ and $\pi_y$ respectively to $f_i$ and $g_i$. Let
\begin{equation*} \Psi : \C^2 \to \C^2, \qquad \Psi := \left( \alpha_2^{-1} \circ \varphi(f_2), \, \alpha_2^{-1} \circ \varphi(g_2) \right) =: \left(R_1(x, \, y), \, T_1(x, \, y) \right) \end{equation*}
be a function such that the following diagram is commutative:
\begin{equation*} \begin{tikzcd} \M_1 \ar[r,"\varphi"] & \M_2 \\ \C(X_1)\ar[u, "\alpha_1"] \ar[r, "\Psi^\ast"]& \C(X_2) \ar[u, "\alpha_2"] \end{tikzcd} \end{equation*}
Therefore $\Psi$ induces a map between Riemann surfaces
\begin{equation*} \Psi : X_2 \to X_1 \qquad \text{with $\Psi^\ast = \alpha_2^{-1} \circ \varphi \circ \alpha_1$}, \end{equation*}
and, actually, it turns out that
\begin{equation*} 0 = G(f_2, \, g_2) \implies 0 = \alpha_2^{-1} \circ \varphi \circ G(f_2, \, g_2) \end{equation*}
which, in turn, implies that
\begin{equation*} 0 = G \left( \alpha_2^{-1} \circ \varphi(f_2), \, \alpha_2^{-1} \circ \varphi(g_2) \right) = G(R_1, \, T_1) \in \M_2 \cong \C(X_2). \end{equation*}

\paragraph{Step 2.} In this paragraph, we give the idea behind the correspondence
\begin{equation*} \left\{ \begin{gathered} \text{Smooth algebraic} \\ \text{projective curves} \end{gathered} \right\}  \xleftrightarrow{\quad \sim \quad} \left\{ \begin{gathered} \text{Field of the form $\C(X)$} \\ \text{with transcendental degree} \\ \text{one over $\C$} \end{gathered} \right\}.\end{equation*}
Let $X \subseteq \p^N(\C)$ be a smooth algebraic curve; by \hyperref[prop:oewoe]{Proposition \ref{prop:oewoe}} it turns out that it is birational to an algebraic curve $X^\prime \subseteq \p^2(\C)$. On the other hand
\begin{equation*} Y \subseteq \p^2(\C) \xrightarrow{\sim} \widetilde{Y} \subseteq \C^2 \end{equation*}
is also a birational correspondence, and hence
\begin{equation*} X \xrightarrow{\sim} \widetilde{Y} \subseteq \C^2 \end{equation*}
is a birational map, as it is the composition of birational maps. We conclude by noticing that
\begin{equation*} \C\left(\widetilde{Y}\right) \cong \mathrm{Frac} \left( \faktor{ \C[x, \, y] }{(F)} \right). \end{equation*}

\paragraph{Step 3.} The third equivalence was already proved in \hyperref[th:eccc]{Theorem \ref{th:eccc}}. The morphisms are easily defined using the commutativity of the diagram.
\end{proof}

\section{Existence of Globally Defined Meromorphic Functions}
\label{sec:merglo}

Let $X$ be a compact Riemann surface, let $p \in X$ be a point, and let
\begin{equation*} D := n \cdot p, \qquad n \geq 2 \, g(X) -1 \end{equation*}
be a \textit{simple} divisor. By \hyperref[exas]{Proposition \ref{exas}} there is a short exact sequence
\begin{equation*} 0 \xrightarrow{} \mathcal{O}_X \xrightarrow{} \mathcal{O}_X[D] \xrightarrow{} \mathcal{O}_D \xrightarrow{} 0, \end{equation*}
which induces a long exact sequence in cohomology (see \hyperref[longcomo]{Theorem \ref{longcomo}}):
\begin{equation*} \begin{aligned} 0 \xrightarrow{}  H^0 \left( X, \, \mathcal{O}_X \right) \xrightarrow{} & H^0 \left(X, \,  \mathcal{O}_X[D] \right)  \xrightarrow{} H^0 \left(X, \,  \mathcal{O}_D \right) \xrightarrow{} \dots \\[1em] & \dots \xrightarrow{} H^1 \left(X, \,  \mathcal{O}_X \right) \xrightarrow{} H^1 \left(X, \,  \mathcal{O}_X[D] \right) \xrightarrow{} 0. \end{aligned} \end{equation*}
As a consequence of \hyperref[hdd]{Theorem \ref{hdd}}, it turns out that
\begin{equation*} n \geq 2 \, g(X) - 1 \implies H^1 \left( \mathcal{O}_X[D] \right) = 0, \end{equation*}
and hence
\begin{equation*}0 \xrightarrow{}  H^0 \left( X, \, \mathcal{O}_X \right) \xrightarrow{} H^0 \left(X, \,  \mathcal{O}_X[D] \right) \xrightarrow{} H^0 \left(X, \,  \mathcal{O}_D \right) \xrightarrow{} H^1 \left(X, \,  \mathcal{O}_X \right) \xrightarrow{} 0. \end{equation*}
is an exact sequence. It follows that
\begin{equation*} h^0 \left( \mathcal{O}_X[D] \right) = n - g(X) + 1 \geq g(X), \end{equation*}
thus the second term of the sequence above cannot be trivial, that is,
\begin{equation*} H^0 \left( \mathcal{O}_X[D] \right) \neq 0. \end{equation*}
In particular, there exists a global meromorphic function $f$ such that $\mathrm{div}_\infty (f) \leq n \cdot p$, that is, $f$ has a pole of order at most $n$ at $p$.

\section{Low Degree Divisors}
\label{sec:ldd} \index{Divisor!of low degree}

In this section, we denote by $X$ a compact connected Riemann surface (since we require both of the assumptions to hold true in most of the results we will be presenting).

\paragraph{Recall.} We have proved that the following result holds true for divisors of (relatively) \textit{high degree}:

\begin{theorem}[High-Degree Divisors] Let $X$ be a compact connected Riemann surface of genus $g(X)$, and let $D \in \Div(X)$ be a divisor of degree $d$. \mbox{}
\begin{enumerate}[label=\textbf{(\alph*)}]
\item If $d \geq 2 \,g(X) - 1$, then
\begin{equation*}H^1 \left( X, \, \mathcal{O}_X[D] \right) = \{0\}. \end{equation*}
\item If $d \geq 2 \, g(X)$, then $|D|$ is a b.p.f. linear system.
\item If $f \geq 2 \, g(X) + 1$, then $D$ is very ample.
\end{enumerate}
\end{theorem}

In this section, we shall be mainly concerned with the properties of small order divisors (precisely: divisors of order $0$, $1$ or $2$.) We start the discussion with two simple remarks.

\begin{remark}\index{Divisor!of degree $0$}Let $D \in \Div(X)$ be a divisor of degree $\mathrm{deg} \, D = 0$. Then
\begin{equation*}h^0 \left(X, \, \mathcal{O}_X[D] \right) = 1 \iff D \sim 0, \end{equation*}
or, equivalently,
\begin{equation*} h^0 \left(X, \, \mathcal{O}_X[D] \right) = 0 \iff D \not \sim 0.\end{equation*}\end{remark}

\begin{proof}We have proved in \hyperref[lemma:consdodfkf]{Lemma \ref{lemma:consdodfkf}} that - assuming $X$ is a compact Riemann surface - there is a $1$-$1$ correspondence
\begin{equation*} \p \left( H^0 \left(X, \, \mathcal{O}_X[D] \right) \right) \cong |D|. \end{equation*}
In particular, it turns out that
\begin{equation*}h^0 \left(X, \, \mathcal{O}_X[D] \right) = 1 \iff \mathrm{dim} \, |D| = 0 \iff |D| = \{0\}, \end{equation*}
that is, if and only if $D \sim 0$. In a similar fashion, one could prove the equivalent formulation, i.e.,
\begin{equation*}h^0 \left(X, \, \mathcal{O}_X[D] \right) = 0 \iff \mathrm{dim} \, |D| = -1 \iff |D| = \emptyset. \end{equation*} \end{proof}

\begin{remark} Let $D \in \Div(X)$ be a divisor of degree $2 \, g(X) - 2$. The \hyperref[serreduality1]{Serre Duality Theorem \ref{serreduality1}}, together with the remark above, proves that
\begin{equation*} h^1 \left(X, \, \mathcal{O}_X[D] \right) = 1 \iff D \sim K_X,\end{equation*}
or, equivalently,
\begin{equation*} h^1 \left(X, \, \mathcal{O}_X[D] \right) = 0 \iff D \not \sim K_X. \end{equation*}\end{remark}

\paragraph{Low Degree Divisors.} Recall that we have proved in \hyperref[prop:stimah0]{Proposition \ref{prop:stimah0}} that there is a rough estimate on the dimension of the $0$th cohomology group if $D$ is a divisor of positive degree:
\begin{equation} \label{roughestimate} h^0 \left(X, \, \mathcal{O}_X[D] \right) \leq \mathrm{deg} \, D + 1. \end{equation}

\begin{proposition}\label{prop:ccdsdsd}Let $X$ be a compact connected Riemann surface, and let $D \in \Div(X)$ be a divisor of degree equal to $1$. Then
\begin{equation*} h^0 \left(X, \, \mathcal{O}_X[D] \right) \geq 2 \iff \begin{gathered}\text{$h^0 \left(X, \, \mathcal{O}_X[D] \right) = 2$, $X = \p^1(\C)$,} \\[1em] \text{and $\varphi_{|D|} : X \xrightarrow{\sim} \p^1(\C)$ is the isomorphism}. \end{gathered} \end{equation*}\end{proposition}

\begin{proof}The estimate \eqref{roughestimate} proves that
\begin{equation*}\mathrm{deg} \, D = 1 \implies \left( h^0 \left(X, \, \mathcal{O}_X[D] \right) \geq 2 \iff h^0 \left(X, \, \mathcal{O}_X[D] \right) = 2 \right). \end{equation*}
The idea is to show that
\begin{equation*}\varphi_{|D|} : X \to \p \left( H^0 \left(X, \, \mathcal{O}_X[D] \right)^v \right) \cong \p^1(\C) \end{equation*}
is a morphism (i.e., the linear divisor system $|D|$ is b.p.f.) of degree $1$. For any $p \in X$ there is a short exact sequence
\begin{equation*} 0 \xrightarrow{} H^0 \left( X, \, \mathcal{O}_X[D - p] \right) \xrightarrow{} H^0 \left( X, \, \mathcal{O}_X[D] \right) \xrightarrow{} H^0\left(X, \, \C_p \right) \end{equation*}
from which it follows that
\begin{equation*} \begin{cases} \mathrm{deg}(D - p) = 0 \\[0.5em]  h^0 \left( X, \, \mathcal{O}_X[D] \right) = 2 \\[0.5em] h^0\left(X, \,\C_p \right) = 1\end{cases} \implies \begin{cases} h^0 \left( X, \, \mathcal{O}_X[D - p] \right) = 1 \\[1em] H^0 \left( X, \, \mathcal{O}_X[D] \right) \longtwoheadrightarrow \C_p.\end{cases} \end{equation*}
By \hyperref[doasodls]{Theorem \ref{doasodls}} it turns out that $|D|$ is b.p.f., and the morphism $\varphi_{|D|}$ has degree equal to $\mathrm{deg} \, D = 1$\footnote{The reader should pay attention that this is not always true; see \cite[pp 164-165]{miranda}. In this case it is true since $\p^1(\C)$ is a Riemann surface.}, which is exactly what we wanted to prove.

\paragraph{Alternative Conclusion.} For every $p \in X$
\begin{equation*} h^0 \left( X, \, \mathcal{O}_X[D - p] \right) = 1 \implies D \sim p, \end{equation*}
and this, in turn, implies that
\begin{equation*} \mathcal{O}_X[D] \cong \mathcal{O}_X[p] \qquad \forall \, p \in X. \end{equation*}
More precisely, the points are all equivalent; since this is a property that characterizes $\p^1(\C)$, we infer that $X$ is the complex projective line\footnote{It is easy to prove that $p, \, q \in \p^1(\C)$ are always equivalent, e.g. consider the function
\begin{equation*}f(z) = \frac{ p_0 \, z_1 - p_1 \, z_0 }{ q_0 \, z_1 - q_1 \, z_0 }.\end{equation*}}.
\end{proof}

\begin{definition}[Hyperelliptic]\index{Hyperelliptic curve} A compact connected Riemann surface $X$ of genus  $g(X) \geq 2$ is \textit{hyperelliptic} if there exists a divisor $D \in \Div(X)$ such that
\begin{equation*} \mathrm{deg} \, D = 2 \qquad \text{and} \qquad h^0 \left(X, \, \mathcal{O}_X[D] \right) = 2. \end{equation*} \end{definition}

\begin{remark}Equivalently, a Riemann surface $X$, satisfying the same assumptions as above, is \textit{hyperelliptic} if there exists a divisor $D \in \Div(X)$ such that: \mbox{}
\begin{enumerate}[label=\textbf{(\arabic*)}]
\item The linear divisor system $|D|$ is b.p.f. (as a consequence of \hyperref[prop:ccdsdsd]{Proposition \ref{prop:ccdsdsd}}).
\item The morphism $\varphi_{|D|} : X \xrightarrow{\cdot 2} \p^1(\C)$ has degree $2$. In the remainder of the section, we shall denote it by $g_2^1$.
\end{enumerate}\end{remark}

\begin{figure}[h]
\centering
\includegraphics[width = 10cm, height = 8cm]{images/GEOMALGC51.png}
\caption{The point $w$ is called \textit{Weierstrass ramification point}\index{Weierstrass ramification point}. By Riemann-Hurwitz there are only $2 \, g(X) + 2$.}
\label{fig:wrp}
\end{figure}

\newpage

\begin{proposition}Let $X$ be a hyperelliptic Riemann surface. The morphism \index{$g_2^1$}$g_2^1 : X \to \p^1(\C)$ is unique. \end{proposition}

\begin{proof}We argue by contradiction.

\paragraph{Step 1.} Let $D_1, \, D_2 \in \Div(X)$, and let $\varphi_{|D_i|} : X \xrightarrow{\cdot 2} \p^1(\C)$ be the associated morphisms. By assumption
\begin{equation*} \mathrm{deg}(D_i) = 2 \qquad \text{and} \qquad h^0 \left(X, \, \mathcal{O}_X[D_i] \right) = 2, \end{equation*}
hence there are divisors in $|D_i|$ (which will still be denoted by $D_i$) such that
\begin{equation*} D_1 = p + q \qquad \text{and} \qquad D_2 = p + r. \end{equation*}

\paragraph{Step 2.} Let $L := p + q + r$ be the minimal divisor containing both; we claim that
\begin{equation*}h^0 \left(X, \, \mathcal{O}_X[L] \right) = 3. \end{equation*}
Suppose that $h^0 \left(X, \, \mathcal{O}_X[L] \right) < 3$. Then it is necessarily equal to $2$, and we can easily derive a contradiction looking at the exact sequences below:
\begin{equation*} \begin{aligned} & 0 \xrightarrow{} H^0 \left(X, \, \mathcal{O}_X[p+q] \right)\xrightarrow{} H^0 \left(X, \, \mathcal{O}_X[L] \right)\xrightarrow{} H^0 \left( X, \, \C_r \right), \\[1em] & 0 \xrightarrow{} H^0 \left(X, \, \mathcal{O}_X[p+r] \right)\xrightarrow{} H^0 \left(X, \, \mathcal{O}_X[L] \right)\xrightarrow{} H^0 \left( X, \, \C_q \right).\end{aligned} \end{equation*}
The middle terms are the same, hence there are exact sequences
\begin{equation*} \begin{aligned} & 0 \xrightarrow{} H^0 \left(X, \, \mathcal{O}_X[p+q] \right)\xrightarrow{} H^0 \left(X, \, \mathcal{O}_X[L] \right)\xrightarrow{} H^0 \left( X, \, \C_q \right), \\[1em] & 0 \xrightarrow{} H^0 \left(X, \, \mathcal{O}_X[p+r] \right)\xrightarrow{} H^0 \left(X, \, \mathcal{O}_X[L] \right)\xrightarrow{} H^0 \left( X, \, \C_r \right),\end{aligned} \end{equation*}
and by the assumption on the dimension it turns out that
\begin{equation*}H^0 \left(X, \, \mathcal{O}_X[p+q] \right) = H^0 \left(X, \, \mathcal{O}_X[L] \right) = H^0 \left(X, \, \mathcal{O}_X[p+r] \right) \end{equation*}
which is absurd.

\paragraph{Step 3.} Let $s, \, t \in X$ be points, and let us set $\Delta := s + t$. There is an exact sequence
\begin{equation*} 0 \xrightarrow{} H^0 \left(X, \, \mathcal{O}_X[L - s - t] \right)\xrightarrow{} H^0 \left(X, \, \mathcal{O}_X[L] \right)\xrightarrow{} H^0 \left( X, \, \C_\Delta \right), \end{equation*}
and we immediately observe that $\mathrm{dim} \, \C_\Delta = 2$, and the divisor $L - s - t$ has degree equal to $1$. Thus, by \hyperref[prop:ccdsdsd]{Proposition \ref{prop:ccdsdsd}}, it turns out that 
\begin{equation*} \mathrm{dim} \, H^0 \left(X, \, \mathcal{O}_X[L - s - t] \right) \leq 1, \end{equation*}
which, in turn, implies that
\begin{equation*} H^0 \left(X, \, \mathcal{O}_X[L] \right) \twoheadrightarrow H^0 \left( X, \, \C_\Delta \right). \end{equation*}

\paragraph{Step 4.} The numerical criterion (see \hyperref[doasodls]{Theorem \ref{doasodls}}) implies that the divisor $L$ is very ample. We have already proved that
\begin{equation*} \mathrm{deg}(L) = h^0  \left( X, \, \mathcal{O}_X[L] \right) = 3, \end{equation*}
hence the morphism $\varphi_{|L|} : X \hookrightarrow \p^2(\C)$ is an embedding, whose image $\varphi_{|L|}(X)$ is an algebraic curve of degree equal to $3$, in contradiction with the fact that $g(X) \geq 2$ since
\begin{equation*}g\left( \text{algebraic curve of degree $3$ in $\p^2$} \right) = \frac{2 \cdot (2 - 1)}{2} = 1.\end{equation*}
\end{proof}

\begin{remark}Let $X$ be a compact connected Riemann surface of genus $g(X) = 2$. A canonical divisor $K_X$ has the additional properties
\begin{equation*} \mathrm{deg}(K_X) = h^0  \left( X, \, \mathcal{O}_X[K_X] \right) = 2, \end{equation*}
and hence $X$ is always hyperelliptic. \end{remark}

\section{Canonical Map}
\index{Canonical map}

In this section, the primary goal is to study the \textit{canonical map}, that is, the map associated to the canonical divisor $K_X$.

More precisely, we will prove that, if $X$ is a compact connected Riemann surface of genus $g(X) \geq 2$ which is not hyperelliptic, then $\varphi_{|K_X|}$ is an embedding.

\begin{theorem} \label{thbpf} Let $X$ be a compact connected Riemann surface of genus $g(X) \geq 2$. Then the canonical divisor $K_X$ is b.p.f., i.e., the map
\begin{equation*} \varphi_{|K_X|} : X \to \p^{g(X) - 1}(\C) \end{equation*}
is a morphism. \end{theorem}

\begin{proof}Fix $p \in X$. There is a short exact sequence (see \hyperref[exas]{Proposition \ref{exas}}) given by
\begin{equation*} 0 \xrightarrow{} \mathcal{O}_X[K_X - p] \xrightarrow{} \mathcal{O}_X[K_X] \xrightarrow{} \C_p \xrightarrow{} 0\end{equation*}
which induces a long sequence in cohomology (see \hyperref[longcomo]{Theorem \ref{longcomo}}):
\begin{equation*} \begin{aligned} 0 \xrightarrow{} H^0 \left(X, \, \mathcal{O}_X[K_X - p] \right) & \xrightarrow{} H^0 \left(X, \, \mathcal{O}_X[K_X] \right) \xrightarrow{} H^0\left(X, \, \C_p \right) \xrightarrow{} \dots \\[1em] & \dots \xrightarrow{} H^1 \left(X, \, \mathcal{O}_X[K_X - p] \right)  \xrightarrow{} H^1 \left(X, \, \mathcal{O}_X[K_X] \right) \xrightarrow{} 0. \end{aligned}\end{equation*}
It follows from the \hyperref[serreduality1]{Serre Duality Theorem \ref{serreduality1}} that
\begin{equation*} \begin{aligned} & H^1 \left(X, \, \mathcal{O}_X[K_X] \right) \cong H^0 \left(X, \, \mathcal{O}_X \right)^v \cong \C, \\[1em] & H^1 \left(X, \, \mathcal{O}_X[K_X-p] \right) \cong H^0 \left(X, \, \mathcal{O}_X[p] \right)^v \cong \C,  \end{aligned}\end{equation*}
where the latter isomorphism is a consequence of the fact that $\mathrm{deg} \, p = 1$ is an effective divisor, but $X$ is not isomorphic to $\p^1(\C)$ since the genus is strictly greater than zero (see \hyperref[prop:ccdsdsd]{Proposition \ref{prop:ccdsdsd}}). As a consequence, there is an isomorphism
\begin{equation*} H^1 \left(X, \, \mathcal{O}_X[K_X-p] \right) \xrightarrow{ \sim} \C \xrightarrow{\sim} H^1 \left(X, \, \mathcal{O}_X[K_X] \right), \end{equation*}
which, in turn, implies that
\begin{equation*}H^0 \left(X, \, \mathcal{O}_X[K_X-p] \right) \longtwoheadrightarrow H^0 \left( X, \, \C_p \right) \end{equation*}
is a surjective map.

By arbitrariness of $p \in X$, we conclude that the thesis holds true as a consequence of the numerical criterion (see \hyperref[doasodls]{Theorem \ref{doasodls}}).

\paragraph{Equivalent Approach.} By \hyperref[RiemannRoch]{Riemann-Roch \ref{RiemannRoch}} it turns out that
\begin{equation*}h^0 \left(X, \, \mathcal{O}_X[K_X-p] \right) - \underbrace{h^1 \left(X, \, \mathcal{O}_X[K_X-p] \right)}_{=1} = \underbrace{\mathrm{deg}(K_X - p)}_{=2 \, g(X) - 3} + 1 - g(X),\end{equation*}
which, in turn, implies that
\begin{equation*}h^0 \left(X, \, \mathcal{O}_X[K_X-p] \right) = g(X) - 1 = h^0 \left(X, \, \mathcal{O}_X[K_X] \right) - 1.\end{equation*}
Since this equality holds for every $p \in X$, the numerical criterion allows us again to infer that the thesis holds true. \end{proof}

\begin{theorem}\index{Canonical map!for nonhyperelliptic curve} Let $X$ be a compact connected Riemann surface of genus $g(X) \geq 2$. The canonical divisor $K_X$ is very ample if and only if $X$ is not hyperelliptic.\end{theorem}

\begin{proof}First, we observe that \hyperref[thbpf]{Theorem \ref{thbpf}} asserts that the linear divisor system $|K_X|$ is b.p.f. under these assumptions. Let $p, \, q \in X$, set $\Delta := p + q$, and consider the exact sequence in cohomology given by
\begin{equation*} \begin{aligned} 0 \xrightarrow{} H^0 \left(X, \, \mathcal{O}_X[K_X - p - q] \right) & \xrightarrow{} H^0 \left(X, \, \mathcal{O}_X[K_X] \right) \xrightarrow{} H^0\left(X, \, \C_\Delta \right) \xrightarrow{} \dots \\[1em] & \dots \xrightarrow{} H^1 \left(X, \, \mathcal{O}_X[K_X - p - q] \right)  \xrightarrow{} H^1 \left(X, \, \mathcal{O}_X[K_X] \right) \xrightarrow{} 0. \end{aligned}\end{equation*}
By \hyperref[serreduality1]{Serre Duality Theorem \ref{serreduality1}} it turns out that
\begin{equation*} \begin{aligned}& H^1 \left(X, \, \mathcal{O}_X[K_X] \right) \cong H^0 \left(X, \, \mathcal{O}_X \right)^v \cong \C, \\[1em] & H^1 \left(X, \, \mathcal{O}_X[K_X - p - q] \right) \cong H^0 \left(X, \, \mathcal{O}_X[p + q] \right)^v.\end{aligned} \end{equation*}
In conclusion, it follows from \hyperref[doasodls]{Theorem \ref{doasodls}} that
\begin{equation*}\text{$D$ is very ample} \iff H^0 \left(X, \, \mathcal{O}_X[K_X] \right) \longtwoheadrightarrow H^0 \left(X, \, \C_\Delta \right),\end{equation*}
and hence it is enough to observe that
\begin{equation*} H^0 \left(X, \, \mathcal{O}_X[K_X] \right) \longtwoheadrightarrow H^0 \left(X, \, \C_\Delta \right) \iff  h^0 \left(X, \, \mathcal{O}_X[p + q] \right) = 1 \end{equation*}
for every $p, \, q \in X$; or, equivalently,
\begin{equation*} H^0\left(X, \, \mathcal{O}_X[K_X] \right) \not \longtwoheadrightarrow H^0 \left(X, \, \C_\Delta \right) \iff \exists \, p, \, q \in X \: : \:  h^0 \left(X, \, \mathcal{O}_X[p + q] \right) = 2\end{equation*}
which means $X$ is hyperelliptic by definition.\end{proof}

\begin{theorem}\label{yperikfks}\index{Canonical map!for hyperelliptic curve} Let $X$ be a hyperelliptic Riemann surface. The morphism $\varphi_{|K_X|} : X \to \p^{g(X) - 1}$ can be factorized as follows:
\begin{equation*} \varphi_{|K_X|} : X \xrightarrow{ g_2^1 } \p^1(\C) \xrightarrow{ \nu_{g-1} } \p^{g(X) - 1}(\C), \end{equation*}
where $\nu_{g-1}$ is the Veronese embedding of degree $g(X) - 1$.\end{theorem}

\begin{proof}The morphism $\varphi_{|K_X|}$ sends $X$ into an algebraic curve $Y \subseteq \p^{g(X) - 1}(\C)$, but it is not $1$-$1$ as a consequence of the previous characterization. Hence $\mathrm{deg} \, \varphi_{|K_X|} \geq 2$ and, if we set $d := \mathrm{deg} \, Y$, then
\begin{equation*}\mathrm{deg} \, K_X = 2 \, g(X) - 2 = d \cdot \mathrm{deg} \, \varphi_{|K_X|} \implies d \leq g(X) - 1. \end{equation*}
The reader should convince herself that it suffices to prove that
\begin{equation*} Y = \nu_{g-1} \left( \p^1(\C) \right), \end{equation*}
that is, $Y$ is a normal rational curve\footnote{A smooth, rational curve of degree $n$ in the projective space $\p^n(\C)$.}, to conclude the proof.

\paragraph{Step 1.} Let $\pi : \widetilde{Y} \to Y$ be the resolution of the singularities of $Y$ (see \hyperref[sec:blow]{Subsection \ref{sec:blow}}), and let $H \subseteq Y$ be a divisor hyperplane.

The linear system $\left|\pi^\ast(H) \right|$ is the $(g(X) - 1)$-dimensional space associated to the pullback divisor $\pi^\ast(H)$, and hence
\begin{equation*} \varphi_{|\pi^\ast(H)|} : \widetilde{Y} \longrightarrow \p^{g(X) - 1}(\C) \end{equation*}
is a morphism such that
\begin{equation*} \mathrm{deg} \,\left(  \varphi_{|\pi^\ast(H)|} \left(\widetilde{Y} \right) \subseteq \p^{g(X) - 1} \right) = g(X) - 1. \end{equation*}

\paragraph{Step 2.} The \hyperref[RiemannRoch]{Riemann-Roch Theorem \ref{RiemannRoch}} proves that $\widetilde{Y} \cong \p^1(\C)$, and also
\begin{equation*} H^0 \left( \widetilde{Y}, \, \mathcal{O}_{\widetilde{Y}} [ \pi^\ast \, H ] \right) = H^0 \left( \p^1(\C), \, \mathcal{O}_{\p^1(\C)} \right). \end{equation*} 
It follows that
\begin{equation*} \varphi_{|\pi^\ast(H)|} : \p^1(\C) \longrightarrow \p^{g(X) - 1}(\C), \qquad x \longmapsto x \longmapsto \left(x_0^\alpha \, x_1^\beta \right)_{\alpha + \beta = g(X) - 1}, \end{equation*}
that is, $\varphi_{|\pi^\ast(H)|}$ is the Veronese embedding; consequently, we infer that
\begin{equation*} d = g(X) - 1 \qquad \text{and} \qquad \mathrm{deg} \, \varphi_{|K_X|} = 2. \end{equation*}
Moreover, for any $p, \, q \in X$ it turns out that
\begin{equation*} \varphi_{|K_X|} (p) = \varphi_{|K_X|}(q) \iff h^0 \left(X, \, \mathcal{O}_X[p + q] \right) = 2 \iff |p + q| = g_2^1, \end{equation*}
that is,
\begin{equation*} \varphi_{|K_X|} (p) = \varphi_{|K_X|}(q) \iff g_2^1(p) = g_2^1(q), \end{equation*}
and this proves that
\begin{equation*}\varphi_{|K_X|} : X \xrightarrow{ g_2^1 } \p^1(\C) \xrightarrow{ \nu_{g-1} } \p^{g(X) - 1}(\C). \end{equation*} \end{proof}

\section{Riemann-Roch: Geometric Form}
\index{Riemann-Roch Theorem!Geometric Form}

In this section, we denote by $X$ a nonhyperelliptic Riemann surface, by $\varphi$ the canonical map $\varphi_{|K_X|}$, and we identify $X$ with its image via $\varphi$ (which is an embedding).

\paragraph{Geometric form.} Let $D = p_1 + \dots + p_d \in \Div(X)$ be a divisor of degree $d$. We may always think of $\{p_1, \, \dots, \, p_d\}$ as a set of points in $\p^{g(X) - 1}(\C)$; in particular, it makes sense to define
\begin{equation*} \mathrm{Span}(D) := \mathrm{Span} \left(p_1, \, \dots, \, p_d \right) \subseteq\p^{g(X) - 1}(\C). \end{equation*}

\begin{theorem}[Riemann Roch, Geometric Form] \label{rrth:gf} The projective dimension of $D$ is $\mathrm{deg} \, D - 1$ minus the dimension of its span, that is,
\begin{equation} \label{rrth:gf:eq} \mathrm{dim} \, |D| = \mathrm{deg} \, D - 1 - \mathrm{dim}\,\mathrm{Span}(D) . \end{equation} \end{theorem}

\begin{proof} The short exact sequence of sheaf maps
\begin{equation*} 0 \xrightarrow{} \mathcal{O}_X[K_X - D] \xrightarrow{} \mathcal{O}_X[K_X] \xrightarrow{} \C_D \xrightarrow{} 0\end{equation*}
induces a long exact sequence in cohomology, that is,
\begin{equation*} 0 \xrightarrow{}  H^0 \left(X, \, \mathcal{O}_X[K_X - D] \right) \xrightarrow{} H^0 \left(X, \,\mathcal{O}_X[K_X] \right) \xrightarrow{} H^0 \left(X, \,\C_D \right).\end{equation*}
Clearly
\begin{equation*} \begin{aligned} H^0 \left(X, \, \mathcal{O}_X[K_X - D] \right) & = \left\{ \begin{gathered} \text{hyperplanes $H$ of $\p^{g(X) - 1}(\C)$ such that} \\[0.8em] \text{$H$ vanishes on $D$} \end{gathered} \right\} \\[1em] & =  \left\{ \begin{gathered} \text{hyperplanes $H$ of $\p^{g(X) - 1}(\C)$ such that} \\[0.8em] \text{$H$ vanishes on $\mathrm{Span}(D)$} \end{gathered} \right\},\end{aligned} \end{equation*}
that is,
\begin{equation} \label{rrth:gf:eq2} \mathrm{dim} \, \mathrm{Span}(D) + \mathrm{dim} \, |K_X - D| = g(X) - 2. \end{equation}
Since $\p^{g(X) - 1}$ is canonically isomorphic to $\p \left( H^0 \left(X, \, \mathcal{O}_X[K_X] \right)^v \right)$ we infer that
\begin{equation*} h^1\left(X, \, \mathcal{O}_X[D] \right) = h^0\left(X, \, \mathcal{O}_X[K_X - D] \right) =  \mathrm{dim} \, |K_X - D| + 1, \end{equation*}
and, if we substitute it into the identity \eqref{rrth:gf:eq2}, we obtain
\begin{equation*}h^1\left(X, \, \mathcal{O}_X[D] \right) = g(X) - 1 - \mathrm{dim} \, \mathrm{Span}(D). \end{equation*}
By \hyperref[RiemannRoch]{Riemann-Roch \ref{RiemannRoch}} it follows that
\begin{equation*}\begin{aligned} h^0\left(X, \, \mathcal{O}_X[D] \right) & = h^1\left(X, \, \mathcal{O}_X[D] \right) + \mathrm{deg} \, D + 1 - g(X) = \\[1em] & = \mathrm{deg} \, D - \mathrm{dim} \, \mathrm{Span}(D), \end{aligned} \end{equation*}
and thus \eqref{rrth:gf:eq} is proved.
\end{proof}
%\section{Normal Rational Curves}

%Let $X$ be a Riemann surface of genus equal to $0$, and let $P \in X$ and $D := n \cdot P$. If $X \cong \p^1$ and $P = [1 \: : \: 0]$, then $h^0(D) = n + 1$ and $\phiD : X \to \p^n$ sends $(x_0 \: : \: x_1)$ to $(x_0^n \: : \dots : \: x_1^n)$, i.e., it is the Veronese embedding.

%Therefore $\phiD$ is an actual embedding and the image $\phiD(X)$ is an algebraic curve of degree $n$, with
%\begin{equation*} \mathrm{rank} \, \begin{pmatrix} y_0 & \dots & y_{n-1} \\ y_1 & \dots & y_n \end{pmatrix} = 1. \end{equation*}

\section{Clifford Theorem}
\index{Clifford Theorem}

\paragraph{Recall.} Let $D$ be a divisor on a compact connected Riemann surface $X$ of genus $g(X)$. The linear system of divisors $|D|$ is isomorphic (see \hyperref[lemma:consdodfkf]{Lemma \ref{lemma:consdodfkf}}) to
\begin{equation*} \p \left( H^0 \left(X, \, \mathcal{O}_X[D] \right) \right), \end{equation*}
and, in particular, it turns out that
\begin{equation*} \mathrm{dim} \, |D| = h^0 \left(X, \, \mathcal{O}_X[D] \right) - 1. \end{equation*}
If $\mathrm{deg} \, D \geq 2 \, g(X) - 1$, then we already know that
\begin{equation*} h^1\left(X, \, \mathcal{O}_X[D] \right) = 0 \qquad \text{and} \qquad \mathrm{dim} \, |D| = \mathrm{deg} \, D - g(X). \end{equation*}
In this section, we state and prove the so-called \textit{Clifford theorem}, which concerns divisors of smaller degree; precisely, it gives a bound on the dimension of $|D|$ when $\mathrm{deg} \, D \leq 2 \, g(X) - 2$.

\begin{lemma}\label{lemma:charks} Let $X$ be a compact connected Riemann surface, and let $D \in \Div(X)$ be a divisor. Then
\begin{equation*} \mathrm{dim} \, |D| \geq k \iff \forall \, \{p_1, \, \dots, \, p_k\} \subset X, \, \, \, \exists \, D^\prime \in |D| \: : \: \mathrm{spt} \left( D^\prime \right) = \{p_1, \, \dots, \, p_k\}. \end{equation*}\end{lemma}

\begin{proof}We argue by induction. The base step ($k = 0$) is trivially true, hence we only focus on the inductive step $k \implies k +1$.

\paragraph{Inductive step: "$\impliedby$".} Assume that for any collection of $k + 1$ points of $X$ there exists a divisor $D^\prime \in |D|$ such that
\begin{equation*}\mathrm{spt} \left(D^\prime \right) \supseteq \{p_1, \, \dots, \, p_{k+1} \}. \end{equation*}
By inductive assumption, this is enough to infer that - at least - the projective dimension of $|D|$ is $\geq k$. Let us pick $p_{k+1} \in \mathrm{Basis}\left( \mathrm{Span} \, |D| \right)$, and let us consider the divisor $D_1 := D - p_{k + 1}$. Clearly
\begin{equation*}\forall \, \{p_1, \, \dots, \, p_k\} \subset X, \, \, \, \exists \, D_1^\prime \in |D_1| \: : \: \mathrm{spt} \left( D_1^\prime \right) = \{p_1, \, \dots, \, p_k\}, \end{equation*}
thus by inductive assumption it turns out that $\mathrm{dim} \, |D_1| \geq k$. On the other hand, we chose $p_{k+1}$ in such a way that $\left| D - p_{k+1} \right| \subsetneqq |D|$; hence
\begin{equation*}\mathrm{dim} \, |D| > \mathrm{dim} \, |D_1| \geq k \implies \mathrm{dim} \, |D| \geq k +1,\end{equation*}
which is exactly what we wanted to prove.

\paragraph{Inductive step: "$\implies$".} Assume that $\mathrm{dim} \, |D| \geq k + 1$. As remarked in the introduction of the section, it implies that $h^0 \left(X, \, \mathcal{O}_X[D] \right) \geq k + 2$. Let $p_1, \, \dots, \, p_{k+1} \in X$ be a given collection of points, and let us set
\begin{equation*} \Delta := p_1 + \dots + p_{k + 1}. \end{equation*}
By \hyperref[exas]{Proposition \ref{exas}} there is a short exact sequence
\begin{equation*} 0 \xrightarrow{} \mathcal{O}_X[D - \Delta] \xrightarrow{} \mathcal{O}_X[D] \xrightarrow{} \C_\Delta \xrightarrow{} 0,\end{equation*}
which induces a long exact sequence in cohomology
\begin{equation*} 0 \xrightarrow{} H^0 \left( X, \, \mathcal{O}_X[D - \Delta] \right) \xrightarrow{} H^0 \left(X, \, \mathcal{O}_X[D]\right) \xrightarrow{} H^0 \left(X, \, \C_\Delta \right) \xrightarrow{} \dots\end{equation*}
and we immediately notice that, by assumption, the middle term has dimension $\geq k + 2$, while the last has dimension equal to $k + 1$. In particular, there exists a section $s \in H^0 \left(X, \, \mathcal{O}_X[D] \right)$ which vanishes on $\Delta$, i.e.,
\begin{equation*} s(p_i) = 0, \qquad \forall \, i = 1, \, \dots, \, k+1. \end{equation*}
The proof is now concluded, but it is worth underlining that the right arrow does not need any inductive assumption since we have never used it in our argument.\end{proof}

\begin{corollary} \label{corollary:ewe}Let $D_1, \, D_2 \in \Div(X)$. Then
\begin{equation*} \mathrm{dim} \, |D_1| + \mathrm{dim} \, |D_2| \leq \mathrm{dim} \, \left|D_1 + D_2 \right|. \end{equation*} \end{corollary}

\begin{proof}Let us set $d_i := \mathrm{dim} \, |D_i|$ for $i = 1, \, 2$. Given a collection of $d_1 + d_2$ points
\begin{equation*} \{ p_1, \, \dots, \, p_{d_1}, \, q_1, \, \dots, \, q_{d_2} \} \subset X, \end{equation*}
it follows from \hyperref[lemma:charks]{Lemma \ref{lemma:charks}} that there are $D_1^\prime \in |D_1|$ and $D_2^\prime \in |D_2|$ divisors such that
\begin{equation*}\mathrm{spt} \left(D_1^\prime \right) \supseteq \{p_1, \, \dots, \, p_{d_1} \} \qquad \text{and} \qquad \mathrm{spt} \left(D_2^\prime \right) \supseteq \{q_1, \, \dots, \, q_{d_2} \}. \end{equation*}
On the other hand, the divisor $D_1^\prime + D_2^\prime$ belongs to $|D_1 + D_2|$, and thus by \hyperref[lemma:charks]{Lemma \ref{lemma:charks}} we infer that
\begin{equation*} \mathrm{dim} \, |D_1 + D_2| \geq d_1 + d_2 = \mathrm{dim} \, |D_1| + \mathrm{dim} \, |D_2|. \end{equation*}\end{proof}

\begin{remark}Equivalently, the corollary asserts that the image of the map
\begin{equation*}\mu : H^0 \left(X, \, \mathcal{O}_X[D_1] \right) \times H^0 \left(X, \, \mathcal{O}_X[D_2] \right) \to H^0 \left(X, \, \mathcal{O}_X[D_1 + D_2] \right) \end{equation*}
has dimension
\begin{equation*} \mathrm{dim} \, \mathrm{Im}(\mu) \geq h^0 \left(X, \, \mathcal{O}_X[D_1] \right) + h^0 \left(X, \, \mathcal{O}_X[D_2] \right) - 1. \end{equation*} \end{remark}

\begin{theorem}[Clifford Theorem] Let $X$ be a compact connected Riemann surface of genus $g(X)$, and let $D \in \Div(X)$ be an effective divisor of degree $\mathrm{deg} \, D \leq 2 \, g(X) - 2$. Then
\begin{equation*} \mathrm{dim} \, |D| \leq \frac{1}{2} \, \mathrm{deg} \, D, \end{equation*}
and the equality holds if and only if either \mbox{}
\begin{enumerate}[label=\textbf{(\arabic*)}]
\item $D = 0$, $D = K_X$; or
\item $X$ is hyperelliptic and $|D| = r \cdot |E|$, where $|E| = g_2^1$ and $r$ is the number of the couples formed by hyperelliptic divisors.
\end{enumerate}\end{theorem}

\begin{proof} The argument is rather involved; hence we divide the proof into four main steps.

\paragraph{Step 1: Inequality.} Assume that $h^1 \left(X, \, \mathcal{O}_X[D] \right) = 0$. By \hyperref[RiemannRoch]{Riemann-Roch \ref{RiemannRoch}} it turns out that
\begin{equation*} h^0 \left(X, \, \mathcal{O}_X[D] \right) = \mathrm{deg} \, D - g(X) + 1 \: {\color{red}\leq} \: \frac{1}{2} \, \mathrm{deg} \, D + 1, \end{equation*} 
where the {\color{red}red} inequality follows from the assumption on the degree of $D$.

If $h^1 \left(X, \, \mathcal{O}_X[D] \right) \neq 0$, then we can also reduce to the Riemann-Roch formula using a simple trick. Indeed, by duality it turns out that
\begin{equation*} h^1 \left(X, \, \mathcal{O}_X[D] \right) \neq 0 \iff h^0 \left(X, \, \mathcal{O}_X[K_X - D] \right) \neq 0, \end{equation*}
and hence it suffices to consider the linear systems $|D|$ and $|K_X - D|$. The \hyperref[RiemannRoch]{Riemann-Roch \ref{RiemannRoch}}, together with \hyperref[serreduality1]{Serre Duality Theorem \ref{serreduality1}}, implies that
\begin{equation} \label{cliff1} \mathrm{dim} \, |D| - \mathrm{dim} \, |K_X - D| = \mathrm{deg} \, D - g(X) + 1, \end{equation}
while the previous \hyperref[corollary:ewe]{Corollary \ref{corollary:ewe}} implies that
\begin{equation} \label{cliff2} \mathrm{dim} \, |D| + \mathrm{dim} \, |K_X - D| \leq \mathrm{dim} \, |K_X| = g(X) - 1. \end{equation}
We conclude the first part of the proof combining \eqref{cliff1} and \eqref{cliff2} to obtain the sought inequality:
\begin{equation*} \mathrm{dim} \, |D| \leq \frac{1}{2} \, \mathrm{deg} \, D, \end{equation*}

\paragraph{Step 2: Equality "$\impliedby$".} This implication is trivial since for $D = 0$ or $D = K_X$ the equality holds as a straightforward application of the \hyperref[serreduality1]{Serre Duality Theorem \ref{serreduality1}}.

On the other hand, if $X$ is a hyperelliptic surface and $|D| = r \cdot g_2^1$, then by definition $\mathrm{deg} \, D = 2 \cdot r$, and hence it is enough to notice that $r = \mathrm{dim} \, |D|$. 

\paragraph{Step 3: Equality "$\implies$".} Let $D \neq 0, \, K_X$ be any divisor such that
\begin{equation*} \mathrm{deg} \, D = 2 \cdot \mathrm{dim} \, |D|. \end{equation*}
We argue by induction on the projective dimension of $|D|$. If $\mathrm{dim} \, |D| = 1$, then the degree of $D$ is equal to $2$ and thus there is nothing to be proved.

\vspace{0.4mm}
Assume that $\mathrm{deg} \, D \geq 4$. Let us consider a divisor $E \in |K_X - D|$, and let us pick two points $p, \, q \in X$ such that $p \in \mathrm{spt}(E)$ and $q \notin \mathrm{spt}(E)$. The dimension of $|D|$ is greater or equal than $2$, hence by \hyperref[lemma:charks]{Lemma \ref{lemma:charks}} there exists $D^\prime \in |D|$ such that
\begin{equation*} \{ p, \, q\} \subseteq \mathrm{spt} (D^\prime). \end{equation*}
Set $\widetilde{D} := D^\prime \cap E$, where the intersection between divisors is to be intended as follows:
\begin{equation*} \widetilde{D}(s) = \min \{ D^\prime(s), \, E(s) \}, \qquad \forall \, s \in X. \end{equation*}
By construction $q \notin \mathrm{spt}(E)$, thus $\mathrm{deg} \, \widetilde{D} < \mathrm{deg} \, D$; similarly $p \in \mathrm{spt}(E)$ implies that
\begin{equation*}p \in \mathrm{spt}(E) \cap \mathrm{spt}(D^\prime) \implies \deg \, \widetilde{D} > 0. \end{equation*}
Let us consider the short exact sequence
\begin{equation*} 0 \xrightarrow{} \mathcal{O}_X[ \widetilde{D}] \xrightarrow{\psi} \mathcal{O}_X[D] \oplus  \mathcal{O}_X[E]  \xrightarrow{\varphi}  \mathcal{O}_X[D + E - \widetilde{D}]  \xrightarrow{} 0,\end{equation*}
where $\psi = (\imath_1, \, \imath_2)$ is given by the pair of inclusions
\begin{equation*} \begin{aligned} & \imath_1 : \mathcal{O}_X[\widetilde{D}] \hookrightarrow \mathcal{O}_X[D], \\[1em] & \imath_2 : \mathcal{O}_X[D] \hookrightarrow \mathcal{O}_X[E], \end{aligned} \end{equation*}
and $\varphi = r_1 - r_2$ is the difference between the maps
\begin{equation*} \begin{aligned} & r_1 : \mathcal{O}_X[D] \longrightarrow \mathcal{O}_X\left[D + (E - \widetilde{D}) \right], \\[1em] & r_2 : \mathcal{O}_X[E] \longrightarrow \mathcal{O}_X\left[D + (E - \widetilde{D}) \right]. \end{aligned} \end{equation*}
The long exact sequence in cohomology gives us the inequality
\begin{equation*} \begin{aligned} h^0 \left(X, \, \mathcal{O}_X[D] \oplus \mathcal{O}_X[E] \right) & = h^0 \left(X, \, \mathcal{O}_X[D] \right) + h^0 \left(X, \, \mathcal{O}_X[E] \right) \leq \\[1em] & \leq h^0 \left(X, \, \mathcal{O}_X[\widetilde{D}] \right) + h^0 \left(X, \, \mathcal{O}_X\left[D + E - \widetilde{D} \right] \right), \end{aligned}\end{equation*}
while the fact that $E \sim K_X - D$ implies that
\begin{equation*} h^0 \left(X, \, \mathcal{O}_X[D] \right) + h^0 \left(X, \, \mathcal{O}_X[E] \right) \leq h^0 \left(X, \, \mathcal{O}_X [\widetilde{D}]\right) + h^0 \left(X, \, \mathcal{O}_X\left[K_X - \widetilde{D} \right] \right),\end{equation*}
that is,
\begin{equation*} \mathrm{dim} \, |D| + \mathrm{dim} \, |K_X - D| \leq \mathrm{dim} \, \left|\widetilde{D} \right| + \mathrm{dim} \, \left|K_X - \widetilde{D} \right|.\end{equation*}
At this point, we observe that: \mbox{}
\begin{enumerate}[label=\textbf{(\arabic*)}]
\item The left-hand side may be computer explicitly, i.e.,
\begin{equation*} \mathrm{dim} \, |D| + \mathrm{dim} \, |K_X - D|  = g(X) - 1.\end{equation*}
Indeed, from \eqref{cliff1} we infer that
\begin{equation*}\mathrm{dim} \, |D| = \frac{1}{2} \, \mathrm{deg} \, D \implies  \mathrm{dim} \, |K_X - D|  = g(X) - 1 - \frac{1}{2} \, \mathrm{deg} \, D,\end{equation*}
and this implies that
\begin{equation*} \mathrm{dim} \, |D| + \mathrm{dim} \, |K_X - D|  = g(X) - 1 - \frac{1}{2} \, \mathrm{deg} \, D + \frac{1}{2} \, \mathrm{deg} \, D = g(X) - 1.\end{equation*}
\item The right-hand side, by \hyperref[corollary:ewe]{Corollary \ref{corollary:ewe}}, satisfies the following inequality:
\begin{equation*} \mathrm{dim} \, \left|\widetilde{D} \right| + \mathrm{dim} \, \left|K_X - \widetilde{D} \right| \leq g(X) - 1.\end{equation*}
\end{enumerate}
Therefore the projective dimension of the linear system associated with $\widetilde{D}$ is equal to half of the degree, and by inductive assumption, it turns out that $X$ is \textit{hyperelliptic} and 
\begin{equation*} \left| \widetilde{D} \right| = \bar{r} \cdot g_2^1. \end{equation*}
It remains to prove that $D$ itself is a multiple of $g_2^1$. Let $E$ be the hyperelliptic divisor of $X$, that is, the divisor such that $|E| = g_2^1$, and let us set $s := \mathrm{dim} \, |D|$; we want to prove that
\begin{equation*}D + \left( g(X) - 1 - s \right) \cdot E = K_X.\end{equation*}
First, we notice that
\begin{equation*} \mathrm{dim} \, |(g(X) - 1 - s) \cdot E| = g(X) - 1 - s, \end{equation*}
since it can be obtained by repeating $g(X) - 1 - s$ the linear system $g_2^1$; hence by \hyperref[corollary:ewe]{Corollary \ref{corollary:ewe}} it turns out that
\begin{equation*}\begin{aligned} \mathrm{dim} & \, |D| + \mathrm{dim} \, |(g(X) - 1 - s) \cdot E| = g(X) - 1 \implies \\[1em] & \implies h^0 \left(X, \, \mathcal{O}_X\left[D + \left(g(X) - 1 - s \right) \cdot E \right] \right) \geq g(X).\end{aligned} \end{equation*}
On the other hand, a straightforward computation proves that 
\begin{equation*}\mathrm{deg} \, D + \mathrm{deg} \, (g(X) - 1 - s) \cdot E = \mathrm{deg} \, D + 2 \cdot \left( g(X) - 1 - \frac{1}{2} \, \mathrm{deg} \, D \right) = 2 \, g(X) - 2,  \end{equation*}
and thus the Clifford inequality allows us to infer that
\begin{equation*}\begin{aligned} \mathrm{dim} & \, \left| D + \left(g(X) - 1 - s\right) \cdot E \right| \leq g(X) - 1 \implies \\[1em] & \implies h^0 \left(X, \, \mathcal{O}_X\left[D + \left(g(X) - 1 - s \right) \cdot E \right] \right) \leq g(X).\end{aligned} \end{equation*}
In conclusion, the claim is proved since $K_X$ is the unique divisor - up to the equivalence relation - satisfying the two properties
\begin{equation*} \mathrm{deg} \, K_X = 2 \, g(X) - 2 \qquad \text{and} \qquad h^0 \left(X, \, \mathcal{O}_X[K_X] \right) = g(X). \end{equation*}
In particular, the divisor $D$ is equal to $K_X - \left( g(X) - 1 - s \right) \cdot E$ and $X$ is hyperelliptic; hence
\begin{equation*} K_X = \left( g(X) - 1 \right) \cdot E \implies D = s \cdot E = s \cdot g_2^1, \end{equation*}
which is exactly what we wanted to prove. \end{proof}

\part{Abel-Jacobi Map}
\chapter{Abel's Theorem} \thispagestyle{empty}
\label{chap:11}

In this final chapter, we first investigate the relation between the divisor group $\left(\Div(X), \, +\right)$ with the Picard group $\left( \Pic(X), \, \otimes \right)$, where $X$ is a compact Riemann surface.

Successively, we introduce the \textit{Jacobian manifold} associated with a compact Riemann surface $X$, and we prove that there is an isomorphism
\begin{equation*} \Pic^0(X) \cong \Jac(X). \end{equation*}
In particular, we derive this result from the Abel theorem (simply stated) and the Jacobi inversion theorem (entirely proved).

\section{Picard Group}
\label{sec:pic}

In this section, we denote by $\mathcal{O}_X$ and $\mathcal{M}_X$ the set of all the holomorphic (respectively, meromorphic) functions $f: X \to \C$ - where $X$ is a compact Riemann surface -. Moreover, we denote by $\mathcal{O}_X^\ast$ and $\mathcal{M}_X^\ast$ respectively, the nonzero elements.

\begin{proposition} Let $X$ be a Riemann surface. There is a canonical isomorphism
\begin{equation*}\varphi : \left( \Div(X), \, + \right) \xrightarrow{\quad\sim \quad} H^0 \left( X, \, \faktor{\mathcal{M}_X^\ast}{\mathcal{O}_X^\ast} \right), \end{equation*}
which is also a group homomorphism. \end{proposition}

\begin{proof}In the literature, the sheaf $\faktor{\mathcal{M}_X^\ast}{\mathcal{O}_X^\ast}$ is called the \textit{divisor sheaf}\index{Divisor!sheaf} of $X$.

\paragraph{Step 1.} Let
\begin{equation*}\sigma \in H^0 \left( X, \, \faktor{\mathcal{M}_X^\ast}{\mathcal{O}_X^\ast} \right)\end{equation*}
be a global section, and let $\mathcal{U} = \{ U_\alpha \}_{\alpha \in I}$ be an open covering of $X$ satisfying the usual properties, that is,
\begin{equation*}\sigma \, \big|_{U_\alpha} = f_\alpha, \qquad \text{$f_\alpha : U_\alpha \to \C$ meromorphic function} \end{equation*}
such that
\begin{equation*}f_\alpha \neq 0 \qquad \text{and} \qquad \frac{f_\alpha}{f_\beta} \in \mathcal{O}_X^\ast \left( U_\alpha \cap U_\beta \right). \end{equation*}
In particular, for any $p \in U_\alpha \cap U_\beta$ it turns out that
\begin{equation*} \mathrm{ord}_p \, f_\alpha = \mathrm{ord}_p \, f_\beta, \end{equation*}
and hence the divisor
\begin{equation*}D = \sum_{p \in X} \mathrm{ord}_p \, f_{\alpha_p} \cdot p \end{equation*}
is well-defined, where $U_{\alpha_p}$ is an element of $\mathcal{U}$ that contains $p$.

\paragraph{Step 2.} Let $D = \sum_{p \in X} n(p) \cdot p$ be a divisor of $X$. There exists an open covering $\{ U_\alpha \}_{\alpha \in I}$ of $X$ with the property that for $p$ there is a neighborhood $U_p$ and there is a holomorphic function $g_{p, \, \alpha} \in \mathcal{O}_X(U_\alpha)$ such that
\begin{equation*} \mathrm{div} \left( g_{p, \, \alpha} \, \big|_{U_p \cap U_\alpha} \right) = p. \end{equation*}
We define
\begin{equation*}f_\alpha = \prod_{p \in X} \left( g_{\alpha, \, p} \right)^{n(p)} \in \mathcal{M}_X^\ast(U_\alpha),\end{equation*}
and it is easy to prove that the collection $\{f_\alpha\}_{\alpha}$ defines a global section of $\faktor{\mathcal{M}_X^\ast}{\mathcal{O}_X^\ast}$.
\end{proof}

\paragraph{Picard Group.} Recall that the invertible sheaves on $X$ form a group, called the \textit{Picard group} of $X$, with the tensor product; more precisely, we have the isomorphism
\begin{equation*} \left( \Pic(X), \, \otimes \right) \cong H^1 \left( X, \, \mathcal{O}^\ast \right). \end{equation*}
The short exact sequence
\begin{equation*} 0 \xrightarrow{} \mathcal{O}_X^\ast \xrightarrow{} \mathcal{M}_X^\ast \xrightarrow{} \faktor{\mathcal{M}_X^\ast}{\mathcal{O}_X^\ast} \xrightarrow{} 0 \end{equation*}
induces a long exact sequence in cohomology, i.e.,
\begin{equation*} \begin{aligned} 0 \xrightarrow{} \: & H^0 \left(X, \,  \mathcal{O}_X^\ast \right) \xrightarrow{} H^0 \left(X, \,  \mathcal{M}_X^\ast \right) \xrightarrow{} H^0 \left(X, \,  \faktor{\mathcal{M}_X^\ast}{\mathcal{O}_X^\ast} \right) \xrightarrow{} \dots \\[1em] & \dots \xrightarrow{} H^1 \left(X, \,  \mathcal{O}_X^\ast \right) \xrightarrow{} H^1 \left(X, \,  \mathcal{M}_X^\ast \right) \xrightarrow{} H^1 \left(X, \,  \faktor{\mathcal{M}_X^\ast}{\mathcal{O}_X^\ast} \right) \xrightarrow{} 0. \end{aligned} \end{equation*}
By \hyperref[prop:abj12]{Proposition \ref{prop:abj12}} - which is proved afterwards - we have
\begin{equation*} H^1 \left(X, \, \mathcal{M}_X^\ast \right) = H^1 \left(X, \, \mathcal{M}_X \right) = 0. \end{equation*}
Hence the isomorphisms above, along with the definitions, prove that the long exact sequence reduces to the shorter exact sequence given by
\begin{equation*} 0 \xrightarrow{} \C^\ast \xrightarrow{} H^0 \left(X, \,  \mathcal{M}_X^\ast \right) \xrightarrow{} \Div(X)  \xrightarrow{} \Pic(X) \xrightarrow{} 0,\end{equation*}
and hence
\begin{equation*} \Div(X) \longtwoheadrightarrow \Pic(X). \end{equation*}

\paragraph{Equivalent Assertion.} Given an invertible sheaf $\mathcal{L}$, there is a covering $\{ U_i \}_{i \in I}$ of $X$ and there are maps
\begin{equation*} \psi_i : \mathcal{L} \, \big|_{U_i} \to \mathcal{O}_X(U_i), \qquad \sigma \mapsto f_i \cdot \sigma, \end{equation*}
and hence we can take $D \, \big|_{U_i} = \mathrm{div}(f_i)$.

\begin{proposition}Let $\sim$ be the equivalence relation on $\Div(X)$ given by
\begin{equation*} D_1 \sim D_2 \iff D_1 - D_2 = \mathrm{div}(g). \end{equation*}
The map $\varphi : \Div(X) \longrightarrow \Pic(X)$ passes to the quotient, and it induces an isomorphism
\begin{alignat*}{2}
  \widetilde{\varphi} : \faktor{\Div(X)}{\sim} & \longrightarrow & \Pic(X) \\[1em]
  D&\longmapsto& \: \mathcal{O}_X[D],
\end{alignat*}
which is also a group homomorphism.\end{proposition}

\begin{proof} Let $\{ U_i \}_{i \in I}$ be a covering of $X$, and let $f_i$ be the local equation of $D$, that is,
\begin{equation*} \mathrm{div} \, f_i = D \, \big|_{U_i}. \end{equation*}
The map $\psi_i : \mathcal{O}_X(U_i) \longrightarrow \mathcal{O}_X[D](U_i)$ sends $1$ to $\frac{1}{f_i}$ for every $i \in I$, and hence we have a nice local explicit expression for $\Psi$.

\paragraph{Injective.} We claim that
\begin{equation*} \varphi(D) = 0 \iff D = \mathrm{div}(h) \iff D \sim 0. \end{equation*}
If $D = \mathrm{div} \, h$, then the sheaf $\mathcal{O}_X[D]$ is globally defined by $\frac{1}{h}$, and hence it is isomorphic to $\mathcal{O}_X$, that is, $\varphi(D) = 0$.

Vice versa, if $\varphi(D) = 0$, then the map $\mathcal{O}_X[D] \longrightarrow \mathcal{O}_X$ is an isomorphism and it defined by the product with a function $h$, i.e.,
\begin{equation*} \Psi : \mathcal{O}_X[D] \to \mathcal{O}_X, \qquad \sigma \longmapsto h \cdot \sigma. \end{equation*}
In particular, the divisor $D$ is equal to $\mathrm{div} \, \frac{1}{h}$ by definition, and this is enough to prove that the claim holds.

\paragraph{Group Homomorphism.} We want to prove that
\begin{equation*} \varphi(D_1 + D_2) \: \stackrel{?}{=} \: \mathcal{O}_X[D_1] \otimes \mathcal{O}_X[D_2]. \end{equation*}
But this is an easy consequence of the fact that there is an isomorphism given by
\begin{equation*} \mathcal{O}_X[D_1] \otimes \mathcal{O}_X[D_2] \xrightarrow{\sim} \mathcal{O}_X[D_1 + D_2], \qquad \sigma_1 \otimes \sigma_2 \longmapsto \sigma_1 \cdot \sigma_2. \end{equation*}
\end{proof}

\paragraph{Structure of $\Pic(X)$.} The Picard group of $X$ may also be seen as the countable disjoint union of "Picard strips", that is,
\begin{equation*} \Pic(X) = \bigsqcup_{d \in \Z} \Pic^d(X), \end{equation*}
where
\begin{equation*} \Pic^d(X) = \left\{ \mathcal{L} \in \Pic(X) \: \left| \: \text{deg} \, \mathcal{L} = d  \right.\right\}. \end{equation*}
The \textit{degree} of an invertible sheaf may be defined by the surjective map $\Div(X) \twoheadrightarrow \Pic(X)$ simply considering the image of $\Div^d(X)$. In a similar fashion, one may define it as
\begin{equation*} \text{deg} \, \mathcal{L} := \chi \left(X, \,  \mathcal{L} \right) - \chi \left(X, \,  \mathcal{O}_X \right), \end{equation*}
coherently with the fact that
\begin{equation*} \mathrm{deg} \, D = d \implies \text{deg} \, \mathcal{O}_X[D] = d, \end{equation*}
as a straightforward consequence of the \hyperref[RiemannRoch]{Riemann-Roch Theorem \ref{RiemannRoch}}

\begin{remark} For every divisor $d \in \Z$ there is an isomorphism
\begin{equation*} \Pic^d(X) \xrightarrow{\quad \sim \quad} \Pic^0(X), \qquad \mathcal{L} \longmapsto \mathcal{L} \otimes \mathcal{O}_X[- d \cdot p], \end{equation*}
which can also be seen via $\varphi$ as
\begin{equation*} \Div^d(X) \xrightarrow{\quad \sim \quad} \Div^0(X), \qquad D \longmapsto D - d \cdot p. \end{equation*} \end{remark}

\section{Jacobian of $X$}

Let $X$ be a compact Riemann surface of genus $g(X) := g$, and let us consider the symplectic basis\footnote{\textbf{Definition.} A symplectic basis is a basis $e_1, \, f_1, \, \dots, \, e_n, \, f_n$ of a vector space endowed with a nondegenerate alternating bilinear form satisfying \eqref{10239dldlsd}.} of the first homology group $H_1(X, \, \Z)$ given by the closed paths $a_1, \, \dots, \, a_g, \, b_1, \, \dots, \, b_g$ (see \hyperref[Figure:Toro]{Figure \ref{Figure:Toro}}), satisfying the intersection conditions
\begin{equation} \begin{cases} a_i \cdot b_i = 1 & \forall \, i =1, \, \dots, \, g, \\[0.5em] a_j \cdot b_k = 0 & otherwise.\end{cases} \label{10239dldlsd} \end{equation}.

\begin{figure}[h]
\centering
\includegraphics[width = 14cm, height =8cm]{images/GAC201.png}
\caption{The generators of the $\pi_1(\T_g)$.}
\label{Figure:Toro}
\end{figure} 

\begin{remark} Recall that, for every divisor $D \in \Div(X)$,
\begin{equation*} \Omega_1^X[D] = \mathcal{O}_X[K_X + D]. \end{equation*}
By \hyperref[serreduality1]{Serre Duality Theorem \ref{serreduality1}} it turns out that
\begin{equation*} H^0 \left(X, \, \Omega_X^1 \right) \cong H^0 \left(X, \, \mathcal{O}_X[K_X] \right) \implies H^0 \left(X, \, \Omega_X^1 \right) \cong H^1 \left(X, \, \mathcal{O}_X \right) . \end{equation*} \end{remark}

\begin{definition}[Period]\index{Period} A \textit{period} is a linear functional
\begin{equation*} H^0 \left(X, \, \Omega_X^1 \right) \ni \omega \longmapsto \int_{[\mathcal{C}]} \omega \in \C, \end{equation*}
where $\left[\mathcal{C}\right]$ is a class of $H_1 \left(X, \, \Z \right)$. \end{definition}

\paragraph{Periods Matrix.}\index{Periods Matrix} Fix a basis $\{ \omega_1, \, \dots, \, \omega_g \}$ of $H^0 \left(X, \, \Omega_X^1 \right)$, and let us consider the symplectic basis $\{a_1, \, \dots, \, a_g, \, b_1, \, \dots, \, b_g\}$ of $H_1 \left(X, \, \Z \right)$. The matrix
\begin{equation*} \begin{pmatrix} \int_{a_1} \omega_1 & \dots & \dots & \int_{a_g} \omega_1 & \int_{b_1} \omega_1 & \dots & \dots & \int_{b_g} \omega_1 \\ \vdots & \ddots & & \vdots & \vdots & \ddots & & \vdots \\ \vdots &  & \ddots& \vdots & \vdots & & \ddots & \vdots \\ \int_{a_1} \omega_g & \dots & \dots & \int_{a_g} \omega_g & \int_{b_1} \omega_g & \dots & \dots & \int_{b_g} \omega_g  \end{pmatrix} \in \mathrm{M}\left(\C, \, g \times 2g \right),\end{equation*}
is called the \textit{periods matrix}, and it has the form $(A \left| B \right. )$. From now on, we shall denote by
\begin{equation*} \begin{aligned} & A_i = \left( \int_{a_i} \omega_j \right)_{j = 1, \, \dots, \, g}, \\[1em] & B_i = \left( \int_{b_i} \omega_j \right)_{j = 1, \, \dots, \, g}, \end{aligned} \end{equation*}
the columns of the these two matrices.

\begin{lemma} \mbox{}
\begin{enumerate}[label=\textbf{(\arabic*)}]
\item The matrices $A, \, B \in \mathrm{M}\left(\C, \, g \times g \right)$ are invertible.
\item The column vectors $A_i, \, B_i$ are a real basis of $\C^g$, i.e., they are $2g$ $\R$-linear vectors.
\item The transpose commutes, that is,
\begin{equation*}A^T B = B^T A. \end{equation*}
\end{enumerate}
\end{lemma}

\begin{definition}[Jacobian]\index{Jacobian} The Jacobian of $X$ is the quotient between the dual space of $\Omega_X^1$ and the lattice induced by the matrix of periods, that is,
\begin{equation*} \Jac(X) := \frac{H^0 \left( X, \, \Omega_X^1  \right)^v}{ \left( \int_{a_i} \omega_j, \, \int_{b_i} \omega_j \right)_{i, \, j = 1, \, \dots, \, g}}. \end{equation*} \end{definition}

\begin{remark}The Jacobian of $X$ is also equal to the quotient
\begin{equation*} \Jac(X) = \frac{H^0\left(X, \, \mathcal{O}_X[K_X] \right)^v}{ \imath \left( H^1(X, \, \Z) \right)}, \end{equation*}
where $\imath : H^1(X, \, Z) \to H^0 \left(X, \, \Omega_X^1 \right)^v$ is the map sending a class of equivalence $[\mathcal{C}]$ to the functional
\begin{equation*}H^0 \left(X, \, \Omega_X^1 \right) \ni \omega \longmapsto \int_{[\mathcal{C}]} \omega \in \C. \end{equation*}\end{remark}

\begin{remark}Let us consider $\Lambda$ - the lattice in $\C^g$ generated by $a_1, \, \dots, \, a_g, \, b_1, \, \dots, \, b_g$ -, that is,
\begin{equation*} \Lambda := \left\{ \sum_{i = 1}^g m_i \cdot A_i + \sum_{j = 1}^g n_j \cdot B_j \: \left| \: m_i, \, n_j \in \Z \right. \right\}. \end{equation*}
The Jacobian of $X$ is isomorphic to the quotient $\faktor{\C^g}{\Lambda}$ via the inclusion maps, and it is hence a complex (compact) torus. \end{remark}

\section{Abel-Jacobi Map}

Let us fix a point $p_0 \in X$. The Abel-Jacobi map\index{Abel-Jacobi map} is defined by
\begin{equation*} A_{p_0} : X \longrightarrow \Jac(X) \qquad p \longmapsto \begin{pmatrix}  \displaystyle\int_{p_0}^p \omega_1 \\[1em] \vdots \\[1em]  \displaystyle\int_{p_0}^p \omega_g \end{pmatrix} \qquad \left( \text{mod $\Lambda$} \right), \end{equation*}\index{Abel-Jacobi map!on points}
where
\begin{equation*} \int_{p_0}^p = \int_\gamma, \end{equation*}
for any path $\gamma$ starting from $p_0$ and ending in $p$.

\begin{remark}The Abel-Jacobi map is well-defined. Indeed, if $\gamma$ and $\gamma^\prime$ are two paths between $p_0$ and $p$, then
\begin{equation*} \int_\gamma \omega - \int_{\gamma^\prime} \omega = \int_\eta \omega \in \Lambda, \end{equation*} 
where $\eta$ is a closed path with base point $p_0$ (see \hyperref[fig:201]{Figure \ref{fig:201}}).\end{remark}

\begin{figure}[h]
\centering
\includegraphics[width = 12cm, height =8cm]{images/GAC202.png}
\caption{Well-definition of the Abel-Jacobi map}
\label{fig:201}
\end{figure} 

The map $A_{p_0} : X \to \Jac(X)$ may be extended by linearity to $\Div(X)$, and from now on we shall denote by $A_0$ the restriction of $A_{p_0}$ to $\Div^0(X)$. \index{Abel-Jacobi map!on divisors}

\begin{remark}The map $A_0$ does not depend on the base point $p_0$. Indeed, if we consider a divisor
\begin{equation*} D = \sum_{i = 1}^n p_i - \sum_{i = 1}^n q_i \in \Div^0(X), \end{equation*}
then
\begin{equation*} A_0(p_i - q_i) = \begin{pmatrix}  \displaystyle\int_{\alpha_i} \omega_1 \\[1em] \vdots \\[1em]  \displaystyle\int_{\alpha_i} \omega_g \end{pmatrix} - \begin{pmatrix}  \displaystyle\int_{\beta_i} \omega_1 \\[1em] \vdots \\[1em]  \displaystyle\int_{\beta_i} \omega_g \end{pmatrix} = \begin{pmatrix}  \displaystyle\int_{\alpha_i} \omega_1 - \int_{\beta_i} \omega_1 \\[1em] \vdots \\[1em]  \displaystyle\int_{\alpha_i} \omega_g - \int_{\beta_i} \omega_g \end{pmatrix}\qquad \left( \text{mod $\Lambda$} \right).\end{equation*}
The reader may jump to \hyperref[fig:202]{Figure \ref{fig:202}} to have a better understating of what is going on here. For every $i$ the closed path $\eta_i := \alpha_i - \gamma_i - \beta_i$ belongs to $H_1 \left(X, \, \Z \right)$, and hence
\begin{equation*}\left( \int_{\eta_i} \omega_1, \, \dots, \, \int_{\eta_i} \omega_g \right)^T  = (0, \, \dots, \, 0)^T.\end{equation*}
As a consequence, we obtain
\begin{equation*} A_0(p_i - q_i) = \begin{pmatrix}  \displaystyle\int_{\gamma_i} \omega_1 \\[1em] \vdots \\[1em]  \displaystyle\int_{\gamma_i} \omega_g \end{pmatrix} \qquad  \left( \text{mod $\Lambda$} \right),\end{equation*}
which implies that $A_0 : \Div^0(X) \to \Jac(X)$ does not depend on the base point $p_0$.\end{remark}

\begin{figure}[h]
\centering
\includegraphics[width = 12cm, height =12cm]{images/GAC203.png}
\caption{Independence of $A_0$ from $p_0$.}
\label{fig:202}
\end{figure} 

\section{Abel-Jacobi Theorems}

In this final section, we state and partially prove two fundamental results concerning the Abel-Jacobi map $A_0$, and we conclude by showing the main consequences they lead us to.

\begin{theorem}[Abel's Theorem]\index{Abel's Theorem}\label{th:abel} Let $X$ be a compact Riemann surface, and let $D \in \Div^0(X)$. Then $A_0(D) = 0$ if and only if there exists a meromorphic function $f : X \to \C$ such that $D = \mathrm{div}(f)$, i.e., $D \sim 0$. \end{theorem}

\begin{theorem}[Jacobi Inversion Theorem]\index{Jacobi Theorem}\label{th:jacobi} Let $X$ be a compact Riemann surface, and fix $p_0 \in X$. For any $\lambda \in \Jac(X)$, there are $p_1, \, \dots, \, p_g \in X$ points such that
\begin{equation*} A_0 \left( \sum_{i = 1}^g (p_i - p_0) \right) = \lambda. \end{equation*}
Moreover, if $\lambda$ is a generic point, then the divisor $D = \sum_{i =1}^g p_i$ is unique.\end{theorem}

\begin{corollary}There are isomorphisms
\begin{equation*} \mathrm{Pic}^0(X) \cong \faktor{\Div^0(X)}{\sim} \cong \Jac(X). \end{equation*}\end{corollary}

\paragraph{Abel-Jacobi Map in Positive Degree.}\index{Abel-Jacobi map!on divisors of degree $d$} Fix $p_0 \in X$, and let us consider the set of all positive divisor of degree $d$, that is,
\begin{equation*} \Div_+^d(X) := \left\{ D \in \Div^d(X) \: \left| \: D \geq 0 \right. \right\}. \end{equation*}
The map $A_d$ is defined by taking the restriction of $A_{p_0}$ to $\Div_+^d(X)$ and, in particular, it is defined by setting
\begin{equation*} A_d : \Div_+^d(X) \longrightarrow \Jac(X), \qquad D \longmapsto \begin{pmatrix}  \displaystyle \sum_{i = 1}^{d} \int_{p_0}^{p_i} \omega_1 \\[1em] \vdots \\[1em]  \displaystyle\sum_{i = 1}^{d} \int_{p_0}^{p_i} \omega_g \end{pmatrix} \qquad \left( \text{mod $\Lambda$} \right). \end{equation*}
The Abel-Jacobi map induces a commutative diagram given by
\begin{equation*} \begin{tikzcd} \Div_+^d(X) \ar[d, "\tau"] \ar[r, "A_d"] & \Jac(X) \ar[d] \ar[r, dotted, "\sim"] & \Pic^d(X) \ar[d, "t"]
\\[1em] \Div^0(X) \ar[r, "A_0"] & \Jac(X)\ar[u] \ar[r, dotted, "\sim"] & \Pic^0(X)  \end{tikzcd} \end{equation*}
where
\begin{equation*} \tau(D) = D - d \cdot p_0 \qquad \text{and} \qquad t(\mathcal{L}) = \mathcal{L} \otimes \mathcal{O}_X[-d \cdot p_0]. \end{equation*}
Observe that $X$ is canonically isomorphic to $\Div_+^1(X)$, hence the Abel-Jacobi map induces
\begin{equation*} A_1 : X \longrightarrow \Jac(X), \qquad D \longmapsto \begin{pmatrix}  \displaystyle \int_{p_0}^{p} \omega_1 \\[1em] \vdots \\[1em]  \displaystyle \int_{p_0}^{p} \omega_g \end{pmatrix} \qquad \left( \text{mod $\Lambda$} \right). \end{equation*}

\begin{proposition}\label{prop:211}Let $X$ be a compact connected Riemann surface of genus $g(X) \geq 1$. Then the Abel-Jacobi map $A_1 : X \to \Jac(X)$ is injective. \end{proposition}

\begin{proof}We argue by contradiction. Let $p, \, q \in X$ be points such that $A_1(p) = A_1(q)$, and observe that the divisor $p - q$ has degree zero. Therefore
\begin{equation*} A_1(p) - A_1(q) = A_0(p - q) = 0, \end{equation*}
and by \hyperref[th:abel]{Abel Theorem} it follows that $p - q = \mathrm{div}(f)$, where $f$ is a function with a zero and a pole. The resulting morphism $f : X \to \p^1(\C)$ has degree one, and thus $X \cong \p^1(\C)$, which is absurd since the genii are different by assumption. \end{proof}

\begin{remark}Let $\lambda \in \Jac(X)$ be a point such that
\begin{equation*} \lambda = A_{p_0}(D) \end{equation*}
for some divisor $D \in \Div_+^d(X)$. Then
\begin{equation*} A_{p_0}^{-1}(\lambda) = |D| \end{equation*}
since $A_{p_0}(D) = A_{p_0}(D^\prime)$ if and only if $D^\prime \sim D$ (see the proof of \hyperref[prop:211]{Proposition \ref{prop:211}}).\end{remark}

\begin{proposition}\label{prop:212} Let $X$ be a compact connected Riemann surface, and let $D \in \Div_+^{g(X)}(X)$ be a generic divisor. Then $h^1 \left(X, \, \mathcal{O}_X[D] \right) = 0$. \end{proposition}

\begin{proof}By \hyperref[th:jacobi]{Jacobi Theorem} it turns out that for any $\lambda \in \Jac(X)$ generic point there exists a unique divisor $D \in \Div_+^{g(X)}(X)$ such that $A_{p_0}(D) = \lambda$. The previous remark, on the other hand, asserts that
\begin{equation*} A_{p_0}^{-1}(\lambda) = |D|, \end{equation*}
and hence $|D| = \{D\}$ is given by the divisor itself only. Hence $h^0 \left(X, \, \mathcal{O}_X[D] \right) = 1$, and by the \hyperref[RiemannRoch]{Riemann-Roch Theorem} it follows that
\begin{equation*}  h^1 \left(X, \, \mathcal{O}_X[D] \right) = h^0 \left(X, \, \mathcal{O}_X[D] \right) - \mathrm{deg} \, D - 1 + g(X) = g(X) + 1 - g(X) - 1 = 0. \end{equation*}
\end{proof}

\begin{proposition}\label{prop:abj12}Let $X$ be a compact Riemann surface, and let $\mathcal{M}$ be the field of all the meromorphic functions defined on $X$, and let $\mathcal{M}_X$ be the constant sheaf on $X$. Then
\begin{equation*} h^1 \left(X, \, \mathcal{M}_X \right) = 0. \end{equation*} \end{proposition}

\begin{proof}Let $D \in \Div(X)$ be a divisor such that $h^1\left(X, \, \mathcal{O}_X[D] \right) = 0$ (which exists by \hyperref[prop:212]{Proposition \ref{prop:212}}). There is a long exact sequence in cohomology given by
\begin{equation*} \begin{aligned} 0 \xrightarrow{} & H^0\left(X, \, \mathcal{O}_X[D] \right) \xrightarrow{} H^0\left(X, \, \mathcal{M}_X \right) \xrightarrow{} H^0\left(X, \, \tau_X[D] \right) \xrightarrow{} \dots \\[1em]
&\dots \xrightarrow{} H^1\left(X, \, \mathcal{O}_X[D] \right) \xrightarrow{} H^1\left(X, \, \mathcal{M}_X \right) \xrightarrow{} H^1\left(X, \, \tau_X[D] \right). \end{aligned} \end{equation*}
The sheaf $\tau_X[D]$ is supported on a finite set of points, hence the dimension $h^1\left(X, \, \tau_X[D]) \right)$ is equal to zero; by construction we also have that $h^1\left(X, \, \mathcal{O}_X[D] \right) = 0$, thus the thesis holds true.\end{proof}

\paragraph{Conclusion.} In this last paragraph, we only give a proof of the Jacobi inversion theorem. The Abel theorem requires a lot of work, and we will give it for granted (the reader may consult \cite[pp. 250-263]{miranda} for a lengthy proof).

\begin{proof}[Proof of Jacobi Theorem] Let us denote by $X^{(g)}$ the symmetric product of $g(X)$ copies of $X$, that is, the quotient
\begin{equation*} X^{(g)} = \faktor{X \times \dots \times X}{\sigma_g}. \end{equation*}
First, we observe that $\mathrm{dim}_\C \, X = 1$ is a particular case because it allows us to conclude that $X^{(g)}$ is a $g$-manifold\footnote{The reader may refer to \href{https://arbourj.wordpress.com/2014/05/11/jacobis-inversion-theorem/}{this post} for a proof of this fact.} (i.e., $X^{(g)}$ is not singular, as it could have been in higher dimension). Let
\begin{equation*} A^{(g)} : X^{(g)} \longrightarrow \Jac(X), \qquad \begin{pmatrix}p_1 \\[1em] \vdots \\[1em] p_g \end{pmatrix} \longmapsto \begin{pmatrix}  \displaystyle \sum_{i = 1}^{g} \int_{p_0}^{p_i} \omega_1 \\[1em] \vdots \\[1em]  \displaystyle\sum_{i = 1}^{g} \int_{p_0}^{p_i} \omega_g \end{pmatrix} \qquad \left( \text{mod $\Lambda$} \right). \end{equation*}
be the Abel-Jacobi map naturally defined on $X^{(g)}$. It is enough to prove that for any generic point $(p_1, \, \dots, \, p_g) \in X^{(g)}$, the Abel-Jacobi map $A^{(g)}$ is a \textbf{local isomorphism}.

\paragraph{Step 1.} Let $\{\omega_1, \, \dots, \, \omega_g\}$ be a basis for $H^0 \left( X, \, \Omega_X^1 \right)$, and take $p_1, \, \dots, \, p_g \in X$ distinct points such that
\begin{equation*} \omega_i = h_i \, \mathrm{d}z_i, \end{equation*}
where $z_i$ is the local coordinate relative to $p_i$ and $(z_1, \, \dots, \, z_g)$ local coordinates of $X^{(g)}$, and also satisfying the following property: the $g \times g$ complex matrix
\begin{equation*} \left( h_j(p_i) \right)_{i, \, j = 1, \, \dots, \, g} \end{equation*}
is upper triangular, and it is not degenerate.

\paragraph{Step 2.} In a neighborhood of $(p_1, \, \dots, \, p_g) \in X^{(g)}$ it turns out that
\begin{equation*} A^{(g)} (p_1, \, \dots, \,  p_g) = \begin{pmatrix}  \displaystyle \sum_{i = 1}^{g} \int_{p_0}^{p_i + z_i} \omega_1 \\[1em] \vdots \\[1em]  \displaystyle\sum_{i = 1}^{g} \int_{p_0}^{p_i + z_i} \omega_g \end{pmatrix} \qquad \left( \text{mod $\Lambda$} \right), \end{equation*}
and hence
\begin{equation*} \frac{\partial}{\partial \, z_i} \, A^{(g)} (p_1, \, \dots, \,  p_g) = \frac{\partial}{\partial \, z_i} \,  \begin{pmatrix}  \displaystyle \sum_{i = 1}^{g} \int_{p_0}^{p_i + z_i} h_1 \, \mathrm{d}z_1\\[1em] \vdots \\[1em]  \displaystyle\sum_{i = 1}^{g} \int_{p_0}^{p_i + z_i} h_g \, \mathrm{d}z_g \end{pmatrix} = \begin{pmatrix} h_1(p_1) \\[1em] \vdots \\[1em] h_g(p_g) \end{pmatrix}. \end{equation*}
The matrix associated with the differential $\mathrm{d} A^{(g)}$ is thus given by
\begin{equation*} \left( h_j(p_i) \right)_{i, \, j = 1, \, \dots, \, g}, \end{equation*}
which is a maximal rank matrix by construction (recall that we chose the $p_i$'s in such a way to have this property), and hence $A^{(g)}$ is a local isomorphism.

\paragraph{Step 4.} The surface $X$ is irreducible and compact. Hence the map $A^{(g)}$ is proper, the manifold $\Jac(X)$ is irreducible and $A^{(g)}$ is also a dominant map\footnote{The reader may jump \href{http://stacks.math.columbia.edu/tag/01RI}{here} for an overview of dominant morphisms between schemes.

\begin{definition} A morphism $f:X \to S$ of schemes is \textit{dominant} if the image of $f$ is a dense subset of $S$. \end{definition}

\begin{lemma}Let $f : X \to S$ be a morphism of schemes. If every generic point of every irreducible component of $S$ is in $\mathrm{Im} \, f$, then $f$ is dominant.\end{lemma}}, that is, $A^{(g)}$ is surjective.

\paragraph{Step 5.} The fiber of $A^{(g)}$ is isomorphic to the linear system $|p_1 + \dots + p_g|$, and this is isomorphic to the projectivization of the global sections, i.e.,
\begin{equation*}|p_1 + \dots + p_g| \cong \p \left( H^0 \left(X, \, \mathcal{O}_X[p_1 + \dots + p_g] \right) \right). \end{equation*}
Since $A^{(g)}$ is a local isomorphism, it turns out that the dimension of the fiber is zero and hence there exists one and only one divisor $p_1 + \dots + p_g$ (if a projective subspace contains two points, then it contains the whole line between them).
\end{proof}

\printindex

\bibliography{Bibliografia}{} % BIBLIOGRAFIA
\bibliographystyle{plain}
\end{document}
