\chapter{The Riemann-Roch Theorem and Serre Duality} \thispagestyle{empty}
\label{chap:8}

Let $X$ be a Riemann surface, and let be given a divisor $D \in \Div(X)$. The primary goals of \hyperref[chap:8]{Chapter \ref{chap:8}} and of \hyperref[chap:10]{Chapter \ref{chap:10}} are the following: \mbox{}
\begin{enumerate}[label=\textbf{\arabic*)}]
\item Find an isomorphism for the $0$th cohomology group $H^0\left(X, \, \mathcal{O}_X[D] \right)$, or, at least, an estimate of the dimension $h^0\left(X, \, \mathcal{O}_X[D] \right)$.
\item Study the map $\varphi_{|D|}$. More precisely, we would like to know if $|D|$ is a b.p.f. linear system (i.e., if $\varphi_{|D|}$ is a morphism) and, in that case, if $\varphi_{|D|}$ is injective (or, even better, an embedding).
\end{enumerate}

\section{Rough Estimate of $h^0 \left(X, \, \mathcal{O}_X[D] \right)$}

\begin{remark}Let $X$ be a compact Riemann surface, and let $p \in X$. There exists a short exact sequence of sheaves, given by
\begin{equation*} 0 \xrightarrow{} \mathcal{I}_p \xrightarrow{} \mathcal{O}_X \xrightarrow{} \C_p \xrightarrow{} 0,\end{equation*}
where $\mathcal{I}_p$ is the sheaf of the ideals of the function vanishing at $p$, $\mathcal{O}_X$ is the holomorphic function sheaf and $\C_p$ is the skyscraper sheaf.\end{remark}

\begin{proposition}Let $X$ be a compact Riemann surface, and let $p \in X$. There is an isomorphism of sheaves
\begin{equation*} \mathcal{I}_p \cong \mathcal{O}_X[-p]. \end{equation*}\end{proposition}

\begin{proof}Notice that, locally, $p$ is the divisor of a function $\varphi : U \to \Delta \subset \C$ that sends $p$ to $0$. The reader may fill in the details of the proof as an exercise, following one of the two possibilities below:\mbox{}
\begin{enumerate}[label=\textbf{(\arabic*)}]
\item Prove that $\mathcal{O}_X[-p]$ is locally generated by $\varphi$, as a consequence of the fact that the sheaf $\mathcal{O}_X[p]$ is (locally) generated by the function $\frac{1}{\varphi}$.
\item Prove that for any subset $U \subseteq X$
\begin{equation*} f \in \mathcal{O}_X[-p](U) \iff f = \varphi \cdot g, \end{equation*}
where $g$ is an holomorphic function defined on $U$. 
\end{enumerate}\end{proof} 

\begin{proposition} \label{exas}Let $D \in \Div(X)$ be a divisor, and let $p \in X$ be a point. Then there is a short exact sequence of sheaf maps
\begin{equation*} 0 \xrightarrow{} \mathcal{O}_X[D - p] \xrightarrow{} \mathcal{O}_X[D] \xrightarrow{} \C_p \xrightarrow{} 0.\end{equation*}
\end{proposition}

\begin{proof} It suffices to prove the exactness in each position.

\paragraph{Left Exactness.} There is a natural inclusion
\begin{equation*} \mathcal{O}_X[D - p] \hookrightarrow \mathcal{O}_X[D], \end{equation*}
as a consequence of the fact that
\begin{equation*}\mathrm{div} \, f + D - p \geq 0 \implies \mathrm{div} \, f + \underbrace{(D - p) + p}_{= D} \geq 0. \end{equation*}

\paragraph{Middle/Right Exactness.} Let $U_y \subseteq X$ be an open neighborhood of a point $y \neq p \in X$, and assume that $p \notin U_y$. In this case it is straightforward to prove that
\begin{equation*} \mathcal{O}_X[D - p](U_y) \cong \mathcal{O}_X[D](U_y) \end{equation*}
is an isomorphism, coherently with the fact that
\begin{equation*} \C_p(U_y) = 0. \end{equation*}
Let $U_p \subseteq X$ be an open neighborhood of $p$. If we set $m := \mathrm{ord}_p \, D^\prime$, then it turns out that there is a meromorphic function 
\begin{equation*} f(z) = z^{-(m+1)} \, h(z) \qquad \text{locally in $U_p$,} \end{equation*}
for some $h$ non-vanishing at $p$, such that
\begin{equation*}\mathrm{div} \, f \, \big|_{U_p} + (D - p) \, \big|_{U_p} = - p. \end{equation*}
It follows that the function $f$ generates the cokernel, and thus
\begin{equation*} \faktor{\mathcal{O}_X[D]}{\mathcal{O}_X[D - p} \cong \C \cdot \{ z^{m+1} \} \cong \C \cong \C_p, \end{equation*}
since $\C \cdot \{ z^{m+1} \}$ is supported at $p$.
\end{proof}

\begin{remark}More generally, there is a functor
\begin{equation*} F : \left( \Div(X), \, + \right) \longrightarrow \left( \{ \text{invertible sheaves} \}, \, \otimes \right), \end{equation*}
defined by
\begin{equation*} D \mapsto \mathcal{O}_X[D] \qquad \text{and} \qquad D_1 + D_2 \mapsto \mathcal{O}_X[D_1] \otimes \mathcal{O}_X[D_2] \cong \mathcal{O}_X[D_1 + D_2]. \end{equation*}
It is not hard to prove that this functor is exact, since
\begin{equation*} \left( 0 \xrightarrow{} \mathcal{O}_X[-p] \xrightarrow{} \mathcal{O}_X \xrightarrow{} \C_p \xrightarrow{} 0 \right) \otimes_{\mathcal{O}_X} \mathcal{O}_X[D]\end{equation*}
is isomorphic to
\begin{equation*} 0 \xrightarrow{} \mathcal{O}_X[D-p] \xrightarrow{} \mathcal{O}_X[D] \xrightarrow{} \C_p \xrightarrow{} 0 .\end{equation*}
This functor is analyzed more in-depth in \hyperref[sec:pic]{Section \ref{sec:pic}} using the language of the Picard group. \end{remark}

\begin{remark} Recall that
\begin{equation*} \mathrm{deg} \, D < 0 \implies \begin{cases} H^0\left(X, \, \mathcal{O}_X[D] \right) = 0, \\[0.5em] |D| = \emptyset. \end{cases} \end{equation*} \end{remark}

\begin{proposition}\label{prop:stimah0} Let $X$ be a compact connected Riemann surface, and let $D$ be a divisor of positive degree $d \geq 0$. Then the following estimate holds true:
\begin{equation} \label{eq:stimah0} h^0 \left(X, \, \mathcal{O}_X[D] \right) \leq d + 1. \end{equation} \end{proposition}

\begin{proof} We first distinguish between divisor with empty linear system and divisor with a nontrivial linear system, and then we proceed by induction on the degree of $D$.

\paragraph{Case $|D| = \emptyset$.} If $|D| = \emptyset$, then $\mathrm{deg} \, D = 0$ (since it is positive by assumption) and thus we infer by the previous remark that
\begin{equation*}h^0 \left(X, \, \mathcal{O}_X[D] \right) = 0. \end{equation*}

\paragraph{Case $|D| \neq \emptyset$, Base Step.} Let $D$ be a divisor of degree $0$, and let $E \in |D|$ be an effective divisor linearly equivalent to $D$.

Clearly $E$ is the null divisor $0$ (the coefficients are positive, and they sum to zero;) thus $|D| = \{0\}$ and the dimension satisfies the estimate \eqref{eq:stimah0} as expected:
\begin{equation*}h^0 \left(X, \, \mathcal{O}_X[D] \right) = 1. \end{equation*}

\paragraph{Case $|D| \neq \emptyset$, Inductive Step.}  Let $D$ be a divisor of degree $d$, let $E \in |D|$ be an effective divisor, and take $p \in \mathrm{spt}(E)$. By \hyperref[exas]{Proposition \ref{exas}} there is a short exact sequence
\begin{equation*} 0 \xrightarrow{} \mathcal{O}_X[E - p] \xrightarrow{} \mathcal{O}_X[E] \xrightarrow{} \C_p \xrightarrow{} 0,\end{equation*}
which induces a long sequence in cohomology (see \hyperref[longcomo]{Theorem \ref{longcomo}}):
\begin{equation*} 0 \xrightarrow{} H^0 \left(X, \, \mathcal{O}_X[E - p] \right) \xrightarrow{} H^0 \left(X, \, \mathcal{O}_X[E] \right) \xrightarrow{} H^0\left(X, \, \C_p \right) \xrightarrow{} \dots\end{equation*}
The map $H^0 \left(X, \, \mathcal{O}_X[E] \right) \xrightarrow{} H^0\left(X, \, \C_p \right)$ is not necessarily surjective, therefore we can only infer an inequality between the dimensions, i.e.,
\begin{equation*} h^0 \left(X, \, \mathcal{O}_X[E] \right) \leq h^0 \left(X, \, \mathcal{O}_X[E - p] \right) + h^0 \left(X, \, \C_p \right). \end{equation*}
In conclusion, recall that there is an isomorphism $H^0\left(X, \, \C_p \right) \cong \C$, and apply the inductive hypothesis to obtain the sought inequality:
\begin{equation*} h^0 \left(X, \, \mathcal{O}_X[E] \right) \leq (d - 1) + 1 + 1 = d+1. \end{equation*}
\end{proof}

\paragraph{Skyscraper Sheaf.}\index{Sheaf!skyscraper} Let $X$ be a Riemann surface, let $p \in X$ be a point and take any natural number $n \in \N$. Take $D \in \Div(X)$ and set $\Delta := n \cdot p$; by \hyperref[exas]{Proposition \ref{exas}} there is a short exact sequence
\begin{equation*} 0 \xrightarrow{} \mathcal{O}_X[D - \Delta] \xrightarrow{} \mathcal{O}_X[D] \xrightarrow{} \mathcal{O}_\Delta \xrightarrow{} 0.\end{equation*}
The sheaf $\mathcal{O}_\Delta$ is the skyscraper sheaf, supported in $\{p\}$, and such that
\begin{equation*}\left( \mathcal{O}_\Delta \right)_q = \begin{cases} 0 & \text{if $q \neq p$} \\[0.8em] \faktor{\mathcal{O}_{X, \, p}}{\mathcal{M}_{X, \, p}^n} \cong \C^n & \text{if $q = p$}, \end{cases} \end{equation*}
where $\mathcal{O}_{X, \, p}$ is the stalk of the holomorphic function sheaf at $p$, and $\mathcal{M}_{X, \, p}$ is the maximal ideal at $p$. More precisely, in the local coordinate we have $p = 0$ and $\mathrm{div}(z) = p$, which in turn implies that
\begin{equation*} \faktor{\mathcal{O}_{X, \, p}}{(x^n)} = \left\{ a_0 + \dots + a_{n-1} \, x^{n-1} \right\}. \end{equation*}

\begin{remark}The above argument can be easily generalized to a linear combination of points. \mbox{}
\begin{enumerate}[label=\textbf{(\alph*)}]
\item The isomorphism $H^0 \left( X, \, \mathcal{O}_\Delta \right) \cong \C^n$ proves that
\begin{equation*} h^0 \left( X, \, \mathcal{O}_\Delta \right) = n. \end{equation*}
\item By \hyperref[prop:sptttt]{Proposition \ref{prop:sptttt}} it turns out that
\begin{equation*} H^1 \left( X, \, \mathcal{O}_\Delta \right) = 0,  \end{equation*}
since the dimension of the support is zero - i.e., strictly less than the index of the cohomology group.
\item If $\Delta = n_1 \cdot p_1 + \dots + n_k \cdot p_k$, then $\mathcal{O}_\Delta$ is a skyscraper sheaf supported in $\{p_1, \, \dots, \, p_k\}$. In a similar fashion, the reader may prove that
\begin{equation*} h^0 \left( X, \, \mathcal{O}_\Delta \right) = \sum_{i = 1}^{k} n_i \qquad \text{and} \qquad h^1 \left( X, \, \mathcal{O}_\Delta \right) = 0.\end{equation*}
\end{enumerate}
\end{remark}

\section{Riemann-Roch Formula}

The major result we want to achieve in this section is the \textit{Riemann-Roch} formula for compact connected Riemann surfaces.

\begin{remark}Recall that the dimension of $X$ over $\C$ is, by definition, equal to $1$. Therefore Betti's numbers are all zero except for $h^0$ and $h^1$, that is,
\begin{equation*} h^j \left( X, \, \mathcal{O}_X[D] \right) = 0 \qquad \forall \, j \geq 2, \, \, \forall \, D \in \Div(X). \end{equation*} \end{remark}

\begin{definition}[Holomorphic Euler Characteristic]\index{Euler-Poincaré characteristic!holomorphic} The \textit{Euler (holomorphic) characteristic} of a surface $X$ with respect to a divisor $D \in \Div(X)$ is defined by
\begin{equation*} \chi \left(X, \,  \mathcal{O}_X[D] \right) = h^0\left(X, \, \mathcal{O}_X[D] \right) - h^1\left(X, \, \mathcal{O}_X[D] \right). \end{equation*} \end{definition}

\begin{definition}[Arithmetic Genus]\index{Arithmetic genus} The \textit{arithmetic genus} of a Riemann surface $X$ is defined by
\begin{equation*} p_a(X) := 1 - \chi \left( X, \, \mathcal{O}_X \right). \end{equation*}
\end{definition}

\begin{remark}If $X$ is a compact connected Riemann surface, then $h^0 \left( X, \, \mathcal{O}_X \right) = 1$ and hence
\begin{equation*} p_a(X) = h^1 \left(X, \, \mathcal{O}_X \right). \end{equation*} \end{remark}

\begin{theorem}[Riemann-Roch]\index{Riemann-Roch Theorem} \label{RiemannRoch} Let $X$ be a connected compact Riemann surface, and let $D \in \Div(X)$ be any divisor. Then it turns out that
\begin{equation}\label{r-r:eq1} \chi \left(X, \,  \mathcal{O}_X[D] \right) = \mathrm{deg} \, D + \chi \left( X, \, \mathcal{O}_X \right), \end{equation}
or, equivalently, that
\begin{equation}\label{r-r:eq2} h^0 \left( X, \, \mathcal{O}_X[D] \right) - h^1 \left( X, \, \mathcal{O}_X[D] \right) = \mathrm{deg} \, D +1 - p_a(X). \end{equation}\end{theorem}

\begin{proof} We first assume that the divisor $D$ is effective and we derive \eqref{r-r:eq1} by induction on the degree of $d = \mathrm{deg} \, D$; only then we solve the general case.

\paragraph{Effective Divisor, Base Step.} If $\mathrm{deg} \, D = 0$, then $D = 0$ (since the coefficients are positive and their sum is equal to zero). It follows that
\begin{equation*} \mathcal{O}_X[D] \cong \mathcal{O}_X \implies \eqref{r-r:eq1}. \end{equation*}

\paragraph{Effective Divisor, Inductive Step.} Suppose that $\mathrm{deg} \, D = d > 0$, and let $p \in \mathrm{spt}(D)$ be a point. By \hyperref[exas]{Proposition \ref{exas}} there is a short exact sequence
\begin{equation*} 0 \xrightarrow{} \mathcal{O}_X[D - p] \xrightarrow{} \mathcal{O}_X[D] \xrightarrow{} \C_p \xrightarrow{} 0,\end{equation*}
which induces an identity on the Euler characteristics, that is,
\begin{equation*}  \chi \left(X, \,  \mathcal{O}_X[D] \right) =  \chi \left(X, \,  \mathcal{O}_X[D - p] \right) + \chi \left(X, \,  \C_p \right). \end{equation*}
As we have already observed, the Euler characteristic of $\C_p$ is given by the difference between $h^0\left(X, \, \C_p \right)$ and $h^1\left(X, \, \C_p \right)$; since the dimension of the support is less than $1$, it turns out that
\begin{equation*} \chi \left(X, \,  \C_p \right) = h^0 \left( X, \, \C_p \right) - h^1 \left( X, \, \C_p \right) = 1 - 0 = 1. \end{equation*}
Using the induction hypothesis, we immediately obtain the thesis for an effective divisor:
\begin{equation*}  \chi \left(X, \,  \mathcal{O}_X[D] \right) =  \chi \left(X, \,  \mathcal{O}_X \right) + \mathrm{deg}(D) - \mathrm{deg}(p) + 1 =  \chi \left(X, \,  \mathcal{O}_X \right) + \mathrm{deg}(D). \end{equation*}

\paragraph{General Divisor.} Let $D$ be any divisor, and let
\begin{equation*}D = D^+ - D^-\end{equation*}
be the decomposition of $D$ in the positive part and the negative part (both of which are effective). As usual, by \hyperref[exas]{Proposition \ref{exas}} there is a short exact sequence
\begin{equation*} 0 \xrightarrow{} \mathcal{O}_X[D^+ - D^-] \xrightarrow{} \mathcal{O}_X[D^+] \xrightarrow{} \C_{D^-} \xrightarrow{} 0,\end{equation*}
which induces an identity of Euler characteristics, that is,
\begin{equation*} \begin{aligned} \chi \left(X, \,  \mathcal{O}_X[D] \right) & = \chi \left(X, \,  \mathcal{O}_X[D^+] \right) - \chi \left(X, \, \C_{D^-} \right) = \\[1em] & = \chi \left(X, \, \mathcal{O}_X \right) + \mathrm{deg} \, D^+ - \mathrm{deg} \, D^- = \\[1em] & = \chi \left( X, \, \mathcal{O}_X \right) + \mathrm{deg} \, D, \end{aligned} \end{equation*}
and this concludes the proof.\end{proof}

\section{Serre Duality}

The primary goal of this section is to use every tool we have introduced so far to prove the notorious \textit{Serre Duality Theorem}, which will come to handy to justify different results in the following sections.

\begin{theorem}[Serre] \label{serreduality1} Let $X$ be a compact connected Riemann surface, let $K_X$ be a canonical divisor and let $D$ be any divisor on $X$. Then there is an isomorphism
\begin{equation*} H^1 \left(X, \, \mathcal{O}_X[D] \right)^{v} \cong H^0 \left(X, \, \mathcal{O}_X \left[K_X - D \right] \right), \end{equation*}
where $^v$ denotes the dual vector space.\end{theorem}

\subsection{Mittag-Leffler Problem}
\index{Mittag-Leffler problem}

Let $X$ be a compact Riemann surfaces. Let $p_1, \, \dots, \, p_s \in X$ be given points, and suppose that for any $i = 1, \, \dots, \, s$ there is a \textit{polar polynomial}, that is,
\begin{equation*} h_i(z) = \sum_{k = - n_i}^{-1} a_k \, z^k, \qquad \text{in $U_{p_i} \cong \Delta$ neighborhood of $p_i$ with local coordinate $z$}. \end{equation*}
In this section, we investigate the Mittag-Leffler problem, that is, we want to determine if there exists a function meromorphic on $X$ such that: \mbox{}
\begin{enumerate}[label=\textbf{(\arabic*)}]
\item The function $f: X \to \C$ is holomorphic outside of the finite set $\{p_1, \, \dots, \, p_s\}$.
\item The principal part of $f$ in $U_{p_i}$ is given by the polar polynomial $h_i$.
\end{enumerate}
A meromorphic function $f: X \to \C$ satisfying these properties exists locally, but the problem is to find one globally defined. The answer, as we shall be able to prove soon, depends on
\begin{equation*} H^0 \left( X, \, \mathcal{O}_X[D] \right) \qquad \text{and} \qquad H^1 \left( X, \, \mathcal{O}_X[D] \right), \end{equation*}
where the divisor is simply defined by
\begin{equation*} D := \sum_{i = 1}^{s} n_i \cdot p_i. \end{equation*}

\paragraph{Laurent Tails.}\index{Laurent tails} Let $X$ be a Riemann surface, let $p \in X$ be a point, and let $U_p \ni p$ be an open neighborhood with coordinate $z_p$. A \textit{Laurent tail} with respect to $p$ is a function of the form
\begin{equation}\label{lttttt} r_p(z_p) = \sum_{i = - n_p}^{k_p} a_i \, z_p^i,  \end{equation}
where $a_i \in \C$ are complex coefficients.

\begin{definition}[Laurent Tail Divisor]\index{Divisor!Laurent tail} A \textit{Laurent tail divisor} on $X$ is a finite formal sum
\begin{equation*} \sum_{p \in X} r_p(z_p) \cdot p, \end{equation*}
where $r_p(-)$ is a Laurent polynomial in the local coordinate $z_p$, that is, a Laurent series of the form \eqref{lttttt} with a finite number of terms. \end{definition}

\begin{notation}Let $X$ be a Riemann surface. We denote by $\mathcal{T}_X$ the set of all the Laurent tail divisors defined on $X$. \end{notation}

\begin{definition}[Laurent Tail Sheaf]\index{Sheaf!Laurent tails} Let 
\begin{equation*} D = \sum_{p \in X} D(p) \cdot p \in \Div(X).\end{equation*}
The \textit{Laurent tail divisor sheaf} associated to $D$ is defined by setting
\begin{equation}\label{sjesd} U \longmapsto \mathcal{T}_X[D](U) := \left\{\left. \sum_{p \in X} r_p( - ) \cdot p \: \right| \: \forall \, p \in U \: : \: k_p < - D(p) \right\}, \end{equation}
where $k_p$ is the maximal order of the function $r_p$, as defined in \eqref{lttttt}. \end{definition}

The reader may check by herself that \eqref{sjesd} actually defines a sheaf. For every divisor $D \in \Div(X)$, there is a truncation map
\begin{equation*} t_D : \mathcal{T}_X(U) \longrightarrow \mathcal{T}_X[D](U), \end{equation*}
which is defined by
\begin{equation*} \sum_{p \in X} r_p(-) \cdot p \longmapsto \sum_{p \in X} t_D(r_p)(-) \cdot p, \end{equation*}
where
\begin{equation*} t_D(r_p)(z_p) = \sum_{i = - n_p}^{- D(p) - 1} a_i \, z_p^i.\end{equation*}

\paragraph{Meromorphic Field.} Let us consider the field
\begin{equation*} \M := \left\{ \text{field of meromorphic function on $X$} \right\}. \end{equation*}
The constant presheaf may be also defined by setting
\begin{equation*} \M_X(U) := \left\{ f : U \to \M \: \left| \: \text{$f$ continuous and $\M$ has the discrete topology} \right. \right\}, \end{equation*}
in such a way that
\begin{equation*} \text{$U$ connected} \implies \M_X(U) \cong \M, \end{equation*}
and the restriction maps are the identity maps. If we denote by $\M_X$ the associated sheaf, then one can prove that \mbox{}
\begin{enumerate}[label=\textbf{(\alph*)}]
\item $H^0 \left(X, \, \M_X \right) \cong \M$, and
\item $H^1 \left(X, \, \M_X \right) = 0$.
\end{enumerate}
In particular, for every divisor $D \in \Div(X)$ there exists a homomorphism of sheaves
\begin{equation*} \alpha_D : \M_X \longrightarrow \mathcal{T}_X[D], \end{equation*}
which can be easily defined \textit{locally} as
\begin{equation*} p \in U_p \implies f(z_p) = \sum_{i \geq - n_p} a_i \, z_p^i \longmapsto r_p(z_p) = \sum_{i = - n_p}^{- D(p) - 1} a_i \, z^i. \end{equation*}
By definition, the kernel of $\alpha_D$ is isomorphic to $\mathcal{O}_X[D]$; hence there is a short exact sequence of sheaf maps
\begin{equation*} 0 \xrightarrow{} \mathcal{O}_X[D] \xrightarrow{} \M_X \xrightarrow{\alpha_D} \mathcal{T}_X[D] \xrightarrow{} 0 \end{equation*}
inducing a long exact sequence in cohomology, that is,
\begin{equation*} 0 \xrightarrow{} H^0 \left(X, \, \mathcal{O}_X[D] \right) \xrightarrow{} H^0 \left(X, \, \M_X \right) \xrightarrow{\alpha_D} H^0 \left(X, \, \mathcal{T}_X[D] \right) \xrightarrow{} H^1 \left(\mathcal{O}_X[D] \right) \xrightarrow{} 0. \end{equation*}
By \hyperref[rmk:sdksdkskdksd]{Remark \ref{rmk:sdksdkskdksd}} we can infer that
\begin{equation*} H^0 \left(X, \, \M_X \right) \cong \M(X) \qquad \text{and} \qquad H^0 \left(X, \, \mathcal{T}_X[D] \right) \cong \mathcal{T}_X[D](X), \end{equation*}
and thus it follows that
\begin{equation*} L(D) := H^0 \left(X, \, \mathcal{O}_X[D] \right) \cong \mathrm{ker}(\alpha_D) \qquad \text{and} \qquad H^1 \left(X, \, \mathcal{O}_X[D] \right) \cong \mathrm{coker}(\alpha_D). \end{equation*}
By definition, there is an isomorphism
\begin{equation*} \mathcal{O}_X[K_x - D] \cong \Omega_X^1[-D],\end{equation*}
from which it follows that
\begin{equation*} \begin{aligned} H^0 \left(X, \, \mathcal{O}_X[K_x - D]\right) & \cong H^0 \left(X, \, \Omega_X^1[-D] \right) = \\[1em] & = \left\{ \omega = f(z) \, \mathrm{d}z \: \left| \: \text{$f$ meromorphic and $\mathrm{ord}_p(f) \geq D(p)$} \right. \right\}. \end{aligned} \end{equation*}

\subsection{Proof of Serre Duality Theorem}
\index{Serre Duality Theorem}

\paragraph{Road Map.} In this section, we finally demonstrate the \hyperref[serreduality1]{Serre Theorem \ref{serreduality1}} based on what we have proved so far. The road map of the proof is the following:  \mbox{}
\begin{enumerate}[label=\textbf{(\arabic*)}]
\item There exists a pairing
\begin{equation*} \mathrm{Res}(\cdot, \, \cdot) : H^0 \left( X, \, \Omega_X^1[-D] \right) \times \mathcal{T}_X[D](X) \to \C. \end{equation*}
\item The map defined above pass to the quotient. More precisely, it turns out that
\begin{equation*} \mathrm{Res}(\cdot, \, T) \equiv 0 \qquad \forall \, T \in \mathrm{Im}(\alpha_D), \end{equation*}
and hence there exists a pairing
\begin{equation*} \mathrm{Res}(\cdot, \, \cdot) : H^0 \left( X, \, \Omega_X^1[-D] \right) \times \mathrm{coker}(\alpha_D) \to \C. \end{equation*}
\item The pairing defined in the previous step is non degenerate.
\end{enumerate}

\begin{proof}[Proof of Theorem \ref{serreduality1}] The argument is rather involved. Hence we divide it into many different steps.

\paragraph{Step 1.} Let
\begin{equation*} \mathrm{Res}(\cdot, \, \cdot) : H^0 \left( X, \, \Omega_X^1[-D] \right) \times \mathcal{T}_X[D](X) \to \C, \qquad (\omega, \, T) \mapsto \mathrm{Res}(\omega, \, T) \end{equation*}
be the map defined by
\begin{equation*} \mathrm{Res}(\omega, \, T) := \sum_{p \in X} \mathrm{Res}_p(T_p \, \omega). \end{equation*}
More precisely, if $U_p$ is an open neighborhood of a point $p \in X$ with local coordinate $z_p$, then it turns out that
\begin{equation*}\omega = \sum_{i \geq D(p)} \left( c_i \, z_p^i \right) \, \mathrm{d}z_p \qquad \text{and} \qquad T_p = \sum_{i = - s_p}^{-D(p) - 1} a_i \, z_p^i, \end{equation*}
and thus
\begin{equation*}\mathrm{Res}_p(T_p \, \omega) = \sum_{i \geq D(p)} a_{-i - 1} \cdot c_i \end{equation*}
is exactly equal to the coefficient of $z_p^{-1}$. 

\paragraph{Step 2.} In this step, the primary goal is to prove that the map defined above descends to the quotient in the second variable, that is, to
\begin{equation*} \faktor{\mathcal{T}_X[D](X)}{\mathrm{Im}(\alpha_D)}. \end{equation*}
Let $f \in \mathcal{M}$ be a meromorphic function, let $p \in X$ be any point, and let $z_p$ be the associated local coordinate; then
\begin{equation*} f(z_p) = \sum_{i \geq - n_p} a_i \, z_p^i \longmapsto \alpha_D(f)(z_p) = \sum_{i \geq - n_p}^{- D(p) - 1} a_i \, z_p^i. \end{equation*}
The residue at $p$ is thus given by
\begin{equation*}\mathrm{Res}_p(f \cdot \omega) = \sum_{i \geq D(p)} a_{-i - 1} \cdot c_i = \mathrm{Res}_p(\alpha_D(f) \cdot \omega) \end{equation*}
since the terms whose index is $j \geq - D(p)$ of $\alpha_D(f)$ do not give any contribution to the sum above. The \hyperref[residui]{Residue Theorem \ref{residui}} immediately implies that
\begin{equation*}\sum_{p \in X} \mathrm{Res}_p(f \cdot \omega) = 0 \implies \mathrm{Res} \left( \alpha_D(f) \cdot \omega \right) = 0,\end{equation*}
which is exactly what we wanted to prove.

\paragraph{Step 3.} In this step, we want to prove that the functional
\begin{equation} \label{serre:fun} \mathrm{Res} : H^0 \left(X, \, \mathcal{O}_X[K_X - D] \right) \to \left( H^1 \left(X, \, \mathcal{O}_X[D] \right) \right)^v, \qquad \omega \mapsto \mathrm{Res}(\omega, \, -) \end{equation}
is an isomorphism (i.e., the pairing is non degenerate). \mbox{}
\begin{enumerate}[label=\textbf{(\alph*)}]
\item \textbf{Linear.} The linearity of the map \eqref{serre:fun} follows easily from the properties of the residue.
\item \textbf{Injective.} Let $D = \sum_{p \in X} D(p) \cdot p$, and let $\omega$ be such that
\begin{equation*} \mathrm{Res}\left(T, \, \omega \right) = 0, \qquad \forall \, T = \sum_{p \in X} T_p \cdot p. \end{equation*}
Let $p \in X$ be a point, let $z_p$ be the local coordinate, and let $k = \mathrm{ord}_p(\omega)$ (in particular, $- 1 - k < - D(p)$). It follows that
\begin{equation*} z_p^{-1-k} \cdot p \in \mathcal{T}_X[D](X), \end{equation*}
and also that, if we set $\omega := \sum_{i \geq k} \left( c_i \, z_p^i \right) \, \mathrm{d}z_p$, with the lowest coefficient $c_k$ different from $0$, then one can easily check that
\begin{equation*} \mathrm{Res}(\omega, \, z^{-1-k} \cdot p) = \mathrm{Res}_p (z_p^{-1-k} \cdot \sum_{i \geq k} c_i \, z_p^i \, \mathrm{d}z_p^i = c_k, \end{equation*}
which is not zero. This contradiction shows that $\mathrm{Res}(\omega, \, -)$ cannot be the identically zero map on $H^1\left(X, \, \mathcal{O}_X[D] \right)$, unless $\omega = 0$.
\item \textbf{Surjective.} Recall that
\begin{equation*} H^1\left(X, \, \mathcal{O}_X[D] \right) \cong \faktor{ \mathcal{T}_X[D](X) }{\mathrm{Im}(\alpha_D)}. \end{equation*}
Let us consider a functional $\Phi : \mathcal{T}_X[D](X) \to \C$ vanishing on the image of $\alpha_D$, that is, assume that
\begin{equation*} \Phi \, \big|_{\mathrm{Im}(\alpha_D)} \equiv 0. \end{equation*}
We want to construct a differential form $\omega \in H^0 \left(X, \, \Omega_X^1[-D] \right)$ such that
\begin{equation*} \Phi(-) = \mathrm{Res}(\omega, \, -). \end{equation*}
The proof of this property is a consequence of two technical lemmas; hence we interrupt the argument for a few pages and resume it when we are ready to conclude.
\end{enumerate}
\end{proof}

\paragraph{Truncation Maps.} Let $D_1, \, D_2 \in \Div(X)$ be two divisors, and assume that $D_1 \leq D_2$. There exists a truncation map
\begin{equation*} t_{D_2}^{D_1} : \mathcal{T}_X[D_1](X) \longrightarrow \mathcal{T}_X[D_2](X), \end{equation*}
which is defined by
\begin{equation*} \sum_{i \geq -n_p}^{- D_1(p) - 1} a_i \, z^i \longmapsto \sum_{i \geq -n_p}^{- D_2(p) - 1} a_i \, z^i. \end{equation*}
Let $D \sim D^\prime$ be linearly equivalent divisors (i.e., $D^\prime = D - \mathrm{div}(f)$). Let $p \in X$ be a point, and let $r_p \in \mathcal{T}_X[D](X)$ be the Laurent tail given by
\begin{equation*} r_p(z_p) = \sum_{i \geq - n_p}^{-D(p) - 1} a_i \, z_p^i. \end{equation*}
There is a unique integer $h$ such that $f(z_p) = z_p^h$ (i.e., it is a map of order $h$ at $p$, and $z_p$ is the local coordinate at $p$), and thus
\begin{equation*} \left(f \cdot r_p\right) (z_p) = \sum_{i \geq - n_p}^{-D(p) - 1} a_i \, z^{i + h} \qquad \text{and} \qquad \mathrm{deg}(f \cdot r_p) < - D(p) + \mathrm{ord}_p \, f = - D^\prime(p). \end{equation*}
We conclude that there exists an isomorphism
\begin{equation*} \mu_f : \mathcal{T}_X[D](X) \xrightarrow{ \quad \sim \quad } \mathcal{T}_X[D - \mathrm{div} \, f](X), \end{equation*}
which is defined by
\begin{equation*}\sum_{p \in X} r_p(-) \cdot p \longmapsto \sum_{p \in X} \left( f \cdot r_p \right)(-) \cdot p. \end{equation*}
To prove that $\mu_f$ is an actual isomorphism, it is enough to check that the map
\begin{equation*} \mu_{\frac{1}{f}} : \mathcal{T}_X[D - \mathrm{div}(f)](X) \longrightarrow \mathcal{T}_X[D](X) \end{equation*}
is the inverse.

\begin{remark} \mbox{}
\begin{enumerate}[label=\textbf{(\alph*)}]
\item It may be useful to rewrite the isomorphism as
\begin{equation*} \mu_f : \mathcal{T}_X[D + \mathrm{div} \, f](X) \xrightarrow{\quad \sim \quad} \mathcal{T}_X[D](X). \end{equation*}
\item Let $\Phi : \mathcal{T}_X[D](X) \longrightarrow \C$ be a linear functional. If $\Phi$ vanishes on $\mathrm{Im} \, \alpha_{D}$, then the composition $\Phi \circ \mu_f$ vanishes on the whole image of $\alpha_{D + \mathrm{div} \, f}$.
\end{enumerate} \end{remark}

\begin{lemma}[\cite{miranda}] \label{lemma:serre1} Let $\Phi_1$ and $\Phi_2$ be two linear functionals defined on $H^1\left(X, \, \mathcal{O}_X[A]\right)$ for some divisor $A \in \Div(X)$. There is a positive divisor $C$ and nonzero meromorphic functions $f_1, \, f_2 \in H^0\left(X, \, \mathcal{O}_X[C]\right)$ such that
\begin{equation*} \Phi_1 \circ t_{A}^{A - C - \mathrm{div} \, f_1} \circ \mu_{f_1} = \Phi_2 \circ t_A^{A - C - \mathrm{div} \, f_2} \circ \mu_{f_2} \end{equation*}
as functionals on $H^1\left(X, \, \mathcal{O}_X[A - C]\right)$. In other words, the two maps on $\mathcal{T}_X[A - C](X)$ in the diagram
\begin{equation*} \begin{tikzcd}& \mathcal{T}_X[A - C - \mathrm{div} \, f_1](X) \ar[r, "t"] & \mathcal{T}_X[A](X) \ar[dr, "\Phi_1"] & 
\\ \mathcal{T}_X[A - C](X) \ar[ur, "\mu_{f_1}"] \ar[dr, "\mu_{f_2}"]& & & \C \\
& \mathcal{T}_X[A - C - \mathrm{div}\, f_2](X) \ar[r, "t"] & \mathcal{T}_X[A](X) \ar[ur, "\Phi_2"] & \end{tikzcd} \end{equation*}
are equal for some $C$ and some $f_1, \, f_2 \in H^0\left(X, \, \mathcal{O}_X[C]\right) \setminus \{0\}$. \end{lemma}

\begin{proof}We argue by contradiction. Suppose that no such divisor $C$ and functions $f_i$ exist. Then for every positive divisor $C$ it turns out that the $\C$-linear map
\begin{equation*} H^0\left(X, \, \mathcal{O}_X[C]\right) \times H^0\left(X, \, \mathcal{O}_X[C]\right) \to \left(H^1\left(X, \, \mathcal{O}_X[A-C]\right)\right)^v \end{equation*} 
defined by sending a pair $(f_1, \, f_2)$ to
\begin{equation*} \Phi_1 \circ t_{A}^{A - C - \mathrm{div}(f_1)} \circ \mu_{f_1} - \Phi_2 \circ t_A^{A - C - \mathrm{div}\, f_2} \circ \mu_{f_2} \end{equation*}
is injective. In particular, for every such $C$ we must have
\begin{equation} \label{lemma:serre1:eq1}  h^1 \left(X, \, \mathcal{O}_X[A-C]\right) \geq 2 \cdot h^0\left(X, \, \mathcal{O}_X[C]\right), \end{equation}
and, as a consequence of the \hyperref[RiemannRoch]{Riemann-Roch Theorem \ref{RiemannRoch}}, we can also infer that
\begin{equation} \label{lemma:serre1:eq2} h^1 \left(X, \, \mathcal{O}_X[A-C]\right) = h^0 \left(X, \, \mathcal{O}_X[A-C]\right) + g(X) - 1 - \mathrm{deg}(A - C). \end{equation}
The divisor $C$ is positive; hence
\begin{equation*}h^0 \left(X, \, \mathcal{O}_X[A-C]\right) \leq h^0 \left(X, \, \mathcal{O}_X[A]\right) \qquad \text{and} \qquad \mathrm{deg}(A - C) \leq \mathrm{deg}(A). \end{equation*}
It follows from \eqref{lemma:serre1:eq1} and the \hyperref[RiemannRoch]{Riemann-Roch Theorem \ref{RiemannRoch}} that
\begin{equation*}h^1 \left(X, \, \mathcal{O}_X[A-C]\right) \geq 2 \cdot h^0\left(X, \, \mathcal{O}_X[C]\right) \geq 2 \left[ \mathrm{deg}(C) + 1 - g(X) \right] =2 \, \mathrm{deg}(C) + K_1, \end{equation*}
where $K_1$ is a constant, and it follows from  \eqref{lemma:serre1:eq2} that
\begin{equation*} h^1 \left(X, \, \mathcal{O}_X[A-C]\right) \leq \mathrm{deg}(C) + \left( h^0\left(X, \, \mathcal{O}_X[A] \right) + g(X) - 1 - \mathrm{deg}(A) \right) = \mathrm{deg}(C) + K_2, \end{equation*}
where $K_2$ is another constant. These growth rate are clearly incompatible for $\mathrm{deg}(C)$ sufficiently big, and this gives the sought contradiction.
\end{proof}

\begin{lemma}[\cite{miranda}] \label{lemma:serre2} Let $D_1 \in Div(X)$ be a divisor, and let $\omega \in H^0\left(X, \, \Omega_X^1[-D_1] \right)$ be a differential form. Suppose that there is another divisor $D_2 \geq D_1$ such that the residue map
\begin{equation*} \mathrm{Res}(\omega, \, -) : \mathcal{T}_X[D_1](X) \to \C \end{equation*}
vanishes on the kernel
\begin{equation*} \mathrm{Ker}\left( t_{D_2}^{D_1}  :  \mathcal{T}_X[D_1](X) \longrightarrow \mathcal{T}[D_2](X) \right).\end{equation*}
Then $\omega$ belong to $H^0 \left(X, \, \Omega_X^1[-D_2] \right)$. \end{lemma}

\begin{proof}We argue by contradiction. If $\omega \notin H^0 \left(X, \, \Omega_X^1[-D_2] \right)$, then there exists a point $p \in X$ with $k = \mathrm{ord}_p(\omega) < D_2(p)$. Let us consider the Laurent tail divisor
\begin{equation*} Z = z_p^{-k-1} \cdot p. \end{equation*}
Then $Z \in \mathrm{ker}(t_{D_2}^{D_1})$, but the residue map does not vanish; this contradiction proves the lemma.\end{proof}

\begin{proof}[Proof of Theorem \ref{serreduality1}, Part II] We are now ready to finish the proof of the Serre duality theorem. \mbox{}
\begin{enumerate}[label=\textbf{(\alph*)}]
\setcounter{enumi}{3}
\item \textbf{Surjective, Part II.} Let $\Phi : H^1 \left(X, \, \mathcal{O}_X[D] \right) \to \C$ be a functional, which we consider as a functional on $\mathcal{T}_X[D](X)$, vanishing on $\alpha_D(\mathcal{M}_X)$.

Let $\omega$ be a holomorphic $1$-form, and let $K = \mathrm{div}(\omega)$ be a canonical divisor so that
\begin{equation*} \omega \in H^0 \left(X, \, \mathcal{O}_X[K] \right) = H^0 \left(X, \, \Omega_X^1 \right). \end{equation*}
Let $A \in \Div(X)$ be a divisor such that $A \leq D$ and $A \leq K$, so that $\omega \in  H^0 \left(X, \, \Omega_X^1[-A] \right)$. Let us set $\Phi_A := \Phi \circ t_D^A : \mathcal{T}_X[A](X) \to \C$. By \hyperref[lemma:serre1]{Lemma \ref{lemma:serre1}} it turns out that there exists a divisor $C \geq 0$ and $f_1, \, f_2$ meromorphic functions such that
\begin{equation} \label{proofserre:eq1} \Phi_A \circ t_A^{A - C - \mathrm{div} \, f_1} \circ \mu_{f_1} = \mathrm{Res}(\omega, \, -) \circ t_A^{A - C - \mathrm{div} \, f_2} \circ \mu_{f_2}. \end{equation}
In the right-hand side of \eqref{proofserre:eq1} we have the map $\mathrm{Res}(\omega, \, -) \circ t_A^{A - C - \mathrm{div} \, f_2}$, which is nothing else than the residue map $\mathrm{Res}(\omega, \, -)$ acting on $\mathcal{T}_X[A - C - \mathrm{div} \, f_2](X)$; on the other hand, the composition $\mathrm{Res}(\omega, \, -) \circ \mu_{f_2}$ is exactly equal to
\begin{equation*} \mathrm{Res}(f_2 \cdot \omega, \, -) : \mathcal{T}_X[A-C](X) \to \C, \end{equation*}
and hence the identity \eqref{proofserre:eq1} becomes
\begin{equation*} \Phi_A \circ t_A^{A - C - \mathrm{div} \, f_1} \circ \mu_{f_1} = \mathrm{Res}(f_2 \cdot \omega, \, -). \end{equation*}
Composing with $\mu_{1/f_1}$ it turns out that
\begin{equation*} \Phi_A \circ t_A^{A - C - \mathrm{div}\, f_1} = \mathrm{Res}\left(\frac{f_2}{f_1} \cdot \omega, \, - \right) \end{equation*}
as functionals on $\mathcal{T}_X[A - C - \mathrm{div}\, f_1](X)$.

We observe that $(f_2/f_1) \, \omega$ belongs to $H^0\left(X, \, \Omega_X^1[C + \mathrm{div}\, f_1 - A] \right)$, and also that
\begin{equation*} \mathrm{Res}\left(\frac{f_2}{f_1} \cdot \omega, \, - \right) \equiv 0 \quad \text{on $\mathrm{Ker}\left( t_A^{A - C - \mathrm{div} \, f_1 } \right)$}. \end{equation*}
By \hyperref[lemma:serre2]{Lemma \ref{lemma:serre2}} we have that $(f_2/f_1) \, \omega \in H^0 \left( X, \, \Omega_X^1[-A] \right)$, and hence
\begin{equation*} \mathrm{Res}\left(\frac{f_2}{f_1} \cdot \omega, \, - \right) = \Phi_A. \end{equation*}
By definition, the map $\Phi_A$ is the composition between $\Phi$ and $t_D^A$; hence the residue map above vanishes on the kernel of $\mathrm{Ker}(t_D^A)$, which, in turn, implies that
\begin{equation*} \frac{f_2}{f_1} \cdot \omega \in H^0 \left( X, \, \Omega_X^1[- D] \right) \implies \Phi = \mathrm{Res} \left( \frac{f_2}{f_1} \cdot \omega, \, - \right) : H^1 \left(X, \, \mathcal{O}_X[D]\right) \longrightarrow \C, \end{equation*}
and this completes the proof of the theorem.
\end{enumerate} \end{proof}

\section{The Equality of the Three Genera}

\begin{corollary} \label{serredualitycor}Let $X$ be a compact connected Riemann surface and let $K_X$ be a canonical divisor. There is an isomorphism
\begin{equation*} H^1 \left(X, \, \mathcal{O}_X \right)^{v} \cong H^0 \left(X, \, \mathcal{O}_X \left[K_X \right] \right). \end{equation*}\end{corollary}

\begin{definition}[Geometric Genus]\index{Geometric genus} The \textit{geometric genus} of a Riemann surface $X$ is defined by
\begin{equation*} p_g(X) := h^0 \left( X, \, \mathcal{O}_X[K_X] \right). \end{equation*}
\end{definition}

\begin{corollary} Let $X$ be a compact connected Riemann surface. The three notions of genus are equivalent, that is,
\begin{equation*} p_a(X) = p_g(X) = g(X). \end{equation*}\end{corollary}

\begin{proof} The \hyperref[serreduality1]{Serre Duality Theorem \ref{serreduality1}} immediately implies that
\begin{equation*}p_a(X) = h^1 \left(X, \, \mathcal{O}_X \right) = h^0 \left(X, \, \mathcal{O}_X[K_X] \right) = p_g(X),\end{equation*}
therefore it remains to prove that one of them also coincides with the \textit{number of holes} $g(X)$.

The \hyperref[th:rh]{Riemann-Hurwitz formula \ref{th:rh}} asserts that $\mathrm{deg}(K_X) = 2 \left(g(X) - 1 \right)$, while the \hyperref[RiemannRoch]{Riemann-Roch Theorem \ref{RiemannRoch}} asserts that
\begin{equation*}h^0 \left( X, \, \mathcal{O}_X[D] \right) - h^1 \left( X, \, \mathcal{O}_X[D] \right) = \mathrm{deg} \, D +1 - p_a(X). \end{equation*}
The \hyperref[serreduality1]{Serre Duality Theorem \ref{serreduality1}} implies
\begin{equation*}2g - 1 = 2 \, p_a(X) - h^1 \left( X, \, \mathcal{O}_X[K_X] \right), \end{equation*}
hence it suffices to prove that $h^1 \left( X, \, \mathcal{O}_X[K_X] \right) = 1$. But this is, once again, a simple consequence of the Serre duality:
\begin{equation*} h^1 \left( X, \, \mathcal{O}_X[K_X] \right) = h^0 \left( X, \, \mathcal{O}_X \right)=  1. \end{equation*}\end{proof}

\section{Analytic Interpretation (Hodge)}
\index{Hodge analytic interpretation}

Let $\Omega_X^1$ be the sheaf of holomorphic $1$-form. There exists a short exact sequence of sheaves
\begin{equation*} 0 \xrightarrow{} \C \xrightarrow{} \mathcal{O}_X \xrightarrow{f \mapsto \mathrm{d}f} \Omega_X^1 \xrightarrow{} 0,\end{equation*}
where the middle map is locally defined as follows:
\begin{equation*}f(z) \longmapsto \mathrm{d}f(z) := f^\prime(z) \, \mathrm{d}z.\end{equation*}
The long exact sequence in cohomology is thus given by
\begin{equation*}\begin{aligned} 0 \xrightarrow{} H^0 \left(X, \, \C \right) & \xrightarrow{} H^0 \left(X, \, \mathcal{O}_X \right) \xrightarrow{} H^0 \left( X, \, \Omega_X^1 \right) \xrightarrow{} H^1 \left(X, \, \C \right) \xrightarrow{} \dots \\[1em] & \dots \xrightarrow{} H^1 \left(X, \, \mathcal{O}_X \right) \xrightarrow{} H^1 \left( X, \, \Omega_X^1 \right) \xrightarrow{} H^2 \left( X, \, \C \right) \xrightarrow{} 0. \end{aligned} \end{equation*}
Clearly $H^i \left(X, \, \C \right) \cong H^i \left(X, \, \R \right) \otimes_{\R} \C$ implies that
\begin{equation*} \begin{cases} H^0 \left(X, \, \C \right) \cong \C, \\[0.6em] H^1 \left(X, \, \C \right) \cong \C^{2g}, \\[0.6em] H^2 \left(X, \, \C \right) \cong \C, \end{cases} \end{equation*}
while
\begin{equation*} \begin{cases} H^0 \left(X, \, \mathcal{O}_X \right) \cong \C \implies H^0 \left(X, \, \mathcal{O}_X \right) \cong H^0 \left(X, \, \C \right), \\[0.6em] H^0 \left(X, \, \Omega_X^1 \right) = H^0 \left(X, \, \mathcal{O}_X[K_X] \right), \\[0.6em] H^1 \left(X, \, \Omega_X^1 \right) = H^1 \left(X, \, \mathcal{O}_X[K_X] \right) \cong H^0 \left(X, \, \mathcal{O}_X \right) \cong \C \cong H^2 \left(X, \, \C \right). \end{cases} \end{equation*}
We infer that there is a short exact sequence
\begin{equation*}0 \xrightarrow{} H^0 \left( X, \, \Omega_X^1 \right) \xrightarrow{} H^1 \left(X, \, \C \right) \xrightarrow{}  H^1 \left(X, \, \mathcal{O}_X \right) \xrightarrow{}  0, \end{equation*}
and it induces an equality on the dimensions of the cohomology groups, i.e.,
\begin{equation*} p_a(X) + p_g(X) = 2 \, g. \end{equation*}