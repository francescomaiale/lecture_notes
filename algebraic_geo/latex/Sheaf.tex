\chapter{Sheaf Theory} \thispagestyle{empty}

In this chapter, we want to introduce and develop the sheaf language to simplify the comprehension of the next topics.

In particular, we present the cohomology groups (which will ease the discussion of the divisor vector space $L(D)$), and we also prove the long exact sequence in cohomology theorem (which will be used extensively in the next chapters.)

The reader may consult \href{https://www2.bc.edu/dawei-chen/Lecture-2013.pdf}{these notes} for a more in-depth dissertation.

\section{Definitions and First Properties}

\begin{definition}[Sheaf]\index{Sheaf} Let $X$ be a topological space. A \textit{sheaf} $\F$ on $X$ associates to each open set $U \subset X$ an abelian group $\F(U)$, along with a \textit{restriction map} $\rho_{V}^{U} : \F(V) \to \F(U)$ for any open sets $U \subset V$, satisfying the following conditions: \mbox{}
\begin{enumerate}[label=\textbf{(\arabic*)}]
\item \textbf{Compatibility Conditions.} 
\begin{enumerate}[label=\textit{(1.\arabic*)}]
\item  $\F(\emptyset) = 0$.
\item  $\rho_{U}^{U} = \mathrm{id}_U$.
\item If $W \subset V \subset U$, then $\rho_{W}^{U} = \rho_{W}^{V} \circ \rho_{V}^{U}$.
\end{enumerate}
\item \textbf{Locality Conditions.} 
\begin{enumerate}[label=\textit{(2.\arabic*)}]
\item If $\mathcal{U} := \{U_i\}_i$ is a covering of $U$, then
\begin{equation*} \sigma \in \F(U) \: : \: \sigma \, \big|_{U_i} = 0 \quad \forall \, i \implies \sigma \equiv 0. \end{equation*}
\item For any covering $\mathcal{U} := \{U_i\}_{i \in I}$ of $U$ and any collection $\{\sigma_i \}_{i \in I}$ of sections $\sigma_i \in \F(U_i)$ it turns out that, if
\begin{equation*}\sigma_i \, \big|_{U_i \cap U_j} = \sigma_j \, \big|_{U_i \cap U_j}\end{equation*}
for any $i, \, j$, then there exists $\sigma \in \F(U)$ such that $\sigma \, \big|_{U_i} = \sigma_i$.
\end{enumerate}
\end{enumerate}
If $\F$ satisfies only the compatibility conditions, then we say that $\F$ is a \textit{presheaf}. \end{definition}

\begin{definition}[Stalks]\index{Sheaf!stalks} Let $\F$ be a sheaf on a topological space $X$, and let $p \in X$. The \textit{stalk} at $p$, denoted by $\F_p$, is the direct limit of all sections containing $p$. \end{definition}

More precisely, suppose that $U$ and $V$ are two open subsets, both containing $p$, with two section $\sigma_U$ and $\sigma_V$; define an equivalence relation
\begin{equation*}\sigma_U \sim \sigma_V \iff \exists \, W \subset U \cap V \: : \: \sigma_U \, \big|_{W} = \sigma_V \, \big|_{W}. \end{equation*}
Then the stalk at $p$ is defined by setting
\begin{equation*} \F_p := \varinjlim_{U \ni p} \F(U) = \left( \bigsqcup_{U \ni p} \F(U) \right) \, / \, \sim. \end{equation*}
There is a group homomorphism $\rho_U : \F(U) \to \F_p$ mapping a section $\sigma_\alpha$ to its equivalence class, and the image is called the \textit{germ} of $\sigma_\alpha$.

\begin{example} To get accustomed with the definitions, the reader may try to check that the following are all sheaves. \mbox{}
\begin{enumerate}[label=\textbf{(\alph*)}]
\item The (locally) constant sheaf $U \mapsto \underline{\C}(U) = \C$,  denoted by $\underline{\C}$, whose restriction maps are the identities between the copies of $\C$.
\item The sheaf of holomorphic functions $U \mapsto \mathcal{O}_X(U) := \left\{ f : U \to \C \: \left| \: \text{$f$ is holomorphic in $U$} \right. \right\}$. For any $p \in X$ the stalk is given by
\begin{equation*}\mathcal{O}_{X, \, p} := \left\{ f : X \to \C \: \left| \: \text{$f$ is meromorphic outside of $p$} \right. \right\}. \end{equation*}
\item The sheaf of meromorphic functions $U \mapsto \mathcal{M}_X(U) := \left\{ f : U \to \C \: \left| \: \text{$f$ is meromorphic in $U$} \right. \right\}$.
\item The sheaf of the holomorphic differentials $U \mapsto \Omega_X^1(U) := \left\{ f(z) \, \mathrm{d}z \: \left| \: \text{$f$ is holomorphic in $U$} \right. \right\}$.
\end{enumerate}
\end{example}

\paragraph{Morphisms of Sheaves.}\index{Sheaf!morphism} Let $\mathcal{E}$ and $\F$ be sheaves on $X$. A morphism $f : \mathcal{E} \to \F$ is a collection of group homomorphisms $f_U : \mathcal{E}(U) \to \F(U)$ such that the following diagram is commutative
\begin{equation*}\begin{tikzcd}
 \mathcal{E}(U) \arrow{rr}{f_U} \arrow{dd}{\rho_{U}^{V}} &&  \F(U) \arrow{dd}{\rho_{U}^{V}} \\ \\
 \mathcal{E}(V)  \arrow{rr}{f_V} & & \F(V)
\end{tikzcd}  \end{equation*}
that is, $\left(f_U(\sigma) \right) \, \big|_{V} = f_V \left( \sigma \, \big|_V \right)$.

\begin{example}[Inclusion Maps] The first kind of morphisms between sheaves we study are the inclusion maps. Indeed, they come up whenever, for any $U \subset X$, the group $\F(U)$ is a subgroup of the group $\G(U)$. \mbox{}
\begin{enumerate}[label=\textbf{(\arabic*)}]
\item \textit{Constant sheaves}: $\underline{\Z} \subset \underline{\R} \subset \underline{\C}$.
\item \textit{Holomorphic/Meromorphic sheaves}: $\underline{\C} \subset \mathcal{O}_X \subset \mathcal{M}_X$.
\item \textit{Nonzero holomorphic/meromorphic sheaves}: $\mathcal{O}_X^\ast \subset \mathcal{M}_X^\ast$.
\item \textit{Sheaves of functions with bounded poles}: if $D_1 \leq D_2$ are divisors on $X$, then $\mathcal{O}_X[D_1] \subset \mathcal{O}_X[D_2]$.
\end{enumerate}\end{example}

\paragraph{Kernel.}\index{Sheaf!kernel} Suppose that $\phi : \F \to \G$ is sheaf map between two group sheaves on $X$. Define a subsheaf $\K \subset \F$, called the \textit{kernel} of $\phi$, by setting
\begin{equation*} \K(U) := \mathrm{ker}(\phi_U) \end{equation*}
for any $U \subseteq X$, that is, the group associated to the open set $U$ is exactly the kernel of the group homomorphism $\phi_U : \F(U) \to \G(U)$.

\begin{proposition} Let $\phi : \F \to \G$ be a sheaf map between two group sheaves on $X$. \mbox{}
\begin{enumerate}[label=\textbf{(\alph*)}]
\item The kernel $\K$ is a sheaf.
\item The cokernel (which is defined in the same way) is a presheaf, but it is generally not a sheaf.\end{enumerate}\end{proposition}

\paragraph{Associated Sheaf.}\index{Sheaf!associated sheaf} If $\F$ is a presheaf, then it is always possible to extend it to a sheaf $\widetilde{F}$, which is usually called associated sheaf.

For example, we define $\mathrm{coker}(f) (U)$ to be a collection of sections $\sigma_\alpha \in \G(U_\alpha)$ for an open covering $\{U_\alpha\}_{\alpha}$ of $U$, such that for all $\alpha$ and $\beta$ it turns out that
\begin{equation*} \sigma_\alpha \, \big|_{U_\alpha \cap U_\beta} - \sigma_{\beta} \, \big|_{U_\alpha \cap U_\beta} \in f_{U_\alpha \cap U_\beta} \left( \mathcal{F}(U_\alpha \cap U_\beta) \right). \end{equation*}
The definition depends on the choice of an open covering. Thus we need to find a way to get rid of this obstacle.

The idea, as we shall see also later in the course, is to use the direct limit. More precisely, we identify two collections $\left\{ (U_\alpha, \, \sigma_\alpha) \right\}$ and $\left\{ (V_\beta, \, \sigma_\beta) \right\}$ if for all $p \in U_\alpha \cap V_\beta$ there exists an open set $W$ satisfying $p \in W \subset U_\alpha \cap V_\beta$, such that
\begin{equation*} \sigma_\alpha \, \big|_{W} - \sigma_{\beta} \, \big|_{W} \in f_{W} \left( \mathcal{F}(W) \right). \end{equation*}
This identification yields an equivalence relation and correspondingly we define the coker sheaf as the group of equivalence classes of the above sections.

\section{Exact Sequences}

In this section, we introduce a fundamental notion that is used vastly in algebra and geometry: short exact sequences, and the long exact sequence in cohomology theorem.

\paragraph{Short Exact Sequences of Sheaves.}\index{Sheaf!short exact sequence} We say that a sequence of sheaf maps
\begin{equation*} 0 \xrightarrow{} \K \xrightarrow{} \F \xrightarrow{\phi} \G \xrightarrow{} 0\end{equation*}
is a \textit{short exact sequence} if the sheaf map $\phi$ is surjective and $\K$ is the kernel sheaf associated to $\phi$.

\begin{remark}There is an equivalent - and maybe more useful - definition of a short exact sequence of sheaf maps, which relies on the notion of a short exact sequence of abelian groups. More precisely, the sequence of sheaf maps
\begin{equation*} 0 \xrightarrow{} \F \xrightarrow{\phi} \G \xrightarrow{\psi} \mathcal{H} \xrightarrow{} 0\end{equation*}
is an exact short sequence if and only if
\begin{equation*} 0 \xrightarrow{} \F_p \xrightarrow{\phi_p} \G_p \xrightarrow{\psi_p} \mathcal{H}_p \xrightarrow{} 0\end{equation*}
is an exact sequence of abelian groups, for every $p \in X$. \end{remark}

\begin{example} Here is a brief list of short exact sequences. \mbox{}
\begin{enumerate}[label=\textbf{(\alph*)}]
\item On a Riemann surface, the sequence
\begin{equation*} 0 \xrightarrow{} \underline{\C} \xrightarrow{} \mathcal{O}_X \xrightarrow{\mathrm{d} = \partial} \Omega_X^1 \xrightarrow{} 0\end{equation*}
is exact, since the kernel sheaf of the differential map is exactly the (locally) constant sheaf.
\item The sequence
\begin{equation*} 0 \xrightarrow{} \underline{\Z} \xrightarrow{} \mathcal{O}_X \xrightarrow{\mathrm{e}^{2 \pi \, i \, \cdot}} \mathcal{O}_X^\ast \xrightarrow{} 0\end{equation*}
is exact, since the kernel sheaf of the exponential map is exactly the integer-valued (locally) constant sheaf.
\item The sequence
\begin{equation*} 0 \xrightarrow{} \Omega_X^1 \xrightarrow{} \mathcal{E}_X^{1, \, 0} \xrightarrow{\bar{\partial}} \mathcal{E}_X^{2} \xrightarrow{} 0\end{equation*}
is exact.
\item The sequence
\begin{equation*} 0 \xrightarrow{} \underline{\C} \xrightarrow{} C^\infty \xrightarrow{d} \mathrm{ker} \left( d : \mathcal{E}_X^1 \to \mathcal{E}_X^2 \right) \xrightarrow{} 0\end{equation*}
is exact, since the kernel of $d$ in this setting is exactly the constant functions space.
\end{enumerate}
\end{example}

\paragraph{Sheaves on Riemann surfaces.}\index{Sheaf!on Riemann surfaces} Let $X$ be a compact Riemann surface and let $p \in X$. The \textit{skyscraper sheaf} centered at $p$ is defined as
\begin{equation*} \left( \C_p \right)_x = \begin{cases} 0 & \text{if $x \neq p$} \\ \C & \text{if $x = p$}. \end{cases} \end{equation*}
We can also define the sheaf of holomorphic functions such that $p$ is a zero, that is,
\begin{equation*} \mathcal{J}_{X, \, p} = \left\{ \text{$f$ is holomorphic and $f(p) = 0$}   \right\}, \end{equation*}
which can be easily denoted using the divisors. In fact, if we let $f \in \mathcal{J}_p$, then it is easy to prove that $\mathrm{div}(f) \geq p$, and thus we may denote it by
\begin{equation*} \mathcal{J}_{X, \, p} = \mathcal{O}_X[-p]. \end{equation*}

\begin{proposition} Let $X$ be a compact Riemann surface. There exists an exact sequence of sheaf maps\begin{equation*} 0 \xrightarrow{} \mathcal{O}_X[-p] \xrightarrow{} \mathcal{O}_X \xrightarrow{f \mapsto f(p)} \C_p \xrightarrow{} 0. \end{equation*} \end{proposition}

\begin{proof}[Idea of the Proof]Notice that, if $x \neq p$, then
\begin{equation*} (\C_p)_x = 0 \qquad \text{and} \qquad (\mathcal{O}_X[-p])_x \cong \mathcal{O}_{X, \, x} \end{equation*}
and that
\begin{equation*} (\mathcal{O}_X[-p])_p \cong \left\{ \text{maximal ideals in $\mathcal{O}_{X, \, p}$} \right\} \qquad \text{and} \qquad \faktor{\mathcal{O}_{X, \, p}}{\mathcal{M}_X }\cong \C_p. \end{equation*} \end{proof}

\section{Čech Cohomology of Sheaves}

In this section will assume that every covering is \textit{locally finite}. This assumption is by no means necessary at this point, but it will come in handy (and, actually, necessary) soon.

\paragraph{Čech Cochains.}\index{Cech!Cochains} Let $\F$ be a sheaf of abelian groups on a topological space $X$. Let $\mathcal{U} := \{U_i\}_i$ be an open covering of $X$, and fix an integer $n \geq 0$. For every collection of indices $(i_0, \, \dots, \, i_n)$, we denote the intersection of the corresponding open sets by
\begin{equation*} U_{i_0, \, \dots, \, i_n} := U_{i_0} \cap \dots \cap U_{i_n}. \end{equation*}
The deletion of the $k$-th index is denoted by the symbol $\hat{i_k}$, and it is clear by the definition that
\begin{equation*} U_{i_0, \, \dots, \, i_n} \subset  U_{i_0, \, \dots, \widehat{i_k}, \, \dots \, i_n}. \end{equation*}

\begin{definition} A \textit{Čech $n$-cochain} for the sheaf $\F$ over the open cover $\mathcal{U}$ is a collection of sections of $\F$, one over each $U_{i_0, \, \dots, \, i_n}$.
\end{definition}

The space of Čech $n$-cochains for $\F$ over $\mathcal{U}$ is denoted by $\widecheck{C}^n(\mathcal{U}, \, \F)$. In particular, a Čech $0$-cochain is simply a collection of sections, that is, one gives a section of $\F$ over each open set in the cover. Similarly, a $1$-cochain is a collection of sections of $\F$ over every intersection of two open sets of the cover; the typical notation for a $1$-cochain is $(f_{i, \, j}) \in \F  \left(U_i \cap U_j \right)$.

\begin{remark}If $\phi : \F \to \G$ is a sheaf map, then it induces map on the cochains space
\begin{equation*} \phi : \widecheck{C}^n(\U, \, \F) \to \widecheck{C}^n(\U, \, \G) \end{equation*}
for any open covering $\U$, defined by
\begin{equation*}(f_{i_0, \, \dots, \, i_n}) \longmapsto \left( \phi(f_{i_0, \, \dots, \, i_n}) \right). \end{equation*} \end{remark}

\paragraph{Čech Cochains Complexes.}\index{Cech!complexes} Define a co-boundary operator
\begin{equation*} d : \widecheck{C}^n(\U, \, \F) \to \widecheck{C}^{n+1}(\U, \, \F) \end{equation*}
by setting
\begin{equation*} d \left( (f_{i_0, \, \dots, \, i_n}) \right) := (g_{i_0, \, \dots, \, i_{n+1}}),\end{equation*}
where
\begin{equation*} g_{i_0, \, \dots, \, i_{n+1}} = \sum_{k = 0}^{n+1} (-1)^k \, \rho\left(f_{i_0, \, \dots, \, \widehat{i_k}, \, \dots, \, i_{n+1}} \right). \end{equation*}
In the above formula $\rho$ denotes the restriction map for the sheaf $\F$ corresponding to the inclusion $U_{i_0, \, \dots, \, i_n} \subset  U_{i_0, \, \dots, \widehat{i_k}, \, \dots \, i_n}$.

Any $n$-cochain $c$ with $d(c) = 0$ is called a \textit{$n$-cocycle}\index{Cech!cocycle}; the space of $n$-cocycles is denoted by $ \widecheck{Z}^n(\U, \, \F)$ and it is simply the kernel of $d$ at the $n$-th level.

Any $n$-cochain $c$ with $c = d(c^\prime)$ for some $(n-1)$-cochain $c^\prime$ is called a \textit{$n$-coboundary}\index{Cech!coboundary}; the space of $n$-coboundaries is denoted by $\widecheck{B}^n(\U, \, \F)$.

It is straightforward, but tedious, to prove that $d \circ d = 0$. Thus we have a \textit{Čech cochain complex}
\begin{equation*} 0 \xrightarrow{0} \widecheck{C}^0(\U, \, \F) \xrightarrow{d} \widecheck{C}^1(\U, \, \F) \xrightarrow{d} \widecheck{C}^2(\U, \, \F) \xrightarrow{d} \dots. \end{equation*}

\paragraph{Cohomology with respect to a Cover.}\index{Cech!cohomology} The fact that $d^2 = 0$ implies that for every $n \in \N$
\begin{equation*} \widecheck{B}^n(\U, \, \F) \subset \widecheck{Z}^n(\U, \, \F). \end{equation*}

\begin{definition}[Cohomology] The $n^{th}$ Čech cohomology group $\widecheck{H}^n(\U, \, \F)$ of $\F$ with respect to the open cover $\mathcal{U}$ is the quotient group
\begin{equation*} \widecheck{H}^n(\U, \, \F) = \faktor{\widecheck{Z}^n(\U, \, \F)}{\widecheck{B}^n(\U, \, \F)}. \end{equation*}\end{definition}

There is a lot of work behind the following definition (see \cite[pp. 295--297]{miranda}), but the main point is that we can define a Čech cohomology group \textbf{independent} of the cover $\mathcal{U}$. In fact, one introduces a refinement of $\mathcal{U}$ and proves that the cohomologies can be compared and they depend only on the particular coverings.

\begin{definition}[Čech Cohomology] Fix a sheaf $\F$ and a integer $n \geq 0$. The $n^{th}$ Čech cohomology group of $\F$ on $X$ is the group
\begin{equation*} H^n(X, \, \F) = \varinjlim_{\mathcal{U}} \widecheck{H}^n(\U, \, \F).\end{equation*}\end{definition}

\begin{proposition} Let $X$ be a complex manifold, paracompact and smooth. If $\underline{\R}$ is the constant sheaf on $X$, then
\begin{equation*} H^n(X, \, \underline{\R}) = H_{\mathrm{dr}}^n(X, \, \underline{\R}) = H_{\mathrm{sing}}^n(X, \, \underline{\R}).\end{equation*}\end{proposition}

\begin{remark} \label{rmk:sdksdkskdksd} There is an isomorphism
\begin{equation*} H^0(X, \, \F) \cong \F(X). \end{equation*}
In fact, the $0$-coboundary is given by $\{0\}$, while the $0$-cycle is given by
\begin{equation*} \widecheck{Z}^0(X, \, \F) = \left\{ \{f_\gamma\}_{\gamma \in \Gamma} \: \left| \: f_\alpha - f_\beta = 0\, \, \, \text{in $U_\alpha \cap U_\beta$ for any $\alpha, \, \beta \in \Gamma$} \right. \right\}.\end{equation*}
Therefore $f_\alpha = f_\beta$ in the intersection $U_\alpha \cap U_\beta$ easily implies that it is possible to extend both $f_\alpha$ and $f_\beta$ to a function $f$ in $X$. \end{remark}

In particular, if $X$ is a compact Riemann surface, then the $0$ cohomology group of the holomorphic sheaf is given by
\begin{equation*} H^0(X, \, \mathcal{O}_X) = \mathcal{O}_X(X) = \C, \end{equation*}
since a holomorphic function defined on the whole compact $X$ is bounded and thus constant.

\section{Sheaves of $\mathcal{O}_X$-modules}

\paragraph{Sheaves of $\mathcal{O}_X$-modules.} Let $X$ be a complex manifold (i.e., a manifold with a holomorphic structure).

\begin{definition}[Coherent]\index{Sheaf!coherent} A sheaf $\F$ of $\mathcal{O}_X$-modules is \textit{coherent} if and only if for any $p \in X$ there exist an open neighborhood $U \subset X$ and an exact sequence
\begin{equation*} \mathcal{O}_X^s(U) \xrightarrow{} \mathcal{O}_X^r(U) \xrightarrow{} \F(U) \xrightarrow{} 0, \end{equation*}
i.e, if and only if it is finitely presented. \end{definition}

\begin{example} Let $X$ be a Riemann surface. Then the holomorphic $1$-form sheaf $\Omega_X^1$ is invertible and the isomorphism is given by
\begin{equation*} \Omega_X^1(U) \ni f(z) \, \mathrm{d}z \longmapsto f(z) \in \mathcal{O}_X(U). \end{equation*}
\end{example}

\begin{example}If $\mathrm{dim}_\C X = n$, then
\begin{equation*} \mathcal{O}(X)^n \twoheadrightarrow \Omega_X^1(U), \qquad (f_1, \, \dots, \, f_n) \longmapsto f_1 \, \mathrm{d}z_1 + \dots + f_n \, \mathrm{d}z_n, \end{equation*}
where $z_1, \, \dots, \, z_n$ are local coordinates. Clearly the sheaf is coherent but it is not invertible.
\end{example}

\paragraph{Fundamental Properties.} In this paragraph we briefly discuss some of the most important properties of the particular class of sheaves we just introduced.

\begin{definition}[Support]\index{Sheaf!support} Let $\F$ be a sheaf on $X$. The \textit{support} of $\F$ is defined as the set of the nontrivial stalks, that is,
\begin{equation*} \mathrm{spt}\left(\F \right) = \left\{ x \in X \: \left| \: \F_x \neq 0 \right. \right\}. \end{equation*}
\end{definition}

\begin{proposition} \label{prop:sptttt} Let $X$ be a compact complex manifold (with holomorphic structure) and let $\F$ be a coherent sheaf over $X$. \mbox{}
\begin{enumerate}[label=\textbf{(\alph*)}]
\item The $p$-th cohomology group $H^p(X, \, \F)$ is a finite-dimensional $\C$-vector space.
\item The $p$-th cohomology group is zero for any $p > \mathrm{dim}_\C X$.
\end{enumerate} \end{proposition}

\begin{corollary} Let $X$ be a compact Riemann surface and let $\F$ be a coherent sheaf over $X$. Then the $p$-th cohomology group is given by
\begin{equation*} H^p(X, \, \F) \, \,  \begin{cases} = 0 & \text{if $p \geq 2$} \\ \neq 0 & \text{if $p = 0, \, 1$}. \end{cases} \end{equation*}
\end{corollary}

\begin{theorem} \index{Sheaf!long exact sequence}\label{longcomo} Let $X$ be a compact complex manifold (with holomorphic structure). If
\begin{equation*} 0 \xrightarrow{} \F \xrightarrow{\varphi}\G \xrightarrow{\psi} \h \xrightarrow{} 0 \end{equation*}
is an exact sequence of coherent $\mathcal{O}_X$-modules, then there is a long exact sequence in cohomology, that is, there exists $\partial$ such that
\begin{equation*} 0 \xrightarrow{} H^0(\F) \xrightarrow{H^0(\varphi)}H^0(\G) \xrightarrow{H^0(\psi)} H^0(\h) \xrightarrow{\partial} H^1(\F) \xrightarrow{} \dots \end{equation*}
is an exact long sequence.\end{theorem}

\begin{proof}[Idea of the Proof] We can always choose a covering $\U := \{U_\alpha\}_{\alpha \in I}$ such that, for any $U_\alpha$, it turns out that
\begin{equation*} 0 \xrightarrow{} \F(U_\alpha) \xrightarrow{\varphi_\alpha} \G(U_\alpha) \xrightarrow{\psi_\alpha} \h(U_\alpha) \xrightarrow{} 0 \end{equation*}
is a short exact sequence. Consequently, the sequence
\begin{equation*} 0 \xrightarrow{} H^0(\F) \xrightarrow{H^0(\varphi)}H^0(\G) \xrightarrow{H^0(\psi)} H^0(\h) \end{equation*}
is exact, but the latter map may not be surjective (in general, it will not be). We want to define a map $\partial : H^0(\h) \to H^1(\F)$ that makes the sequence exact at $H^0(\h)$.

Let $\sigma \in H^0(\h)$ and let us consider two open sets of the covering, e.g., $U_\alpha$ and $U_\beta$. We already know that there exists $s_\alpha \in \G(U_\alpha)$ such that $s_\alpha \mapsto \sigma \, \big|_{U_\alpha}$ and there exists $s_\beta \in \G(U_\beta)$ such that $s_\beta \mapsto \sigma \, \big|_{U_\beta}$. Let us set $\tilde{\delta} (\sigma) := s_\alpha - s_\beta = g_{\alpha, \, \beta}$.

Clearly $g_{\alpha, \, \beta} \in Z^1 \left( U_\alpha \cap U_\beta \right) \subseteq \G\left( U_\alpha \cap U_\beta \right)$ and by the exactness of the local sequence, it turns out that $\psi(g_{\alpha, \, \beta}) = 0$ in $\h \left( U_\alpha \cap U_\beta \right)$ and therefore there exists $f_{\alpha, \, \beta} \in \F \left( U_\alpha \cap U_\beta \right) $ such that
\begin{equation*} \varphi( f_{\alpha, \, \beta} ) = g_{\alpha, \, \beta}. \end{equation*}
We can set
\begin{equation*} \partial (\sigma) := \left\{ f_{\alpha, \, \beta} \right\}_{\alpha, \, \beta} \in H^1(\F) \end{equation*}
and, by using the induction principle, we conclude the proof.
\end{proof}

\begin{corollary} Let $X$, $\F$, $\G$ and $\h$ be as above. Then
\begin{equation*} \chi(\G) = \chi(\F) + \chi(\h), \qquad \text{where} \quad \chi(\F) := \sum_{i = 0}^{\mathrm{dim}_\C\left(\mathrm{spt}(\F) \right)} (-1)^i \, \mathrm{dim}_\C \, \left(H^i(\F) \right). \end{equation*} \end{corollary}

\section{GAGA Principle}
\index{GAGA Principle}

Let $X$ be a projective manifold, equipped with the Zariski topology, and let
\begin{equation*}\mathcal{O}_X^{\mathrm{alg}} := \left\{ f : X \to \C \: \left| \: \text{$f$ algebraic} \right. \right\}. \end{equation*}
On the other side, let $X$ be a compact holomorphic manifold, equipped with the Hausdorff topology, and let
\begin{equation*}\mathcal{O}_X^{\mathrm{h}} := \left\{ f : X \to \C \: \left| \: \text{$f$ holomorphic} \right. \right\}. \end{equation*}

\begin{theorem} Let $X$ be a smooth projective manifold. Then there exists an application
\begin{equation*} \F \longmapsto \F^h := \F \bigotimes_{\mathcal{O}_X^{\mathrm{alg}}} \mathcal{O}_X^{\mathrm{h}}, \end{equation*}
where $\F$ is a coherent sheaf of $\mathcal{O}_X^{\mathrm{alg}}$-modules and $\F^h$ a coherent sheaf of $\mathcal{O}_X^{\mathrm{h}}$-modules, such that
\begin{equation*} H_{\mathrm{Zar}}^i(X, \, \F) \cong H_{\mathrm{Hau}}^i(X, \, \F^h). \end{equation*} \end{theorem}


\section{Invertible $\mathcal{O}_X$-module Sheaves}

\begin{definition}[Invertible]\index{Sheaf!invertible} A sheaf $\mathcal{L}$ of $\mathcal{O}_X$-modules is \textit{invertible} if and only if there exists a covering $\mathcal{U} := \{ U_i \}_{i \in I}$ of $X$ such that: \mbox{}
\begin{enumerate}[label=\textbf{(\alph*)}]
\item For any $i \in I$, there is an isomorphism $\phi_i : \mathcal{L}(U_i) \xrightarrow{\sim} \mathcal{O}_X(U_i)$.
\item For any $i, \, j \in I$, there is an invertible function $f_{i, \, j}$, defined on $U_i \cap U_j$, such that $\phi_i = f_{i, \, j} \cdot \phi_j$.
\end{enumerate} \end{definition}

\begin{remark} Let $\mathcal{L}$ be an invertible sheaf of $\mathcal{O}_X$-modules and let $U_i, \, U_j, \, U_k \in \mathcal{U}$. It follows from the definition that, in the triple intersection, the following relation holds true:
\begin{equation*} f_{i, \, k} = f_{i, \, j} \cdot f_{j, \, k}. \end{equation*} \end{remark}

\begin{remark} Equivalently, a sheaf $\F$ of $\mathcal{O}_X$-modules is \textit{invertible} if and only if for every $p \in X$ there is an open neighborhood $U$ of $p$, such that $\mathcal{O} \, \big|_U \cong \F \, \big|_U$ as sheaves of $\mathcal{O} \, \big|_U$-modules on the space $U$.

The invertible sheaves are locally free rank one $\mathcal{O}$-modules. An isomorphism $\phi_U : \mathcal{O} \, \big|_U \to \F \, \big|_U$ is called a \textit{trivialization} of $\F$ over $U$. \end{remark}

\begin{remark}By definition, if $U$ is an open neighborhood of $p$ such that $\mathcal{O} \, \big|_U \cong \F \, \big|_U$, then there is an isomorphism $\phi_U : \mathcal{O}(U) \to \F(U)$.\end{remark}

Let us consider the complex projective line $\p^1(\C)$ endowed with the usual atlas given by $U_0 = \left\{ [1 : z_1] \right\}$, with local coordinate $z := z_1/z_0$, and $U_1 = \left\{ [z_0 : 1] \right\}$, with local coordinate $w := z_0/z_1$.

\begin{example}\label{ex:p23psadapdppewpw}For any fixed $m \in \Z$, we consider the invertible sheaf $\mathcal{O}_{\p^1}[m]$, which is defined as follows: \begin{enumerate}[label=\textbf{(\alph*)}]
\item for $i = 1$ and $i=2$, there are isomorphisms $\mathcal{O}_{\p^1}[m](U_i) \cong \mathcal{O}(U_i) \cong \mathcal{O}(\C)$;
\item the transition map is given by
\begin{equation*} f_{1, \, 0} = \left( \frac{z_1}{z_0} \right)^m. \end{equation*}
\end{enumerate}
We are interested in computing the $0$-cohomology group of $\mathcal{O}_{\p^1}[m]$ for $m = 0$, $m = 1$ and $m \in \Z$.
\begin{enumerate}[label=\textbf{(\arabic*)}]
\item If $m = 0$, then it is straightforward to prove that
\begin{equation*} \mathcal{O}_{\p^1}[0] = \mathcal{O}_{\p^1} \qquad \text{and} \qquad H^0 \left( \p^1, \, \mathcal{O}_{\p^1} \right) = \C. \end{equation*}
\item Let $m = 1$. By definition of the sheaf $\mathcal{O}_{\p^1}[1]$, it turns out that
\begin{equation*} \begin{aligned} & \exists \, f(z) \in \mathcal{O}_{\p^1}(U_0) \quad \text{such that $f$ is holomorphic in $U_0$,} \\[1em] & \exists \, g(z) \in \mathcal{O}_{\p^1}(U_1) \quad \text{such that $g$ is holomorphic in $U_1$}. \end{aligned} \end{equation*}
To compute the group $H^0 \left( \p^1, \, \mathcal{O}_{\p^1} \right)$ we need to check how $f$ and $g$ glue in the intersection $U_0 \cap U_1$. The assumption \textbf{(b)} on the transition map easily implies that
\begin{equation*} f(z) = z \, g \left( \frac{1}{z} \right), \qquad \forall \, z \in U_0 \cap U_1. \end{equation*}
If we use the Laurent develop, it turns out that
\begin{equation*} f(z) = \sum_{i \geq 0} a_i \, z^i \qquad \text{and} \qquad g\left(z\right) = \sum_{i \geq 0} b_i \, z^{-i}, \end{equation*}
hence $f(z) = z \cdot g(1/z)$ in the intersection if and only if
\begin{equation*} f(z) = a_0 + a_1 \, z \qquad \text{and} \qquad g(z) = \frac{b_{-1}}{z} + b_0. \end{equation*}
Therefore we can easily conclude that
\begin{equation*}H^0 \left( \p^1, \, \mathcal{O}_{\p^1}[1] \right) \cong \left\{ \begin{gathered} \text{homogeneous polynomials} \\ \text{of degree $1$ in $z_0$ and $z_1$} \end{gathered} \right\}.  \end{equation*}
\item Let $m > 0$ be any positive integer. There is an isomorphism
\begin{equation*}\left\{ p(z_0, \, z_1) \: \left| \: \begin{gathered} \text{homogeneous polynomials} \\ \text{of degree $m$ in $z_0$ and $z_1$} \end{gathered} \right. \right\} \cong H^0 \left( \p^1, \, \mathcal{O}_{\p^1}[m] \right).  \end{equation*}
which is defined by
\begin{equation*} p(z_0, \, z_1) \longmapsto \begin{cases} \frac{p(z_0, \, z_1)}{z_0^m} & \text{in $U_0$} \\[0.8em] \frac{p(z_0, \, z_1)}{z_1^m} & \text{in $U_1$}. \end{cases} \end{equation*}
The same result holds true for $m < 0$, but there are no polynomials of degree less than zero. In particular, the $H^0$ cohomology group is the trivial one, that is,
\begin{equation*} H^0 \left( \p^1, \, \mathcal{O}_{\p^1}[m] \right) = 0, \qquad \forall \, m \in \Z^{-}.  \end{equation*}
\end{enumerate}
\end{example}

\section{Operations on Sheaves}

\paragraph{Tensor Product.}\index{Sheaf!tensor product} Let $\mathcal{L}$ and $\F$ be invertible sheaves of $\mathcal{O}_X$-modules. Their tensor product is the sheaf denoted by $\mathcal{L} \otimes \F$ which is defined, in terms of co-cycles, as follows:
\begin{equation*} \left( \mathcal{L}, \, \F \right) \ni \left(\ell_{i, \, j}, \, f_{i, \, j} \right) \longmapsto \ell_{i, \, j} \cdot f_{i, \, j} \in \mathcal{L} \otimes \F. \end{equation*}
For example, if $m, \, k \in \Z$, it is relatively easy to prove that the following isomorphism exists:
\begin{equation*} \mathcal{O}_{\p^1}[m] \otimes \mathcal{O}_{\p^1}[k] \cong \mathcal{O}_{\p^1}[m + k]. \end{equation*}

\paragraph{Inverse.}\index{Sheaf!inverse} Let $\mathcal{L}$ be an invertible sheaf. The inverse is denoted by $\mathcal{L}^{-1}$ and it is the unique sheaf such that
\begin{equation*} \mathcal{L} \otimes \mathcal{L}^{-1} = \mathcal{O}_X. \end{equation*}
In particular if $X$ is a Riemann surface or, more generally, a complete holomorphic manifold
\begin{equation*} \left( \left\{ \text{invertible sheaves} \right\}, \, \bigotimes \right) \quad \text{is a group}. \end{equation*}
For a more precise formulation of the above argument, the reader may jump directly to \hyperref[sec:pic]{Section \ref{sec:pic}}.

\paragraph{Line Bundle.}\index{Sheaf!line bundles equivalence} There is a correspondence between invertible sheaf on a smooth manifold $X$ and line bundles, that is,
\begin{equation*}\left\{ \begin{gathered} \text{Invertible sheaves} \\ \text{on $X$ smooth} \end{gathered}  \right\} \stackrel{\sim}{\longleftrightarrow} \left\{ \text{Line bundles} \right\}.  \end{equation*}
If $X$ is an invertible Riemann surface, then
\begin{equation*} \left\{ \text{$F \to X$ Line bundle} \right\}  \stackrel{\sim}{\longleftrightarrow} \begin{gathered} \text{$\exists \, \mathcal{U} := \{U_i\}_{i \in I}$ covering such that} \\ \text{$F_i : F \, \big|_{U_i} \to U_i \times \C$ which sends $z$ to $(z, \, f_i(z))$} \\ \text{and $f_i = g_{i, \, j} \cdot f_j$ in the intersection $U_i \cap U_j$}. \end{gathered} \end{equation*}
A holomorphic section of $F$ is simply the holomorphic mapping $f_i : U_i \to \C$ such that
\begin{equation*} f_i = g_{i, \, j} \cdot f_j \qquad \text{in $U_i \cap U_j$}. \end{equation*}
In particular, there is a correspondence
\begin{equation*}\text{$\F$ sheaf of the sections of $F$} \stackrel{\sim}{\longleftrightarrow} \text{$F \to X$ Line bundles}, \end{equation*}
given by
\begin{equation*} \text{$f_{i}$ (and $g_{i, \, j}$)} \longmapsto F \to U_i \times \C \: : \: z \mapsto \left(z, \, f_i(z) \right). \end{equation*}

\paragraph{Algebraic Curves.}\index{Sheaf!and algebraic curves} Let $\F$ be an invertible sheaf. Clearly it corresponds locally to $\phi_i = z^{n_i}$ which is $0$ if and only if $z = 0$ is a point of multiplicity $n_i$ (an analogous argument works for $\infty$).

If $X \subset \p^2(\C)$ is a smooth algebraic curve of degree $d$, then we can always consider the vector subspace made up of fixed degree smooth algebraic curves, that is,
\begin{equation*} W := \left\{ \text{smooth algebraic curves of fixed degree $h$} \right\} \subset \p^2(\C). \end{equation*}
By Bezout's theorem\footnote{Let $X$ and $Y$ be plane algebraic curves of degree $n$ and $m$ respectively. Then there are $m \cdot n$ points in the intersection $X \cap Y$, counted with the respective multiplicities, provided that $X$ and $Y$ have no common components.}, it follows that for any $Y \in W$, the intersection $Y \cap X$ is given by $h \cdot d$ points with the right multiplicities (this assertion is not very precise, but we only want to give a rough idea of this construction).

Let $Y_1, \, Y_2 \in W$ and consider the points $p_i \in X \cap Y_1$ and $q_i \in Y_2$; since $W$ is a vector subspace, also the linear combinations of these two elements will individuate $Y_3 \in W$, such that the points $r_i \in X \cap Y_3$ are linear combinations of the previous points (see the \hyperref[fig:ksad2]{Figure \ref{fig:ksad2}}).

\begin{figure}[t]
\centering
\includegraphics[width = 15cm, height = 9cm]{images/GC12.png}
\caption{Idea of the construction}
\label{fig:ksad2}
\end{figure} 