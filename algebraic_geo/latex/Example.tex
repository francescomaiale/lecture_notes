\chapter{More Examples of Riemann Surfaces} \thispagestyle{empty}

In the first part of this chapter, we show a simple application of the Hurwitz's formula: we compute the genus for a smooth algebraic curve $X \subset \p^2(\C)$, finally proving what we have mentioned several times, that is,
\begin{equation*}g(X) = \frac{(d - 1)(d -2)}{2}. \end{equation*}
Next, we study the group of automorphisms for compact Riemann surfaces of genus $0$ and $1$; in the final part, we investigate the action of finite groups $\mathcal{G}$, and we prove an estimate on $|\mathcal{G}|$ which follows from the Hurwitz's formula.

\section{Application of Hurwitz's Formula}

\paragraph{Genus of Algebraic Curves.} Let $X = \{ f(z_1, \, z_2, \, z_0) = 0 \} \subset \p^2(\C)$ be an algebraic curve, defined by a homogeneous polynomial $f$ of degree equal to $d$.

Assume that $X$ is smooth (so that $X$ is a Riemann surface, by Dini's theorem\footnote{See \hyperref[implfunc]{Theorem \ref{implfunc}}.}) and assume also that, up to a change of coordinates, the following properties are satisfied: \mbox{}
\begin{enumerate}[label=\textbf{(\alph*)}]
\item $p = [1 : 0 : 0] \notin X$;
\item $\{z_2 \neq 0\}$ is not tangent at any point of $X$.
\end{enumerate}
Let us consider the projective line $L = \{ z_0 = 0 \} \cong \p^1(\C)$, and let us denote by $\pi : X \to L$ the associated projection. As usual, we can work in the chart $U_0 = \{ z_0 \neq 0\}$ with coordinates $z = z_1/z_0$ and $w = z_2/z_0$ so that
\begin{equation*}\begin{tikzcd}[contains/.style = {draw=none,"\in" description,sloped}]
 & X \cap U_0 \ar[r, "f"] & L \cap U_0 \\
  & (z, \, w)  \ar[u,contains] \ar[r,mapsto] & z \ar[u,contains]  \\
\end{tikzcd} \end{equation*}
First, we observe that by assumption \textbf{(a)}, the degree of the map associated to $f$ is exactly equal to $d$ on the intersection $X \cap \{ z_0 \neq 0 \}$.

On the other hand, at infinity there are no ramifications, and thus $R = R \, \big|_{U_0}$. More precisely, the ramification points of $X \cap U_0$ may be explicitly found as a solution of the system
\begin{equation*} \begin{cases} \widetilde{f}(z, \, w) = 0 \\[0.8em] \frac{\partial \, \widetilde{f}}{\partial \, w}(z, \, w) = 0, \end{cases} \end{equation*}
or, equivalently, of the system
\begin{equation*} \begin{cases} f(z_1, \, z_2, \, z_0) = 0 \\[0.8em] \frac{\partial \, f}{\partial \, z_0}(z_1, \, z_2, \, z_0) = 0. \end{cases} \end{equation*}
Hence by Bezout theorem the sum of the multiplicities of the ramification points is equal to the product of the degrees, that is,
\begin{equation*}R = d \cdot (d-1). \end{equation*}
Finally, from the Hurwitz's formula \eqref{eq:hur}, it turns out that
\begin{equation*} 2 \, g(X) - 2 = -2 \, d + R \implies g(X) = \frac{(d - 1)(d-2)}{2},\end{equation*}
as we suggested many times in the previous chapters.

The reader may consult \cite[pp. 144-145]{miranda} for a different approach to the problem, which still results in a simple application of the Hurwitz's formula.

\section{Automorphism of Riemann Surfaces}
\label{sec:42}
\index{Automorphisms!of genus $0$ surfaces}

\paragraph{Genus $0$ Automorphisms.} Let $X$ be a compact Riemann surface of genus zero. If $f : \p^1(\C) \to \p^1(\C)$ is an automorphism, then its degree is necessarily $\mathrm{deg} \, f = 1$, and thus
\begin{equation*} f(z_0, \, z_1) = (a \, z_1 + b \, z_0, \, c \, z_1+d \,z_0) .\end{equation*}
A necessary condition for $f$ to be an automorphism is that the two linear polynomials have no zero in common, that is,
\begin{equation*} \mathrm{det} \, \begin{pmatrix} a & b \\ c & d \end{pmatrix} = ad - bc \neq 0. \end{equation*}
In the affine setting (e.g. in $U_0 = \{z_0 \neq 0\}$), it turns out that, with respect to the coordinate $z = z_1/z_0$, the mapping is given by
\begin{equation*} \bar{f}: \C \to \C, \qquad z \longmapsto\frac{a\,z + b}{c \, z + d}.\end{equation*}
In particular, the group of automorphism of $\p^1(\C)$ can be completely characterized as
\begin{equation*}\p \, \mathrm{GL}(2; \; \C) = \faktor{\mathrm{GL}(2; \; \C)}{\C^\ast} \cong \mathrm{Aut} \left( \p^1(\C) \right),\end{equation*}
and the isomorphism is explicitly given by
\begin{equation*} \faktor{\mathrm{GL}(2; \; \C)}{\C^\ast} \ni A = \begin{pmatrix} a & b \\ c & d \end{pmatrix} \longmapsto  f(z_0, \, z_1) = (a \, z_1 + b \, z_0, \, c \, z_1+d \,z_0) \in \mathrm{Aut}\left( \p^1(\C) \right).\end{equation*}

\index{Automorphisms!of genus $1$ surfaces}
\paragraph{Genus $1$ Automorphisms.} In this paragraph, we characterize the holomorphic mappings $f$ between complex tori and give a criterion to decide if $f$ is an isomorphism or not.

\begin{proposition} \mbox{}
\begin{enumerate}[label=\textbf{(\arabic*)}]
\item Let $X$ be a compact Riemann surface of genus $g(X) = 1$.

Then $X$ is isomorphic to a complex torus $\faktor{\C}{\Lambda}$, with $\Lambda = \Z \, \omega_1 + \Z \, \omega_2$ lattice generated by $\R$-linearly independent elements $\omega_1$ and $\omega_2$.
\item Let $\omega_1, \, \omega_2 \in \C$ be two elements such that
\begin{equation*} \tau := \frac{\omega_1}{\omega_2} \in \C \setminus \R, \end{equation*}
and denote by $\Lambda$ the associated lattice. Then the quotient $\faktor{\C}{\Lambda}$ is a compact Riemann surface of genus one.
\end{enumerate}
\end{proposition}

We are now ready to prove the main theorem about holomorphic maps between complex tori. In particular, by the end of the section, we shall show that there are non-isomorphic complex tori (and hence non-isomorphic Riemann surfaces of genus one).

\begin{theorem}[\cite{miranda}] \label{theorem:124}Let $X = \faktor{\C}{\Lambda}$ and let $Y = \faktor{\C}{\Gamma}$ be compact Riemann surfaces of genus $1$. A holomorphic map $f : X \to Y$ is induced by a function $G : \C \to \C$ defined by $G : z \mapsto \gamma \, z + \alpha$, where $\alpha$ and $\gamma$ are fixed complex numbers. Moreover, the following properties hold true: \mbox{}
\begin{enumerate}[label=\textbf{(\alph*)}]
\item If $0 \mapsto 0$, then $\alpha = 0$ and $f$ is a group homomorphism.
\item The mapping $f$ is an isomorphism if and only if $\gamma \cdot \Lambda = \Gamma$.
\end{enumerate}
\end{theorem}
 
\begin{proof} By composing $f$ with a suitable translation on $Y$ we may always assume that $f(0) = 0$.

\paragraph{Step 1.} Since $g(X) = g(Y) = 1$, the Hurwitz's formula \eqref{eq:hur} proves that $f$ is an unramified map. In particular, it is a topological covering, and hence so is the composition $f \circ \pi_X : \C \to Y$.

Since the domain is simply connected, this must be isomorphic - as a covering - to the universal covering $p : \C \to Y$. Therefore there is a map $G : \C \to \C$ and a commutative diagram
\begin{equation*}\begin{tikzcd}
\C \arrow{rr}{G} \arrow{dd}{\pi_X} && \C \arrow{dd}{\pi_Y} \\
\\
X \arrow{rr}{f} && Y
\end{tikzcd} \end{equation*}

\paragraph{Step 2.} The map $G$ is induced on the universal coverings by the commutativity of the diagram above and sends $0$ to a lattice point; we may assume in fact that $G(0) = 0$, since composing with translation by a lattice point does not affect the projection map $\pi$. Moreover $f$ is a well-defined map of quotients, thus
\begin{equation*} f \left( X \right) \subseteq Y \implies G(\Lambda) \subseteq \Gamma, \end{equation*}
that is,
\begin{equation*} G(z + \ell) \equiv_{\Gamma} G(z), \qquad \forall \, \ell \in \Lambda. \end{equation*}
Therefore there exists a lattice point $\omega(z, \, \ell) \in \Gamma$ such that
\begin{equation*} \omega(z, \, \ell) = G(z + \ell)- G(z) \in \Gamma. \end{equation*}
But $\Gamma$ is a discrete subset and $\C$ is connected, thus we infer that $\omega(z, \, \ell)$ is locally constant in the variable $z$. In particular, it turns out that
\begin{equation*} \partial_z \, \left[G(z + \ell)- G(z) \right] = 0, \qquad \forall \, \ell \in \Lambda, \end{equation*}
and thus $G^\prime$ is invariant for $\Lambda$ (i.e., up to translations for elements of the lattice). As a consequence, $G^\prime$ is uniquely determined by its values on a fundamental parallelogram $P_{\Lambda}$.

\paragraph{Step 3.} Therefore $G^\prime : \C \to \C$ is a holomorphic function, whose value is determined by $G^\prime \, \big|_{P_{\Lambda}}$, and thus it is bounded (since $P$ is compact). By Liouville's Theorem\footnote{See \hyperref[Liov]{Corollary \ref{Liov}}.}, there exists $\gamma \in \C$ such that $G^\prime(z) = \gamma$ and this concludes the proof of the point \textbf{(a)}.

\paragraph{Step 4.} Finally, if $\gamma \cdot \Lambda = \Gamma$, then $\gamma^{-1} \cdot \Gamma = \Lambda$, and so the map $H(z) = \gamma^{-1} \left(z - \alpha \right)$ induces a holomorphic map from $Y$ to $X$ which is an inverse for $G$.\end{proof}

\begin{remark} The degree of $f$ is also given by the index of $\gamma \cdot \Lambda$ in $\Gamma$, that is,
\begin{equation*} \mathrm{deg}\, f = \left| \faktor{\Gamma}{\gamma \Lambda} \right|. \end{equation*}
In particular, if $f$ is an isomorphism, its degree is equal to $1$ and $\gamma \cdot \Lambda = \Gamma$. \end{remark}

\begin{proposition}Let $X = \faktor{\C}{\Lambda}$ be a compact Riemann surface of genus $1$. The holomorphic map
\begin{equation*} f : X \to X, \qquad z \longmapsto \gamma \cdot z \end{equation*}
is an automorphism of $X$ - sending $0$ to $0$ - if and only if either
\begin{enumerate}[label=\textbf{(\arabic*)}]
\item $\Lambda$ is a squared lattice, and $\gamma$ is a $4^{th}$-root of unity; or
\item $\Lambda$ is a hexagonal lattice, and $\gamma$ is a $6^{th}$-root of unity; or
\item $\Lambda$ is neither squared or hexagonal, and $\gamma = \pm 1$.
\end{enumerate}
\end{proposition}

\begin{proof} By \hyperref[theorem:124]{Theorem \ref{theorem:124}} a necessary condition for $f : X \to X$ to be an automorphism (sending $0$ to $0$) it that $\gamma \cdot \Lambda = \Lambda$, and thus $\|\gamma\| = 1$. If $\gamma = \pm 1$ there is nothing else to prove.

Assume that $\gamma \notin \R$ and let $\ell \in \Lambda \setminus \{0\}$ be an element of minimal length. Then so is $\gamma \cdot \ell$ and it belongs to $\Lambda$. Clearly $\gamma \cdot (\gamma \ell) \in \Lambda$, thus there exists $m, \, n \in \Z$ such that
\begin{equation*}\gamma^2 \, \ell = n \, \ell + m \, \gamma \ell, \end{equation*}
therefore $\gamma$ is a root (of norm equal to $1$) of the polynomial
\begin{equation*}p(\lambda) = \lambda^2 - m \, \lambda + n, \qquad m, \, n \in \Z. \end{equation*}
The proof is now complete since
\begin{equation*} \begin{aligned}&p(\lambda) = \lambda^2 \pm 1 \leadsto \text{$\Lambda$ is a square}, \\[1em] &
p(\lambda) = \lambda^2  \pm \lambda \pm 1 \leadsto \text{$\Lambda$ is a hexagonal}. \end{aligned}\end{equation*}
\end{proof}

\begin{corollary} Let $X$ be a compact Riemann surface of genus $1$. Then $X \cong \faktor{\C}{\Lambda}$, where $\Lambda = \langle 1, \, \tau \rangle$ and $\tau = \xi + \imath \, \eta$ is a complex number such that $\eta > 0$. \end{corollary}

\begin{proof}If $\widetilde{\Lambda} =  \left< \omega_1, \, \omega_2 \right>$, then one can consider the isomorphism of lattices given by
\begin{equation*}G : \widetilde{\Lambda} \longrightarrow \Lambda, \qquad \omega_1 \mapsto 1, \quad \omega_2 \mapsto \frac{\omega_2}{\omega_1}. \end{equation*}
If the imaginary part is not positive, then we may consider the isomorphism of lattices given by
\begin{equation*}H : \widetilde{\Lambda} \longrightarrow \Lambda, \qquad \omega_1 \mapsto \frac{\omega_1}{\omega_2} \quad \omega_2 \mapsto 1.\end{equation*}\end{proof}

\begin{corollary} Let $\Lambda = \left< 1, \, \tau \right>$ and let $\Lambda^\prime = \left< 1, \, \tau^\prime \right>$ be two lattices defined over $\C$. Set $X = \faktor{\C}{\Lambda}$ and $X^\prime = \faktor{\C}{\Lambda^\prime}$. Then $X \cong X^\prime$ if and only if there exists
\begin{equation*} \begin{pmatrix} a & b \\ c & d \end{pmatrix} \in \mathrm{SL}(2, \, \Z), \end{equation*}
such that
 \begin{equation*}\tau = \frac{a + b \, \tau^\prime}{c + d \, \tau^\prime}. \end{equation*}\end{corollary}
 
\begin{proof}By \hyperref[theorem:124]{Theorem \ref{theorem:124}} a sufficient condition for $X$ to be isomorphic to $X^\prime$ is the existence of a complex number $\gamma \in \C$ such that $\gamma \cdot \Lambda = \Lambda^\prime$. 

Equivalently, we only need to prove that there is $\gamma$ such that $\left< \gamma, \,  \gamma \, \tau \right>$ generates the lattice $\Lambda^\prime$. The inclusion $\subseteq$ is satisfied if there are integers $a, \, b, \, c, \, d$ such that 
\begin{equation*} \gamma = c \, \tau^\prime + d, \qquad \gamma \, \tau = a \, \tau^\prime + b. \end{equation*}
Eliminating $\gamma$ from these equations gives a relation between $\tau$ and $\tau^\prime$, that is,
\begin{equation*}\tau = \frac{a + b \, \tau^\prime}{c + d \, \tau^\prime}. \end{equation*}
Finally, for $\gamma$ and $\gamma \, \tau$ to generate $\Lambda^\prime$, the determinant of the matrix (i.e., $ad - bc$) must be equal to $\pm 1$. But it is easy to see that it is exactly equal to $1$, since both $\tau$ and $\tau^\prime$ lie in the upper half-plane.\end{proof}
 
\paragraph{Conclusion.} We proved that the group of automorphisms of a compact Riemann surface $X$ fixing $0$, denoted by $\mathrm{Aut}_0(X)$, is isomorphic to
\begin{equation*} \begin{aligned}& \mathrm{Aut}_0(X) \cong \faktor{\Z}{4} & \text{if $\Lambda$ is square}; \\[1em] & \mathrm{Aut}_0(X) \cong \faktor{\Z}{6} & \text{if $\Lambda$ is hexagonal}; \\[1em]
& \mathrm{Aut}_0(X) \cong \faktor{\Z}{2} & \text{otherwise}. \end{aligned}\end{equation*}

This simple result yields to a surprising fact: the complex torus defined using a square lattice is not isomorphic to a complex torus defined using a hexagonal lattice.

Thus there are non-isomorphic complex tori, i.e., if $g \geq 1$ then there exist surfaces of the same genus which are not isomorphic (this does not happen for $g = 0$ since the only surface of genus zero is the projective space).

\section{Group Actions on Riemann Surfaces}

\paragraph{Finite Group Actions.} In this section, $\G$ will denote a finite group and $X$ a Riemann surface. In the last paragraph, we will briefly talk about the case of $\G$ infinite.

\begin{definition}[Action]\index{Group action} An \textit{action} of a group $\G$ on a set $X$ is a map $\mu : \G \times X \to X$, denoted by $\mu(g, \, p) := g \cdot p$, which satisfies the following properties:
\begin{enumerate}[label=\textbf{(\arabic*)}]
\item $(g \, h) \cdot p = g \cdot (h \cdot p)$ for any $g, \, h \in \G$ and $p \in X$;
\item $e \cdot p = p$ for $p \in X$, where $e \in \G$ is the identity.
\end{enumerate} \end{definition}

\vspace{2mm}
The reader who is already familiar with the basic definitions may skip this paragraph. The \textit{orbit} of a point $p \in X$ is the set $\G \cdot p := \left\{ g \cdot p \: \left| \: g \in \G \right. \right\}$. If $A \subset X$ is any subset, we denote by $\G \cdot A$ the set of orbits of points in $A$, that is,
\begin{equation*}\G \cdot A = \bigcup_{p \in A} \G \cdot p. \end{equation*}
The \textit{stabilizer} of a point $p \in X$ is the set of the elements of the group $\G$ not moving $p$, i.e., $\G_p = \left\{ g \in \G \: \left| \: g \cdot p = p \right. \right\}$.

\begin{theorem}[Class Formula]\index{Group action!class formula} Let $\G$ be a finite group acting on a set $X$. For any $p \in X$ it turns out that
\begin{equation} \left| \G \cdot p \right| \cdot \left| \G_p \right| = \left| \G \right|.\label{classformula}\end{equation}
\end{theorem}

\begin{definition}[Effective Action]\index{Group action!effective} Let $\G$ be a finite group acting on a set $X$. The action is said to be \textit{effective} if the associated kernel is trivial. \end{definition}

More precisely, the kernel $K$ associated to an action is the intersection of all stabilizer subgroups
\begin{equation*} K = \bigcap_{p \in X} \G_p. \end{equation*}
Therefore it is a normal subgroup of $\G$, and thus the quotient group $\faktor{\G}{K}$ acts on $X$ with trivial kernel and identical orbits to the action of $G$. In particular, we can always assume without loss of generality that an action is effective.

\begin{definition}[Holomorphic Action]\index{Group action!holomorphic} Let $\G$ be a finite group acting on a set $X$. The action is said to be \textit{holomorphic} if for every $g \in \G$, the bijection 
\begin{equation*}\phi_g : X \ni p \longmapsto g \cdot p \in X\end{equation*}
 is a holomorphic map from $X$ to itself, i.e., $\phi_g$ belongs to $\mathrm{Aut}(X)$. \end{definition}

\begin{remark}The quotient space associated to an action, denoted by $\faktor{X}{\G}$, is the set of the orbits. Recall that the topology on $\faktor{X}{\G}$ is easily defined via the natural projection, i.e.,
\begin{equation*}\text{$U \subset \faktor{X}{\G}$ is open} \iff \text{$\pi^{-1}(U)$ is open in $X$}.\end{equation*}
Recall also that $\pi$ is an open map when the action is continuous (or, even better, holomorphic). \end{remark}

\paragraph{Stabilizer Subgroups.} In this short paragraph, we list some facts about the stabilizer subgroup of a finite group $\G$, acting holomorphically and effectively on a Riemann surface.

\begin{proposition}Let $\G$ be a finite group acting holomorphically and effectively on a Riemann surface $X$, and let $p \in X$ be a fixed point. \mbox{}
\begin{enumerate}[label=\textbf{(\arabic*)}]
\item The stabilizer subgroup $\G_p$ is a finite cyclic group.
\item If $\G$ is not finite, then $\G_p$ is still a cyclic group if it is finite.
\item The points of $X$ with nontrivial stabilizers are a discrete subset.
\end{enumerate} \end{proposition}

\paragraph{The Quotient Riemann Surface.}\index{Riemann Surface!quotient} In order to put a complex structure on the quotient surface $\faktor{X}{G}$, we must find complex charts.

\begin{proposition}\label{prop:acts}Let $\G$ be a finite group acting holomorphically and effectively on a Riemann surface $X$, and let $p \in X$ be a fixed point. Then there exists an open neighborhood $U$ of $p$ such that: \mbox{}
\begin{enumerate}[label=\textbf{(\alph*)}]
\item $U$ is invariant under the action of the stabilizer subgroup $\G_p$;
\item $U \cap \left(g \cdot U\right) = \emptyset$ for every $g \notin \G_p$;
\item the map $\faktor{U}{\G_p} \to \faktor{X}{\G}$, which sends a point $q \in U$ to its orbit $[q]$, is a homeomorphism onto an open subset of the quotient $\faktor{X}{\G}$;
\item no point of $U$, except $p$, is fixed by an element of $\G_p$.
\end{enumerate}\end{proposition}

\begin{proof}[Sketch of the Proof] Suppose that $\G \setminus \G_p = \{g_1, \, \dots, \, g_n\}$ are the elements of $\G$ not fixing $p$. A Riemann surface is, in particular, Hausdorff, thus, for each $i =1, \, \dots, \, n$, we can find open neighborhoods $V_i$ of $p$ and $W_i$ of $g_i \cdot p$ such that $V_i \cap W_i = \emptyset$.

In particular, $g_i^{-1} \cdot W_i$ is an open neighborhood of $p$ as $i$ ranges in $\{1, \, \dots, \, n\}$. Let us consider $R_i = V_i \cap \left(g_i^{-1} \cdot W_i \right)$, and let us set
\begin{equation*} U := \bigcap_{g \in \G_p} g \cdot R, \quad \text{where $R = \bigcup_{i = 1}^{n} R_i$}. \end{equation*}
This is exactly the sought open neighborhood of $p$, and it is now easy to check that it satisfies the properties \textbf{(a)}-\textbf{(d)}.
\end{proof}

\begin{theorem}Let $\G$ be a finite group acting holomorphically and effectively on a Riemann surface $X$. The quotient $\faktor{X}{\G}$ is a Riemann surface, whose complex charts are given by \hyperref[prop:acts]{Proposition \ref{prop:acts}}.

Moreover, $\pi : X \to \faktor{X}{\G}$ is a holomorphic map, whose degree is equal to $|\G|$, such that $\mathrm{mult}_p(\pi) = |\G_p|$ for any point $p \in X$. \end{theorem}

\paragraph{Ramification of the Quotient Map.} Let $\G$ be a finite group acting holomorphically and effectively on a Riemann surface $X$, and denote by $Y = \faktor{X}{\G}$ the quotient space.

\vspace{1.7mm}
Suppose that $y \in Y$ is a branch point, and let $x_1, \, \dots, \, x_s$ be the points of $X$ lying above $y$, i.e., $\pi^{-1}(y) = \{x_1, \, \dots, \, x_s\}$. Clearly the $x_i$'s are all in the same orbit by definition, thus they all have conjugate stabilizer subgroups, and each one of them is of the same order $r$.

Moreover, the number $s$ of points in this orbit is the index of the stabilizer, and so is equal to $|\G|/r$. This argument proves the following lemma:

\begin{lemma} Let $\G$ be a finite group acting holomorphically and effectively on a Riemann surface $X$, and let $Y = \faktor{X}{\G}$ be the quotient.

For every branch point $y \in Y$, there is an integer $r \geq 2$ such that $\pi^{-1}(y)$ consists of exactly $|\G|/r$ points of $X$, and at each of these preimage points $\pi$ has multiplicity exactly equal to $r$.  \end{lemma}

\begin{corollary}  \label{cor:123} Let $\G$ be a finite group acting holomorphically and effectively on a Riemann surface $X$, and let $Y = \faktor{X}{\G}$ be the quotient. Suppose that there are $k$ branch points $y_1, \, \dots, \, y_k \in Y$ such that, for each $i = 1, \, \dots, \, n$, $\pi$ has multiplicity $r_i$ at the $|G|/r_i$ points above $y_i$. Then
\begin{equation*} \begin{aligned} 2 \, g(X) - 2 & = \left|G\right| \, \left( 2 \, g(Y) - 2 \right) + \sum_{i=1}^{k} \frac{|G|}{r_i} \, (r_i - 1) = \\[1em] & = \left|G\right| \, \left[ 2 \, g(Y) - 2 + \sum_{i=1}^{k} \left( 1 - \frac{1}{r_i} \right) \right]. \end{aligned} \end{equation*}\end{corollary}

\begin{lemma}\label{lemma:231923} Let $r_1, \, \dots, \, r_k$ be given integers such that $r_i \geq 2$ for each $i$. Let
\begin{equation*}R := \sum_{i=1}^{k} \left(1 - \frac{1}{r_i}\right). \end{equation*}
Then it turns out that\mbox{}
\begin{enumerate}[label=\textbf{(\alph*)}]
\item \begin{equation*} R < 2  \iff \left(k, \, \{ r_i \}\right) = \begin{cases} k = 1 & \text{any $r_1$}; \\[0.4em] k = 2 & \text{any $r_1, \, r_2$}; \\ k = 3 & \text{$\{r_i\} = \{2, \, 2, \, r_3\}$; or} \\ k = 3 & \text{$\{r_i\} = \{2, \, 3, \, 3\}, \, \{2, \, 3, \, 4\}$ or $\{2, \, 3, \, 5\}$.} \end{cases} \end{equation*}
\item \begin{equation*} R = 2 \iff \left(k, \, \{ r_i \}\right) = \begin{cases} k = 3 & \text{$\{r_i\} = \{2, \, 3, \, 6\}$, $\{2, \, 4, \, 4\}$ or $\{3, \, 3, \, 3\}$; or} \\ k = 4 & \text{$\{r_i\} = \{2, \, 2, \, 2, \, 2\}$}. \end{cases} \end{equation*}
\item If $R > 2$ then $R \geq 2 + \frac{1}{42}$.
\end{enumerate} \end{lemma}

\paragraph{Hurwitz's Theorems on Automorphism} For compact Riemann surfaces of genus bigger or equal to $2$, \hyperref[cor:123]{Corollary \ref{cor:123}} leads to a bound on the order of the group $\G$ acting holomorphically and effectively.

\begin{theorem}[Hurwitz's Theorem]\index{Hurwitz's Theorem} \label{htf} Let $\G$ be a finite group acting holomorphically and effectively on a compact Riemann surface $X$ of genus $g(X) \geq 2$. Then
\begin{equation*} |\G| \leq 84 \cdot \left(g(X) - 1 \right). \end{equation*} \end{theorem}

\begin{proof} By \hyperref[cor:123]{Corollary \ref{cor:123}} it turns out that
\begin{equation} \label{111} 2\,g(X) - 2 = \left| \mathcal{G} \right| \, \left[ 2\,g \left(\faktor{X}{\G} \right) - 2 + R \right], \end{equation}
where $R$ is defined as in the Lemma above. \mbox{}
\begin{enumerate}[label=\textbf{(\arabic*)}]
\item Suppose that $g \left(\faktor{X}{\G} \right) \geq 1$. If there is no ramification, i.e., $R=0$, then $g \left(\faktor{X}{\G} \right) \geq 2$ (since $g(X) - 2 > 0$), and this implies immediately that
\begin{equation*}|\G| \leq g(X) - 1.\end{equation*}
If the ramification is nonzero, i.e., $R \neq 0$, then $R \geq 1/2$. Therefore $2 \, g \left(\faktor{X}{\G} \right) - 2 + R \geq 1/2$, and from \eqref{111} it follows that
\begin{equation*} |\G| \leq 4 \cdot \left(g(X) - 1 \right).\end{equation*}
\item Assume then that $g\left(\faktor{X}{\G} \right) = 0$. Then \eqref{111} reduces to
\begin{equation*} 2\,g(X) - 2 = \left| \mathcal{G} \right| \, \left[ R - 2 \right], \end{equation*}
which forces $R > 2$. Therefore Lemma \ref{lemma:231923} implies that $R - 2 \geq 1/42$, i.e,
\begin{equation*} \left| \mathcal{G} \right| \leq 84 \cdot (g - 1), \end{equation*}
as claimed.
\end{enumerate}
\end{proof}

In fact, the group of all automorphisms of a compact Riemann surface of genus at least two is a finite group. It implies that for such a Riemann surface, we have
\begin{equation*} \left| \mathrm{Aut}(X) \right| \leq 84 \cdot \left( g(X) - 1 \right), \end{equation*}
since the full group of the automorphisms certainly acts holomorphically and effectively on $X$; we shall prove this later on in the course.