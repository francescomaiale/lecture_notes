\title{Metodi Topologici in Analisi Globali - Lecture Notes}
\author{Francesco Paolo Maiale}


\documentclass[a4paper,10 pt]{report}

\usepackage{graphicx}
\usepackage{amsfonts}
\usepackage[pass]{geometry}
\usepackage{amsthm}
\usepackage{amsmath}
\usepackage[english]{babel}
\usepackage{tikz-cd}
\usepackage[utf8]{inputenc}
\usepackage{mathtools}
\usepackage{enumitem}
\usepackage{accents}
\usepackage{mathrsfs}  
\usepackage{pifont}
\usepackage{xcolor}
\usepackage{xparse}
\usepackage{hyperref}

\definecolor{cadmiumgreen}{rgb}{0.0, 0.42, 0.24}

\hypersetup{
    colorlinks=true,
    linkcolor=cadmiumgreen,
    filecolor=magenta,      
    urlcolor=cyan,
}

\pagestyle{plain}
\setlength{\topmargin}{0.0in}
\setlength{\headheight}{0.2in}
\setlength{\headsep}{0.0in}
\setlength{\footskip}{0.5in}
\setlength{\textheight}{8.3in}
\setlength{\textwidth}{6.0in}
\setlength{\oddsidemargin}{0.5in}
\setlength{\evensidemargin}{0.5in}
\setlength{\parindent}{0.2 in}

\newtheorem{theorem}{Theorem}[chapter]
\newtheorem{lemma}[theorem]{Lemma}
\newtheorem{proposition}[theorem]{Proposition}
\newtheorem{corollary}[theorem]{Corollary}
\theoremstyle{definition}
\newtheorem{definition}[theorem]{Definition}
\newtheorem*{scolium}{Problem}
\newtheorem{remark}{Remark}[chapter]
\newtheorem{example}{Example}[chapter]
\newtheorem*{notation}{Notation}
\newtheorem{exercise}{Exercise}[chapter]


\numberwithin{equation}{chapter}

\newcommand{\smallO}[1]{\scriptstyle\mathcal{O}}
\DeclarePairedDelimiter\floor{\lfloor}{\rfloor}
\newcommand*\conj[1]{\overline{#1}}
\newcommand{\R}{\mathbb R}
\newcommand{\C}{\mathbb C}
\newcommand{\F}{\mathcal F}
\newcommand{\N}{\mathbb N}
\newcommand{\sgn}{\mathrm{sgn}}
\newcommand{\Z}{\mathbb Z}
\newcommand{\Q}{\mathbb Q}
\newcommand{\parallelsum}{\mathbin{\!/\mkern-5mu/\!}}

\newcommand*{\double}[2][.1ex]{%
  \mathrel{\vcenter{\offinterlineskip%
  \hbox{$#2$}\vskip#1\hbox{$#2$}}}}
\newcommand*{\doublerightarrow}{\double{\longrightarrow}}

\newenvironment{steps}[1]
{ \begin{enumerate}[label = #1 \arabic*] }
{\end{enumerate}}

\makeatletter
\def\step{ 
\@ifnextchar [ \@step{\@noitemargtrue\@step[\@\itemlabel]}}
\def\@step[#1] {\item[#1] \mbox{} \\ \hspace*{\dimexpr-\labelwidth-\labelsep}}
\makeatother

%%%%%%%%%%%%%%%%%%%%%%%%%%%%%%%%%%%%%%%%%%%%%%%%%%%%%%%
% %   The first thanks indicates your affiliation
% %
% %  Just the name here.
% %
% % Your mailing address goes at the end.
% %
% % \thanks is also how you indicate grant support
% %
%%%%%%%%%%%%%%%%%%%%%%%%%%%%%%%%%%%%%%%%%%%%%%%%%%%%%%%

\begin{document}
\newpage
\maketitle
%%%%%%%%%%%%%%%%%%%%%%%%%%%
% abstract, keywords and Subject classification are optional.
%%%%%%%%%%%%%%%%%%%%%%%%%%%
{\Huge \textbf{Disclaimer}\par}

\vspace{0.4in} % if you wish

These notes arise from the \textbf{Topological Methods in Global Analysis} course, held by Professor Antonio Marino in the second semester of the academic year 2016/2017.

They have not been properly reviewed yet, hence to report oversights, mistakes, etc., one can send an email to frank094 (at) hotmail (dot) it.

\tableofcontents
\newpage
\chapter{Introduction}

In this chapter, we introduce the main topics of the course and give a brief overview of what we will see and what we will be able to prove by the end of the course.

\section{Plateau's Problem}

The primary goal and the motivating example of this course is the \textbf{Plateau's problem}, that is, the problem to find the $d$-dimensional surface $\Sigma$ of the minimal area with prescribed $(d-1)$-dimensional boundary $\Gamma$.

By the end, we will be able to prove that a solution indeed exists, but we will not find it explicitly since it is a $NP$ (hard) numerical problem.

As of now, the problem is not well defined. In fact, the notions of \textit{surface}, \textit{area}, and \textit{boundary} make sense in the smooth setting but, as the examples below show, we need to work in a less regular setting.

More precisely, requiring the surface to be smooth is not enough for modeling reasons (e.g., dip a wire frame into a soap solution, form a soap film, and look for the minimal surface whose boundary is the wire frame), and also for existence reasons.

\begin{example} Here we give a list of Plateau's problems with prescribed boundary conditions, and we write down the correct solutions, without proving anything. \mbox{}
\begin{enumerate}[label=\textbf{(\alph*)}]
\item Let us identify $\R^4 \cong \C \times \C$ and, if $d = 2$, let us consider the smooth boundary given by
\begin{equation*} \Gamma_1 := \left( S^1 \times \{0\} \right) \cup \left( \{0\} \times S^1 \right). \end{equation*}
Surprisingly, every minimizing sequence of smooth surfaces converges to a surface which is not smooth at all. Indeed, the solution of the problem is
\begin{equation*} \Sigma_1 := \left( D^2 \times \{0\} \right) \cup \left( \{0\} \times D^2 \right). \end{equation*}
The surface $\Sigma_1$ is clearly singular at the origin, but the singularity may be removed (by factorizing it into two nonsingular surfaces).
\item Let us identify $\R^4 \cong \C \times \C$ and, if $d = 2$, let us consider the smooth boundary given by
\begin{equation*} \Gamma_2 := \left\{ (z^2, \, z^3) \: : \: z \in S^1 \right\}. \end{equation*}
The solution to the Plateau's problem is
\begin{equation*} \Sigma_2 := \left\{ (z^2, \, z^3) \: : \: z \in D^2 \right\}, \end{equation*}
which is a non-smooth surface, whose singularity cannot be removed (since the polynomial $z_1^3 = z_2^2$ cannot be factorized).
\item Let us identify $\R^8 \cong \R^4 \times \R^4$ and, if $d = 7$, let us consider the smooth boundary given by
\begin{equation*} \Gamma_3 := S^3 \times S^3. \end{equation*}
The minimal surface of prescribed boundary $\Gamma_3$ is
\begin{equation*} \Sigma_3 := \left\{ (x_1, \, x_2) \in \R^4 \times \R^4 \: : \: |x_1| = |x_2| \leq 1 \right\}. \end{equation*}
\end{enumerate}
\end{example}

To conclude this introductive chapter, we give a brief overview of the main approaches (studied in this course) to the Plateau's problem, as $d$ ranges between $1$ and $\infty$.

\section{Geodesics problem ($d=1$)}

The geodesics problem (that is, find the shortest curve connecting two points) is, surprisingly, still an open in the non-Riemannian setting. However, in the Riemannian setting, the geodesics problem is completely solved.

Indeed, if we consider the curves parametrized by paths, the \textit{length} is a well-defined notion, and the associated functional is lower semi-continuous and coercive; hence the compactness is easy to prove.

There are many possible approaches to the geodesics problem, e.g., the Steiner approach and the set theoretical approach, which we describe briefly in the remainder of the section.

\paragraph{Steiner Problem.} It is also called networks approach, and it is used to prove the existence of the geodesics and find the explicit expression for it. The reader may consult \cite{steinerpro} for a detailed dissertation on the topic.

\paragraph{Set Theoretical Approach.} The main idea is to find a closed and connected set $\Sigma$ of minimum \textit{length}, containing a given finite set $\Gamma$. As we shall see later in the course, in this case the length is a well defined concept: the \textit{Hausdorff distance}.

In fact, if $X$ is a suitable space (metric, endowed with Hausdorff distance, etc...), then the class defined by
\begin{equation*}\mathcal{X} := \left\{ K \subseteq X \: : \: K \, \, \text{compact and connected} \right\} \end{equation*}
is compact and, by Gotab theorem\footnote{\cite{falconer} Let $\mathscr{C}$ be an infinite collection of non-empty compact sets all lying in a bounded portion $B$ of $\R^n$. Then there exists a sequence $\{E_j\}$ of distinct sets of $\mathfrak{C}$ convergent in the Hausdorff metric to a non-empty compact set $E$.}, $\mathcal{H}^1$ is lower semi-continuous on $X$. 

\section{"Surface" Problem ($d>1$)}

\paragraph{3.} The Plateau's problem is much harder when $d = 2$, but there are still many approaches possible some of which relying, in a certain sense, on the work already done in the geodesics case.

\paragraph{Set Theoretical Approach.} This approach is highly nontrivial. For example, one may ask what does it mean that a compact set $\Sigma$ spans a boundary $\Gamma$? Moreover, there is another problem one should deal with: the $2$-dimensional Hausdorff measure $\mathcal{H}^2$ is, generally, not lower semi-continuous. The reader may consult \cite{Reifenberg1960} for a complete treatise of the topic.

\begin{remark}Suppose that $d = 2$, $n = 3$ and that $\Sigma$ is a surface with boundary $\Gamma$. If $\gamma$ is another closed curve, linked to $\Gamma$ (by a nonzero linking number), then $\gamma \cap \Sigma \neq \emptyset$. \end{remark}

\paragraph{Parametric Approach.} This method is essentially due to Douglas \cite{douglas}. The main idea is the following: since a parametrization $\phi : D^2 \to \R^n$ defines surfaces in $\R^n$, the area functional is well-defined and given by the formula
\begin{equation*}A(\phi) := \int_{D^2} \left| \frac{\partial \, \phi}{\partial \, s_1} \wedge \frac{\partial \, \phi}{\partial \, s_2} \right| \, \mathrm{d}s_1 \, \mathrm{d}s_2. \end{equation*}
On the other hand, the existence through lower semi-continuity and the compactness are a delicate matter, since coercivity is not an easy property to obtain (the integrand is similar to a determinant).

There is a trick which is similar to the one we can use to find geodesics in the differential geometry setting. More precisely, we consider the functional
\begin{equation*}E(\phi) := \frac{1}{2} \, \int_{D^2} \left| \nabla \phi \right|^2 \, \mathrm{d}s_1 \, \mathrm{d}s_2. \end{equation*}
If we find a minimal point $\phi$ for $E$, then $\phi$ will be a \textbf{conformal parametrized} minimum for $A$. This trick, on the other hand, heavily depends on a nontrivial theorem: every such $\Sigma$ admits a conformal re-parametrization.

The lack of conformal parametrization, though, is what stop us from extending the same trick to dimension $d$ strictly bigger than $2$.

\paragraph{Higher Dimension.} If the codimension of $\Sigma$ is equal to $1$ (that is, $n = d+1$), then finite perimeter sets generalize the notion of open $(d+1)$-dimensional sets with smooth boundary in $\R^n$.

The class of finite perimeter sets has excellent compactness properties and a notion of area lower semi-continuous. 

This approach is called "weak" surfaces approach, and it is essentially due to Caccioppoli \cite{cacio} and De Giorgi \cite{degi}. A different approach, working for any $d$ and $n$, referred to as \textit{integral currents}, was introduced by Federer and Flaming in their joint paper \cite{fed}.
\chapter{Topological Degree Modulo $2$}

The primary goal of this chapter is to introduce the notion of topological degree modulo two, prove some of  its leading properties and apply it to global analysis problems.

\section{Topological Preliminaries}

In this section, $X$ and $Y$ will be metric spaces, unless stated otherwise.

\begin{definition}[Proper map] Let $A \subseteq X$ be a subset of $X$, and let $\Phi : A \to Y$ be a continuous map.  We say that $\Phi$ is \textit{proper} if, for any sequence $(x_n)_{n \in \N} \subset A$ such that $\left(\Phi(x_n) \right)_{n \in \N}$ converges to some $y \in Y$, there exists a converging subsequence $(x_{n_k})_{k \in \N}$ in $X$. \end{definition}

\begin{lemma} \label{lemma:cpt} Let $\Phi : A \subset X \to Y$ be a proper map. \mbox{}
\begin{enumerate}[label=\textbf{ (\alph*)}]
\item If $A$ is closed as a subset of $X$, then $\Phi$ is a closed map.
\item If $A$ is closed as a subset of $X$ and $K \subset Y$ is compact, then $\Phi^{-1}(K)$ is compact in $X$.
\end{enumerate} \end{lemma}

\begin{proof} \mbox{}
\begin{enumerate}[label=\textbf{ (\alph*)}]
\item Let $F$ be any closed subset of $A$: we need to prove that $\Phi(F)$ is closed.

Let $(y_n)_{n \in \N} \subset \Phi(F)$ be any converging sequence in $Y$, let $y$ be its limit and let $(x_n)_{n \in \N} \subset F$ be any sequence such that $\Phi(x_n) = y_n$.

Since $\Phi$ is proper, there exists a subsequence $(x_{n_k})_{k \in \N} \subset F$ such that $x_{n_k} \to x \in F$ (since $F$ is closed). Consequently, by continuity of $\Phi$, we conclude that $y = \Phi(x) \implies y \in F$.
\item Let $(x_n)_{n \in \N} \subset \Phi^{-1}(K)$ be any sequence. The image sequence $(\Phi(x_n))_{n \in \N} \subset K$ is contained in $K$, and thus, by compactness, there exists a subsequence $(\Phi(x_{n_k}))_{k \in \N}$ which converges to $y \in K$.

Since $\Phi$ is proper, there exists a subsequence $(x_{n_{k_h}})_{h \in \N} \subset \Phi^{-1}(K)$ such that it converges to $x \in X$. But $A$ is closed, thus $x \in A$ and, by continuity of $\Phi$, it turns out that $\Phi(x) = y$ and $x \in \Phi^{-1}(K)$.
\end{enumerate}
\end{proof}

\section{Cardinality of the Preimage }

\begin{definition}[Critical point/values] Let $\Omega \subset \R^N$ be an open subset and let $\Phi \in C^1 \left( \Omega, \, \R^M \right)$ be a differentiable map. A point $x_0 \in \Omega$ is a \textit{critical point} if the differential at $x_0$, denoted by $\mathrm{d}\,\Phi(x_0)$, is not surjective. If we define
\begin{equation} \label{eq:locus} Z_\Phi := \left\{ x \in \Omega \, \left| \, \mathrm{d} \, \Phi(x) \, \, \, \text{not surjective} \right. \right\}, \end{equation}
then $y \in \R^M$ is a \textit{critical value} if $\Phi^{-1}(y) \cap Z_{\Phi} \neq \emptyset$.
 \end{definition}

\begin{lemma} \label{lemma:fond} Let $\Omega \subset \R^N$ and let $\Phi : \overline{\Omega} \to \R^N$ be a proper map of class $C^1(\Omega) \cap C^0(\overline{\Omega})$. If $y$ is a regular value not in the image of the border of $\Omega$, that is,
\begin{equation*} y \in \R^N \setminus \Phi \left( \partial \, \Omega \cup Z_\Phi \right), \end{equation*}
then it turns out that \mbox{}
\begin{enumerate}[label=\textbf{(\alph*)}]
\item the cardinality of the fiber is finite, i.e., $| \Phi^{-1}(y) | < \infty$;
\item if $y \in \Phi(\Omega)$ and $\Phi^{-1}(y) = \{x_1, \, \dots, \, x_k\}$, then there is a neighborhood $V$ of $y$ and there are neighborhoods $U_1, \, \dots, \, U_k$ of $x_1, \, \dots, \, x_k$ such that
\begin{equation*}U_1, \, \dots, \, U_k \subset \Omega \quad \text{and} \quad V \cap \Phi \left( \partial \, \Omega \cup Z_\Phi \right) = \emptyset. \end{equation*}
Moreover, the restriction $\Phi \, \big|_{U_i} : U_i \to V$ is a diffeomorphism for all $i = 1, \, \dots, \, k$ and 
\begin{equation*} \Phi^{-1}(V) \subset \bigcup_{i = 1}^k U_i . \end{equation*}
\end{enumerate}
\end{lemma}

\begin{proof} \mbox{}
\begin{enumerate}[label=\textbf{(\alph*)}]
\item By \hyperref[lemma:cpt]{Lemma \ref{lemma:cpt}}, the subset $\Phi^{-1}(y)$ is compact and, by the local invertibility theorem, it is also discrete. Since a compact and discrete set is finite, we conclude that the cardinality of the fiber of $y$ is finite, i.e., $| \Phi^{-1}(y) | < \infty$.
\item Let us assume that $\Phi^{-1}(y) = \{x_1, \, \dots, \, x_k\}$. By the local invertibility theorem, there are $U_i$ open neighborhoods of $x_i$ and $V_i$ open neighborhoods of $y$ such that $\Phi \, \big|_{U_i} : U_i \to V_i$ is a diffeomorphism for all $i = 1, \, \dots, \, k$. We claim that there exists a neighborhood $V$ of $y$ such that
\begin{equation*} \Phi^{-1} (V) \subset \bigcup_{i = 1}^{k} U_i. \end{equation*}
Suppose that the claim is not true. Then there exist a converging sequence $(y_n)_{n \in \N}$ (to $y$) and a sequence $(z_n)_{n \in \N} \subset \Omega$ such that $\Phi(z_n) = y_n$ for any $n \in \mathbb{N}$, and
\begin{equation*} z_n \notin \bigcup_{i=1}^{k} U_i\end{equation*}
for any $n \geq N$. Since $\Phi$ is proper there is a subsequence $(z_{n_k})_{k \in \N}$ such that $z_{n_k} \to x$ and $\Phi(x) = y$, but this is absurd since
\begin{equation*} \Phi^{-1}(y) = \{ x_1, \, \dots, \, x_k \} \subset \bigcup_{i=1}^k U_i. \end{equation*}
In conclusion, we just need to consider the following new neighborhoods for the $x_i$'s
\begin{equation*} U_i \mapsto U_i^\prime := U_i \cap \Phi^{-1}(V). \end{equation*}
\end{enumerate}
\end{proof}

\begin{corollary} \label{corollary:same} If the same assumption as of \hyperref[lemma:fond]{Lemma \ref{lemma:fond}} are met, then the cardinality function
\begin{equation*} \symbol{35} \, \Phi^{-1} : \R^N \setminus \Phi \left( \partial \, \Omega \cup Z_\Phi \right) \to \mathbb{N},\end{equation*} is continuous. In particular, it is constant on the connected component $\mathcal{C}$ of the domain. \end{corollary}

\begin{theorem} \label{theorem:surj} Let $N \geq 2$ and let $\Phi \in C^1 \left(\R^N; \; \R^N\right)$ be a proper map such that $Z_\Phi$ is a finite set. Then $\Phi$ is surjective, i.e.
\begin{equation*} \forall \, y \in \R^N, \, \, \exists \, x \in \R^N \: : \: \Phi(x) = y. \end{equation*}\end{theorem}

\begin{proof}For any $N \geq 2$, the set $\R^N \setminus \Phi(Z_\Phi)$ is connected, hence \hyperref[corollary:same]{Corollary \ref{corollary:same}} implies that the map $\symbol{35} \, \Phi^{-1} $ is globally constant, that is, the fiber of all the regular values have the same cardinality.

Assume, by contradiction, that it is equal to $0$. Then the image of $\Phi$ is a discrete subset of $\R^N$, i.e., $\Phi(\R^N) = \Phi(Z_\Phi) = \{ y_1, \, \dots, \, y_k\}$. On the other hand, $\Phi$ is continuous and $\R^N$ is connected, hence $k = 1$. Therefore $\Phi$ is a constant map, and this is absurd since we assumed $\Phi$ to be proper.
\end{proof}

\begin{lemma} Let $\Phi : \R^N \to \R^N$ be any function. Then
\begin{equation*}\text{$\Phi$ is proper} \iff \lim_{|x| \to + \infty} |\Phi(x)| = + \infty. \end{equation*} \end{lemma}

\begin{theorem}[Fundamental Theorem of Algebra] Let $p(z) := a_n \, z^n + \dots + a_0$ be a nonconstant complex polynomial from $\C$ to $\C$. Then $p$ is surjective. \end{theorem}

\begin{proof}Clearly $p$ is a function of class $C^\infty(\C; \; \C)$ and there is a characterization of critical points in terms of the derivative, that is,
\begin{equation*}z \in Z_p \iff p^\prime(z) = 0. \end{equation*}
By the long division algorithm, there are only finitely many such points. Therefore we conclude the proof by noticing that
\begin{equation*} \lim_{|z| \to + \infty} |p(z)| = + \infty, \end{equation*}
that is, $p$ is a proper map and \hyperref[theorem:surj]{Theorem \ref{theorem:surj}} applies.\end{proof}

\begin{remark} The function $p : \C \to \C$ is clearly differentiable (it follows easily from the definition) and the differential is given by
\begin{equation*} \mathrm{d} \, p(z)[h] := p^\prime(z) \, h, \qquad \forall z, \, h \in \C. \end{equation*} \end{remark}

\begin{proposition} Let $N \geq 2$ and let $\Phi:\Omega \to \R^N$ be a function of class $C^1 \left(\Omega; \; \R^N \right)$. If $x_0 \in \Omega$ is an isolated critical point, then $\Phi(x_0)$ is an internal point of $\mathrm{Ran}(\Phi)$. \end{proposition}

\begin{proof}This proposition requires many steps to be proved. To clarify the method to the reader, we divide the proof into five little steps.

\paragraph{Step 1.} By assumption $x_0$ is an isolated critical point. Therefore there exists $\rho_0 > 0$ such that
\begin{equation*} \overline{B(x_0, \, \rho)} \cap Z_\Phi = \{x_0\}. \end{equation*}

\paragraph{Step 2.} We claim that there exists $r \in (0, \, \rho_0]$ such that $\Phi(x_0) \notin \Phi\left(\partial \, B(x_0, \, r)\right)$. We argue by contradiction. If $r$ does not exist, then for all $\rho \in (0, \, \rho]$ there is a point $x_\rho \in \partial \, B(x_0, \, \rho)$ such that $\Phi(x_\rho) = \Phi(x_0)$.

\paragraph{Step 3.} For any $\rho \in [1/2 \, \rho_0, \, \rho_0]$ we can always find $x_\rho \in \partial \, B(x_0, \, \rho)$ with the same image as $x_0$. Consequently, there exists a sequence $(x_{\rho_n})_{n \in \N}$ such that $x_{\rho_n} \to \tilde{x}$ and $\Phi(x_{\rho_n}) \to \Phi(x_0)$ (since it is the constant sequence).

The limit $\tilde{x}$ is an accumulation point by definition, and its image is equal to $\Phi(x_0)$. Thus $\tilde{x} \in Z_\Phi$ and $\tilde{x} \in \partial \, B(x_0, \, 1/2 \, \rho_0)$. But we chose $\rho > 0$ in such a way that
\begin{equation*}\overline{B(x_0, \, \rho)} \cap Z_\Phi = \{x_0\},\end{equation*}
thus we have finally derived a contradiction, and the claim is now proved.

\paragraph{Step 4.} The restriction
\begin{equation*} \Psi := \Phi \, \big|_{\overline{B(x_0, \, r)}} : \overline{B(x_0, \, r)} \to \R^N\end{equation*}
is a map of class $C^1$, which is proper and such that the set of critical points is finite ($Z_\Psi = \{x_0\}$).

Since $\Phi(x_0) \notin \Phi(\partial \, B(x_0, \, r))$, it follows from \hyperref[corollary:same]{Corollary \ref{corollary:same}} that $\symbol{35} \, \Phi^{-1} $ is constant in a neighborhood $\mathcal{U}$ of $\Phi(x_0)$ deprived of that point.

\paragraph{Step 5.} Up to a rescaling, we may always assume that $\mathcal{U} \cap \Phi(\partial \, B(x_0, \, r)) = \emptyset$. If the cardinality is constantly zero, i.e., $\symbol{35} \, \Phi^{-1}(z) = 0$ for any $z \in \mathcal{U} \setminus \{ \Phi(x_0)\}$, then $\Phi$ is a constant map which is defined on the subset $\Phi^{-1}(\mathcal{U})$. Hence $\Phi^{-1}(\mathcal{U}) \subset Z_\Phi$ and this is impossible, since $x_0$ is the only point of $Z_\Phi$.
\end{proof}

\begin{corollary} \label{corollary:openmap} Let $N \geq 2$ and let $\Phi:\Omega \to \R^N$ be a function of class $C^1 \left(\Omega; \; \R^N \right)$. If every critical point of $\Phi$ is isolated, then $\Phi(\Omega)$ is open (that is, $\Phi$ is an open map). \end{corollary}

\begin{corollary} Let $\Phi :\R^N \to \R^N$ be a function of class $C^1\left(\R^N; \; \R^N \right)$. Assume that every point of the set $Z_\Phi$ is isolated, and also assume that
\begin{equation*}\lim_{|x| \to + \infty} |\Phi(x)| = + \infty. \end{equation*}
Then $\Phi$ is a surjective map. \end{corollary}

\begin{proof}It follows from \hyperref[corollary:openmap]{Corollary \ref{corollary:openmap}} that $\Phi$ is an open map. On the other hand $\Phi$ is proper, thus it is also a closed map.

Since $\R^N$ is connected and $\Phi \neq 0$, it turns out that $\Phi(\R^N)$ is open and closed, hence it is necessarily equal to the whole $\R^N$. \end{proof}

\begin{exercise}Prove that from \hyperref[lemma:fond]{Lemma \ref{lemma:fond}} it follows that, for any connected component $\mathcal{C}$ of $\R^N \, \setminus \, \Phi \left( \partial \, \Omega \cup Z_\Phi \right)$, there exist $k \in \mathbb{N} \setminus \{0\}$ and $\varphi_1, \, \dots, \, \varphi_k : \mathcal{C} \to \Omega$ continuous functions such that $\Phi^{-1}(y) = \left\{\varphi_1(y), \, \dots, \, \varphi_k(y) \right\}$. \end{exercise}

\begin{proof}[\textbf{Solution}] If $y \notin \Phi(\Omega)$, then the thesis is obvious since $\Phi^{-1}(y) = \emptyset$. If $y \in \Phi(\Omega)$, then the second assertion of Lemma \ref{lemma:fond} holds true, that is there exist $k \in \mathbb{N}$, a neighborhood $V$ of $y$ and neighborhoods $U_1, \, \dots, \, U_k$ of $x_1, \, \dots, \, x_k$ such that $U_i \subseteq \Omega$, $V \cap \Phi(\partial \, \Omega \cup Z_\Phi) = \emptyset$ and $\Phi:U_i \to V$ is a diffeomorphism such that $\Phi^{-1}(V) \subseteq \cup_{i = 1}^n U_i$.

\vspace{2.5mm}
\noindent Moreover, since $y \in \mathcal{C}$ and $\mathcal{C}$ is a connected component, the same assertion holds true for any $z \in \mathcal{C}$ with a uniform $k$. Clearly, for any $z \in \mathcal{C}$, we have
\begin{equation*}\Phi^{-1}(z) = \left\{ \varphi_{1, \, z}(z), \, \dots, \, \varphi_{k, \, z}(z) \right\}, \end{equation*}
where $\varphi_{i, \, z} := \Phi \, \big|_{U_{i, \, z}}$ is a diffeomorphism for any $i = 1, \, \dots, \, k$. If we set
\begin{equation*} \varphi_i \stackrel{\mathrm{def}}{=} \bigcup_{z \in \mathcal{C}} \varphi_{i, \, z}, \end{equation*}
the proof is complete, if we are able to prove that $\varphi_{i, \, z}$ coincides in nonempty intersections. But this is immediate from the definition, since they are all restriction of $\Phi$ to appropriate sets. \end{proof} 

\begin{remark}It's not necessary to work with functions in $\R^N$, since it's more than enough to work in a Banach space with a local invertibility theorem or in a $C^1$ variety. \end{remark}

\section{Sard's Lemma}

In this section, we want to state and prove a fundamental result, which is widely used in this course, in differential geometry, in global analysis, etc.

On the other hand, in this course, we will need only a particular case of Sard's lemma (which is easier to prove as a standalone), which is when $M \geq N$.

The proof in the general case is rather involved. We will include in an optional subsection the proof in any dimension, provided that the map is $C^\infty(\Omega)$.

\begin{lemma}[Sard's Lemma] \label{lemma:sard} Let $\Phi : \Omega \subseteq \R^N \to \R^M$ be a function of class $C^k(\Omega)$, where $k \geq \min \, \{N - M + 1, \, 1\}$. Then the Lebesgue measure of the set $\Phi(Z_\Phi)$ is zero.   \end{lemma}

\begin{proposition} Let $M \geq N$ and assume that $\Phi \in C^1(\Omega)$. Then $\Phi(Z_\Phi)$ is a null set in $\R^M$. \end{proposition}

\begin{proof}Let $Q$ be a closed cube, and let $\ell$ be the length of its edge. Assume that $\ell$ is small enough so that $Q \subseteq \Omega$, and also assume that $Q \cap Z_\Phi \supseteq \{x\}$. Let $X$ be a $(M-1)$-dimensional linear subspace of $\R^M$ such that
\begin{equation*} \mathrm{d} \, \Phi(x) \left[\R^N\right] \subseteq X, \end{equation*}
which exists because $\mathrm{d}\phi_x$ is not surjective, and let $P : \R^n \to X$ be the orthogonal projection onto $X$ and $P_\perp : \R^n \to X^\perp$ the orthogonal projection defined by setting $P_\perp := \mathrm{Id} - P$. For any $y \in Q$, it turns out that
\begin{equation*} \phi(y) - \phi(x) = \int_{0}^{1} \mathrm{d} \, \Phi \left(x + t \, (y - x) \right)[y - x] \, \mathrm{d}t. \end{equation*}
Then the projection along $X$ is given by
\begin{equation*} P\left(\phi(y) - \phi(x) \right) = \int_{0}^{1} P \left( \mathrm{d} \, \Phi \left(x + t \, (y - x) \right)[y - x] \right) \, \mathrm{d}t, \end{equation*}
while the perpendicular projection is given by
\begin{equation*} P_\perp \left(\phi(y) - \phi(x) \right) = \int_{0}^{1} P_\perp \left( \mathrm{d} \, \Phi \left(x + t \, (y - x) \right)[y - x] \right) \, \mathrm{d}t, \end{equation*}
which is also equal to
\begin{equation*} \int_{0}^{1} P_\perp \left( \mathrm{d} \, \Phi \left(x + t \, (y - x) \right)[y - x] - \mathrm{d} \, \Phi(x) [y - x] \right) \, \mathrm{d}t  \end{equation*}
since $P_\perp \left( \mathrm{d} \, \Phi(x) \right) = 0$ by definition. Therefore
\begin{equation*} \left| P \left( \Phi(y) - \Phi(x) \right) \right| \leq H_Q \cdot \left| y - x \right| \leq H_Q \cdot \sqrt{N} \cdot \ell, \end{equation*}
where $H_Q := \sup \left\{ \|\mathrm{d} \, \Phi(z)  \| \: \left| \: z \in Q \right. \right\}$, and
\begin{equation*} \left| P_\perp \left( \Phi(y) - \Phi(x)\right) \right| \leq \sigma_Q \cdot \left| y - x \right| \leq \sigma_Q \cdot \sqrt{N} \cdot \ell, \end{equation*}
where $\sigma_Q := \sup \left\{ \| \mathrm{d} \, \Phi(z) -  \mathrm{d} \, \Phi(x)  \| \: \left| \: z \in Q  \right.\right\}$.

\vspace{2.5mm}
Consequently $\Phi(Q)$ is contained in the product between a ball in $X$ of radius $H_Q \cdot \sqrt{N} \cdot \ell$ and an interval of the $1$-dimensional subspace $\mathrm{Ker} \, (P)$ of amplitude $\sigma_Q \cdot \sqrt{N} \cdot \ell$. If we let $\mathcal{L}$ be the Lebesgue measure, then
\begin{equation*}\mathcal{L} \left( \Phi(Q) \right) \leq c_Q \, \sigma_Q \, \ell^M, \qquad c_Q := 2 \, \omega_{M - 1} \, H_{Q}^{M - 1} \, N^{\frac{M}{2}}, \end{equation*}
where $\omega_{M-1}$ denotes the volume of the unity ball in $\R^{M-1}$.

Let us divide the cube $Q$ in $n^N$ cubes whose edges' length is $\frac{\ell}{n}$, and let us consider only the little cubes whose intersection with $Z_\Phi$ is nonempty. Then
\begin{equation*}\begin{aligned} \mathcal{L} \left( \Phi(Q \cap Z_\Phi) \right) & \leq \sum_{i}\mathcal{L} \left( \Phi(Q_i) \right) \leq \sum_i C_{Q_i} \, \sigma_{Q_i} \, \left(\frac{\ell}{n}\right)^M \leq \\& \leq C_Q \, \sigma(n) \, \left( \frac{\ell}{n} \right)^M \, n^N, \end{aligned}\end{equation*}
where
\begin{equation*} \sigma(n) := \sup \left\{ \| \mathrm{d}\,\Phi(z) - \mathrm{d} \, \Phi(z^\prime) \| \: \left| \: z, \, z^\prime \in Q, \, \, |z - z^\prime| \leq \sqrt{N} \, \frac{\ell}{n} \right. \right\}. \end{equation*}
Therefore, for any $n \in \mathbb{N}$, it turns out that
\begin{equation*} \mathcal{L} \left( \Phi(Q \cap Z_\Phi) \right) \leq c_Q \, \ell^M \, \frac{1}{n^{M-N}} \, \sigma(n),\end{equation*}
thus, if we take the limit for $n \to + \infty$, the function $\sigma(n)$ goes to zero (since $\mathrm{d} \, \Phi$ is uniformly continuous in the cube).

Finally, $M \geq N$ implies that $\mathcal{L} \left( \Phi(Z_\Phi \cap Q) \right) = 0$ for any cube $Q$ contained in $\Omega$. But $Z_\Phi$ can be easily covered using countable many cubes, thus what we proved is enough to conclude that the image of the set of critical points is a null set, that is, $\mathcal{L}\left( \Phi(Z_\Phi )\right) = 0$.
\end{proof}

\begin{corollary} Let $\Phi : \Omega \subset \R^N \to \R^M$ be a function of class $C^k(\Omega)$, where $k \geq \min \, \{N - M + 1, \, 1\}$. \mbox{}
\begin{enumerate}[label=\textbf{(\alph*)}]
\item If $M > N$, then the Lebesgue measure of $\mathrm{Ran}(\Phi) = \Phi(\Omega)$ is zero.
\item If $M \leq N$, then $\Phi(\Omega)$ is the union of an open set and a null set. \end{enumerate}\end{corollary}

\section{Topological Degree for $C^2$ Maps}

In this section $\Omega$ denotes an open bounded subset of $\R^N$, $I$ an open interval in $\R$ containing both $0$ and $1$, and we assume that
\begin{equation*}F : I \times \overline{\Omega} \to \R^N \end{equation*}
is a function of class $C^2$ such that $\Phi_0(x) := F(0, \, x)$,  $\Phi_1(x) := F(1, \, x)$.

\begin{lemma} \label{lemma:fundmod2}Let $y \in F\left(I \times \overline{\Omega} \right) \setminus F \left(Z_F \cup \left(I \times \partial \, \Omega \right) \right)$, and assume that $y \notin \Phi_0(Z_{\Phi_0}) \cup \Phi_1(Z_{\Phi_1})$. \mbox{}
\begin{enumerate}[label=\textbf{(\alph*)}]
\item The subset $F^{-1}(y) \subseteq \Omega$ is a $1$-manifold of class $C^2$ with no border.
\item If $(0, \, x) \in F^{-1}(y)$, then the tangent line to $F^{-1}(y)$ at $(0, \, x)$ does not lie on the border $\{0, \, 1\} \times \R^N$. Similarly if we replace $(0, \, x)$ with $(1, \, x)$.
\item Let $\Gamma$ be a connected component of $F^{-1}(y) \cap \left([0, \, 1] \times \Omega\right)$. If 
\begin{equation*}\Gamma \cap \left( \{0, \, 1\} \times \Omega \right) \neq \emptyset, \end{equation*}
then there exists a curve $\gamma : [a, \, b] \to [0, \, 1] \times \Omega$ of class $C^2$, such that $\alpha([a, \, b]) = \Gamma$, $\alpha^\prime(\tau) \neq 0$ for any $\tau \in [a, \, b]$ and 
\begin{equation*} \{ \alpha(a), \, \alpha(b) \} = \Gamma \cap \left( \{ 0, \, 1 \} \times \Omega \right). \end{equation*}
\item If $\alpha$ is the curve given in the previous point, then $\alpha^\prime(a)$ and $\alpha^\prime(b)$ do not belong to $\{0, \, 1\} \times \R^N$.
\end{enumerate}  \end{lemma}

\begin{proof}  \mbox{}
\begin{enumerate}[label=\textbf{(\alph*)}]
\item From the implicit function theorem it follows easily that, for any point $(t, \, x) \in F^{-1}(y)$, there exists a neighborhood $U_{(t, \, x)}$ such that $F^{-1}(y) \cap U_{(t, \, x)}$ is a $1$-manifold.

In fact, since $\mathrm{d}F_{(t, \, x)}$ is surjective there exist charts such that the composition with $F$ is locally the projection over the last $N$ coordinates (i.e., it is locally the graph of a real function).

Since $F^{-1}(y)$ is locally a $1$-manifold for any point $(t, \,x) \in F^{-1}(y)$, then it is globally a $1$-manifold. Moreover, for any $p := (t, \,x) \in F^{-1}(y)$, it turns out that
\begin{equation*} T_p \, \left(F^{-1}(y) \right) = \mathrm{Ker} \left(\mathrm{d} \, F_p \right). \end{equation*}
\item We want to prove that the manifold $F^{-1}(y)$ is not tangent to the manifolds $\{0\} \times \Omega$ and $\{1\} \times \Omega$. Let $x_0 \in \Omega$ be a point such that $(0, \, x_0) \in F^{-1}(y)$. Then $\Phi_0(x_0) = y$ and the tangent line to $F^{-1}(y)$ at $(0, \, x_0)$ is given by the unidimensional kernel of $\mathrm{d} \, F(0, \, x_0)$:
\begin{equation*} \left\{ (h, \, k) \in \R \times \R^N \: : \: \mathrm{d} \, F(0, \, x_0)[h, \, k] = 0 \right\}. \end{equation*}
Recall that the differential is a linear map, thus for any $(h, \, k) \in \R \times \R^N$
\begin{equation*}\mathrm{d} \, F (0, \, x) [h, \, k] = \mathrm{d} \, F(0, \, x)[h, \, 0] + \mathrm{d} \, F(0, \, x)[0, \, k]. \end{equation*}
Let $h = 0$: we want to prove that $k = 0$. By linearity this implies that
\begin{equation*}0 = \mathrm{d} \, F(0, \, x)[h, \, 0] + \mathrm{d} \, F(0, \, x)[0, \, k] \implies  \mathrm{d} \, F(0, \, x)[0, \, k] = 0, \end{equation*}
but $ \mathrm{d} \, F(0, \, x)[0, \, k] = \mathrm{d} \, \Phi_0(x)[k]$ and $\mathrm{d} \, \Phi_0$ is an isomorphism, thus we infer that $k = 0$.
\item If we set $V := F^{-1}(y) \cap \left([0, \, 1] \times \Omega\right)$, then it is easy to prove that $V$ is a compact $1$-manifold. Also, it is not tangent to $\{0\} \times \Omega$ or $\{1\} \times \Omega$, and its border, if any, is given by the intersection with $\{0, \, 1\} \times \Omega$.

We are interested in the connected components of $V$ whose intersection with $\{0, \, 1\} \times \Omega$ is nontrivial. If there are none, then $\Phi_0^{-1}(y) = \Phi_1^{-1}(y) = \emptyset$ and the thesis is trivial. 

Assume that $\Gamma$ is a compact connected component of $V$, whose intersection is nonempty. The classification theorem for compact connected $1$-manifolds implies that $\Gamma$ is either homeomorphic to a closed interval or to $S^1$. Since the intersection with the border is nontrivial, it cannot be $S^1$, and so $\Gamma \cong [a, \, b]$ (i.e., it is diffeomorphic to the interval $[a, \, b]$).

\noindent In particular, there exists $\alpha : [a, \, b] \to \Gamma \subseteq [0, \, 1] \times \Omega$ such that $\alpha([a, \, b]) = \Gamma$ and $\alpha$ diffeomorphism (and this implies point \textbf{(d)} as well). Notice that
\begin{equation*} \{ \alpha(a), \, \alpha(b) \} = \Gamma \cap \left( \{ 0, \, 1 \} \times \Omega \right), \end{equation*}
and there are only two possibility: two point belong to $\{0\} \times \Omega$ or to $\{1\} \times \Omega$ (i.e. $\Gamma$ diffeomorphic to a bended interval) or one point in $\{0\} \times \Omega$ and one point in $\{1\} \times \Omega$.
\end{enumerate}  \end{proof}

What happens can be visualized in a simple way using smooth maps between manifolds (with dimension $n+1$ and $n$ respectively). The example in \hyperref[fig:c1]{Figure \ref{fig:c1}} is the standard one in the degree theory of manifolds with no boundary.

\begin{figure}[h]
\centering
\includegraphics[width=12cm, height=8cm]{Images/MTGPA1.png}
\caption{Topological degree modulo $2$}
\label{fig:c1}
\end{figure}

\begin{lemma}\label{lemma:grad2} Let $y \in F\left(I \times \overline{\Omega} \right) \setminus F \left(Z_F \cup \left(I \times \partial \, \Omega \right) \right)$, and assume that $y \notin \Phi_0(Z_{\Phi_0}) \cup \Phi_1(Z_{\Phi_1})$. Then the topological degree is equal modulo $2$, that is,
\begin{equation} \label{eq:grado} \symbol{35} \, \Phi_0^{-1}(y) = \symbol{35} \, \Phi_1^{-1}(y) \quad \text{(mod 2)}. \end{equation}\end{lemma}

\begin{proof} It is a straightforward corollary of \hyperref[lemma:fundmod2]{Lemma \ref{lemma:fundmod2}}. In fact, each connected component $\Gamma$ whose intersection with the boundary is nontrivial is diffeomorphic to a compact interval:
\begin{equation*} \Gamma \cong [a, \, b] \implies \left| \partial \, \Gamma \right| = 2. \end{equation*}
Since each $\Gamma$ contributes with\mbox{}
\begin{enumerate}[label=\textbf{(\alph*)}]
\item two points to $\{0\} \times \Omega$; or
\item two points to $\{1\} \times \Omega$; or
\item one point to $\{0\} \times \Omega$ and one point to $\{1\} \times \Omega$,
\end{enumerate}
then the thesis follows easily, as in \hyperref[fig:c1]{Figure \ref{fig:c1}}.
\end{proof}

\begin{proposition}\label{prop:tg2} Let $\Omega \subset \R^N$ be an open and bounded subset and let $\Phi : \overline{\Omega} \to \R^N$ be a function of class $C^2$. Assume that $y_0, \, y_1 \in \mathcal{C} \subseteq \R^N \setminus \Phi(\partial \, \Omega)$, where $\mathcal{C}$ is a connected component, and assume also that $y_0, \, y_1 \notin \Phi(Z_\Phi)$. Then
\begin{equation*} \symbol{35} \, \Phi^{-1}(y_0) = \symbol{35} \, \Phi^{-1}(y_1) \quad \text{(mod 2)}. \end{equation*}\end{proposition}

\begin{remark} Let $\Phi \in C^2 \left(\overline{\Omega}; \; \R^N \right)$ be a function, and let $y \in \R^N \setminus \Phi(Z_\Phi \cup \partial \, \Omega)$. If we let
\begin{equation*} \Psi(x) := \Phi(x) - y, \end{equation*}
then $\Psi \in C^2 \left(\overline{\Omega}; \; \R^N \right)$, $Z_\Psi = Z_\Phi$, $\Phi^{-1}(y) = \Psi^{-1}(0)$ and $0 \in \R^N \setminus \Psi(Z_\Psi \cup \partial \, \Omega)$.\end{remark}

\begin{proof}We divide the argument into three steps to make it more clear to the reader.

\paragraph{Step 1.} Let $\pi \subset \mathcal{C}$ be a polygonal curve linking $y_0$ and $y_1$. By \hyperref[lemma:sard]{Sard's Lemma} $\Phi(Z_\Phi)$ is a null set, thus we can slightly move the vertexes of $\pi$, if necessary, to coincide with regular points.

In particular we find a sequence of regular points $p_0 := y_0, \, \dots, \, p_s := y_1$ linked through segments, thus we can assume without loss of generality that the segment $y_0 \, y_1$ is entirely contained in $\mathcal{C}$.

\paragraph{Step 2.} Let us set
\begin{equation*} F(t, \, x) := \Phi(x) - \gamma(t), \qquad \text{where} \quad \gamma(t) = (1 - t) \, y_0 + t \, y_1. \end{equation*}
We may always assume that there exists an open bounded interval $I \supset [0, \, 1]$ such that $\gamma(t) \in \mathcal{C}$ for any $t \in \bar{I}$.

Let $\Phi_0(x) := F(0, \, x) = \Phi(x) - y_0$ and $\Phi_1(x) := F(1, \, x) = \Phi(x) - y_1$. The previous Remark proves that $0$ is a regular value for both. Clearly $0 \notin F(\bar{I} \times \partial \, \Omega)$, thus in order to apply \hyperref[lemma:grad2]{Lemma \ref{lemma:grad2}} we need to prove that $0$ is not a critical value for $F$ (i.e., it doesn't belong to $F(Z_F)$).  

\paragraph{Step 3.} In general, however, this is not true. Thus we cannot direct apply the previous lemma. On the other hand, by \hyperref[lemma:sard]{Sard's Lemma} there exists $z$ arbitrarily near to $0$ such that
\begin{equation*} z \notin F \left( Z_F \cup (\bar{I} \times \partial \, \Omega) \right) \end{equation*}
since $F(Z_F)$ is a thin set and $F(\bar{I} \times \partial \, \Omega)$ is compact. For the same reason we can choose $z$ so that it is not a critical value for $\Phi_0$ and $\Phi_1$ and
\begin{equation*}  \symbol{35} \, \Phi_0^{-1}(0) =  \symbol{35} \, \Phi_0^{-1}(z), \qquad \symbol{35} \, \Phi_1^{-1}(0) =  \symbol{35} \, \Phi_1^{-1}(z). \end{equation*}
We can now apply \hyperref[lemma:grad2]{Lemma \ref{lemma:grad2}} to $z$ and it turns out that
\begin{equation*}  \symbol{35} \, \Phi_0^{-1}(0) =  \symbol{35} \, \Phi_0^{-1}(z) = \symbol{35} \, \Phi_1^{-1}(z) =  \symbol{35} \, \Phi_1^{-1}(0), \end{equation*}
which, in turn, implies the thesis:
\begin{equation*} \symbol{35} \, \Phi^{-1}(y_0) = \symbol{35} \, \Phi^{-1}(y_1) \quad \text{(mod 2)}. \end{equation*}
 \end{proof}
 
\begin{definition}[Topological Degree] Let $\Omega \subset \R^N$ be an open and bounded subset, let $\Phi : \overline{\Omega} \to \R^N$ be a function of class $C^2$ and let $y \notin \Phi(\partial \, \Omega)$. The topological degree modulo $2$ is defined as
\begin{equation*} \mathrm{deg}_2\left(y, \, \Phi, \, \overline{\Omega}\right) \stackrel{\mathrm{def}}{=} \begin{cases} 1 & \text{if odd cardinality and  } y \notin \Phi(Z_\Phi) \\ 0 & \text{if even cardinality and  } y \notin \Phi(Z_\Phi) \\ \mathrm{deg}_2\left(\tilde{y}, \, \Phi, \, \overline{\Omega}\right) & y \in \Phi(Z_\Phi), \, \, \tilde{y} \in \mathcal{U}_y \, \text{regular}. \end{cases} \end{equation*} \end{definition}

\begin{remark}\label{rmk:propregls}The topological degree modulo $2$ is a function
\begin{equation*} \mathrm{deg}_2\left(y, \, \Phi, \, \overline{\Omega}\right) : \R^N \setminus \Phi(\partial \, \Omega) \to \{0, \, 1\} \end{equation*} 
continuous and well defined. Moreover, if $y \notin \Phi(\partial \, \Omega)$, then
\begin{equation*} \mathrm{deg}_2\left(y, \, \Phi, \, \overline{\Omega}\right) = 1 \implies y \in \accentset{\circ} \Phi(\Omega).\end{equation*} 
In fact, if $y \notin \Phi(\Omega)$, then $y \notin \Phi \left(\overline{\Omega} \right)$ and this clearly implies that $y$ is a regular value whose topological degree is equal to $0$.

On the other hand, if $y \in \Phi(\Omega)$, then there exists a neighborhood $U \ni y$ such that, for any $y^\prime \in U$, we have $y^\prime \notin \Phi \left( \partial \, \Omega \right)$ and $\mathrm{deg}_2\left(y^\prime, \, \Phi, \, \overline{\Omega}\right) = 1$ (i.e., $y$ is an interior point).\end{remark}

\section{Topological Degree for Continuous Maps}

In this section we denote by $\Omega$ an open subset of $\R^N$ and by $\Phi : \Omega \to \R^N$ a continuous map, unless we need it to be more regular (in which case, we will specify it).

\begin{definition}Let $y \in \R^N$ be any point. The set of all the continuous maps with the property that $y$ does not belong to the image of the border of $\Omega$ is denoted by
\begin{equation*} \mathcal{F}_y = \left\{ \Phi \in C^0 \left(\overline{\Omega}; \; \R^N \right) \: \left| \: y \notin \Phi \left(\partial \, \Omega \right) \right. \right\}. \end{equation*}
Similarly, for any $k \in \N$, the subset of all the functions of class $C^k$ is denoted by
\begin{equation*} \mathcal{F}_{y}^{k} = \left\{ \Phi \in C^0\left(\overline{\Omega}; \; \R^N \right) \: \left| \: y \notin \Phi \left(\partial \, \Omega \right) \right. \right\} \cap C^k \left( \overline{\Omega}; \; \R^N \right). \end{equation*}
\end{definition}

\begin{remark}Clearly $\F_y$ is a subset of $C^0 \left(\overline{\Omega}; \; \R^N \right)$, which is a normed space with the uniform norm. Therefore we can equip $\F_y$ with the topology $\tau$ induced by taking the restriction of the uniform norm. It turns out that $\mathcal{F}_y$ is an open subset of $C^0 \left(\overline{\Omega}, \, \R^N \right)$. \end{remark}

\begin{proof} If $d(y, \, \Phi(\partial \, \Omega)) = \delta > 0$, then
\begin{equation*} B\left(\Phi, \, \delta \right) \subset \mathcal{F}_y.\end{equation*}
Indeed, if $\Psi \in \mathcal{F}_y$ is a function arbitrarily near $\Phi$, that is, a function such that $\| \Phi - \Psi \|_\infty < \delta$, then it follows from the reverse triangular inequality that
\begin{equation*}\left| \Psi(x) - y \right| \geq \left| \left| \Phi(x) - y \right| - \| \Psi - \Phi \|_\infty \right| > \delta - \delta = 0.\end{equation*} \end{proof}

\begin{remark}For any $k \geq 1$, the subset $\mathcal{F}_y^k$ is dense in $\mathcal{F}_y$. \end{remark}

\begin{lemma}Fix $y \in \R^N$. The map
\begin{equation*} \mathcal{F}_y^2 \ni \Phi \mapsto \mathrm{deg}_2 \left(y, \, \Phi, \, \Omega \right) \in \{0, \, 1\}\end{equation*}
is continuous and constant on the connected components of $\mathcal{F}_y^2$. \end{lemma}

\begin{proof}Let $\Phi \in \mathcal{F}_y^2$ be any function, let $\delta = d(y, \, \Phi(\partial \, \Omega)) > 0$ and let $\Psi \in \mathcal{F}_y^2$ be a function such that
\begin{equation*}\|\Phi - \Psi\|_\infty < \delta. \end{equation*}
Let $I$ be an open interval containing $[0, \, 1]$. The map
\begin{equation*} F(t, \, x) := (1 - t) \, \Phi(x) + t \, \Psi(x) : \bar{I}\times\overline{\Omega} \to \R^N \end{equation*}
is well defined, with the additional property that $F(t, \, \cdot) \in \mathcal{F}_y^2$ for any $t \in \bar{I}$. By definition it follows that $y \notin F \left(\bar{I} \times \partial \, \Omega \right)$, but a priori it is not clear if $y$ is either a regular value for $F$, $\Psi$ and $\Phi$ or not.

On the other hand, by \hyperref[lemma:sard]{Sard's Lemma \ref{lemma:sard}} there exists $\tilde{y}$ arbitrarily close to $y$, such that $\tilde{y}$ is regular for $F$, $\Psi$ and $\Phi$. The \hyperref[lemma:grad2]{Homotopy Lemma \ref{lemma:grad2}}, along with the definition of topological degree modulo $2$, implies that
\begin{equation*} \mathrm{deg}_2(y, \, \Phi, \, \Omega) = \mathrm{deg}_2(\tilde{y}, \, \Phi, \, \Omega) = \mathrm{deg}_2(\tilde{y}, \, \Psi, \, \Omega) = \mathrm{deg}_2(y, \, \Psi, \, \Omega), \end{equation*}
which is exactly what we wanted to prove.\end{proof}

\begin{corollary} \label{corololr23} If $\mathcal{C}$ is a connected component of $\mathcal{F}_y$, then $\mathrm{deg}_2 \left(\cdot, \, \Phi, \, \Omega \right)$ is constant on the intersection
\begin{equation*} \mathcal{C} \cap C^2 \left(\overline{\Omega}, \, \R^N \right) = \mathcal{C} \cap \F_y^2. \end{equation*} \end{corollary}

\begin{proof}This is an easy consequence of the fact that $\mathcal{C}\cap C^2\left(\overline{\Omega}, \, \R^N \right)$ is path-connected, and thus it is connected.

To prove that it is path-connected, it suffices to demonstrate that one can find a polygonal path between any two points (with a finite number of vertexes).

A priori we cannot guarantee that the path lies in the intersection (and this is generally false), but using \hyperref[lemma:sard]{Sard's Lemma} we may always assume that, at least the vertexes, are in $\mathcal{C}\cap C^2\left(\overline{\Omega}, \, \R^N \right)$.

If the number of the vertexes gets bigger, it's easier to find segments (entirely contained in the intersection) linking two of them.\end{proof} 

\begin{definition}[Topological Degree Modulo $2$] Let $\Phi \in C^0\left( \overline{\Omega}, \, \R^N \right)$, let $y \in \R^N \setminus \Phi(\partial \, \Omega)$ and let $\mathcal{C}$ be the connected component of $\mathcal{F}_y$ containing $\Phi$. The topological degree modulo $2$ of $\Phi$ is
\begin{equation*} \mathrm{deg}_2(y, \, \Phi, \, \Omega) = \mathrm{deg}_2(y, \, \Psi, \, \Omega),\end{equation*}
where $\Psi \in \mathcal{C} \cap C^2 \left(\overline{\Omega}, \, \R^N \right)$ is any function of class $C^2$ in the same connected component. \end{definition}

\section{Properties of the Topological Degree}

In this section, we state and prove some of the main properties of the topological degree modulo 2, which, in part, explain the reason why we did so much work to introduce this (powerful) tool.

\begin{lemma}[Homotopy Invariance] \label{lemma:ivs} Let $H : [0, \, 1] \times \overline{\Omega} \to \R^N$ be a homotopy, and denote $\Phi(\cdot) := F(0, \, \cdot)$ and $\Psi(\cdot) := F(1, \, \cdot)$. If $y \notin H \left([0, \, 1] \times \partial \, \Omega \right)$, then
\begin{equation*} \mathrm{deg}_2 \left(y, \, \Phi, \, \Omega \right) = \mathrm{deg}_2 \left(y, \, \Psi, \, \Omega \right).\end{equation*} \end{lemma}

\begin{proof}The curve
\begin{equation*} [0, \, 1] \ni t \longmapsto F(t, \, \cdot) \end{equation*}
belongs to $\F_y$, since it does not intersect the image of the border of $\Omega$ as $t$ ranges in $[0, \, 1]$. Besides, it is continuous with respect to the uniform norm, and thus $\Phi$ and $\Psi$ belong to the same connected component of $\mathcal{F}_y$. Finally, \hyperref[corololr23]{Corollary \ref{corololr23}} concludes the proof. \end{proof}

\begin{lemma}[Solution Property] \label{lemma:imagje} Let $\Phi \in C^0 \left(\overline{\Omega}, \, \R^N \right)$ and let $y \in \R^N \setminus \Phi(\partial \, \Omega)$ be any point such that
\begin{equation*} \mathrm{deg}_2(y, \, \Phi, \, \Omega) \neq 0. \end{equation*}
Then the equation $\Phi(x) = y$ admits (at least) one solution, that is, $y \in \Phi(\Omega)$. \end{lemma}

\begin{proof}We argue by contradiction. If $y\notin \Phi(\Omega)$, it turns out that $y \notin \Phi \left(\overline{\Omega} \right)$ and this means that there exists $\rho > 0$ such that
\begin{equation*} B(y, \, \rho) \cap \Phi \left(\overline{\Omega} \right) = \emptyset. \end{equation*}
By density, there exists $\Psi$ of class $C^2 \left(\overline{\Omega}, \, \R^N \right)$ in the same connected component of $\Phi$ such that $\| \Psi - \Phi \|_{\infty} < \rho$. Therefore \hyperref[rmk:propregls]{Remark \ref{rmk:propregls}} concludes the proof.
\end{proof}

\begin{proposition}\label{prop:1239}Let $\Phi, \, \Psi \in C^0 \left(\overline{\Omega}, \, \R^N \right)$ and let $y \in \R^N \setminus \Phi(\partial \, \Omega) = \Psi(\partial \, \Omega)$. Assume also that $\Phi \, \big|_{\partial \Omega} \equiv \Psi \, \big|_{\partial \Omega}$. Then the topological degree modulo $2$ depends only on the value on the border, that is,
\begin{equation*} \mathrm{deg}_2 \left(y, \, \Phi, \, \Omega \right) =  \mathrm{deg}_2 \left(y, \, \Psi, \, \Omega \right). \end{equation*} \end{proposition}

\begin{proof}The reader can easily check that the map
\begin{equation*} H(t, \, x) = t \, \Phi(x) + (1 - t) \, \Psi (x)\end{equation*}
is an homotopy with the additional property that, for any $x \in \partial \Omega$ and any $t \in [0, \, 1]$, $H(t, \, x) = \Phi(x)$.

Thus $y \notin H_t (\partial \, \Omega)$ for any $t \in [0, \, 1]$, and thus we can apply the homotopy invariance property (see \hyperref[lemma:ivs]{Lemma \ref{lemma:ivs}}). \end{proof}

\begin{corollary}Let $\Phi \in C^0 \left(\overline{\Omega}, \, \R^N \right)$ be a function such that $\Phi(x) = x$ for any $x \in \partial \, \Omega$. Then
\begin{equation*}\Omega \subseteq \Phi(\Omega). \end{equation*} \end{corollary}

\begin{theorem}[Brouwer Fixed Point] \label{th:brouwer}Let $K$ be a compact subset of $\R^N$, and assume that $K$ is homeomorphic to the closed ball $D := \overline{B}$. If $\Phi: K \to K$ is continuous, then there exists (at least) a fixed point. \end{theorem}

\begin{proof}Let us consider the map $H(t, \, x) := t \, \Phi(x) - x$. Clearly 
\begin{equation*}0 \notin H \left([0, \, 1) \times \partial \, D \right), \end{equation*}
since $\|x\| = 1$ and $\|t \, \Phi(x)\| = |t| \, \|\Phi(x)\| < 1$, hence it cannot vanish for any $x \in \partial \, D$. Consequently only two alternatives are possible: \mbox{}
\begin{enumerate}[label=\textbf{(\arabic*)}]
\item There exists $x \in \partial \, D$ such that $\Phi(x) = x$, that is, there exists a fixed point on the boundary.
\item There is no point $x \in \partial \, D$ such that $\Phi(x) = x$, that is,
\begin{equation*}0 \notin H([0, \, 1] \times \partial \, B). \end{equation*}
Therefore by \hyperref[lemma:ivs]{Lemma \ref{lemma:ivs}} (homotopy invariance property) we conclude that
\begin{equation*} \mathrm{deg}_2(0, \, \Phi - \mathrm{Id}, \, B) = \mathrm{deg}_2(0,  \,- \mathrm{Id}, \, B) = 1,\end{equation*} 
that is, there exists an internal fixed point.
\end{enumerate} \end{proof}

\begin{theorem} Let $\Phi \in C^0\left(\overline{\Omega}; \; \R^N \right)$ be a continuous map, and let $y_1, \, y_2 \in \mathcal{C} \subset \R^N \setminus \Phi(\partial \, \Omega)$ be points in the same connected component. Then
\begin{equation*} \mathrm{deg}_2\left(y_1, \, \Phi, \, \Omega\right) = \mathrm{deg}_2\left(y_2, \, \Phi, \, \Omega\right).\end{equation*}  \end{theorem}

\begin{proof} Let $H : [0, \, 1] \times \overline{\Omega} \to \R^N$ be the homotopy (it should be checked) defined as
\begin{equation*} H(t, \, x) = \Phi(x) - \gamma(t), \end{equation*}
where $\gamma(\cdot)$ is a continuous function from $[0, \, 1]$ to $\R^N$, such that $\gamma(0) = y_1$ and $\gamma(1) = y_2$. Then $0 \notin H([0, \, 1] \times \partial \, \Omega)$ since $\gamma(t) \notin \Phi(\partial \, \Omega)$ for any $t \in [0, \, 1]$ and thus \hyperref[lemma:ivs]{Lemma \ref{lemma:ivs}} implies that
\begin{equation*} \mathrm{deg}_2\left(0, \, \Phi - y_1, \, \Omega\right) = \mathrm{deg}_2\left(0, \, \Phi - y_2, \, \Omega\right).\end{equation*}  
We conclude by noticing that the topological degree of $\Psi := \Phi - y_1$ at the point $0$ is equal to the topological degree of $\Phi$ at the point $y_1$.\end{proof}

\begin{theorem}[Tietze] \label{tiet}Let $X$ be a metric space and let $F \subset X$ be a closed subset. If $f : F \to \R$ is a continuous function, then there exists an extension of $f$ to the whole space, that is, a function $\tilde{f}: X \to \R$ such that
\begin{equation*} f \equiv \tilde{f} \, \big|_{F}. \end{equation*} \end{theorem}

\begin{example}Let $E = S^1$ and let $D = B^1$. If $f : S^1 \to S^1$ is the identity map, then no extension to a continuous function $\tilde{f} : S^1 \to B^1$ is possible (in fact, there's no retraction by deformation). \end{example}

\begin{definition} Let $\Phi \in C^0\left(\partial \, \Omega; \; \R^N\right)$ be a continuous function. For any $y \in \R^N \setminus \Phi(\partial \, \Omega)$ we define the topological degree modulo $2$ as
\begin{equation*} \mathrm{deg}_2\left(y, \, \Phi, \, \Omega \right) := \mathrm{deg}_2\left(y, \, \Psi, \, \Omega \right), \end{equation*}
where $\Psi \in C^0(\overline{\Omega})$ is a continuous extension of $\Phi$. \end{definition}

\begin{theorem} Let $\Omega \subset \R^N$ be an open bounded subset. Then its border $\partial \, \Omega$ is not contractile in itself, i.e., no homotopy between the identity and a constant map $x_0$ exists. \end{theorem}

\begin{proof} We argue by contradiction. Suppose that such an homotopy $H(t, \, x) : [0, \, 1] \times \partial \, \Omega \to \R^N$ exists and let $y \in \Omega$ be a point such that $y \notin H([0, \, 1] \times \partial \, \Omega)$. The topological degree modulo $2$ of $y$ is well defined and the homotopy invariance property implies that
\begin{equation*} \mathrm{deg}_2\left(y, \, \mathrm{Id}, \, \Omega\right) = 1.\end{equation*}
On the other hand, $y \notin \Phi(\partial \, \Omega)$ and $y \neq x_0$, thus for any $\Psi$ extension of $\Phi$ such that $\Psi(x) = x_0$ for any $x \in \overline{\Omega}$, it turns out that
\begin{equation*} \mathrm{deg}_2\left(y, \, \Psi, \, \Omega\right) = 0\end{equation*} 
and this yields to a contradiction:
\begin{equation*} 0 =  \mathrm{deg}_2\left(y, \, \Phi, \, \Omega\right) = \mathrm{deg}_2\left(y, \, \mathrm{Id}, \, \Omega\right) = 1.\end{equation*}\end{proof}

\section{Applications}

Let $X$ and $Y$ be Banach spaces, and let us denote by $\mathcal{L}(X, \, Y)$ the space of continuous linear forms. Recall that it is a complete space endowed with the operators norm, defined by
\begin{equation*} \| L \|  := \sup_{x \neq 0} \frac{ \|L \, x\| }{ \|x\| } = \sup_{\|x\| \leq 1} \| L \, x \|. \end{equation*}

We are particularly interested in the case $X = Y$ since it is a \textbf{Banach algebra}. Indeed, the composition $\circ$ between linear continuous operators is the product operation in the algebra $\mathcal{L}(X)$, and it satisfies the inequality
\begin{equation*} \| A \circ B \| \leq \|A \|\, \|B\|, \qquad \forall \, A, \, B \in \mathcal{L}(X). \end{equation*}
Let us denote the subspace of invertible continuous linear forms by
\begin{equation*} \mathrm{G}_L(X) := \left\{ L \in \mathcal{L}(X) \: \left| \: \text{$L$ is an isomorphism} \right. \right\}.\end{equation*}

\begin{lemma} Let $A \in \mathrm{G}_L(X)$ and let $B \in \mathcal{L}(X, \, X)$ be linear continuous forms. If
\begin{equation*} \|B-A\| < \frac{1}{\|A^{-1}\|}, \end{equation*}
then also $B$ is invertible (i.e. $B \in \mathrm{G}_L(X)$). \end{lemma}

\begin{proof}Recall that, if $(Y, \, \|\cdot\|_Y)$ is a Banach space, then for any $(y_n)_{n \in \N} \subset Y$ such that
\begin{equation*} \sum_{n \in \N} \|y_n\|_Y < \infty, \end{equation*}
it turns out that
\begin{equation*} \sum_{n \in \N} y_n < \infty. \end{equation*}

\paragraph{Step 1.} We may always assume, without loss of generality, that $A = \mathrm{Id}_X$. Indeed, if
\begin{equation*} \|B-I\| < 1 \implies B \in \mathrm{G}_L(X),\end{equation*}
then, for any isomorphism $A \in \mathrm{G}_L(X)$, it turns out that
\begin{equation*} \| B - A \| < \frac{1}{\|A^{-1}\|} \implies \|A^{-1} B - \mathrm{Id}_X \| < 1 \implies A^{-1}B \in \mathrm{G}_L(X), \end{equation*}
which, in turn, implies that $B \in \mathrm{G}_L(X)$ since $A$ is invertible.

\paragraph{Step 2.} Let us define the series
\begin{equation*} C := \sum_{n \geq 0} \left(\mathrm{Id}_X - B\right)^n.\end{equation*}
It is straightforward to prove that it converges in norm (it is the geometric series of ratio $\| \mathrm{Id}_X - B \| < 1$) , hence it also converges in $\mathcal{L}(X)$. It remains to prove that $C$ is the inverse of $B$, i.e.,
\begin{equation*} BC = CB = \mathrm{Id}_X. \end{equation*}
If we consider the finite sums, it turns out that
\begin{equation*}\left(\mathrm{Id}_X - B \right)\cdot \sum_{n \leq N} \left(\mathrm{Id}_X - B \right)^n - \sum_{n \leq N} \left(\mathrm{Id}_X - B \right)^n = \left(\mathrm{Id}_X - B \right)^N - \mathrm{Id}_X, \end{equation*}
and thus, by passing to the limit as $N \to + \infty$, we obtain
\begin{equation*}\left(\mathrm{Id}_X - B \right)\cdot C - \mathrm{Id}_X \cdot C = - \mathrm{Id}_X \implies BC = \mathrm{Id}_X. \end{equation*}
The opposite relation ($CB = \mathrm{Id}_X$) follows in a similar fashion, thus we leave the details to the reader.
\end{proof}

\begin{theorem} Let $\Phi : \R^N \to \R^N$ be a continuous function such that
\begin{equation}\label{eq:suplim} \limsup_{|x| \to + \infty} \frac{|\Phi(x) - x|}{|x|} < 1. \end{equation}
Then $\Phi$ is a surjective mapping. Moreover, for any linear isomorphism $A \in \mathrm{G}_L(\R^N)$, the same conclusion holds true if
\begin{equation*} \limsup_{|x| \to + \infty} \frac{|\Phi(x) - A \, x|}{|x|} < \frac{1}{\|A^{-1}\|}. \end{equation*} \end{theorem}

\begin{proof}We may always assume, without loss of generality, that $A = \mathrm{Id}_X$ (the same argument we have already used in the previous proof, works also here).

\paragraph{Step 1.} Let $y \in \R^N$ be any point, and let us consider the homotopy
\begin{equation*} H : [0, \, 1] \times \R^N \to \R^N, \qquad (t, \, x) \mapsto x + t \, \left(\Phi(x) - x \right). \end{equation*}
We can easily estimate its absolute value, i.e.,
\begin{equation*}\left| H(t, \, x) \right| \geq \|x\| - \|\Phi(x) - x\| = \|x\| \, \left(1 - \frac{\|\Phi(x) - x\|}{\|x\|}\right), \end{equation*}
and thus, by the assumption on the superior limit \eqref{eq:suplim}, we infer that there exist $\delta \in (0, \, 1)$ and $r > 0$ such that $\|x\| \geq r$ implies $\left| H(t, \, x) \right| \geq (1 - \delta) \, r$.

\paragraph{Step 2.} Let us take $r$ such that
\begin{equation*}r > \frac{\|y\|}{1 - \delta}, \end{equation*}
so that the homotopy may be restricted to $H : [0, \, 1] \times \overline{B(0, \, r)} \to \R^N$, with the additional property $y \notin H([0, \, 1] \times \partial \, \overline{B})$. The \hyperref[lemma:ivs]{Homotopy Invariance Lemma \ref{lemma:ivs}} concludes the proof since
\begin{equation*} \mathrm{deg}_2\left(y, \, \Phi, \, B(0, \, r)\right) = \mathrm{deg}_2\left(y, \, \mathrm{Id}_X, \, B(0, \, r)\right) = 1.\end{equation*}\end{proof}

\begin{exercise} Let $\Phi : \R^N \to \R^N$ be a continuous function such that
\begin{equation*} \lim_{|x| \to + \infty} |\Phi(x)| = + \infty, \end{equation*}
and let $A : \R^N \to \R^N$ be a linear isomorphism. If
\begin{equation*} \inf_{x \in \R^N} \left\{ (\Phi(x), \, Ax) \right\} < \infty, \end{equation*}
then $\Phi$ is a surjective mapping. \end{exercise}

\paragraph{Fixed-Point Theorems.} Recall that, if $H : [0, \, 1] \times S^n \to S^n$ is an homotopy such that $H(0, \, x) = x$ for any $x \in S^n$, then $H(1, \, \cdot)$ is a surjective map. Surprisingly a similar result holds, under few assumptions, for a generic set $\Omega \subset \R^N$.

\begin{lemma}\label{lemma:dkoskd} Let $\Omega \subset \R^N$ be an open bounded subset such that $\partial \, \Omega = \partial \, \overline{\Omega}$. If $H : [0, \, 1] \times \partial \, \Omega \to \partial \, \Omega$ is a homotopy such that $H(0, \, x) = x$ for any $x \in \partial \, \Omega$, then $H(1, \, \cdot)$ is a surjective map. \end{lemma}

\begin{remark}This assertion is not true if we drop the hypothesis on the boundary of $\Omega$, even in dimension $2$. Indeed, let $B^1(0, \, 1) \subset \R^2$ be the unitary ball, and let
\begin{equation*} s = \{ (x, \, 0) \in \R^2 \: \left| \: 0 \leq x \leq 1 \right. \} \end{equation*}
be a segment. If we set $\Omega = B^1(0, \, 1) \setminus s$, then the boundary is given by
\begin{equation*}\partial \, \Omega = S^1 \cup s\end{equation*}
and it is straightforward to prove that $s$ can be retracted to a point with $H$, in such a way that $ \partial \, \Omega \mapsto S^1$. Consequently $H(1, \, \cdot)$ is equal to a constant on the whole segment $s$, and thus it cannot be surjective (see \hyperref[fig:dkkds]{Figure \ref{fig:dkkds}}). \end{remark} 

\begin{figure}[h]
\centering
\includegraphics[width=8cm, height=6cm]{Images/MTDG1.png}
\caption{Counterexample to \hyperref[lemma:dkoskd]{Lemma \ref{lemma:dkoskd}}}
\label{fig:dkkds}
\end{figure}

\begin{proof}Assume by contradiction that there exists $y_0 \in \partial \, \Omega$ such that $y_0 \notin H(1, \, \partial \, \Omega)$. The homotopy $H$ is continuous and the boundary $\partial \, \Omega$ is compact, therefore $H(1, \, \partial \, \Omega)$ is compact, and thus there exist a positive $\delta$ and a ball $B(y_0, \, \delta)$ such that
\begin{equation*} y^\prime \in B(y_0, \, \delta) \cap \partial \, \Omega \implies y^\prime \notin H(1, \, \partial \, \Omega). \end{equation*}
The assumption on the boundary of $\Omega$ allows us to find two points $y_1, \, y_2 \in B(y_0, \, \delta)$ such that $y_1 \in \Omega$ and $y_2 \notin \overline{\Omega}$ (see \hyperref[fig:2dsads]{Figure \ref{fig:2dsads}}). Therefore $y_i \notin H(t, \, \partial \, \Omega)$ for any $t \in [0, \, 1]$ and thus the topological degree modulo $2$ is well defined; in particular, it turns out that
\begin{equation*} \begin{cases} \mathrm{deg}_2\left(y_1, \, H(1, \, \cdot), \, \Omega \right) = \mathrm{deg}_2\left(y_1, \, H(0, \, \cdot), \, \Omega \right) = 1 \\ \mathrm{deg}_2\left(y_2, \, H(1, \, \cdot), \, \Omega \right) = \mathrm{deg}_2\left(y_2, \, H(0, \, \cdot), \, \Omega \right) = 0. \end{cases} \end{equation*}
On the other hand, the ball $B(y_0, \, \delta)$ is contained in a connected component of $\R^N \setminus H(1, \, \partial \, \Omega)$ and therefore the topological degree should be constant. \end{proof}

\begin{figure}[t]
\centering
\includegraphics[width=8cm, height=6cm]{Images/OGKE.png}
\caption{Idea of the Proof}
\label{fig:2dsads}
\end{figure}

\begin{remark}If $H$ is an infinite-dimensional space, the fixed-point theorem does not hold true. Indeed, suppose that $H$ is an Hilbert space and define the map
\begin{equation*} \Phi : \bar{B}(0, \, 1) \longrightarrow \bar{B}(0, \, 1), \qquad \sqrt{1 - \|x\|^2} \, e_0 + \sum_{i = 0}^{+\infty} \langle x, \, e_i \rangle \, e_{i+1}. \end{equation*}
It is well defined since $\|\Phi(x) \| = 1$ for any $x \in  \bar{B}(0, \, 1)$, and it does not have any fixed point.

To prove the latter assertion, it suffices to argue by contradiction that $\Phi(x) = x$, and then notice that
\begin{equation*} x_0 = \sqrt{1 - x_0^2}, \qquad x_i = 0 \quad \forall \, i \in \N. \end{equation*}\end{remark}

\begin{theorem}[Schauder Fixed-Point] Let $K$ be a closed convex subset of a Banach reflexive space $B$, and let $\Phi : K \to K$ be a compact continuous operator. Then there exists a fixed point $x \in K$, that is, a point such that $\Phi(x) = x$. \end{theorem}

\begin{lemma}Let $K \subset \R^N$ be a convex bounded closed subset and let $\Phi : K \to K$ be a continuous map. Then $\Phi$ admits a fixed point. \end{lemma}

\begin{proof}Let $P : \R^N \to K$ be the projection and let $P_B$ be its restriction to any closed ball containing $K$. The map $\Phi \circ P_B : \overline{B} \to \overline{B}$ admits a fixed point (by \hyperref[th:brouwer]{Brouwer Fixed Point Theorem \ref{th:brouwer}}) $x \in \overline{B}$, therefore
\begin{equation*} \Phi(P(x)) = x \implies x \in K \implies P(x) = x \implies \Phi(x) = x. \end{equation*} \end{proof}

\begin{theorem} Let $K \subset \R^N$ be a compact subset contractible in itself. Assume that there exists $U \supset K$ such that $K$ is a deformation retracted of $U$ (that is, there exists a retraction $r : U \to K$). Then any $\Phi : K \to K$ admits a fixed point. \end{theorem}

\begin{proof}Let $H : [0, \, 1] \times K \to K$ be the homotopy between the identity and the constant map $x_0$, and let $\tilde{H}(t, \, x) := - H\left(t, \, \Phi \circ r(x) \right) + x$.

We may always assume without loss of generality that $r : \bar{U} \to K$, therefore $\tilde{H} : [0, \, 1] \times U \to \R^N$ is a well defined map such that $\tilde{H}(0, \, x) = x - \Phi(r(x))$ and $\tilde{H}(1, \, x) = x - x_0$.

Clearly $0 \notin \tilde{H}(t, \, \partial \, U)$, since $x$ belongs to $\partial \, U$ and $H(t, \, \Phi(r(x)))$ belongs to $K$ and they do not intersect in any point. Therefore the topological degree is well defined and it turns out that
\begin{equation*} \mathrm{deg}_2\left(0, \, \tilde{H}(1, \, \cdot), \, U \right) = \mathrm{deg}_2\left(0, \, \tilde{H}(0, \, \cdot), \, U \right) = 1.\end{equation*}
Hence there exists $x \in U$ such that
\begin{equation*} x = \Phi (r(x)) \implies x \in K \implies r(x) = x \implies x = \Phi(x). \end{equation*}\end{proof}


\chapter{Brouwer Topological Degree}

\begin{scolium} Let $\mathrm{id}: S^n \to S^n$ be the identity and let $\imath: S^n \to S^n$ be the antipodal map, both defined on the $n$-dimensional sphere. Is there an \textbf{homotopy} $H : [0, \, 1] \times S^n \to S^n$ such that
\begin{equation*} H(0, \, x) = x \qquad \text{and}\qquad H(1, \, x) = - x \end{equation*}
for every $x \in S^n$?

\vspace{1.8mm}
\noindent The answer is \textbf{yes} if the dimension is even and \textbf{no} if the dimension is odd. Unfortunately the topological degree modulo $2$ is not enough to prove our claims (since it cannot distinguish $1$ from $-1$), thus we need to introduce a more powerful tool, that is, the Brouwer topological degree. \end{scolium}

\section{Representation and Orientation}

In this section we denote by $X$ and $Y$ real vector spaces of dimensions $N$ and $M$ respectively.

\begin{definition}[Representation] Let $\mathcal{B} := \{b_1, \, \dots, \, b_N\}$ be a basis for $X$. A \textit{representation} of $X$ on $\R^N$ (with respect to the basis $\mathcal{B}$) is the isomorphism $r : X \to \R^N$ such that
\begin{equation*} x = \sum_{i = 1}^{N} x_i \, b_i \implies r(x) = (x_1, \, \dots, \, x_N). \end{equation*} \end{definition}

Let $\left(\mathcal{B}, \, r\right)$ be a representation of $X$ and let $\left(\mathcal{B}^\prime, \, s\right)$ be a representation of $Y$. The matrix associated to a linear application $L \in \mathcal{L}(X, \, Y)$ is defined as follows:
\begin{equation*} A_L := s^{-1} \circ L \circ r. \end{equation*}

\begin{remark}Let $L \in \mathcal{L}(X, \, X)$ be a linear application. The determinant of the associated matrix $A_L$ does not depend on the particular choice of representations of $X$.\end{remark}

\begin{proof} Indeed, if we let $r$ and $r^\prime$ be representation of $X$, it turns out that
\begin{equation*}A_L = r^{-1} \circ L \circ r \implies A_L = r^{-1} \circ r^\prime \circ A_L^\prime \circ \left(r^\prime\right)^{-1} \circ r, \end{equation*}
and the conclusion follows from Binet's theorem\footnote{Let $A, \, B \in M\left(n, \, \mathbb{K}\right)$ be two square matrices. Then $\mathrm{det} \left(A \cdot B \right) = \mathrm{det} A \cdot \mathrm{det} B$.}.  \end{proof}

\begin{definition}[Determinant] Let $L \in \mathcal{L}(X, \, X)$ be a linear application. The \textit{determinant} of $L$ is the determinant of the associated matrix $A_L$, for any choice of the representations.\end{definition}

\begin{definition}[Orientation] Let $L \in \mathcal{L}(X, \, X)$ be a linear isomorphism. We say that $L$ is  \textit{orientation-preserving} if $\mathrm{det}(L) > 0$, and \textit{orientation-reversing} otherwise.\end{definition}

\begin{definition} Let $\mathcal{B}$ and $\mathcal{B}^\prime$ be bases of $X$. We say that they have the same orientation if the transition matrix $A_{\mathrm{id}_X}$ has positive determinant.\end{definition}

\begin{remark}\label{rmk:cf}If $(\mathcal{B}_t)_{t \in [0, \,1]}$ is a continuous family of basis for $X$, then $\mathcal{B}_0$ and $\mathcal{B}_1$ have the same orientation.\end{remark}

\begin{remark}The automorphisms group is made up of two connected components: positively-oriented basis and negatively-oriented basis.\end{remark}

\begin{exercise} Let $X$ be a $N$-dimensional vector space and let $r$ and $r^\prime$ be two representations (of two different bases). If $A$ is an isomorphism and $s$ represents a basis oriented as $r$, then 
\begin{equation*} \mathrm{sgn} \left[ \mathrm{det} \, A_{\mathcal{B}, \, \mathcal{B}^\prime} \right] = \mathrm{sgn} \left[ \mathrm{det} \, A_{\mathcal{B}^{\prime\prime}, \, \mathcal{B}^\prime} \right]. \end{equation*} \end{exercise}

\begin{proof} By definition
\begin{equation*} A_{\mathcal{B}, \, \mathcal{B}^\prime}  = r^\prime \circ \left(r^\prime\right)^{-1} \circ A_{\mathcal{B}^{\prime\prime}, \, \mathcal{B}^\prime} \circ r^{\prime\prime} \circ r^{-1} = A_{\mathcal{B}^{\prime\prime}, \, \mathcal{B}^\prime} \circ (r^{\prime\prime} \circ r^{-1}) \end{equation*}
and the thesis follows from the fact that $\mathrm{det} \, (r^{\prime\prime} \circ r^{-1})  > 0$ along with the Binet theorem.\end{proof}

\section{The Brouwer Degree}

\paragraph{Degree for Differential Mappings.} Let $\Omega \subset \R^N$ be a bounded open subset, let $\Phi \in C^1( \overline{\Omega}, \, \R^N)$ and let $y \in \R^N \setminus \Phi \left( \partial \, \Omega \cup Z_\Phi \right)$.  The primary goal of this chapter is to study the sum
\begin{equation} \sum_{x \in f^{-1}(y)} \mathrm{sgn} \, J(x). \end{equation}

\begin{lemma}[Dog on Leash] \label{lemmagui} Let $v : [a, \, b] \to \R \times \R^N$ be a path of class $C^2\left([a, \, b]\right)$ such that
\begin{equation*} v(\tau) \neq 0 \qquad \forall \, \tau \in [a, \, b].\end{equation*}
Assume that $\mathcal{B}_a := \{ v(a), \, z_1, \, \dots, \, z_N \}$ is a basis of $\R^{N+1} = \R \times \R^N$. There are differentiable paths $\beta_1, \, \dots, \, \beta_N : [a, \, b] \to \R \times \R^{N}$ such that $\beta_i(a) = z_i$ and
\begin{equation*} \mathcal{B}_\tau := \left\{ v(\tau), \, \beta_1(\tau), \, \dots, \, \beta_N(\tau) \right\} \end{equation*}
is also a basis of $\R^{N+1}$ for every $\tau \in [a, \, b]$.\end{lemma}

\begin{proof}Let us set
\begin{equation*} F(\tau, \, x) := \left \langle x, \, \frac{v(\tau)}{|v(\tau)|^2} \right\rangle \, v^\prime(\tau), \qquad (\tau, \, x) \in [a, \, b] \times \R^{N+1}, \end{equation*}
and let us consider the associated differential equation
\begin{equation} \label{eq:edo1} u^\prime(\tau) = F(\tau, u). \end{equation}
The initial path $v$ is clearly a solution of the differential equation. Moreover, for any given initial condition $x \in \R^{N + 1}$, by Lipschitz-Cauchy theorem the initial value problem \eqref{eq:edo1} admits one and only one solution: we denote it by $S(\tau, \, x) = u_x(\tau)$.

For any fixed $\tau$, the map $S(\tau, \, \cdot) : \R^{N + 1} \to \R^{N + 1}$ is an isomorphism and, more precisely, $S(0, \, \cdot)$ is the identity. Therefore, it suffices to set
\begin{equation*}\beta_j(\tau) := S(\tau, \, z_j), \end{equation*}
and the thesis follows easily from what we have already proved above. \end{proof}

\begin{remark} The isomorphism $S(\tau, \, \cdot)$ introduced in the proof is, actually, an isometry for every $\tau \in [a, \, b]$. Indeed, the reader may prove as an exercise that
\begin{equation*} \langle F(\tau, \, x), \, y \rangle = - \langle F(\tau, \, y), \, x \rangle, \qquad \forall \, x, \, y \in \R^N\end{equation*}
for any fixed $\tau \in [a, \, b]$.

In particular, if the first basis is \textit{orthonormal}, then the intermediate bases are orthonormal and so is the final basis. \end{remark}

\begin{lemma} \label{lemma:imspd} Let $I \supset [0, \, 1]$ be an open interval of $\R$, let $F : I \times \Omega \to \R^N$ be a $C^1$ map and let $\alpha : [a, \, b] \to I \times \Omega$ be a curve of class $C^1$ such that
\begin{equation*} \alpha^\prime(\tau) \neq 0, \qquad F\circ \alpha(\tau) = c \quad \text{and} \quad \alpha(\tau) \notin Z_F, \qquad \forall \, \tau \in I. \end{equation*}
Let $z_1, \, \dots, \, z_N \in \R \times \R^{N}$ be vectors such that $\mathcal{B}_a := \{ \alpha^\prime(a), \, z_1, \, \dots, \, z_N \}$ is a basis of $\R^{N+1}$. \mbox{}
\begin{enumerate}[label=\textbf{(\alph*)}]
\item There are vectors $z_1^\prime, \, \dots, \, z_N^\prime \in \R \times \R^N$ such that $\mathcal{B}_b := \{ \alpha^\prime(b), \, z_1^\prime, \, \dots, \, z_N^\prime \}$ is a basis, with the same orientation as $\mathcal{B}_a$.
\item If $F_1, \, \dots, \, F_N$ are the components of $F$ with respect to the canonical base, then the determinant of the $N\times N$-minor
\begin{equation*} \left( F_i^\prime(\alpha(a)) (z_j) \right)_{i, \, j = 1, \, \dots, \, N}  \end{equation*}
is nonzero, and it has the same sign as of the determinant of the $N\times N$-minor
\begin{equation*} \left( F_i^\prime(\alpha(b)) (z_j^\prime) \right)_{i, \, j = 1, \, \dots, \, N}.\end{equation*}
\item If $\alpha^\prime(b) \notin \{0\} \times \R^N$, then we may always choose the $z_i^\prime$'s in $\{0\} \times \R^N$.
\end{enumerate} \end{lemma}

\begin{proof} \mbox{}
\begin{enumerate}[label=\textbf{(\alph*)}]
\item If we set $v(\tau) := \alpha^\prime(\tau)$, then it follows from \hyperref[lemmagui]{Lemma \ref{lemmagui}} that there are $N$  differentiable paths $\beta_1, \, \dots, \, \beta_N : [a, \, b] \to \R \times \R^N$ such that $\beta_i(a) = z_i$ and, for every $\tau \in [a, \, b]$,
\begin{equation*}\left\{ \alpha^\prime(\tau), \, \beta_1(\tau), \, \dots, \, \beta_N(\tau) \right\}\end{equation*}
is a basis of $\R \times \R^N$. Finally, \hyperref[rmk:cf]{Remark \ref{rmk:cf}} proves that the orientations need to be the same.
\item Since $\alpha(\tau)$ does not belong to $Z_F$, the differential at $\alpha(\tau)$ is surjective and thus the matrix
\begin{equation*} \begin{pmatrix} F_1^\prime(\alpha(\tau))(\alpha^\prime(\tau)) & F_1^\prime(\alpha(\tau))(\beta_1(\tau)) & \dots & F_1^\prime(\alpha(\tau))(\beta_N(\tau)) \\ \vdots & \vdots & \vdots & \vdots \\ \vdots & \vdots & \vdots & \vdots \\ F_N^\prime(\alpha(\tau))(\alpha^\prime(\tau)) & F_N^\prime(\alpha(\tau))(\beta_1(\tau)) & \dots & F_N^\prime(\alpha(\tau))(\beta_N(\tau)) \end{pmatrix} \end{equation*}
has maximal rank (that is, $N$). By assumption $F$ is constant along the path $\alpha$, hence the first column of the matrix is identically zero, that is, the $N \times N$-minor is uniquely determined.
\item Let $Q : \R \times \R^N \to \{0\} \times \R^N$ be the projection with kernel
\begin{equation*} \mathrm{Ker}(Q) = \mathrm{Span} \left< \alpha^\prime(b) \right>. \end{equation*}
By assumption, the vectors $z_1^\prime = Q(\beta_1(b)), \, \dots, \, z_N^\prime = Q(\beta_N(b))$ are also a basis of $\R^N$ in $\{0\} \times \R^N$, hence is suffices to prove that the bases
\begin{equation*} \left\{ \alpha^\prime(a), \, z_1, \, \dots, \, z_n \right\}, \qquad \left\{ \alpha^\prime(b), \, \beta_1(b), \, \dots, \, \beta_N(b) \right\} \quad \text{and} \quad \left\{ \alpha^\prime(b), \, z_1^\prime, \, \dots, \, z_N^\prime \right\}  \end{equation*}
have all the same orientation.

Indeed, the middle and the latter ones are linked via the continuous family of bases
\begin{equation*} \left\{ \alpha^\prime(b), \, \beta_1^\ast(\tau), \, \dots, \, \beta_N^\ast(\tau) \right\}, \qquad\beta_i^\ast(\tau) := (1 - \tau) \, \beta_i(b) + \tau \, z_i^\prime. \end{equation*}
Finally, it remains to prove that the determinant of the $N \times N$-minor
\begin{equation*} \left( F_i^\prime(\alpha(a)) (\beta_j(b)) \right)_{i, \, j = 1, \, \dots, \, N}  \end{equation*}
is nonzero, and that it has the same sign as of the determinant of the $N\times N$-minor
\begin{equation*} \left( F_i^\prime(\alpha(b)) (z_j^\prime) \right)_{i, \, j = 1, \, \dots, \, N}.\end{equation*}
Since $\alpha(b) \notin Z_F$ it turns out that the differential at $\alpha(b)$ is surjective, hence the matrix
\begin{equation*} \begin{pmatrix} F_1^\prime(\alpha(b))(\alpha^\prime(b)) & F_1^\prime(\alpha(b))(\beta_1^\ast(\tau)) & \dots & F_1^\prime(\alpha(b))(\beta_N^\ast(\tau)) \\ \vdots & \vdots & \vdots & \vdots \\ \vdots & \vdots & \vdots & \vdots \\ F_N^\prime(\alpha(b))(\alpha^\prime(b)) & F_N^\prime(\alpha(b))(\beta_1^\ast(\tau)) & \dots & F_N^\prime(\alpha(b))(\beta_N^\ast(\tau)) \end{pmatrix}. \end{equation*}
has maximal rank, i.e., $N$. By assumption $F$ is constant along the path $\alpha$, thus the first column is zero and the thesis follows easily.
\end{enumerate}
\end{proof}


\paragraph{Fundamental Lemma.} Let $\Omega \subset \R^N$ be an open set, let $I \subset \R$ be an open interval such that $0, \, 1 \in I$ and let $F \in C^2 \left(I \times \overline{\Omega}; \; \R^N \right)$. Set
\begin{equation*} \Phi_0 (x) := F(0, \, x) \qquad \text{and} \qquad \Phi_1(x) := F(1, \, x), \end{equation*}
and let $y \in \R^N$ be a point such that $y \notin F \left(I \times \partial \, \Omega \cup Z_F \right)$ and $y \notin \Phi_0 \left( Z_{\Phi_0} \right) \cup \Phi_1 \left( Z_{\Phi_1} \right)$.

\begin{lemma} \label{fund} In this setting, it turns out that
\begin{equation*} \sum_{x \in \Phi_0^{-1}(y)} \sgn \left( J_{\Phi_0}(x) \right) = \sum_{x \in \Phi_1^{-1}(y)} \sgn \left( J_{\Phi_1}(x) \right), \end{equation*} 
where $J_{\Phi_i}(x)$ denotes the determinant of the Jacobian of $\Phi_i$ at $x \in \Omega$.\end{lemma}

\begin{proof} As a result of \hyperref[lemma:fundmod2]{Lemma \ref{lemma:fundmod2}}, it turns out that
\begin{equation*} Y := F^{-1}(y) \cap \left( [0, \, 1] \times \partial \, \Omega \right) \end{equation*}
is a smooth $1$-manifold, with a finite number of connected components. Let $\Gamma$ be one of them, and assume that it intersects $\{0, \, 1\} \times \Omega$; then $\Gamma$ is diffeomorphic to an interval, that is, there exists a diffeomorphism
\begin{equation*} \gamma : [a, \, b] \xrightarrow{\sim} [0, \, 1]\end{equation*}
such that $\gamma \left([a, \, b]\right) = \Gamma$, $\gamma^\prime(\tau) \neq 0$ for any $\tau \in [a, \, b]$ and
\begin{equation*} \Gamma \cap \left(\{0, \, 1\} \times \Omega \right) = \left\{ \alpha(a), \, \alpha(b) \right\}.\end{equation*}
There are two possible scenarios to discuss: \mbox{} %
\begin{enumerate}[label=\textbf{(\arabic*)}]
\item If $\alpha(a) = (0, \, x_a)$ and $\alpha(b) = (0, \, x_b)$, then we expect to find that
\begin{equation*}\sgn \left( J_{\Phi_0}(x_a) \right) + \sgn \left( J_{\Phi_0}(x_b) \right) = 0. \end{equation*}
Let us set $z_1 := (0, \, e_1), \, \dots, \, z_N := (0, \, e_N)$. Since $\gamma^\prime(\tau)$ is non-vanishing in $[a, \, b]$, and it does not intersect $\{0\} \times \Omega$ (see \hyperref[lemma:fundmod2]{Lemma \ref{lemma:fundmod2}}), then
\begin{equation*} \mathcal{B} = \left\{ \gamma^\prime(a), \, z_1, \, \dots, \, z_N \right\} \end{equation*}
is a basis of $\R \times \R^N$. By \hyperref[lemma:imspd]{Lemma \ref{lemma:imspd}} there are vectors $z_1^\prime, \, \dots, \, z_N^\prime \in \{0\} \times \Omega$ such that
\begin{equation*} \mathcal{B}^\prime = \left\{ \gamma^\prime(b), \, z_1^\prime, \, \dots, \, z_N^\prime \right\} \end{equation*}
is also a basis of $\R \times \R^N$, with the same orientation as $\mathcal{B}$, such that the determinant of
\begin{equation*} \left( F_i^\prime(\gamma(a)) (z_j) \right)_{i, \, j = 1, \, \dots, \, N}  \end{equation*}
is nonzero and it has the same sign of the determinant of
\begin{equation*} \left( F_i^\prime(\gamma(b)) (z_j^\prime) \right)_{i, \, j = 1, \, \dots, \, N}.\end{equation*}
Since $F_i^\prime(\gamma(a) (z_j) = \Phi_{0, \, i}^\prime (x_a) (e_j)$, it turns out that
\begin{equation*} \mathrm{det} \, \left[ F_i^\prime(\gamma(a)) (z_j) \right]_{i, \, j = 1, \, \dots, \, N} = J_{\Phi_0}(x_a).  \end{equation*}
If we set $z_i^\prime := (0, \, e_i^\prime)$ for $i = 1, \, \dots, \, N$, then it is easy to prove that $F_i^\prime(\gamma(b) (z_j^\prime) = \Phi_{0, \, i}^\prime (x_b) (e_j^\prime)$, and thus
\begin{equation*} \mathrm{det} \, \left[ F_i^\prime(\gamma(b)) (z_j^\prime) \right]_{i, \, j = 1, \, \dots, \, N} = \pm \, J_{\Phi_1}(x_b).  \end{equation*}
To find the right sign, recall that $\mathcal{B}$ and $\mathcal{B}^\prime$ have the same orientation. On the other hand, since $\alpha^\prime(a)$ and $\alpha^\prime(b)$ have opposite orientations, then $\{z_1, \, \dots, \, z_N\}$ and $\{ z_1^\prime, \, \dots, \, z_N^\prime\}$ necessarily have the opposite orientations, and thus
\begin{equation*}\sgn \, J_{\Phi_0}(x_a) =\sgn \, \mathrm{det} \, \left[ F_i^\prime(\gamma(b)) (z_j^\prime) \right]_{i, \, j = 1, \, \dots, \, N} = - \sgn \, J_{\Phi_1}(x_b).  \end{equation*}

\item If $\alpha(a) = (0, \, x_a)$ and $\alpha(b) = (1, \, x_b)$, then we expect to find that
\begin{equation*}\sgn \left( J_{\Phi_0}(x_a) \right) = \sgn \left( J_{\Phi_1}(x_b) \right). \end{equation*}
The proof of this identity is exactly as the previous one, so we will not give any detail. The sole difference is the fact that here $\alpha^\prime(a)$ and $\alpha^\prime(b)$ have the same orientation.
\end{enumerate} \end{proof}

\begin{definition}[Topological Degree] Let $\Omega \subset \R^N$ be an open and bounded subset, let $\Phi : \overline{\Omega} \to \R^N$ be a function of class $C^2$ and let $y \notin \Phi \left(\partial \, \Omega \cup Z_\Phi \right)$. The topological degree is defined by setting
\begin{equation*} \mathrm{deg}\left(y, \, \Phi, \, \Omega\right) := \sum_{x \in F^{-1}(y)}  \sgn \left( J_{\Phi}(x) \right). \end{equation*} \end{definition}

\begin{proposition}\label{prop:tg0} Let $\Omega \subset \R^N$ be an open and bounded subset, and let $\Phi : \overline{\Omega} \to \R^N$ be a function of class $C^2(\Omega)$. Assume that $y_0, \, y_1 \in \mathcal{C} \subseteq \R^N \setminus \Phi(\partial \, \Omega)$, where $\mathcal{C}$ is a connected component, and assume also that $y_0, \, y_1 \notin \Phi \left(Z_\Phi \right)$. Then
\begin{equation*} \mathrm{deg}\left(y_0, \, \Phi, \, \Omega\right) =  \mathrm{deg}\left(y_1, \, \Phi, \, \Omega\right). \end{equation*} 
\end{proposition}

\begin{remark}As a result of \hyperref[prop:tg0]{Proposition \ref{prop:tg0}} and \hyperref[lemma:sard]{Sard's Lemma}, the topological degree is a well defined notion for every $y \in \R^N \setminus \Phi(\partial \, \Omega)$. Moreover, if $\Phi \in C^2 \left( \overline{\Omega}; \; \R^N \right)$, then the map
\begin{equation*}\R^N \setminus \Phi (\partial \, \Omega) \ni y \longmapsto \mathrm{deg}\left(y, \, \Phi, \, \Omega\right) \in \Z \end{equation*}
is continuous, and thus constant on connected components.\end{remark}

\begin{definition}Let $y \in \R^N$. We define the sets of continuous mappings such that $y$ doesn't belong to the image of the border of $\Omega$ as
\begin{equation*} \mathcal{F}_y = \left\{ \Phi \in C^0\left(\overline{\Omega}, \, \R^N \right) \: \left| \: y \notin \Phi(\partial \, \Omega) \right. \right\} \end{equation*}
equipped with the uniform norm. Also, for any $k \in \N$, we define
\begin{equation*} \mathcal{F}_{y}^{k} = \left\{ \Phi \in C^0 \left(\overline{\Omega}, \, \R^N \right) \: \left| \: y \notin \Phi(\partial \, \Omega) \right. \right\} \cap C^k \left( \overline{\Omega}, \, \R^N \right). \end{equation*}
\end{definition}

\begin{remark}With the topology of subspaces (induced by the restriction of the uniform norm), it turns out that $\mathcal{F}_y$ is an open subset of $C^0\left(\overline{\Omega}, \, \R^N \right)$. Moreover, for any $k \geq 1$, it turns out that $\mathcal{F}_y^k$ is dense in $\mathcal{F}_y$. \end{remark}

\begin{lemma}The map
\begin{equation*} \mathcal{F}_y^2 \ni \Phi \longmapsto \mathrm{deg}_2(y, \, \Phi, \, \Omega) \in \Z\end{equation*}
is continuous, and thus constant on the connected component of $\mathcal{F}_y$. \end{lemma}

\begin{definition}[Topological Degree] Let $\Phi \in C^0\left( \overline{\Omega}, \, \R^N \right)$, let $y \in \R^N \setminus \Phi(\partial \, \Omega)$ and let $\mathcal{C}$ be the connected component of $\mathcal{F}_y$ containing $\Phi$. The topological degree of $\Phi$ is defined as follows:
\begin{equation*} \mathrm{deg}(y, \, \Phi, \, \Omega) = \mathrm{deg}(y, \, \Psi, \, \Omega), \qquad \Psi \in \mathcal{C} \cap C^2 \left(\overline{\Omega}, \, \R^N \right). \end{equation*} \end{definition}

\paragraph{Main Properties of the Topological Degree.} Here we give a list of the main properties of the topological degree; since the proofs are similar to the ones we have already written for the modulo $2$ degree, we won't give any.

\begin{lemma}[Homotopy Invariance] \label{lemma:ivsjj}Let $H : [0, \, 1] \times \overline{\Omega} \to \R^N$ be a continuous homotopy and let $\Phi(\cdot) = F(0, \, \cdot)$ and $\Psi(\cdot) = F(1, \, \cdot)$. If $y \notin H \left([0, \, 1] \times \partial \, \Omega \right)$, then
\begin{equation*} \mathrm{deg}(y, \, \Phi, \, \Omega) = \mathrm{deg}(y, \, \Psi, \, \Omega).\end{equation*} \end{lemma}

\begin{lemma} \label{lemma:imagjejj}Let $\Phi \in C^0\left(\overline{\Omega}, \, \R^) \right)$ and let $y \in \R^N \setminus \Phi(\partial \, \Omega)$ be any point such that $\mathrm{deg}_(y, \, \Phi, \, \Omega) \neq 0$. Then $y \in \Phi(\Omega)$. \end{lemma}

\begin{lemma}[Solution Property]\label{prop:1239}Let $\Phi, \, \Psi \in C^0 \left(\overline{\Omega}, \, \R^N \right)$ and let $y \in \R^N \setminus \Phi(\partial \, \Omega) = \Psi(\partial \, \Omega)$. Assume also that $\Phi \, \big|_{\partial \Omega} \equiv \Psi \, \big|_{\partial \Omega}$. Then the topological degree depends only on the value on the border, that is,
\begin{equation*} \mathrm{deg} \left(y, \, \Phi, \, \Omega \right) =  \mathrm{deg} \left(y, \, \Psi, \, \Omega \right). \end{equation*} \end{lemma}

\section{Applications}

The motivating example to develop the topological degree theory was the following problem:

\begin{scolium} Let $\mathrm{id}: S^{N-1} \to S^{N-1}$ be the identity and let $\imath: S^{N-1} \to S^{N-1}$ be the antipodal map both defined on the $(N-1)$-sphere. Is there an \textbf{homotopy} $H : [0, \, 1] \times S^{N-1} \to S^{N - 1}$ such that
\begin{equation*} H(0, \, x) = x \qquad \text{and}\qquad H(1, \, x) = - x \end{equation*}
for any $x \in S^{N-1}$? \end{scolium}

\noindent We are now ready to give a negative answer to this question, provided that $N$ is an odd number, as a consequence of a way more general theorem.

\begin{theorem} \label{fundamentaltheorem} Let $\Omega \subset \R^N$ be an open bounded subset of $\R^N$, and let $H : [0, \, 1] \times \partial \, \Omega \to \partial \, \Omega$ be a homotopy between 
\begin{equation*} \mathrm{id}_{ \partial \, \Omega } :\partial \, \Omega \ni x \mapsto x \in \partial \, \Omega \quad \text{and} \quad \imath \, \big|_{\partial \, \Omega} : \partial \, \Omega \ni x \mapsto -x \in \partial \, \Omega. \end{equation*}
If $N$ is odd, then 
\begin{equation*} \Omega \cup \left(- \Omega \right) \subset \mathrm{Ran}(H). \end{equation*} \end{theorem}

\begin{proof}We argue by contradiction. Suppose that there is a point $p \in \Omega$ such that $p \notin \mathrm{Ran}(H)$. Then the topological degree of $H(0, \, \cdot)$, computed at the point $p$, is clearly equal to $1$ since $p$ is a point of $\Omega$, and $H(0, \, \cdot)$ is the identity map.

The homotopy invariance property (\hyperref[lemma:ivsjj]{Lemma \ref{lemma:ivsjj}}) implies that also the degree of $H(1, \, \cdot)$, computed at $p$, needs to be equal to $1$. On the other hand, a simple argument proves that
\begin{equation*} \mathrm{deg} \left(y, \, H(1, \, \cdot), \, \Omega \right) = \begin{cases} (-1)^N & \text{if $-p \in \Omega$} \\ 0 & \text{if $-p \notin \Omega$}, \end{cases} \end{equation*}
and this is a contradiction since we assumed $N$ to be odd. The case $p \in - \Omega$ is exactly the same, hence we will not work out the details.
\end{proof}

Note that this assertion is \textbf{false} if $N$ is an even number. A simple counterexample is given by the circumference in the plane $\R^2$.

Indeed, if we denote by $R_\theta$ the counterclockwise rotation of angle $\theta$ around the origin, then it turns out that the map
\begin{equation*} H : [0, \, 1] \times S^1 \to S^1, \qquad (t, \, p) \longmapsto R_{t \cdot \pi}(p) \end{equation*}
is a homotopy between the identity $R_0 = \mathrm{id}_{S^1}$, and the antipodal map $R_{\pi} = \imath \, \big|_{S^1}$.

\vspace{1.6mm}
On the other hand, the theorem we have just proved easily implies that the answer to the motivating \textit{problem} is negative if $N$ is odd. More precisely, any homotopy between the identity and the antipodal map necessarily covers the whole ball, hence $\mathrm{Ran}(H) \not \subset S^{N - 1}$, that is, the homotopy cannot achieve the result without trespassing the boundary.

\begin{corollary}[Eigenvalue] Let $\Phi : S^{N-1} \to \R^N$ be a continuous mapping and assume that $N$ is odd. There exist $x \in S^{N-1}$ and $\lambda \in \R$ such that
\begin{equation*} \Phi(x) = \lambda \cdot x. \end{equation*} \end{corollary}

\begin{proof}The map defined by setting
\begin{equation*} H(t, \, x) := \cos(t) \, \Phi(x) + \sin(t) \, x, \qquad t \in \left[ - \frac{\pi}{2}, \, \frac{\pi}{2} \right] \end{equation*}
is a homotopy between the identity and the antipodal map on the sphere. Since $N$ is odd \hyperref[fundamentaltheorem]{Theorem \ref{fundamentaltheorem}} implies that there exists a point $(t, \, x) \in  \left[ - \frac{\pi}{2}, \, \frac{\pi}{2} \right] \times S^{N-1}$ such that $H(t, \, x) = 0$, that is,
\begin{equation*} \cos(t) \, \Phi(x) + \sin(t) \, x = 0. \end{equation*}
Clearly $\cos(t)$ cannot be equal to zero, otherwise $x$ would be also equal to zero and this is impossible (since $x$ is a point of the sphere). Therefore, if we divide by $\cos(t)$, it turns out that
\begin{equation*}\Phi(x) = - \frac{\sin (t)}{\cos(t)} \cdot x \implies \lambda := -\frac{\sin (t)}{\cos(t)}. \end{equation*}
\end{proof}

The above corollary proves a much stronger result (assuming $N$ odd): \textit{one cannot comb a hairy ball flat without creating a cowlick}. More precisely, there is no nonvanishing continuous tangent vector field on the $(N-1)$-spheres for any $N$ odd.

Moreover, the eigenvalue property can be generalized to different subsets of $\R^N$, as we prove in the next theorem.

\begin{theorem} Let $\Omega \subset \R^N$ be a bounded open set containing the origin, and let $\Phi : \partial \, \Omega \to \R^N$ be a continuous map. If $N$ is odd, then there are $x \in \partial \, \Omega$ and $\lambda \in \R$ such that $\Phi(x) = \lambda \cdot x$. \end{theorem}

\begin{proof}The map defined by setting
\begin{equation*} H(t, \, x) := \cos(t) \, \Phi(x) + \sin(t) \, x, \qquad t \in \left[ - \frac{\pi}{2}, \, \frac{\pi}{2} \right] \end{equation*}
is a homotopy between the identity and the antipodal map on the boundary of $\Omega$. Since $0$ belongs to $\Omega$ the topological degree of the identity at $0$ is equal to one. For the same reason, the topological degree of the antipodal map is $(-1)^N = -1$.

It follows from \hyperref[fundamentaltheorem]{Theorem \ref{fundamentaltheorem}} that there exists a point $(t, \, x) \in \left[ - \frac{\pi}{2}, \, \frac{\pi}{2} \right] \times \partial \, \Omega$ such that $H(t, \, x) = 0$, i.e.,
\begin{equation*} \cos(t) \, \Phi(x) + \sin(t) \, x = 0, \end{equation*}
and the thesis follows easily as in the previous proof.
\end{proof}

\begin{theorem}[Fixed Point] Let $\Phi : S^{N-1} \to S^{N-1}$ be a continuous map homotopic to the identity. If $N$ is odd, then there is $x \in S^{N-1}$ such that $\Phi(x) = x$. \end{theorem}

\begin{proof} The map defined by setting
\begin{equation*} H(t, \, x) : [0, \, 1] \times S^{N-1} \to S^{N-1}, \qquad (t, \, x) \longmapsto (1 - t) \cdot \Phi(x) - t \cdot x \end{equation*} 
is a homotopy between $\Phi$ and the antipodal map.

The topological degree, at the origin, of the map $\Phi$ is equal to $1$ (since it is homotopic to the identity), while the topological degree, at the origin, of the antipodal map is equal to $(-1)^N = -1$.

It follows from \hyperref[fundamentaltheorem]{Theorem \ref{fundamentaltheorem}} that there exists a point $(t, \, x) \in \left[ 0, \, 1\right] \times S^{N-1}$ such that $H(t, \, x) = 0$, i.e.,
\begin{equation*}(1 - t) \cdot \Phi(x) - t \cdot x = 0.  \end{equation*}
Clearly $t$ cannot be equal to $1$, otherwise $x$ would be zero which is not a point of the sphere. Hence, it turns out that
\begin{equation*}\Phi(x) = \frac{t}{1 - t} \cdot x.  \end{equation*}
The multiplicative constant $\frac{t}{1 - t}$ is positive and its absolute value needs to be equal to $1$ (the point belongs to the sphere). In conclusion, it turns out that $t = 1/2$ is the unique possible value and
\begin{equation*}\Phi(x) = x.  \end{equation*}\end{proof}

\section{More Properties of the Topological Degree}

Let $\Omega \subset \R^N$ be a bounded and open set and let $\Phi : \overline{\Omega} \to \R^N$ be a continuous map.

\begin{theorem}[Locality and Additivity] \label{locality} Let $\Omega_1, \, \dots, \, \Omega_k$ be disjoint open subsets of $\Omega$, and let $y \in \R^N$ be a point such that
\begin{equation*} y \notin \Phi \left( \overline{\Omega} \setminus \bigcup_{i = 1}^{k} \Omega_i \right). \end{equation*}
Then the topological degree is additive, that is,
\begin{equation*} \mathrm{deg} \left(y, \, \Phi, \, \Omega \right) = \sum_{i = 1}^{k}  \mathrm{deg} \left(y, \, \Phi, \, \Omega_i \right). \end{equation*} \end{theorem}

\begin{proof}Let us set 
\begin{equation*} K_\Phi := \Phi \left( \overline{\Omega} \setminus \bigcup_{i = 1}^{k} \Omega_i \right). \end{equation*}
Since $K_\Phi$ is the difference between a compact set and a finite union of open subsets it is still compact, hence there exists $\rho > 0$ such that
\begin{equation*} B(y, \, 2 \, \rho) \cap K_\Phi = \emptyset. \end{equation*}
By density, there exists $\psi \in C^2\left( \overline{\Omega} \right)$ such that $\| \Psi - \Phi \|_\infty < \rho$ and the above property is preserved, that is,
\begin{equation*} B(y, \, 2 \, \rho) \cap K_\Psi = \emptyset. \end{equation*}
The ball $B(y, \, \rho)$ is connected, hence for any $y^\prime \in B(y, \, \rho)$ the topological degree is equal to the one computed at $y$, i.e.,
\begin{equation*}\begin{aligned} & \mathrm{deg} \left(y, \, \Phi, \, \Omega \right) = \mathrm{deg} \left(y^\prime, \, \Phi, \, \Omega \right), \\ & \mathrm{deg} \left(y, \, \Phi, \, \Omega_i \right) = \mathrm{deg} \left(y^\prime, \, \Phi, \, \Omega_i \right), \\ & \mathrm{deg} \left(y, \, \Psi, \, \Omega \right) = \mathrm{deg} \left(y^\prime, \, \Psi, \, \Omega \right), \\ & \mathrm{deg} \left(y, \, \Psi, \, \Omega_i \right) = \mathrm{deg} \left(y^\prime, \, \Psi, \, \Omega_i \right). \end{aligned} \end{equation*}
On the other hand, by \hyperref[lemma:sard]{Sard's Lemma} there exists $y^\prime \in B(y, \, \rho) \setminus \Psi(Z_\Psi)$. The additive property holds true for a regular value, that is,
\begin{equation*} \mathrm{deg} \left(y^\prime, \, \Psi, \, \Omega \right) = \sum_{i = 1}^{k}  \mathrm{deg} \left(y^\prime, \, \Psi, \, \Omega_i \right),\end{equation*}
and this relation concludes the proof since the topological degree is also locally constant with respect to the function.
\end{proof}

\begin{exercise} Let $\Psi : \R^N \to \R^N$ be a continuous map such that
\begin{equation*} \lim_{|x| \to + \infty} \left| \Psi(x) \right| = + \infty. \end{equation*}
Then for any $y \in \R^N$ there exists $R> 0$ such that
\begin{equation*} \mathrm{deg} \left(y, \, \Psi, \, B_r \right) =  \mathrm{deg} \left(y, \, \Psi, \, B_R \right) =  \mathrm{deg} \left(0, \, \Psi, \, B_r \right), \qquad \forall \, r \geq R. \end{equation*} \end{exercise}

\begin{proof}[\textbf{Solution}] The assumption on the limit of $\Psi$ implies that, for any $y \in \R^N$ (uniformly with respect to the norm of $y$), there exists $R > 0$ such that
\begin{equation*} \left| \Psi(x) \right| > |y|, \qquad \forall \, x \in \R^N \: : \:  |x| \geq R. \end{equation*}
In particular, the fiber $\Psi^{-1}(y)$ is contained in the ball of radius $R$ and, by the previous \hyperref[locality]{Theorem \ref{locality}}, it follows that
\begin{equation*} \mathrm{deg} \left(y, \, \Psi, \, B_r \right) =  \mathrm{deg} \left(y, \, \Psi, \, B_R \right) , \qquad \forall \, r \geq R. \end{equation*}
The last equality follows from the fact that
\begin{equation*}\Psi \left( \partial \, B(0, \, R) \right) \cap B\left(0, \, |y| \right) = \emptyset, \end{equation*}
since the topological degree depends only on the values on the border.
\end{proof}

\begin{exercise} Let $\Phi, \, \Psi : \R^N \to \R^N$ be continuous maps such that
\begin{equation*} \lim_{|x| \to + \infty} \left| \Psi(x) \right| = + \infty \qquad \text{and} \qquad \lim_{R \to + \infty} \mathrm{deg} \left(0, \, \Psi, \, B(0, \, R) \right) \neq 0.\end{equation*}
If $\Phi - \Psi$ is a bounded function, then $\Phi$ is surjective.\end{exercise}

\begin{proof}[\textbf{Solution}] Let us consider the following homotopy between the two maps:
\begin{equation*} H(t, \, x) := \Psi(x) + t \cdot \left( \Phi(x) - \Psi(x) \right), \qquad t \in [0, \, 1]. \end{equation*}
We can easily give an estimate a priori to the module of the homotopy:
\begin{equation*}\left| H(t, \, x) \right| \geq \left|\Psi(x) \right| - \left| \Phi(x) - \Psi(x) \right| \geq \left| \Psi(x) \right| - \| \Phi - \Psi \|_\infty. \end{equation*}
Since $\Psi$ goes to $+ \infty$ and the $\infty$-norm of the difference is bounded, it turns out that for any $y \in \R^N$ there is $R > 0$ such that
\begin{equation*}\left| H(t, \, x) \right| \geq |y|, \qquad \forall \, x \in \R^N \: : \: |x| = R. \end{equation*}
Therefore $H$ is an actual homotopy between $\Phi$ and $\Psi$ such that $y \notin H\left(t, \, \partial \, B(0, \, R) \right)$ for any $t \in [0, \, 1]$ and, by the homotopy invariance property, we infer that
\begin{equation*} \mathrm{deg} \left(y, \, \Psi, \, B_R \right) =  \mathrm{deg} \left(y, \, \Phi, \, B_R \right). \end{equation*}
By taking the limit as $R \to + \infty$, we conclude that $\mathrm{deg} \left(y, \, \Phi, \, B_R \right) \neq 0$ for any $y \in \R^N$ and thus $\Phi$ is surjective.
\end{proof}

\begin{exercise} Let $P : \C \to \C$ the application defined by a complex-valued polynomial of degree equal to $n \geq 1$. For any $w \in \C$ there exists $R > 0$ such that
\begin{equation*} P^{-1}(w) \subset B(0, \, R), \end{equation*}
and also it turns out that
\begin{equation*} \mathrm{deg} \left( w, \, P, \, B_r \right) = n, \qquad \forall \, r \geq R. \end{equation*}\end{exercise}
\chapter{Borsuk Theorem}

\section{Introduction}

In this brief chapter, we want to prove the Borsuk theorem, the Borsuk-Ulam theorem and then show some of the primary applications of these results.

\begin{theorem}[Borsuk] \label{b} Let $\Omega \subset \R^N$ be an open bounded subset and suppose that $0 \in \Omega$ and $\Omega$ symmetric with respect to the origin. Let $\Phi : \overline{\Omega} \to \R^N$ be a continuous and odd map such that $0 \notin \Phi(\partial \, \Omega)$. Then it turns out that
\begin{equation*} \text{$\mathrm{deg}(0, \, \Phi, \, \Omega)$ is odd.} \end{equation*}\end{theorem}

\begin{theorem}[Borsuk-Ulam] \label{b_u} Let $\Omega \subset \R^N$ be an open bounded subset and suppose that $0 \in \Omega$ and $\Omega$ symmetric with respect to the origin. If $X$ is a subspace of dimension strictly less than $N$ and $\Phi : \partial \, \Omega \to X$ is continuous, then there exists $x \in \partial \, \Omega$ such that
\begin{equation*} \Phi(x) = \Phi(-x). \end{equation*} \end{theorem}

We now prove that the Borsuk theorem implies the Borsuk-Ulam theorem. In the next section we introduce the technical results needed for the Borsuk theorem, and then we prove it.

\begin{proof} We argue by contradiction. Suppose that there is no point $x \in \partial \, \Omega$ such that $\Phi(x) = \Phi(-x)$, and let
\begin{equation*} \Phi_d(x) := \Phi(x) - \Phi(-x) : \partial \, \Omega \to X \end{equation*}
be the odd part of $\Phi$. Since $0 \notin \Phi_d(\partial \, \Omega)$ the \hyperref[b]{Borsuk Theorem \ref{b}} immediately implies that $\mathrm{deg}(0, \, \Phi, \, \Omega)$ is odd and, in particular, different from zero.

By \hyperref[lemma:imagjejj]{Lemma \ref{lemma:imagjejj}} it turns out that $0$ is an internal point of $\Phi_d\left(\overline{\Omega} \right)$ - for a suitable extension $\Phi$ - and this is absurd since the image of $\Phi_d$ is contained in a subspace $X$ of dimension strictly less than $N$, i.e., the image is contained in a set whose internal part is empty.\end{proof}

\section{Proof of Borsuk Theorem}

In the first part of this section, we state and prove three technical lemmas which can be combined to give a relatively straightforward proof of the Borsuk theorem.

\begin{lemma} \label{lemmab1} Let $K \subset \R^M$ be a compact set, let $\Phi : K \to \R^N$ be a continuous map and assume that $N > M$ and $y_0 \in \R^N \setminus \Phi(K)$. Then there exists $\Psi : \R^M \to \R^N$ continuous such that
\begin{equation*} \Psi \, \big|_{K} \equiv \Phi \quad \text{and} \quad y_0 \notin \Psi(\R^M). \end{equation*}
\end{lemma}

\begin{remark}The assumption on the dimensions $N > M$ is necessary. Indeed, it suffices to consider the identity map between a compact set contained in a ring of $\R^N$ and $\R^N$ itself (see \hyperref[fig:c111]{Figure \ref{fig:c111}}). \end{remark} 

\begin{figure}[h]
\centering
\includegraphics[width=12cm, height=6cm]{Images/MTDP1.png}
\caption{Counterexample}
\label{fig:c111}
\end{figure}

\begin{proof}This requires a rather technical argument. Therefore we divide it into many small steps to make it easier to understand.

\paragraph{Step 1.} Since $K$ is compact, the image $\Phi(K)$ is also compact, and thus it is closed in $\R^N$. Therefore there exists $\rho > 0$ such that
\begin{equation*} B(y_0, \, \rho) \cap \Phi(K) = \emptyset,\end{equation*}
and there exists an extension $\tilde{\Phi} : \R^M \to \R^N$ such that $\tilde{\Phi} \, \big|_K \equiv \Phi$ (but it is not the sought extension since, a priori, it may contain $y_0$ in the image).

\paragraph{Step 2.} Let $0 < \epsilon < \frac{\rho}{2}$ and let $\tilde{\Psi} \in C^\infty \left(\R^M; \; \R^N \right)$ be a function arbitrarily near to $\tilde{\Phi}$, that is, $\| \tilde{\Phi} - \tilde{\Psi} \|_{\infty} \leq \epsilon$. It follows that
\begin{equation*} \tilde{\Psi}(K) \cap B(y_0, \, \rho - \epsilon) = \emptyset, \end{equation*}
and by \hyperref[lemma:sard]{Sard's Lemma} the image of $\tilde{\Psi}$ is a null-set in $\R^N$ (since the differential sends $\R^M$ to $\R^N$, thus it cannot be surjective). Consequently there is a point $\bar{y}$, arbitrarily close to $y_0$, such that it does not belong to the image of $\tilde{\Psi}$ (while $y_0$ may still be there).

\paragraph{Step 3.} There is a retraction
\begin{equation*} r : \R^N \setminus \{\bar{y}\} \to \R^N \setminus B(y_0, \, \rho - \epsilon), \end{equation*}
so that we can consider the composition with $\tilde{\Psi}$
\begin{equation*} r \circ \tilde{\Psi} : \R^M \to \R^N \setminus B(y_0, \, \rho - \epsilon), \end{equation*}
which is clearly well defined. We conclude the proof by defining
\begin{equation*} \Psi := r \circ \tilde{\Psi} + \left( \tilde{\Phi} - \tilde{\Psi} \right) \end{equation*}
which is the sought function since it satisfies the following properties: \mbox{}
\begin{enumerate}[label = \textbf{(\arabic*)}]
\item $\Psi \, \big|_K = \tilde{\Phi} \, \big|_K = \Phi$;
\item $\Psi(\R^M) \subseteq U_\epsilon \left( r \circ \tilde{\Psi}(\R^M) \right) \implies \Psi(\R^M) \cap B(y_0, \, \rho - 2 \, \epsilon) = \emptyset$.
\end{enumerate}
\end{proof}

\begin{lemma} \label{lemmab2} Let $K$ and $K_0$ be compact sets in $\R^M$, both symmetrical with respect to the origin and such that $K \subseteq K_0 \not \ni 0$. Let $\Phi : K \to \R^N$ be a continuous odd function such that $0 \notin \Phi(K)$, and let $M < N$. Then there exists $\Psi : K_0 \to \R^M$ continuous odd function such that
\begin{equation*} \Psi \, \big|_{K} \equiv \Phi \quad \text{and} \quad 0 \notin \Psi(K_0). \end{equation*}
\end{lemma}

\begin{proof}We argue by induction on the dimension of the starting space , that is, $M = 1, \, \dots, \, N - 1$.

\paragraph{Base Step.} Suppose that $M = 1$. The function $\Phi : K \to \R^N$ is continuous and odd, thus by \hyperref[lemmab1]{Lemma \ref{lemmab1}} there exists a continuous function $\tilde{\Phi} : \R \to \R^N$ which extends $\Phi$ such that $0 \notin \tilde{\Phi}(\R)$. If we set
\begin{equation*} \Psi(x) := \begin{cases} \tilde{\Phi}(x) & \text{if $x \in K_0$ and $x > 0$} \\ -\tilde{\Phi}(-x) & \text{if $x \in K_0$ and $x < 0$} \end{cases}\end{equation*}
then it is easy to prove that $\Psi$ has the sought properties.

\paragraph{Inductive Step.} Suppose that the thesis is true for $M - 1$, and let $M \leq N - 1$. There exists a function  $\tilde{\Phi} : K_0 \cap \R^{M - 1} \to \R^N$ such that $\tilde{\Phi}$ is continuous, odd, and
\begin{equation*} \tilde{\Phi} \, \big|_{K \cap \R^{M - 1}} \equiv \Phi \, \big|_{K \cap \R^{M-1}} \quad \text{and} \quad 0 \notin \tilde{\Phi}\left(K_0 \cap \R^{M - 1} \right). \end{equation*}
The function
\begin{equation*} \tilde{\Psi}(x) := \begin{cases} \tilde{\Phi}(x) & \text{if $x \in K_0 \cap \R^{M - 1}$} \\ \Phi(x) & \text{if $x \in K$}, \end{cases}\end{equation*}
is continuous, odd and well-defined. We can now extend $\tilde{\Psi}$ to $\R^M$ in such a way that $0 \notin \tilde{\Psi}(\R^M)$ (as a consequence of \hyperref[lemmab1]{Lemma \ref{lemmab1}}). We conclude by setting
\begin{equation*} \Psi(x) := \begin{cases} \tilde{\Psi}(x) & \text{if $x \in K_0$ and $x_M > 0$} \\ -\tilde{\Psi}(-x) & \text{if $x \in K_0$ and $x_M < 0$}, \end{cases}\end{equation*}
which is clearly the function with the required properties.
\end{proof}

\begin{remark} Let $\Omega \subset \R^N$ be an open bounded subset, and let $\Phi \in C^0 \left( \Omega; \; \R^N \right)$ be a continuous map. If $A$ is a linear isomorphism of $\R^N$ and $y \notin \Phi(\partial \, \Omega)$, it turns out that 
\begin{equation*} \mathrm{deg} \left( A \,y, \, A \circ \Phi, \, \Omega \right) = \mathrm{sgn} \left( \mathrm{det}(A) \right) \cdot \mathrm{deg}(y, \, \Phi, \, \Omega). \end{equation*}
If $W$ is an open set of $\R^N$ such that $\Omega = A \, W$, then
\begin{equation*} \mathrm{deg} \left( y, \, \Phi, \, \Omega \right) = \mathrm{sgn} \left( \mathrm{det}(A) \right) \cdot \mathrm{deg}(y, \, \Phi \circ A, \, W). \end{equation*}\end{remark}
 
\begin{lemma} \label{lemmab3} Let $\Omega \subset \R^N$ be an open bounded subset and suppose that $0 \notin \overline{\Omega}$ and $\Omega$ symmetric with respect to the origin. Let $\Phi : \overline{\Omega} \to \R^n$ be a continuous and odd map such that $0 \notin \Phi(\partial \, \Omega)$. Then it turns out that
\begin{equation*} \text{$\mathrm{deg}(0, \, \Phi, \, \Omega)$ is even.} \end{equation*}
\end{lemma}

\begin{proof} Let $\R^{N - 1} = \left\{ x \in \R^N \: \left| \: x_N = 0 \right. \right\}$, and let us set
\begin{equation*} K := \R^{N - 1} \cap \partial \, \Omega \quad \text{and} \quad K_0 := \R^{N - 1} \cap \overline{\Omega}. \end{equation*}
By \hyperref[lemmab2]{Lemma \ref{lemmab2}} there exists $\tilde{\Phi} : K_0 \to \R^N$ such that $\tilde{\Phi} \, \big|_K \equiv \Phi$ and $0 \notin \tilde{\Phi}(K_0)$. Therefore, there exists a continuous map, defined on $\partial \, \Omega \cup \left( \overline{\Omega} \cap \R^{N - 1} \right)$, which coincides with $\Phi$ on $\partial \, \Omega$ such that the origin does not belong to its image.

By \hyperref[tiet]{Tietze Extension Theorem \ref{tiet}} there exists $\tilde{\Psi} : \R^N \to \R^N$ continuous, which extends the previous map. Let us set $\Psi := \tilde{\Psi}_d \, \big|_{\overline{\Omega}}$ (i.e., the odd part) and observe that it is odd and
\begin{equation*}\Psi \, \big|_{\partial \, \Omega} \equiv \Phi \, \big|_{\partial \, \Omega}, \qquad \Psi \, \big|_{K_0} \equiv \tilde{\Phi} \, \big|_{K_0}. \end{equation*}
Let us consider
\begin{equation*} \Omega^+ := \left\{ x \in \Omega \: \left| \: x_N > 0 \right. \right\} \quad \text{and} \quad \Omega^- := \left\{ x \in \Omega \: \left| \: x_N < 0 \right. \right\}, \end{equation*}
and notice that $0 \notin \Psi( \partial \, \Omega \cup \partial \, \Omega^+ \cup \partial \, \Omega^{-} )$. If we set $\Psi^+$ and $\Psi^-$ to be the restrictions to $\overline{\Omega}^{\pm}$, it turns out
\begin{equation*} \mathrm{deg}(0, \, \Phi, \, \Omega) = \mathrm{deg}(0, \, \Psi, \, \Omega) = \mathrm{deg}(0, \, \Psi^+, \, \Omega^+) + \mathrm{deg}(0, \, \Psi^-, \, \Omega^-), \end{equation*}
and this concludes the proof as a consequence of the previous Remark (using the isomorphism $\alpha(x) := -x$). Indeed, the two topological degrees are equal and thus $ \mathrm{deg}(0, \, \Phi, \, \Omega)$ is even.
\end{proof}

We are finally ready to prove the Borsuk theorem, using the technical lemmas above.

\begin{proof}By assumption there exists $\rho > 0$ such that 
\begin{equation*} \overline{B(0, \, \rho)} \subseteq \Omega. \end{equation*}
Let $\Psi : \overline{\Omega} \to \R^N$ be a continuous odd map, such that
\begin{equation*} \Psi \, \big|_{\partial \, \Omega} \equiv \Phi \qquad \text{and} \qquad \Psi \, \big|_{\overline{B(0, \, \rho)}} \equiv \mathrm{id}_{\overline{B(0, \, \rho)}}. \end{equation*}
By the additivity property of the topological degree (see \hyperref[locality]{Theorem \ref{locality}}), it turns out that
\begin{equation*}\mathrm{deg} \left(0, \, \Phi, \, \Omega \right) = \mathrm{deg} \left(0, \, \Psi, \, \Omega \right) = \mathrm{deg} \left(0, \, \Psi, \, \Omega \setminus \overline{B(0, \, \rho)} \right) + \mathrm{deg} \left(0, \, \Psi, \, B(0, \, \rho) \right), \end{equation*}
and by \hyperref[lemmab3]{Lemma \ref{lemmab3}} the first addendum is even, while the second is equal to $1$ (because it is the identity, and $0$ belongs to $B(0, \, \rho)$), hence the total topological degree is odd.
\end{proof}

\section{Applications of Borsuk Theorem}

\begin{corollary}Let $\Phi : S^{N-1} \to S^{N-1}$ be an odd function. Then $\Phi$ is surjective. \end{corollary}

\begin{proof}Argue by contradiction. Suppose that $\Phi$ is not surjective, and let $y_0 \in S \setminus \Phi(S)$ be a point not in the image (the sphere is symmetric, so $- y_0 \in S \setminus \Phi(s)$).

\vspace{2mm}
\noindent The stereographic projection
\begin{equation*} P : S^{N - 1} \setminus\{y_0, \, - y_0\} \to S^{N-2}\end{equation*}
is and odd function. Therefore its composition with $\Phi$, that is, $P \circ \Phi : S^{N - 1} \to S^{N - 2}$, is odd. By \hyperref[b_u]{Borsuk-Ulam Theorem \ref{b_u}} it turns out that
\begin{equation*} P \circ \Phi(x) = P \circ \Phi(-x) = - P \circ \Phi(x) \implies P \circ \Phi(x) = 0, \end{equation*} 
and this is absurd since $P \circ \Phi(x)$ is an element of the sphere. \end{proof}

\begin{corollary} Let $\Omega \subseteq \R^N$ be an open bounded subset, which is symmetrical with respect to the origin, and let $\Phi : \partial \, \Omega \to \partial \, \Omega$ be an odd continuous function. Assume also that: \mbox{}
\begin{enumerate}[label=\textbf{(\arabic*)}]
\item The boundary is regular, i.e., $\partial \, \Omega = \partial \, \overline{\Omega}$.
\item The origin is contained in $\Omega$.
\item $\Omega$ is connected.
\item The complement of $\Omega$ is connected.
\end{enumerate}
Prove that, under these assumptions, $\Phi$ is a surjective map. Discuss the validity of the conclusion when one of the above assumption is not satisfied.\end{corollary}

\begin{proof} We argue by contradiction. Suppose that there exists $y \in \partial \, \Omega$ such that $y \notin \Phi \left( \partial \, \Omega \right)$. 

The border of $\Omega$ is compact, hence the image through $\Phi$ is closed. In particular, there is $\rho > 0$ such that
\begin{equation*}B(y_0, \, \rho) \cap \Phi \left( \partial \, \Omega \right) = \emptyset, \end{equation*}
and the assumption on the boundary allows us to find two points $y_1 \in B(y_0, \, \rho) \cap \Omega$ and $y_2 \in B(y_0, \, \rho) \setminus \Omega$. The ball $B(y_0, \, \rho)$ is connected and it does not intersect the image of the boundary (by construction), therefore
\begin{equation*} \mathrm{deg}\left(y_1, \, \Phi, \, \Omega \right) = \mathrm{deg}\left(y_2, \, \Phi, \, \Omega \right). \end{equation*}
By the Borsuk Theorem \ref{b}, the topological degree $\mathrm{deg}\left(0, \, \Phi, \, \Omega \right)$ is nonzero (since it is odd) and, by the connectedness of $\Omega$, it turns out that
\begin{equation*}\mathrm{deg}\left(0, \, \Phi, \, \Omega \right) = \mathrm{deg}\left(y_1, \, \Phi, \, \Omega \right) \neq 0 \implies \mathrm{deg}\left(y_2, \, \Phi, \, \Omega \right) \neq 0. \end{equation*}
We are now ready to derive the contradiction. The complement of $\Omega$ is connected, therefore the topological degree of $\Phi$ in the point $y_2$ is necessarily $0$ (since it does not belong to the image).

\vspace{2.5mm}
The proof of the corollary is complete, but we want to discuss briefly the assumptions and show that there is a counterexample if only $3$ out of $4$ hold true (at least for $N \geq 3$). \mbox{}
\begin{enumerate}[label=\textbf{(\arabic*)}]
\item Suppose that \textbf{(a)} is the only false assumption. Then Figure \ref{fig:cex1} shows a counterexample in both the plane and in the space.
\item Suppose that \textbf{(b)} is the only false assumption. Then Figure \ref{fig:cex2} shows a counterexample in the space (since in the plane this assumption is implied by the others).
\item Suppose that \textbf{(c)} is the only false assumption. Then Figure \ref{fig:cex3} shows a counterexample in both the plane and in the space.
\item Suppose that \textbf{(d)} is the only false assumption. Then Figure \ref{fig:cex4} shows a counterexample in both the plane and in the space.
\end{enumerate}
\end{proof}

\begin{figure}[p]
\centering
\includegraphics[width=12cm, height=8cm]{Images/Cex1.png}
\label{fig:cex1}
\caption{Counterexample \textbf{(a)}}
\end{figure}

\begin{figure}[p]
\centering
\includegraphics[width=12cm, height=8cm]{Images/Cex2.png}
\label{fig:cex2}
\caption{Counterexample\textbf{(b)}}
\end{figure}

\begin{figure}[p]
\centering
\includegraphics[width=12cm, height=8cm]{Images/Cex3.png}
\label{fig:cex3}
\caption{Counterexample \textbf{(c)}}
\end{figure}

\begin{figure}[p]
\centering
\includegraphics[width=12cm, height=8cm]{Images/Cex4.png}
\label{fig:cex4}
\caption{Counterexample \textbf{(d)}}
\end{figure}
\chapter{Topological Method in Calculus of Variations}

\section{Introduction}

Let $f : \R \to \R$ be a continuous function defined on the real line. A sufficient condition for $f$ to admit a critical point is that
\begin{equation*} \lim_{x \to \pm \infty}f(x) = + \infty. \end{equation*}
The situation is a lot more complex in higher dimension, even when $N = 2$, since a continuous function $f : \R^2 \to \R^2$ such that
\begin{equation*} \lim_{|(0, \, y)| \to \infty}f(x, \, y) = + \infty \qquad \text{and} \qquad  \lim_{|(x, \, 0)| \to \infty}f(x, \, y) = - \infty \end{equation*}
does not necessarily admit a stationary point, let alone a minimum/maximum.

\paragraph{Main Idea.} Let $X \subset \R^N$ be a submanifold (or a Riemann manifold), and let $f$ be a continuous function defined on $X$. For any $a \in \R$, we denote by $f^{(a)}$ the sub-level of $f$ given by
\begin{equation*} f^{(a)} = \left\{ x \in X \: \left| \: f(x) \leq a \right. \right\}. \end{equation*}
We shall prove that, if $a < b$ and $f^{(b)} \setminus f^{(a)}$ does not contain any critical point of $f$, then, assuming that there is some compactness, it turns out that $f^{(a)}$ is a deformation retract of $f^{(b)}$ which is constructed along the curves of maximal decreasing of $f$, i.e., 
\begin{equation*}u^\prime(t) = - \mathrm{grad} \left(f(u(t)) \right) . \end{equation*}

\section{Deformation Lemma}

In this section we shall make more precise the main idea introduced in the previous discussion. We denote by $X$ either \mbox{}
\begin{enumerate}[label=\textbf{(\arabic*)}]
\item a finite/infinite dimensional Hilbert space; or
\item a complete submanifold (of class $C^2$) of $\R^N$ with respect to the induced metric; or
\item a complete submanifold (of class $C^2$) of a Hilbert space $H$ with respect to the induced metric.
\end{enumerate}

\begin{definition}[Gradient] Let $A \subseteq \R^N$ be an open subset such that $X \subseteq A$. The \textit{gradient} of a differentiable function $f : A \to \R$ is the unique vector, denoted by $\mathrm{grad} \, f(u)$, such that
\begin{equation*} \mathrm{d}f(u)[v] = \left( \mathrm{grad} \, f(u), \, v \right), \qquad \forall \, v \in \R^N. \end{equation*} \end{definition}

The notion of \textit{gradient} depends on the scalar product. Also, if the function $f$ is defined on a submanifold $X \subseteq \R^N$, then the differential at $u \in X$ sends the tangent $T_u \, X$ to $\R$ and hence the $X$-gradient can be identified to the restriction of the usual gradient $X$, i.e.,
\begin{equation*} \mathrm{grad}_X \, f(p) = \mathrm{grad} \, f(p) \, \big|_{T_u \, X}. \end{equation*}

\begin{remark}If $X$ is a submanifold of codimension one (either of $\R^N$ or a Hilbert space), then the gradient on $X$ is given by the formula
\begin{equation*} \mathrm{grad}_X \, f(u) = \mathrm{grad} \, f(u) - \frac{\left( \mathrm{grad} \, f(u), \, \nu_X(u) \right)}{\| \nu_X(u) \|^2} \, \nu_X(u), \end{equation*}
where $\nu_X(u)$ is the normal vector of $X$ at the point $u$. \end{remark}

We are now ready to introduce the compactness mentioned in the previous section, which, as the reader may check, is a much weaker notion than the usual one.

\begin{definition}[Palais-Smale] Let $c \in \R$ be any real number, and let $f : X \to \R$ be any function. We say that $f$ satisfies the \textit{Palais-Smale condition} if any sequence $(u_n)_{n \in \N} \subset X$ such that \mbox{}
\begin{enumerate}[label=\textbf{\arabic*)}]
\item $f(u_n) \to c$ and
\item $\mathrm{grad} \, f(u_n) \to 0$
\end{enumerate}
is \textit{precompact}, that is, it admits a converging subsequence.\end{definition}

\begin{definition}[Retract] Let $A \subseteq B \subseteq X$ be subspaces, and let $X$ be a metric space. \mbox{}
\begin{enumerate}[label=\textbf{(\alph*)}]
\item $A$ is a \textit{retract} of $B$ if there exists $r : B \to A$ continuous retraction (i.e. $r \, \big|_A = \mathrm{Id}_A$).
\item $A$ is a \textit{deformation retract} of $B$ in $X$ if there exists an homotopy $H$ between the retraction $r : B \to A$ and the inclusion $\imath : A \hookrightarrow B$.
\item $A$ is a \textit{strong deformation retract} of $B$ in $X$ if it is a \textit{deformation retract} of $B$ in $X$, and the homotopy $H$ has an additional property: $H(t, \, \cdot) \, \big|_A = \mathrm{Id}_A$ for any $t \in [0, \, 1]$.
\end{enumerate}\end{definition}

\begin{remark}\label{rmk:cris} If the function $f : X \to \R$ satisfies the $\left( \mathrm{PS} \right)_c$ condition for every $c \in [a, \, b]$, and if there are no critical points of $f$ in that interval, then there exists $\epsilon_0 > 0$ such that
\begin{equation*} \inf_{ u \in f^{-1} \left( I_\epsilon \right) } \left\| \mathrm{grad} \, f(u) \right\| > 0, \end{equation*}
where $I_\epsilon = \left[ a - \epsilon_0, \, b + \epsilon_0 \right]$. \end{remark}

\begin{proof}Suppose, by contradiction, that for any $\epsilon_0 > 0$ the infimum is equal to $0$. Then there is a sequence $(\epsilon_n)_{n \in \N}$, decreasingly converging to $0$, and a sequence $(u_n)_{n \in \N} \subset X$ such that
\begin{equation*} \left\| \mathrm{grad} \, f(u_n) \right\| \leq \epsilon_n \qquad \text{and} \qquad f(u_n) \in \left[ a- \epsilon_n, \, b + \epsilon_n \right]. \end{equation*}
Therefore, up to subsequences, $f(u_n) \to c$ and $\mathrm{grad} \, f(u_n) \to 0$ and by Palais-Smale condition $(u_n)_{n \in \N}$ is precompact.  But this is absurd since it would imply the existence of an element $u \in X$ such that $u_{n_k} \to u$ and $u$ critical point.
\end{proof}

\begin{lemma}[Deformation Lemma] \label{deflemma}Let $f : X \to \R$ be a function satisfying the $\left( \mathrm{PS} \right)_c$ condition for every $c \in [a, \, b]$, and assume that $\mathrm{grad} \, f(u) \neq 0$ for any $u \in f^{-1}\left([a, \, b] \right)$.

Then there exists $\epsilon_0 > 0$ such that, for any $\epsilon \in [0, \, \epsilon_0]$ and any $\delta \in [0, \, \epsilon_0]$, the sublevel $f^{(a-\epsilon)}$ is a strong deformation retract of $f^{(b + \delta)}$. \end{lemma}

\begin{proof}This theorem requires many steps to be proved. Therefore we shall divide the argument into four little steps, to clarify it for the reader.

\paragraph{Step 1.} The previous \hyperref[rmk:cris]{Remark \ref{rmk:cris}} implies that there exists $\epsilon_0 > 0$ such that
\begin{equation*} \sigma := \inf_{ u \in f^{-1} \left( I_\epsilon \right) } \left\| \mathrm{grad} \, f(u) \right\| > 0. \end{equation*}
Therefore it suffices to prove that $f^{(\alpha)}$ is a strong deformation retract of $f^{(\beta)}$ for any $\alpha \in [a-\epsilon_0, \, a]$ and for any $\beta \in [b, \, b+\epsilon_0]$.

\paragraph{Step 2.} Let $u \in f^{(\beta)}$ be any point, and let us consider the initial value problem
\begin{equation} \label{eq:sdosd} \begin{cases} U^\prime(t) = - \frac{\mathrm{grad} \, f \left(U(t) \right)}{ \left\| \mathrm{grad} \, f \left(U(t) \right) \right\|^2} \\ U(0) = u. \end{cases} \end{equation}
The main idea is to prove that there exists a unique solution of the initial problem \eqref{eq:sdosd} and that it can be extended up to $f^{(\alpha)}$.

The first assertion is a straightforward application of the Cauchy-Lipschitz theorem, according to which there exists $\delta > 0$ and a unique solution $U : [0, \, \delta] \to X \in C^1$ such that $U$ solves \eqref{eq:sdosd}.

\paragraph{Step 3.} Assume that $f(u) \geq \alpha$ (otherwise the Lemma is trivial).

It is enough to prove that $U$ is defined on $[0, \, f(u) - \alpha]$, that is, the maximal interval of definition $I$ has supremum strictly bigger than $f(u) - \alpha$. We argue by contradiction: suppose that $\sup(I) \leq f(u) - \alpha$. A simple computation shows that
\begin{equation*} \frac{\mathrm{d}}{\mathrm{d} \, t} \left(f \circ U\right)(t) = \left( \mathrm{grad} \, f \left(U(t) \right), \, - \frac{\mathrm{grad} \, f \left(U \right)}{ \left\| \mathrm{grad} \, f \left(U \right) \right\|^2} \right) = -1, \end{equation*}
so that $f(u) - \alpha \leq f \left(U(t) \right) \leq f(u)$. It follows that
\begin{equation*} \left\| \mathrm{grad} \, f \left( U(t) \right) \right\| \geq \sigma \implies \left\| U^\prime(t) \right\| \leq \frac{1}{ \left\| \mathrm{grad} \, f \left(U \right) \right\| } \leq \frac{1}{\sigma}, \end{equation*}
therefore $U^\prime$ is bounded, $U$ is a Lipschitz function and the limit $\lim_{T \to \sup(I)} U(T)$ exists.

In particular, the maximum of $I$ exists and it is equal to $f(u) - \alpha$: if not, from the Cauchy-Lipschitz theorem, we would be able to extend the solution $U$ starting from $U(\sup(I))$, and this is absurd (since $I$ is maximal).

\paragraph{Step 4.} Let us set $S(u, \, t) := U(t)$, where $U$ is the solution of \eqref{eq:sdosd} with initial data $u$. Then the function
\begin{equation*} H(t, \, u) := S\left(u, \, t \cdot \left (f(u) - \alpha \right) \right) : [0, \, 1] \times \left( f^{(\beta)} \setminus f^{(\alpha)} \right) \to \R \end{equation*}
is the sought homotopy. Indeed, the main properties are easy to check, and it can be prolonged to $f^{(\alpha)}$ by continuity (i.e., by setting it equal to the identity for any $u$ such that $f(u) \leq \alpha$).
\end{proof}%RIVEDI

\begin{theorem}\label{inusual}Let $f : X \to \R$ be a function such that $\alpha := \inf_{x \in X} f(x) > - \infty$. If $f$ satisfies the $\left( \mathrm{PS} \right)_\alpha$ condition, then $f$ admits a minimum. \end{theorem}

\begin{proof}Suppose by contradiction that there is no minimum point. There exists $\epsilon > 0$ such that $f^{(c - \epsilon)}$ is a deformation retract of $f^{(c + \epsilon)}$ (see \hyperref[deflemma]{Lemma \ref{deflemma}}), but the first one is empty and the second one is not.  \end{proof} %RIVEDI

\begin{example}Let $X$ be a Hilbert space, and let $A \in \mathscr{L}_c^{sym}(H)$ be a compact symmetric operator defined on $H$. If we set
\begin{equation*} f(u) := \frac{1}{2} \left< A \, u, \, u \right> \:  : \: S_H(0, \, 1) \longrightarrow \R, \end{equation*}
then the following assertions hold true: \mbox{}
\begin{enumerate}[label=\textbf{(\alph*)}]
\item If there exists $u \in S_H(0, \, 1)$ such that $f(u) > 0$ strictly, then there exist a maximum point $u_0 \in S_H(0, \, 1)$ for the function $f$ and a real number $\lambda_0 \in \R$ such that
\begin{equation*} A \, u_0 = \lambda_0 \, u_0.\end{equation*}
\item If there exists $u \in S_H(0, \, 1)$ such that $f(u) < 0$ strictly, then there exist a minimum point $u_0 \in S_H(0, \, 1)$ for the function $f$ and a real number $\lambda_0 \in \R$ such that
\begin{equation*} A \, u_0 = \lambda_0 \, u_0.\end{equation*}
\end{enumerate}
We only discuss \textbf{(a)} since the proofs are essentially the same. The operator $A$ is symmetric, thus $A$ is equal to the gradient of $f$ in $X$, that is,
\begin{equation*} A = \mathrm{grad}_X \, f. \end{equation*}
The operator $A$ is compact, hence it satisfies the Palais-Smale condition at any level $c \neq 0$. Therefore, by \hyperref[inusual]{Theorem \ref{inusual}} it turns out that there exists a maximum point $u_0 \in S_H(0, \, 1)$ for $f$, i.e., a point such that
\begin{equation*}  \mathrm{grad}_S \, f(u_0) = 0 \implies \mathrm{grad} \, f(u_0) \parallelsum u_0 \implies A \, u_0 = \lambda_0 \, u_0. \end{equation*}
It remains to justify why the compactness of the operator $A$ is enough to infer that $f$ satisfies the $\left( \mathrm{PS} \right)_c$ condition at every $c \neq 0$. Let $\left(u_n \right)_{n \in \N} \subset S_H(0, \, 1)$ be a Palais-Smale sequence at level $c$, that is,
\begin{equation*} \begin{cases} f(u_n) \xrightarrow{n \to + \infty} c \\ \mathrm{grad}_S \, f(u_n) = A(u_n) - \frac{ \left(A \, u_n, \, u_n \right)}{\|u_n\|^2} \xrightarrow{n \to + \infty} 0. \end{cases} \end{equation*}
The operator $A$ is compact, thus there exists a subsequence  $\left(u_{n_k} \right)_{k \in \N} \subset S_H(0, \, 1)$ such that $A(u_{n_k}) \to v \in X$. This concludes the proof since the subsequence cannot converge to $0$, otherwise the gradient of $f$ would not satisfy the Palais-Smale condition.
\end{example}

\section{Mountain Pass Theorem}

In this section we want to discuss the \textit{Saddle Point Theorem} and the \textit{Mountain Pass Theorem}. These results are fundamental in the study of nonlinear partial differential equations.

\begin{theorem}[Saddle points] \label{th:saddpoi} Let $X$ be a Hilbert space, and let $f \in C^1 \left(X; \; \R \right)$ be a differentiable functional. Suppose that there are $X_0, \, X_1 \subset X$ subspaces such that $\mathrm{dim} (X_0) < + \infty$, and $X$ is equal to their direct sum, that is,
\begin{equation*} X = X_0 \oplus X_1. \end{equation*}
Assume that there exists $\rho_0 > 0$ such that, if we set $S_0 := S_X(0, \, \rho_0) \cap X_0$, then
\begin{equation} \label{eq:sup} \sup_{x \in S_0} f(x) < \inf_{x \in X_1} f(x). \end{equation}
Assume also that $f$ satisfies the Palais-Smale condition $\left( \mathrm{PS} \right)_c$ at every level $c \in [a, \, b]$, where
\begin{equation*} a := \inf_{x \in X_1} f(x) \qquad \text{and} \qquad b := \sup_{x \in B_0} f(x). \end{equation*}
Then there exists a critical point $u \in X$ for the functional $f$, such that $f(u) \in [a, \, b]$.
\end{theorem}

\begin{lemma}\label{lemma:tech1} Let $X, \, X_0, \, X_1$ and $\rho_0 > 0$ be as in the assumptions of the \hyperref[th:saddpoi]{Saddle Point Theorem}. If $\Phi : \overline{B_0} \to X$ is a continuous function such that
\begin{equation*} \Phi \, \big|_{S_0} \equiv \mathrm{Id} \, \big|_{S_0}, \end{equation*}
then the image of the ball intersects the space $X_1$, that is,
\begin{equation*}\Phi( B_0 ) \cap X_1 \neq \emptyset. \end{equation*} \end{lemma}

\begin{proof} Let $P_0 : X \to X_0$ be the orthogonal projection on $X_0$ with kernel equal to $X_1$, and let us consider the continuous composition
\begin{equation*} P_0 \circ \Phi : \overline{B_0} \to X_0. \end{equation*}
Let $u \in S_0$ be any point. It turns out that $P_0 \circ \Phi(u) = P_0(u) = u$, and thus the restriction of $P_0 \circ \Phi$ on the sphere $S_0$ is the identity map. Therefore
\begin{equation*} B_0 \subset P_0 \circ \Phi \left(B_0 \right), \end{equation*}
which, in turn, implies that there exists $u \in B_0$ such that $P_0 \circ \Phi(u) = 0$. Since the kernel of $P_0$ is $X_1$, this is equivalent to $\Phi(u) \in \mathrm{Ker}(P_0) = X_1$, that is exactly what we wanted to prove.\end{proof}

\begin{proof}[Proof of Saddle Point Theorem] We argue by contradiction. Let $\epsilon > 0$ be small enough that the following inequality holds:
\begin{equation*} \sup_{x \in S_0} f(x) < \inf_{x \in X_1} f(x) - \epsilon. \end{equation*}
By \hyperref[deflemma]{Deformation Lemma \ref{deflemma}} there exists a retraction $r : f^{(b)} \to f^{(a - \epsilon)}$, and it follows easily that
\begin{equation*} \overline{B_0} \subseteq f^{(b)}, \qquad S_0 \subseteq f^{(a - \epsilon)} \quad \text{and} \quad X_1 \cap f^{(a - \epsilon)} = \emptyset. \end{equation*}
The map $\Phi := r \, \big|_{\overline{B_0}}$ satisfies the assumptions of \hyperref[lemma:tech1]{Lemma \ref{lemma:tech1}}, hence
\begin{equation*} \Phi \left(B_0 \right) \cap X_1 \neq \emptyset. \end{equation*}
In particular, there exists $u \in \overline{B_0}$ such that $\Phi(u) > a - \epsilon$ strictly, and this is absurd since $\Phi(B_0) \subseteq f^{(a - \epsilon)}$, that is, $\Phi(u) \leq a - \epsilon$ follows by assumption \eqref{eq:sup}. %VEDI INC.
\end{proof}

\begin{theorem}[Mountain Pass] Let $X$ be a Hilbert space. Let $S := S_H(0, \, \rho)$ for any $\rho > 0$, and suppose that there exist $u_0, \, u_1 \in X$ such that
\begin{equation*} u_0 \in \accentset{\circ}{B}_\rho \qquad \text{and} \qquad u_1 \notin \overline{B_\rho}. \end{equation*}
Suppose that there exists a functional $f \in C^1 \left(X; \; \R \right)$ such that $f(u_0), \, f(u_1) < \inf_{x \in S} f(x)$. If $\gamma : [0, \, 1] \to X$ is a continuous path from $u_0$ to $u_1$, and if $f$ satisfies the Palais-Smale condition $\left( \mathrm{PS} \right)_c$ at every level
\begin{equation*} c \in \left[ \inf_{x \in S} f(x), \, \sup_{t \in I} f(\gamma(t)) \right] \end{equation*}
then there exists a critical level $c$ (in the same interval) for the functional $f$. \end{theorem}

\begin{proof} We argue by contradiction. There exists $\epsilon > 0$ such that $f(u_0), \, f(u_1) < \inf_{x \in S} f(x) - \epsilon$ and, by the \hyperref[deflemma]{Deformation Lemma}. it turns out that there exists a retraction $r : f^{(b)} \to f^{(a - \epsilon)}$. Then
\begin{equation*} \gamma \left( [0, \, 1] \right) \subseteq f^{(b)} \qquad \text{and} \qquad \text{$r(u_i) = u_i$ since $u_0, \, u_1 \in f^{(a-\epsilon)}$}. \end{equation*}
This is absurd since
\begin{equation*}r \left( \mathrm{Ran}(\gamma) \right) \cap S = \emptyset,\end{equation*}
as a consequence of the fact that $r \circ \gamma (t) \in f^{(a-\epsilon)}$ for any $t \in [0, \, 1]$, and also using the assumption $f^{(a - \epsilon)} \cap S = \emptyset$.
\end{proof}

\begin{lemma}[Deformation Lemma II] \label{deflemma2}Let $f : X \to \R$ be a functional, and let $c$ be a critical value of $f$. If we set
\begin{equation*} \mathcal{Z}_c := \left\{ u \in X \: \left| \: \mathrm{grad} \, f(u) = 0, \, f(u) = c \right. \right\} \end{equation*}
to be the set of critical points, and we assume that the Palais-Smale condition at the level $c$ holds true and that there exists $V$ open neighborhood of $\mathcal{Z}_c$, then there exist $\epsilon > 0$ and a homotopy
\begin{equation*} H : [0, \, 1] \times \left( f^{(c+\epsilon)} \setminus V \right) \to f^{(c+\epsilon)} \end{equation*}
such that $H(0, \, u) = u$, $H(1, \, u) \in f^{(c - \epsilon)}$ and $H(t, \, u) = u$ for any $t \in [0, \, 1]$ and any $u \in \left(f^{(c-\epsilon)} \setminus V \right)$. \end{lemma}

\begin{proof}This theorem requires many steps to be proved. Therefore we shall divide the argument into five little steps, to clarify it for the reader.

\paragraph{Step 1.} The Palais-Smale condition at the level $c$ implies that $\mathcal{Z}_c$ is a compact set; the reader may prove this fact as a simple exercise. In particular, it turns out that
\begin{equation*} d \left(V^c, \, Z_c \right) > 0, \end{equation*}
and thus there exists an open neighborhood $\mathcal{U}$ of $\mathcal{Z}_c$ such that
\begin{equation*} \overline{\mathcal{U}} \subseteq V \qquad \text{and} \qquad \delta := d(U, \, V)  > 0. \end{equation*}
Moreover, since there are no critical points in $X \setminus \mathcal{Z}_c$, there exists $\epsilon_0 > 0$ such that
\begin{equation*} \sigma := \inf \left\{ \left\| \mathrm{grad} \, f(u) \right\| \: \left| \: u \in X \setminus U, \, c - \epsilon_0 \leq f(u) \leq c + \epsilon_0 \right. \right\} > 0. \end{equation*}

\paragraph{Step 2.} Let $u \in f^{(c + \epsilon_0)} \setminus V$ be any point, and let us consider the initial value problem
\begin{equation} \label{eq:sdosd2} \begin{cases} U^\prime(t) = - \frac{\mathrm{grad} \, f \left(U(t) \right)}{ \left\| \mathrm{grad} \, f \left(U(t) \right) \right\|^2} \\ U(0) = u. \end{cases} \end{equation}
The key idea is proving that there exists a unique solution of the initial problem \eqref{eq:sdosd2}, and that it can be extended up to $f^{(c-\epsilon_0)}$ without intersecting $\mathcal{Z}_c$ in any point.

The first assertion is a straightforward application of the Cauchy-Lipschitz theorem, according to which there exists $\delta > 0$ and a unique curve $U : [0, \, \delta] \to X$ of class $C^1$, such that $U$ solves \eqref{eq:sdosd2}.

\paragraph{Step 3.} Let $I$ be an interval such that $\mathrm{min}(I) = 0$, and suppose that the solution $U$ is defined on $I$. If there exists a time $t \in I$ such that
\begin{equation*} U(t) \in \mathcal{U} \quad \text{and} \quad f \left(U(t) \right) \geq c - \epsilon_0, \end{equation*}
then it turns out that
\begin{equation} \label{timecond} t \geq \delta \cdot \sigma. \end{equation}
This estimate on the time $t$ follows easily from the fact that
\begin{equation*} t \geq t_0 := \inf \left\{ \tau \in I \: \left| \: U(\tau) \in \mathcal{U} \right. \right\}. \end{equation*}
Indeed, the length of the curve from $0$ and $t_0$ satisfies the following inequality
\begin{equation*} \int_{0}^{t_0} \left\| U^\prime(\tau) \right\| \, \mathrm{d}\tau = \int_{0}^{t_0} \frac{1}{\| \mathrm{grad} \, f\left(U(\tau) \right) \|} \, \mathrm{d}\tau \leq \frac{t_0}{\sigma}, \end{equation*}
since $U(0) = u \in f^{(c+ \epsilon_0)} \setminus V$. On the other hand, the length of the curve needs to be at least equal to $\delta$ to intersect $\mathcal{U}$, thus
\begin{equation*} \frac{t_0}{\sigma} \geq \delta \implies t \geq t_0 \geq \sigma \cdot \delta \implies \eqref{timecond}. \end{equation*}

\paragraph{Step 4.} From the previous step, it follows that
\begin{equation*} f \left( U(t) \right) \leq f(u) - \delta \cdot \sigma. \end{equation*}
Let $\epsilon$ be a real number such that
\begin{equation*} 0 < \epsilon < \min \left\{ \epsilon_0, \, \frac{1}{2} \, \sigma \cdot \delta \right\}. \end{equation*}
If $u \in f^{(c + \epsilon)} \setminus V$ and $u \notin f^{(c-\epsilon)}$, then the solution curve $U$ reaches $f^{(c - \epsilon)} \setminus \mathcal{U}$, that is, there is a point such that $U(t) \in f^{(c - \epsilon)}$ and $U(s) \notin \mathcal{U}$ for any $s \leq t$.  Moreover, from the estimate
\begin{equation*} \| U^\prime \| \leq \frac{1}{\| \mathrm{grad} \, f(U) \|} \leq \frac{1}{\sigma}, \end{equation*}
it turns out that the curve can be extended in such a way that $U$ is well defined in the interval
\begin{equation*} J := \left[0, \, f(u) - (c - \epsilon) \right]. \end{equation*}

\paragraph{Step 5.} In conclusion, the sought homotopy is given by
\begin{equation*} H(t, \, u) := U_u \left(t \cdot \left(f(u) - (c-\epsilon)\right) \right). \end{equation*}
By continuity we may extend it to any $u$ such that $f(u) \leq c -\epsilon$, by setting it identically equal to zero.
\end{proof}

\section{Critical Points in Symmetrical Problems}

Let $E$ be a normed space and let $X \subseteq E$ be a complete submanifold of class $C^2$ without boundary such that
\begin{equation*} u \in X \implies - u \in X. \end{equation*}
Let $f : X \to \R$ be an \textbf{even} functional of class $C^1(X)$. Then the topological degree of $f$ is odd, and it turns out that
\begin{equation*} \text{$u$ critical point} \iff \text{$-u$ critical point}. \end{equation*}
More precisely, if $U$ is the solution of the the initial-value problem
\begin{equation*} \begin{cases} U^\prime(t) = - \frac{\mathrm{grad} \, f \left(U(t) \right)}{ \left\| \mathrm{grad} \, f \left(U(t) \right) \right\|^2} \\ U(0) = u. \end{cases} \end{equation*}
then the solution of the problem
\begin{equation} \label{eq:osdoos} \begin{cases} V^\prime(t) = - \frac{\mathrm{grad} \, f \left(V(t) \right)}{ \left\| \mathrm{grad} \, f \left(V(t) \right) \right\|^2} \\ V(0) = -u. \end{cases} \end{equation}
is $V \equiv - U$. Therefore, if $X$ is a symmetric submanifold, we can restate the deformation lemmas requiring an additional property (i.e., the homotopy is odd at each level).

\begin{lemma}[Deformation Lemma I] \label{deflemmasym}Let $f : X \to \R$ be an even functional satisfying the $\left( \mathrm{PS} \right)_c$ condition for every $c \in [a, \, b]$, and assume that $\mathrm{grad} \, f(u) \neq 0$ for any $u \in f^{-1}\left([a, \, b] \right)$.

Then there exists $\epsilon_0 > 0$ such that, for any $\epsilon \in [0, \, \epsilon_0]$ and any $\delta \in [0, \, \epsilon_0]$, the sublevel $f^{(a-\epsilon)}$ is a strong deformation retract of $f^{(b + \delta)}$, that is, there exists an homotopy
\begin{equation*}H : [0, \, 1] \times f^{(b + \delta)} \to f^{(b + \delta)} \end{equation*}
satisfying the following properties:
\begin{enumerate}[label=\textbf{(\arabic*)}]
\item $H(0, \, u) = u$ for every $u \in f^{(b + \delta)}$.
\item $H(1, \, u) \in f^{(a - \epsilon)}$ for every $u \in f^{(b + \delta)}$.
\item $H(t, \, u) = u$ for every $u \in f^{(a - \epsilon)}$ and every $t \in [0, \, 1]$.
\item $H_t(u) = - H_t(-u)$, that is, the homotopy is odd at each level.
\end{enumerate} \end{lemma}

\begin{lemma}[Deformation Lemma II] \label{deflemma2sym}Let $f : X \to \R$ be an even functional, and let $c$ be a critical value of $f$. If we set
\begin{equation*} \mathcal{Z}_c := \left\{ u \in X \: \left| \: \mathrm{grad} \, f(u) = 0, \, f(u) = c \right. \right\} \end{equation*}
to be the set of critical points, and we assume that the Palais-Smale condition at the level $c$ holds true and that there exists $V$ open neighborhood of $\mathcal{Z}_c$, then there exist $\epsilon > 0$ and a homotopy
\begin{equation*} H : [0, \, 1] \times \left( f^{(c+\epsilon)} \setminus V \right) \to f^{(c+\epsilon)} \end{equation*}
satisfying the following properties:
\begin{enumerate}[label=\textbf{(\arabic*)}]
\item $H(0, \, u) = u$ for every $u \in f^{(c + \epsilon)} \setminus V$.
\item $H(1, \, u) \in f^{(c - \epsilon)}$ for every $u \in f^{(c + \epsilon)} \setminus V$.
\item $H(t, \, u) = u$ for every $u \in f^{(c - \epsilon)} \setminus V$ and every $t \in [0, \, 1]$.
\item $H_t(u) = - H_t(-u)$, that is, the homotopy is odd at each level.
\end{enumerate} \end{lemma}
\chapter{Genus}

In this section we say that a subset $\Omega \subset E$ is \textit{symmetric} if it is symmetric with respect to the origin of $E$, that is,
\begin{equation*} u \in \Omega \implies - u \in \Omega. \end{equation*}

\paragraph{Definition and Main Properties.} Let $E$ be a normed space, and let us denote by $\Gamma$ the class of all the symmetric subsets $A \subseteq E \setminus \{0\}$ such that $A$ is closed in $E \setminus \{0\}$.

\begin{definition}[Genus] Let $A \in \Gamma$. The \textit{genus} of $A$ is defined by
\begin{equation*} \gamma(A) := \min \left\{k \in \N \: \left| \: \text{$\exists \, \Phi : A \to \R^k \setminus \{0\}$ continuous and odd} \right. \right\}, \end{equation*}
and $+ \infty$ if such a natural number $k$ does not exist. \end{definition}

\begin{notation} If $A = \emptyset$, then we set $\gamma(A) := 0$.\end{notation}

\begin{remark}The request that $0$ does not belong to the codomain of $\Phi$ is crucial. Otherwise, the genus would not be an attractive notion since one could take the null map
\begin{equation*} \Phi : A \to \R^k, \qquad \Phi \equiv 0 \end{equation*}
for any $k \in \N$, and for every element $A \in \Gamma$. \end{remark}

\begin{remark}\label{rmk:genussphere}The definition of genus does not change if we require $\Phi$ to be a function with values in the sphere $S^{k-1}$ instead of $\R^{k} \setminus \{0\}$ since one may always compose it with the projection
\begin{equation*} \pi : \R^k \setminus \{0\} \to S^{k-1}, \qquad x \longmapsto \frac{x}{|x|}. \end{equation*}\end{remark}

\begin{example}If $E = L^2(\R^N)$ and $A = S_E(0, \, 1)$ is the sphere (with respect to the $L^2$-norm), then the genus of $A$ is $\infty$.

\paragraph{Proof.} If there exists $k \in \N$ such that there is a continuous odd map $\Phi : A \to \R^k \setminus \{0\}$, then we can apply \hyperref[b]{Borsuk Theorem} and obtain a contradiction (since $A$ contains the spheres of every $\R^m$ for $m > k$). \end{example}

\begin{lemma} \mbox{}
\begin{enumerate}[label=\textbf{(\alph*)}]
\item If $A \in \Gamma$ is finite and nonempty, then $\gamma(A) = 1$.
\item If $A \subseteq \R^N$ and $0 \notin A$, then $\gamma(A) \leq N$.
\item If $0 \in A$, then $\gamma(A) = + \infty$.
\end{enumerate} \end{lemma}

\begin{proof} \mbox{}
\begin{enumerate}[label=\textbf{(\alph*)}]
\item By assumption $A$ is a set of the form
\begin{equation*} A = \left\{ u_1, \, - u_1, \, \dots, \, u_n, \, - u_n \right\}.\end{equation*}
The map $\Phi : A \to \R \setminus \{0\}$ sending $u_i$ to $+ 1$ and $- u_i$ to $- 1$ for every $i \in \{1, \, \dots, \, n\}$ is well defined, continuous and odd.
\item The identity map $\mathrm{id}_{\R^N}$ is continuous and odd, therefore $\gamma(A) \leq N$.
\item Every odd map sends $0$ to $0$, hence for every $k \in \N$ a map $\Phi : A \to \R^{k} \setminus \{0\}$ cannot be odd and continuous (i.e., the minimum $k$ is not attained and, by definition, the genus is infinite).
\end{enumerate} \end{proof}

\begin{proposition} \label{propgenus}\mbox{}
\begin{enumerate}[label=\textbf{(\alph*)}]
\item Let $A \in \Gamma$. Then $\gamma(A) = 0$ if and only if $A = \emptyset$.
\item Let $A, \, B \in \Gamma$. If $\Phi : A \to B$ is a continuous odd map, then $\gamma(A) \leq \gamma(B)$. In particular
\begin{equation*} A \subseteq B \implies \gamma(A) \leq \gamma(B). \end{equation*}
\item Let $A, \, B \in \Gamma$. The genus is subadditive, that is,
\begin{equation} \label{genussub} \gamma (A \cup B) \leq \gamma(A) + \gamma(B). \end{equation}
\item Let $A \in \Gamma$. There exists an open neighborhood $U$ of $A$ such that
\begin{equation*}u \in U \implies - u \in U, \qquad 0 \notin \overline{U} \qquad \text{and} \qquad \gamma \left( \overline{U} \right) = \gamma(A). \end{equation*}
\end{enumerate} \end{proposition}

\begin{proof} \mbox{}
\begin{enumerate}[label=\textbf{(\alph*)}]
\item Obvious.
\item If there exists $k \in \N$ such that there is a continuous odd function $\Psi : B \to \R^k \setminus \{0\}$, then the composition
\begin{equation*}\Psi \circ \Phi : A \to \R^k \setminus \{0\}\end{equation*}
is a continuous odd map, and thus $\gamma(A) \leq \gamma(B)$.
\item Let $k, \, h \in \N$ be natural numbers such that there are $\Phi_1 : A \to \R^k \setminus \{0\}$ and $\Phi_2 : B \to \R^h \setminus \{0\}$ continuous odd maps.

Let us denote by $\tilde{\Phi}_1 : A \cup B \to \R^k$ and $\tilde{\Phi}_2 : A \cup B \to \R^h$ the continuous odd extensions of $\Phi_1$ and $\Phi_2$. More precisely, take the odd parts of the continuous extensions given by \hyperref[tiet]{Tietze Theorem} (since $A$ and $B$ are closed subsets of the union $A \cup B$).

Let us consider the map
\begin{equation*} \Psi(u, \, v) := \left(\tilde{\Phi}_1(u), \, \tilde{\Phi}_2(v) \right) : A \cup B \to \R^k \times \R^h.\end{equation*}
It is straightforward to check that $\Psi$ is a continuous odd map, such that no point $u \in A \cup B$ goes to the origin $(0, \, 0) \in \R^{k + h}$ via $\Psi$.
\item Let $k := \gamma(A)$. By \hyperref[rmk:genussphere]{Remark \ref{rmk:genussphere}} there exists a continous  odd map
\begin{equation*} \Phi :A \to S^{k-1}. \end{equation*}
The sphere is closed in $E$, hence there exists a continuous odd function $\tilde{\Phi} : E \to \R^k$, extending $\Phi$, but a priori $0$ may belong to its image. If we define
\begin{equation*} U := \left\{ u \in E \: \left| \: \left| \tilde{\Phi}(u) \right| > \frac{1}{2} \right. \right\} \end{equation*}
then one can easily check that it satisfies the following properties: \mbox{}
\begin{enumerate}[label=\textbf{(\arabic*)}]
\item $U$ is open and symmetric.
\item $0 \notin \overline{U}$.
\item $A \subseteq U$ and $\overline{U} \in \Gamma$.
\end{enumerate}
\end{enumerate} \end{proof}

\begin{theorem}[Genus of the Sphere]\label{gen:sph} Let $\Omega \subseteq \R^N$ be an open, bounded, symmetric (with respect to the origin) subset containing $0$. Then
\begin{equation*}\gamma \left( \partial \, \Omega \right) = N. \end{equation*}\end{theorem}

\begin{proof}On the one hand, we have already proved that $\gamma \left( \partial \, \Omega \right) \leq N$. On the other hand, if
\begin{equation*}\Phi : \partial \, \Omega \subseteq \R^N \to \R^k\end{equation*}
is a continuous odd map, then \hyperref[b]{Borsuk Theorem} implies that $0 \in \Phi \left( \partial \, \Omega \right)$ for every $k < N$  \end{proof}

\begin{corollary} If $S^{N-1} \subseteq \R^N$ is the sphere centered at $0$, then $\gamma(S) = N$. \end{corollary}

\begin{corollary} If $E$ is an infinite-dimensional normed space, then $\Omega$ has infinite genus.\end{corollary}

\section{Genus in Calculus of Variations}

Let $H$ be a Hilbert space, and let $X \subseteq E$ be a complete submanifold without boundary of class $C^2$ and symmetric with respect to the origin, i.e.,
\begin{equation*} u \in X \implies - u \in X. \end{equation*}
In this section, the functional $f : X \to \R$ will always be assumed to be \textbf{even} and of class $C^1(X)$.

\begin{proposition}Let $a, \, b \in \R$ be real numbers such that $a < b$, and assume that $f$ satisfies the $\left( \mathrm{PS} \right)_c$ condition at every level $c \in [a, \, b]$. If $\gamma(f^{(a)}) < \gamma(f^{(b)})$, then there exists a critical level $c \in [a, \, b]$ for $f$. \end{proposition}

\begin{proof}We argue by contradiction. Suppose that there is no critical point $c\in [a, \, b]$; then there is an odd retraction $r: f^{(b)} \to f^{(a)}$ given by the symmetric \hyperref[deflemmasym]{Deformation Lemma \ref{deflemmasym}}. But this is absurd since \hyperref[propgenus]{Proposition \ref{propgenus}} implies that 
\begin{equation*}\gamma(f^{(a)}) \geq \gamma(f^{(b)}), \end{equation*}
which is in contradiction with the assumption. \end{proof}

\begin{notation}Let $k \in \N$ be a natural number such that $1 \leq k \leq \gamma(X)$. We denote by $\gamma_k$ the infimum of all the sublevels such that the genus is at least $k$, that is,
\begin{equation*} \gamma_k := \inf \left\{ c \in \R \: \left| \: \gamma \left( f^{(c)} \right) \geq k \right. \right\}. \end{equation*}
Moreover, it is possible that $\gamma \left( f^{(c)} \right) \geq k$ is not satisfied for any real number $c \in \R$. In this case, we set $\gamma_k := + \infty$ and, clearly, the supremum of $f$ must be $+ \infty$. \end{notation}

\begin{lemma}[Fundamental Lemma] \label{gen:fl}Let $k \in \N$ be a natural number such that $1 \leq k \leq \gamma(X)$. \mbox{}
\begin{enumerate}[label=\textbf{(\alph*)}]
\item $\inf_{x \in X} f(x) = \gamma_1 \leq \gamma_2 \leq \dots \leq \gamma_k \leq \sup_{x \in X} f(x)$.
\item If $\gamma_k \in \R$ and $f$ satisfies the $\left( \mathrm{PS} \right)_{\gamma_k}$ condition, then $\gamma_k$ is a critical level for the functional $f$. In particular, if $\gamma_1 \in \R$, then $\gamma_1 = \min_{x \in X} f(x)$.
\item If $\gamma_k = \gamma_{k+1} = \dots = \gamma_{k + h} \in \R$ and $f$ satisfies the $\left( \mathrm{PS} \right)_{\gamma_k}$ condition, then for the set of all the critical points of $f$, denoted by $Z_{\gamma_k}$, it results
\begin{equation*} \gamma \left( Z_{\gamma_k} \right) \geq h + 1. \end{equation*}
In particular, if $0 \in Z_{\gamma_k}$, then it turns out that it is an infinite set.
\end{enumerate} \end{lemma}

\begin{proof}
\begin{enumerate}[label=\textbf{(\alph*)}]
\item The first identity follows from the fact that
\begin{equation*} \gamma_1 := \inf \left\{ c \in \R \: \left| \: \gamma \left( f^{(c)} \right) \geq 1 \right. \right\} = \inf \left\{ c \in \R \: \left| \:f^{(c)} \neq \emptyset \right. \right\} = \inf_{x \in X} f(x).\end{equation*}
For any $k \geq 1$ it turns out that
\begin{equation*}\left\{ c \in \R \: \left| \: \gamma \left( f^{(c)} \right) \geq k \right. \right\} \supseteq \left\{ c \in \R \: \left| \: \gamma \left( f^{(c)} \right) \geq k+1 \right. \right\},\end{equation*}
hence $\gamma_k \leq \gamma_{k+1}$ by taking the infimum of both the left-hand and the right-hand sides.
\item If $\gamma_k$ is not a critical level for $f$, then the first \hyperref[deflemmasym]{Deformation Lemma} for symmetric maps implies that there exist $\epsilon > 0$ and an odd retraction
\begin{equation*} r : f^{(\gamma_k + \epsilon)} \to f^{(\gamma_k - \epsilon)}. \end{equation*}
But $r$ is odd, hence \hyperref[propgenus]{Proposition \ref{propgenus}} asserts that %INC + RET
\begin{equation*} \gamma \left( f^{(\gamma_k + \epsilon)} \right) = \gamma \left( f^{(\gamma_k - \epsilon)} \right), \end{equation*}
and this is absurd since
\begin{equation*} \gamma \left( f^{(\gamma_k - \epsilon)} \right) \leq k -1 < k \leq \gamma \left( f^{(\gamma_k + \epsilon)} \right). \end{equation*}
\item The set of all critical levels is an element of $\Gamma$, hence by \hyperref[propgenus]{Proposition \ref{propgenus}} there exists a symmetric open neighborhood $U$ of $Z_{\gamma_k}$ such that
\begin{equation*} \overline{U} \in \Gamma \qquad \text{and} \qquad \gamma(U) = \gamma( Z_{\gamma_k} ). \end{equation*}
From the second symmetric \hyperref[deflemma2sym]{Deformation Lemma} it follows that there exist $\epsilon > 0$ and an odd retraction
\begin{equation*} r : f^{(\gamma_k + \epsilon)} \setminus U \to f^{(\gamma_k - \epsilon)}. \end{equation*}
Clearly $f^{(\gamma_k + \epsilon)} \setminus U$ is closed and it belongs to $\Gamma$, hence
\begin{equation} \label{123123} f^{(\gamma_k + \epsilon)} \subseteq \left(f^{(\gamma_k + \epsilon)} \setminus U \right) \cup \overline{U}. \end{equation}
Then $k + h \leq \gamma \left( f^{(c + \epsilon)} \right)$ and by the subadditivity of the genus (see \hyperref[propgenus]{Proposition \ref{propgenus}}), from \eqref{123123} it follows that
\begin{equation*} \begin{aligned} \gamma \left( f^{(c + \epsilon)} \right) & \leq \gamma \left( f^{(c + \epsilon)} \setminus U \right) + \gamma \left( \overline{U} \right) \leq \gamma \left( f^{(c - \epsilon)} \right) \gamma \left( \overline{U} \right) \leq \\ \\& \leq k - 1 + h, \end{aligned} \end{equation*}
and this is clearly absurd.
\end{enumerate} \end{proof}

\paragraph{Consequences.} We are finally ready to state and prove an exceptional result due to the two mathematicians who introduced first the notion of the genus.

\begin{theorem}[Lusternik-Schnirelman] \label{lussch}Let $f : S_\rho^{N-1} \subset \R^N \to \R$ be an even functional of class $C^1$. Then there are at least $n$ couples of critical points for $f$ of the form
\begin{equation*} (-u_i, \, u_i) \in S_\rho^{N-1} \times S_\rho^{N-1} \qquad i = 1, \, \dots, \, n. \end{equation*}
\end{theorem} 
 
\begin{proof}The sphere $S_\rho^{N-1}$ is a compact set, hence $f$ satisfies the Palais-Smale $\left( \mathrm{PS} \right)_c$ condition at every level $c \in \R$. Since $g \left(S^{N-1}\right) = N$ (see \hyperref[gen:sph]{Theorem \ref{gen:sph}}) it follows from \hyperref[gen:fl]{Lemma \ref{gen:fl}} that $\gamma_1 \leq \dots \leq \gamma_N$ are critical levels for the functional $f$, and this concludes the proof.\end{proof}
 
\begin{remark} If $W \supset S_\rho^{N-1}$ is an open set of $\R^N$, then
\begin{equation*}\text{$u \in W$ is a critical point for $f : W \to \R$} \iff \exists \, \lambda \in \R \setminus \{0\} \: : \: \mathrm{grad} \, f(u) = \lambda \, u. \end{equation*} \end{remark}

\begin{definition}[Regular] A subset $\Omega \subseteq \R^N$ is \textit{regular} (see \hyperref[regset]{Figure \ref{regset}}) if for every $x \in \partial \, \Omega$ there is a neighborhood $U \subseteq \R^N$ of $x$ and a diffeomorphism $\varphi : U \xrightarrow{\sim} \R^N$ such that \mbox{}
\begin{enumerate}[label=(\arabic*)]
\item $\varphi(x) = 0$, and
\item $\varphi \left( \partial \, \Omega \cap U \right) \subseteq \R^{N-1}$.
\end{enumerate}\end{definition}

\begin{figure}[h]
\centering
\includegraphics[width=11cm, height=6cm]{Images/MTINAG1.png}
\caption{Regular subset}
\label{regset}
\end{figure}

\begin{theorem}[Lusternik-Schnirelman, II] Let $f : \partial \, \Omega \to \R$ be an even functional of class $C^1$, and assume that $\Omega \subset \R^N$ is an open bounded $C^2$-regular set such that
\begin{equation*} 0 \in \Omega \qquad \text{and} \qquad u \in \Omega \implies - u \in \Omega. \end{equation*}
Then $\gamma(\partial \, \Omega) = N$, and $f$ has (at least) $n$ couples of critical points of the form
\begin{equation*} (-u_i, \, u_i) \in \partial \, \Omega \times \partial \, \Omega \qquad i = 1, \, \dots, \, n. \end{equation*}
\end{theorem}

\begin{proof}The argument is similar to the one used to prove \hyperref[lussch]{Theorem \ref{lussch}}. \end{proof}

\begin{proposition}\label{prop:genus3} Let $X$ be a complete $C^2$-submanifold without boundary of a Hilbert space, and let $f : X \to \R$ be an even functional of class $C^1(X)$.

Assume that there are real numbers $a < b$ such that $f$ satisfies the Palais-Smale condition $\left( \mathrm{PS}\right)_c$ at every level $c \in [a, \, b]$, and $0 \notin f^{-1} \left([a, \, b] \right)$. Then
\begin{equation} \label{eq:gnusld} \gamma\left(f^{(b)} \right)\leq \gamma\left(f^{(a)} \right) + \symbol{35} \left(Z_f \cap f^{-1} \left( [a, \,b] \right) \right). \end{equation} \end{proposition}

\begin{proof} Let $k \in \N$ be any integer number satisfying the estimate
\begin{equation*} \gamma\left(f^{(a)} \right)< k \leq \gamma\left(f^{(b)} \right). \end{equation*}
It follows from \hyperref[gen:fl]{Lemma \ref{gen:fl}} that $\gamma_k$ is a critical value for the functional $f$, and this is enough to infer that \eqref{eq:gnusld} holds true.
\end{proof}

\begin{proposition}Under the same assumptions of \hyperref[prop:genus3]{Proposition \ref{prop:genus3}} it turns out that
\begin{equation*} \text{$\gamma \left(f^{(a)} \right)$ is finite} \implies \text{$\gamma \left(f^{(b)} \right)$ is finite}. \end{equation*}\end{proposition}

\begin{proof} We do not prove this result now, but we may come back to it in the future. \end{proof}

\begin{proposition} Let $X$ be a Hilbert space, and let $f \in C^1 \left(X; \; \R \right)$ be an even functional. Suppose that there are $X_0, \, X_1 \subset X$ subspaces such that $X$ is equal to their direct sum, that is,
\begin{equation*} X = X_0 \oplus X_1. \end{equation*}
Assume that $f$ is bounded from below, that is,
\begin{equation*} \inf_{x \in X} f(x) := M > - \infty, \end{equation*}
and assume also that there exists $\rho_0 > 0$ such that, if we set $S_0 := S_X(0, \, \rho_0) \cap X_0$, then
\begin{equation} \label{eq:sup} \sup_{x \in S_0} f(x) < \inf_{x \in X_1} f(x). \end{equation}
Let $a \in \R$ be any real number such that $a < M$, and assume that $f$ satisfies the Palais-Smale condition $\left( \mathrm{PS} \right)_c$ at every level $c \in [a, \, b]$, where
\begin{equation*} b := \sup_{x \in S_0} f(x). \end{equation*}
Then it turns out that
\begin{equation*} \gamma \left( f^{(a)} \right) = 0 \qquad \text{and} \qquad  \gamma \left( f^{(b)} \right) \geq \gamma \left(S_0 \right) = \mathrm{dim} \, X_0. \end{equation*} \end{proposition}

\begin{proof} The reader may check by herself that for any $k \in \N$ such that $1 \leq k \leq \gamma \left( f^{(b)} \right)$, it turns out that
\begin{equation*} \gamma_k \leq b. \end{equation*}
Moreover, $0$ does not belong to $f^{(b)}$ by definition, and thus $f$ admits at least $\gamma \left( f^{(b)} \right)$ critical points in the sublevel $f^{(b)}$.\end{proof}

\begin{lemma} \label{gen:equiv} Let $E$ be a normed space, and let $A \in \Gamma$. Then
\begin{equation}\label{eq:equiv} \gamma(A) = \min \left\{ k \in \N \: \left| \: \begin{gathered} \text{$\exists \, A_1, \, \dots, \, A_k \subset E$ closed,} \\ \text{$\left(A_i \cap (-A_i) \right) = \emptyset$ and $A \subseteq \bigcup_{i =1}^k \left(A_i \cup (-A_i) \right)$}  \end{gathered} \right. \right\}. \end{equation} \end{lemma}

\begin{figure}[h]
\centering
\includegraphics[width=6cm, height=6cm]{Images/MTINAG2.png}
\caption{Example: the genus of $S^1$ is equal to $2$.}
\end{figure}

\begin{proof}Let $A_1, \, \dots, \, A_k$ be closed subspaces of $E$ such that
\begin{equation*}\left(A_i \cap (-A_i) \right) = \emptyset \quad \text{and} \quad A \subseteq \bigcup_{i =1}^k \left(A_i \cup (-A_i) \right). \end{equation*}
For every $i = 1, \, \dots, \, k$ we can define a continuous odd map by considering the function
\begin{equation*} \varphi_i(x) := \begin{cases} 1 & x \in A_i \\ - 1 & x \in - A_i \end{cases} \end{equation*}
and then extend it to a continuous odd map $\psi_i : A \to \R$. Then
\begin{equation*} \Psi := (\psi_1, \, \dots, \, \psi_k) : A \to \R^k \setminus \{0\} \end{equation*}
is a well defined map since $0$ does not belong to the image (by assumption the family $\{A_i, \, -A_i\}_{i = 1, \, \dots, \, n}$ forms a covering of $A$), and it is easy to check that $\psi$ is continuous and odd, that is,
\begin{equation*}\gamma(A) \leq k. \end{equation*}
Vice versa, let $\Phi : A \to \R^{k} \setminus \{0\}$ be a continuous odd map. Up to composing with a projection, we may always assume (see \hyperref[rmk:genussphere]{Remark \ref{rmk:genussphere}}) that $\Phi$ sends $A$ to the sphere $S^{k -1} \subset \R^k$. Clearly for every $u \in A$ it turns out that
\begin{equation*} \Phi(u) \in S^{k -1} \implies \sum_{i = 1}^k \Phi_i(u)^2 = 1 \implies \exists \, i \in \{1, \, \dots, \, n\} \: : \: |\Phi_i(u)| \geq \frac{1}{\sqrt{k}}, \end{equation*}
and thus we can define
\begin{equation*} A_i := \left\{u \in A \: \left| \: \Phi_i(u) \geq \frac{1}{\sqrt{k}} \right. \right\} \implies - A_i = \left\{u \in A \: \left| \: \Phi_i(u) \leq -\frac{1}{\sqrt{k}} \right. \right\}.\end{equation*}
The reader may check by herself that the family $\{A_i, \, - A_i\}_{i = 1, \, \dots, \, n}$ satisfies the required properties, and thus we can conclude that \eqref{eq:equiv} holds true.\end{proof}

\begin{corollary}Let $K$ be a compact subset of $E$ such that $0 \notin K$ and $K \in \Gamma$. Then the genus of $K$ is finite, that is,
\begin{equation*} \gamma(K) < + \infty. \end{equation*} \end{corollary}

\begin{proof}It is a straightforward consequence of \hyperref[gen:equiv]{Lemma \ref{gen:equiv}}. Indeed, for every $u \in K$ there exists an open neighborhood $V_u \ni u$ such that
\begin{equation*} \overline{V_u} \cap \left( \overline{- V_u} \right) = \emptyset. \end{equation*}
The family
\begin{equation*} \left\{ V_u, \, - V_u \right\}_{u \in K} \end{equation*}
is an open covering of $K$, hence there exists a finite subfamily 
\begin{equation*} \left\{ V_{u_i}, \, - V_{u_i} \right\}_{i = 1, \, \dots, \, M} \end{equation*}
that still covers $K$.
\end{proof}

\section{Relative Genus}

\paragraph{Definition and Main Properties.} Let $E$ be a normed space, and let us denote by $\Gamma$ the class of all the symmetric subsets $A \subseteq E \setminus \{0\}$ such that $A$ is closed in $E \setminus \{0\}$.

\begin{definition}[Genus] Let $A, \, B \in \Gamma$ such that $A \subseteq B$. The \textit{relative genus} of $B$ with respect to $A$ is defined by
\begin{equation*} \gamma(B, \, A) := \min \left\{k \in \N \: \left| \: \begin{gathered} 
\text{$\exists \, F_0, \, F \in \Gamma$ closed subsets such that} \\ \text{$A \subseteq F_0$, $A$ is a deformation retract of $F_0$,} \\ \text{$B = F_0 \cup F$ and $\gamma(F) \geq k$} \end{gathered} \right. \right\}, \end{equation*}
and $+ \infty$ if such a natural number $k$ does not exist. \end{definition}

\begin{remark} If $A = \emptyset$, then it turns out that $\gamma(B, \, A) = \gamma(B)$.\end{remark}

\begin{proposition} \label{propgenusrel}\mbox{}
\begin{enumerate}[label=\textbf{(\alph*)}]
\item $\gamma(\emptyset,  \, \emptyset) = 0$.
\item Let $A, \, B, \, C \in \Gamma$. If $A \subseteq B, \, C$ and there exists a continuous odd map $\Phi : B \to C$ such that $\Phi \, \big|_A = \mathrm{id}_A$, then
\begin{equation*}\gamma(C, \, A) \geq \gamma(B, \, A). \end{equation*}
\item Let $A, \, B, \, C \in \Gamma$. If $A \subseteq B, \, C$, then
\begin{equation*}\gamma\left( B \cup C, \, A \right) \leq \gamma(B, \, A) + \gamma(C). \end{equation*}
\end{enumerate} \end{proposition}

\begin{proof}\mbox{}
\begin{enumerate}[label=\textbf{(\alph*)}]
\setcounter{enumi}{1}
\item Let $F, \, F_0 \in \Gamma$ be the closed subsets given by the definition of $\gamma(C, \, A)$. Clearly
\begin{equation*} F^\prime := \Phi^{-1}(F) \qquad \text{and} \qquad F_0^\prime := \Phi^{-1}(F_0) \end{equation*}
belong to $\Gamma$, and their union is equal to $B$. Moreover, the map $\Phi$ is the identity on $A$, hence
\begin{equation*} A \subseteq F_0 \implies A = \Phi^{-1}(A) \subseteq \Phi^{-1}(F_0) = F_0^\prime.   \end{equation*}
Let $r : F_0 \to A$ be the deformation retraction; then
\begin{equation*} r \circ \Phi : F_0^\prime \to A \end{equation*}
is also a deformation retraction, and hence it follows from the definitions that
\begin{equation*}\gamma(C, \, A) \geq \gamma(B, \, A). \end{equation*}
\item Let $F, \, F_0 \in \Gamma$ be the closed subsets given by the definition of $\gamma(B, \, A)$. Hence
\begin{equation*} B \cup C = F_0 \cup F \cup C, \end{equation*}
and $A \subseteq F_0$ is still a deformation retract of $F_0$. Moreover, it turns out that
\begin{equation*} \gamma \left(F \cup C \right) \leq \gamma(F) + \gamma(C) = \gamma(B, \, A) + \gamma(C), \end{equation*}
and the latter term is independent of $F$, hence we infer that
\begin{equation*}\gamma\left( B \cup C, \, A \right) \leq \gamma(B, \, A) + \gamma(C). \end{equation*}
\end{enumerate} \end{proof}

\section{Relative Genus in Calculus of Variations}

Let $H$ be a Hilbert space, and let $X \subseteq E$ be a complete submanifold without boundary of class $C^2$ symmetric with respect to the origin, i.e.,
\begin{equation*} u \in X \implies - u \in X. \end{equation*}
In this section, the functional $f : X \to \R$ will always be assumed to be \textbf{even} and of class $C^1(X)$.

\begin{notation}Let $a_0 \in \R$ be a real number, and let $k \in \N$ be a natural number such that
\begin{equation*} 1 \leq k \leq \gamma \left(X, \, f^{(a_0)} \right). \end{equation*}
We denote by $\gamma(a_0)_k$ the infimum of all the sublevels such that the relative genus is at least $k$, that is,
\begin{equation*} \gamma(a_0)_k := \inf \left\{ c \in \R_{\geq a_0} \: \left| \: \gamma \left( f^{(c)}, \, f^{(a_0)} \right) \geq k \right. \right\}. \end{equation*}
Moreover, it is possible that $\gamma \left( f^{(c)}, \, f^{(a_0)} \right) \geq k$ is not satisfied for any real number $c \in \R$. In this case, we set $\gamma(a_0)_k := + \infty$ and, clearly, the supremum of $f$ must be $+ \infty$. \end{notation}

\begin{lemma}[Fundamental Lemma] \label{genrel:fl}Let $a_0 \in \R$ be a real number, and let $k \in \N$ be a natural number such that $1 \leq k \leq \gamma\left(X, \, f^{(a_0)} \right)$. \mbox{}
\begin{enumerate}[label=\textbf{(\alph*)}]
\item $a_0 \leq \gamma(a_0)_1 \leq \gamma(a_0)_2 \leq \dots \leq \gamma(a_0)_k \leq \sup \, f$.
\item If $\gamma(a_0)_k \in \R$ and $f$ satisfies the $\left( \mathrm{PS} \right)_{\gamma(a_0)_k}$ condition, then $\gamma(a_0)_k$ is a critical level for the functional $f$.
\item If $1 \leq k \leq k + h \leq \gamma \left(X, \, f^{(a_0)} \right)$ are natural numbers such that $\gamma(a_0)_k = \gamma(a_0)_{k+1} = \dots = \gamma(a_0)_{k + h} \in \R$, and $f$ satisfies the $\left( \mathrm{PS} \right)_{\gamma(a_0)_k}$ condition, then for the set of all the critical points of $f$, denoted by $Z_{\gamma(a_0)_k}$, it results
\begin{equation*} \gamma \left( Z_{\gamma(a_0)_k} \right) \geq h + 1. \end{equation*}
\end{enumerate} \end{lemma}

The proof of this fundamental result relies on a more precise formulation of the deformation lemma (i.e., we do not require $\epsilon$ and $\delta$ to be equal).

\begin{lemma}[Deformation Lemma, Variant II] \label{deflemma2symrel}Let $f : X \to \R$ be a functional, and let $c$ be a critical value of $f$. If we set
\begin{equation*} \mathcal{Z}_c := \left\{ u \in X \: \left| \: \mathrm{grad} \, f(u) = 0, \, f(u) = c \right. \right\} \end{equation*}
to be the set of critical points, and we assume that the Palais-Smale condition at the level $c$ holds true and that there exists $V$ open neighborhood of $\mathcal{Z}_c$, then there exist $\epsilon_0 > 0$ such that for any $\epsilon, \, \delta \in [0, \, \epsilon_0]$ there is an homotopy
\begin{equation*} H : [0, \, 1] \times \left( \left( f^{(c+\epsilon)} \setminus V \right) \cup f^{(c - \delta)} \right) \to f^{(c+\epsilon)} \end{equation*}
satisfying the following properties:
\begin{enumerate}[label=\textbf{(\arabic*)}]
\item $H(0, \, u) = u$.
\item $H(1, \, u) \in f^{(c - \epsilon)}$.
\item $H(t, \, u) = u$ for every $u \in f^{(c - \delta)} \setminus V$ and every $t \in [0, \, 1]$.
\end{enumerate}
Moreover, if $f$ is an even functional, then the homotopy is odd at every level (i.e., $H_t$ is odd for every $t \in [0, \, 1]$). \end{lemma}

\begin{proof}[Proof of Lemma \ref{genrel:fl}] We observe that it suffices to prove \textbf{(c)}. Indeed, the first assertion is obvious by definition, while the second immediately follows from the third one.

\paragraph{Step 0.} By \hyperref[gen:fl]{Lemma \ref{gen:fl}} one can always find an open symmetric neighborhood $V$ of $Z_c$ such that
\begin{equation*} \gamma \left( \overline{V} \right) = \gamma(Z_c). \end{equation*}

\paragraph{Case 1 ($c > a_0$).} There exists $\epsilon > 0$ such that $c - \epsilon > a_0$, and there exists a $\gamma$-retraction
\begin{equation*} r : \left( f^{(c+ \epsilon)} \setminus V \right) \cup f^{(c - \epsilon)} \to f^{(c - \epsilon)} \end{equation*}
such that $r \, \big|_{f^{(a_0)}} \equiv \mathrm{id}_{f^{(a_0)}}$. It follows that
\begin{equation*} \begin{aligned} k + h & \leq \gamma \left(f^{(c+ \epsilon)}, \, f^{(a_0)} \right) \, {\color{red}\leq} \, \gamma \left( \left( f^{(c+\epsilon)} \setminus V \right) \cup \overline{V}, \, f^{(a_0)} \right) \, {\color{blue}\leq} \\ & \, {\color{blue}\leq} \, \gamma \left( \left( f^{(c+\epsilon)} \setminus V \right) \cup f^{(c-\epsilon)}, \, f^{(a_0)} \right) + \gamma \left( \overline{V} \right) \, {\color{green}\leq} \\ & \, {\color{green}\leq} \, \gamma \left( f^{(c-\epsilon)}, \, f^{(a_0)} \right) + \gamma \left( \overline{V} \right), \end{aligned}\end{equation*}
where the {\color{red}red} inequality is the monotony of the genus, the {\color{blue}blue} inequality is the subadditivity of the genus and the {\color{green}green} inequality is given by the fact that $c - \epsilon > a_0$. Finally, since $c = \gamma(a_0)_k$, then it turns out that
\begin{equation*}\gamma \left( f^{(c-\epsilon)}, \, f^{(a_0)} \right) + \gamma \left( \overline{V} \right) \leq k + \gamma(Z_c),\end{equation*}
which is exactly what we wanted to prove.

\paragraph{Case 2 ($c = a_0$).} It is easy to prove that $k = 1$ and
\begin{equation*} \gamma(a_0)_1 = \dots = \gamma(a_0)_{1 + h} = c. \end{equation*}
Hence there exists a symmetric open neighborhood $V$ of $Z_c$ such that
\begin{equation*} \gamma \left( \overline{V} \right) = \gamma(Z_c), \end{equation*}
and by \hyperref[deflemma2symrel]{Deformation Lemma Variant \ref{deflemma2symrel}} it turns out that
\begin{equation*} \begin{aligned} 1 + h & \leq \gamma \left( f^{(c+\epsilon)}, \, f^{(a_0)} \right) \leq \\ & \leq \gamma \left( \left( f^{(c+\epsilon)}\setminus V \right) \cup f^{(c)}, \, f^{(a_0)} \right) + \gamma \left( \overline{V} \right) \, {\color{blue}=} \\ & \, {\color{blue}=}\,  \gamma \left( \overline{V} \right) = \gamma(Z_c), \end{aligned}  \end{equation*} 
where the {\color{blue}blue} equality is given by the fact that
\begin{equation*}\left( f^{(c+\epsilon)}\setminus V \right) \cup f^{(c)} = \emptyset. \end{equation*}
\end{proof}

%TEOREMA FALSO

\paragraph{Motivational Example.} Let $E$ be a Hilbert space, and let $f : E \to \R$ be an even functional of class $C^2(X)$ such that $f(0) = 0$. Suppose that there is a decomposition
\begin{equation*} E = X_0 \oplus X_1 \oplus X_2, \end{equation*}
and let us denote by $S_\rho$ the sphere in $X_2 \oplus X_3$ of radius $\rho$, and by $S_R$ the sphere in $X_1\oplus X_2$ of radius $R$. Let $0 < \rho < R$, and assume that
\begin{equation} \label{sdrrrd} \sup_{x \in S_R} f(x) \leq 0 = f(0) \leq \inf_{x \in S_\rho} f(x). \end{equation}
The functional $f$ is even, hence the gradient $\mathrm{grad} \, f$ is odd and $\mathrm{grad} \, f(0) = 0$. Let us denote by $f^{\prime \prime}(0) \, |h|^2$ the Hessian matrix associated to $f$, that is,
\begin{equation*} f^{\prime \prime}(0) \, |h|^2 = \left( Hf(0) \, h, \, h \right). \end{equation*}
The direct sum $X_2 \otimes X_3$ is the space of eigenvectors associated to positive eigenvalues, i.e.,
\begin{equation} \label{sdrrrd2} \forall \, h \in X_2 \otimes X_3 \leadsto f^{\prime\prime}(0) \, |h|^2 \geq k \, |h|^2, \end{equation}
and it follows from \eqref{sdrrrd} that
\begin{equation*} \lim_{|r| \to + \infty} f(r) = - \infty. \end{equation*}

\begin{theorem}\label{thhhhtt}Let
\begin{equation*} b := \sup_{x \in B_R} \, f(x) \qquad \text{and} \qquad a := \inf_{x \in S_\rho} \, f(x), \end{equation*}
and assume that $b \in \R$ (i.e., $b$ is finite). Then
\begin{equation*} \gamma \left( f^{(b)}, \, f^{(a)} \right) \geq \mathrm{dim} \, X_2. \end{equation*}\end{theorem}

To prove this theorem, we first need to state and prove some technical results.

\begin{remark} If $A \in \Gamma$ and $\varphi : A \to \R^k$ is a continuous odd map, then the genus of $A_0 := \varphi^{-1} \left( \{0\} \right)$ is bounded from below by
\begin{equation} \label{belowesd} \gamma(A_0) \geq \gamma(A) - k. \end{equation} \end{remark}

\begin{proof}By \hyperref[propgenus]{Proposition \ref{propgenus}} there exists an open neighborhood $V_0$ of $A_0$ such that
\begin{equation*} A_0 \subseteq \accentset{\circ}{V_0} \qquad \text{and} \qquad \gamma\left( \overline{V_0} \right) = \gamma(A_0). \end{equation*} %RIC.
The map $\varphi$ sends $A \setminus \accentset{\circ}{V_0}$ in $\R^k \setminus \{0\}$, hence
\begin{equation*} \gamma \left( A \setminus \accentset{\circ}{V_0} \right) \leq k. \end{equation*}
On the other hand, it turns out that
\begin{equation*} A = \left( A \setminus \accentset{\circ}{V_0} \right) \cup \overline{V_0}, \end{equation*}
hence by the subadditivity of the genus
\begin{equation*} \gamma(A) \leq \gamma \left( A \setminus \accentset{\circ}{V_0} \right) + \underbrace{\gamma \left( \overline{V_0} \right)}_{= \gamma(A_0)} \leq k + \gamma(A_0),\end{equation*}
which is exactly \eqref{belowesd}.
\end{proof}

\begin{lemma} Let $h : B_R \to E$ be a continuous odd map such that $h \, \big|_{S_R} \equiv \mathrm{id}_{S_R}$. Then
\begin{equation} \label{belowesd1} \gamma \left( h^{-1}(S_{\rho}) \right) \geq \mathrm{dim} \, X_2. \end{equation} \end{lemma}

\begin{proof} Let us define the set
\begin{equation*} U:= \left\{ u \in \overline{B_R} \: \left| \: \| h(u) \| < \rho \right. \right\}. \end{equation*}
Clearly, $U$ is an open symmetric set containing the origin $0$. On the other hand, if $u \in S_R$, then by assumption
\begin{equation*} h(u) = u \implies \|h(u)\| = R > \rho, \end{equation*}
and this implies that
\begin{equation*} \overline{U} \subseteq B_R \qquad \text{and} \qquad \|h(u)\| = \rho \quad \forall \, u \in \partial \, U, \end{equation*}
where $\partial \, U$ denotes the boundary in $X_1 \oplus X_2$. Let $P : E \to X_1$ be the projection onto $X_1$ such that $\mathrm{ker}(P) = X_2 \oplus X_3$; then
\begin{equation*}h^{-1} (S_\rho) \supseteq \left\{ u \in \partial \, U \: \left| \: P \circ h (u) = 0 \right. \right\}. \end{equation*}
But $P \circ h$ is an odd map, hence, if we assume $\mathrm{dim} \, X_1 = k <+ \infty$ and $\mathrm{dim} \, X_2, \, \mathrm{dim} \, X_3 < + \infty$, then
\begin{equation} \label{eq12233} \gamma \left(  \left\{ u \in \partial \, U \: \left| \: P \circ h (u) = 0 \right. \right\} \right) \geq \gamma( \partial \, U ) - k. \end{equation}
By \hyperref[b]{Borsuk Theorem \ref{b}} it turns out that
\begin{equation*} \gamma(\partial \, U) = \mathrm{dim} \, X_1 +\mathrm{dim} \, X_2 \end{equation*}
since $U$ is a neighborhood of $0$ in $X_1 \oplus X_2$, and thus \eqref{eq12233} concludes the proof.
\end{proof}

\begin{proposition} Let $A, \, B \in \Gamma$ be sets such that $A \subseteq B$, $\overline{B_R} \subseteq B$, and let $A := f^{(a - \epsilon)}$. Then
\begin{equation*} A \cap S_\rho = \emptyset \qquad \text{and} \qquad A \supseteq S_R \quad \text{if $a - \epsilon > 0$}, \end{equation*}
and it turns out that
\begin{equation} \label{ths.dd} \gamma(B, \, A) \geq \mathrm{dim} \, X_2. \end{equation} \end{proposition}

\begin{proof} Let $F, \, F_0 \in \Gamma$ be such that $B = F_0 \cup F$. Suppose that $A \subseteq F_0$, and let $r :F_0 \to A$ be a $\gamma$-retraction. The thesis \eqref{ths.dd} is equivalent to proving that
\begin{equation*}\gamma(F) \geq \mathrm{dim} \, X_2. \end{equation*}
In particular, there exists a odd deformation $r^\ast : \overline{B_R} \to E$ which is the identity on the sphere $S_R$; more precisely, $r^\ast$ is a retraction of $F_0 \cap \overline{B_R}$ to $A$. Then
\begin{equation*} F \subseteq (r^\ast)^{-1}(S_\rho) \end{equation*}
since if $u$ is an element such that $r^\ast(u) \in S_\rho$, then $u \notin F_0$ ($F_0 \supseteq A$) since $r^\ast$ is fixed on $A$ and $S_\rho \cap A = \emptyset$.
Then it follows from \hyperref[belowesd1]{Lemma \ref{belowesd1}} that
\begin{equation*} \gamma(F) \geq \gamma \left( h^{-1} (S_\rho) \right) \geq \mathrm{dim} \, X_2, \end{equation*}
and this concludes the proof.\end{proof}

We proved that, if $a - \epsilon >0$, then
\begin{equation*} \gamma \left( f^{(b)}, \, f^{(a - \epsilon)} \right) \geq \mathrm{dim} \, X_2, \end{equation*}
that is, the statement of \hyperref[thhhhtt]{Theorem \ref{thhhhtt}} is proved.

\paragraph{Conclusion.} To conclude this section, we introduce some fundamental results without proving them; the reader may do it as an exercise since they quickly follow from the tools we have introduced so far.

\begin{proposition}Let $a < b$ be real numbers, and assume that $f(0) \notin [a, \, b]$ and $f$ satisfies the Palais-Smale condition $(\mathrm{PS})_c$ for every $c \in [a, \, b]$. The following assertion hold true: \mbox{}
\begin{enumerate}[label=\textbf{(\alph*)}]
\item  If $\gamma \left( f^{(a)} \right)$ is finite, then $\gamma \left( f^{(b)} \right)$ is also finite.
\item The relative genus is finite, i.e.,
\begin{equation*} \gamma \left( f^{(b)}, \, f^{(a)} \right) < + \infty. \end{equation*}
\item Assume that $f$ is bounded from below on $X$, $0 \notin f^{(b)}$, and assume also that $f$ satisfies the Palais-Smale condition $(\mathrm{PS})_c$ for every $c \leq b$. Then
\begin{equation*} \gamma \left( f^{(b)}\right) < + \infty. \end{equation*}
\end{enumerate}\end{proposition}

\begin{theorem}Assume that $f$ is a functional bounded from below on $X$, and assume also that $f$ satisfies the  Palais-Smale condition $(\mathrm{PS})_c$ for every $c$. If $\gamma(X) = + \infty$, then
\begin{enumerate}[label=\textbf{(\alph*)}]
\item there are infinitely many critical points, i.e.,
\begin{equation*} \symbol{35} \, Z_f = + \infty; \end{equation*}
\item the supremum is infinite, i.e.
\begin{equation*} \sup_{x \in X} f(x) = \sup \left(Z_f \right) = + \infty. \end{equation*}
\end{enumerate}
\end{theorem}

\section{Lusternik-Schnirelman Category}

Let $X$ be a topological space. We will introduce the notion of the \textit{Lusternik-Schnirelman category} using the class of closed set, but this is by no means necessary (though, it is the most used in literature), and one could make a different choice.

\begin{definition}[Category] Let $A \subseteq X$ be a closed and nonempty set. The category of $A$ in $X$ is given by
\begin{equation*} \mathrm{cat}_X(A) := \min \left\{k \in \N \setminus \{0\} \: \left| \: \begin{gathered}\text{$\exists \, F_1, \, \dots, \, F_k \subset X$ closed contractible} \\ \text{subsets such that $A \subset \bigcup_{i = 1}^{k} F_i$} \end{gathered} \right. \right\} \end{equation*}
and, if no such $k$ exists, then we set $ \mathrm{cat}_X(A) := + \infty$. Moreover,
\begin{equation*} \mathrm{cat}_X(\emptyset) = 0. \end{equation*}\end{definition}

\begin{example}Let $X = \R^N \setminus \{0\}$, and let $S$ be the sphere $S^{N-1} \subset \R^N$. It turns out that
\begin{equation*} \mathrm{cat}_X(S) = 2 \end{equation*}
since the sphere is not contractible in $X$. Indeed, by \hyperref[43203o3]{Theorem \ref{43203o3}} the homotopy would cover the whole ball $B^{N-1}$, but this is not possible since $0$ doesn't belong to the space. In a similar fashion one can argue that, if $\Omega \subseteq \R^N$ and $p_0 \in \Omega$, then
\begin{equation*} \mathrm{cat}_{\R^N \setminus \{p_0\}}( \partial \, \Omega) \geq 2. \end{equation*}\end{example}

\begin{example}The category of the torus $T = S^1 \times S^1$ in itself is equal to $3$, i.e.,
\begin{equation*}\mathrm{cat}_T(T) = 3, \end{equation*}
but this is a highly nontrivial result. %REF
\end{example}

\begin{remark} \mbox{}
\begin{enumerate}[label=\textbf{(\alph*)}]
\item If $X$ is connected and $A$ is a finite set, then
\begin{equation*}\mathrm{cat}_X(A) = 1. \end{equation*}
\item If for all $p \in X$ there exists a closed and contractible neighborhood $U_p$ of $p$, then a compact set $A \subseteq X$ has finite category, i.e.,
\begin{equation*}\mathrm{cat}_X(A) < + \infty. \end{equation*}
\end{enumerate}\end{remark}

\begin{proposition} \label{propcat}\mbox{}
\begin{enumerate}[label=\textbf{(\alph*)}]
\item Let $A \subseteq X$ be a closed set. Then $\mathrm{cat}_X(A) = 0$ if and only if $A = \emptyset$.
\item Let $A \subseteq B \subseteq X$ be closed sets. The category is monotone, i.e.,
\begin{equation*} \mathrm{cat}_X(A) \leq \mathrm{cat}_X(B). \end{equation*}
Moreover, if $B$ contains a deformation of $A$, then
\begin{equation*} \mathrm{cat}_X(A) \leq \mathrm{cat}_X(B). \end{equation*}
\item Let $A, \, B \subseteq X$ be closed sets. The category is subadditive, i.e.,
\begin{equation*} \mathrm{cat}_X(A \cup B) \leq \mathrm{cat}_X(A) + \mathrm{cat}_X(B). \end{equation*}
\item If $X$ is a complete Banach manifold of class $C^1$, then for any $A \subseteq X$ closed there is a neighborhood $V$ of $A$ whose closure has the same category as $A$, i.e.,
\begin{equation*} \mathrm{cat}_X\left( \overline{V} \right) = \mathrm{cat}_X(A). \end{equation*}
\end{enumerate} \end{proposition}

\begin{proof}We only prove the second assertion of \textbf{(b)}. Let
\begin{equation*} H : [0, \, 1] \times A \to B \end{equation*}
be the deformation of $A$, and let $F_0, \, \dots, \, F_k$ be the closed sets given by the definition of category associated with $B$, i.e.,
\begin{equation*} B \subset \bigcup_{i = 1}^k F_i. \end{equation*}
For every $i = 1, \, \dots, \, k$ it turns out that
\begin{equation*} F_i^\prime := \left\{ u \in A \: \left| \: H(1, \, u) \in F_i \right. \right\} \end{equation*}
is a closed set, and it is easy to check that
\begin{equation*} A \subset \bigcup_{i=1}^k F_i^\prime. \end{equation*}
Let $H_i : [0, \, 1] \times F_i \to X$ be the homotopy between $\mathrm{id}_{F_i}$ and the constant map $u_{0, \, i} \in F_i$. Then we can easily define a deformation retraction of $F_i^\prime$ onto a point as follows:
\begin{equation*} H_i^\prime : [0, \, 1] \times F_i^\prime \to X, \qquad H_i^\prime(t, \, u) := \begin{cases} H(2t, \, u) & t \in \left[0, \, \frac{1}{2} \right] \\[1em] H_i(2t - 1, \, u) & t \in \left[ \frac{1}{2}, \, 1 \right].  \end{cases} \end{equation*}
The reader may check by herself that $H_i^\prime$ is well defined at $t = 1/2$ for every $i = 1, \, \dots, \, k$. \end{proof}

\begin{remark}Let $A \subseteq Y \subseteq X$. \mbox{}
\begin{enumerate}[label=\textbf{(\arabic*)}]
\item If $Y$ is closed in $X$ and $A$ is closed in $Y$, then
\begin{equation*} \mathrm{cat}_X \, A \leq \mathrm{cat}_X \, B. \end{equation*}
\item If $Y$ is a deformation retract of $X$, then
\begin{equation*} \mathrm{cat}_X \, A = \mathrm{cat}_X \, B. \end{equation*}
\item If $F \subseteq X$ is contractible in $X$, then $F \cap Y$ is contractible in $Y$.
\end{enumerate}\end{remark}

\begin{remark}Let $X$ be a Riemannian manifold, i.e., a complete manifold of class $C^2$ without boundary.
\begin{enumerate}[label=\textbf{(\arabic*)}]
\item If $X$ is locally contractible (that is, for every $x \in X$ there is a neighborhood $U_x \ni x$ contractible in $X$), then the category of a compact subset $K \subseteq X$ is finite.
\item If $X$ is connected and $A \subset X$ is finite, then $\mathrm{cat}_X \, A = 1$.
\end{enumerate}\end{remark}

\begin{remark}The main issue of the category is that it highly depends on the ambient space. For example, if $X = \R^2 \setminus \{0\}$, $A = [0, \, 1]$ and $B = S^1$, then one can check that
\begin{equation*} \mathrm{cat}_X \, A < \mathrm{cat}_X \, B. \end{equation*} \end{remark}

\section{Category in Calculus of Variations}

Let $X$ be a Hilbert space (or, more generally, a complete Riemannian manifold of class $C^2$ without border), and let $f : X \to \R$ be a functional of class $C^1(X)$.

\begin{definition} Let $k$ be an integer such that $1 \leq k \leq \mathrm{cat}_X(X)$. The $k$-th essential critical level of $f$ is defined by setting
\begin{equation*} c_k = \inf \left\{ c \in \R \: \left| \: \mathrm{cat}_X \, f^{(c)} \geq k \right. \right\}, \end{equation*}
and, if no such $c$ exists, we set $c_k = + \infty$.
\end{definition}

\begin{lemma}\label{cat:fl} Let $k \in \N$ be a natural number such that $1 \leq k \leq \mathrm{cat}_X(X)$. \mbox{}
\begin{enumerate}[label=\textbf{(\alph*)}]
\item The sequence is bounded and increasing, i.e.,
\begin{equation*} \inf_{x \in X} f(x) = c_1 \leq c_2 \leq \dots \leq c_k \leq \sup_{x \in X} f(x). \end{equation*}
\item If $c_k \in \R$ and $f$ satisfies the $\left( \mathrm{PS} \right)_{c_k}$ condition, then $c_k$ is a critical level for the functional $f$.
\item Let $1 \leq k \leq k + h \leq \mathrm{cat}_X(X)$. If $c_k = c_{k+1} = \dots = c_{k + h} \in \R$ and $f$ satisfies the $\left( \mathrm{PS} \right)_{c_k}$ condition, then for the set of all the critical points of $f$, denoted by $Z_{c_k}$, it results
\begin{equation*} \mathrm{cat}_X \, Z_{c_k}  \geq h + 1. \end{equation*}
\end{enumerate} \end{lemma}

\begin{proof} We observe that it suffices to prove \textbf{(c)}. Indeed, the first assertion is obvious by definition, while the second immediately follows from the third one.

\paragraph{Step 0.} By \hyperref[propcat]{Proposition \ref{propcat}} one can always find an open neighborhood $V$ of $Z_c$ such that
\begin{equation*} \mathrm{cat}_X \, \overline{V} =\mathrm{cat}_X \, Z_c. \end{equation*}

\paragraph{Step 1.} By \hyperref[deflemma2]{Deformation Lemma \ref{deflemma2}} there is a real number $\epsilon > 0$ and there is a deformation retraction
\begin{equation*} H : [0, \, 1] \times \left( f^{(c+ \epsilon)} \setminus V \right) \to f^{(c - \epsilon)}. \end{equation*}
It follows that
\begin{equation*} \begin{aligned} k + h & \leq \mathrm{cat}_X \, f^{(c+ \epsilon)} \, {\color{blue}\leq} \\ & \, {\color{blue}\leq} \,  \mathrm{cat}_X \, \left( f^{(c+\epsilon)} \setminus V \right) +  \mathrm{cat}_X \, \left( \overline{V} \right) \, {\color{green}\leq} \\ & \, {\color{green}\leq} \,  \mathrm{cat}_X \, f^{(c-\epsilon)} +  \mathrm{cat}_X \, Z_c, \end{aligned}\end{equation*}
where the {\color{blue}blue} inequality is the subadditivity of the genus and the {\color{green}green} inequality is given by the fact that $f^{(c - \epsilon)}$ is a retraction by deformation of $ f^{(c+\epsilon)} \setminus V$ (see \hyperref[propcat]{Proposition \ref{propcat}}, \textbf{(b)}). It turns out that
\begin{equation*} k + h \leq \mathrm{cat}_X \, f^{(c-\epsilon)} + \mathrm{cat}_X \, Z_c < k + \mathrm{cat}_X \, _c \implies k + h \leq k - 1 + \mathrm{cat}_X \, Z_c, \end{equation*}
which is the thesis.
\end{proof}

\begin{theorem}Assume that $X$ is compact. \mbox{}
\begin{enumerate}[label=\textbf{(\alph*)}]
\item There are at least $\mathrm{cat}_X \, X$ critical points, i.e.,
\begin{equation*} \symbol{35} \, Z_f \geq \mathrm{cat}_X \, X. \end{equation*}
\item Assume that $X$ is connected. If $ \symbol{35} \, Z_f  < + \infty$, then there are exactly $\mathrm{cat}_X \, X$ distinct critical levels.
\end{enumerate} \end{theorem}

\begin{theorem}Assume that $a, \, b \in \R$ are two real numbers such that $a < b$, and assume that $f$ satisfies the Palais-Smale condition $\left( \mathrm{PS} \right)_c$ condition at every level $c \in \R$. Then
\begin{equation*} \mathrm{cat}_X \, f^{(b)} \leq \mathrm{cat}_X \, f^{(a)} + \symbol{35} \left\{ u \in Z_f \: \left| \: f(u) \in [a, \, b] \right. \right\}. \end{equation*} \end{theorem}

\begin{theorem}Let $X$ be a Hilbert space, and assume that there are two subspaces $X_0$ and $X_1$ such that
\begin{equation*} X = X_0 \oplus X_1 \qquad \text{and} \qquad \mathrm{dim} \, X_0 = N < + \infty. \end{equation*}
Let $f \in C^1(X; \; \R)$ be a differentiable functional, and let $W$ be an open subspace of $X_0$ containing the origin. If
\begin{equation*} \sup_{x \in \partial \, W} f(x) < \inf_{x \in X_1} f(x) \qquad \text{and} \qquad \inf_{x \in X} f(x) > - \infty \end{equation*}
and the Palais-Smale condition $(\mathrm{PS})_c$ holds for every $c <\sup_{x \in \partial \, W} f(x)$, then $f$ admits (at least) two critical points $u_1$ and $u_2$ such that $f(u_i) \leq \sup_{x \in \partial \, W} f(x) $. \end{theorem}

\begin{proof}Let us consider any real number
\begin{equation*} b \in \left( \sup_{x \in \partial \, W} f(x) , \, \inf_{x \in X_1} f(x) \right). \end{equation*}
By assumption it turns out that
\begin{equation*} f^{(b)} \supseteq \partial \, W \qquad \text{and} \qquad f^{(b)} \cap X_1 = \emptyset, \end{equation*}
hence the category $\mathrm{cat}_{f^{(b)}} \, f^{(b)}$ is bigger or equal than $\mathrm{cat}_{f^{(b)}} \, \partial \, W \geq 2$, since $\partial \, W$ is not contractible in $X \setminus X_1$.

By assumption the Palais-Smale condition holds at the level $b$; thus the number of critical points in $f^{(b)}$ is at least two. \end{proof} 

\begin{theorem}Assume that $f$ is bounded from below, i.e.,
\begin{equation*} \inf_{x \in X} f(x) > - \infty, \end{equation*}
and assume that $f$ satisfies the Palais-Smale condition $\left( \mathrm{PS} \right)_c$ at every level $c \in \R$. If $\mathrm{cat}_X \, X = + \infty$, then there are infinitely many critical points and the supremum of $f$ is $+ \infty$, that is,
\begin{equation*} \symbol{35} \, Z_f = \sup_{x \in Z_f} f(x) = + \infty. \end{equation*}\end{theorem}

\paragraph{Strict Sublevels.} Let $\bar{b}$ be any real number, and set
\begin{equation*} X_{\bar{b}} := \left\{u \in X \: \left| \: f(u) < \bar{b} \right. \right\}.\end{equation*}
Let $k \in \N$ be a natural number such that $1 \leq k \leq \mathrm{cat}_{X_{\bar{b}}} \, X_{\bar{b}}$, and let us denote by $\bar{c}_k$ the real number defined by
\begin{equation*} \bar{c}_k := \inf \left\{ c < \bar{b} \: \left| \: \mathrm{cat}_{X_{\bar{b}}} \, f^{(c)} \geq k \right. \right\}.\end{equation*}

\begin{lemma}\label{cat:flstrict} Let $\bar{b}$ be a real number grater or equal than $\inf_{x \in X}f(x)$, and let $k \in \N$ be a natural number such that $1 \leq k \leq \mathrm{cat}_{X_{\bar{b}}} \, X_{\bar{b}}$. \mbox{}
\begin{enumerate}[label=\textbf{(\alph*)}]
\item The sequence is bounded and increasing, i.e.,
\begin{equation*} \inf_{x \in X} f(x) = \bar{c}_1 \leq \bar{c}_2 \leq \dots \leq \bar{c}_k \leq \sup_{x \in X} f(x). \end{equation*}
\item If $\bar{c}_k \in \R$ and $f$ satisfies the $\left( \mathrm{PS} \right)_{\bar{c}_k}$ condition, then $\bar{c}_k$ is a critical level for the functional $f$.
\item Let $1 \leq k \leq k + h \leq \mathrm{cat}_{X_{\bar{b}}} \, X_{\bar{b}}$. If $\bar{c}_k = \bar{c}_{k+1} = \dots = \bar{c}_{k + h} \in \R$ and $f$ satisfies the $\left( \mathrm{PS} \right)_{\bar{c}_k}$ condition, then for the set of all the critical points of $f$, denoted by $Z_{\bar{c}_k}$, it results
\begin{equation*} \mathrm{cat}_{X_{\bar{b}}} \, Z_{\bar{c}_k}  \geq h + 1. \end{equation*}
In particular, even if $X_{\bar{b}}$ is not connected, it turns out that
\begin{equation*} \symbol{35} \, Z_{\bar{c}_k}  \geq h + 1. \end{equation*}
\end{enumerate} \end{lemma}

\begin{proof} The argument is similar to the one used in \hyperref[cat:fl]{Lemma \ref{cat:fl}}, provided that there exists an open neighborhood $V \supset Z_{\bar{c}_k}$ in $X_{\bar{b}}$ such that
\begin{equation*} \mathrm{cat}_{X_{\bar{b}}} \, Z_{\bar{c}_k} =  \mathrm{cat}_{X_{\bar{b}}} \, \overline{V}. \end{equation*}
The existence of $V$ is a straightforward consequence of the fact that $X_{\bar{b}}$ is a regular ($C^2$) manifold without border (here requiring that every $u \in X_{\bar{b}}$ satisfies the strict inequality $f(u) < \bar{b}$ is actually necessary: with this definition $X_{\bar{b}}$ is an open subset of $X$). \end{proof}

\begin{theorem}Assume that $a, \, b \in \R$ are two real numbers such that $a < b$, and assume that $f$ satisfies the Palais-Smale condition $\left( \mathrm{PS} \right)_c$ condition at every level $c \in \R$. If $\bar{b}$ is a real number strictly greater than $b$, then
\begin{equation} \label{usefulformula} \mathrm{cat}_{X_{\bar{b}}} \, f^{(b)} \leq \mathrm{cat}_{X_{\bar{b}}} \, f^{(a)} + \symbol{35} \left\{ u \in Z_f \: \left| \: f(u) \in [a, \, b] \right. \right\}. \end{equation} \end{theorem}

\begin{theorem}Let $X$ be a Hilbert space, and assume that there are two subspaces $X_0$ and $X_1$ such that
\begin{equation*} X = X_0 \oplus X_1 \qquad \text{and} \qquad \mathrm{dim} \, X_0 = N < + \infty. \end{equation*}
Let $f \in C^1(X; \; \R)$ be a differentiable functional, and assume that $f$ is bounded from belows and satisfies the saddle-point inequality, i.e.,
\begin{equation*} \sup_{x \in S_0} f(x) < \inf_{x \in X_1} f(x) \qquad \text{and} \qquad \inf_{x \in X} f(x) > - \infty. \end{equation*}
Let $\bar{b} \in \left( \sup_{x \in S_0} f(x), \, \inf_{x \in X_1} f(x) \right)$. Assume that the Palais-Smale condition $(\mathrm{PS})_c$ holds for every $c \in \left[\inf_{x \in X} f(x), \, \bar{b} \right]$. Then $f$ admits (at least) two critical points $u_1$ and $u_2$ such that
\begin{equation*} f(u_i) \leq \sup_{x \in S_0} f(x). \end{equation*} \end{theorem}

\begin{proof}Set
\begin{equation*} b := \sup_{x \in S_0} f(x), \qquad \text{and} \qquad a < \inf_{x \in X} f(x). \end{equation*}
By construction
\begin{equation*} S_0 \subseteq f^{(b)} \subseteq f^{\bar{b}} \qquad \text{and} \qquad f^{(\bar{b})} \cap X_1 = \emptyset, \end{equation*}
which implies that $f^{(b)}$ is not contractible in $X_{\bar{b}}$ (here we need the assumption on the dimension of $X_0$, otherwise $S_0$ would be an infinite-dimensional sphere $\leadsto$ contractible), and hence the category $\mathrm{cat}_{X_{\bar{b}}} \, f^{(b)}$ is bigger or equal than $2$.

Moreover, $f^{(a)} = \emptyset$ and consequently the category is equal to $0$. In conclusion, if we apply formula \eqref{usefulformula}, then it turns out that
\begin{equation*} \underbrace{\mathrm{cat}_{X_{\bar{b}}} \, f^{(b)}}_{=2} \leq \underbrace{\mathrm{cat}_{X_{\bar{b}}} \, f^{(a)}}_{=0} + \symbol{35} \left\{ u \in Z_f \: \left| \: f(u) \in [a, \, b] \right. \right\} \implies \symbol{35} \left\{ u \in Z_f \: \left| \: f(u) \in [a, \, b] \right. \right\} \geq 2, \end{equation*}
which is exactly what we wanted to prove.\end{proof} %RIUSA PER ES. SOPRA

\begin{remark}In the previous theorem, one can equivalently consider the subspaces X and X + v, for some vector $v \in X$. In this case, we take the sphere centered at $v$, but the assertion still holds true (since the $f$ need not be symmetric in the category setting, oppositely to what would happen in the genus setting). \end{remark}

\begin{exercise}Let $X$ be a Hilbert space, and assume that there are two subspaces $X_0$ and $X_1$ such that
\begin{equation*} X = X_0 \oplus X_1. \end{equation*}
Let $f \in C^1(X; \; \R)$ be an \textbf{even} differentiable functional, and assume that $f$ is bounded from belows and satisfies the saddle-point inequality, i.e.,
\begin{equation*} \sup_{x \in S_0} f(x) < \inf_{x \in X_1} f(x) \qquad \text{and} \qquad \inf_{x \in X} f(x) > - \infty. \end{equation*}
Let $\bar{b} \in \left( \sup_{x \in S_0} f(x), \, \inf_{x \in X_1} f(x) \right)$. Assume that the Palais-Smale condition $(\mathrm{PS})_c$ holds for every $c \in \left[\inf_{x \in X} f(x), \, \bar{b} \right]$. Then $f$ admits (at least) two critical points $u_1$ and $u_2$ such that
\begin{equation*} f(u_i) \leq \sup_{x \in S_0} f(x). \end{equation*} \end{exercise}

\section{Linking Theorem}

In this final section, we introduce a result, extending the mountain pass theorem, which deals with a more complex topological setting.

\begin{notation}Let $X$ be a Hilbert space, and assume that there are two subspaces $X_0$ and $X_1$ such that
\begin{equation*} X = X_0 \oplus X_1. \end{equation*}
Given $\rho_0, \, \rho_1 > 0$ and $e \in X_0$ a nonzero element, we can introduce the notation used in this section: 
\begin{equation*} \begin{aligned} & B_0 = \mathrm{Int}\left( B(0, \, {\rho_0}) \right) \cap X_0, \\[1em] & S_0 = \partial \, B(0, \, \rho_0) \cap X_0, \\[1em] & B_1 = \mathrm{Span}<e> \oplus \left(\mathrm{Int}\left( B(0, \, {\rho_1}) \right) \cap X_1\right), \\[1em] & S_1 = \mathrm{Span}<e> \oplus \left( \partial \, B(0, \, {\rho_1}) \cap X_1\right).\end{aligned} \end{equation*}\end{notation}

\begin{lemma}\label{lemma:linking} Assume that \mbox{}
\begin{enumerate}[label=\textbf{(\arabic*)}]
\item $X_0$ is finite-dimensional, and
\item the spheres $S_0$ and $S_1$ link, i.e.,
\begin{equation} \label{eq:link} - \rho_0 < \| e \| - \rho_1 < \rho_0 < \|e\| + \rho_1. \end{equation}
\end{enumerate}
Then the following assertions hold true: \mbox{}
\begin{enumerate}[label=\textbf{(\alph*)}]
\item For every continuous map $\Phi : \overline{B_0} \to X$ such that $\Phi \, \big|_{S_0} = \mathrm{id}_{S_0}$, it turns out that
\begin{equation*} \Phi(B_0) \cap S_1 \neq \emptyset. \end{equation*}
\item For every (continuous) homotopy $H : [0, \, 1] \times S_0 \to X$ such that
\begin{equation*} H(0, \, u) = u \qquad \text{and} \qquad H(t, \, u) \notin S_1\end{equation*}
for any $u \in S_0$ and $t \in [0, \, 1]$, it turns out that
\begin{equation*} H(1, \, S_0) \cap B_1 \neq \emptyset. \end{equation*}
\end{enumerate}\end{lemma}

\begin{remark}\label{remark:linking}Let $X^\prime$ be a topological space together with a homeomorphism $\Phi : X \to X^\prime$. The reader may check as a simple exercise that the assertions of \hyperref[lemma:linking]{Lemma \ref{lemma:linking}} hold true for $X^\prime$, where
\begin{equation*} \begin{aligned} & B_0^\prime = \Phi(B_0), \\[1em] & S_0^\prime = \Phi(S_0), \\[1em] & B_1^\prime = \Phi(B_1), \\[1em] & S_1^\prime = \Phi(S_1). \end{aligned} \end{equation*}
\end{remark}

\begin{proof}[Proof of Lemma \ref{lemma:linking}] We divide the argument into three steps.

\paragraph{Step 1.} As a consequence of \hyperref[remark:linking]{Remark \ref{remark:linking}}, we may always consider without loss of generality that $X_0 \perp X_1$ and $\|e\| \geq \rho_0$. Let $Q$ be the orthogonal projection onto $\mathrm{Span}<e> \oplus X_1$, and let us set
\begin{equation*} P := \mathrm{id}_X - Q : X \to P(X). \end{equation*}
Clearly $P(X) = \mathrm{Ker}(Q)$, hence $P$ is the orthogonal projection to a subspace $P(X)$ of $X_0$ which is orthogonal to the vector $e$, i.e,
\begin{equation*} X_0 = P(X) \oplus \mathrm{Span}<e>.\end{equation*}
Let us consider the map $\Psi : X \to X_0$ defined by setting
\begin{equation*} \Psi(u) = P(u) + \left( \|e\| - \| Q(u) - e \| \right) \, \frac{e}{\|e\|},\end{equation*}
and set $y_0 := \left( \|e\| - \rho_1 \right) \, \frac{e}{\|e\|}$. Notice that the condition \eqref{eq:link} implies that $y_0 \in B_0$. 

\paragraph{Step 2.} The main goal of this brief step is to prove that $\Psi$ satisfies the following properties:
\begin{enumerate}[label=\textbf{(\roman*)}]
\item If $u \in S_0$, then $\Psi(u) = u$.
\item If $u \in X$ and $\Psi(u) = \lambda \, \frac{e}{\|e\|}$, then $P(u) = 0$ and $\|u - e\| = \|e\| - \lambda$. In particular:
\begin{equation*}\begin{aligned} & u \in X \: : \: \Psi(u) = \lambda \, \frac{e}{\|e\|} \implies \lambda \leq \|e\|, \\[1em] & u \in X \: : \: \Psi(u) = y_0 \implies u \in S_1. \end{aligned} \end{equation*}
\end{enumerate}
Let $u \in S_0$. There exists $\lambda \in \R$ such that $Q(u) = \lambda \, \frac{e}{\|e\|}$ with $|\lambda| \leq \|e\|$. Moreover, it turns out that
\begin{equation*} \left\| Q(u) - e \right\| = \|e\| \, \left( \frac{\lambda}{\|e\|} - 1 \right) = \|e\| - \lambda, \end{equation*}
and thus
\begin{equation*}\Psi(u) = P(u) + \left( \|e\| - \|e\| + \lambda \right) \, \frac{e}{\|e\|} = P(u) + Q(u) = u. \end{equation*}
We now check the second assertion. Let $u \in X$ be a point such that
\begin{equation*}\Psi(u) = \lambda \, \frac{e}{\|e\|}. \end{equation*}
The projection $P$ is orthogonal to the linear span of $e$, hence $P(u) = 0$; thus
\begin{equation*}\lambda \, \frac{e}{\|e\|} = \Psi(u) = \left( \|e\| - \|e\| + \lambda \right) \, \frac{e}{\|e\|} \implies \lambda \leq \|e\|. \end{equation*}
Let $u \in X$ be a point such that $\Psi(u) = y_0 = \left( \|e\| - \rho_1 \right) \, \frac{e}{\|e\|}$. As before $P(u) = 0$ and $Q(u) = u$, hence
\begin{equation*} \left( \|e\| - \rho_1 \right) \, \frac{e}{\|e\|}= \Psi(u) = \left( \|e\| - \left\| u - e \right\| \right) \, \frac{e}{\|e\|} \implies \| u - e \| = \rho_1. \end{equation*}

\paragraph{Step 3.} We consider the map $\tilde{\Phi} := \Psi \circ \Phi : \overline{B_0} \to X_0$ (since we only developed the topological degree in finite-dimensional spaces). Clearly, by property \textbf{(i)} it turns out that
\begin{equation*}u \in S_0 \implies \Psi \circ \Phi(u) = \Psi(u) = u \implies \tilde{\Phi} \, \big|_{S_0} = \mathrm{id}_{S_0}, \end{equation*}
hence $\mathrm{deg}\left(y_0, \, \tilde{\Phi}, \, B_0 \right) = 1$. The solution property (\hyperref[prop:1239]{Proposition \ref{prop:1239}}) implies that there exists $u \in B_0$ such that $\tilde{\Phi}(u) = y_0$. The second assertion of \textbf{(ii)} proves that $\Psi(\Phi(u)) =y \implies \Phi(u) \in S_1$.

\vspace{1mm}
To prove \textbf{(b)}, we consider the homotopy $\tilde{H} := \Psi \circ H : [0, \, 1] \times S_0 \to X_0$ (for the very same reason of above). Clearly, by property \textbf{(i)} it turns out that
\begin{equation*}u \in S_0 \implies \Psi \circ H(u) = \Psi(u) = u \implies \tilde{H} \, \big|_{\{0\} \times S_0} = \mathrm{id}_{S_0}, \end{equation*}
and $y_0 \notin \tilde{H}(t, \, u)$ for any $(t, \, u) \in [0, \, 1] \times S_0$, hence $\mathrm{deg}\left(y_0, \, \tilde{H}(1, \, \cdot), \, B_0 \right) = 1$. By assumption, the point
\begin{equation*} y_1 := \left( \|e\| + \rho_1 \right) \, \frac{e}{\|e\|} \end{equation*}
does not belong to $B_0$, and $y_1 \notin \tilde{H}(t, \, u)$ for any $(t, \, u) \in [0, \, 1] \times S_0$ since $y_1 \notin \Psi(X)$ as it follows easily by the first assertion of \textbf{(ii)}. In particular, it turns out that
\begin{equation*}\mathrm{deg}\left(y_1, \, \tilde{H}(1, \, \cdot), \, B_0 \right) = 0. \end{equation*}
Therefore $y_0$ and $y_1$ cannot belong to the same connected component, and thus there must be a real number $\alpha_0 \in (\|e\| - \rho_1, \, \|e\| + \rho_1)$ and a point $u \in S_0$ such that
\begin{equation*} \tilde{H}(1, \, u) = \alpha_0 \, \frac{e}{\|e\|}. \end{equation*}
The property \textbf{(ii)} now implies that $P \left( H(1, \, u) \right) = 0$ and $\| H(1, \, u) - e \| = \|e\| - \alpha_0 < \rho_1$, that is, $H(1, \, u) \in B_1$ which is exactly what we wanted to prove. \end{proof}

\begin{theorem}[Linking] \label{theorem:linking} Assume that \mbox{}
\begin{enumerate}[label=\textbf{(\arabic*)}]
\item $X_0$ is finite-dimensional,
\item $\sup_{x \in S_0} f(x) < \inf_{x \in S_1} f(x)$, and
\item the spheres $S_0$ and $S_1$ link, i.e.,
\begin{equation} \label{eq:link} - \rho_0 < \| e \| - \rho_1 < \rho_0 < \|e\| + \rho_1. \end{equation}
\end{enumerate}
Then the following assertions hold true: \mbox{}
\begin{enumerate}[label=\textbf{(\alph*)}]
\item If $f$ satisfies the Palais-Smale condition $\left( \mathrm{PS} \right)_c$ for every $c \in \left[ \inf \, f(S_1), \, \sup \, f(B_0) \right]$, then there exists (at least) a critical level in that interval.
\item If $f$ is bounded from below on $B_1$, i.e.,
\begin{equation*} \inf_{x \in B_1} \, f(x) > - \infty, \end{equation*}
and $f$ satisfies the Palais-Smale condition $\left( \mathrm{PS} \right)_c$ for every $c \in \left[ \inf \, f(B_1), \, \sup \, f(S_0) \right]$, then there exists (at least) a critical level in that interval.
\end{enumerate}\end{theorem}

\begin{proof}\mbox{}
\begin{enumerate}[label=\textbf{(\alph*)}]
\item We argue by contradiction. Let $a := \inf_{x \in S_0} \, f(x)$ and $b := \sup_{x \in B_0} \, f(x)$; by the \hyperref[deflemma]{Deformation Lemma \ref{deflemma}} it turns out that there exists $\epsilon > 0$ such that
\begin{equation*} a < b - \epsilon \qquad \text{and} \qquad r : f^{(b)} \to f^{(a - \epsilon)} \end{equation*}
is a deformation retraction. The reader may easily check that
\begin{equation*}\overline{B_0} \subseteq f^{(b)}, \qquad S_0 \subseteq f^{(a-\epsilon)} \quad \text{and} \quad f^{(a-\epsilon)} \cap S_1 = \emptyset.\end{equation*}
Thus the restriction $\Phi := r \, \big|_{\overline{B_0}}$ is a continuous map which is the identity on $S_0$; but
\begin{equation*}\Phi(B_0) \cap S_1 = \emptyset, \end{equation*}
and this is in contradiction with \textbf{(a)} of \hyperref[lemma:linking]{Lemma \ref{lemma:linking}}.
\item We argue by contradiction. Let $a := \inf_{x \in S_0} \, f(x)$ and $b := \sup_{x \in S_0} \, f(x)$; by the \hyperref[deflemma]{Deformation Lemma \ref{deflemma}} it turns out that there exists $\epsilon > 0$ and a homotopy $H : [0, \, 1] \times f^{(b)} \to f^{(b)}$ with the following properties: \mbox{}
\begin{enumerate}[label=\textbf{\arabic*)}]
\item $H(0, \, u) = u$ for every $u \in f^{(b)}$.
\item $H(1, \, u) \in f^{(a - \epsilon)}$ for every $u \in f^{(b)}$.
\item $H(1, \, u) = u$ for every $u \in f^{(a - \epsilon)}$.
\end{enumerate}
On the other hand, the reader may check that
\begin{equation*}S_0 \subseteq f^{(b)}, \qquad S_1 \cap f^{(b)} = \emptyset \quad \text{and} \quad f^{(a-\epsilon)} \cap \overline{B_1} = \emptyset.\end{equation*}
Therefore the restriction $\tilde{H} := H \, \big|_{[0, \, 1] \times S_0}$ satisfies the assumptions of the assertion \textbf{(b)} in \hyperref[lemma:linking]{Lemma \ref{lemma:linking}}, but $\tilde{H}(1, \, u) \notin B_1$ for any $u \in S_0$ (in contradiction with the conclusion of the mentioned Lemma).
\end{enumerate} \end{proof}
\chapter{Jordan Theorem}

\section{Multiplicative Property of the Topological Degree}

\paragraph{Setting.} Let $\Omega, \, W \subseteq \R^N$ open subsets of $\R^N$, and assume that $\Omega$ is bounded and
\begin{equation*} \Phi : \overline{\Omega} \to W \qquad \text{and} \qquad \Psi : W \to \R^N\end{equation*}
are continuous maps.

\begin{notation}We denote by $\left(W_i\right)_{i \in \N}$ the collection of the connected components of $\R^N \setminus \Phi(\partial \, \Omega)$, and we reserve $W_0$ for the unbounded one (see \hyperref[fig:jt1]{Figure \ref{fig:jt1}}). \end{notation}

\begin{figure}[h]
\centering
\includegraphics[width=10cm, height=6cm]{Images/MTAGTJ1.png}
\caption{Setting}
\label{fig:jt1}
\end{figure}

\begin{remark}It may happen that an element of the collection $W_i$, for some $i \geq 1$, is not contained in $W$ (see \hyperref[fig:jt2]{Figure \ref{fig:jt2}}).  \end{remark}

\begin{figure}[h]
\centering
\includegraphics[width=10cm, height=6cm]{Images/MTAGTJ2.png}
\caption{$W_1$ is not contained in the torus $W$}
\label{fig:jt2}
\end{figure}

\begin{theorem}[Multiplicative Property] Let $y \in \R^N \setminus \Psi \circ \Phi\left( \partial \, \Omega \right)$. The topological degree at $y$ of the composition map is given by
\begin{equation} \label{mpdegree} \mathrm{deg} \left(y, \, \Psi \circ \Phi, \, \Omega \right)= \sum_{i \geq 0} \mathrm{deg} \left(y, \, \Psi, \, W_i \right) \cdot \mathrm{deg} \left(W_i, \, \Phi, \, \Omega \right). \end{equation}\end{theorem}

\begin{remark}If $z_1, \, z_2 \in W_i$ are points in the same connected component, then
\begin{equation*}\mathrm{deg} \left(z_1, \, \Phi, \, \Omega \right) = \mathrm{deg} \left(z_2, \, \Phi, \, \Omega \right),\end{equation*}
hence the topological degree $\mathrm{deg} \left(W_i, \, \Phi, \, \Omega \right)$ is well defined, and it is equal to the topological degree of \textbf{any} point in $W_i$.\end{remark}

\begin{remark}\label{rmk:sdosod} If $\mathrm{deg} \left(W_i, \, \Phi, \, \Omega \right) \neq 0$, then $W_i \subseteq \Phi(\Omega)$. In particular,
\begin{equation*} \overline{W_i} \subset W \end{equation*}
is bounded (since $\Phi \left(\overline{\Omega} \right)$ is compact), and hence the topological degree $\mathrm{deg} \left(y, \, \Psi \circ \Phi, \, \Omega \right)$ is well-defined since $y \notin \Psi(\partial \, W_i)$, as a consequence of the fact that $\partial \, W_i \subseteq \Phi(\partial \, \Omega)$. \end{remark}

\begin{remark}Fix $y \in \R^N \setminus \Psi \circ \Phi\left( \partial \, \Omega \right)$. The set of indices
\begin{equation*} \left\{ i \in \N \: \left| \: \Psi^{-1}(y) \cap W_i \cap \Phi(\Omega) \right. \right\} \end{equation*}
is finite, so that the sum \eqref{mpdegree} is well-defined. Indeed, the intersection
\begin{equation*}\Psi^{-1}(y) \cap \Phi \left( \overline{\Omega} \right) \end{equation*}
is a compact set, hence there is a finite collection of $W_i$'s covering it. \end{remark}

\begin{proof} \end{proof}

\section{Open Mapping Theorem}

In this section, we want to apply the multiplicative property of the topological degree, introduced above, to prove a famous - particularly simple - open mapping theorem.

\begin{theorem}[Open Mapping Theorem] Let $\Omega \subseteq \R^N$ be an open set. If $\Phi : \Omega \to \R^N$ is a continuous injective map, then $\Phi$ is also open. \end{theorem}

\begin{proof} Let $B$ be an open ball such that $\overline{B} \subseteq \Omega$. We consider the restriction
\begin{equation*} \tilde{\Phi} := \Phi \, \big|_{\overline{B}} \: : \: \overline{B} \to \R^N, \end{equation*}
and we denote by $\left(W_i \right)_{i \in \N}$ the collection of the connected components (as in the previous section) of the following set:
\begin{equation*} \R^N \setminus \tilde{\Phi} \left( \partial \, \overline{B} \right). \end{equation*}
The map $\tilde{\Phi}$ is continuous and injective on a compact set, hence there exists the left-inverse
\begin{equation*} \tilde{\Phi}^{-1} : \tilde{\Phi} \left(\overline{B} \right) \to \overline{B}, \end{equation*}
which can be extended to a continuous map $\Psi : \R^N \to \R^n$ by \hyperref[tiet]{Tietze Theorem}. The composition
\begin{equation*} \Psi \circ \tilde{\Phi} : \overline{B} \to \R^N \end{equation*}
is, by construction, the identity on the boundary $\partial \, \overline{B}$ (since there $\Psi$ coincides with the left inverse of $\Phi$), and hence for any $y \in B$ it turns out that
\begin{equation*} \mathrm{deg} \left(y, \, \Psi \circ \Phi, \, B \right) = 1. \end{equation*}
The multiplicative property \eqref{mpdegree} implies that
\begin{equation*} \sum_{i \geq 0} \mathrm{deg} \left(y, \, \Psi, \, W_i \right) \cdot \mathrm{deg} \left(W_i, \, \Phi, \, B \right) = 1, \end{equation*}
and thus there exists an index $j > 0$ such that
\begin{equation*} \mathrm{deg} \left(y, \, \Psi, \, W_j \right) \neq 0 \qquad \text{and} \qquad \mathrm{deg} \left(W_j, \, \Phi, \, B \right)\neq 0.\end{equation*}
By \hyperref[rmk:sdosod]{Remark \ref{rmk:sdosod}} we infer that $W_j \subseteq \Phi(B)$; since $W_j$ is connected, $\Phi(B)$ is connected and $\partial \, W_j \cap \Phi(B) = \emptyset$, we also infer that $W_j = \Phi(B)$, and this concludes the proof.
\end{proof}
\end{document}
