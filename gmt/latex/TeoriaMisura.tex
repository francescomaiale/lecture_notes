\title{Teoria della Misura - Lecture Notes}
\author{Francesco Paolo Maiale
        }
        
\documentclass[a4paper,10 pt]{report}

\usepackage{graphicx}
\usepackage{amsfonts}
\usepackage[pass]{geometry}
\usepackage{amsthm}
\usepackage{amsmath, amssymb}
\usepackage{setspace}
\usepackage[english]{babel}
\usepackage{tikz-cd}
\usepackage{makeidx}         % permette di generare l'indice
\usepackage[utf8]{inputenc}
\usepackage{mathtools}
\usepackage{enumitem}
\usepackage{accents}
\usepackage{faktor}
\usepackage{mathrsfs}  
\usepackage{pifont}
\usepackage{xcolor}
\usepackage{hyperref}
\usepackage{scalerel}[2014/03/10]
\usepackage[usestackEOL]{stackengine}
\usepackage{bbm}

\makeindex %DDD

\hypersetup{
    colorlinks=true,
    linkcolor=cyan,
    filecolor=magenta,      
    urlcolor=cyan,
}


\DeclareRobustCommand\longtwoheadrightarrow
     {\relbar\joinrel\twoheadrightarrow}

\newcommand{\notimplies}{%
  \mathrel{{\ooalign{\hidewidth$\not\phantom{=}$\hidewidth\cr$\implies$}}}}
  
\pagestyle{plain}
\setlength{\topmargin}{0.0in}
\setlength{\headheight}{0.2in}
\setlength{\headsep}{0.0in}
\setlength{\footskip}{0.5in}
\setlength{\textheight}{8.3in}
\setlength{\textwidth}{6.0in}
\setlength{\oddsidemargin}{0.5in}
\setlength{\evensidemargin}{0.5in}
\setlength{\parindent}{0.2 in}


\newtheorem{theorem}{Theorem}[chapter]
\newtheorem{lemma}[theorem]{Lemma}
\newtheorem{proposition}[theorem]{Proposition}
\newtheorem{corollary}[theorem]{Corollary}
\theoremstyle{definition}
\newtheorem{definition}[theorem]{Definition}

\newtheorem{remark}{Remark}[chapter]
\newtheorem{example}{Example}[chapter]
\newtheorem*{notation}{Notation}
\newtheorem*{claim}{Claim}
\newtheorem{exercise}{Exercise}[chapter]

\newcommand{\smallO}[1]{\scriptstyle\mathcal{O}}
\DeclarePairedDelimiter\floor{\lfloor}{\rfloor}
\newcommand*\conj[1]{\overline{#1}}
\newcommand{\R}{\mathbb R}
\newcommand{\C}{\mathbb C}
\newcommand{\N}{\mathbb N}
\newcommand{\G}{\mathcal G}
\newcommand{\Z}{\mathbb Z}
\newcommand{\con}{\mathcal{C}\left(x, \, V, \, \alpha \right)}
\newcommand{\Q}{\mathbb Q}
\newcommand{\p}{\mathbb P}
\newcommand{\Le}{\mathcal{L}}
\newcommand{\can}{\symbol{35}}
\newcommand*{\double}[2][.1ex]{%
  \mathrel{\vcenter{\offinterlineskip%
  \hbox{$#2$}\vskip#1\hbox{$#2$}}}}
\newcommand*{\doublerightarrow}{\double{\longrightarrow}}

\newcommand{\restr}{%
  \,\raisebox{-.127ex}{\reflectbox{\rotatebox[origin=br]{-90}{$\lnot$}}}\,%
}

\makeatletter
\renewcommand*\env@matrix[1][*\c@MaxMatrixCols c]{%
  \hskip -\arraycolsep
  \let\@ifnextchar\new@ifnextchar
  \array{#1}}
\makeatother

\newcommand{\bsquare}{\item[\color{magenta}\ding{110}]} 
\newcommand{\barrow}{\item[\color{blue}\ding{228}]}
\newcommand{\bwarrow}{\item[\color{gray}\ding{227}]}

\def\dashint{\,\ThisStyle{\ensurestackMath{%
  \stackinset{c}{.2\LMpt}{c}{.5\LMpt}{\SavedStyle-}{\SavedStyle\phantom{\int}}}%
  \setbox0=\hbox{$\SavedStyle\int\,$}\kern-\wd0}\int}
\def\ddashint{\,\ThisStyle{\ensurestackMath{%
  \stackinset{c}{.2\LMpt}{c}{.5\LMpt+.2\LMex}{\SavedStyle-}{%
    \stackinset{c}{.2\LMpt}{c}{.5\LMpt-.2\LMex}{\SavedStyle-}{%
      \SavedStyle\phantom{\int}}}}\setbox0=\hbox{$\SavedStyle\int\,$}\kern-\wd0}\int}


\begin{document}
\newpage

\begin{center}

\begin{spacing}{1.5}
Lecture Notes\\
\vspace*{\fill}
\end{spacing}
\begin{spacing}{2.5}
\textbf{\huge Geometric Measure Theory}\\[0.5cm]
\vspace*{\fill}
\begin{minipage}{5cm}
\centering {\textit{Course held by}}
\end{minipage}
\hspace{5cm}
\begin{minipage}{5cm}
\centering {\textit{Notes written by}} \\
\end{minipage}
\end{spacing}

\begin{spacing}{1.3}

\begin{minipage}{5cm}
\centering {\textbf{\large Prof. Giovanni Alberti}}
\end{minipage}
\hspace{5cm}
\begin{minipage}{5cm}
\centering {\textbf{\large Francesco Paolo Maiale}}
\end{minipage}

\vspace*{\fill}

\textnormal{\large Department of Mathematics \\[0.4em] Pisa University \\[0.4em] \today}

\end{spacing}
\end{center}

\newpage
{\Huge \textbf{Disclaimer}\par}

\vspace{0.4in} % if you wish

These notes came out of the \textit{Geometric Measure Theory} course, held by Professor Giovanni Alberti in the second semester of the academic year 2016/2017.

They include all the topics that were discussed in class; I added some remarks, simple proof, etc.. for my convenience.

I have used them to study for the exam; hence they have been reviewed thoroughly. Unfortunately, there may still be many mistakes and oversights; to report them, send me an email to frank094 (at) hotmail (dot) it.


\tableofcontents

\newpage
\chapter{Introduction}

In this chapter, we introduce the main topics of the course and give a brief overview of what we will see and what we will be able to prove by the end of the course.

\section{Plateau's Problem}

The primary goal and the motivating example of this course is the \textbf{Plateau's problem}, that is, the problem to find the $d$-dimensional surface $\Sigma$ of the minimal area with prescribed $(d-1)$-dimensional boundary $\Gamma$.

By the end, we will be able to prove that a solution indeed exists, but we will not find it explicitly since it is a $NP$ (hard) numerical problem.

As of now, the problem is not well defined. In fact, the notions of \textit{surface}, \textit{area}, and \textit{boundary} make sense in the smooth setting but, as the examples below show, we need to work in a less regular setting.

More precisely, requiring the surface to be smooth is not enough for modeling reasons (e.g., dip a wire frame into a soap solution, form a soap film, and look for the minimal surface whose boundary is the wire frame), and also for existence reasons.

\begin{example} Here we give a list of Plateau's problems with prescribed boundary conditions, and we write down the correct solutions, without proving anything. \mbox{}
\begin{enumerate}[label=\textbf{(\alph*)}]
\item Let us identify $\R^4 \cong \C \times \C$ and, if $d = 2$, let us consider the smooth boundary given by
\begin{equation*} \Gamma_1 := \left( S^1 \times \{0\} \right) \cup \left( \{0\} \times S^1 \right). \end{equation*}
Surprisingly, every minimizing sequence of smooth surfaces converges to a surface which is not smooth at all. Indeed, the solution of the problem is
\begin{equation*} \Sigma_1 := \left( D^2 \times \{0\} \right) \cup \left( \{0\} \times D^2 \right). \end{equation*}
The surface $\Sigma_1$ is clearly singular at the origin, but the singularity may be removed (by factorizing it into two nonsingular surfaces).
\item Let us identify $\R^4 \cong \C \times \C$ and, if $d = 2$, let us consider the smooth boundary given by
\begin{equation*} \Gamma_2 := \left\{ (z^2, \, z^3) \: : \: z \in S^1 \right\}. \end{equation*}
The solution to the Plateau's problem is
\begin{equation*} \Sigma_2 := \left\{ (z^2, \, z^3) \: : \: z \in D^2 \right\}, \end{equation*}
which is a non-smooth surface, whose singularity cannot be removed (since the polynomial $z_1^3 = z_2^2$ cannot be factorized).
\item Let us identify $\R^8 \cong \R^4 \times \R^4$ and, if $d = 7$, let us consider the smooth boundary given by
\begin{equation*} \Gamma_3 := S^3 \times S^3. \end{equation*}
The minimal surface of prescribed boundary $\Gamma_3$ is
\begin{equation*} \Sigma_3 := \left\{ (x_1, \, x_2) \in \R^4 \times \R^4 \: : \: |x_1| = |x_2| \leq 1 \right\}. \end{equation*}
\end{enumerate}
\end{example}

To conclude this introductive chapter, we give a brief overview of the main approaches (studied in this course) to the Plateau's problem, as $d$ ranges between $1$ and $\infty$.

\section{Geodesics problem ($d=1$)}

The geodesics problem (that is, find the shortest curve connecting two points) is, surprisingly, still an open in the non-Riemannian setting. However, in the Riemannian setting, the geodesics problem is completely solved.

Indeed, if we consider the curves parametrized by paths, the \textit{length} is a well-defined notion, and the associated functional is lower semi-continuous and coercive; hence the compactness is easy to prove.

There are many possible approaches to the geodesics problem, e.g., the Steiner approach and the set theoretical approach, which we describe briefly in the remainder of the section.

\paragraph{Steiner Problem.} It is also called networks approach, and it is used to prove the existence of the geodesics and find the explicit expression for it. The reader may consult \cite{steinerpro} for a detailed dissertation on the topic.

\paragraph{Set Theoretical Approach.} The main idea is to find a closed and connected set $\Sigma$ of minimum \textit{length}, containing a given finite set $\Gamma$. As we shall see later in the course, in this case the length is a well defined concept: the \textit{Hausdorff distance}.

In fact, if $X$ is a suitable space (metric, endowed with Hausdorff distance, etc...), then the class defined by
\begin{equation*}\mathcal{X} := \left\{ K \subseteq X \: : \: K \, \, \text{compact and connected} \right\} \end{equation*}
is compact and, by Gotab theorem\footnote{\cite{falconer} Let $\mathscr{C}$ be an infinite collection of non-empty compact sets all lying in a bounded portion $B$ of $\R^n$. Then there exists a sequence $\{E_j\}$ of distinct sets of $\mathfrak{C}$ convergent in the Hausdorff metric to a non-empty compact set $E$.}, $\mathcal{H}^1$ is lower semi-continuous on $X$. 

\section{"Surface" Problem ($d>1$)}

\paragraph{3.} The Plateau's problem is much harder when $d = 2$, but there are still many approaches possible some of which relying, in a certain sense, on the work already done in the geodesics case.

\paragraph{Set Theoretical Approach.} This approach is highly nontrivial. For example, one may ask what does it mean that a compact set $\Sigma$ spans a boundary $\Gamma$? Moreover, there is another problem one should deal with: the $2$-dimensional Hausdorff measure $\mathcal{H}^2$ is, generally, not lower semi-continuous. The reader may consult \cite{Reifenberg1960} for a complete treatise of the topic.

\begin{remark}Suppose that $d = 2$, $n = 3$ and that $\Sigma$ is a surface with boundary $\Gamma$. If $\gamma$ is another closed curve, linked to $\Gamma$ (by a nonzero linking number), then $\gamma \cap \Sigma \neq \emptyset$. \end{remark}

\paragraph{Parametric Approach.} This method is essentially due to Douglas \cite{douglas}. The main idea is the following: since a parametrization $\phi : D^2 \to \R^n$ defines surfaces in $\R^n$, the area functional is well-defined and given by the formula
\begin{equation*}A(\phi) := \int_{D^2} \left| \frac{\partial \, \phi}{\partial \, s_1} \wedge \frac{\partial \, \phi}{\partial \, s_2} \right| \, \mathrm{d}s_1 \, \mathrm{d}s_2. \end{equation*}
On the other hand, the existence through lower semi-continuity and the compactness are a delicate matter, since coercivity is not an easy property to obtain (the integrand is similar to a determinant).

There is a trick which is similar to the one we can use to find geodesics in the differential geometry setting. More precisely, we consider the functional
\begin{equation*}E(\phi) := \frac{1}{2} \, \int_{D^2} \left| \nabla \phi \right|^2 \, \mathrm{d}s_1 \, \mathrm{d}s_2. \end{equation*}
If we find a minimal point $\phi$ for $E$, then $\phi$ will be a \textbf{conformal parametrized} minimum for $A$. This trick, on the other hand, heavily depends on a nontrivial theorem: every such $\Sigma$ admits a conformal re-parametrization.

The lack of conformal parametrization, though, is what stop us from extending the same trick to dimension $d$ strictly bigger than $2$.

\paragraph{Higher Dimension.} If the codimension of $\Sigma$ is equal to $1$ (that is, $n = d+1$), then finite perimeter sets generalize the notion of open $(d+1)$-dimensional sets with smooth boundary in $\R^n$.

The class of finite perimeter sets has excellent compactness properties and a notion of area lower semi-continuous. 

This approach is called "weak" surfaces approach, and it is essentially due to Caccioppoli \cite{cacio} and De Giorgi \cite{degi}. A different approach, working for any $d$ and $n$, referred to as \textit{integral currents}, was introduced by Federer and Flaming in their joint paper \cite{fed}.
\chapter{Measure Theory}

In this chapter, the chief goal is to give a brief introduction to the theory of measure and acquire the fundamental notions needed for the remainder of the course. The reader with an adequate background, may immediately jump to the next section, and use this one as a reference.

In the first part, we introduce the only three classes of measures we shall be concerned about in this course, and then we study the weak-$\ast$ topology and some of the fundamental properties of the weak-$\ast$ convergence.

In the second part, we introduce the notion of \textit{outer measure}, and we show how it can be used to construct a measure that belongs to one of the classes mentioned above.

In the final part, we construct the Hausdorff $d$-dimensional measure, denoted by $\mathcal{H}^d$, and we prove some of the main properties (e.g., the relation with Lipschitz functions). The last section is devoted to the computation of the Hausdorff dimension of a Cantor-type set.

\section{Definitions and Elementary Properties}

\paragraph{Introduction.} In this course, we will essentially be only concerned with measures that belong to one of the following classes: \mbox{}
\begin{enumerate}[label=\textbf{(\arabic*)}]
\item Positive, $\sigma$-additive measures, defined on the Borel $\sigma$-algebra of a reasonable space $X$, e.g. we might assume $X$ metric, locally compact and separable.\footnote{We are not very specific here since the reader may try, whenever possible, to derive, as an exercise, the minimal assumption on $X$ for an assertion to be true. }
\item Positive, $\sigma$-subadditive measures, defined on the set of all parts $P(X)$. These are generally called \textit{outer measures}.
\item Real-valued and vector-valued (bounded) $\sigma$-additive measures, defined on the Borel $\sigma$-algebra.
\end{enumerate}

\begin{remark}The Borel $\sigma$-algebra $\mathcal{B}(X)$ is not closed under the action of continuous maps.\end{remark}

\begin{proof}Let $X = [0, \, 1] \subset \R$, and let us consider a space-filling curve\index{space-filling curve} $g : [0, \, 1] \longrightarrow [0, \, 1]^2$ and the projection $\pi : [0, \, 1]^2 \longrightarrow [0, \, 1]$; the reader may easily check that the map $f := \pi \circ g : [0, \, 1] \longrightarrow [0, \, 1]$ is a continuous function.

Thus, given a Borel set $S \subset [0, \, 1]^2$ with $\pi \left(S \right)$ not Borel, it is easy to prove that the image of the Borel set $g^{-1}(S)$ under $f$ is not Borel. \end{proof}

In this course, we will often introduce a suitable\footnote{See \hyperref[sec:caco]{Section \ref{sec:caco}} for a more precise definition of what suitable means here.} outer measure defined on $X$ and, by taking the restriction to the Borel $\sigma$-algebra (see \hyperref[sigmachar]{Theorem \ref{sigmachar}}), we automatically obtain a measure which belongs to the class \textbf{(1)}.

\begin{remark}  If $X$ is a separable Banach space\footnote{It is not necessary to require $X$ to be a separable Banach space, but it is more than enough for our purposes. }, then the class of measures \textbf{(3)} is the dual space of a particular subspace of $C^0(X)$. In the next section, we will also prove that, if the space of measures is endowed with the weak-$\ast$ topology, then this class has good compactness properties (that is, the Banach-Alaoglu theorem holds true). \end{remark}

\begin{definition}[Vector-valued Measure] \index{measure} \index{measure!vector-valued} Let $\mathcal{B}(X)$ be a Borel $\sigma$-algebra on $X$, and let $F$ be a normed space. A function $\mu : X \longrightarrow F$ is a $\sigma$-additive \textit{$F$-valued} measure if
\begin{equation} \label{eq:sum} \sum_{n \in \N} \mu \left(E_n \right) = \mu \left( \bigcup_{n \in \N} E_n \right), \end{equation}
for any countable disjoint family of Borel sets $\{E_n\}_{n \in \N} \subset \mathcal{B}(X)$. \end{definition}

\begin{remark} The identity \eqref{eq:sum} does not only imply the $\sigma$-additivity of the measure, but it also gives us a stronger property.

More precisely, the sum on the left-hand side does not depend on the order of the indexes, and thus, if $F$ is a finite-dimensional space, then it is enough to infer that the sum converges absolutely.\end{remark}

\paragraph{Representation Theorem.} In this final brief paragraph, we sketch the proof of the \textit{Riesz representation theorem}, according to which the class of measures \textbf{(3)} is the dual of a suitable subset of the space of all continuous functions.

\begin{definition}[Total Variation] \index{measure}\index{measure!total variation} Let $\mu$ be a measure. The \textit{total variation} of a set $E \subseteq X$ is the supremum of the measure over the possible countable partitions, that is,
\begin{equation} \label{eq:totvar} \left| \mu \right| (E) := \sup \left\{ \left. \sum_{n = 0}^{+\infty} \left| \mu(E_n) \right| \: \right| \: \text{$\{E_n\}_{n \in \N}$ countable partition of $E$} \right\}. \end{equation}
\end{definition}

Clearly, the total variation is a positive bounded measure, defined on the Borel $\sigma$-algebra $\mathcal{B}(X)$. The \textbf{mass}\index{measure!mass} of $\mu$ is defined by setting
\begin{equation} \label{eq:normame} \| \mu \| := \left| \mu \right|(X). \end{equation}
The formula \eqref{eq:normame} defines a norm on the space of all the $F$-valued $\sigma$-Borel measures, which is also complete. Moreover, it is easy to see that $\mu$ is absolutely continuous with respect to its total variation $\left|\mu \right|$, and hence by \textbf{Radon-Nikodym Theorem} there exists a Borel function $f : X \longrightarrow F$ such that
\begin{equation*} \mu = f \cdot \left| \mu \right| \leadsto \mu(E) := \int_{E} f(x) \, \mathrm{d} \left| \mu \right|(x), \end{equation*}
with the additional property that its norm is almost everywhere equal to one, that is, $|f(x)|_F = 1$ for $\left|\mu\right|$-almost every $x \in X$. 

Consequently, vector-valued measures may be identified with the product between a positive measure $\lambda$ and a function $\lambda$-summable, that is, there exists $f \in L^1(\lambda)$ such that
\begin{equation*} \mu = f \cdot \lambda. \end{equation*}
This identification is particularly useful when we need to integrate a function $g : X \longrightarrow F$ with respect to a vector valued measure. Indeed, it turns out that
\begin{equation*} \int_{X} g(x) \, \mathrm{d} \mu(x) = \int_{X} g(x) \cdot f(x) \, \mathrm{d} |\mu|(x), \end{equation*}
where $\cdot$ is a suitable notion of product\footnote{For example it could be a scalar product, or an external product.} between $f$ and $g$.

\begin{notation}Let $\mu$ be an $\R^n$-valued measure defined on $X$. We denote by $\Lambda_\mu$ the functional given by
\begin{equation} \label{fucntiosnd} g \longmapsto \int_{X} g \, \mathrm{d}\mu = \int_{X} g(x) \cdot f(x) \, \mathrm{d} \left|\mu\right|(x), \end{equation}
where $\cdot$ is the scalar product in $\R^n$. \end{notation}

The map \eqref{fucntiosnd} is well-defined at every $\left| \mu \right|$-summable function $g : X \longrightarrow \R^n$ function; thus, the functional is well-defined at every continuous function $h : X \longrightarrow \R^n$ which is infinitesimal at $\infty$ (i.e., equal to zero in the one-point compactification of $\R^n$.)

In particular, $\Lambda_\mu$ is well-defined on $C_0 \left(X; \; \R^n \right)$ - or, equivalently, on $C\left(X; \; \R^n \right)$ if $X$ is compact -, which is a separable Banach spaces with respect to the supremum norm. The functional 
\begin{equation*}\Lambda_\mu : C_0 \left(X; \; \R^n \right) \longrightarrow \R\end{equation*}
is linear and bounded (i.e., continuous). More precisely, from \eqref{fucntiosnd} it follows that
\begin{equation*} \left| \Lambda_\mu(g) \right| \leq \int_{X} |g(x)| \, \mathrm{d} \left|\mu\right|(x) \leq \|g\|_{C_0(X; \; \R^n)} \, \|\mu\|, \end{equation*}
which, in turn, implies that $\| \Lambda_\mu \| \leq \|\mu\|$. If we set $g := f$, then it turns out that $\|g\| = \|f\| = 1$ and that the equality holds true, i.e.
\begin{equation*} \| \Lambda_{\mu} \| = \|\mu\|. \end{equation*}

\begin{theorem}[Riesz] \index{Riesz Theorem} \label{RieszTheorem} Let $\mathcal{M} \left(X; \; \R^n\right)$ be the space of all the $\R^n$-valued measures. The map
\begin{equation*}\mathcal{M} \left(X; \; \R^n\right) \longrightarrow \left(C_0\left(X; \; \R^n \right)\right)^\ast, \qquad \mu \longmapsto \Lambda_\mu \end{equation*}
is an isometry. Moreover, if $X$ is a compact space, then
\begin{equation*}\mathcal{M} \left(X; \; \R^n\right) \longrightarrow \left(C\left(X; \; \R^n \right)\right)^\ast, \qquad \mu \longmapsto \Lambda_\mu \end{equation*}
is also an isometry. \end{theorem}

\begin{theorem}[Riesz] Let $F$ be a finite-dimensional normed space. The map
\begin{equation*}\mathcal{M} \left(X; \; F^\ast \right) \longrightarrow \left(C_0(X; \; F^\ast)\right)^\ast, \qquad \mu \longmapsto \Lambda_\mu \end{equation*}
is an isometry. Moreover, if $F$ is a separable Banach space, then
\begin{equation*}\mathcal{M} \left(X; \; F^\ast \right) \longrightarrow \left(C_0(X; \; F^\ast)\right)^\ast, \qquad \mu \longmapsto \Lambda_\mu \end{equation*}
is also an isometry, provided that $F^\ast$ is \textit{good} enough.  \end{theorem}

\section{Weak-$\ast$ Topology on Measures Space}

In this section, we endow the space of measures $\mathcal{M} \left(X; \; \R^n \right)$ with the weak-$\ast$ topology since the associated notion of convergence is particularly pleasant.

\begin{definition}[Measures convergence]\index{measure!weak-$\ast$ convergence} A sequence $(\mu_n)_{n \in \mathbb{N}} \subset \mathcal{M} \left(X; \; \R^n \right)$ of $\R^n$-valued measures converges weakly-$\ast$ to a measure $\mu \in \mathcal{M} \left(X; \; \R^n \right)$ if and only if
\begin{equation} \label{convweaksast} \lim_{n \to \infty} \int_{X} g(x) \, \mathrm{d} \mu_n(x) = \int_{X} g(x) \, \mathrm{d}\mu(x), \qquad \forall g \in C_0^0 \left(X; \; \R^n \right). \end{equation}\end{definition}

\begin{remark}If $X$ is a compact space, then $(\mu_n)_{n \in \mathbb{N}} \subset \mathcal{M} \left(X; \; \R^n \right)$ converges weakly-$\ast$ to a measure $\mu \in \mathcal{M} \left(X; \; \R^n \right)$ if and only if
\begin{equation} \label{eq:sodoskdottdd} \lim_{n \to \infty} \int_{X} g(x) \, \mathrm{d} \mu_n(x) = \int_{X} g(x) \, \mathrm{d}\mu(x), \qquad \forall g \in C^0 \left(X; \; \R^n \right). \end{equation} \end{remark}

\begin{remark} The notion of strong convergence in the space $\mathcal{M} \left(X; \; \R^n \right)$ is, unfortunately, "\textit{too strong}", and, in general, it is not interesting at all. For this reason, from now on we say that $\mu_n$ converges (in the sense of measures) to $\mu$ if \eqref{convweaksast} is satisfied. \end{remark}

\begin{remark} \mbox{}
\begin{enumerate}[label=\textbf{(\arabic*)}]
\item The \textbf{Banach-Alaoglu Theorem}, in the particular case of $\mathcal{M} \left(X; \; \R^n \right)$, may be stated in the following, particularly simple, way.

\begin{theorem}[Banach-Alaoglu] \index{measure!Banach-Alaoglu} Let $(\mu_n)_{n \in \mathbb{N}} \subset \mathcal{M} \left(X; \; \R^n \right)$ be a sequence with uniformly bounded masses, that is, there exists $C > 0$ such that
\begin{equation*} \| \mu_n \| \leq C < \infty \qquad \forall \, n \in \mathbb{N}. \end{equation*}
Then, up to subsequences, it converges in the sense of measures to an element $\mu \in \mathcal{M} \left(X; \; \R^n \right)$. \end{theorem}

\item Let $(\mu_n)_{n \in \mathbb{N}} \subset \mathcal{M} \left(X; \; \R^n \right)$ be a sequence of measures with uniformly bounded masses. Then $\mu_n$ converges to $\mu \in \mathcal{M} \left(X; \; \R^n \right)$ if and only if
\begin{equation*} \lim_{n \to \infty} \int_{X} g(x) \, \mathrm{d} \mu_n(x) = \int_{X} g(x) \, \mathrm{d}\mu(x), \qquad \forall g \in D, \end{equation*}
where $D$ is a dense subset of $C_0 \left(X; \; \R^n \right)$, e.g. the space of compactly supported functions.

\item The "local version of the theory" works in a very similar way. For example, if we assume that $(\mu_n)_{n \in \mathbb{N}} \subset \mathcal{M} \left(X; \; \R^n \right)$ is a sequence of measures with locally bounded masses, that is, for any $K \subseteq X$ compact there exists $C(K) > 0$ such that
\begin{equation*}\| \mu_n \|_{K} \leq C(k) < \infty, \qquad \forall n \in \mathbb{N}, \end{equation*}
then we can infer that $\mu_n$ converges in the sense of measures.
\end{enumerate}
\end{remark}

\begin{proposition} \label{proposition:posme} Let $(\mu_n)_{n \in \mathbb{N}} \subset \mathcal{M} \left(X; \; \R \right)$ be a sequence of real-valued positive measures, and assume that it converges to $\mu \in  \mathcal{M} \left(X; \; \R \right)$. \mbox{}
\begin{enumerate}[label=\textbf{(\alph*)}]
\item For any open subset $A \subseteq X$, it turns out that
\begin{equation*}\liminf_{n \to + \infty} \mu_n(A) \geq \mu(A). \end{equation*}
\item For any compact subset $K \subseteq X$, it turns out that
\begin{equation*}\limsup_{n \to + \infty} \mu_n(K) \leq \mu(K). \end{equation*}
\item For any relatively compact $E \subseteq X$ such that $\mu(\partial E) = 0$, it turns out that
\begin{equation*}\lim_{n \to + \infty} \mu_n(E) = \mu(E). \end{equation*}
\end{enumerate}
\end{proposition}

\begin{proof} \mbox{}
\begin{enumerate}[label=\textbf{(\alph*)}]
\item The first assertion is equivalent to the fact that the functional
\begin{equation*}\Phi_A : \mathcal{M} \left(X; \; \R \right) \longrightarrow \R, \qquad \lambda \longmapsto \int_{X} \chi_A \, \mathrm{d}\lambda \end{equation*}
is (weakly-$\ast$) lower semi-continuous for every $A \in \mathcal{B}(\R)$. The characteristic function of an open set can be approximated by an increasing sequence of continuous function, e.g.,
\begin{equation*} f_n^{(A)}(x) := \begin{cases} 0 & \text{if $x \notin A$} \\[0.4em] \min \left\{ 1, \, \displaystyle\sup_{r > 0} \{ n \cdot r \: \left| \: B(x, \, r) \subseteq A \right. \} \right\} & \text{if $x \in A$}. \end{cases} \end{equation*}
The supremum of weakly-$\ast$ lower semi-continuous function\footnote{The sequence $(f_n)_{n \in \N}$ is made up of continuous function, but the supremum will lose the upper semi-continuity.} is weakly-$\ast$ lower semi-continuous; thus from the relation
\begin{equation*} \chi_A(x) = \sup_{n \in \mathbb{N}} f_n(x),  \end{equation*}
it follows easily that $\chi_A$ is lower semi-continuous. Therefore, if we apply the standard Fatou Lemma, it turns out that
\begin{equation*}\liminf_{n \to + \infty} \mu_n(A) \geq \mu(A), \qquad \forall \, A \in \mathcal{B}(\R). \end{equation*}
\item This assertion follows immediately from the previous one by taking the complement. On the other hand, we can mimic the proof above and obtain the result directly by proving that
\begin{equation*}\Phi_K : \mathcal{M} \left(X; \; \R \right) \longrightarrow \R, \qquad \lambda \longmapsto \int_{X} \chi_K \, \mathrm{d}\lambda \end{equation*}
is weakly-$\ast$ upper semi-continuous for every $K \subseteq X$ compact. The characteristic function of a compact set can be approximated by a decreasing sequence of continuous function, e.g.
\begin{equation*} g_n(x) := \begin{cases} 1 & \text{if $x \in K$} \\[0.4em] 1 - \min \left\{ 0, \, \displaystyle\sup_{r > 0} \{ n \cdot r \: \left| \: B(x, \, r) \right. \subseteq K^c \} \right\} & \text{if $K^c$}. \end{cases} \end{equation*}
The infimum of a collection of (weakly-$\ast$) upper semi-continuous functions is (weakly-$\ast$) upper semi-continuous; thus from the equality
\begin{equation*} \chi_K(x) = \inf_{n \in \mathbb{N}} g_n(x),  \end{equation*}
it follows that $\chi_K$ is upper semi-continuous. Therefore, if we apply the reverse inequality of the Fatou Lemma, it turns out that
\begin{equation*}\limsup_{n \to + \infty} \mu_n(K) \leq \mu(K). \end{equation*}
\item Let $E \subseteq X$ be a relatively compact set such that $\mu\left(\partial E \right) = 0$. The interior part $\mathrm{Int} \, E$ has the same measure of $E$ and it is open; thus it follows from \textbf{(a)} that
\begin{equation*}\liminf_{n \to + \infty} \mu_n(E) = \liminf_{n \to + \infty} \mu_n \left(\mathrm{Int} \, E \right) = \geq \mu \left( \mathrm{Int} \, E \right) = \mu(E). \end{equation*}
In a similar fashion, since $E$ is relatively compact, the closure $\overline{E}$ is compact and has the same measure of $E$; thus it follows from \textbf{(b)} that
\begin{equation*}\limsup_{n \to + \infty} \mu_n \left(\overline{E} \right) = \limsup_{n \to + \infty} \mu_n(E) \leq \mu(E) = \mu\left(\overline{E}\right), \end{equation*}
and this proves the sought identity, i.e., $\lim_{n \to + \infty} \mu_n(E) = \mu(E)$.
\end{enumerate}
\end{proof}

\begin{proposition} \label{proposition:posme2} Let $(\mu_n)_{n \in \mathbb{N}} \subset \mathcal{M} \left(X; \; \R \right)$ be a sequence of real-valued positive measures, and assume that it converges to $\mu \in  \mathcal{M} \left(X; \; \R \right)$. \mbox{}
\begin{enumerate}[label=\textbf{(\alph*)}]
\item If $X$ is compact, then
\begin{equation*}\lim_{n \to \infty} \|\mu_n\| = \|\mu\|. \end{equation*}
\item If $X$ is locally compact, then
\begin{equation*}\liminf_{n \to \infty} \|\mu_n\| \geq \|\mu\|. \end{equation*}
In particular, there exists a sequence $(\mu_n)_{n \in \N}$ of measures, converging to some $\mu$, such that the mass $\|\mu_n\|$ does not converge to $\|\mu \|$.
\end{enumerate}
\end{proposition}

\begin{proof}\mbox{}
\begin{enumerate}[label=\textbf{(\alph*)}]
\item The norm is weakly-$\ast$ lower semi-continuous; hence it suffices to prove that
\begin{equation*}\mu_n \to \mu \text{   and $X$ compact} \implies \limsup_{n \to \infty} \|\mu_n\| \leq \|\mu\|. \end{equation*}
By assumption $(\mu_n)_{n \in \N}$ is a sequence of positive measures, which means that the mass is simply given by $ \|\mu_n\| = \mu_n(X)$, that is,
\begin{equation*}\|\mu_n\| = \int_{X} \mathrm{d}\mu_n, \qquad \forall \, n \in \mathbb{N}. \end{equation*}
By compactness of $X$, the convergence in sense of measures is equivalent to \eqref{eq:sodoskdottdd}, and thus
\begin{equation*}\|\mu_n\| = \int_{X} \mathrm{d}\mu_n \to \int_{X} \mathrm{d}\mu = \|\mu\|.  \end{equation*}
\item It is enough to provide a counterexample to the convergence. Let $X = \R$ and let $(x_n)_{n \in \N}$ be a sequence of points in $\R$ such that $x_n \to + \infty$ as $n \to + \infty$. If we define
\begin{equation*} \mu_n := \delta_{x_n}, \end{equation*}
where $\delta_{y}$ is the delta of Dirac at $y$, then it is straightforward to prove that $\mu_n \to 0$ and $\|\mu_n\| = 1$ for every $n \in \mathbb{N}$, that is,
\begin{equation*} \liminf_{n \to + \infty} \|\mu_n\| = 1 > \|\mu\| = 0. \end{equation*}
\end{enumerate} \end{proof}

\begin{definition}[Tightness] \index{measure!tightness} Let $(\mu_n)_{n \in \mathbb{N}} \subset \mathcal{M} \left(X; \; \R \right)$ be a sequence of real-valued positive measures. The sequence is \textit{tight} if, for every $\epsilon > 0$, there exists a compact subset $K_\epsilon \subset X$ such that
\begin{equation*} \mu_n \left(X \setminus K_\epsilon\right) \leq \epsilon, \qquad \forall \, n \in \mathbb{N}. \end{equation*}\end{definition}

\begin{lemma}Let $(\mu_n)_{n \in \mathbb{N}} \subset \mathcal{M} \left(X; \; \R \right)$ be a sequence of real-valued positive measures, converging to an element $\mu \in  \mathcal{M} \left(X; \; \R \right)$. Then the sequence is tight if and only if the mass converges, that is,
\begin{equation} \label{eq:thight} \|\mu_n\| \xrightarrow{n \to + \infty} \|\mu\|. \end{equation} \end{lemma}

\begin{proof}If the sequence of the masses converges to $\|\mu\|$, then the tightness of $(\mu_n)_{n \in \N}$ follows from the definition (namely, the tail of the masses sequence is as small as we want.)

Vice versa, suppose that the sequence $(\mu_n)_{n \in \N}$ is tight. The norm $\| \cdot \|$ is weakly-$\ast$ lower semi-continuous; hence it suffices to prove the opposite inequality, that is,
\begin{equation*}\limsup_{n \to \infty} \|\mu_n\| \leq \|\mu\|. \end{equation*}
Let $\epsilon > 0$ and let $K_\epsilon \subset X$ be the compact subset satisfying \eqref{eq:thight}. If we consider the decomposition of the ambient space $X = K_\epsilon \cup K_\epsilon^c$, then it turns out that
\begin{equation*}\mu_n(X) = \int_{K_\epsilon} \mathrm{d}\mu_n + \int_{X \setminus K_\epsilon} \mathrm{d}\mu_n \leq \int_{X} \chi_{K_\epsilon}(x) \, \mathrm{d}\mu_n(x) + \epsilon.  \end{equation*}
Finally, we take the limit as $n \to + \infty$, and we notice that from \textbf{(b)} of \hyperref[proposition:posme]{Proposition \ref{proposition:posme}} it follows that that
\begin{equation*}\limsup_{n \to +\infty} \mu_n(X) \leq \limsup_{n \to + \infty} \int_{X} \chi_{K_\epsilon} \, \mathrm{d}\mu_n + \epsilon \leq \mu(K_\epsilon) + \epsilon,  \end{equation*}
which is enough to conclude the proof since $\epsilon > 0$ may be taken arbitrarily small.\end{proof} 

\paragraph{"Strong" Convergence.} In this final paragraph, we introduce a stronger notion of convergence, which turns out to be the right replacement for the convergence in norm for the space of all measures.

\begin{definition}[Convergence in Variation] \index{measure!convergence in variation} Let $(\mu_n)_{n \in \mathbb{N}} \subset \mathcal{M} \left(X; \; \R \right)$ be a sequence of real-valued measures. The sequence \textit{converges in variation} to some $\mu \in \mathcal{M} \left(X; \; \R \right)$ if and only if
\begin{equation} \label{eq:convergencevariation} \mu_n \longrightarrow \mu \qquad \text{and} \qquad \|\mu_n\| \xrightarrow{n \to + \infty} \|\mu\|.\end{equation}
\end{definition}

\begin{remark} The convergence in variation induces a finer topology than the weak-$\ast$ one, and it is the right replacement for the norm convergence. \end{remark}

\begin{remark} Let $(\mu_n)_{n \in \N}$ be a sequence of positive measures converging in variation. The assertions of \hyperref[proposition:posme]{Proposition \ref{proposition:posme}} hold true, provided that we modify them accordingly with the new definition: \mbox{}
\begin{enumerate}
\item[$\mathbf{(b_1)}$] For any closed subset $C \subseteq X$, it turns out that
\begin{equation*}\limsup_{n \to + \infty} \mu_n(C) \leq \mu(C). \end{equation*}
\item[$\mathbf{(c_1)}$] For any $E \subseteq X$ such that $\mu \left(\partial E \right) = 0$, it turns out that
\begin{equation*}\lim_{n \to + \infty} \mu_n(E) = \mu(E). \end{equation*}
\end{enumerate}
\end{remark}

\begin{remark}If $\mu_n \to \mu$ is a converging sequence of $\R^n$-valued measures (not necessarily positive), then there is a subsequence $(n_k)_{k \in \N} \subset (n)_{n \in \N}$ such that
\begin{equation*} \left|\mu_{n_k}\right| \to \lambda,\end{equation*}
where $\lambda$ is a positive measure satisfying $\lambda \geq \left|\mu\right|$. Moreover, given a relatively compact set $E \subseteq X$ with $\lambda(E) = 0$, the reader may prove, as an exercise, that
\begin{equation*}\mu_n(E) \to \mu(E). \end{equation*}\end{remark}

\begin{exercise}Let $(\mu_n)_{n \in \N}$ be a sequence of $\R^n$-valued measures converging in variation to an element $\mu$. Prove that:\mbox{}
\begin{enumerate}[label=\textbf{(\arabic*)}]
\item The sequence of positive measures $\left( \left|\mu_n\right| \right)_{n \in \N}$ weakly-$\ast$ converges to $\left| \mu \right|$.
\item For every subset $E \subseteq X$ such that $\left|\mu\right|(E) = 0$, it turns out that $\mu_n(E) \to \mu(E)$.
\end{enumerate}
\end{exercise}
 
\begin{exercise}Let $X$ be a suitable ambient space, let $(x_n)_{n \in \N}$ be a sequence of points in $X$ such that $x_n \to x$ as $n \to + \infty$, and set $\mu_n := \delta_{x_n} - \delta_x$. Prove that
\begin{equation*} \left|\mu_n\right| = \delta_{x_n} \qquad \text{and} \qquad  \left|\mu_n\right| \to \left|\mu\right| = 0. \end{equation*}
\end{exercise} 
 
\section{Outer Measures}

In this section, we introduce the notion of \textit{outer measure}, and we set the ground for the main result of this chapter: the Carathéodory construction of a measure defined on the Borel $\sigma$-algebra $\mathcal{B}(X)$.
 
\begin{definition}[Outer Measure]\index{measure!outer measure} Let $X$ be a set. An \textit{outer measure} on $X$ is a set function $\mu : P(X) \longrightarrow [0, \, \infty]$, where $\mathcal{P}(X)$ denotes the power set, such that \mbox{}
\begin{enumerate}[label=\textbf{(\alph*)}]
\item $\mu(\varnothing) = 0$;
\item $\mu$ is monotone, i.e. $A \subseteq B \implies \mu(A) \leq \mu(B)$;
\item $\mu$ is $\sigma$-additive, i.e. $\{A_n\}_{n \in \mathbb{N}} \subset P(X) \implies \mu \left( \cup_{n} A_n \right) \leq \sum_{n} \mu \left(A_n \right)$.
\end{enumerate}
\end{definition}

\begin{example} A simple example of an outer measure, which is defined on every set $X$, is given by
\begin{equation*} \mu(E) := \begin{cases} n & \text{if $\left|E\right| = n$}, \\[0.4em] + \infty & \text{if $\left| E \right| \geq \aleph_0$}. \end{cases} \end{equation*} \end{example}


\begin{example}Let $X := \R$. The outer measure from which the Lebesgue measure derives is defined by
\begin{equation*}  \mu(E) := \inf \left\{ \left. \sum_{n \in \mathbb{N}} \left| I_n \right| \: \right| \: E \subset \bigcap_{n \in \mathbb{N}} I_n \right\},\end{equation*}
where the $I_n$ are, for example, open intervals.
\end{example}

\begin{definition}[Carathéodory measurable] \index{Carathéodory measurable} Let $\mu$ be an outer measure defined on $X$. A subset $A \subseteq X$ is \textit{Carathéodory $\mu$-measurable} if and only if
\begin{equation}\label{cars} \mu(E) = \mu(E \cap A) + \mu(E \cap A^c), \qquad \forall \, E \subseteq X. \end{equation} \end{definition}

\begin{theorem}[Carathéodory]\label{sigmachar}\index{Carathéodory Theorem}Let $\mathcal{M}$ be the class of Carathéodory $\mu$-measurable sets in $X$. Then $\mathcal{M}$ is a $\sigma$-algebra and the restriction of $\mu$ to $\mathcal{M}$, denoted by $\mu \restr \mathcal{M}$, is $\sigma$-additive. \end{theorem}

\begin{proof}To ease the notation for the reader we divide the proof into three steps.

\paragraph{Step 0.} The relation \eqref{cars} is symmetric with respect to the complement; hence $A \in \mathcal{M}$ immediately implies $A^c \in \mathcal{M}$, and vice versa. In particular, the empty set $\varnothing$ belongs to $\mathcal{M}$.

\paragraph{Step 1.} Let $A_1, \, A_2 \in \mathcal{M}$. For every $E \subseteq X$ it turns out that
\begin{equation*} \begin{aligned} \mu(E)  & = \mu(E \cap A_1) + \mu(E \cap A_1^c) =  \\[1em]
& = \mu(E \cap A_1 \cap A_2) + \mu(E \cap A_1 \cap A_2^c) + \mu(E \cap A_1^c) \geq  \\[1em]
 & \geq \mu\left(E \cap (A_1 \cap A_2)\right) + \mu\left(E \cap (A_1 \cap A_2)^c\right), \end{aligned}\end{equation*}
where the last inequality follows from the monotonicity property of $\mu$. The opposite inequality is always true as a consequence of the subadditivity of the measure; hence we can infer that
\begin{equation*} A_1, \, A_2 \in \mathcal{M} \implies A_1 \cap A_2 \in \mathcal{M}. \end{equation*}

\paragraph{Step 2.} Let $\{A_n\}_{n \in \mathbb{N}}$ be a disjoint family of elements of $\mathcal{M}$. Then
\begin{equation*} \begin{aligned}\mu(E) & = \mu(E \cap A_1) + \mu(E \cap A_1^c) =  \\[1em]
& = \mu(E \cap A_1) + \mu(E \cap A_1^c \cap A_2) + \mu(E \cap A_1^c \cap A_2^c) =  \\[1em]
& = \mu(E \cap A_1) + \mu(E \cap A_2) + \mu(E \cap A_1^c \cap A_2^c) = \\[1em]
& = \sum_{n = 1}^{m} \left[ \mu(E \cap A_n) \right] + \mu\left(E \cap \left(\bigcup_{n=1}^{m} A_n \right)^c \, \right). \end{aligned} \end{equation*}
If we take the supremum, the identity above yields to
\begin{equation*} \mu(E) \geq  \sum_{n = 1}^{+ \infty} \left[ \mu(E \cap A_n) \right] + \mu\left(E \cap \left(\bigcup_{n=1}^{+ \infty} A_n \right)^c \, \right), \end{equation*}
and thus, using the $\sigma$-subadditivity of the outer measure $\mu$, we obtain the nontrivial inequality
\begin{equation*} \mu(E) \geq \mu\left(E \cap \left(\bigcup_{n=1}^{+ \infty} A_n \right) \right) + \mu\left(E \cap \left(\bigcup_{n=1}^{+ \infty} A_n \right)^c \, \right), \end{equation*}
that is, the countable union of disjoint elements of $\mathcal{M}$ still belongs to $\mathcal{M}$.

\paragraph{Step 3.} In a similar fashion, one can prove that $\mu \restr \mathcal{M}$ is $\sigma$-additive. In fact, one can take a family of disjoint elements $\{A_n\}_{n \in \mathbb{N}}$ and take $E := \bigcup_{n \in \N} A_n$.
\end{proof}

This theorem, despite the simple proof, is incredibly powerful. Indeed, it is relatively easy to find an outer measure on $X$ and derive a $\sigma$-additive measure, while it is much harder to define it directly.

\begin{remark} The Carathéodory result, on the other hand, gives no information whatsoever on the $\sigma$-algebra $\mathcal{M}$. For example, if we consider the outer measure
\begin{equation*} \mu(E) := \begin{cases} 0 \qquad \text{if  } E = \varnothing \\ 1 \qquad \text{if  } E \neq \varnothing , \end{cases} \end{equation*}
then it is easy to prove that the $\sigma$-algebra associated via \hyperref[sigmachar]{Theorem \ref{sigmachar}}) is the trivial one, that is,
\begin{equation*}\mathcal{M} = \left\{ \varnothing, \, X \right\}, \end{equation*}
and thus it is not an interesting object to deal with. \end{remark}

The next theorem will give us an easy-to-check criterion for an outer measure to be "\textit{interesting}," in the sense that the $\sigma$-algebra $\mathcal{M}$ contains (at least) the Borel $\sigma$-algebra.

\begin{theorem} \label{theore:sigma}Let $X$ be a metric space and let $\mu$ be an outer measure defined over $X$. If
\begin{equation*} \mathrm{dist}(A_1, \, A_2) > 0 \implies \mu(A_1 \cup A_2) = \mu(A_1) + \mu(A_2), \end{equation*}
then $\mathcal{M}$ contains the Borel algebra. \end{theorem}

\begin{proof} First, we notice that it is enough to prove that closed sets belong to $\mathcal{M}$. We divide the argument into two steps: a particular, fairly straightforward, case and the general case.

\paragraph{Step 1.} Let $X$ be a Cantor-type set (in particular, it is totally disconnected). The identity
\begin{equation*} \mu(E) = \mu(E \cap I) + \mu(E \cap I^c), \qquad \forall \, E \subseteq X, \end{equation*}
holds for any interval $I$ contained in $X$. By construction, the interval $I$ and the complement $I^c$ are distant, and this concludes the proof since the intervals generate the Borel $\sigma$-algebra of $X$.

\paragraph{Step 2.} Let $C \subseteq X$ be a closed subset, and let us consider the sequence given by
\begin{equation*} A_n := \left\{ x \in X \: \left| \: d(x, \, C) \geq \frac{1}{n} \right. \right\}. \end{equation*}
The sequence $(A_n)_{n \in \N}$ is increasing, and its limit is the complement of $C$, that is,
\begin{equation*}A_n \uparrow \left(X \setminus C \right) = C^c. \end{equation*}
We may always assume, without loss of generality, that $\mu(E) < + \infty$. It follows from the monotonicity of $\mu$ that
\begin{equation*} \mu(E) \geq \mu(E \cap C) + \mu(E \cap A_n), \qquad \forall \, n \in \mathbb{N}, \end{equation*}
and hence, by taking the limit as $n \to + \infty$, an application of \hyperref[lemma:tech]{Lemma \ref{lemma:tech}} proves that
\begin{equation*} \mu \left(E \cap A_n \right) \to \mu \left(E \cap C^c \right), \end{equation*}
which is the nontrivial inequality.  \end{proof}

\begin{lemma} \label{lemma:tech} Let $A_n \uparrow A$ be an increasingly converging sequence of subsets, and assume that
\begin{equation} \label{ass:1230123}\mathrm{dist}(A_n, \, A_{n+1}^c) > 0 \qquad \forall \, n \in \mathbb{N}.\end{equation}
If the measure of the limit is finite ($\mu(A) < + \infty$), then $\mu(A_n) \uparrow \mu(A)$.\end{lemma}

\begin{proof} Let $a \in \R$ be the real number such that $\mu(A_n) \uparrow a$ (it exists, as a consequence of the monotonicity of the sequence). The reader can easily prove that
\begin{equation*}a \leq \mu(A). \end{equation*}
To prove the opposite inequality, we consider the sequence of sets defined by induction as
\begin{equation*} \begin{cases} B_1 := A_1, \\[0.4em] B_n := A_n \setminus B_{n-1}. \end{cases} \end{equation*}
By construction, for every $k \in \N$ it turns out that
\begin{equation*} \mu(A_k) \geq \mu(A) - \sum_{n = k+1}^{+\infty} \mu(B_n), \end{equation*}
and thus, if we can prove that the sum $\sum_{n \in \N} \mu(B_n)$ is finite, then we can take the limit as $k \to + \infty$ and find that
\begin{equation*} a = \liminf_{k \to + \infty} \mu(A_k) \geq \mu(A).  \end{equation*}
The assumption \eqref{ass:1230123} implies that $B_n$ and $B_{n + 2}$ are distant for every $n \in \mathbb{N}$; therefore we can decompose the sum as
\begin{equation*}\sum_{n \in \N} \mu(B_n) = \sum_{k = 0}^{+ \infty} \mu(B_{2k}) + \sum_{k = 0}^{+ \infty} \mu(B_{2k+1}). \end{equation*}
The measure $\mu$ satisfies the assumption of \hyperref[theore:sigma]{Theorem \ref{theore:sigma}}; hence
\begin{equation*}\sum_{k \in \N} \mu(B_{2k}) = \mu \left( \bigcup_{k = 0}^{+ \infty} B_{2k} \right) \leq \mu(A).\end{equation*}
In conclusion, the assumption $\mu(A) < \infty$ implies that
\begin{equation*}\sum_{n \in \N} \mu(B_n) \leq 2 \mu(A) < \infty, \end{equation*}
which is exactly the estimate we needed.
\end{proof}

\begin{lemma} \label{lemma:ousd} Let $(\mu_n)_{n \in \N}$ be a sequence of outer measures. If $\mu_n \nearrow \mu$ or $\mu_n \searrow \mu$, then also $\mu$ is an outer measure.\end{lemma}

\begin{proof} We may assume that $(\mu_n)_{n \in \N}$ is an increasing sequence of outer measures converging to some $\mu$, that is, $\mu_n \nearrow \mu$. The opposite case is obtained in a similar way.

\paragraph{Step 1.} The equality $\mu(\varnothing) = 0$ is trivial. If $A \subset B$, then
\begin{equation*} \mu_n(A) \leq \mu_n(B) \implies\mu(A) := \sup_{n \in \N} \mu_n(A) \leq \sup_{n \in \N} \mu_n(B) =: \mu(B). \end{equation*}

\paragraph{Step 2.} Let $\{A_k\}_{k \in \N}$ be any countable collection of subsets of $X$. Then
\begin{equation*} \mu_n\left(\bigcup_{k \in \N} A_k \right) \leq \sum_{k \in \N} \mu_n(A_k)\qquad \forall \, n \in \N, \end{equation*}
and it follows from Fatou's lemma that
\begin{equation*}\begin{aligned} \mu \left( \bigcup_{k \in \N}A_k \right) & = \limsup_{n \to + \infty} \mu_n\left(\bigcup_{k \in \N} A_k \right) \leq \\[1em] & \leq \limsup_{n \to +\infty} \sum_{k \in \N} \mu_n(A_k) \leq \\[1em] &\leq \sum_{k \in \N} \limsup_{n \to + \infty} \mu_n(A_k) = \\[1em] &= \sum_{k \in \N} \mu(A_k). \end{aligned}\end{equation*}\end{proof}
 
\begin{proposition} Let $X$ be a compact space, and let $\mu$ be a positive finite measure, defined on the Borel $\sigma$-algebra. For every $E \subseteq X$ Borel it turns out that
\begin{equation*} \begin{aligned} \mu(E) & = \inf \left\{ \mu(A) \: \left| \: \text{$A \supseteq E$, $A$ is open} \right. \right\} = \\[1em] & = \inf \left\{ \mu(K) \: \left| \: \text{$K \subseteq E$, $K$ compact} \right. \right\}. \end{aligned}\end{equation*}
The first identity is usually referred to as \textit{outer-regularity} of a measure; the second one \textit{inner-regularity} of a measure. \end{proposition}

\begin{proof}[Sketch of the Proof] First, we notice that the space $X$ is compact, and thus the inner-regularity follows from the outer-regularity by passing to the complement. If we define the measure
\begin{equation*}\nu(E) := \inf \left\{ \mu(A) \: \left| \: \text{$A \supseteq E$, $A$ is open} \right. \right\} ,\end{equation*}
then it is easy to prove that this is an outer measure (thus outer-regular by definition), which is additive on distant sets.

It follows from \hyperref[theore:sigma]{Theorem \ref{theore:sigma}} that $\nu$ is a positive measure on the Borel $\sigma$-algebra. Finally, since $\mu(A) = \nu(A)$ for every open subset $A$, a straightforward application of the \textit{Monotone Class Theorem}\footnote{A monotone class is a collection of sets which is closed under countable monotone union and intersection.} allows us to infer that $\mu \equiv \nu$ on the whole $\sigma$-algebra of Borel.\end{proof}
 
\begin{remark}Let $X$ and $Y$ be spaces satisfying suitable assumption (e.g., metric and locally compact.) If $E$ is a Borel subset of $X$, and $f:E\longrightarrow Y$ is a continuous function, then $f(E)$ is universally measurable\footnote{\textbf{Definition.} The set $f(E) \subset Y$ is universally measurable if and only if, for every finite positive measure $\mu$ on $Y$, there are two Borel sets $A \subseteq f(E) \subseteq B$ such that $\mu(A) = \mu(B)$.}\index{universally measurable}, but it is, in general, not Borel. \end{remark}

\section{Carathéodory Construction}
\label{sec:caco}

Let $X$ be a metric space, let $\mathcal{F} \subseteq \mathcal{P}(X)$ be a family of subsets, and let $\rho : \mathcal{F} \longrightarrow [0, \, + \infty]$ be the associated \textbf{gauge} function. We also assume that
\begin{enumerate}[label=(\alph*)]
\item $\varnothing \in \mathcal{F}$;
\item $\rho(\varnothing) = 0$.
\end{enumerate}
For every positive real number $\delta \in (0, \, + \infty]$ and every $E \in \mathcal{P}(X)$, we define
\begin{equation} \label{outmes} \Psi_\delta(E) := \inf \left\{ \sum_{n \in \mathbb{N}} \rho(F_n) \: \left| \: \{F_n\}_{n \in \N} \subset \mathcal{F}, \, \mathrm{diam}(F_n) < \delta, \, \bigcup_{n \in \mathbb{N}} F_n \supseteq E \right. \right\}, \end{equation}
where the diameter\index{set diameter} of a set $F$ is defined by setting
\begin{equation*} \mathrm{diam}(F) := \sup \left\{ d(x, \, x^\prime)  \: \left| \: x, \, x^\prime \in F \right. \right\}. \end{equation*}
It may not be clear at this moment, but it is crucial to take the infimum only over countable covers in the definition \eqref{outmes}. We assume that
\begin{equation*} \inf \, \varnothing = + \infty, \end{equation*}
in such a way that the function
\begin{equation*}(0, \, + \infty] \ni \delta \longmapsto \Psi_\delta \end{equation*}
is (weakly) decreasing. In particular, for any $E \in \mathcal{P}(X)$ we define
\begin{equation}\label{mess}\Psi(E) := \lim_{\delta \to 0^+} \Psi_\delta(E) = \sup_{\delta > 0} \Psi_\delta(E). \end{equation}

\begin{lemma} \label{lemma:prop1} \mbox{}
\begin{enumerate}[label=\textbf{(\alph*)}]
\item For any $\delta > 0$, the set function $\Psi_\delta$ is an outer measure and it is additive on sets which are distant more than $\delta$ in $\mathcal{F}$, that is, for every $A_1, \, A_2 \in \mathcal{F}$ such that $d(A_1, \, A_2) > \delta$, it turns out that
\begin{equation*} \Psi_\delta(A_1 \cup A_2) = \Psi_\delta(A_1) + \Psi_\delta(A_2). \end{equation*}
\item The limit set function $\Psi$ is an outer measure and it is additive on distant set of $\mathcal{F}$, that is, for every $A_1, \, A_2 \in \mathcal{F}$ such that $d(A_1, \, A_2) > 0$, it turns out that
\begin{equation*} \Psi(A_1 \cup A_2) = \Psi(A_1) + \Psi(A_2). \end{equation*}
In particular, the outer measure $\Psi$ satisfies the assumption of \hyperref[theore:sigma]{Theorem \ref{theore:sigma}}.
\end{enumerate} \end{lemma}

\begin{proof} \mbox{}
\begin{enumerate}[label=\textbf{(\alph*)}]
\item The reader may prove that $\Psi_\delta$ is an outer measure, as a relatively straightforward check of the characterizing properties; here we only show that it is additive on distant sets.

Let $\{ E_n \}_{n \in \N}$ be an admissible countable cover of $A_1 \cup A_2$, with $\mathrm{diam}(E_n) < \delta$. By assumption
\begin{equation*} d(A_1, \, A_2) > \delta, \end{equation*}
therefore it is immediate to check that, for any $n \in \N$, either $E_n \cap A_1 \neq \varnothing$ or $E_n \cap A_2 \neq \varnothing$. For $i = 1, \, 2$, let us set
\begin{equation*} I_i := \left\{ n \in \N \: : \ : E_n \cap A_i \neq \varnothing \right\}, \end{equation*}
and consider the countable covers $\mathscr{E} := \{ E_n \}_{n \in I_1}$ and $\mathscr{F} := \{ E_n \}_{n \in I_2}$ of $A_1$ and $A_2$ respectively. Then the nontrivial inequality follows easily:
\begin{equation*} \begin{aligned} \Psi_\delta(A_1) + \Psi_\delta(A_2) & \leq \sum_{n \in I_1} \rho(E_n) + \sum_{n \in I_2} \rho(E_n) = \\[1em]
& = \sum_{n \in \N} \rho(E_n) = \Psi_\delta(A_1 \cup A_2). \end{aligned} \end{equation*}
\item This assertion is an immediate consequence of \textbf{(a)} and \hyperref[lemma:ousd]{Lemma \ref{lemma:ousd}}.
\end{enumerate} \end{proof} 

\begin{corollary} The limit set function $\Psi$ is a $\sigma$-additive measure defined on the $\sigma$-algebra of Borel. \end{corollary}

\begin{example}[Lebesgue Measure] Let $X = \R^d$, and let us consider the collection of all \textit{rectangles}, that is,
\begin{equation*}\mathcal{F} = \left\{ I_1 \times \dots \times I_d \: \left| \: I_i \subset \R \, \, \text{interval} \right. \right\}. \end{equation*}
The gauge function associated to this collection is given by
\begin{equation*}\rho \left(I_1 \times \dots \times I_d \right) = \prod_{i = 1}^{d} \left| I_i \right|, \end{equation*}
where $|I|$ denotes the length of the interval $I \subset \R$. For every $\delta \in (0, \, + \infty]$, it is easy to prove that
\begin{equation*} \Psi_\delta = \Psi = \mathcal{L}^d, \end{equation*}
where $\mathcal{L}^d$ denotes the $d$-dimensional Lebesgue measure. Moreover, the limit function coincides with the gauge function on rectangles set, i.e.,
\begin{equation*} \Psi \left(I_1 \times \dots \times I_d \right) = \rho \left(I_1 \times \dots \times I_d \right). \end{equation*} \end{example}

\begin{exercise}Let $X = \mathbb{Q}$ and $\mathcal{F}$ be the family of $1$-dimensional rectangles, that is,
\begin{equation*}\mathcal{F} = \left\{ I \: \left| \: I \subset \R \, \, \text{interval} \right. \right\}. \end{equation*}
Prove that
\begin{equation*} \rho \left(I \right) = \left| I \right| \implies \Psi_\delta \equiv 0.\end{equation*}
\textit{Hint:} the set of all rational numbers $\mathbb{Q}$ is totally disconnected countable space, and thus we can take countable covers made up of points, whose length - in the sense of $\rho$ - is zero.)
\end{exercise}

\section{Hausdorff $d$-dimensional Measure $\mathcal{H}^d$}

Let $X$ be a metric space, let $d \in [0, \, \infty)$ be any positive real number, let $\mathcal{F} = \mathcal{P}(X)$ be the family of all subsets, and let
\begin{equation*} \rho \left(E \right) = \left( \mathrm{diam} (E) \right)^d \end{equation*}
be the gauge function. The Hausdorff outer measure is defined by formula \eqref{outmes}, but there is also a multiplicative constant (depending on $d$ only), that is,
\begin{equation*} \mathcal{H}_\delta^d(E) = c_d \Psi_\delta(E). \end{equation*}
The Hausdorff measure\index{Hausdorff measure} is the limit as $\delta \to 0^+$, that is, it is defined by formula \eqref{mess} up to the same multiplicative constant:
\begin{equation*}\mathcal{H}^d(E) = c_d \Psi(E). \end{equation*}
The constant $c_d$, for integer values of $d$, is defined as follows:
\begin{equation*} c_d := \frac{\alpha_d}{2^d}, \end{equation*}
where $\alpha_d$ is the measure of the $d$-dimensional unitary ball, that is,
\begin{equation*} \alpha_d = \frac{\pi^{d/2}}{\Gamma( \frac{d}{2} + 1 )}. \end{equation*}

\begin{lemma}[Properties of the Hausdorff Measure] \label{lemma:hausdorffprop} \mbox{}
\begin{enumerate}[label=\textbf{(\alph*)}]
\item If $d = 0$, then $\mathcal{H}^d$ is the counting measure.
\item If $X = \R^n$, then
\begin{equation*} \mathcal{H}^d \left( \lambda E \right) = \lambda^d \mathcal{H}^d \left( E \right), \qquad \forall \, \lambda > 0, \, \, \forall \, E \subseteq \R^n. \end{equation*}
\item If $f : E \subseteq X \longrightarrow Y$ is an isometry, then
\begin{equation*} \mathcal{H}^d \left( f(E) \right) = \mathcal{H}^d \left( E \right). \end{equation*}
\item If $f : E \subseteq X \longrightarrow Y$ is an $L$-Lipschitz map, then
\begin{equation*} \mathcal{H}^d \left( f(E) \right) \leq L^d \mathcal{H}^d \left( E \right). \end{equation*}
\item Let $X := \R^n$, and let $\Psi_\delta$ be the outer measure associated to the rectangles gauge function. For every $\delta \in (0, \, + \infty]$ and every $E \subseteq \R^n$, it turns out that
\begin{equation*} \mathcal{H}_\delta^d(E) = \Psi_\delta(E).\end{equation*}
Moreover, the Hausdorff $d$-dimensional measure and the Lebesgue $d$-dimensional measure coincide on $\R^n$, that is,
\begin{equation*} \mathcal{H}^d \equiv \mathcal{L}^d. \end{equation*}
\end{enumerate} \end{lemma}

\begin{proof} \mbox{}
\begin{enumerate}[label=\textbf{(\alph*)}]
\item Let $E$ be a finite set with $n := |E|$. Since $E$ is discrete, there exists a real number $\epsilon > 0$ such that any countable cover of $E$ made up by sets of diameter less than $\epsilon$, must be of cardinality at least $n$.

On the other hand, there is a trivial covering by $n$ nonempty sets of diameter less than $\epsilon$ (i.e., small neighborhoods of the $n$ points); hence $\mathcal{H}^0(E) = n$. Employing the monotonicity of the outer measures, this proves the claim.
\item First, we observe that any subset $A \subset \R^n$ has the following property: for every positive constant $\lambda > 0$, it turns out that
\begin{equation*} \mathrm{diam}\left(\lambda A \right) = \lambda \cdot \mathrm{diam}(A). \end{equation*}
Let $E \subseteq \R^n$ be given, and let $\{E_n\}_{n \in \N}$ be an optimal cover\footnote{A cover realizing the infimum in definition \eqref{outmes}.} for $E$ of diameter $\delta > 0$. It follows easily that $\{ \lambda E_n \}_{n \in \N}$ is a countable cover for the rescaling $\lambda E$, of diameter $\lambda \delta$. Hence
\begin{equation*}\begin{aligned} \mathcal{H}_{\lambda \delta}^d \left( \lambda E \right) & \leq c_d \sum_{n \in \mathbb{N}} \left(\mathrm{diam}\left( \lambda E_n \right) \right)^d = \\[1em] & = c_d \lambda^d \sum_{n \in \mathbb{N}} \left(\mathrm{diam}\left(E_n \right) \right)^d = \lambda^d  \mathcal{H}_{\delta}^d(E), \end{aligned}\end{equation*}
and, by taking the limit as $\delta$ approaches $0^+$, we infer that
\begin{equation*} \mathcal{H}^d \left( \lambda E \right) \leq \lambda^d \mathcal{H}^d \left( E \right). \end{equation*}
In a similar fashion, let $\{F_n\}_{n \in \N}$ be an optimal cover for $\lambda E$ of diameter $\delta > 0$. It follows easily that $\{ \lambda^{-1} E_n \}_{n \in \N}$ is a countable cover for $E$, of diameter $\lambda^{-1} \delta$. Hence
\begin{equation*}\begin{aligned} \mathcal{H}_{\lambda^{-1} \delta}^d \left( E \right) & \leq c_d \sum_{n \in \mathbb{N}} \left(\mathrm{diam}\left( E_n \right) \right)^d = \\[1em] & = c_d \lambda^{-d} \sum_{n \in \mathbb{N}} \left(\mathrm{diam}\left( \lambda E_n \right) \right)^d = \lambda^d  \mathcal{H}_{\delta}^d(\lambda E), \end{aligned}\end{equation*}
and, by taking the limit as $\delta$ approaches $0^+$, we infer that
\begin{equation*} \mathcal{H}^d \left( \lambda E \right) \geq \frac{1}{\lambda^d} \mathcal{H}^d \left( E \right). \end{equation*}
\item First, we notice that for every isometry $f : X \longrightarrow Y$ and every subset $E \subseteq X$, it turns out that
\begin{equation*} d(x, \, x^\prime) = d(f(x), \, f(x^\prime)) \implies \mathrm{diam}(E) = \mathrm{diam}\left(f(E) \right). \end{equation*}
The argument shown in \textbf{(b)} applies here without any significant change, provided that we take into account the property above.
\item Let $\{E_n\}_{n \in \mathbb{N}}$ be a countable cover of $E$ with diameter less than or equal to $\delta > 0$. The map $f$ is $L$-Lipschitz, and thus $\{f(E_n)\}_{n \in \mathbb{N}}$ is a countable cover of $f(E)$ with diameter less than or equal to $L \delta$. Therefore
\begin{equation*}\begin{aligned} \mathcal{H}_{L \delta}^d \left( f(E) \right) & \leq c_d \sum_{n \in \mathbb{N}} \left(\mathrm{diam}\left(f(E_n) \right) \right)^d \leq \\[1em] & \leq c_d  L^d  \sum_{n \in \mathbb{N}} \left(\mathrm{diam}\left(E_n \right) \right)^d = L^d \mathcal{H}_{\delta}^d(E), \end{aligned}\end{equation*}
and the thesis follows by taking the limit as $\delta$ approaches $0^+$.
\item This assertion is rather nontrivial, and we give a sketch of the proof after we introduce the Steiner symmetrization and the isoperimetrical inequality (see \hyperref[theorem:H=L]{Theorem \ref{theorem:H=L}}.)
\end{enumerate}\end{proof} 

\begin{lemma} \label{lemma:hausdsurf} Let $E$ be a Borel set contained in a $d$-dimensional surface $\Sigma \subset \R^n$ of class $C^1$. Then the $d$-dimensional Hausdorff measure of $E$ is the $d$-dimensional volume, that is,
\begin{equation*} \mathcal{H}^d(E) = \mathrm{vol}_{d}(E). \end{equation*} \end{lemma}

\begin{proof}The proof presented here is \textbf{not} rigorous, but the reader may try to expand it, fill the gaps and fix the imprecision as a particularly useful exercise.

\paragraph{Step 1.} Fix $\delta > 0$ and cover the surface $\Sigma$ with a countable collection $\mathcal{U} := \left\{U_n\right\}_{n \in \mathbb{N}}$ of sets such that, for every $n \in \N$, there is an almost-isometry
\begin{equation*} f_n : U_n \xrightarrow{ \: \sim \: } A_n \subseteq \R^d, \end{equation*}
where $A_n$ is a flat subset of $\R^d$. More precisely, there exists $\delta > 0$ such that
\begin{equation*} (1 - \delta) \cdot d\left(x, \, x^\prime\right) \leq d\left(f_n(x), \, f_n(x^\prime)\right) \leq  (1 + \delta) \cdot d\left(x, \, x^\prime\right), \end{equation*}
where $d$ is the distance of the metric space $X$.

\paragraph{Step 2.} The set $E$ may be written as the disjoint union of a collection of subsets $\mathcal{E} := \left\{E_n\right\}_{n \in \mathbb{N}}$ satisfying the inclusion $E_n \subseteq U_n$ for every $n \in \mathbb{N}$, that is,
\begin{equation*} E = \bigsqcup_{n \in \N} E_n. \end{equation*}
The $d$-dimensional volume of $E$ is given by the sum of the $d$-dimensional volumes of the $E_n$s, i.e.,
\begin{equation*}  \mathrm{vol}_{d}(E) = \sum_{n \in \mathbb{N}}  \mathrm{vol}_{d}(E_n), \end{equation*}
which is equal, taking the preimages via the almost-isometries, to
\begin{equation*}\mathrm{vol}_{d}(E) = \left( \sum_{n \in \mathbb{N}} \mathcal{L}^d \left(f_n^{-1}(A_n) \cap E_n\right) \right) \cdot (1 + \mathcal{O}(\delta)). \end{equation*}
The function $f_n$ is $(1 + \delta)$-Lipschitz; hence
\begin{equation*} \mathcal{H}^d(E) = \sum_{n \in \mathbb{N}} \mathcal{H}^d(E_n) = \left( \sum_{n \in \mathbb{N}} \mathcal{H}^d \left(f_n(E_n)\right) \right) \cdot (1 + \mathcal{O}(\delta)), \end{equation*}
and this concludes the proof since the right-hand side coincides with $\mathrm{vol}_{d}(E)$.\end{proof}
 
\begin{remark} The Hausdorff measure $\mathcal{H}^d$ does not change if we replace the family $\mathcal{F}$ with the following alternatives:
\begin{enumerate}[label=\textbf{(\arabic*)}]
\item \textbf{Closed Sets.} The diameter of a set $A$ coincides with the diameter of its closure $\overline{A}$.
\item \textbf{Open Sets.} Given a set $A$, for every $\epsilon > 0$ one can find an open set $B_\epsilon$ such that
\begin{equation*}A \subseteq B_\epsilon \qquad \text{and} \qquad \mathrm{diam}(B_\epsilon) \leq \mathrm{diam}(A) + \epsilon. \end{equation*}
\item \textbf{Convex Sets.} In the particular case $X = \R^n$, the reader may prove that the diameter of the convex hull of $A$ is equal to the diameter of $A$.
\end{enumerate}
On the other hand, one cannot replace the family $\mathcal{F}$ with, e.g., the family of balls\footnote{The reader may prove this assertion formally, but the rough idea behind it is evident: in general, a $r$-diameter set is \textbf{not} contained in a $r$-diameter ball (see, e.g., an equilateral triangle).}.\end{remark}

\paragraph{Hausdorff Spherical Measure.}\index{Hausdorff spherical measure} Let $X$ be a metric space, let $d \in [0, \, \infty)$ be any positive real number, let $\mathcal{F}_s$ be the family of all the balls, and let
\begin{equation*} \rho \left(E \right) = \left( \mathrm{diam} (E) \right)^d \end{equation*}
be the gauge function. The spherical Hausdorff outer measure is defined by formula \eqref{outmes}, but there is also a multiplicative constant (depending on $d$ only), that is,
\begin{equation*} \mathcal{H}_{\delta, \, s}^d(E) = c_d \Psi_{\delta, \, s}(E). \end{equation*}
The spherical Hausdorff measure is the limit as $\delta \to 0^+$, that is, it is defined by formula \eqref{mess} up to the same multiplicative constant:
\begin{equation*}\mathcal{H}_{s}^{d}(E) = c_d \Psi(E). \end{equation*}

\begin{lemma} There is a constant $C > 0$ such that
\begin{equation*} \mathcal{H}^d(E) \leq \mathcal{H}_s^d(E) \leq C \cdot \mathcal{H}^d(E), \qquad \forall \, E \subseteq X. \end{equation*} \end{lemma}
 
\begin{proof} Let $E \subseteq X$ be a subset of $X$, and let $\{E_n\}_{n \in \N}$ be a countable covering of $E$ with sets whose diameter is strictly less than $\delta$.

\paragraph{Step 1.} Let $x_n \in E_n$ be a sequence of points, and let us consider the family of balls defined by
\begin{equation*} B_n := B\left(x_n, \, \mathrm{diam}(E_n) \right). \end{equation*}
By construction, one can easily check that
\begin{equation*} E_n \subset B_n \qquad \text{and} \qquad \mathrm{diam} \left( B_n \right) = 2 \mathrm{diam}(E_n), \end{equation*}
and thus $\{B_n\}_{n \in \N}$ is a covering of $E$, made up of balls whose diameter does not exceed $2 \delta$. 

\paragraph{Step 2.} A straightforward application of the definitions proves the inequality
\begin{equation*}  \mathcal{H}_{2\delta, \, s}^d(E) \leq \sum_{n \geq 0} c_d  \left(\mathrm{diam}(B_n) \right)^d = 2^d  c_d  \sum_{n \geq 0} \left( \mathrm{diam}(E_n) \right)^d. \end{equation*}
If we take the infimum over all such families $\{E_n\}_{n \in \N}$, then we get the thesis with $C = 2^d$, that is,
\begin{equation*} \mathcal{H}^d(E) \leq \mathcal{H}_s^d(E) \leq 2^d \cdot \mathcal{H}^d(E), \qquad \forall \, E \subseteq X. \end{equation*} 
One easily notices that $2^d$ is not the sharp constant, and it can be improved up to
\begin{equation*} C(d) := \left(\frac{2 n}{n + 1} \right)^{\frac{d}{2}}, \end{equation*}
but we will not furnish a proof of this result in the course.
\end{proof}

\begin{lemma} \label{lemma:hausdsurf2} Let $E$ be a Borel set contained in a $d$-dimensional surface $\Sigma \subset \R^n$ of class $C^1$. Then
\begin{equation*} \mathcal{H}^d(E) = \mathrm{vol}_{d}(E) = \mathcal{H}_s^d(E). \end{equation*} \end{lemma}

\begin{proof}The argument is along the lines of the one used in \hyperref[lemma:hausdsurf]{Lemma \ref{lemma:hausdsurf}}. \end{proof}

\begin{lemma}Let $d < d^\prime$ be two real numbers, and let $E \subseteq X$ be any subset.\mbox{}
\begin{enumerate}[label=\textbf{(\arabic*)}]
\item If $\mathcal{H}^d(E)$ is finite, then $\mathcal{H}^{d^\prime}(E) = 0$.
\item If $\mathcal{H}^{d^\prime}(E) > 0$, then $\mathcal{H}^d(E) = + \infty$.
\end{enumerate}
In particular, for a fixed set $E$, the function $d \longmapsto \mathcal{H}^d(E)$ is decreasing, and it attains a finite nonzero value at most once.\end{lemma}

\begin{proof} \mbox{}
\begin{enumerate}[label=\textbf{(\arabic*)}]
\item Let $\{E_n\}_{n \in \N}$ be a covering of $E$, whose diameter does not exceed a fixed $\delta > 0$. By definition, it turns out that
\begin{equation} \label{103029931203} \mathcal{H}_\delta^{d^\prime}(E) \leq   \sum_{n \geq 0} \left(\mathrm{diam}(E_n) \right)^{d^\prime} \leq \delta^{d^\prime - d} \cdot \sum_{n \geq 0} \left(\mathrm{diam}(E_n) \right)^d, \end{equation}
which, in turn, implies that
\begin{equation*} \mathcal{H}_\delta^{d^\prime}(E) \leq \delta^{d^\prime - d} \cdot \mathcal{H}_\delta^d(E). \end{equation*}
In conclusion, we notice that $d^\prime - d$ is strictly greater than $0$, and hence the thesis follows by taking the limit as $\delta \to 0^+$.
\item The inequality \eqref{103029931203} may be rewritten as
\begin{equation*} \mathcal{H}_\delta^{d}(E) \geq \delta^{d - d^\prime} \cdot \mathcal{H}_\delta^{d^\prime}(E), \end{equation*}
and thus we conclude as in \textbf{(1)} since $d - d^\prime$ is strictly less than $0$.
\end{enumerate} \end{proof} 

\begin{definition}[Hausdorff Dimension]\index{Hausdorff dimension} Let $E \subseteq X$. The Hausdorff dimension of $E$, denoted by $\mathrm{dim}_{\mathcal{H}} \, E$, is the unique real number such that
\begin{equation*} \begin{cases}\mathcal{H}^{d} (E) = 0 & d > \mathrm{dim}_{\mathcal{H}} \, E \\[1em] \mathcal{H}^d(E) = + \infty & d < \mathrm{dim}_{\mathcal{H}} \, E. \end{cases} \end{equation*} \end{definition}

\begin{remark}[Basic Properties] \label{rmk:2o1dokk1} \mbox{} 
\begin{enumerate}[label=\textbf{(\alph*)}]
\item If $\underline{d} = \mathrm{dim}_{\mathcal{H}} \, E$ is the Hausdorff dimension of $E$, then $\mathcal{H}^{\underline{d}}(E)$ might be either zero or infinite (i.e., it is not necessarily finite).
\item The Hausdorff dimension of a countable union is equal to the supremum of the Hausdorff dimensions, that is,
\begin{equation*} \mathrm{dim}_\mathcal{H} \, \bigcup_{n \in \mathbb{N}}E_n = \sup_{n \in \mathbb{N}} \left\{ \mathrm{dim}_{\mathcal{H}} \, E_n\right\}. \end{equation*}
\item Let $(E_n)_{n \in \mathbb{N}} \subset \mathcal{P}(\R)$ and assume that $\mathrm{dim}_{\mathcal{H}} \, E_n \nearrow d$. The dimension of the union is equal to $d$, that is,
\begin{equation*} \mathrm{dim}_{\mathcal{H}} \, \bigcup_{n \in \mathbb{N}} E_n = d, \end{equation*}
and the Hausdorff measure of the union is zero:
\begin{equation*}\mathcal{H}^d \left( \bigcup_{n \in \mathbb{N}} E_n \right) = 0. \end{equation*}
\end{enumerate} 
 \end{remark}

\begin{exercise}Let $E \subseteq X$. The following properties are equivalent: \mbox{}
\begin{enumerate}[label=\textbf{(\alph*)}]
\item The Hausdorff measure of $E$ is zero, i.e. $\mathcal{H}^d(E) = 0$.
\item For every $\epsilon > 0$ there is a countable cover $\{E_n\}_{n \in \mathbb{N}}$, whose diameters satisfy the inequality
\begin{equation*}\sum_{n \geq 0} \left(\mathrm{diam}\left(E_n \right) \right)^d \leq \epsilon. \end{equation*}
\item The $\infty$-Hausdorff measure of $E$ is zero, i.e. $\mathcal{H}_\infty^d(E) = 0$.
\end{enumerate}
\end{exercise}

\begin{theorem}[$\mathcal{H}^d \equiv \mathcal{L}^d$] Let $X := \R^n$. For every $\delta > 0$ and every $E \subseteq X$, it turns out that
\begin{equation*} \mathcal{H}^d(E) = \mathcal{H}_\delta^d(E) = \mathcal{L}^d(E). \end{equation*}
\label{theorem:H=L} \end{theorem}

Before we can give the proof of this result, we need to introduce two fundamental tools (the isoperimetrical property of balls and the Steiner symmetrization), which are also extremely important per se.

\begin{theorem}[Isoperimetrical Property of Balls] \label{theorem:ismp} Let $B(r)$ be the \textbf{closed} ball centered at the origin of radius $r$ in $\R^d$. Among all closed sets with given diameter in $\R^d$, the ball $B(r)$ has maximum Lebesgue measure, that is,
\begin{equation*}c_d \left( \mathrm{diam}(B_r) \right)^d = \mathcal{L}^d\left(B(r) \right) \geq \mathcal{L}^d(E), \qquad \forall \, E \subseteq X \: : \: \mathrm{diam}(E) = 2r. \end{equation*}\end{theorem} 

\begin{proof}[Sketch] The proof of the isoperimetrical property is a direct consequence of the Steiner symmetrization (the reader may consult \cite[pp 195--198]{fef12d1} for a rigorous argument.)

\paragraph{Symmetrization Construction.} Let $E \subseteq \R^d$ be a bounded closed subset and let $V$ be an affine hyperplane in $\R^d$. For every point $x \in V$, we may consider the $1$-dimensional subspace $e_x$, orthogonal to $V$, and replace the intersection $e_x \cap E$ by a closed segment, centered at $x_0$, with the same length\footnote{Hausdorff measure.}. From now on, we denote by $\widetilde{E}$ the symmetrization of $E$.

\paragraph{Symmetrization Properties.} Let $E \subseteq \R^d$ be a bounded closed subset and let $V$ be an affine hyperplane in $\R^d$. The reader may prove that the Lebesgue measure does not change, that is,
\begin{equation*}\mathcal{L}^d(E) = \mathcal{L}^d \left( \widetilde{E} \right), \end{equation*}
while the diameter decreases, that is,
\begin{equation*}\mathrm{diam}(E) \geq \mathrm{diam} \left( \widetilde{E} \right). \end{equation*}
Moreover, one could easily prove that the set $\widetilde{E}$ is closed.

\paragraph{Proof ($d = 2$).} In this particular case the proof is fairly straightforward (it suffices to look at \hyperref[fig:sss]{Figure \ref{fig:sss}}), but the idea for $d > 2$ is exactly the same.

Let $E$ be any subset of $\R^2$, and consider an orthogonal basis $\{e_1, \, e_2\}$ of the real plane. If we denote by $\widetilde{E}$ the Steiner symmetrization of $E$ with respect to $e_1$ first, and $e_2$ after, then it is easy to prove that $\widetilde{E}$ symmetric with respect to the origin, and it is thus contained in a ball of diameter $\mathrm{diam}\left(\widetilde{E} \right)$. \end{proof}

\begin{figure}[h]
\centering
\includegraphics[width=15cm, height=8cm]{images/TGM11.png}
\caption{Idea of the proof in dimension $2$}
\label{fig:sss}
\end{figure}

%\begin{lemma} \label{lemma:ballsf} Let $B := B(x, \, r)$ be a ball in $\R^d$, and let $\delta, \, \epsilon > 0$ be positive real numbers. Then there exist a collection of countably many balls $B_n := B_n(x_n, \, r_n)$ such that the
%\begin{equation*} r_n \leq \frac{\delta}{2} \qquad \text{and} \qquad  \sum_{n \in \mathbb{N}} \mathcal{L}^d(B_n) \leq \mathcal{L}^d(B) + \epsilon. \end{equation*} \end{lemma}

%\begin{proof}This result follows easily from the \hyperref[theorem:vct]{Vitali Covering Theorem \ref{theorem:vct}}, which is introduced in the next chapter. \end{proof}

\begin{proof}[Proof of Theorem \ref{theorem:H=L} ] Recall that the normalization constant of the Hausdorff measure is given by
\begin{equation*} c_d = \frac{\alpha_d}{2^d}, \end{equation*}
and thus, for every ball $B_{x, \, r} := B(x, \, r) \subseteq \R^d$, it turns out that
\begin{equation*}\mathrm{vol}_d(B_{x, \, r}) = c_d \left( \mathrm{diam} (B_{x, \, r}) \right)^d. \end{equation*}

\paragraph{Step 1.} First, we notice that it suffices to prove that
\begin{equation*} \mathcal{H}_\infty^d(E) \geq \mathcal{L}^d(E), \qquad \forall \, E \subseteq \R^d, \end{equation*}
in order to obtain the first inequality, that is,
\begin{equation*} \mathcal{H}^d(E) \geq \mathcal{L}^d(E), \qquad \forall \, E \subseteq \R^d. \end{equation*}
Let $\{E_n\}_{n \in \N}$ be a countable cover of $E$. It follows from the isoperimetrical property of balls (see \hyperref[theorem:ismp]{Theorem \ref{theorem:ismp}}) that
\begin{equation*} c_d \left( \mathrm{diam}(E_n) \right)^d \geq \mathcal{L}^d(E_n),\end{equation*}
and thus
\begin{equation*} c_d \sum_{n \in \mathbb{N}} \left( \mathrm{diam}(E_n) \right)^d = \sum_{n \in \mathbb{N}} c_d \left( \mathrm{diam}(E_n) \right)^d \geq  \mathcal{L}^d(E).\end{equation*}
By taking the infimum of the left-hand side, we conclude that
\begin{equation*} \mathcal{H}_\infty^d(E) \geq  \mathcal{L}^d(E).\end{equation*}

\paragraph{Step 2.} The opposite inequality follows if we are able to prove that
\begin{equation*} \mathcal{H}_\delta^d(E) \leq \mathcal{L}^d(E), \qquad \forall \, E \subseteq \R^d, \, \, \, \forall \, \delta > 0. \end{equation*}
Let $\{E_n\}$ be a countable optimal cover for computing the Lebesgue measure\footnote{In particular, the cover does not need to be optimal for computing the Hausdorff measure.} of $E$. The optimal cover cannot be made up of cubes or rectangles\footnote{The reader may prove, as an exercise, that covers made up of cubes or rectangles yield to the sought inequality up to a constant strictly bigger than one.}, but by \hyperref[lemma:vcl]{Vitali's Covering Lemma \ref{lemma:vcl}} it follows that it can be chosen among all covers made up of (closed) balls. Moreover, for every fixed $\epsilon > 0$, there exists a collection of balls $\{B_n\}_{n \in \mathbb{N}}$ such that
\begin{equation*} \epsilon + \mathcal{L}^d(E) \geq \sum_{n \in \mathbb{N}} \mathcal{L}^d(B_n) = \sum_{n \in \mathbb{N}} c_d \left(\mathrm{diam}(B_n) \right)^d \geq \mathcal{H}_\delta^d(E), \end{equation*}
and this concludes the proof by arbitrariness of $\delta > 0$.
\end{proof} 

\section{Hausdorff Dimension of Cantor-Type Sets}
\label{sec:cts}

In this brief section, we compute the Hausdorff dimension of the standard Cantor set $\mathcal{C}$, and we generalize the process to more elaborate sets.

\begin{lemma} \label{lemma:cset} Let $\mathcal{C}$ be the standard Cantor set. Its Hausdorff dimension is given by
\begin{equation*}\underline{d} := \mathrm{dim}_{\mathcal{H}} \left(  \mathcal{C} \right) = \frac{\ln 2}{\ln 3}, \end{equation*}
and the Hausdorff measure is finite, that is,
\begin{equation*} 0 < \mathcal{H}^{\underline{d}} \left( \mathcal{C} \right) < + \infty. \end{equation*} \end{lemma}

\begin{proof} In this proof for simplicity purposes we assume that the renormalization constant $c_d$ is equal to $1$, and we prove that the Hausdorff measure of $\mathcal{C}$ is bounded as follows:
\begin{equation*} \frac{1}{2} \leq \mathcal{H}^{\underline{d}}( \mathcal{C} ) \leq 1. \end{equation*}

\paragraph{Upper bound.} The bound from above is, as usual, relatively easy to prove: it is enough to find a cover for the Cantor set $\mathcal{C}$ satisfying the bound. By definition
\begin{equation*} \mathcal{C} = \bigcap_{k = 0}^{\infty} \left( \bigcup_{i = 1}^{2^k} I_{k, \, i} \right), \end{equation*}
where the diameter of each interval is given by
\begin{equation*} \left| I_{k, \, i} \right| = 3^{-k}. \end{equation*}
Fix $\delta > 0$, and consider the minimal integer $k \in \N$ such that $3^{-k} < \delta$. Clearly
\begin{equation*} \mathcal{C} \subset \bigcup_{i = 1}^{2^k} I_{k, \, i}, \end{equation*}
and this implies, by $\sigma$-subadditivity, that
\begin{equation*} \mathcal{H}_{\delta}^{\underline{d}} \left( \mathcal{C} \right) \leq \sum_{i=1}^{2^k} \mathcal{H}_{\delta}^{\underline{d}} \left( I_{k, \, i} \right) \leq 2^k \frac{1}{3^{\underline{d} k}} = 1.\end{equation*}

\paragraph{Lower bound.} First, we notice that
\begin{equation*} \mathcal{H}_\delta^{\underline{d}}\left( \mathcal{C} \right) \geq \mathcal{H}_\infty^{\underline{d}}\left( \mathcal{C} \right), \end{equation*}
and thus it is enough to prove the lower bound for the $\infty$ measure, that is,
\begin{equation*} \mathcal{H}_\delta^{\underline{d}}\left( \mathcal{C} \right) \geq \mathcal{H}_\infty^{\underline{d}}\left( \mathcal{C} \right) \stackrel{?}{\geq} \frac{1}{2}. \end{equation*}
Let $\{E_n\}_{n \in \mathbb{N}}$ be a countable cover of $\mathcal{C}$, and assume that each $E_n$ is open and convex. Moreover, by compactness, we may assume that there are only finitely many. Fix $n \in \mathbb{N}$ and take the smallest interval $I_{k, \, i} := I_n$ that contains $E_n$. The reader may prove by herself that
\begin{equation*}\left| I_n \right| \leq 3 \left|E_n\right|, \end{equation*}
as a consequence of the fact that, if $I_n$ splits into $I_{n}^{\prime}$ and $I_n^{\prime \prime}$, then $E_n$ must intersect both\footnote{\textbf{N.B.} There could be points of $E_n$ lying between $I_{n}^{\prime}$ and $I_n^{\prime \prime}$, but, since they do not belong to the Cantor set, we can ignore them (with some care).} (otherwise $I_n$ would not be minimal, as required.) As a consequence of our claim, it turns out that
\begin{equation*} \begin{aligned} \sum_{n \in \mathbb{N}} \left(\mathrm{diam}\left(E_n \right) \right)^d & \geq 3^{-d} \sum_{n \in \mathbb{N}} \left(\mathrm{diam}\left(I_n \right) \right)^d = \\[1em] & =3^{-d} = \frac{1}{2}.\end{aligned}\end{equation*}
The equality is left as an exercise for the reader. The rough idea behind it is to throw away the repeated intervals $I_n$ in the previous sum; it is not hard to prove that, in this way, we end up with finitely many.
\end{proof}

\begin{remark}Let $\mathcal{C}$ be a Cantor set and let $\underline{d}$ be its Hausdorff dimension. The lower bound is not sharp, and one could prove that
\begin{equation*} \mathcal{H}^{\underline{d}}(\mathcal{C}) = 1.\end{equation*}
\end{remark}

\begin{exercise}[Cantor-Type Set] Fix $0 < \lambda < 1$, and let us consider the following partition of the unitary interval:
\begin{equation*} I_\lambda = \left[0, \, \frac{\lambda}{2} \right] \sqcup \left[\frac{\lambda}{2}, \, 1 - \frac{\lambda}{2} \right] \sqcup \left[1 - \frac{\lambda}{2}, \, 1 \right]. \end{equation*}
Let $\mathcal{C}_\lambda$ be the generalized Cantor set, that is
\begin{equation*} \mathcal{C}_\lambda = \bigcap_{k = 0}^{\infty} \left( \bigcup_{i = 1}^{2^k} I_{k, \, i}^{(\lambda)} \right), \end{equation*}
where $|I_{k, \, i}^{(\lambda)}| = \left(\frac{\lambda}{2} \right)^k$. Prove that the Hausdorff dimension of $\mathcal{C}_\lambda$ is given by the solution of the equation
\begin{equation*} 2 \left( \frac{\lambda}{2} \right)^{d_\lambda} = 1, \end{equation*}
and, actually, it turns out that
\begin{equation*} 0 < \mathcal{H}^{d_\lambda}( \mathcal{C}_\lambda ) \leq 1. \end{equation*}\end{exercise}

\begin{remark} The Hausdorff dimension of $\mathcal{C}_\lambda$ is explicitly given by
\begin{equation*} d_\lambda = \frac{\log 2}{\log 2 - \log \lambda}, \end{equation*}
and thus we get any Hausdorff dimension in $(0, \, 1)$ as $\lambda$ ranges in $[0, \, 1]$. \end{remark}

\chapter{Covering Theorem}

 In this chapter, we investigate several covering theorems, that is, the possibility, given a family $\mathcal{F}$ of balls covering $E$, to extract a subfamily $\mathcal{F}^\prime$ which covers $E$ and behaves "better."

More precisely, let $X$ be a metric space and $\mu$ a measure define on $\mathcal{B}(X)$; we would like to extract a subfamily $\mathcal{F}^\prime$ satisfying one of the following properties: \mbox{}
\begin{enumerate}[label=\textbf{(\alph*)}]
\item The collection $\mathcal{F}^\prime$ is a disjoint covering of $E$. It is, clearly, the best possibility we could be hoping for, but, unfortunately, it is false\footnote{\textbf{Exercise.} Prove that the unitary square $[0, \,1]^2$ in $\R^2$ cannot be covered by disjoint balls.} even in $\R^2$.
\item The collection $\mathcal{F}^\prime$ is a covering of $E$ made up of balls that do not overlap "too much" (i.e., the measure of the overlapping portion is arbitrarily small.)
\item The collection $\mathcal{F}^\prime$ is a disjoint covering of $\mu$-almost all of $E$, that is, the portion of $E$ that is not covered has measure zero.
\item The collection $\mathcal{F}^\prime$ is a covering of $E$ satisfying the inequality
\begin{equation*} \sum_{B \in \mathcal{F}^\prime} \mu(B) \leq \mu(E) + \epsilon. \end{equation*}
\end{enumerate}
In this chapter, we are mainly concerned with two classes of covering theorems, depending on the ambient space: \mbox{}
\begin{enumerate}[label=\textbf{(\arabic*)}]
\item Covering theorems that works on \textit{every} metric space $X$, provided that the measure $\mu$ satisfies some extra assumptions.
\item Covering theorems that works only on $X := \R^n$, with no requirement on the measure $\mu$.
\end{enumerate}

\section{Vitali Covering Theorem}

In this section, to ease the notation we denote by $B(x, \, r)$ the \textbf{closed} ball of center $x$ and radius $r$, and by $\widehat{B}(x, \, r)$ the closed ball of center $x$ and radius $5r$.

\begin{lemma}[Vitali Covering Lemma] \index{Vitali Covering Lemma} \label{lemma:vcl}Let $X$ be a metric space, let $\mathcal{F} := \{ B(x_i, \, r_i) \}_{i \in I}$ be a family of closed balls with uniformly bounded radii that covers a set $E \subseteq X$. Then there exists a disjoint subfamily $\mathcal{F}^\prime$ such that the rescaling
\begin{equation*} \widehat{\mathcal{F}}^\prime := \{ \widehat{B}(x_i, \, r_i) \: \left| \: B(x_i, \, r_i) \in \mathcal{F}^\prime \right. \} \end{equation*}
is a covering of $E$. \end{lemma}

\begin{remark}The uniform bound on the radii of the family $\mathcal{F}$ is fundamental, and it is impossible to drop it. Indeed, there is a simple counterexample in the real plane $\R^2$. The family
\begin{equation*} \mathcal{F} := \left\{ B(n, \, n) \: \left| \: n \in \N \right. \right\}, \end{equation*}
is an increasing covering of the upper-half plane $E$. Therefore, any disjoint subfamily is necessarily formed by a single ball, and hence it cannot cover $E$ for any $n \in \N$ (see Figure \ref{fig:example1}.)\end{remark}

\begin{figure}[h]
\centering
\includegraphics[width=14cm, height=7cm]{images/TGM1.jpeg}
\label{fig:example1}
\caption{Necessity of the uniform boundedness on radii}
\end{figure}

\begin{proof}We first prove the result by assuming some additional properties, and then we reduce the general case to it with a simple trick.

\paragraph{Step 1.} Assume that the family $\mathcal{F} = \{ B(x_n, \, r_n) \}_{n \in \N}$ is a countable cover of $E$, and assume also that the sequence of the radii is weakly decreasing, i.e. $r_1 \geq r_2 \geq \dots$. To construct the disjoint subfamily, we proceed as follows: \mbox{}
\begin{enumerate}[label=\textbf{(\arabic*)}]
\item The first ball $B_1$ automatically belongs to $\mathcal{F}^\prime$.
\item The second ball $B_2$ either intersect $B_1$ or it does not. If the latter holds true, then we add $B_2$ to the family $\mathcal{F}^\prime$; if the former alternative holds true, we solely throw it away.
\item Iterate the process for every $n \geq 3$ - taking into account all of the intersections with the balls that were previously added to the subfamily $\mathcal{F}^\prime$.
\end{enumerate}
At this point, we only need to prove that the rescaled collection of balls $\widehat{\mathcal{F}}^\prime$ is a covering of $E$. Clearly, we may equivalently show that every ball $B_n$ is a subset of an element in $\widehat{\mathcal{F}}^\prime$. In particular, for every $n \in \N$ there are only two possible outcomes:  \mbox{}
\begin{enumerate}[label=\textbf{(\roman*)}]
\item The ball $B_n$ already belongs to $\mathcal{F}^\prime$, and thus there is nothing to prove.
\item The ball $B_n$ intersects the ball $B_m$ for some $m < n$. The distance between the centers $x_n$ and $x_m$ is less than or equal to the sum of the radii, which means that
\begin{equation*} d(x_m, \, x_n) \leq r_m + r_n \leq 2 r_m \implies B(x_n, \, r_n) \subset B(x_m, \, 3 r_m) \subset \widehat{B}_m. \end{equation*}
\end{enumerate}

\paragraph{Step 2.} In the general case, we cannot assume that the family is countable or that the radii give an increasing sequence, and thus everything depends on the boundedness assumption. More precisely, there exists a real number $R > 0$ such that
\begin{equation*} B(x_i, \, r_i) \in \mathcal{F} \implies r_i \leq R.\end{equation*}
Let us consider the following partition
\begin{equation*} \mathcal{F}_n := \left\{ B \in \mathcal{F} \: \left| \: \frac{R}{2^{n+1}} < r \leq \frac{R}{2^n} \right. \right\}, \qquad \forall \, n \in \N, \end{equation*}
in such a way that we can extract a subfamily from each $\mathcal{F}_n$ as follows: \mbox{}
\begin{enumerate}[label=\textbf{(\roman*)}]
\item Extract a maximal (with respect to the inclusion) disjoint subfamily $\mathcal{F}_0^\prime$ from $\mathcal{F}_0$.
\item Extract a maximal disjoint subfamily $\mathcal{F}_1^\prime$ from $\mathcal{F}_1$, satisfying the additional requirement that it does not intersect any element of $\mathcal{F}_0^\prime$.
\item Iterate the process for $n \geq 3$ - taking into account all of the intersections with the subfamilies already chosen in the previous steps.
\end{enumerate}
Let $\mathcal{F}^\prime = \cup_{n \in \N} \mathcal{F}_n^\prime$: we need to show that the rescaled family $\widehat{\mathcal{F}}^\prime$ is a covering of $E$. Clearly, we may equivalently show that every ball $B_n$ is a subset of an element in $\widehat{\mathcal{F}}^\prime$. In particular, for every $B(x, \, r) \in \mathcal{F}_N$ there are only two possible outcomes: \mbox{}
\begin{enumerate}[label=\textbf{(\roman*)}]
\item The ball $B(x, \, r)$ already belongs to $\mathcal{F}_M^\prime$ for some $M \in \N$, and thus there is nothing to prove.
\item The ball $B(x, \, r)$ intersects the ball $B(y, \, s)$, with $B(y, \, s) \in \mathcal{F}_M^\prime$ for some $M \leq N$. It follows that $r \leq 2s$, and therefore the distance between the centers is less than or equal to the sum of the radii, which means that
\begin{equation*} d(x, \, y) \leq r + s \leq 3 s \implies B(x, \, r) \subseteq \widehat{B}(y, \, s). \end{equation*}
\end{enumerate}
\end{proof}

\begin{theorem}[Vitali Covering Theorem] \index{Vitali Covering Theorem}\label{theorem:vct} Let $X$ be a metric space, let $\mu$ be a doubling\index{measure!doubling}\footnote{\textbf{Definition.} A measure $\mu$ is doubling if and only if there exists a positive constant $C > 0$ such that
\begin{equation*} \mu \left( B(x, \, 2r) \right) \leq C \, \mu \left( B(x, \, r) \right). \end{equation*}} locally finite measure\index{measure!locally finite}\footnote{\textbf{Definition.} A measure $\mu$ is locally finite if and only if for every $x \in X$ there exists a neighborhood $U_x \ni x$ such that $\mu(U_x) < + \infty$.}, and let $E \subseteq X$ be a Borel set. Then, for every $\epsilon > 0$ and every $\mathcal{F}$ fine cover\footnote{\textbf{Definition.} Let $\mathcal{F}$ be a family of closed balls that covers $E$. We say that $\mathcal{F}$ is fine if and only if each $x \in E$ is the center of a closed ball in $\mathcal{F}$ with arbitrarily small radius.}. of $E$ made up of closed balls, there exists a disjoint subfamily $\mathcal{F}^\prime \subset \mathcal{F}$ which covers $\mu$-almost all of $E$, that is,
\begin{equation*} \mu \left( E \setminus \bigcup_{B \in \mathcal{F}^\prime} B \right) = 0, \end{equation*}
and the mass does not exceed $\mu(E)$ "too much", that is,
\begin{equation} \label{eq:vtsssdt} \sum_{B \in \mathcal{F}^\prime} \mu(B) \leq \mu(E) + \epsilon. \end{equation} \end{theorem}

\begin{remark}The statement of the Vitali covering theorem makes sense, but there is a topological issue which may not be apparent: The disjoint subfamily needs to be countable, and thus we need some additional assumptions on $X$ (locally compact, separable, etc.)\end{remark}

\begin{figure}[h]
\centering
\includegraphics[width=14cm, height=7cm]{images/TEOGEOM2.jpeg}
\label{fig:example2}
\caption{Idea of the proof}
\end{figure}

\begin{proof}Assume, without loss of generality, that $\mu(E) < \infty$, and fix an open neighborhood $A_0$ of $E$ satisfying the additional property
\begin{equation}\label{oqwe912jidoe109e1} \mu \left(A_0 \setminus E \right) \leq \epsilon \wedge \frac{\mu(E)}{2  C^{3}}, \end{equation}
where $C$ is the doubling constant.

\paragraph{Step 1.} Let us consider the collection of balls
\begin{equation*} \mathcal{F}_0 := \left\{ B(x, \, r) \in \mathcal{F} \: \left| \: B(x, \, r) \subseteq A_0  \right.\right\}. \end{equation*}
By construction, this is still a cover of $E$, and thus, by \hyperref[lemma:vcl]{Lemma \ref{lemma:vcl}}, it follows that there exists a disjoint subfamily of balls $\mathcal{F}_0^\prime$ such that the rescaling $\widehat{\mathcal{F}}_0^\prime$ covers $E$.

\paragraph{Step 2.} We apply the doubling property of $\mu$ three times (since $2^2 < 5 < 2^3$), and it is easy to see that we can find a lower bound for the measure of $\mathcal{F}_0^\prime$:
\begin{equation*} \mu(E) \leq \sum_{B \in \mathcal{F}_0^\prime} \mu\left(\widehat{B} \right) \leq C^3 \sum_{B \in \mathcal{F}_0^\prime} \mu(B) \implies \sum_{B \in \mathcal{F}_0^\prime} \mu(B) \geq \frac{\mu(E)}{C^3}.\end{equation*}
In a similar fashion\footnote{\textit{Hint.} Decompose $E$ as the disjoint union of the interior part ($\cap$) and the exterior part ($\setminus$). Then, apply the subadditivity of the measure and the assumption \eqref{oqwe912jidoe109e1} on the measure of the open set$A_0$.}, we obtain the following estimate\footnote{\textbf{Notation.} The union of all the elements in a collection is denoted by
\begin{equation*} \bigcup \mathcal{F} := \bigcup_{B \in \mathcal{F}} B.\end{equation*} }:
\begin{equation*}\begin{aligned}  C^3 \sum_{B \in \mathcal{F}_0^\prime} \mu(B) & \leq C^3 \left( \mu \left( \bigcup \mathcal{F}_0^\prime \cap E \right) +\mu \left( \bigcup \mathcal{F}_0^\prime \setminus E \right)  \right) \leq \\[1em] & \leq C^3 \mu \left( \bigcup \mathcal{F}_0^\prime \cap E \right) + \frac{\mu(E)}{2}, \end{aligned} \end{equation*}
from which it follows that
\begin{equation*} \frac{\mu(E)}{2 \, C^3} \leq \mu \left( \left( \bigcup \mathcal{F}_0^\prime \right) \cap E \right).\end{equation*}
As an immediate consequence, the measure of the portion of $E$ that has not been covered yet by $\mathcal{F}_0^\prime$ may be estimated as follows:
\begin{equation} \label{91jido3ie0392jf} \mu\left(E \setminus \bigcup \mathcal{F}_0^\prime \right) \leq \left( 1 - \frac{1}{2 C^3} \right) \mu(E). \end{equation}

\paragraph{Step 3.} Let us take a suitable \textbf{finite} subfamily $\mathcal{F}_0^{\prime \prime}$, and let $E_1 := E \setminus \bigcup \mathcal{F}_0^{\prime \prime}$. The inequality \eqref{91jido3ie0392jf} proves that
\begin{equation*} \mu\left(E_1 \right) \leq \left( 1 - \frac{1}{3 C^3} \right) \mu(E). \end{equation*}
The process can now be iterated in the following way: Let $A_1$ be an open neighborhood of $E_1$ satisfying the additional requirement\footnote{Here we use the finiteness of the subfamily $\mathcal{F}_0^{\prime \prime}$ to ensure that the complement is an open set.} that $E_1 \subset A_1 \subset \left(\bigcup \mathcal{F}_0^{\prime \prime} \right)^c$, and notice that everything works out smoothly as in the previous steps. In particular, it turns out that
\begin{equation*} \mu\left(E_k \right) \leq \left( 1 - \frac{1}{3 C^3} \right)^k \mu(E), \end{equation*}
and thus the inevitable candidate is given by
\begin{equation*} \mathcal{F}^\prime := \bigcup_{n \in \N} \mathcal{F}_n^{\prime \prime}. \end{equation*}
The reader may check, as an exercise, that the following properties hold (and conclude the proof): \mbox{}
\begin{enumerate}[label=\textbf{(\arabic*)}]
\item The family $\mathcal{F}^\prime$ is disjoint.
\item The family $\mathcal{F}^\prime$ is a covering of $\mu$-almost all of $E$. Hint: prove that
\begin{equation*} E \setminus \bigcup_{n \in \N} \mathcal{F}_n^{\prime \prime} = \bigcap_{n \in \N} E_n. \end{equation*}
\item The measure is arbitrarily near to the measure of $E$, that is,
\begin{equation*} \sum_{B \in \mathcal{F}^\prime} \mu(B) \leq \mu(E) + \epsilon. \end{equation*}
\end{enumerate}
\end{proof}

\begin{lemma} \label{lemma:measind} Let $X$ be a metric space, and let $\mu$ be a doubling locally finite measure. Let $E \subseteq X$ be a Borel subset, and let $\mathcal{F}$ be a fine cover of $E$. Then there exist a finite disjoint subfamily $\mathcal{F}^{\prime \prime}$ and a positive real number $\delta > 0$ such that 
\begin{equation*}\mu \left(E \setminus \bigcup \mathcal{F}^{\prime\prime} \right) \leq \delta \cdot \mu(E). \end{equation*}\end{lemma}

\begin{proof} The statement follows easily from the proof of the \hyperref[theorem:vct]{Vitali's Covering Theorem \ref{theorem:vct}}. \end{proof}

\begin{lemma} \label{lemma:meas0} Let $X$ be a metric space, and let $\mu$ be a doubling locally finite measure. Let $E_0 \subset X$ be a Borel null-set, let $\epsilon > 0$ be a positive real number, and let $\mathcal{F}$ be a fine cover of $E$. Then there exist a subfamily $\mathcal{F}^{\prime\prime}$, which covers $E_0$, such that
\begin{equation*} \sum_{B \in \mathcal{F}^{\prime\prime}} \mu(B) \leq \epsilon. \end{equation*}
\end{lemma}

\begin{proof} Let $A_0$ be an open neighborhood of $E_0$ satisfying the inequality
\begin{equation*} \mu(A_0) \leq \frac{\epsilon}{C^3}. \end{equation*}
Let us consider the family of balls
\begin{equation*} \mathcal{G} := \left\{ B_i := B(x_i, \, r_i) \: \left| \: \text{$B_i \subset A_0$ and $\widehat{B}_i \in \mathcal{F}$} \right. \right\}. \end{equation*}
The reader may check, as an exercise, that $\mathcal{G}$ is a fine cover of $E_0$. By \hyperref[lemma:vcl]{Vitali's Covering Lemma \ref{lemma:vcl}} we can find a disjoint subfamily $\mathcal{G}^\prime$ such that $\widehat{\mathcal{G}}^\prime$ covers $E_0$; hence, we set
\begin{equation*} \mathcal{F}^{\prime \prime} := \widehat{\mathcal{G}}^{\prime}. \end{equation*}
By construction $\mathcal{F}^{\prime\prime} \subset \mathcal{F}$, and it is also a covering for $E_0$ such that
\begin{equation*} \sum_{B \in \mathcal{G}^{\prime}} \mu\left(\widehat{B} \right) \leq C^3 \sum_{B \in \mathcal{G}^{\prime}} \mu(B) \leq C^3 \mu(A_0) \leq \epsilon, \end{equation*}
which is exactly what we wanted to prove.
\end{proof}

\begin{remark}The cover $\mathcal{F}$ need not be fine in the statement of \hyperref[lemma:meas0]{Lemma \ref{lemma:meas0}}, but, since this is the only result in this section whose assumptions can be weakened, we will just ignore it.\end{remark}

\begin{theorem} Let $X$ be a metric space, and let $\mu$ be a doubling locally finite measure. Let $E \subseteq X$ be a Borel subset, let $\epsilon > 0$ be a positive real number, and let $\mathcal{F}$ be a fine cover of $E$. Then there exists a subfamily $\mathcal{F}^\prime$, which covers $E$, such that
\begin{equation*} \sum_{B \in \mathcal{F}^\prime} \mu(B) \leq \mu(E) + \epsilon, \end{equation*}
that is we can cover $E$ completely with balls, whose measure does not exceed too much $\mu(E)$, by dropping the disjoint requirement. \end{theorem}

\begin{proof} First, we notice that the \hyperref[theorem:vct]{Vitali's Covering Theorem \ref{theorem:vct}} allows us to restrict out attention to null-set (i.e., $\mu(E) = 0$.) Then, we apply \hyperref[lemma:meas0]{Lemma \ref{lemma:meas0}} to conclude. \end{proof}

\section{Besicovitch Covering Theorem}

In this section, we investigate a different type of covering theorem, which is, actually, very similar to the Vitali's result.

The originality of the Besicovitch covering theorem is that it works without further assumption on the measure $\mu$, provided that $X$ is the Euclidean space $\R^n$.

\begin{lemma}[Besicovitch Covering Lemma] \index{Besicovitch Covering Lemma}\label{lemma:bcl} Let $X := \R^n$, and let
\begin{equation*} \mathcal{F} := \{ B(x_i, \, r_i)  \}_{i \in I} \end{equation*}
be a family of closed balls with uniformly bounded radii, which is also a fine cover of a Borel set $E \subset \R^n$. Then there exist a natural number $N := N(n)$, which depends only on the dimension $n$, and $\mathcal{F}_1, \, \dots, \, \mathcal{F}_{N + 1} \subset \mathcal{F}$ disjoint subfamilies such that
\begin{equation*} \bigcup_{i=1}^{N + 1} \mathcal{F}_i \supseteq E. \end{equation*} \end{lemma}

Before we work out the details of the proof of this result, we need to state a technical lemma that allows us to infer easily that the natural number $N(n)$ depends solely on the dimension of the ambient space.

\begin{lemma}\label{lemma:limballs} There exists a constant $N := N(n) \in \N$, depending only on the dimension of the ambient space, such that for any finite collection of closed balls $B_0, \, \dots, \, B_p \subset \R^n$ satisfying\mbox{}
\begin{enumerate}[label=\textbf{(\alph*)}]
\item $B_0 \cap B_i \neq \varnothing$ for every $i = 1, \, \dots, \, p$;
\item the $0$th ball has the smaller radius, that is $r_0 \leq r_i$, for any $i = 1, \, \dots, \, p$;
\item the center $x_i$ of the ball $B_i$ does not belong to the ball $B_j$ for every $i \neq j \in \{1, \, \dots, \, p\}$;
\end{enumerate}
it turns out that $p \leq N(n)$. Moreover, the conclusion does not change if we replace the assumption \textbf{(b)} with a slightly different one, that is
\begin{enumerate}
\item[\textbf{(b)}$^\prime$] $r_0 \leq 2 \, r_i$, for any $i = 1, \, \dots, \, p$.
\end{enumerate}
\end{lemma}

\begin{proof} See \cite[pp. 99--101]{git}. \end{proof}

\begin{proof}[Proof of Lemma \ref{lemma:bcl}] We first prove the result by assuming some additional properties, and then we reduce the general case to it with a simple trick.

\paragraph{Step 1.} Assume that the family $\mathcal{F} = \{ B(x_n, \, r_n) \}_{n \in \N}$ is a countable fine cover of $E$, and assume also that the sequence of the radii is weakly decreasing, i.e. $r_1 \geq r_2 \geq \dots$. To construct the disjoint subfamilies, we proceed as follows: \mbox{}
\begin{enumerate}[label=\textbf{(\arabic*)}]
\item Select any index $j \in \{1, \, \dots, \, N + 1\}$ and add the ball $B(x_0, \, r_0)$ to the subfamily $\mathcal{F}_j$.
\item Assume that $n - 1$ balls have already been placed in $\mathcal{F}_1, \, \dots, \, \mathcal{F}_{N+1}$, or thrown away. Let $\{i_1, \, \dots, \, i_k\} \subset \{1, \, \dots, \, n-1\}$ be the subset of the indices of the kept balls.
\item We consider the $n$-th ball, and we notice that there are only two possible outcomes:
\mbox{}
\begin{enumerate}[label=\textbf{(\roman*)}]
\item If $x_n$ belongs to $\bigcup_{j = 1}^k B_{i_j}$, then we throw the ball $B_n$ away.
\item If $x_n$ does not belong to $\bigcup_{j = 1}^k B_{i_j}$, then there is an index $i \in \{1, \, \dots, \, N+1\}$ such that $B_n$ can be added to $\mathcal{F}_i$, and that subfamily is still disjoint.

We argue by contradiction. If such an index $i$ does not exist, then, for any $j \in \{1, \, \dots, \, N + 1\}$, there exists a ball $B_j \in \mathcal{F}_j$ such that $B_j \cap B_n \neq \varnothing$. By \hyperref[lemma:limballs]{Lemma \ref{lemma:limballs}} we immediately derive the contradiction $N + 1 \leq N$, which is exactly what we needed to conclude the first part of the proof.
\end{enumerate}
\end{enumerate}

\paragraph{Step 2.} In the general case, we cannot assume that the family is countable or that the radii give an increasing sequence, and thus everything depends on the boundedness assumption. More precisely, there exists a real number $R > 0$ such that
\begin{equation} \label{dksa0e2iomkmd023} B(x_i, \, r_i) \in \mathcal{F} \implies r_i \leq R.\end{equation}
Let us put a well order on $\mathcal{F} := \{ B_\alpha \}_{\alpha < \omega}$ for some ordinal $\omega$ satisfying the additional property
\begin{equation*} \alpha^\prime < \alpha \implies r_\alpha < 2 \, r_{\alpha^\prime},\end{equation*}
which is always possible as a result of \eqref{dksa0e2iomkmd023}. The construction is no different than the previous one, but we do need to use the transfinite induction. In particular, we proceed as follows:
\begin{enumerate}[label=\textbf{(\roman*)}]
\item Select any index $j \in \{1, \, \dots, \, N+1\}$ and add the ball $B(x_0, \, r_0)$ to the subfamily $\mathcal{F}_j$.
\item Assume that $\alpha^\prime$ balls have already been placed in $\mathcal{F}_1, \, \dots, \, \mathcal{F}_{N + 1}$,  or thrown away. Let $\tau \subset \alpha^\prime$ be the subset of the indices of the kept balls.
\item Let $\alpha > \alpha^\prime$. There are two possible outcomes: \mbox{}
\begin{enumerate}[label=\textbf{(\roman*)}]
\item If $x_\alpha$ belongs to $\bigcup_{j \in \tau} B_j$, then we throw the ball $B_\alpha$ away.
\item If $x_\alpha$ does not belong to $\bigcup_{j \in \tau} B_j$, then there is an index $i \in \{1, \, \dots, \, N + 1\}$ such that $B_\alpha$ can be added to $\mathcal{F}_i$, and that subfamily is still disjoint.
\end{enumerate}
\end{enumerate}
\end{proof}

\begin{lemma} \label{lemma:measind2} Let $X = \R^n$, and let $\mu$ be a locally finite measure. Let $E \subseteq X$ be a Borel subset, and let $\mathcal{F}$ be a fine cover of $E$. Then there exist a finite disjoint subfamily $\mathcal{F}^{\prime \prime}$ and a positive real number $\delta > 0$ such that 
\begin{equation*}\mu \left(E \setminus \bigcup \mathcal{F}^{\prime\prime} \right) \leq \delta \cdot \mu(E). \end{equation*}\end{lemma}

\begin{proof}By \hyperref[lemma:bcl]{Besicovitch Covering Lemma \ref{lemma:bcl}} there are $N := N(n)$ disjoint subfamilies $\mathcal{F}_1, \, \dots, \, \mathcal{F}_N \subset \mathcal{F}$, whose union covers $E$, that is
\begin{equation*} E \subset \bigcup_{i = 1}^{N} \mathcal{F}_i . \end{equation*}
For every $i \in \{1, \, \dots, \, N\}$ we set
\begin{equation*} E_i := E \setminus \bigcup \mathcal{F}_i, \end{equation*}
and the reader may readily check that there exists an index $i \in \{1, \, \dots, \, N\}$ such that
\begin{equation*} \mu(E_i) \leq \frac{\mu(E)}{N}. \end{equation*}
The thesis follows immediately by taking to a proper finite subfamily of $\mathcal{F}_i$.\end{proof}

\begin{remark} It is not necessary for $\mathcal{F}$ to be a fine cover, but it is enough to have each point of $E$ be the center of such a ball. \end{remark}

\begin{lemma} \label{lemma:meas02} Let $X := \R^n$, and let $\mu$ be a locally finite measure. Let $E_0 \subset X$ be a Borel null-set, let $\epsilon > 0$ be any positive number, and let $\mathcal{F}$ be a fine cover of $E$. Then there exist a subfamily $\mathcal{F}^{\prime\prime}$, which covers $E_0$, such that
\begin{equation*} \sum_{B \in \mathcal{F}^{\prime\prime}} \mu(B) \leq \epsilon. \end{equation*}
\end{lemma}

\begin{proof} Let $A_0$ be an open neighborhood of $E_0$ satisfying the inequality
\begin{equation*} \mu(A_0) \leq \frac{\epsilon}{N}. \end{equation*}
Let us consider the family of balls
\begin{equation*} \mathcal{G} := \left\{ B_i := B(x_i, \, r_i) \: \left| \: B_i \subseteq A_0 \right. \right\}. \end{equation*}
The reader may check, as an exercise, that $\mathcal{G}$ is a fine cover of $E_0$. By \hyperref[lemma:bcl]{Besicovitch Covering Lemma \ref{lemma:bcl}} we can find disjoint subfamilies $\mathcal{G}_1, \, \dots, \, \mathcal{G}_N$ covering $E_0$. Set
\begin{equation*} \mathcal{F}^{\prime\prime} := \bigcup_{i = 1}^{N} \mathcal{G}_i, \end{equation*}
and notice that it is also a covering of $E_0$ satisfying the additional property
\begin{equation*} \sum_{B \in \mathcal{F}^{\prime \prime}} \mu(B) = \sum_{i = 1}^{N} \left[ \sum_{B \in \mathcal{G}_i} \mu(B) \right] \leq N \cdot \mu(A_0) \leq \epsilon.\end{equation*}
Indeed, the sum of the measures of the balls inside any subfamily $\mathcal{G}_i$ (as a consequence of the disjointness) needs to be equal or less than the measure of $A_0$.
\end{proof}

\begin{theorem}[Besicovitch Covering Theorem] \index{Besicovitch Covering Theorem} \label{theorem:bct} Let $X := \R^n$ and let $\mu$ be a locally finite measure. Let $E \subseteq X$ be a Borel subset, let $\epsilon > 0$ be any positive real number and let $\mathcal{F}$ be a fine cover of $E$. Then there exists a disjoint subfamily $\mathcal{F}^\prime$, which covers $\mu$-almost all of $E$, satisfying the inequality
\begin{equation*} \sum_{B \in \mathcal{F}^\prime} \mu(B) \leq \mu(E) + \epsilon. \end{equation*} \end{theorem}

\begin{proof}This is a direct consequence\footnote{The argument is very similar to the one used in Vitali's lemma. The reader may try to fill-in the details as an exercise.} of the \hyperref[lemma:bcl]{Besicovitch Covering Lemma \ref{lemma:bcl}} and \hyperref[lemma:measind2]{Lemma \ref{lemma:measind2}}. \end{proof}
\chapter{Density of Measures}

In this brief chapter, we investigate the notion of \textit{density} associated with a measure $\mu$. In particular, we first study the upper density related to the $d$-dimensional Hausdorff measure, and then we show that some of its essential properties may be generalized to other measures satisfying certain assumptions.

\section{Density of Doubling Locally Finite Measures}

In this section, we prove that the density of a doubling\footnote{This assumption is necessary if $X$ is a general metric space, but we can drop it if $X = \R^n$.} locally finite measure $\mu$ is a well-defined quantity.

\begin{definition}[Density] \index{measure!density} Let $E \subseteq X$ be a Borel subset of a metric space. The \textit{density} of a measure $\mu$ of the set $E$ is given, at the point $x \in X$, by the following limit:
\begin{equation*} \Theta(\mu, \, E, \, x) := \lim_{r \to 0} \frac{\mu \left( E \cap B(x, \, r) \right)}{\mu \left( B(x, \, r) \right)} . \end{equation*}
\end{definition}

\begin{theorem}Let $\mu$ be a (doubling) locally finite measure defined on a metric space $X$, and let $E \subseteq X$ be a Borel subset. Then the density is either $1$ or $0$, that is,
\begin{equation*} \Theta(\mu, \, E, \, x) := \lim_{r \to 0} \frac{\mu \left( E \cap B(x, \, r) \right)}{\mu \left( B(x, \, r) \right)} = \begin{cases} 1 & \text{for $\mu$-almost every $x \in E$}, \\[0.4em] 0 & \text{for $\mu$-almost every $x \notin E$}. \end{cases} \end{equation*} \end{theorem}

\begin{remark}The theorem holds true even if $\mu$ is not a doubling measure. Indeed, one can require $\mu$ to be  \textit{asymptotically} doubling\index{measure!asymptotically doubling}, that is
\begin{equation*} \lim_{r \to 0} \frac{\mu \left(  B(x, \, 2r) \right)}{\mu \left( B(x, \, r) \right)} < + \infty \end{equation*}for $\mu$-almost every $x \in X$.\end{remark}

\begin{proof}Here we only prove that the density is $1$ for $\mu$-almost every $x \in E$. The reader may either prove the other case with a similar argument or obtain it automatically by taking the complement.

\paragraph{Step 1.} Fix $\lambda \in (0, \, 1)$, and let us consider the set
\begin{equation*} E_\lambda := \left\{ x \in E \: \left| \: \liminf_{r \to 0} \frac{\mu \left( E \cap B(x, \, r) \right)}{\mu \left( B(x, \, r) \right)} < \lambda \right. \right\}. \end{equation*}
It is easy to see that we may equivalently prove that measure\footnote{Here, we would need to check carefully that the set $E_\lambda$ is actually Borel! The intuitive idea behind this goes as follows: The density ratio is increasing with respect to $r$, and thus it is a Borel function. The inferior limit for $r \to 0$, on the other hand, can be computed using rational sequences for the radii; thus it is Borel, and so is the set $E_\lambda$. The reader may try to fill in the details here as an exercise.} of $E_\lambda$ is zero for any $\lambda$, that is
\begin{equation*} \mu(E_\lambda) = 0, \qquad \forall \, \lambda \in (0, \, 1). \end{equation*}

\paragraph{Step 2.} Let $\mathcal{F}$ be the family of all the closed balls $B(x, \, r)$ with center $x \in E_\lambda$ and radius satisfying the inequality
\begin{equation*} \frac{\mu \left( E \cap B(x, \, r) \right)}{\mu \left( B(x, \, r) \right)} \leq \lambda, \end{equation*}
which is possible because the inferior limit of the ratio is estimated by $\lambda$ in $E_\lambda$. Then $\mathcal{F}$ is a fine cover of $E_\lambda$, and thus by \hyperref[theorem:vct]{Vitali's Covering Theorem \ref{theorem:vct}} there exists a disjoint subfamily $\mathcal{F}^\prime$, covering $\mu$-almost all of $E_\lambda$, such that
\begin{equation*} \sum_{B \in \mathcal{F}^\prime} \mu(B) \leq \mu(E_\lambda) + \epsilon. \end{equation*}
An easy estimate for the measure of $E_\lambda$ follows:
\begin{equation*} \mu(E_\lambda) \leq \sum_{B \in \mathcal{F}^\prime} \mu\left(E \cap B \right) \leq \lambda \sum_{B \in \mathcal{F}^\prime} \mu\left( B \right) \leq \lambda (\mu(E_\lambda) + \epsilon). \end{equation*}
In conclusion, if we take the limit as $\epsilon \to 0^+$, then we obtain the estimate
\begin{equation*} \mu(E_\lambda) \leq \lambda \mu(E_\lambda), \end{equation*}
and this the sought contradiction since we assumed $\lambda$ to be strictly less than $1$.
\end{proof}

\section{Upper Density of the Hausdorff Measure}

In this section, we focus our efforts on the upper density of the $d$-dimensional Hausdorff measure since we can find a lower bound and an upper bound explicitly.

\begin{definition}[Upper Density] \index{Hausdorff measure!upper density} Let $E$ be a Borel subset of a metric space $X$. The \textit{upper $d$-dimensional density} (with respect to $\mathcal{H}^d$) of $E$ at the point $x$ is defined by setting
\begin{equation} \Theta_d^\ast(E, \, x) := \limsup_{r \to 0^+} \frac{ \mathcal{H}^d \left(E \cap B(x, \, r)\right)}{\alpha_d r^d}, \label{eq:density} \end{equation}
where $\alpha_d 2^{-d}$ is the renormalization constant introduced the definition of the Hausdorff measure, and $B(x, \, r)$ denotes the closed ball of center $x$ and radius $r$. \end{definition}

\begin{theorem} \label{theorem:densityhasudd}Let $E$ be a Borel subset of a metric space $X$, and assume that $E$ is locally $\mathcal{H}^d$-finite. Then the following properties hold true: \mbox{}
\begin{enumerate}[label=\textbf{(\alph*)}]
\item The upper density of external points (outer density) is almost everywhere zero, that is,
\begin{equation*} \Theta_d^\ast(E, \, x) = 0, \qquad \text{for $\mathcal{H}^d$-almost every $x \notin E$.} \end{equation*}
\item The upper density is bounded from below, that is,
\begin{equation*} \frac{1}{2^d} \leq \Theta_d^\ast(E, \, x), \qquad \text{for $\mathcal{H}^d$-almost every $x \in E$.} \end{equation*}
\item If $X \cong \R^n$, then the upper density is bounded from above by $1$, that is,
\begin{equation*}\Theta_d^\ast(E, \, x) \leq 1, \qquad \text{for $\mathcal{H}^d$-almost every $x \in E$.} \end{equation*}
\item If $X$ is a generic metric space, then the upper density is bounded from above by a slightly different constant, that is,
\begin{equation*}\Theta_d^\ast(E, \, x) \leq 5^d, \qquad \text{for $\mathcal{H}^d$-almost every $x \in E$.} \end{equation*}
\end{enumerate}\end{theorem}

\begin{remark} The reason we are only concerned with upper density is the following: There exists $E \subseteq \R$ with finite and positive Hausdorff measure such that the lower density is $0$ for $\mathcal{H}^d$-almost every $x \in E$.

On the other hand, there is a dual concept of the Hausdorff dimension, called the \textbf{packing dimension}. Similar statements hold for this kind of measure, but the lower density replaces the upper one (see, e.g., \cite[pp. 81--86]{manilla}.)  \end{remark}

\begin{proof}\mbox{}
\begin{enumerate}[label=\textbf{(\alph*)}]
\item Fix $m > 0$ and let us consider the set
\begin{equation*} E_m := \left\{ x \notin E \: \left| \: \Theta_d^\ast(E, \, x) > m \right. \right\}. \end{equation*}
The thesis follows easily if we can prove that the $d$-dimensional Hausdorff measure of $E_m$ is equal to zero for any fixed $m$.

Let $\mu := \mathcal{H}^d \restr E$ the restriction to $E$ of the $d$-dimensional Hausdorff measure, and notice that $\mu$ is locally finite by construction. Let $A$ be an open neighborhood of $E_m$, and let us consider the family of balls
\begin{equation*} \mathcal{F} = \left\{ \overline{B(x, \, r)} \: \left| \: \text{$\overline{B(x, \, r)} \subset A$, $x \in E_m$ and $\mathcal{H}^d\left(E \cap \overline{B(x, \, r)} \right) > m \cdot \alpha_d r^d$ } \right. \right\}. \end{equation*}
The reader may easily prove that $\mathcal{F}$ is a fine cover of $E_m$. Thus, by \hyperref[lemma:vcl]{Vitali's Covering Lemma \ref{lemma:vcl}} it follows that there exists a disjoint subfamily $\mathcal{F}^\prime \subset \mathcal{F}$ such that $\widehat{\mathcal{F}}^\prime$ is also a covering of $E_m$. A straightforward computation proves that
\begin{equation*} \begin{aligned} \mu(A) & \geq \sum_{B \in \mathcal{F}^\prime} \mu(B) \geq m \cdot \alpha_d \sum_{B \in \mathcal{F}^\prime} r(B)^d = \\[1em] & = \frac{m}{5^d} \frac{\alpha_d}{2^d} \sum_{B \in \mathcal{F}^\prime}\left( \mathrm{diam}(\widehat{B}) \right)^d \geq \\[1em] & \geq \frac{m}{5^d} \mathcal{H}_\infty^d(E_m), \end{aligned} \end{equation*}
since $E_m \subseteq \cup_{B \in \mathcal{F}^\prime} \widehat{B}$. By definition, the set $E_m$ is disjoint from $E$, and thus $\mu(E_m) = 0$ (as $\mu$ is the restriction to $E$), and, by regularity (of the measure $\mu$), we can always choose $A$ in such a way that $\mu(A)$ is as small as we want. It turns out that
\begin{equation*} \epsilon \geq  \frac{m}{5^d} \mathcal{H}_\infty^d(E_m), \quad \forall \, \epsilon > 0 \implies \mathcal{H}^d(E_m) = 0  \end{equation*}
since $\mathcal{H}^d(E) = 0$ is equivalent to $\mathcal{H}_\infty^d(E) = 0$.
\item We divide the proof into three steps: we prove it in a very particular case, and then we reduce it to the general case using a simple trick.

Fix $m < 2^{-d}$ and let us consider the set
\begin{equation*} E_m := \left\{ x \in E \: \left| \: \Theta_d^\ast(E, \, x) < m \right. \right\}. \end{equation*}
Let $\mu := \mathcal{H}^d \restr E$ the restriction to $E$ of the $d$-dimensional Hausdorff measure, and notice that $\mu$ is locally finite by construction. By assumption, for every $x \in E_m$ there is a real number $r_0(x) := r_0 > 0$ such that
\begin{equation} \label{eq:sosakoowe22d} \mathcal{H}^d \left( E \cap \overline{B(x, \, r_0)} \right) < m \cdot \alpha_d r_0^d. \end{equation}

\paragraph{Step 1.} Assume that\footnote{We will get rid of this assumption in the next step using a simple trick.} the inequality above holds for every $r > 0$, uniformly with respect to $x$. Let $\mathcal{F} := \left\{ F_i \: \left| \: i \in I \right. \right\}$ be a covering of $E_m$ made up of balls, and consider the family of balls $\mathcal{F}^\prime := \{B_i \: \left| \: i \in I \right. \}$ defined in the following way: \mbox{}
\begin{enumerate}[label=\textbf{\arabic*)}]
\item The radius of $B_i$ is equal to the radius of $F_i$ for any $i \in I$.
\item The center of $B_i$ is a point of $E_m$ for any $i \in I$.
\item The collection of balls $\mathcal{F}^\prime$ covers $E_m$.
\end{enumerate}
We can easily estimate the Hausdorff measure of $E_m$ as follows:
\begin{equation*}2^d \mathcal{H}^d(E_m) \geq \alpha_d \sum_{F_i \in \mathcal{F}} \mu(F_i) \geq \alpha_d  \sum_{B_i \in I}\left( \mathrm{diam}(B_i) \right)^d \geq \frac{1}{m} \sum_{i \in I} \mathcal{H}^d \left(E \cap B_i \right), \end{equation*}
from which it follows that
\begin{equation*}2^d \mathcal{H}^d(E_m) \geq \frac{1}{m} \mathcal{H}^d(E_m) \implies \mathcal{H}^d(E_m) = 0\end{equation*}
since $m < 2^{-d}$ by assumption.

\paragraph{Step 2.} Assume now that the inequality \eqref{eq:sosakoowe22d} does not hold uniformly for every $r > 0$, but there is a uniform constant $r_0 > 0$ such that \eqref{eq:sosakoowe22d} holds true for every $x \in E$ and every $r \leq r_0$.

In a similar fashion, we can estimate the outer measure $\mathcal{H}_\delta^d$ for $\delta < r_0/2$ - up to an arbitrarily small error $\epsilon > 0$ -, that is, for every $\epsilon > 0$ we obtain the inequality
\begin{equation*}2^d \mathcal{H}^d(E_m)  + \epsilon \geq \frac{1}{m} \, \mathcal{H}_\delta^d(E_m),\end{equation*}
which immediately implies
\begin{equation*} \mathcal{H}^d(E_m) = 0, \end{equation*}
since $m < 2^{-d}$ by assumption.

\paragraph{Step 3.} Finally, assume that the inequality \eqref{eq:sosakoowe22d} is not uniform with respect to $x$, that is, for any $x \in E_m$, it holds for $r_0 := r_0(x) > 0$. We consider the set
\begin{equation*} E_{m, \, r_0} := \left\{ x \in E \: \left| \: \mathcal{H}^d\left(E \cap \overline{B(x, \, r)} \right) < m \cdot \alpha_d r^d, \, \, \, \forall \, r \leq r_0 \right. \right\} \end{equation*}
and we prove, using the previous step, that this set has measure zero for any choice of $r_0 > 0$ and $m < 2^{-d}$.

\item We first give a rough idea of the proof of this assertion and then we formalize it correctly. 

\paragraph{Step 0.} Fix $m > 1$ and let us consider the set
\begin{equation*} E_m := \left\{ x \in E \: \left| \: \Theta_d^\ast(E, \, x) > m \right. \right\}. \end{equation*}
Let $\mu := \mathcal{H}^d \restr E$ the restriction to $E$ of the $d$-dimensional Hausdorff measure, and notice that $\mu$ is locally finite by construction. By assumption, there are infinitely many balls such that
\begin{equation*} \mu \left( \overline{B(x, \, r)} \right) = \mathcal{H}^d \left( \overline{B(x, \, r)} \cap E \right) > m \cdot \alpha_d r^d, \end{equation*}
and thus
\begin{equation*} \mu(E_m) \simeq \sum_i \mu(B_i) \geq m \cdot \frac{\alpha_d}{2^d} \sum_i \left(2 r_i \right)^d \geq m \cdot \mathcal{H}^d(E_m).\end{equation*}
In conclusion, since $\mu$ is the restriction of $\mathcal{H}^d$ to $E$, it suffices to notice that $E_m \subset E$ to infer that
\begin{equation*} \mathcal{H}^d(E_m) = \mu(E_m) \geq m \cdot \mathcal{H}^d(E_m), \end{equation*}
which is absurd since $m > 1$.

\paragraph{Step 1.} Fix $\delta > 0$ and let us consider the collection of balls
\begin{equation*}\mathcal{F} := \left\{ \overline{B(x, \, r)} \: \left| \: \text{$x \in E_m$, $r \leq \frac{\delta}{2}$ and $\mathcal{H}^d \left( \overline{B(x, \, r)} \cap E \right) > m \cdot \alpha_d r^d$} \right. \right\}. \end{equation*}
It is easy to prove that $\mathcal{F}$ is a fine cover of $E_m$. By \hyperref[lemma:bcl]{Besicovitch Covering Lemma \ref{lemma:bcl}}, for every $\epsilon > 0$ we can extract a subfamily $\mathcal{F}^\prime \subset \mathcal{F}$ such that $\mathcal{F}^\prime$ is a covering of $E_m$ and
\begin{equation*}\mu(E_m) + \epsilon \geq \sum_{B \in \mathcal{F}^\prime} \mu(B).\end{equation*}
A straightforward computation proves that
\begin{equation*} \begin{aligned} \mu(E_m) + \epsilon \geq & \sum_{B \in \mathcal{F}^\prime} \mu(B) \geq \\[1em] & \geq m \sum_{B \in \mathcal{F}^\prime}\alpha_d r(B)^d = \\[1em] & = m \cdot \frac{\alpha_d}{2^d} \sum_{B \in \mathcal{F}^\prime} (2 r(B))^d \geq \\[1em] & \geq m \mathcal{H}_\delta^d(E_m). \end{aligned}\end{equation*}
Therefore, if we take the limit as $\delta$ and $\epsilon$ goes to $0^+$, then it turns out that
\begin{equation*}\mathcal{H}^d(E_m) + \epsilon = \mu(E_m) + \epsilon \geq  m \, \mathcal{H}_\delta^d(E_m) \implies \mathcal{H}^d(E_m) \geq m \, \mathcal{H}^d(E_m),\end{equation*}
which is absurd since $m > 1$.

\item Let $\mu:= \mathcal{H}^d \restr E$ the restriction to $E$ of the $d$-dimensional Hausdorff measure, and notice that $\mu$ is locally finite by construction. Unfortunately, the measure $\mu$ does not have the doubling property (or the asymptotic one); therefore we need to rely on a different covering theorem.

Fix $m > 5^d$, fix $\delta > 0$, let $A$ be an open neighborhood of $E_m$, and let us consider the family of closed balls
\begin{equation*}\mathcal{F} := \left\{ \overline{B(x, \, r)} \subset A \: \left| \: \text{$x \in E_m$, $r \leq \delta$ and $\mathcal{H}^d \left( \overline{B(x, \, r)} \cap E \right) > m \cdot \alpha_d r^d$} \right. \right\}. \end{equation*}
The cover $\mathcal{F}$ is fine, and thus there exists a disjoint subfamily $\mathcal{F}^\prime \subset \mathcal{F}$ such that $\widehat{\mathcal{F}}^\prime$ covers $E_m$. Then
\begin{equation*} \begin{aligned} \mu(A) & \geq \sum_{B \in \mathcal{F}^\prime} \mu(B) \geq \\[1em] & \geq  m \sum_{B \in \mathcal{F}^\prime}\alpha_d \, r(B)^d = \\[1em] & =  \frac{m}{5^d}\, \frac{\alpha_d}{2^d} \sum_{B \in \mathcal{F}^\prime} \left( 2 r\left(\widehat{B} \right) \right)^d \geq \\[1em] & \geq \frac{m}{5^d} \mathcal{H}_{\delta^\prime}^d(E_m), \end{aligned}\end{equation*}
where $\delta^\prime = 10 \delta$. We conclude the proof by noticing that, since $\mu$ is a regular measure (both inner and outer regular), we have
\begin{equation*} \mathcal{H}^d(E_m) = \mu(E_m) = \inf_{A \supset E_m} \mu(A) \end{equation*}
and hence, by taking the limit as $\delta, \, \epsilon \to 0^+$, it turns out that
\begin{equation*}\mathcal{H}^d(E_m) \geq \frac{m}{5^d} \mathcal{H}_{\delta^\prime}^d(E_m) \implies \mathcal{H}^d(E_m) = 0.\end{equation*}
\end{enumerate} \end{proof}

\begin{remark} In the previous results, we used the covering theorems assuming either that $X = \R^n$ or $\mu$ doubling, but for the applications we need them to be true for a larger class of measures. More precisely, the two statements \mbox{}
\begin{enumerate}[label=\textbf{A\arabic*})]
\item If $E \subset X$ is a Borel subset and $\mathcal{F}$ is a fine cover of $E$, then for any $\epsilon > 0$ there exists a disjoint subfamily $\mathcal{F}^\prime \subset \mathcal{F}$, covering $E$ $\mu$-almost everywhere, such that
\begin{equation*} \sum_{B \in \mathcal{F}^\prime} \mu(B) \leq \mu(E) + \epsilon. \end{equation*}
\item If $E \subset X$ is a Borel subset and $\mathcal{F}$ is a fine cover of $E$, then for any $\epsilon > 0$ there exists a subfamily $\mathcal{F}^\prime \subset \mathcal{F}$, covering $E$, such that
\begin{equation*} \sum_{B \in \mathcal{F}^\prime} \mu(B) \leq \mu(E) + \epsilon.\end{equation*}
\end{enumerate}
hold true even for a measure $\mu = \mu^\prime\restr F$, where $\mu^\prime$ is doubling and $F$ is arbitrary (actually, it is enough to ask $\mu \ll \mu^\prime \restr F$.)
\end{remark}

\section{Upper $d$-Dimensional Density}

\paragraph{Support.}\index{measure!support} Let $\mu$ be a measure defined on a metric space $X$. The \textit{support} of $\mu$ is the smallest closed subset $F \subset X$ such that the complement $F^c$ is a null set, that is, 
\begin{equation*} \mathrm{spt}(\mu) = \inf \left\{ F \subset X \: \left| \: \text{$F$ is closed and $\mu(F^c) = 0$} \right. \right\}. \end{equation*}
The measure $\mu$ is \textit{supported} on a Borel set $E \subset X$ if the complement of $E$ is a null set for $\mu$. In particular, the set $E$ contains the support of $\mu$, that is,
\begin{equation*} \mathrm{spt}(\mu) \subseteq E. \end{equation*}

\begin{definition}\index{measure!orthogonal} Let $\mu$ and $\lambda$ be two measures defined on the same metric space $X$. We say that $\mu$ is \textit{orthogonal} to $\lambda$, and we denote it by $\mu \perp \lambda$, if and only if they are supported on disjoint Borel sets. \end{definition}

\begin{remark} The support of a measure $\mu$ is a well-defined notion, but it is not the optimal one concerning the orthogonality. Indeed, the reader may prove, as an exercise, that there exists a measure $\mu$ orthogonal to a measure $\lambda$ such that
\begin{equation*} \mathrm{spt}(\mu) \cap \mathrm{spt}(\lambda) \neq \varnothing . \end{equation*}
\textit{Hint.} Consider the Lebesgue measure and the Dirac measure on the real line. \end{remark}

\begin{lemma}\label{lemma:kdaso} Let $\lambda$ and $\mu$ be locally finite measures defined on a metric space $X$. Assume that $\lambda$ is orthogonal to $\mu$ and $\lambda$ satisfies the assumption \textbf{A2)}. Then the Radon-Nikodym derivative is zero, that is,
\begin{equation*} \frac{\mathrm{d} \lambda}{\mathrm{d} \mu}(x) := \lim_{r \to 0} \frac{\lambda \left( \overline{B(x, \, r)} \right)}{\mu \left( \overline{B(x, \, r)} \right)} = 0, \end{equation*}
for $\mu$-almost every $x \in X$.  \end{lemma}

\begin{proof} Fix $m > 0$, let $F \subset X$ be a Borel set satisfying
\begin{equation*} \lambda\left( F^c \right) = 0 \qquad \text{and} \qquad \mu(F) = 0, \end{equation*}
and let
\begin{equation*}E_m := \left\{ x \notin F \: \left| \: \limsup_{r \to 0^+} \frac{\mathrm{d}\lambda}{\mathrm{d}\mu}(x) > m \right. \right\}. \end{equation*}
Since $F$ is a $\mu$-null set, we may equivalently prove that $\mu(E_m) = 0$ for every fixed $m > 0$. Consider the family of closed balls
\begin{equation*}\mathcal{F} := \left\{ \overline{B(x, \, r)} \: \left| \: \text{$x \in E_m$, $\lambda\left( \overline{B(x, \, r)} \right) \geq m \mu \left( \overline{B(x, \, r)} \right)$} \right. \right\}. \end{equation*}
The reader may check that $\mathcal{F}$ is a fine cover of $E_m$; hence, by assumption \textbf{A2)} it follows that for every $\epsilon > 0$ there exists a subfamily $\mathcal{F}^\prime \subset \mathcal{F}$ such that
\begin{equation*} E_m \subseteq \bigcup_{B \in \mathcal{F}^\prime} B \qquad \text{and} \qquad \sum_{B \in \mathcal{F}^\prime} \lambda(B) \leq \lambda(E_m) + \epsilon. \end{equation*}
The complement of $F$ is a $\lambda$-null set, and thus we finally infer that
\begin{equation*} \underbrace{\lambda(E_m)}_{=0} + \epsilon \geq m \sum_{B \in \mathcal{F}^\prime} \mu(B) \geq m  \mu(E_m) \implies \mu(E_m) = 0.\end{equation*}
\end{proof}

\begin{corollary}Let $\lambda$ and $\mu$ be locally finite measures defined on a metric space $X$. Assume that $\lambda$ is orthogonal to $\mu$ and $\lambda$ satisfies the assumption \textbf{A2)}. Then the Radon-Nikodym derivative is infinite, that is,
\begin{equation*} \frac{\mathrm{d} \mu}{\mathrm{d} \lambda}(x) = + \infty, \end{equation*}
for $\mu$-almost every $x \in X$. \end{corollary}

\begin{theorem} \label{theorem:lusdnas} Let $\mu$ be locally finite measures defined on a metric space $X$ satisfying the assumption \textbf{A2)}, and let $f \in L_{\mathrm{loc}}^p(X, \, \mu)$ be a function. Then
\begin{equation*} \dashint_{\overline{B(x, \, r)}} \left| f(y) - f(x) \right|^p \, \mathrm{d}\mu(y) \xrightarrow{r \to 0} 0, \end{equation*}
for $\mu$-almost every $x \in X$.  \end{theorem}

\begin{proof}Fix $\epsilon > 0$. By Lusin's Theorem\footnote{\textbf{Lusin's Theorem:} Let $(X, \, \Sigma, \, \mu)$ be a Radon measure space and let $Y$ be a second-countable topological space. If $f : X \to Y$ is a measurable function and $\epsilon > 0$, then for any $A \in \Sigma$ there is a closed set $E$ with $\mu(A \setminus E) < \epsilon$ such that $f \, \big|_{E}$ is continuous.} we can always find a continuous function $\widetilde{f} : X \longrightarrow \R$ and a subset $E \subset X$ such that
\begin{equation*} f \, \big|_E = \widetilde{f} \, \big|_E \qquad \text{and} \qquad \mu\left(X \setminus E \right) < \epsilon. \end{equation*}
The complement of $E$ can be taken arbitrarily small; thus it suffices to prove the theorem for $\mu$-almost every $x \in E$. In particular, given $x \in E$ it turns out that
\begin{equation*} \begin{aligned} \dashint_{\overline{B(x, \, r)}} \left| f(y) - f(x) \right|^p \, \mathrm{d}\mu(y) & =  \dashint_{\overline{B(x, \, r)} \cap E}\left| f(y) - f(x) \right|^p \, \mathrm{d}\mu(y) + \dashint_{\overline{B(x, \, r)} \setminus E}\left| f(y) - f(x) \right|^p \, \mathrm{d}\mu(y) =
\\[1em] & = \frac{1}{\mu\left(\overline{B_r} \right)} \left[ \int_{\overline{B_r} \cap E}\left| \widetilde{f}(y) - \widetilde{f}(x) \right|^p \, \mathrm{d}\mu(y) + \int_{\overline{B_r} \setminus E}\left| f(y) - f(x) \right|^p \, \mathrm{d}\mu(y) \right] \stackrel{(\ast)}{\leq}
\\[1em] & \stackrel{(\ast)}{\leq} \left[ \omega\left(\widetilde{f} \right) \right]^p + \frac{2^{p-1}}{\mu\left(\overline{B_r}\right)} \int_{\overline{B(x, \, r)} \setminus E} \left[ \left| f(x) \right|^p + \left| f(y) \right|^p \right] \, \mathrm{d}\mu(y),
\end{aligned}\end{equation*}
where $\omega \left( \widetilde{f} \right)$ denotes the oscillation of $\widetilde{f}$, and the inequality $(\ast)$ follows from the following trivial fact: If $p \geq 1$, then for any real numbers $a$ and $b$ it turns out that
\begin{equation*}|a - b|^p \leq 2^{p-1} (|a|^p + |b|^p). \end{equation*}
Let us consider the measures
\begin{equation*} \lambda_1 := \left[ \left| f \right|^p \, \mathbbm{1}_{X \setminus E} \right] \cdot \mu \qquad \text{and} \qquad \lambda_2 := \mathbbm{1}_{X \setminus E} \cdot \mu, \end{equation*}
and let $\widetilde{\mu}$ be the restriction of $\mu$ to the set $E$. Then
\begin{equation*} \dashint_{\overline{B(x, \, r)}} \left| f(y) - f(x) \right|^p \, \mathrm{d}\mu(y) \leq \left[ \omega\left(\widetilde{f} \right) \right]^p + 2^{p-1} \frac{\lambda_1\left(\overline{B(x, \, r)} \right)}{\widetilde{\mu}\left(\overline{B(x, \, r)} \right)} + 2^{p-1} \left| f(x) \right|^p \frac{\lambda_2\left(\overline{B(x, \, r)}\right)}{\widetilde{\mu}\left(\overline{B(x, \, r)}\right)},\end{equation*}
from which it easily follows that: \mbox{}
\begin{enumerate}[label=\textbf{\arabic*})]
\item The oscillation $\omega \left( \widetilde{f} \right)$ goes to $0$ as $r \to 0^+$ for \textbf{every} $x \in E$, as a consequence of the continuity of $\widetilde{f}$.
\item The measures $\lambda_1$ and $\lambda_2$ are both orthogonal to $\widetilde{\mu}$; hence both the ratios go to 0 as $r \to 0^+$, for $\widetilde{\mu}$-almost every $x \in X$, which means for $\mu$-almost every $x$ in $E$.
\end{enumerate} \end{proof}

\begin{definition}\index{approximately continuous function} The function $f: X \longrightarrow \R$ is \textit{$L^p$-approximately continuous} if and only if
\begin{equation*} \dashint_{\overline{B(x, \, r)}} \left| f(y) - f(x) \right|^p \, \mathrm{d}\mu(y) \xrightarrow{r \to 0^+} 0 \end{equation*}
for $\mu$-almost every $x \in X$.\end{definition}

\begin{corollary} \label{cor:lploc}Let $\mu$ be locally finite measures defined on a metric space $X$ satisfying the assumption \textbf{A2)}, and let $f \in L_{\mathrm{loc}}^p(X, \, \mu)$ be a function. Then
\begin{equation*} \dashint_{\overline{B(x, \, r)}} f(y) \, \mathrm{d}\mu(y) \xrightarrow{r \to 0^+} f(x), \end{equation*}
for $\mu$-almost every $x \in X$. \end{corollary}

\begin{proof}By Jensen's inequality\footnote{Let $(\Omega, \, \mathcal{G}, \, \mu)$ be a probability space. If $g : \Omega \longrightarrow \R$ is a $\mu$-summable function, and if $\varphi : \R \longrightarrow \R$ is a convex function, then
\begin{equation*}\varphi \left( \int_\Omega g(x) \, \mathrm{d}\mu(x) \right) \leq \int_\Omega \varphi \circ g(x) \, \mathrm{d}\mu(x).\end{equation*}  }, it turns out that
\begin{equation*} L_{\mathrm{loc}}^q(\Omega) \subset L_{\mathrm{loc}}^p(\Omega), \qquad \forall \, q < p, \end{equation*}
and thus we can apply \hyperref[theorem:lusdnas]{Theorem \ref{theorem:lusdnas}} with $p = 1$. \end{proof}

\begin{theorem} Let $\mu$ and $\lambda$ be locally finite measures satisfying the assumption \textbf{A2)}. Assume that there is a decomposition
\begin{equation*} \lambda = f \cdot \mu + \lambda_s, \end{equation*}
where $\lambda_s$ denotes the singular part of the measure $\lambda$. Then the density of $\lambda$ with respect to $\mu$ can be computed pointwise, and it is equal to
\begin{equation*} \frac{\mathrm{d} \lambda}{\mathrm{d} \mu}(x) = f(x), \end{equation*}
for $\mu$-almost every $x \in X$.  \end{theorem}

\begin{proof} By \hyperref[lemma:kdaso]{Lemma \ref{lemma:kdaso}} it turns out that
\begin{equation*} \frac{ \mathrm{d} \lambda_s}{\mathrm{d}\mu}(x) = 0 \end{equation*}
for $\mu$-almost every $x \in X$. On the other hand, by \hyperref[theorem:lusdnas]{Theorem \ref{theorem:lusdnas}} it follows that
\begin{equation*} \frac{ \mathrm{d} \lambda_{ac}}{\mathrm{d}\mu}(x) = f(x), \end{equation*}
where $\lambda_{ac}$ denotes the part of $\lambda$ which is absolutely continuous with respect to $\mu$, that is,
\begin{equation*} \lambda_{ac} = f \cdot \mu. \end{equation*}
\end{proof}

\paragraph{Upper Density.} Let $\mu$ be a locally finite measure defined on a metric space $X$, and let $d > 0$ be a positive real number. The \textit{$d$-dimensional upper density} at the point $x$ with respect to $\mu$ is defined by setting
\begin{equation} \Theta_d^\ast(\mu, \, x) := \limsup_{r \to 0^+} \frac{\mu \left(\overline{B(x, \, r)}\right)}{\alpha_d  r^d}, \label{eq:density2} \end{equation}
where $\frac{\alpha_d}{2^d}$ is the renormalization constant used in the definition of the Hausdorff measure.

\begin{theorem} \label{theorem:densidut} Let $\mu$ be a locally finite measure defined on a metric space $X$, and let $d > 0$ be a positive real number. The following properties are equivalent: \mbox{}
\begin{enumerate}[label=\textbf{(\arabic*)}]
\item The upper density is finite and nonzero, that is,
\begin{equation*} 0 < \Theta_d^\ast(\mu, \, x) < + \infty, \end{equation*}
for $\mu$-almost every $x \in X$.
\item There is a locally summable function $f \in L_{\mathrm{loc}}^1 \left( X, \, \mathcal{H}^d \right)$ such that
\begin{equation*}\mu = f \cdot \mathcal{H}^d. \end{equation*}
\end{enumerate} \end{theorem}

The proof of this theorem is rather involved. Hence, we split it into different propositions and lemmas and, at the end of the section, we get much more than the statement above.

%\begin{proof} The proof of the implication \textbf{(2)} $\implies$ \textbf{(1)} is made up of three steps.

%\vspace{2.5mm}
%\textbf{Step 1.} Suppose that $f = \mathbbm{1}_E$ is the characteristic function of a set. Then the thesis follows easily from Theorem \ref{theorem:densityhasudd}.

%\vspace{2.5mm}
%\textbf{Step 2.} Suppose that there exists $E \subset X$ of arbitrarily large measure such that $0 < m \leq f(x) \leq M < + \infty$ for $\mu$-almost any $x \in E$. Then it turns out that
%\begin{equation*} 0 < m \cdot \Theta_d^\ast(\mathcal{H}^d \, \big|_E, \, x) \leq \Theta_d^\ast(\mu, \, x) \leq M \cdot \Theta_d^\ast(\mathcal{H}^d \, \big|_E, \, x) < + \infty,\end{equation*}
%since both terms are finite for $\mu$-almost every $x \in E$ by Step 1.

%\vspace{2.5mm}
%\textbf{Step 3.} Suppose that $f$ is any locally summable function. For any natural number $n \in \N$ there exists $E_n$ such that 
%\begin{equation*}0 < \frac{1}{n} \leq f_1(x) \leq n < + \infty, \end{equation*}
%for $\mu$-almost every $x \in E_n$, where $f(x) = f_1(x) + f_2(x)$ and $f_2$ is small with respect to the $\mathcal{H}^d$ measure. Notice also that $(E_n)$ is an increasing sequence of sets and
%\begin{equation*}\bigcup_{n \in \N} E_n = X.\end{equation*}
%By assumption $\mu = f_1 \cdot \mathcal{H}^d + f_2 \cdot \mathcal{H}^d$, therefore
%\begin{equation*} \Theta_d^\ast(\mu, \, x) = \Theta_d^\ast(f_1 \cdot \mathcal{H}^d, \, x) + \Theta_d^\ast(f_2 \cdot \mathcal{H}^d, \, x)\end{equation*}
%and it is clearly enough to prove that it is finite for $\mu_1$-almost everywhere and, at the same time, that the second term is zero for $\mu_1$-almost every $x$. But it's easy to see that
%\begin{equation*}  \Theta_d^\ast(f_2 \cdot \mathcal{H}^d, \, x) = \frac{\mu_2(\bar{B})}{\mu_1(\bar{B})} \cdot\frac{\mu_1(\bar{B})}{\alpha_d \, r^d}, \end{equation*}
%and the first factor goes to zero by orthogonality, while the second is finite by Step 2. This remark concludes the proof since the thesis holds for $\mu_1$-almost every $x \in E_n$, thus by taking the limit for $n \to + \infty$ it turns out that it holds for $\mu$-almost every $x$. \end{proof}

\begin{proposition} Let $\mu$ be a locally finite measure on $X$ and let $d > 0$. Suppose that there is $f \in L_{\mathrm{loc}}^1\left(\mathcal{H}^d \right)$ such that $\mu = f \cdot \mathcal{H}^d$. Then
\begin{equation*} 2^{-d} \, f(x) \leq \Theta_d^\ast(\mu, \, x) \leq 5^d \, f(x) \quad \text{for $\mathcal{H}^d$-almost every $x \in X$}. \end{equation*}
In particular the upper-density of $\mu$ is finite and nonzero ($0 < \Theta_d^\ast(\mu, \, x) < + \infty$) for $\mu$-almost any $x \in X$.\end{proposition}


\begin{proof} We first prove the result in some particular cases, and then we generalize it using a simple trick. 

\paragraph{Step 1.} Assume that $X = \R^n$ and assume that there are constants $m < M \in \R$ such that
\begin{equation*} 0 < m < f(x) \leq M < + \infty \quad \text{for $\mathcal{H}^d$-almost every $x \in X$ s.t. $f(x)\neq0$}.\end{equation*}
Set $E := \left\{ x \in X \: \left| \: f(x) \neq 0 \right. \right\}$ and consider the measure $\lambda := \mathbbm{1}_{E} \cdot \mathcal{H}^d$. Clearly $\lambda$ is locally finite, and thus we can easily compute the density ratio, that is,
\begin{equation*} \begin{aligned} \frac{\mu \left(\overline{B(x, \, r)}\right)}{\alpha_d r^d} & = \frac{1}{\alpha_d r^d} \int_{\overline{B(x, \, r)}} f(y) \, \mathrm{d}\mathcal{H}^d(y) = \\[1em] & = \frac{1}{\alpha_d r^d} \int_{\overline{B(x, \, r)}} f(y) \, \mathrm{d}\lambda(y) = \\[1em] & = \frac{\lambda \left( \overline{B(x, \, r)} \right)}{\alpha_d \, r^d} \dashint_{\overline{B(x, \, r)}} f(y) \, \mathrm{d}\lambda(y).  \end{aligned} \end{equation*}
Since $\lambda$ is equal to the restriction to $E$ of the $d$-dimensional Hausdorff measure, it follows from \hyperref[theorem:densityhasudd]{Theorem \ref{theorem:densityhasudd}} that
\begin{equation*}\limsup_{r \to 0} \frac{\lambda \left( \overline{B(x, \, r)} \right)}{\alpha_d r^d}  \in \left[\frac{1}{2^d}, \, 5^d \right] \end{equation*}
for $\mathcal{H}^d$-almost every $x \in E$. On the other hand, the Lebesgue result (see \hyperref[cor:lploc]{Corollary \ref{cor:lploc}}) implies that
\begin{equation*} \dashint_{\overline{B(x, \, r)}} f(y) \, \mathrm{d}\lambda(y) \xrightarrow{r \to 0} f(x), \end{equation*}
for $\mathcal{H}^d$-almost every $x \in E$, which is exactly what we wanted to prove.

\paragraph{Step 2.} Let $X$ be any reasonable\footnote{Here we do not specify what kind of assumptions are needed on $X$, but any space that "looks like" $\R^n$ will do. The reader may try, as an exercise, to find the minimal assumptions that can make the argument of the second step work.} space and assume that there are constants $m < M \in \R$ such that
\begin{equation*} 0 < m < f(x) \leq M < + \infty\end{equation*}
for $\mathcal{H}^d$-almost every $x \in X$ such that $f(x)\neq0$. Let $f_1$ and $f_2$ be finite sums of step functions in such a way that
\begin{equation*} f_1(x) \leq f(x) \leq f_2(x). \end{equation*}
We claim that
\begin{equation*} 2^{-d} f_1(x) \leq \Theta_d^\ast(\mu, \, x) \leq 5^d  f_2(x)\end{equation*}
for $\mathcal{H}^d$-almost every $x \in X$ such that $f(x)\neq0$. 

\paragraph{Step 2.1.} Let $\alpha_1, \, \dots, \, \alpha_m \in \R$ and $A_1, \, \dots, \, A_m \subset X$, and assume that
\begin{equation*} f_1(x) = \sum_{i = 1}^{m} \alpha_i  \mathbbm{1}_{E_i},\end{equation*}
where $E_i = E \cap A_i$ for every $i = 1, \, \dots, \, m$. The following inequality of measures is tirvial
\begin{equation*} \mu \geq \sum_{i = 1}^{m} \alpha_i \,  \mathbbm{1}_{E_i} \cdot \mathcal{H}^d,\end{equation*}
and hence 
\begin{equation*} \Theta_d^\ast(E_i, \, x) \geq \alpha_i 2^{-d} \stackrel{(x \in E_i)}{=} f_1(x) \, 2^{-d}\end{equation*}
for $\mathcal{H}^d$-almost every $x \in E_i$. The union of the $E_i$s is almost all of $E$, and thus it turns out that 
\begin{equation*} 2^{-d} f_1(x) \leq \Theta_d^\ast(\mu, \, x) \end{equation*}
for $\mathcal{H}^d$-almost every $x \in E$.

\paragraph{Step 2.2.} Let us consider the function
\begin{equation*} f_2(x) = \sum_{i = 1}^{m} \beta_i  \mathbbm{1}_{E_i},\end{equation*}
where the family $\{E_i := A_i \cap E \}_{i = 1, \, \dots, \, m}$ is disjoint. Since $E = \bigsqcup_{i = 1, \, \dots, \, m} E_i$, it turns out that $\Theta_d^\ast(\mu, \, x) = 0$ for $\mathcal{H}^d$-almost every $x \notin E$. If $x \in E_j$, then
\begin{equation*}  \Theta_d^\ast(\mu, \, x) \leq \sum_{i = 1}^m \beta_i \Theta_d^\ast \left(E_i, \, x \right) = \beta_j \Theta_d^\ast(E_j, \, x) \leq 5^d f_2(x)\end{equation*} 
for $\mathcal{H}^d$-almost every $x \in E_j$, as a consequence of \hyperref[theorem:densityhasudd]{Theorem \ref{theorem:densityhasudd}}.

\paragraph{Step 3.} Assume $f \in L_{\mathrm{loc}}^1\left(\R^n, \, \mathcal{H}^d \right)$. Fix $m > 1$, set
\begin{equation*} E_m := \left\{ x \in E \: \left| \: \frac{1}{m} \leq f(x) \leq m \right. \right\}, \end{equation*}
and let $\mu = \mu_m + \widetilde{\mu_m}$ be the Radon-Nikodym decomposition of $\mu$ associated to $E_m$. More precisely, we consider the decomposition
\begin{equation*} \mu_m = f \, \mathbbm{1}_{E_m} \cdot \mathcal{H}^d \qquad \text{and} \qquad \widetilde{\mu_m} = f \, \mathbbm{1}_{E \setminus E_m} \cdot \mathcal{H}^d, \end{equation*}
in such a way that
\begin{equation*} \frac{\mu \left( \overline{ B(x, \, r) } \right)}{\mu_m\left( \overline{ B(x, \, r) } \right)} \xrightarrow{r \to 0} 1 \quad \text{for $\mu_m$-almost every $x \in E$.} \end{equation*}
Since $\widetilde{\mu_m}$ is singular, it turns out that
\begin{equation*} \Theta_d^\ast(\mu, \, x) = \Theta_d^\ast(\mu_m, \, x) \leq 5^d f(x)\end{equation*}
for $\mu_m$-almost every $x \in E$, i.e., for $\mathcal{H}^d$-almost every $x \in E_m$
The union of the $E_m$s covers almost all of $E$; thus
\begin{equation*} \Theta_d^\ast(\mu, \, x) \leq 5^d f(x) \end{equation*}
for $\mu$-almost every $x \in E$. If, on the other hand, $x \notin E$, then we can prove\footnote{The reader may prove this assertion in a similar fashion to \hyperref[theorem:densityhasudd]{Theorem \ref{theorem:densityhasudd}}.} that the $d$-dimensional upper density is equal to $0$ for $\mathcal{H}^d$-almost every $x \notin E$.

\paragraph{Step 4.} Let $X$ be a reasonable metric space and let $f$ be a function in $L_{\mathrm{loc}}^1\left(X, \, \mathcal{H}^d \right)$. This step follows from the previous ones since for a locally summable function, for every $\epsilon > 0$, there are constants $m, \, M > 0$ and $E_{m, \, M} \subseteq X$ such that
\begin{equation*} 0 < m < f(x) \leq M < + \infty\end{equation*}
for $\mathcal{H}^d$-almost every $x \in E$ such that $f(x)\neq0$ and $\mathcal{H}^d(E_{m, \, M}) < \epsilon$.
\end{proof}

\begin{lemma} \label{lemma32} Let $\mu$ be a locally finite measure with finite $d$-dimensional upper density, that is, 
\begin{equation*}\Theta_d^\ast(\mu, \, x) < + \infty \end{equation*}
for $\mu$-almost every $x \in X$. Then $\mu$ is absolutely continuous with respect to the Hausdorff measure $\mathcal{H}^d$, that is,
\begin{equation*} \mathcal{H}^d(E) = 0 \implies \mu(E) = 0. \end{equation*} \end{lemma}

\begin{proof}[Sketch of the Proof] By assumption, if $r > 0$ is small enough, it turns out that
\begin{equation*} \frac{ \mu\left( \overline{B(x, \, r)} \right)}{\alpha_d r^d} \leq m. \end{equation*}
Let $E$ be a $\mathcal{H}^d$-null subset of $X$, that is $\mathcal{H}^d(E) = 0$. For any $m, \, \rho > 0$ let us consider the collection of sets
\begin{equation*} E_{m, \, \rho} = \left\{ x \in E \: \left| \: \text{$\frac{ \mu\left( \overline{B(x, \, r)} \right)}{\alpha_d  r^d} \leq m$ for every $r < \rho$} \right. \right\}. \end{equation*}
The countable union over the rational numbers covers almost every point of $E$, that is,
\begin{equation*} \bigcup_{(m, \, \rho) \in \Q^2} E_{m, \, \rho} = \text{$\mu$-almost all of $E$},\end{equation*}
since there may be $A \subset X$ null set such that
\begin{equation*} x \in A \implies \Theta_d^\ast(\mu, \, x) = + \infty.\end{equation*}
It is clearly enough to prove that the measure of $E_{m, \, \rho}$ is equal to $0$ for every fixed couple $(m, \, \rho) \in \Q^2$. 

By assumption $\mathcal{H}^d(E_{m, \, \rho}) = 0$, and thus $\mathcal{H}_\delta^d(E_{m, \, \rho}) = 0$. In particular, for every $\epsilon > 0$ there exists a covering $\{ E_i\}_{i \in I_{m, \, \rho}}$ for $E_{m, \, \rho}$ satisfying the additional property
\begin{equation*} \mathrm{diam}(E_i) < \delta \quad \text{and} \quad \sum_i \left(\mathrm{diam}(E_i) \right)^d \leq \epsilon. \end{equation*}
Let us consider the closed ball $B_i := \overline{B(x_i, \, r_i)}$, centered at $x_i \in E_i \cap E$ with radius $r_i = \mathrm{diam}(E_i)$, for every $i \in I_{m, \, \rho}$. The collection of balls $\{B_i\}_{i \in I_{m, \, \rho}}$ covers $E_{m, \, \rho}$, and thus
\begin{equation*}\mu(E_{m, \, \rho}) \leq \sum_{i \in I_{m, \, \rho}} \mu(B_i) \leq m \alpha_d \sum_{i \in I_{m, \, \rho}} r_i^d \leq m \alpha_d \sum_{i \in I_{m, \, \rho}} \left(\mathrm{diam}(E_i)\right)^d \leq m \alpha_d \cdot \epsilon, \end{equation*}
which is exactly what we wanted to prove.
\end{proof}

\begin{lemma} \label{lemma33} Let $\mu$ be a locally finite measure defined on a metric space $X$. Fix $m > 0$ and let
\begin{equation*}E_m := \left\{ x \in E \: \left| \: \Theta_d^\ast(\mu, \, x) > m \right. \right\}. \end{equation*}
The $d$-dimensional Hausdorff measure of $E_m$ is bounded from above, that is,
\begin{equation*}\mathcal{H}^d(E_m) \leq \frac{5^d}{m} \mu(E_m). \end{equation*}\end{lemma}

\begin{proof}Let $A$ be an open neighborhood of $E_m$, and let us consider the collection of closed balls
\begin{equation*}\mathcal{F} := \left\{ \overline{B(x, \, r)} \: \left| \: \text{$\overline{B(x,\, r)} \subseteq A$ and $\frac{\mu \left( \overline{B(x, \, r)} \right)}{\alpha_d r^d} > m$} \right. \right\}. \end{equation*}
The reader may prove that $\mathcal{F}$ is a fine cover of $E_m$; thus by \hyperref[lemma:vcl]{Vitali's covering Lemma \ref{lemma:vcl}} there exists a disjoint subfamily $\mathcal{F}^\prime \subset \mathcal{F}$ such that $\widehat{\mathcal{F}^\prime}$ covers $E_m$. It follows that
\begin{equation*} \begin{aligned} \mu(A) & \geq \sum_{B \in \mathcal{F}^\prime} \mu(B) \geq
\\[1em] & \geq m \sum_{B \in \mathcal{F}^\prime} \alpha_d r^d = \\[1em] & = \frac{m}{5^d} \frac{\alpha_d}{2^d} \sum_{B \in \mathcal{F}^\prime} (10 r)^d \geq \\[1em] & \geq \frac{m \alpha_d}{10^d} \sum_{B \in \mathcal{F}^\prime} \left( \mathrm{diam}(\hat{B}) \right)^d \geq \frac{m}{5^d} \mathcal{H}_\infty^d(E_m).\end{aligned} \end{equation*}
In particular, to estimate $\mathcal{H}_\delta^d$ it is enough to consider a cover made up of balls whose diameter is less than $\delta/10$. If we take the infimum with respect to $A$ and the supremum with respect to $\delta > 0$, then it turns out that
\begin{equation*}\mathcal{H}^d(E_m) \leq \frac{5^d}{m} \mu(E_m). \end{equation*}
\end{proof}

\begin{corollary}Let $\mu$ be a locally finite measure defined on a metric space $X$ and assume that
\begin{equation*}\Theta_d^\ast(\mu, \, x) > 0 \end{equation*}
for $\mu$-almost every $x \in X$. Then $\mu$ is supported on the set
\begin{equation*}E := \left\{ x \in X \: \left| \: \Theta_d^\ast(\mu, \, x) > 0 \right. \right\}, \end{equation*}
which is $\sigma$-finite with respect to $\mathcal{H}^d$. \end{corollary}

\begin{proof}It follows immediately from the previous theorem since
\begin{equation*} E = \bigcup_{n \in \N} E_{\frac{1}{n}} \end{equation*}
and we have, for every $n \in \N$, the estimate
\begin{equation*} \mathcal{H}^d ( E_{\frac{1}{n}} ) \leq \frac{5^d}{m}\mu( E_{\frac{1}{n}} ) < + \infty. \end{equation*}
\end{proof}

\begin{proposition} Let $\mu$ be a locally finite measure defined on a metric space $X$ and assume that
\begin{equation*} 0 \leq \Theta_d^\ast(\mu, \, x) < + \infty \end{equation*}
for $\mu$-almost every $x \in X$. Then
\begin{equation*} \mu = f \cdot \mathcal{H}^d \end{equation*}
for some locally summable function $f \in L_{\mathrm{loc}}^1(\mathcal{H}^d)$. \end{proposition}

\begin{proof}By \hyperref[lemma32]{Lemma \ref{lemma32}}, the measure $\mu$ is absolutely continuous with respect to the $d$-dimensional Hausdorff measure $\mathcal{H}^d$, while the previous corollary implies that $\mu$ is supported on the set $E$ defined above. Therefore
\begin{equation*} \mu \ll \lambda := \mathbbm{1}_E \cdot \mathcal{H}^d, \end{equation*}
and $\lambda$ is $\sigma$-finite; thus the Radon-Nikodym theorem implies that $\mu = f \cdot \lambda$.
\end{proof}

\begin{remark} Notice that the Radon-Nikodym theorem cannot be applied directly to $\mathcal{H}^d$ since the Hausdorff measure is not $\sigma$-finite (which is a necessary assumption for the result to hold true.) \end{remark}

\section{Applications}

In this final brief section, we investigate some of the main applications of the theory developed so far (especially to Cantor sets.)

\begin{corollary} Let $E$ be a Borel set contained in a metric space $X$, and let $d \geq 0$ be a positive real number. Assume that there exists a finite measure $\mu$, defined on $X$, such that: \mbox{}
\begin{enumerate}[label=\textbf{(\alph*)}]
\item The $d$-dimensional upper density is finite, that is, $\Theta_d^\ast(\mu, \, x) < + \infty$ for $\mu$-almost every $x \in E$.
\item The set $E$ is not null with respect to $\mu$, that is, $\mu(E) > 0$.
\end{enumerate}
Then the $d$-dimensional Hausdorff measure of $E$ is strictly bigger than zero. \end{corollary}

\begin{proof} By \hyperref[theorem:densidut]{Theorem \ref{theorem:densidut}}, the measure $\lambda := \mathbbm{1}_E \cdot \mu$ is absolutely continuous with respect to $\mathcal{H}^d$; therefore the Hausdorff measure of the set $E$ cannot be zero (otherwise also $\mu(E)$ would be equal to zero).\end{proof}

\paragraph{Cantor Set.} We have proved in the previous chapter that
\begin{equation*}d := \frac{\log 2}{\log 3}, \end{equation*}
is the Hausdorff dimension of the Cantor set $\mathcal{C}$, and recall that $\mathcal{H}^d(\mathcal{C}) > 0$.

\begin{remark}The Cantor set is defined by setting
\begin{equation*}\mathcal{C} := \bigcap_{i = 0}^{+ \infty} C_i, \end{equation*}
where $C_0 = [0, \, 1]$, $C_1 = [0, \, 1/3] \cup [2/3, \, 1]$ and $C_k$ is the disjoint union of $2^k$ intervals, whose length is equal to $3^{-k}$. \end{remark}

\begin{proposition}\label{prop:210kdso02} Let
\begin{equation*} \Phi( I_{i, \, j} ) = \Phi( I_{i + 1, \, k} ) + \Phi( I_{i + 1, \, k+1}) \end{equation*}
be a set function (that preserves the mass). Then $\Phi$ can be uniquely extended to a finite measure $\nu$ on $\R$, which is supported on the Cantor set $\mathcal{C}$. \end{proposition}

\begin{proof} Let $\mathcal{F} = \{ I_{i, \, j} \}_{i, \, j \in \N} \cup \varnothing$ and let $\mu$ be the outer measure on $\mathcal{C}$ given by the Carathéodory construction. We claim that the following properties are satisfied: \mbox{}
\begin{enumerate}[label = \textbf{(\alph*)}]
\item For every $i, \, j \in \N$ it turns out that $\mu(I_{i, \, j}) = \Phi(I_{i, \, j})$.
\item The outer measure $\mu$ is additive on distant sets (and thus the restriction to the Borel $\sigma$-algebra is a $\sigma$-additive measure.)
\end{enumerate}
As a consequence of the second claim, the restriction of $\mu$ to the Borel $\sigma$-algebra, denoted by $\nu$, is exactly the sought measure. The reader may prove, as an exercise, that the measure defined in this way is \textbf{unique}. \mbox{}
\begin{enumerate}[label = \textbf{(\alph*)}]
\item Let us consider the outer measure
\begin{equation*}\mu \left(I_{i, \, j} \right) := \inf \left\{ \Phi(E_k) \: \left| \: \text{$E_k \in \mathcal{F}$ and $I_{i, \, j} \subseteq E_k$} \right. \right\}. \end{equation*}
Notice that $\left( E_k \right)_{k \in \N}$ is an open covering of $\mathcal{C}$ and, for every $i, \, j \in \N$, the interval $I_{i, \, j}$ is compact. It follows that there exists a finite subfamily $E_{i_1}, \, \dots, \, E_{i_k}$ such that
\begin{equation*}\mu \left(I_{i, \, j} \right) = \inf_{\ell = 1, \, \dots, \, k} \Phi(E_{i_\ell}). \end{equation*}
This proves that $\mu(I_{i, \, j}) = \Phi(I_{i, \, j})$ for every $i, \, j$ since $\Phi$ is additive on distant sets\footnote{It follows easily from the definition of $\Phi$ and of $\mathcal{C}$.}.
\item We consider the outer measure
\begin{equation*}\mu_\delta \left(I_{i, \, j} \right) := \inf \left\{ \Phi(E_k) \: \left| \: \text{$E_k \in \mathcal{F}$, $\mathrm{diam}(E_k) \leq \delta$ and $I_{i, \, j} \subseteq E_k$} \right. \right\}. \end{equation*}
Then $\mu_\delta$ coincides with $\mu$ on a suitable class of sets, and it is larger on where the distance is bigger than $\delta$. This concludes the proof. \end{enumerate} \end{proof}

\paragraph{Canonical Measure on $\mathcal{C}$.} Let $\mu$ be the measure given by \hyperref[prop:210kdso02]{Proposition \ref{prop:210kdso02}} associated to the set function
\begin{equation*} \Phi(I_{i, \, j}) = 2^{-i}. \end{equation*}
It is enough to prove that $\Theta_d^\ast(\mu, \, x) < + \infty$ for $\mu$-almost every $x \in \mathcal{C}$. First, we notice that for every $x \in \mathcal{C}$, it turns out that
\begin{equation*} \widetilde{\Theta_d^\ast}(\mu, \, x) := \lim_{ \mathrm{diam}(I_{i, \, j}) \to 0, \, x \in I_{i, \, j}} \frac{\mu(I_{i, \, j})}{ \left( \mathrm{diam}(I_{i, \, j}) \right)^d} = \frac{2^{-i}}{3^{-id}} = 1. \end{equation*}
Then
\begin{equation*} \Theta_d^\ast(\mu, \, x) \in \left[c_1 \widetilde{\Theta_d^\ast}(\mu, \, x), \, c_2 \widetilde{\Theta_d^\ast}(\mu, \, x) \right], \end{equation*}
and this is exactly what we wanted to prove. Indeed, the closed ball $\overline{B(x, \, r)}$ satisfies the inequality $3^{-i-1} < r \leq 3^{-i}$, and thus $\overline{B(x, \, r)} \cap \mathcal{C} \subseteq I_{i, \, j}$ for some $j \in \N$. 
\chapter{Self-Similar Sets}

In this brief chapter, we investigate the notion of Hutchinson's \textit{fractals} (or self-similar sets), and we prove that, under suitable assumptions, given a certain number of similarities, there exists a unique compact self-similar set with a precise Hausdorff dimension $d$.

\begin{definition}[Self-Similar] \index{self-similar set} A subset $E \subseteq \R^n$ is \textit{self-similar} if there exist a finite collection $\phi_1, \, \dots, \, \phi_N$ of similarities
\begin{equation*} \phi_i(x) := x_i + \lambda_i \cdot R_i x, \quad \text{where $R_i \in O(n)$ and $\lambda_i \in (0, \, 1)$}, \end{equation*}
such that
\begin{equation*} E = \bigcup_{i = 1}^{N} \phi_i(E). \end{equation*}
\end{definition}

\begin{remark}If the $\phi_i(E)$ are (essentially) disjoint\footnote{More precisely, we say that $\phi_i(E)$ and $\phi_j(E)$ are \textit{essentially} disjoint if and only if $\mathcal{L}^n \left(\phi_i(E) \cap \phi_j(E) \right) = 0$. }, then we expect the Hausdorff dimension of $E$ to be equal to the unique solution of the equation
\begin{equation} \label{equazionesol} \sum_{i = 1}^{N} \lambda_i^d = 1. \end{equation} \end{remark}

It is important to stress that the Hausdorff dimension should be equal to $d$, but, a priori, there is no guarantee that the $d$-dimensional Hausdorff measure of $E$ is finite. We now state two lemmas that explain what we mean by \textit{should be equal to}.

\begin{lemma} The equation \eqref{equazionesol} admits a unique solution $d \geq 0$, provided that the $\lambda_i$s are positive and strictly less than one. \end{lemma}

\begin{lemma} If the $\phi_i(E)$ are (essentially) disjoint and if there exists $d \geq 0$ such that $0 < \mathcal{H}^d(E) < + \infty$, then $d$ is the unique solution of \eqref{equazionesol}.  \end{lemma}

\begin{proof} This is a straightforward application of the properties of the Hausdorff measure:
\begin{equation*} E = \bigsqcup_{i = 1}^{N} \phi_i(E) \implies \mathcal{H}^d(E) = \sum_{i=1}^{N} \mathcal{H}^d \left(\phi_i(E) \right) = \left( \sum_{i=1}^{N} \lambda_i^d \right) \, \mathcal{H}^d(E),\end{equation*}
that is, the factor $\left( \sum_{i=1}^{N} \lambda_i^d \right)$ needs to be equal to one for the identity to hold. \end{proof}

\paragraph{Hutchinson Construction of Self-Similar Sets.} Let be given $\phi_1, \, \dots, \, \phi_N$ similarities with scaling factors $\lambda_i \in (0,\, 1)$. For the next theorem to be true, we need to introduce the \textit{open set condition}.

\paragraph{The open set condition (OSC).}\index{open set condition (OSC)} The finite family of similarities $\phi_1, \, \dots, \, \phi_N$ satisfies the open set condition if and only if there exists a nonempty open set $V \subset \R^n$ such that
\begin{equation*} \bigcup_{i = 1}^{N} \phi_i(V) \subseteq V \qquad \text{and} \qquad \phi_i(V) \cap \phi_j(V) = \varnothing \: \: \text{for $i \neq j$}.\end{equation*}

\begin{theorem}[Hutchinson] \index{Hutchinson Theorem}\label{th:hutchisnds} Let be given a finite collection $\phi_1, \, \dots, \, \phi_N$ of similarities with scaling factors $\lambda_i \in (0,\, 1)$ satisfying the (OSC) condition. Then there exists a unique self-similar compact set $C \subseteq \R^n$, that is,
\begin{equation*} C = \bigcup_{i=1}^{N} \phi_i(C). \end{equation*}
The Hausdorff dimension of $C$ is the unique solution $d$ of the equation \eqref{equazionesol}, and the $d$-dimensional Hausdorff measure of $C$ is finite and nonzero, that is,
\begin{equation*} 0 < \mathcal{H}^d(C) < + \infty. \end{equation*}
\end{theorem}

The proof of this theorem is hard, and we need extra care to deal properly with the open set condition. For this reason, we replace it with a stronger condition: 
\begin{equation*} \phi_i \left (\overline{V} \right) \cap \phi_j \left(\overline{V} \right) = \varnothing, \qquad \forall \, i \neq j. \end{equation*}
More precisely, we assume that the images of $V$ via the similarities are distant.

\begin{proof}Let $X := \overline{V}$, and let $\mathcal{F} := \left\{F \subseteq X \: \left| \: \text{$F$ is closed.} \right. \right\}$ be the family of all the closed (and thus compact) subsets of $X$.

\paragraph{Step 1.} The Hausdorff distance induces a structure of metric space on $\mathcal{F}$. Indeed, if we denote by $\mathcal{U}_r(A)$ an open neighborhood of $A$ with diameter $r$, then one can prove that
\begin{equation*}d_{\mathcal{H}}(C_1, \, C_2) := \inf \left\{ r > 0 \: \left| \: \text{$C_1 \subseteq \mathcal{U}_{r}(C_2)$ and $C_2 \subseteq \mathcal{U}_{r}(C_1)$} \right. \right\}, \end{equation*}
is a metric. Moreover, the following implications hold\footnote{These two statements are not related to the content of this course; hence both are left as an exercise for the reader.}: \mbox{}
\begin{enumerate}[label=\textbf{(\alph*)}]
\item If $X$ is complete, then $\left( \mathcal{F}, \, d_\mathcal{H} \right)$ is complete.
\item If $X$ is compact, then $\left( \mathcal{F}, \, d_\mathcal{H} \right)$ is compact.
\end{enumerate}

\paragraph{Step 2.} Let us consider the following operator
\begin{equation*}\Phi : \mathcal{F} \longrightarrow \mathcal{F}, \qquad F \longmapsto \bigcup_{i = 1}^{N} \phi_i(F). \end{equation*}
between two Banach spaces, and let $\lambda_{max} := \max_{i = 1, \, \dots, \, N} \lambda_i \in (0, \, 1)$. By definition, it turns out that
\begin{equation*} d_{\mathcal{H}} \left(\Phi(F_1), \, \Phi(F_2) \right) \leq \lambda_{max}\cdot d_{\mathcal{H}} (F_1, \, F_2), \end{equation*}
which means that the operator $\Phi$ is a contraction. The Banach fixed point theorem\footnote{\textbf{Theorem.} Let $B$ be a Banach space, and let $T : B \longrightarrow B$ be a contraction. Then there exists a unique $b \in B$ such that $T(b) = b$, and for every $x \in B$ it turns out that $b$ is the limit point of the sequence $\{ x_n := T^n x \}_{n \in \N}$. } proves that there exists a unique fixed point $C$ which is given by
\begin{equation*} \lim_{j \to + \infty} \Phi^j(F), \end{equation*}
for every $F \in \mathcal{F}$. In particular, one can choose $X$ as a starting point for the sequence.

\paragraph{Step 3.} First, we observe that
\begin{equation*} X \supset \Phi(X) \supset \Phi^2(X) \supset \dots \implies \lim_{j \to + \infty} \Phi^j(X) = \bigcap_{j \in \N} \Phi^j(X) = C. \end{equation*}
For any $j \in \N$ and any $j$-tuple of indices $(i_1, \, \dots, \, i_j) \in \{1, \, \dots, \, N\}^j$, we denote by $X_{i_1, \, \dots, \, i_j}$ the iterated image of $X$, that is,
\begin{equation*} X_{i_1, \, \dots, \, i_j} := \phi_{i_j} \circ \dots \circ \phi_{i_1}(X). \end{equation*}
This notation is particularly useful since we can now express $C$ as an infinite intersection of finite unions (a fairly straightforward generalization of the construction of the Cantor set), that is,
\begin{equation} \label{eqsikdkls} \bigcap_{j \in \N} \left( \bigcup_{1 \leq i_1, \, \dots, \, i_j \leq N} X_{i_1, \, \dots, \, i_j} \right) = C. \end{equation}

\paragraph{Step 4.} In this brief step, we want to find an \textbf{upper bound} on the $d$-dimensional Hausdorff measure of $C$ and, more precisely, we prove that
\begin{equation*} \mathcal{H}^d(C) \leq \frac{\alpha_d}{2^d} \left(\mathrm{diam}(X) \right)^d. \end{equation*}
Fix $\delta > 0$, choose $j$ so that $\lambda_{max}^j \cdot \mathrm{diam}(X) < \delta$, and observe that
\begin{equation*} \mathrm{diam}\left(X_{i_1, \, \dots, \, i_j} \right) \leq \lambda_{max}^j \cdot \mathrm{diam}(X) \quad \text{and} \quad \mathrm{diam}\left(X_{i_1, \, \dots, \, i_j} \right) = \lambda_{i_1} \dots \lambda_{i_j} \cdot \mathrm{diam}(X). \end{equation*}
As a consequence of the identity \eqref{eqsikdkls}, it follows that the family $\left\{ X_{i_1, \, \dots, \, i_j} \right\}$ is a covering of $C$ with diameter less than $\delta$ as $(i_1, \, \dots, \, i_j)$ ranges in the set of all the $j$-tuples; hence
\begin{equation*} \begin{aligned} \mathcal{H}_\delta^d(C) & \leq \frac{\alpha_d}{2^d} \sum_{1 \leq i_1, \, \dots, \, i_j \leq N}\left( \mathrm{diam}\left(X_{i_1, \, \dots, \, i_j}\right) \right)^d \leq \\[1em] & \leq \frac{\alpha_d}{2^d} \left( \mathrm{diam}(X) \right)^d \sum_{1 \leq i_1, \, \dots, \, i_j \leq N} \left(\lambda_{i_1} \dots \lambda_{i_j} \right)^d. \end{aligned} \end{equation*}
The right-hand side is, up to a constant, equal to $\left(\mathrm{diam}(X)\right)^d$ since
\begin{equation*} \sum_{1 \leq i_1, \, \dots, \, i_j \leq N} \left(\lambda_{i_1} \dots \lambda_{i_j} \right)^d = \left( \sum_{i = 1}^{N} \lambda_i^d \right)^j = 1. \end{equation*}
In particular, notice that the existence of the fixed point $C$ and the upper bound on the $d$-dimensional Hausdorff measure rely on the completeness of $X$ only: all the other assumptions are necessary for the lower bound!

\paragraph{Step 5.} The lower bound is rather delicate, and the rough idea behind it is to construct a probability measure $\mu$ on $C$ such that
\begin{equation*} \Theta_d^\ast(\mu, \, x) < + \infty, \qquad \forall \, x \in C. \end{equation*}
More precisely, we construct $\mu$ as the weak limit of a sequence of probability measures $(\mu_k)_{k \in \N}$ as follows. Let $x \in C$ be a point, and let $\mu_0 := \delta_{x}$ be the Dirac measure centered at that point. We define the measure
\begin{equation*} \mu_1 := \lambda_1^d  \delta_{\phi_1(x)} + \dots + \lambda_N^d \delta_{\phi_N(x)}, \end{equation*}
where the renormalization constants need to be chosen in this way since the problem is asymmetrical\footnote{The reader may check that a different choice of renormalization constants yields to a different result (e.g., with $1/N$).} as the weight is not uniformly distributed.

\paragraph{Step 5.1.} Let us denote by $x_{i_1, \, \dots, \, i_j}$ the image via the collection $\phi_1, \, \dots, \, \phi_N$ of $x$, that is,
\begin{equation*} x_{i_1, \, \dots, \, i_j} := \phi_{i_j} \circ \dots \circ \phi_{i_1}(x). \end{equation*}
The $j$th element of the sequence is thus given by
\begin{equation*} \mu_j := \sum_{1 \leq i_1, \, \dots, \, i_j \leq N} \left(\lambda_{i_1} \dots \lambda_{i_j} \right)^d \, \delta_{x_{i_1, \, \dots, \, i_j}}. \end{equation*}
The set function $\mu_j$ is a probability measure defined on $C$ for every $j \in \N$; hence there exists a subsequence\footnote{The whole sequence converges to $\mu$, but we do not use this fact in the proof. On the other hand, we will use this property in the next chapters; thus the reader may try to prove it by herself.} $\mu_{j_k}$ that converges to a probability measure $\mu$. We now claim that for every $j \in \N$ and for every $j$-tuple of indices $(i_1, \, \dots, \, i_j) \in \{1, \, \dots, \, N\}^j$, it turns out that
\begin{equation*} \mu \left(X_{i_1, \, \dots, \, i_j} \right) = \lambda_{i_1}^d \dots \lambda_{i_j}^d. \end{equation*}
It is easy to see that for any $k \geq j$, we have
\begin{equation*} \mu_{k} \left(X_{i_1, \, \dots, \, i_j} \right) = \lambda_{i_1}^d \dots \lambda_{i_j}^d,\end{equation*}
and hence our claim follows immediately if we can pass this identity to the limit as $k \to + \infty$.

Here we use the strong replacement for the open set condition: For every $j$-tuple, the set $X_{i_1, \, \dots, \, i_j}$ is both open and closed in $C$, and thus we can take the limit for $k \to + \infty$.

\paragraph{Step 5.2.} Fix $x \in C$. Then $x$ is uniquely identified by a sequence of indices $(i_j)_{j \in \N}$ in such a way that $x \in X_{i_1, \, \dots, \, i_j}$ for every $j \in \N$. Moreover, we have the identity
\begin{equation} \label{eqlsd} \frac{\mu\left(X_{i_1, \, \dots, \, i_j} \right)}{\left[ \mathrm{diam}(X_{i_1, \, \dots, \, i_j}) \right]^d} = \frac{1}{\mathrm{diam}(X)^d}, \end{equation}
since $\mathrm{diam}(X_{i_1, \, \dots, \, i_j}) = \lambda_{i_1} \dots \lambda_{i_j} \cdot \mathrm{diam}(X)$ by construction.

It remains to prove that \eqref{eqlsd} is enough to infer that the $d$-dimensional upper density of $\mu$ is finite. Fix $x \in C$ and fix $0 < r < d_{min}$, where
\begin{equation*}d_{min} := \inf_{i, \, j = 1, \, \dots, \, N} d\left(\phi_i(x), \, \phi_j(x) \right). \end{equation*}
There exists a natural number $j \in \N$ such that
\begin{equation*} d_{min}\cdot \lambda_{i_1} \dots \lambda_{i_{j+1}} < r \leq d_{min} \cdot \lambda_{i_1} \dots \lambda_{i_j}, \end{equation*}
and therefore
\begin{equation*} \overline{B(x, \, r)} \cap C \subseteq X_{i_1, \, \dots, \, i_j}. \end{equation*}
It follows that
\begin{equation*} \mu \left( \overline{B(x, \, r)} \right) \leq \mu \left(X_{i_1, \, \dots, \, i_j} \right), \end{equation*}
and this is enough to estimate the density ratio, that is,
\begin{equation*}\frac{ \mu \left( \overline{B(x, \, r)} \right) }{r^d} \leq \frac{ \mu \left(X_{i_1, \, \dots, \, i_j} \right) }{\left( d_{min} \cdot \lambda_{i_1} \dots \lambda_{i_j} \right)^d} = \left( \frac{1}{d_{min} \cdot \mathrm{diam}(X)} \right)^d < + \infty. \end{equation*}
%In a similar fashion, we notice that
%\begin{equation*} \overline{B(x, \, r)} \cap C \supseteq X_{i_1, \, \dots, \, i_{j+1}}, \end{equation*}
%and this is enough to estimate the density ratio, that is,
%\begin{equation*}\frac{ \mu \left( \overline{B(x, \, r)} \right) }{r^d} > \frac{ \mu \left(X_{i_1, \, \dots, \, i_{j+1}} \right) }{\left( d_{min} \cdot \lambda_{i_1} \dots \lambda_{i_j} \right)^d} \geq 0. \end{equation*}

\paragraph{Step 5.3} Here we rephrase the argument presented in \cite{manilla}. We want to prove that
\begin{equation*} \Theta_d^\ast(\mu, \, x) < + \infty, \qquad \forall \, x \in C \end{equation*}
is enough to infer that $\mathcal{H}^d(C) > 0$ strictly. Suppose that
\begin{equation*} B_r \subseteq X \subseteq B_R, \end{equation*}
and let $\rho > 0$ be a fixed real number. Denote by $\mathscr{S}$ the space of all finite sequences $(i_1, \, \dots, \, i_j)$ such that the following inequality holds:
\begin{equation}\label{eq.53} \lambda_{min} \cdot \rho \leq \lambda_{i_1} \dots \lambda_{i_j} \leq \rho. \end{equation}
It follows that the collection
\begin{equation*} \mathscr{X} := \left\{ X_{i_1, \, \dots, \, i_j} \: : \: (i_1, \, \dots, \, i_j) \in \mathscr{S} \right\} \end{equation*}
is disjoint, and thus every $X_{i_1, \, \dots, \, i_j} \in \mathscr{X}$ contains a ball of radius $r \cdot \lambda_{i_1} \dots \lambda_{i_j}$ and it is contained in a ball of radius $R \cdot \lambda_{i_1} \dots \lambda_{i_j}$. By \eqref{eq.53} we infer that
\begin{equation*}B_{r \cdot \lambda_{min} \cdot \rho} \subseteq X_{i_1, \, \dots, \, i_j}  \subseteq B_{R \cdot \rho}. \end{equation*}
In particular, every ball of radius $\rho$ intersects $q$ sets of the collection $\mathscr{X}$, where
\begin{equation*}q := \left(\frac{1 + 2R}{\lambda_{min} r} \right)^{n}. \end{equation*}
Moreover, by definition the support of $\mu_{i_1, \, \dots, \, i_j}$ is contained in $\overline{X_{i_1, \, \dots, \, i_j}}$ for every $j \geq 1$, and therefore
\begin{equation*} \mu = \sum_{(i_1, \, \dots, \, i_j) \in \mathscr{S}} \left( \lambda_{i_1} \dots \lambda_{i_j} \right)^d \mu_{i_1, \, \dots, \, i_j}. \end{equation*}
For every ball $B_\rho$ of radius $\rho$ such that $B_\rho \cap \overline{X_{i_1, \, \dots, \, i_j}} \neq \varnothing$, it turns out that
\begin{equation*} \mu(B_\rho) \leq \sum_{(i_1, \, \dots, \, i_j) \in \mathscr{S}} \left( \lambda_{i_1} \dots \lambda_{i_j} \right)^d \mu_{i_1, \, \dots, \, i_j}(\R^n), \end{equation*}
which means that
\begin{equation*} \mu(B_\rho) \leq q \rho^d = \frac{q |B_\rho|^d}{2^d} \end{equation*}
whenever $|B_\rho| < |X|$. Let $\{ U_i \}_{i \in I}$ be a countable cover of $C$, and notice that
\begin{equation*} C \subseteq \bigcup_{i \in I} B_i \quad \text{where $|B_i| \leq 2|U_i|$}, \end{equation*}
from which it follows that
\begin{equation*}1 = \mu(X) \leq \sum_{i \in I} \mu(B_i) \leq q \sum_{i \in I} |U_i|^d \simeq q \mathcal{H}^d(E),\end{equation*}
and thus $\mathcal{H}^d(E) \geq q^{-1} > 0$, which is what we wanted to prove.
\end{proof}
\chapter{Other measures and dimensions}

In this chapter, we first introduce the \textit{integralgeometric} measure, and then we investigate the Haar invariant $k$-dimensional measure.

In the second half, we show how the Haar measure may be used to define an invariant measure on the Grassmannian manifold $G(n, \, m)$, which will be extremely useful to study rectifiable sets.

\section{Geometric Integral Measure}

In this first section, we propose an alternative $k$-dimensional measure to the Hausdorff one and, at the same time, we exhibit a motivational example for introducing the notion of \textit{invariant measures}.

\paragraph{Definition ($1$-dimensional).} Let $E \subseteq \R^n$ be a subset, and fix a projection
\begin{equation*} P_L : \R^n \longrightarrow L \end{equation*}
onto a $1$-dimensional linear subspace (i.e., a line). We may easily define (see \hyperref[fig:012dkos02]{Figure \ref{fig:012dkos02}}) a measure, which is invariant under translation but not under rotations, as follows:
\begin{equation} \label{mea1} \int_{x \in L} \can \left( P_L^{-1}(x) \cap E \right) \, \mathrm{d} \mathcal{H}^1(x), \end{equation}
where $\can ( \cdot )$ denotes the cardinality function.

In order to define a rotation-invariant measure, the simplest idea that comes in mind is to consider the average value of \eqref{mea1}, as $L$ ranges in the set of all the $1$-dimensional subspaces of $\R^n$. More precisely, we "define" the $1$-dimensional integralgeometric\index{measure!integralgeometric} measure as follows:
\begin{equation} \label{mea11} \mathcal{I}^1(E) := \dashint_{L \in G(n, \, 1)} \left[ \int_{x \in L} \can \left( P_L^{-1}(x) \cap E \right) \, \mathrm{d} \mathcal{H}^1(x) \right]. \end{equation}
The measure \eqref{mea11} is not well-defined, since we do not know which measure we need to use to integrate over all $L \in G(n, \, 1)$; we will come back to this issue later.

\newpage

\begin{figure}[h]
\centering
\includegraphics[width=14cm, height=8cm]{images/TGMMMS1.png}
\label{fig:012dkos02}
\caption{One-dimensional translation-invariant measure.}
\end{figure}

\paragraph{Definition ($k$-dimensional).} The same construction can be easily generalized to a $k$-dimensional measure. Indeed, if we denote by $G(n, \, k)$ the Grassmannian manifold (i.e, the set of all $k$-dimensional subspaces of $\R^n$) then, it turns out that
\begin{equation} \label{meak} \mathcal{I}^k(E) := c_{k, \, n} \dashint_{V \in G(n, \, k)} \left[ \int_{x \in V} \can \left( P_V^{-1}(x) \cap E \right) \, \mathrm{d} \mathcal{H}^k(x) \right] \end{equation}
is a measure, which is invariant under translations and rotations. The reader may prove that, if the renormalization constant $c_{k, \, n}$ is chosen properly, then
\begin{equation*} \mathcal{I}^k(E) = \mathcal{L}^k(E) = \mathcal{H}^k(E), \end{equation*}
provided that $E$ is contained in a $k$-hyperplane of $\R^n$.

\paragraph{Definition Issue.} The $k$-dimensional integralgeometric measure \eqref{meak} is not well-defined (as we have already mentioned in the $1$-dimensional case.) Indeed, we are taking the average integral over all the elements of the Grassmannian manifold $(n, \, k)$, but we have not introduced yet a measure on that space that is also invariant.

This issue is the main reason why we are so interested in developing (at least partially) the theory of \textit{invariant measures}. At the end of the chapter, we will be able to prove the existence of an invariant measure $\gamma_{n, \, k}$ on the Grassmannian manifold $(n, \, k)$.

\paragraph{Basic Properties.} To conclude this introduction we give a list of relevant and compelling facts about the $k$-dimensional geometric integral measure and leave them as exercises for the reader.

\begin{lemma} Let $E$ be a subset of a $h$-dimensional surface $\Sigma$. Assume that $h > k$ strictly, and that the $h$-dimensional volume of $\Sigma$ is positive. Then $\mathcal{I}^k(E) = + \infty$. \end{lemma}

\begin{remark}The $k$-dimensional Hausdorff measure is, in general, different from the $k$-dimensional geometric integral measure.

More precisely, it may happen that for some subset $E \subset \R^n$, the Hausdorff dimension is $\mathrm{dim}_{\mathcal{H}}(E) > k$, but $\mathcal{I}^k(E) = 0$. \end{remark}

\begin{example}[Cantor Set] Let us consider the set $\mathcal{C}_2 \subset \R^2$ given by the product of two Cantor-type set with scaling factor equal to $1/4$ (see \hyperref[fig:cs]{Figure \ref{fig:cs}}.) 

In particular, we have the similarities $\Phi_1, \, \dots, \, \Phi_4$, all with scaling factor equal to $1/4$, in such a way that $\mathcal{C}_2$ is a self-similar fractal set in the sense of Hutchinson, that is,
\begin{equation*} \mathcal{C}_2 = \bigcup_{i =1}^4 \Phi_i(\mathcal{C}_2). \end{equation*}
By \hyperref[th:hutchisnds]{Theorem \ref{th:hutchisnds}}, the Hausdorff dimension of $\mathcal{C}_2$ is the unique solution to the equation
\begin{equation*} 4 \left(\frac{1}{4} \right)^d = 1 \implies d = 1, \end{equation*}
but the reader may prove, as an exercise, that $\mathcal{I}^1( \mathcal{C}_2) = 0$. \end{example}

\begin{figure}[h]
\centering
\includegraphics[width=12cm, height=6cm]{images/CS.png}
\label{fig:cs}
\caption{Cantor Square}
\end{figure}

\section{Invariant Measures on Topological Groups}

In this section, we examine the assumptions needed for the existence and uniqueness of an invariant measure defined on a topological group $\G$.

\begin{definition}[Push-Forward] \index{measure!push-forward} Let $\mu$ be a positive measure on $X$, and let $f : X \longrightarrow Y$ be a Borel function. The \textit{push-forward} measure of $\mu$ via $f$ is defined by setting
\begin{equation*}f_{\can}\mu(E) := \mu \left( f^{-1}(E) \right), \qquad \forall \, E \in \mathcal{B}(Y). \end{equation*} \end{definition}

\begin{lemma} Let $\left(X, \, \mathcal{B}(X) \right)$ and $\left(Y, \, \mathcal{B}(Y) \right)$ be measurable spaces, and let $\mu$ be a positive measure on $X$. Then the push-forward $f_\can \mu$ is a well-defined measure on the the Borel $\sigma$-algebra of $Y$. \end{lemma}

\paragraph{Topological Groups.} Let $\G$ be a topological group. For any $y \in \G$, we denote by $\tau_y$ the left-multiplication ($x \mapsto y \cdot x$) and by $\tau_y^\ast$ the right-multiplication ($x \mapsto x \cdot y$).

\begin{definition}[Invariant Measure] \index{measure!invariant} Let $\mu$ be a measure defined on a topological group $\G$. The measure $\mu$ is left-invariant on $\G$ if and only if
\begin{equation*} \left( \tau_y \right)_{\can} \mu = \mu\qquad \forall \, y \in \G. \end{equation*} 
In a similar fashion, the measure $\mu$ is right-invariant if and only if
\begin{equation*} \left( \tau_y^\ast \right)_{\can} \mu = \mu\qquad \forall \, y \in \G, \end{equation*} 
and, clearly, $\mu$ is \textit{invariant} if and only if $\mu$ is both left-invariant and right-invariant.
\end{definition}

We are now ready to state the existence and uniqueness results. We will not prove the second theorem in this course, but we will give, at the end of the section, a complete sketch of the proof of the first result (which is the only one we shall be using in this course).

\begin{theorem}\label{theorem:1das} Let $\G$ be a compact group. Then there exists a unique invariant probability measure on $\G$, called Haar measure. \end{theorem}

\begin{theorem}Let $\G$ be a locally compact and separable group. Then there exists a locally finite invariant measure on $\G$, which is unique up to a multiplicative constant. \end{theorem}

\paragraph{Grassmannian Manifold.} The Grassmannian $G(n, \, k)$ is, in general, not a group. Therefore, we cannot apply to it the theorem stated above, but there is another way around since, as we shall see soon, we can identify $G(n, \, k)$ with a quotient $\faktor{\G}{\Gamma}$, where $\Gamma$ satisfies certain properties.

Let $\G$ be a topological group acting on a set $X$, and let $\tau : \G \times X \longrightarrow X$ be the left action map, that is,
\begin{equation*} \tau(g, \, x) = g \cdot x. \end{equation*}
In order to be coherent with the previous paragraph, we denote by $\tau_g(x)$ the element $\tau(g, \, x)$ so that
\begin{equation*} \tau_{g_1 \cdot g_2} = \tau_{g_1} \tau_{g_2}. \end{equation*}
A measure $\mu$ defined on $X$ is \textit{invariant} under the action of $\G$ if and only if
\begin{equation*} \left( \tau_g \right)_{\can} \mu = \mu \end{equation*}
for every $g \in \G$. However, in this general setting, the existence (let alone the uniqueness) of an invariant measure is not guaranteed and, actually, we can exhibit an easy counterexample assuming both $\G$ and $X$ compact.

\begin{example}Let $X = \p^1(\R) \cong \R \cup \{0\}$ and let $\G$ be the group of all the projective transformation of $X$. Since the translation group is contained in $\G$, the only possible invariant measure is the $1$-dimensional Lebesgue measure. On the other hand, the Lebesgue measure is not invariant under homothety, and hence there are no invariant measures. \end{example}

\begin{theorem}\label{theorem:3osds}Let $\G$ be a topological group acting on a set $X$. An invariant measure (not necessarily unique) exists if one of the following conditions is satisfied: \mbox{}
\begin{enumerate}[label=\textbf{(\arabic*)}]
\item The group $\G$ is abelian (or, more generally, it satisfies the Weyl condition\footnote{We do not need to deal with this delicate condition, but the interested reader may find more information \href{https://en.wikipedia.org/wiki/Weyl_group}{here}.}.)
\item The set $X$ is isomorphic to the quotient $\faktor{\G}{H}$, where $H$ is a closed subspace\footnote{Notice that $H$ does not need to be a normal subgroup of $\G$.} of $\G$.
\end{enumerate} \end{theorem}

\begin{remark} If $\mu$ is a left-invariant measure on $\G$ and $\pi : \G \longrightarrow \faktor{\G}{H}$ is a projection onto a closed subset, then the push-forward $\pi_{\can} \mu$ is also an invariant measure.\end{remark}

The non-oriented Grassmannian manifold $G(n, \, k)$ is diffeomorphic to a certain quotient, so that, by \hyperref[theorem:3osds]{Theorem \ref{theorem:3osds}}, an invariant measure $\gamma_{n, \, k}$ exists. In particular, we will finally be able to show that the integralgeometric $k$-dimensional measure \eqref{meak} is well-defined.

\begin{lemma} Let $O(m)$ denote the group of the orthogonal $m\times m$ matrices. Then there is a diffeomorphism
\begin{equation*} G(n, \, k) \cong \faktor{O(n)}{\left(O(k) \times O(n - k) \right)}, \end{equation*}
where $O(k) \times O(n - k)$ is the space of orthogonal matrices made up of a $k\times k$ orthogonal block and a $(n-k)\times (n-k)$ orthogonal block. \end{lemma}

The reader may work out the details of the proof by themselves, but the intuitive idea behind it is simple: Given $W \in G(n, \, k)$ element of the Grassmannian manifold, consider an orthonormal basis $w_1, \, \dots, \, w_k$ for it. Complete it to an orthonormal basis $w_1, \, \dots, \, w_n$ of $\R^n$ and, at this point, define an equivalence class naturally (that is, up to change of orthonormal basis for $W$ and the complement of $W$ separately.)

In conclusion, as promised, we sketch the proof of \hyperref[theorem:1das]{Theorem \ref{theorem:1das}} in the special case of $\G$ Lie group, and we give a complete proof (of the existence, at least) in the case of $\G$ commutative).

\begin{proof}[Proof 1] Let $\G$ be a $k$-dimensional Lie group. The idea is to define a left-invariant $k$-form $\omega$, that is, a $k$-form such that the pull-back according to $\tau_y$ is $\omega$ itself. Then it suffices to check that
\begin{equation*} \mu(E) := \int_E \omega \end{equation*}
is the sought invariant measure, and also that it is unique. \end{proof}

\begin{proof}Let $\G$ be a commutative group, and let $\mathcal{P}$ be the space of probability measures defined on $\G$. For any $g \in \G$, set
\begin{equation*} \mathcal{P}_g := \left\{ \mu \in \mathcal{P} \: \left| \: \left(\tau_g\right)_{\can} \mu = \mu \right. \right\} \end{equation*}
be the subset of $\mathcal{P}$ containing all the $g$-invariant probability measures defined on $X$.

\paragraph{Step 1.} We want to prove that, for every $g \in \G$, the subset $\mathcal{P}_g$ is nonempty. Fix $\mu_0 \in \mathcal{P}$ and let us consider, for every $n \in \N$, the probability measure defined by setting
\begin{equation*} \mu_n := \frac{\mu_0 + \left(\tau_{g} \right)_{\can} \mu_0 + \dots + \left(\tau_{g^n} \right)_{\can} \mu_0}{n+1} \in \mathcal{P}, \end{equation*}
where $g^{n}$ denotes the product of $n$ copies of $g$.

By compactness there exists a subsequence $\mu_{n_k}$ weakly-$\ast$ converging to a measure $\mu_\infty$. We now claim that $\mu_\infty$ is a $\tau_g$-invariant probability measure. Indeed, by definition of $\mu_n$, it follows that
\begin{equation*} \left(\tau_g \right)_{\can} \mu_{n_k} \to \mu_\infty \implies \left(\tau_g\right)_{\can} \mu_\infty = \mu_\infty.  \end{equation*}

\paragraph{Step 2.} We want to prove that the intersection of all the $\mathcal{P}_g$ is nonempty, which is clearly enough to infer the existence of an invariant measure.

Let $g, \, h \in \G$ be two elements, let $\mu_0 \in \mathcal{P}_g$ be an invariant measure, and let $\mu_\infty$ be the weakly-$\ast$ limit of the sequence
\begin{equation*} \mu_n := \frac{\mu_0 + \left(\tau_{h} \right)_{\can} \mu_0 + \dots + \left(\tau_{h^n} \right)_{\can} \mu_0}{n+1} \in \mathcal{P}. \end{equation*}
The set $\mathcal{P}_g$ is weakly-$\ast$ closed; therefore $\mu_\infty \in \mathcal{P}_g \cap \mathcal{P}_h$. By induction we can prove that the family $\{\mathcal{P}_g\}_{g \in \G}$ has the finite intersection property, and thus, by compactness of $\G$, it immediately follows that
\begin{equation*} \bigcap_{g \in \G} \mathcal{P}_g \neq \varnothing. \end{equation*}

\paragraph{Step 3.} We want to prove that the intersection above contains only one element. In order to do that, we define the convolution product of two measures by setting
\begin{equation*} \mu_1 \ast \mu_2 (E) := \left(\mu_1 \times \mu_2 \right) \left( \{ (x_1, \, x_2) \: \left| \: x_1 + x_2 \in E \right. \} \right). \end{equation*}
The reader may prove that the convolution is commutative, and also that
\begin{equation*} \mu_1 \ast \mu_2 = \mu_1,\end{equation*}
if $\mu_1$ is an invariant measure.

This is enough to infer that the invariant measure is unique. Indeed, if $\lambda, \, \mu \in \cap_{g \in \G} \mathcal{P}_g$ are two invariant measures, then the properties above of the convolution implies that
\begin{equation*} \mu = \mu \ast \lambda = \lambda \ast \mu = \lambda \implies \mu = \lambda. \end{equation*}
\end{proof}
\chapter{Lipschitz Functions}

In the first half of the chapter, we present some of the basic properties of Lipschitz maps and their relations to Hausdorff measures. 

In the second half, we derive the so-called \textit{area formula} for Lipschitz maps and we show the close relations with the \textit{coarea formula}. 

\section{Definitions and Main Properties}

In this section, we introduce Lipschitz function and Lipschitz maps (between metric spaces), and we state some of the main properties (which explains why they are such a good replacement in geometric measure theory for $C^1$ functions.)

\begin{definition}[Lipschitz] \index{Lipschitz map} Let $X$ and $Y$ be metric spaces. A map $f : X \longrightarrow Y$ is \textit{Lipschitz} if there exists a positive constant $L > 0$ such that
\begin{equation*} d_Y \left( f(x), \, f(y) \right) \leq L \cdot d_X \left( x, \, y \right) \end{equation*}
for every $x, \, y \in X$. The Lipschitz constant of $f$ is the optimal one, that is,
\begin{equation*} \mathrm{Lip}(f) := \inf  \left\{ L > 0 \: \left| \: \text{$d_Y \left( f(x), \, f(y) \right) \leq L \cdot d_X \left( x, \, y \right)$ for every $x, \, y \in X$} \right. \right\}. \end{equation*}\end{definition}

\paragraph{Compactness.} We now present an Ascoli-Arzelà result, that is, a compactness criteria for Lipschitz maps.

\begin{theorem} \index{Lipschitz map!Ascoli-Arzelà Theorem} Let $(f_n)_{n \in \N}$ be a sequence of maps $f_n : X \longrightarrow Y$ between compact metric spaces, and assume that it is uniformly Lipschitz, that is, each $f_n$ has the same Lipschitz constant.

Then there exists a subsequence $(f_{n_k})_{k \in \N}$ converging uniformly to a Lipschitz map $f_\infty : X \longrightarrow Y$ with the same Lipschitz constant. \end{theorem}

This result follows immediately from an application of the typical Ascoli-Arzelà theorem for metric spaces, but the reader may notice that the assumptions are not optimal. Indeed, one may only assume that $X$ is locally compact and $Y$ is a pre-compact space.

\paragraph{Extension Property.} Lipschitz functions $f: X \longrightarrow \R$ have an excellent extension property, which preserves the Lipschitz constant so that, even if it is defined on a subset $E \subset X$, we can always talk about $f$ as a function defined on $X$.

\begin{lemma}[Mc Shame] \index{Lipschitz map!Mc Shame Theorem}\label{extprop} Let $f : E \subset X \longrightarrow \R$ be a Lipschitz function. There exists $F : X \longrightarrow \R$ such that
\begin{equation*}F \, \big|_E = f \qquad \text{and} \qquad \mathrm{Lip}(F) =\mathrm{Lip}(f). \end{equation*} \end{lemma}

\begin{proof}Let $L := \mathrm{Lip}(f)$, and set
\begin{equation} \label{ext1} F(x) := \inf \left\{ f(y) + L \cdot d_X(x, \, y) \: \left| \: y \in E \right. \right\}. \end{equation}
The reader may prove by themselves that $F$ is an extension of $f$, and also that $F$ is a $L$-Lipschitz map (since it is a lower envelope.)\end{proof}

The extension is, in general, not unique. Indeed, we can easily define a different extension by setting
\begin{equation} \label{ext2} \widetilde{F}(x) := \sup \left\{ f(y) - L \cdot d_X(x, \, y) \: \left| \: y \in E \right. \right\}. \end{equation}
The reader may check that \eqref{ext1} and \eqref{ext2} define, in general, two different extension of $f$. The extension property is also true for $Y := \R^m$-valued functions, but the argument above does not preserve the Lipschitz constant in general.

\begin{lemma} Let $f : E \subset X \longrightarrow \R^m$ be a Lipschitz function. There exists $F : X \longrightarrow \R^m$ such that
\begin{equation*}F \, \big|_E = f \qquad \text{and} \qquad \mathrm{Lip}(F) \leq \sqrt{m} \cdot \mathrm{Lip}(f). \end{equation*} \end{lemma}

\begin{proof}Let $L := \mathrm{Lip}(f)$. We may extend each component by setting
\begin{equation} \label{ext31} F_i(x) := \inf \left\{ f_i(y) + L \cdot d_X(x, \, y) \: \left| \: y \in E \right. \right\}. \end{equation}
The reader may prove by themselves that $F = (F_1, \, \dots, \, F_m)$ is an extension of $f$ satisfying the properties mentioned above.\end{proof}

If $X$ and $Y$ are metric spaces, then there is no guarantee that such an extension exists. On the other hand, if they are Hilbert spaces, then there is a highly nontrivial theorem that proves the existence of an extension that preserves the Lipschitz constant.

\begin{theorem}[Kirszbraun] \index{Lipschitz map!Kirszbraun Theorem} Let $f : E \subseteq X \longrightarrow Y$ be a Lipschitz map between Hilbert spaces. Then there exists $F : X \longrightarrow Y$ such that
\begin{equation*}F \, \big|_E = f \qquad \text{and} \qquad \mathrm{Lip}(F) = \mathrm{Lip}(f). \end{equation*} \end{theorem}

\section{Differentiability of Lipschitz Functions}

In this section, we investigate the Lusin property of Lipschitz maps (from $\R^n$ to $\R^m$) with $C^1$ maps and, in particular, we prove that Lipschitz maps are $\mathcal{L}^n$-almost everywhere differentiable.

\begin{theorem} [Lusin Property] \label{diff:lip} Let $f : \R^n \longrightarrow \R$ be a Lipschitz map. Then, for every $\epsilon > 0$ there exists a function $g_\epsilon \in C^1(\R^n; \; \R)$ satisfying the following properties: \mbox{}
\begin{enumerate}[label=\textbf{(\arabic*)}]
\item The two maps coincide up to a set of measure at most $\epsilon$, that is,
\begin{equation*}\mathcal{L}^n \left( \left\{ x \in \R^n \: \left| \: f(x) \neq g_\epsilon(x)  \right. \right\} \right) \leq \epsilon. \end{equation*}
\item The Lipschitz constant is bounded from above, that is,
\begin{equation*}\mathrm{Lip}(g_\epsilon) \leq \mathrm{Lip}(f). \end{equation*}
\end{enumerate}  \end{theorem}

We will not prove this theorem entirely as we will only sketch the proof of a "local" statement. On the other hand, at the end of the section we briefly show how to obtain the global statement via a \textit{partition of unity} (but this is a highly nontrivial result!)

\paragraph{Regularity.} First, we prove that Lipschitz maps from $\R^n$ to $\R$ (and thus to $\R^m$) are almost everywhere differentiable with respect to the Lebesgue measure.

\begin{theorem}[Rademacher] \index{Rademacher Theorem} \label{th:rad} Let $f : \R^n \longrightarrow \R$ be a Lipschitz map. Then $f$ is $\Le^n$-almost everywhere differentiable. \end{theorem}

\begin{remark}The extension property (i.e., \hyperref[extprop]{Mc Shame Lemma \ref{extprop}}) automatically gives us the Rademacher theorem for Lipschitz maps defined on a subset $E \subset \R^n$. \end{remark}

\begin{remark}The statement of the Rademacher theorem presented here is not the most general possible. Indeed, the proof we are about to give works for Lipschitz maps with values in \textit{any finite-dimensional normed space}.

On the other hand, for infinite-dimensional Banach spaces, the statement is usually false. There is a particular class of Banach spaces for which the theorem holds, usually referred in the literature as \textit{Banach spaces with the Lusin property}).\end{remark}

\begin{lemma} Let $f : \R^n \longrightarrow \R$ be a Lipschitz map. Then $f$ belongs to $W_{\mathrm{loc}}^{1, \, \infty}(\R^n)$, that is, the distributional gradient $D f$ is essentially bounded. \end{lemma}

\begin{proof}[Sketch of the Proof] Given a locally summable $f \in L_{\mathrm{loc}}^{1}(\R^n)$, and given a direction $v \in \R^n$, it is a well-known fact that the distributional directional derivative is given by
\begin{equation*} \frac{\partial f}{\partial v} = \lim_{h \to 0} \frac{f - \tau_{hv} f}{h}. \end{equation*}
Since $f$ is a Lipschitz map, we easily infer that
\begin{equation*} \sup_{x \in \R^n} \left|  \frac{f(x) - \tau_{hv} f(x)}{h} \right| \leq \mathrm{Lip}(f) |v|, \end{equation*}
that is, the directional distributional derivative belongs to $L^\infty(\R^n)$, and so also the distributional gradient.\end{proof}

\begin{remark}For every $p \in [1, \, + \infty)$ it turns out that
\begin{equation*}W_{\mathrm{loc}}^{1, \, \infty}(\R^n) \subset W_{\mathrm{loc}}^{1, \, p}(\R^n). \end{equation*} \end{remark}

\begin{theorem} Let $f : \R^n \longrightarrow \R$ be a continuous Sobolev function, that is $f \in W_{\mathrm{loc}}^{1, \, p}(\R^n)$ for some $p \in (n, \, + \infty)$. Then $f$ is $\Le^n$-almost everywhere differentiable, and the pointwise gradient agrees with the distributional gradient for $\Le^n$-almost every $x \in \R^n$. \end{theorem}

\begin{proof} Fix a ball $B := B \left(\underline{x}, \, r \right)$. 

\paragraph{Step 1.} We claim that
\begin{equation} \label{eq:ineqosc} \omega \left(f, \, B \right) \leq C(n) \cdot r \left( \dashint_B \left| \nabla f(x) \right|^p \, \mathrm{d}x \right)^{1/p}, \end{equation}
where $C(n)$ is a universal constant\footnote{We say that a constant is universal if it depends only on the dimension of the ambient space} and $\omega$ is oscillation function, that is,
\begin{equation*} \omega \left(f, \, B \right) := \sup \left\{ \left|f(x) - f(\underline{x}) \right| \: \left| \: |x - \underline{x}| < r \right. \right\} .\end{equation*}

\paragraph{Step 2.} In order to prove \eqref{eq:ineqosc}, we notice that we can always reduce to the case $B = B \left(0, \, 1 \right)$ by composing with the transformation
\begin{equation*}x \longmapsto \frac{x - \underline{x}}{r}. \end{equation*}
The Poincaré inequality gives the inequality\footnote{The constant depends only on the dimension $n$ since the domain is the unitary ball!}
\begin{equation*} \left| \int_{B(0, \, 1)} f(x) \, \mathrm{d}x - \dashint_{B(0, \, 1)} f(x) \, \mathrm{d}x \right| \leq c(n) \| \nabla f \|_{L^p(B)}, \end{equation*}
from which we infer that
\begin{equation*} \Phi(f) = \| \nabla f \|_{L^p(B(0, \, 1))} + \left| \dashint_{B(0, \, 1)} f(x) \, \mathrm{d}x \right|\end{equation*}
is equivalent to the usual $W^{1,\, p}(B(0, \, 1))$ norm. Therefore, we apply Sobolev embedding theorem and obtain the inequality
\begin{equation*} \|f\|_{\infty, \, B(0, \, 1)} \leq c_1(n) \left[ \| \nabla f \|_{L^p(B(0, \, 1))} + \left| \dashint_{B(0, \, 1)} f(x) \, \mathrm{d}x \right| \right]. \end{equation*}
In conclusion, it turns out that
\begin{equation*} \omega \left(f, \, B \right) \leq 2 \, \|f\|_\infty \leq 2 \, c_1(n)  \left[ \| \nabla \, f \|_{L^p(B(0, \, 1))} + \left| \dashint_{B(0, \, 1)} f(x) \, \mathrm{d}x \right| \right], \end{equation*}
and, by replacing $f$ with $f - \mathrm{av}(f)$, we infer that \eqref{eq:ineqosc} holds true since the left-hand side does not change if we add a constant to the function.

\paragraph{Step 3.} For every $h \in \R^n$ it turns out that
\begin{equation*} \left| f(\underline{x} + h) - f(\underline{x}) \right| \leq C(n)  |h|  \left( \dashint_B \left| \nabla f(x) \right|^p \, \mathrm{d}x \right)^{1/p}.\end{equation*}
Therefore, if $L$ is a linear application, it turns out that
\begin{equation*} \left| f(\underline{x} + h) - f(\underline{x}) - Lh \right| \leq C(n) |h| \left( \dashint_B \left| \nabla f(x)  - L \right|^p \, \mathrm{d}x \right)^{1/p},\end{equation*}
and this concludes the proof. Indeed, if $\underline{x}$ is a point of $L^p$-approximate continuity of $\nabla f \in \L_{\mathrm{loc}}^p(\R^n)$, and if we take $L := \nabla f(\underline{x})$, then it turns out that
\begin{equation*} \left| f(\underline{x} + h) - f(\underline{x}) - \nabla f(\underline{x}) h \right| \leq C(n) |h| \mathcal{O}(1) = C(n) \, \mathcal{O}( \|h\| ).\end{equation*}
\end{proof}

The Rademacher theorem for Lipschitz maps follows easily combining the lemma and the theorem above. We are finally ready to state the local version of the Lusin property theorem and to give, at least, the main ideas behind it.

\begin{theorem} Let $f : \R^n \longrightarrow \R$ be a $\Le^n$-almost everywhere differentiable map, and let $\Omega$ be an open bounded subset of $\R^n$. For every $\epsilon > 0$ there exist a compact subset $K_\epsilon \subset \Omega$ and a function $g_\epsilon \in C^1(\R^n; \; \R)$ satisfying the following properties: \mbox{}
\begin{enumerate}[label=\textbf{(\arabic*)}]
\item The two maps coincide up to a set of measure at most $\epsilon$, that is,
\begin{equation*}f \, \big|_{K_\epsilon} = g_\epsilon \, \big|_{K_\epsilon} \qquad \text{and} \qquad \mathcal{L}^n \left( \Omega \setminus K_\epsilon \right) \leq \epsilon. \end{equation*}
\item The Lipschitz constant is bounded from above, that is,
\begin{equation*}\mathrm{Lip}(g_\epsilon) \leq \mathrm{Lip}(f) + \epsilon \end{equation*}
provided that $f$ is Lipschitz.
\end{enumerate} \end{theorem}

\begin{proof} Let
\begin{equation*} D := \left\{ x \in \R^n \: \left| \: \text{$f$ is differentiable at $x$} \right. \right\}\end{equation*}
be the set of points where $f$ is differentiable. Then, for any $x \in D$, it turns out that
\begin{equation*} \omega(x, \, r) := \sup_{|h| < r} \left| \frac{f(x + h) - \nabla f(x) h}{h} \right| \to 0 \end{equation*}
decreasingly as $r \to 0^+$. For every $\epsilon > 0$ we can find a compact set $K_\epsilon \subset \Omega$ such that \mbox{}
\begin{enumerate}[label=\textbf{(\alph*)}]
\item $\Le^n \left(\Omega \setminus K_\epsilon \right) \leq \epsilon$;
\item $\nabla f$ is continuous at each point of $K_\epsilon$ (Lusin);
\item $\omega_r(x) \searrow 0$ as $r \to 0^+$, uniformly with respect to $x \in K_\epsilon$.
\end{enumerate}
The reader may check these assertions as an exercise\footnote{\textbf{Hint.} The second assertion follows from the Lusin theorem, while the third one follows from the Egorov theorem.}. Let $A = \R^n \setminus K_\epsilon$ be the complement of $K_\epsilon$ in $\Omega$; for any integer $i \in \Z$ we consider the set
\begin{equation*} A_i := \left\{ x \in A \: \left| \: \frac{1}{2^{i+1}} < d(x, \, K_\epsilon) < \frac{1}{2^{i-1}} \right. \right\}. \end{equation*}
Clearly $\{A_i\}_{i \in \Z}$ is a covering of $A$; hence there exists a smooth partition of unity $\{\sigma_i\}_{i \in \Z}$ subordinated to it, with the additional property
\begin{equation*} | \nabla \sigma_i | \leq 2^{i + 3}. \end{equation*}
Let $\rho$ be a regularizing kernel with support contained in the unitary ball, and assume also that $\int \rho = 1$ and $\int x \rho = 0$. If we let $\rho_i := \rho_{ \frac{1}{2^{i+1}}}$ be the rescaling, then we can set
\begin{equation*} g(x) := \begin{cases} f(x) & \text{if $x \in K_\epsilon$}, \\[0.4em] \displaystyle\sum_{i \in \Z} \left(f(x) \sigma_i(x) \right) \ast \rho_i(x) & \text{if $x \notin K_\epsilon$}. \end{cases}\end{equation*}
To conclude the proof the reader may check that
\begin{enumerate}[label=\textbf{(\alph*)}]
\item $f \equiv g$ on $K_\epsilon$;
\item $g$ is smooth on $A$;
\item $g$ is differentiable at every $x \in K_\epsilon$ and $\nabla g = \nabla f$ on $A$.
\end{enumerate}
\end{proof}

\section{Area Formula for Lipschitz Maps}
\label{sec:aflm}

The main result of this section is the following theorem.

\begin{theorem} \label{th:areaformula} Let $\Sigma$ be a $d$-dimensional surface of class $C^1$, and let $f : \Omega \longrightarrow \Sigma$ be a Lipschitz function defined on an open subset $\Omega \subseteq \R^d$. Then for every $E \subseteq \Omega$ Borel it turns out that
\begin{equation} \label{areaformula} \int_{\Sigma} m_{f, \, E}(y) \, \mathrm{d} \mathcal{H}^d(y) = \int_{E} Jf(x) \, \mathrm{d}\mathcal{L}^d(x), \end{equation}
where $m_{f, \, E}(y) := \symbol{35} \, \left(f^{-1}(y) \cap E \right)$ and $Jf(x)$ denotes the Jacobian\footnote{The Jacobian, in this abstract setting, will be defined below in \hyperref[def:jac]{Definition \ref{def:jac}}.} of $f$ at $x$. \end{theorem}

\begin{remark} If $f$ is an injective Lipschitz function, the area formula \eqref{areaformula} reduces to an expression we are more familiar with (the substitution of variables in integral calculus):
\begin{equation} \label{areaformula2} \mathcal{H}^d \left(f(E) \right) = \int_{E} Jf(x) \, \mathrm{d}\mathcal{H}^d(x). \end{equation} \end{remark}

\begin{remark}The codomain $\Sigma$ does not need to be a $d$-dimensional surface, but it is enough to require $\Sigma$ Riemannian manifold since a scalar product is all we need to define orthonormal basis. \end{remark}

\begin{remark}[Jacobian, I] If $\Sigma$ is a $d$-dimensional surface embedded in $\R^n$ for some $n$ and $f$ is differentiable at $x$, then the Jacobian of $f$ at $x$ can be easily computed using the standard formula
\begin{equation} \label{jac} Jf(x) = \sqrt{ \mathrm{det} \left(\nabla_t f(x) \right)^T \, \left(\nabla_t f(x) \right)}, \end{equation}
where $\nabla_t$ denotes \textit{tangent gradient}. More precisely, if we look at $f$ as a function from $\Omega$ to $\R^n$, then the differential at $x$ is a linear map
\begin{equation*} \mathrm{d}f_x : T_x \Omega \cong \R^d \longrightarrow \R^{n}. \end{equation*}
By choosing orthonormal bases of both $\R^d$ and $\R^n$, we obtain a matrix that represents the linear map $ \mathrm{d}f_x$, which is also the matrix representing $\nabla_t f(x)$.

\paragraph{Proof.} We notice that the adjoint map $\left(\mathrm{d}f_x\right)^\ast$ sends $T_{f(x)} \Sigma \cong \R^n$ to $\R^d$ since the scalar product canonically identifies a vector space with its dual. The linear map
\begin{equation*} \left(\mathrm{d}f_x\right)^\ast \circ \mathrm{d}f_x : T_x \Sigma \longrightarrow T_x \Sigma \end{equation*}
is represented by the square matrix $\left(\nabla_t f(x) \right)^T \left(\nabla_t f(x) \right)$ by construction, and thus we infer that the formula \eqref{jac} holds. \end{remark}

% DEF JAC Here the Jacobian , that is, the absolute value of the determinant of the application $\mathrm{grad} \, f : \R^d \to T_{f(x)} \, \Sigma$ - once two orthonormal basis of the tangents have been chosen.

\begin{remark} The main result of this section \eqref{areaformula} is right as it is stated, but the reader may notice that the proof we present completely ignores the measurability issues\footnote{A nice exercise would be to clean up the proof and make it rigorous.}. \end{remark}

A weaker statement, which can be proved as an exercise, with no measurability issue to check, is given by the following theorem.

\begin{theorem}\label{th:areaformulaweak} Let $f : \Omega \subseteq \R^d \longrightarrow \R^n$ be a Lipschitz function defined on an open subset $\Omega \subseteq \R^d$. Then for every $F \subset \R^n$ Borel set it turns out that
\begin{equation} \label{areaformulaw} \int_{F} \symbol{35} \left(f^{-1}(y) \right) \, \mathrm{d} \mathcal{H}^d(y) = \int_{f^{-1}(F)} Jf(x) \, \mathrm{d}x. \end{equation} \end{theorem}

A stronger (more general) statement, which requires a lot of work to be proved rigorously, is given by the following theorem.

\begin{theorem}\label{th:areaformulastr} Let $\Sigma$ be a $d$-dimensional surface of class $C^1$, and let $f : \Omega \longrightarrow \Sigma$ be a Lipschitz function defined on an open subset $\Omega \subseteq \R^d$.For every positive Borel function $h : \Omega \longrightarrow [0, \, + \infty]$, it turns out that
\begin{equation} \label{areaformulas} \int_{\Sigma} [ \sum_{s \in f(x)} h(s)] \, \mathrm{d} \mathcal{H}^d(x) = \int_{\Omega} h(x) Jf(x) \, \mathrm{d}x. \end{equation} \end{theorem}

\paragraph{Area Formula.} In this final paragraph, the main goal is to sketch the proof of \eqref{areaformula}. The first step is, as promised, to give meaning to the Jacobian of a Lipschitz function.

\begin{definition}[Jacobian] \label{def:jac} Let $f : \Omega \subseteq \R^d \longrightarrow \R^n$ be a function differentiable at $x$, and $d \leq n$. The $d$-dimensional Jacobian of $f$ at $x$, denoted by $J_d f(x)$, is defined by setting
\begin{equation*} J_d f(x) = \sup \left\{ \frac{ \mathcal{H}^d \left( \nabla_t f(x)(P) \right)}{\mathcal{H}^d[P]} \: \left| \: \begin{gathered} \text{$P$ is a $d$-dimensional} \\[0.4em] \text{parallelepiped contained in $\R^d$} \end{gathered} \right. \right\}. \end{equation*} \end{definition}

We are now ready to prove the area formula \eqref{areaformula}. We shall see soon that it is an immediate consequence of two technical lemmas.

\begin{lemma} \label{areaforma1} Let $A \subseteq \Omega$ be an open set such that the restriction $f \, \big|_A$ is a diffeomorphism, that is, injective and with maximal rank:
\begin{equation*} \mathrm{rank}\left(\mathrm{d}f(s)\right) = d. \end{equation*}
Then the push-forward measure
\begin{equation*} \mu := f_\symbol{35} \left( \mathbbm{1}_A(x) Jf(x) \cdot \mathcal{L}^d \right) \end{equation*}
is equal to the measure
\begin{equation*} \mathbbm{1}_{f(A)} \cdot \mathcal{H}^d. \end{equation*}
In particular, given the subset $F = f(A) \subseteq \Sigma$ as in \eqref{areaformulaw}, it turns out that
\begin{equation*} \mathcal{H}^d \left(F \right) = \int_{A} Jf(s) \, \mathrm{d}s, \end{equation*}
that is, the area formula \eqref{areaformula} holds if $F = f(E)$ for every $E$ Borel. \end{lemma}

\begin{lemma}  \label{areaforma2} Let $E \subseteq \Omega$ be a Borel set such that $Jf(s) = 0$ for every $s \in E$ (i.e., the rank of the differential map is not maximal). Then
\begin{equation*}\mathcal{H}^d \left(f(E) \right) = 0\end{equation*}
and the area formula \eqref{areaformula} holds for every such $E$. \end{lemma}

Suppose, for now, that both lemmas hold true. Then, for every Borel set $E \subset \Omega$, it is enough to split it into countably many pieces $E_i$ satisfying the following properties: \mbox{}
\begin{enumerate}[label=\textbf{\roman*)}]
\item The restriction $f \, \big|_{E_i}$ is a diffeomorphism for every $i \in I$.
\item The complement set
\begin{equation*}E \setminus \bigcup_{i \in I} E_i \end{equation*}
is the set of all points $s \in E$ such that the differential of $f$ at $s$ has rank strictly less than $d$.
\end{enumerate}

\begin{proof}[Proof of Lemma \ref{areaforma1}] First, we notice that the thesis follows easily if we can prove that the Radon-Nikodym density
\begin{equation} \label{eq.0} \frac{\mathrm{d}\mu}{\mathrm{d} \left( \mathbbm{1}_{f(A)} \cdot \mathcal{H}^d \right) }(s) = 1 \end{equation}
for \textbf{every} $s \in f(A)$.

\paragraph{Step 0.} Suppose that \eqref{eq.0} holds. The measure $\mu$ is supported, by definition, in the set $f(A)$ and it has no singular part\footnote{Indeed, if $\mu$ has a singular part, then there exists a point $s \in f(A)$ such that \eqref{eq.0} is $+ \infty$, against our assumption.}; hence
\begin{equation*}f_\symbol{35} \left( \mathbbm{1}_A \, Jf \cdot \mathcal{L}^d \right) = \mathbbm{1}_{f(A)} \cdot \mathcal{H}^d. \end{equation*}

\paragraph{Step 1.} Recall that
\begin{equation*} \frac{\mathrm{d}\mu}{\mathrm{d} \left( \mathbbm{1}_{f(A)} \cdot \mathcal{H}^d \right)}(s) = 1 \iff \frac{\mu \left(B(s, \, r) \right)}{\mathcal{H}^d \left( \left(B(s, \, r) \right) \cap \Sigma \right)} \xrightarrow{r \to 0^+} 1, \end{equation*}
and thus it is enough to find an asymptotic estimate of both the numerator and the denominator.

\paragraph{Step 2.} If $\pi_s : \R^n \longrightarrow T_s \Sigma$ is the orthogonal projection onto $T_s \Sigma$, then it turns out that
\begin{equation*} \left(1 - o(1) \right) \cdot |x - y| \leq \left| \pi(x) - \pi(y) \right| \leq |x - y| \end{equation*}
that is, the projection is an almost-isometry. It follows that
\begin{equation*} \mathcal{H}^d \left( B(s, \, r) \cap \Sigma \right) \sim \mathcal{H}^d \left( \pi \left( B(s, \, r) \cap \Sigma \right) \right) \sim \alpha_d r^d. \end{equation*}

\paragraph{Step 3.} In a similar fashion, given a ball $B_r := B(0, \, r) \subseteq T_s \Sigma$, one can consider the ellipsoid\footnote{The preimage of a ball via the differential is an ellipsoid because the rank is maximal.}
\begin{equation*} E_r := \mathrm{d}f^{-1} \left(B_r \right). \end{equation*}
The image of $E_r$ according to $f$ has measure
\begin{equation*} \mu \left( f(s + E_r) \right) \sim \mu \left( B(s, \, r) \right), \end{equation*}
and, on the other hand, it is easy to show that
\begin{equation*} \mu \left( f(s + E_r) \right) = \int_{s + E_r} Jf(x) \, \mathrm{d}x = Jf(s) \, |s + E_r| = Jf(s) \, \frac{|B_r|}{Jf(s)} = |B_r|. \end{equation*}
In particular, it turns out that
\begin{equation*} \mu \left( B(s, \, r) \right) \sim \mu \left( f(s + E_r) \right) \sim |B_r| = \alpha_d r^d, \end{equation*}
which is exactly what we wanted to prove.
\end{proof}

\begin{proof}[Proof of Lemma \ref{areaforma2}] By assumption, for every point $s \in E$ and positive real number $r > 0$ small enough, it turns out that
\begin{equation*} f \left(B(s, \, r) \right) \subseteq B^{k < d}\left(f(s), \, \mathcal{O}(r) \right) \times B \left(f(s), \, o(r) \right), \end{equation*}
where $\mathcal{O}(-)$ and $o(-)$ are, respectively, the Landau's symbols. It follows that
\begin{equation*} \mathcal{H}^d \left(f(B(s, \, r)) \right) \simeq \mathcal{O}(r)^k \cdot o(r)^{d - k} \sim o(r^d), \end{equation*}
and this is enough to conclude since we can cover the ellipsoid with a countable union of balls satisfying the estimate above, that is,
\begin{equation*} \sum_i \mathrm{diam}\left( (B_i) \right)^d \sim \left( \frac{\mathcal{O}(r)}{o(r)} \right)^k \cdot o(r)^d \sim o(r^d), \end{equation*}
which is exactly what we wanted to prove.
\end{proof}

\section{Coarea Formula for Lipschitz Maps} %%%%MOD.
\label{sec:coafor}

\paragraph{Coarea.} Let $f : \Omega \subseteq \R^n \longrightarrow \R^m$ be a map of class $C^1$, and assume that $m < n$. If $h : \Omega \longrightarrow [0, \, + \infty]$ is a positive function, then the coarea formula is given by
\begin{equation} \label{coarea} \int_{\R^n} \left[ \int_{f^{-1}(y) \cap \Omega} h(x) \, \mathrm{d} \mathcal{H}^{n - m}(x) \right] \mathrm{d}y = \int_{\Omega} h(x) J(f)(x) \, \mathrm{d}x, \end{equation}
where
\begin{equation*} J(f)(x) = \sqrt{ \mathrm{det}\left( (\nabla f(x)) \left(\nabla f(x) \right)^t \right)} \: \in \: M(n \times n, \, \R) \end{equation*}
is a suitable notion of Jacobian. The proof of formula \eqref{coarea} is simple, and it is left to the reader to fill in the details in what follows.

\paragraph{Sketch of the Proof.} \mbox{}

\paragraph{Particular Case.} If $f : \R^2 \longrightarrow \R$ is the projection over the first coordinate, then the formula reduces to the Fubini-Tonelli theorem.

\paragraph{General Case.} In the general case, the main idea is to split the set $\Omega$ as
\begin{equation*} \Omega = \Omega_s \sqcup \Omega_n, \end{equation*}
where $\Omega_s$ is the set of all singular points (i.e., all the $x \in \Omega$ such that the Jacobian at $x$ has non-maximal rank), and $\Omega_n$ is the set of all points such that the Jacobian is a matrix of maximal rank.

The formula \eqref{coarea} clearly holds at every point of the first kind since the Jacobian is not invertible (i.e., $J(f)(x) = 0$), while the set $f^{-1}(x) \cap \Omega$ is $\mathcal{H}^{n - m}$-null as expected.

The formula for regular points, on the other hand, can be easily deduced from the particular case discussed above ($f : \R^2 \longrightarrow \R$ projection) via a simple change of variables.

\begin{figure}[h]
\centering
\includegraphics[width=12cm, height=8cm]{images/TGMPSP1.png}
\caption{Idea of the coarea formula: integrate first along the blue lines, and then integrate the result along the violet lines}
\end{figure}

%\begin{theorem} Let $L : V \to W$ be a linear map between $d$-dimensional vector spaces (both endowed with a scalar product). Then
%\begin{equation*} \mathcal{H}^d(L(E)) = \left| \mathrm{det}(M) \right| \cdot \mathcal{H}^d(E), \end{equation*}
%where $M$ is the square matrix representing $L$ with respect to two orthonormal basis. \end{theorem}
%In order to reduce to this basic case, one can either prove it through the $\epsilon$-$\delta$ proof (i.e. almost affinity maps), or through the blow-up of measures.

%Let $\mu = \mathbbm{1}_{A} \cdot Jf \cdot \mathcal{H}^d$, and let $\mu^\prime$ be the push-forward according to $f$ of $\mu$. The identity \eqref{areaformula} is equivalent to proving that
%\begin{equation*} f_\symbol{35} \, \mu = \mathcal{H}^d \cdot m_{f, \, E}, \end{equation*}
%but we shall show only the easier case ($f$ injective) since it is enough to solve the general case as well. Therefore, the thesis is now equivalent to proving that
%\begin{equation*} \frac{\mathrm{d} \, \mu^\prime}{\mathcal{H}^d \, \mathbbm{1}_E} = 1 \qquad \text{at a.e. $y \in f(E)$}, \end{equation*}
%which is also equivalent to proving that the blow-up of $\mu^\prime$ at $y \in f(E)$ is equal to $\mathcal{H}^d$, at least for almost every $y$.

%DEF BLOW UP

%At this point one can show that \mbox{}
%\begin{enumerate}[label=\textbf{(\alph*)}]
%\item The blowup at $x$ of $\mu$ is equal to $\mathcal{H}^d \, Jf \, \mathbbm{1}_{B \cap T_x \, \Sigma}$, and
%\item the blowup at $y$ of $\mu^\prime$ is equal to $\mathcal{H}^d \, \mathbbm{1}_{B \cap T_y \, \Sigma^\prime}$.
%\end{enumerate}
%But this immediately implies the equality for linear maps, and thus it proves the area formula.
\chapter{Rectifiable Sets}

In this chapter, we shall be mainly concerned with the notions of $d$-dimensional \textit{rectifiable} and \textit{purely unrectifiable} sets in metric spaces.

In the first section, we furnish the definitions of rectifiable (and purely unrectifiable) set through Lipschitz maps, and we state some of the main criteria in $\R^n$.

In the second section, we work towards a (suitably weak) definition of the tangent bundle for rectifiable sets, and, in the next section, we end up proving that any Borel set $E$ with an approximate tangent cone and $d$-dimensional lower density bounded from below is $d$-rectifiable.

In the last section, we prove the area formula for rectifiable sets using the characterization as the union of the graphs of Lipschitz maps.

\section{Introduction and Elementary Properties}

\begin{definition}[Rectifiable Set] \index{rectifiable set} Let $X$ be a metric space. A Borel set $E \subseteq X$ is a $d$-dimensional \textit{rectifiable set} if and only if there exists a collection of Borel sets $\{E_i\}_{i \in \N}$ such that
\begin{equation*} E = \bigcup_{i = 0}^{+ \infty} E_i, \end{equation*}
satisfying the following properties: \mbox{}
\begin{enumerate}[label=\arabic*)]
\item $E_0$ is a $\mathcal{H}^d$-null set, and
\item $E_i \subset f_i(\R^d)$, where $f_i : \R^d \longrightarrow X$ is a Lipschitz function for every $i \geq 1$.
\end{enumerate}
\end{definition}

\begin{remark} Let $X$ be a metric space, and let $E \subset X$ be a $d$-dimensional rectifiable set. Then the Hausdorff dimension of $E$ is less than or equal to $d$, i.e.,
\begin{equation*} \mathrm{dim}_{\mathcal{H}} E \leq d \end{equation*} \end{remark}

\begin{proof} Let $\{E_i\}_{i \in \N}$ be the collection given by the definition of $d$-rectifiable set. The assumption that $E_0$ is a $\mathcal{H}^d$-null set immediately implies that
\begin{equation*}\mathrm{dim}_{\mathcal{H}} E_0 \leq d. \end{equation*}
On the other hand, by \hyperref[lemma:hausdorffprop]{Lemma \ref{lemma:hausdorffprop}} it follows that a Lipschitz map does not increase the Hausdorff dimension, that is,
\begin{equation*}\mathrm{dim}_{\mathcal{H}} \R^d = d \implies \mathrm{dim}_{\mathcal{H}} f_i(\R^d) \leq d, \end{equation*}
which is enough to infer (as a consequence of \hyperref[rmk:2o1dokk1]{Remark \ref{rmk:2o1dokk1}}) that 
\begin{equation*}E = \bigcup_{i \in \N} E_i \implies \mathrm{dim}_{\mathcal{H}} E = \sup_{i \in \N} \mathrm{dim}_{\mathcal{H}} E_i \leq d. \end{equation*}\end{proof}

\begin{exercise}Prove that $\mathcal{H}^d(E) = 0$ does not imply that $E$ is contained in a countable union of Lipschitz images of $\R^d$, that is,
\begin{equation*}E \not \subseteq \bigcup_{n \in \N} f_n(\R^d). \end{equation*} \end{exercise}

\begin{exercise}Prove that there exists a Borel set $E_0 \subseteq \R^2$ such that $\mathcal{H}^d(E_0) = 0$ which cannot be covered by a rectifiable curve (hint: self-similar sets). \end{exercise}

\begin{proposition}[Criteria of Rectifiability] \label{prop:equivdlsd}Let $X := \R^n$, and let $E \subset X$ be a Borel set. The following assertions are equivalent: \mbox{}
\begin{enumerate}[label=\textbf{(\alph*)}]
\item The set $E$ is $d$-rectifiable.
\item There exists a collection of Borel sets $\{E_i\}_{i \in \N}$ such that
\begin{equation*} E = \bigcup_{i = 0}^{+ \infty} E_i, \end{equation*}
satisfying the following properties: \mbox{}
\begin{enumerate}[label=\textbf{\arabic*)}]
\item $E_0$ is a $\mathcal{H}^d$-null set.
\item $E_i \subset f_i(A_i)$, where $A_i \subseteq \R^d$ is an open subset and $f_i : A_i \longrightarrow \R^n$ is a differentiable function.
\end{enumerate}
\item There exists a collection of Borel sets $\{E_i\}_{i \in \N}$ such that
\begin{equation*} E = \bigcup_{i = 0}^{+ \infty} E_i, \end{equation*}
satisfying the following properties: \mbox{}
\begin{enumerate}[label=\textbf{\arabic*)}]
\item $E_0$ is a $\mathcal{H}^d$-null set.
\item $E_i \subset f_i(A_i)$, where $A_i \subseteq \R^d$ is an open subset and $f_i : A_i \longrightarrow \R^n$ is a diffeomorphism.
\end{enumerate}
\item There exists a collection of Borel sets $\{E_i\}_{i \in \N}$ such that
\begin{equation*} E = \bigcup_{i = 0}^{+ \infty} E_i, \end{equation*}
satisfying the following properties: \mbox{}
\begin{enumerate}[label=\textbf{\arabic*)}]
\item $E_0$ is a $\mathcal{H}^d$-null set.
\item $E_i \subset \Sigma_i$, where $\Sigma_i$ is a $d$-dimensional surface of class $C^1$ contained in $\R^n$.
\end{enumerate}
\end{enumerate}
\end{proposition}

\begin{proof}By definition (differentiable maps are Lipschitz) \textbf{(b)} $\implies$ \textbf{(a)}, while the opposite implications is a straightforward consequence of the following result.

\vspace{2.5mm}
\begin{center}
\noindent\fbox{ 
\parbox{15cm}{ \begin{lemma}Let $f : \R^d \longrightarrow \R^n$ be a Lipschitz map. Then there is a collection $\{f_i\}_{i \in \N}$ of functions of class $C^1$ and a $\mathcal{H}^d$-null Borel set $A_0$ such that 
\begin{equation*}f(\R^d) \subseteq \bigcup_{i \in \N} f_i(\R^d) \cup A_0.  \end{equation*} \end{lemma}
\begin{proof} By \hyperref[diff:lip]{Theorem \ref{diff:lip}}, Lipschitz maps have the Lusin property in the class of differentiable maps $C^1$; hence, for every $i \in \N$, there exists a differentiable map $f_i : \R^d \longrightarrow \R^n$ such that
\begin{equation*} | E_i | := \left| \left\{ x \in \R^d \: \left| \: f_i(x) \neq f(x) \right. \right\} \right| \leq \frac{1}{i}. \end{equation*}
The reader may easily prove that, by construction, it turns out that
\begin{equation*} f(\R^d) \subset \left( \, \bigcup_{i \in \N} f_i(\R^d) \right) \cup \left( f \left( \, \bigcap_{i \in \N} E_i \right) \right).\end{equation*}
In particular, the image of the intersection is a $\mathcal{H}^d$-null set since
\begin{equation*} \mathcal{L}^d \left( \, \bigcap_{i \in \N} E_i \right) = 0 \implies \mathcal{H}^d \left( \, \bigcap_{i \in \N} E_i \right) = 0, \end{equation*}
and $f$ is a Lipschitz map (see \hyperref[lemma:hausdorffprop]{Lemma \ref{lemma:hausdorffprop}}). \end{proof} }}
\end{center}

\vspace{2.5mm}
\noindent Clearly \textbf{(c)} $\implies$ \textbf{(b)} (since diffeomorphisms are also differentiable maps); thus we only need to prove the opposite implication.

\noindent Suppose \textbf{(b)} holds, and let $\{E_i\}_{i \in \N}$ be the collection of Borel sets satisfying the definition, i.e,,
\begin{equation*} E = \bigcup_{i = 0}^{+ \infty} E_i, \end{equation*}
where $E_0$ is a $\mathcal{H}^d$-null set and $E_i \subset f_i(A_i)$ for differentiable maps $f_i : A_i \longrightarrow \R^n$. For every $i \geq 1$ we consider a partition of $A_i$
\begin{equation*} A_i = A_i^{max} \sqcup A_i^{min}, \end{equation*}
where
\begin{equation*} A_i^{max} := \left\{ x \in A_i \: \left| \: \text{$\mathrm{d}f_x$ has maximal rank} \right. \right\} \end{equation*}
and
\begin{equation*} A_i^{min} := \left\{ x \in A_i \: \left| \: \text{$\mathrm{d}f_x$ has NOT maximal rank} \right. \right\}. \end{equation*}
Recall that $A_i^{min}$ is a $\mathcal{H}^d$-null set for every differentiable map $f_i$. Hence, we can consider the collection of Borel sets given by
\begin{equation*}\widetilde{E_0} := E_0 \cup \left( \, \bigcup_{i \geq 1} f \left( A_i^{min} \right) \right) \qquad \text{and} \qquad \widetilde{E_i} := f_i \left(A_i^{max} \right) \end{equation*}
together with the diffeomorphisms $f_i \, \big|_{A_i^{max}}$.

In conclusion, it remains to prove the equivalence \textbf{(c)} $\implies$ \textbf{(d)}. The "only if" part is immediate from the definitions, while the "if" part is left to the reader as a simple exercise (in the same spirit of the implications we have already proved here).
\end{proof}
 
\begin{remark}If $X := \R^n$, we just proved that Lipschitz maps may be replaced by $C^1$ maps in the definition of $d$-rectifiable set. On the other hand, it is not possible to replace $C^1$ by a more regular space $C^{k \geq 2}$ since $C^1$ does not have the Lusin property in $C^2$. \end{remark} 

\begin{definition}[Purely Unrectifiable]  \index{purely unrectifiable set} Let $X$ be a metric space. A Borel set $E \subseteq X$ is a $d$-dimensional \textit{unrectifiable set} if and only if 
\begin{equation*} \mathcal{H}^d \left( E \cap f \left(\R^d \right) \right) = 0 \quad \text{for every Lipschitz map $f : \R^d \longrightarrow X$.} \end{equation*}\end{definition}

\begin{proposition}[Criteria of Unrectifiability] \label{prop:pequivdlsd} Let $X := \R^n$, and let $E \subset X$ be a Borel set. The following assertions are equivalent: \mbox{}
\begin{enumerate}[label=\textbf{(\alph*)}]
\item $E$ is purely $d$-unrectifiable. 
\item For every $C^1$ map $f : \R^d \longrightarrow \R^n$ it turns out that
\begin{equation*} \mathcal{H}^d \left( E \cap f \left(\R^d \right) \right) = 0. \end{equation*}
\item For every diffeomorphism $f : A \subseteq \R^d \longrightarrow \R^n$ defined on an open subset $A$, it turns out that
\begin{equation*} \mathcal{H}^d \left( E \cap f \left(A \right) \right) = 0. \end{equation*}
\item For every $d$-dimensional surface $\Sigma \subseteq \R^n$ of class $C^1$ it turns out that
\begin{equation*} \mathcal{H}^d \left( E \cap \Sigma \right) = 0. \end{equation*}
\end{enumerate}
\end{proposition}

\begin{notation} From now on, we will use purely unrectifiable and $1$-purely unrectifiable as synonyms. \end{notation}

\begin{proposition}Let $X = \R^2$. For every $d \in (0, \, 2]$ there exists a compact set $K_d \subset \R^2$ such that $\mathrm{dim}_{\mathcal{H}} \, K = d$ and $K$ purely unrectifiable. \end{proposition}

\begin{proof}Here we present the proof of the assertion for $d \in (0, \, 2)$. The reader may extend the construction to dimension $2$ as an exercise.

\paragraph{Step 1.} First, we claim that if $K = K_1 \times K_2$ is a product of compact subsets of $\R$, then
\begin{equation*} \mathcal{L}^1 \left(K_1 \right) = \mathcal{L}^1 \left(K_2 \right) = 0 \implies \text{$K$ purely unrectifiable}. \end{equation*}
Indeed, let $\mathcal{C}$ be a curve of class $C^1$ (a submanifold of $\R^2$). Since $\mathcal{C}$ is locally the graph of a function, we may assume, without loss of generality, that
\begin{equation*} \mathcal{C} = \left\{ \left(x_1, \, g(x_1) \right) \: \left| \: g \in C^1\left(\R; \; \R \right) \right. \right\}.\end{equation*}
The $1$-dimensional Hausdorff measure of the intersection $K \cap \mathcal{C}$ may be computed by parts, and thus we can use the assumption on the Lebesgue measure to obtain
\begin{equation*}\mathcal{H}^1 \left( K \cap \mathcal{C} \right) \leq \mathcal{H}^1 \left( (K_1 \times \R) \cap \mathcal{C} \right) = \int_{K_1} \sqrt{1 + \dot{g}^2} \, \mathrm{d}t = 0. \end{equation*}
Therefore, we conclude that $K$ is purely $1$-unrectifiable as a consequence of \hyperref[prop:pequivdlsd]{Proposition \ref{prop:pequivdlsd}}.

\paragraph{Step 2.} If $K_1 = K_2$ are self-similar fractals (in Hutchinson sense) of dimension $d^\prime$, then the product $K = K_1 \times K_2$ has Hausdorff dimension $2 d^\prime$.

Indeed, the reader may prove that for any two separable metric spaces $X$ and $Y$ with $Y$ totally bounded that
\begin{equation*}\mathrm{dim}_{\mathcal{H}} \, X + \mathrm{dim}_{\mathcal{H}} \, Y \leq \mathrm{dim}_{\mathcal{H}} \left(X \times Y \right) \leq \mathrm{dim}_{\mathcal{H}} \, X + \mathrm{dim}_{\mathcal{B}} \, Y, \end{equation*}
where $\mathrm{dim}_{\mathcal{B}} \, Y$ denotes the upper box counting dimension of $Y$ (see, e.g. \cite{manilla}).   In particular, if $Y$ has equal Hausdorff and upper box counting dimension (which holds if $Y$ is a compact interval) then
\begin{equation*}\mathrm{dim}_{\mathcal{H}} \, X + \mathrm{dim}_{\mathcal{H}} \, Y = \mathrm{dim}_{\mathcal{H}} \left(X \times Y \right). \end{equation*}

\paragraph{Step 3.} In conclusion, in \hyperref[sec:cts]{Section \ref{sec:cts}} we have proved that there is a Cantor-type set of every dimension $d \in (0, \, 1)$. Hence, for every $d \in (0, \, 1)$, the product $\mathcal{C}_d \times \mathcal{C}_d$ is a purely unrectifiable set of dimension $2d \in (0, \, 2)$, which is exactly what we wanted to exhibit. \end{proof}

\begin{proposition}Let $E \subset X$ be a Borel set with finite Hausdorff measure $\mathcal{H}^d(E) < + \infty$. Then $E$ is the union of a $d$-rectifiable part and a purely $d$-unrectifiable part, that is,
\begin{equation*} E = E_r \cup E_{pu}. \end{equation*}\end{proposition}

\begin{proof}Let us consider the family
\begin{equation*} \mathcal{F} = \left\{ F \subset E \: \left| \: \text{$F$ is a $d$-rectifiable subset} \right. \right\}. \end{equation*}
Let $m := \sup \left\{ \mathcal{H}^d(F) \: \left| \: F \in \mathcal{F} \right. \right\}$, and let $(F_n)_{n \in \N} \subset \mathcal{F}$ be a maximizing sequence, that is,
\begin{equation*} \mathcal{H}^d(F_n) \nearrow m. \end{equation*}
If we set
\begin{equation*} E_r := \bigcup_{n \in \N} F_n, \end{equation*}
then it turns out that $E_r \in \mathcal{F}$ (being rectifiable is closed under countable unions), and the Hausdorff dimension of $E_r$ is exactly $m$. In conclusion, we set
\begin{equation*} E_{pu} := E \setminus E_r, \end{equation*}
and we prove\footnote{The reader may prove this assertion as an exercise.}, by contradiction, that $E_{pu}$ is purely $d$-unrectifiable.

Indeed, suppose that $E^\prime := E \setminus E_r$ is not purely $d$-unrectifiable. Let us consider the family
\begin{equation*} \mathcal{F}^\prime = \left\{ F \subset E^\prime \: \left| \: \text{$F$ is a $d$-rectifiable subset} \right. \right\}, \end{equation*}
which is not empty by assumption. Let $m^\prime := \sup \left\{ \mathrm{dim}_{\mathcal{H}} \, F \: \left| \: F \in \mathcal{F}^\prime \right. \right\}$, and let $(F_n)_{n \in \N} \subset \mathcal{F}$ be a maximizing sequence, that is,
\begin{equation*} \mathcal{H}^d(F_n) \nearrow m^\prime. \end{equation*}
If we set
\begin{equation*} E_r^\prime := \bigcup_{n \in \N} F_n, \end{equation*}
then it turns out that $E_r \in \mathcal{F}^\prime$, and the Hausdorff dimension of $E_r^\prime$ is exactly $m^\prime$. In particular, it turns out that
\begin{equation*} \widetilde{E_r} := E_r \cup E_r^\prime \end{equation*}
is a $d$-rectifiable set with Hausdorff dimension $\geq m$; thus we have derived a contradiction with the maximality of $E_r$.\end{proof}

\paragraph{Characterization of $d$-rectifiable sets in $\R^n$.} In this final paragraph, we state two fundamental criteria for rectifiable sets (but we do not give the proof of them).

\begin{theorem}[Marstrand-Preiss] Let $E \subseteq \R^n$ be a Borel set. If $\mathcal{H}^d(E) < + \infty$ and
\begin{equation*} \lim_{r \to 0} \frac{ \mathcal{H}^d \left( E \cap B(x, \, r) \right)}{\alpha_d r^d} \neq 0, \, \infty\end{equation*}
exists for $\mathcal{H}^d$-almost every $x \in E$, then $d$ is an integer, and $E$ is a $d$-rectifiable set. \end{theorem}

\begin{remark} If $E \subseteq \R^n$ is a $d$-rectifiable set with positive Hausdorff measure $\mathcal{H}^d(E) > 0$, then for almost every $V \in G(n, \, d)$ it turns out that
\begin{equation*} \mathcal{H}^d \left( P_V(E) \right) > 0. \end{equation*} \end{remark}

\begin{theorem}[Besicovitch-Federer] Let $E \subseteq \R^n$ be a Borel set. If $E$ is purely $d$-unrectifiable, then
\begin{equation*} \mathcal{H}^d \left( P_V(E) \right) = 0\end{equation*}
for $\gamma_{n, \, d}$-almost every $V \in G(n, \, d)$. \end{theorem}

\begin{proof} This result is the main topic of my seminary. I will try to upload some written notes about it after the exam. \end{proof}

\section{Tangent Bundle}

In this section, we want to introduce a weaker notion of the tangent bundle for rectifiable sets. From now on, we will denote by $E$ a $d$-rectifiable set in $\R^n$ unless stated otherwise.

\begin{proposition}Let $\Sigma$ and $\Sigma^\prime$ be $d$-dimensional surfaces of class $C^1$. Then the tangent planes are equal at almost every point in the intersection $x \in \Sigma \cap \Sigma^\prime$, that is,
\begin{equation*} T_x \, \Sigma = T_x \, \Sigma^\prime. \end{equation*} \end{proposition}

The proof of this assertion is a mere consequence of the following Lemma since the set of all points $x \in \Sigma \cap \Sigma^\prime$ such that $\Sigma$ is transversal to $\Sigma^\prime$ (i.e., where the tangent planes do not coincide) is negligible for the Hausdorff/Lebesgue measure.

\begin{lemma}\label{lemma:sok1kdosd}Let $f, \, \widetilde{f} : \R^d \longrightarrow \R$ be functions of class $C^1(\R^d)$. Then
\begin{equation*} \nabla f(x) = \nabla \widetilde{f}(x) \end{equation*}
for $\mathcal{L}^d$-almost every point $x \in \R^d$ such that $f(x) = \widetilde{f}(x)$. \end{lemma}

\begin{proof} We may always assume without loss of generality that $\widetilde{f} \equiv 0$. Let $K$ denote the set of the zeros of $f$, and let $x \in K$ be a density point, that is,
\begin{equation*} \lim_{r \to 0} \frac{ \mathcal{H}^d \left(K \cap B(x, \, r) \right)}{\alpha_d \, r^d} = 1. \end{equation*}
Suppose that $\nabla f(x) \geq \epsilon > 0$. By continuity, there exist a positive constant $\delta_i > 0$ and a point $y_i \in \R^n$ such that $f(y_i) \neq 0$ and $y_i \in B(x, \, \delta_i)$. In particular, by choosing a suitable sequence $\delta_i \searrow 0$, it turns out that
\begin{equation*} B \left(y_i, \, \frac{\delta_i}{2} \right) \cap K = \varnothing. \end{equation*}
We are now ready to derive a contradiction. The Hausdorff density of $K$ at $x$ cannot be equal to one because there is a sequence of balls shrinking down to $x$ that does not intersect $K$. More precisely, it turns out that
\begin{equation*} \lim_{i \to + \infty} \frac{ \mathcal{H}^d \left(K^c \cap B(y_i, \, r_i) \right)}{\alpha_d \, r_i^d} > 0 \implies \lim_{r \to 0} \frac{ \mathcal{H}^d \left(K \cap B(x, \, r) \right)}{\alpha_d \, r^d} < 1. \end{equation*}
\end{proof}

\begin{definition}[Tangent Bundle] \index{tangent bundle} Let $E$ be a Borel $d$-rectifiable set. A map $T$ from $E$ to the Grassmannian manifold $G(n, \, d)$ that sends $x$ to $T(x)$ is a \textit{weak tangent bundle} for the set $E$ if and only if for every $\Sigma$ $d$-dimensional surface of class $C^1$ it turns out that
\begin{equation*} T_x \, \Sigma = T(x) \end{equation*}
for $\mathcal{H}^d$-almost every $x \in \Sigma \cap E$.\end{definition}

\begin{proposition} A $d$-rectifiable Borel set $E \subset \R^n$ admits a "unique" - up to $\mathcal{H}^d$ negligible sets - weak tangent bundle. \end{proposition}

\begin{proof}By \hyperref[prop:equivdlsd]{Proposition \ref{prop:equivdlsd}} there exists a collection of Borel subsets $\{E_i\}_{i \in \N}$ such that
\begin{equation*} E = \bigcup_{i = 0}^{+ \infty} E_i, \end{equation*}
satisfying the following properties: \mbox{}
\begin{enumerate}[label=\textbf{(\arabic*)}]
\item The set $E_0$ is $\mathcal{H}^d$-null.
\item For every $i \geq 1$ there is a $d$-dimensional surface $\Sigma_i \subset \R^n$ of class $C^1$ such that $E_i \subset \Sigma_i$.
\end{enumerate}
Therefore, we set
\begin{equation*} T(x) := \begin{cases} T_x \, \Sigma_1 & x \in E_1, \\ T_x \, \Sigma_2 & x \in E_2 \setminus E_1, \\ \dots, \end{cases} \end{equation*}
and, for every point
\begin{equation*} x \notin \bigcup_{i \geq 1} E_i, \end{equation*}
we set $T(x)$ to be equal to any $d$-dimensional vector space. The reader may easily check that $T$ is the sought map since it is unique up to the negligible set $\left(\bigcup_{i \geq 1} E_i \right)^c$. \end{proof}

\paragraph{Approximate Tangent.} In this brief paragraph, we introduce an even weaker notion called \textit{approximate tangent plane}. From now on, we shall assume that $E$ is a $d$-rectifiable Borel set, which is locally $\mathcal{H}^d$-finite, that is for every $p \in E$ there is an open neighborhood $U_p \ni p$ such that
\begin{equation*}\mathcal{H}^d(U_p) < + \infty. \end{equation*}

\begin{definition}[Cone] \index{cone} Let $\alpha$ be a fixed angle, let $x \in \R^n$ be a point and let $V$ be a $d$-dimensional plane in $\R^n$. The cone of angle $\alpha$ around $V$ centered at $x$ is defined by setting
\begin{equation*} \con := \left\{ x^\prime \in \R^n \: : \: |x^\prime - x| \cdot \sin(\alpha) \geq d(x - x^\prime, \, V) \right\}. \end{equation*}\end{definition}

\begin{definition}[Strong Tangent Plane] \index{strong tangent plane} Let $V \in G(n, \, d)$ be a $d$-dimensional plane. If $E$ is a Borel set and $x \in E$ a point, then $V$ is a \textit{strong tangent plane} to $E$ at $x$ if and only if for every $\alpha > 0$ there exists a positive radius $r_0 > 0$ such that
\begin{equation*}E \cap B(x, \, r) \subseteq \con \quad \text{for every $r \leq r_0$.} \end{equation*} \end{definition}

\begin{definition}[Approximate Tangent Plane] \index{approximate tangent plane} \label{def:tp1} Let $V \in G(n, \, d)$ be a $d$-dimensional plane. If $E$ is a Borel set and $x \in E$ a point, then $V$ is an \textit{approximate tangent plane} to $E$ at $x$ if and only if for every $\alpha > 0$ it turns out that
\begin{equation}\label{apt:eq1}\mathcal{H}^d \left( \left( E \cap B(x, \, r) \right) \setminus \con \right) = o(r^d),\end{equation}
and
\begin{equation}\label{apt:eq2} \mathcal{H}^d \left( \left( E \cap B(x, \, r) \right) \cap \con \right) \sim \alpha_d r^d. \end{equation}
\end{definition}

\begin{theorem}\label{th:dksdks} Let $E \subseteq \R^n$ be a Borel set. If $E$ is a $d$-rectifiable $\mathcal{H}^d$-locally finite set, then the weak tangent bundle $T(x)$ is the approximate tangent plane to $E$ at $x$ for $\mathcal{H}^d$-almost every $x \in E$. 
\end{theorem}

\begin{proof}We divide the argument into different steps to highlight what we are doing here.

\paragraph{Step 1.} First, observe that the thesis is equivalent to the following assertion: \textit{For every $i \geq 1$ the vector space $T_x \, \Sigma_i$ is the approximate tangent plane to $E$ at $x$ for $\mathcal{H}^d$-almost every $x \in E \cap \Sigma_i$}.

\paragraph{Step 2.} Let us define the measures
\begin{equation*} \begin{aligned} & \lambda := \mathbbm{1}_{\Sigma_i} \cdot \mathcal{H}^d, \\[1em]
& \mu^\prime := \mathbbm{1}_{\Sigma_i \setminus E} \cdot \mathcal{H}^d, \\[1em]
& \mu^{\prime\prime} := \mathbbm{1}_{E \setminus \Sigma_i} \cdot \mathcal{H}^d. \end{aligned} \end{equation*}
Fix $\alpha > 0$ and $x \in \Sigma_i$, and let $V := T_x \, \Sigma_i$ be the $d$-dimensional tangent plane. By construction, it turns out that
\begin{equation} \label{dcsdd} \begin{aligned}\mathcal{H}^d \left( \left( E \cap B(x, \, r) \right) \cap \con \right) & = \lambda \left(B(x, \, r) \cap \con \right) - \mu^\prime \left(B(x, \, r) \cap \con \right) +  \\[1em] & \dots + \mu^{\prime \prime} \left(B(x, \, r) \cap \con \right) \end{aligned} \end{equation}
since the restriction $\mathcal{H}^d \restr E$ is equal to $\lambda - \mu^\prime + \mu^{\prime \prime}$.

\paragraph{Step 3.} By definition, as $r$ goes to $0$
\begin{equation*} \lambda \left( B(x, \, r) \cap \con \right) \sim \mathcal{H}^d \left(E \cap B(x, \, r) \right), \end{equation*} 
and thus
\begin{equation*} \lambda \left( B(x, \, r) \cap \con \right) \sim \alpha_d r^d \quad \text{for $r$ sufficiently small}. \end{equation*}

\paragraph{Step 4.} We now claim that
\begin{equation*} \mu^\prime \left( B(x, \, r) \cap \con \right) = o(r^d) \end{equation*}
for $\mathcal{H}^d$-almost every $x \in E \cap \Sigma_i$. Indeed, we observe that
\begin{equation*} \mu^\prime \left( B(x, \, r) \cap \con \right) \leq \mathcal{H}^d \left( B(x, \, r) \cap \left(\Sigma_i \setminus E \right) \right),  \end{equation*}
and thus it suffices to show\footnote{This assertion is left as a simple exercise for the reader.} that the density of $\Sigma_i \setminus E$ is $0$ at almost every point of $E$ with respect to the measure $\lambda$ (which is, by the way, rather intuitive). Indeed, if
\begin{equation*} \Theta \left( \Sigma_i \setminus E, \, \lambda, \, x \right) = \frac{ \mathcal{H}^d \left( B(x, \, r) \cap \left(\Sigma_i \setminus E \right) \right)}{ \mathcal{H}^d \left( \Sigma_i \cap B(x, \, r) \right) } = 0, \end{equation*}
then the numerator must be an element of the class $o(r^d)$ since the denominator is $\sim \alpha_d r^d$.

\paragraph{Step 5.} In a similar fashion, we claim that
\begin{equation*} \mu^{\prime \prime} \left( B(x, \, r) \cap \con \right) = o(r^d) \end{equation*}
for $\mathcal{H}^d$-almost every $x \in E \cap \Sigma_i$. This is an easy consequence of the inequality
\begin{equation*} \mu^{\prime \prime} \left( B(x, \, r) \cap \con \right) \leq \mu^{\prime \prime} \left( B(x, \, r) \right). \end{equation*}
Indeed, the right-hand side is an element of the class $o(r^d)$ since $\mu^{\prime \prime}$ is orthogonal to $\lambda$ by construction, and thus the density with respect to $\lambda$ is zero for $\lambda$-almost every $x$. More precisely,
\begin{equation*} \frac{\mathrm{d} \mu^{\prime \prime}}{\mathrm{d} \lambda}(x) = \lim_{r \to 0} \frac{ \mathcal{H}^d \left( B(x, \, r) \right)}{ \lambda \left( B(x, \, r) \right) } = 0, \end{equation*}
and the denominator is $\sim \alpha_d r^d$ as in the previous step.

\paragraph{Step 6.} In the same spirit, the reader may prove that
\begin{equation*} \mathcal{H}^d \left( \left( E \cap B(x, \, r) \right) \setminus \con \right) = o(r^d) \end{equation*}
using the same decomposition introduced in \eqref{dcsdd}.
\end{proof}

\begin{exercise} Let $\Sigma$ be a line in $\R^2$, and let
\begin{equation*} E:= \Sigma \cup \left( \, \bigcup_{n \in \N} \partial B(x_n, \, r_n) \right), \end{equation*}
where $\{x_n\}_{n \in \N}$ is a dense countable collection of points in $\R^2$ and $(r_n)_{n \in \N}$ is a summable family of positive real numbers. Prove that: \mbox{}
\begin{enumerate}[label=\textbf{\arabic*)}]
\item The set $E$ is $d$-rectifiable.
\item The set $E$ is locally $\mathcal{H}^1$-finite.
\item For almost every $x \in \Sigma$ it turns out that
\begin{equation*} \mathcal{H}^1 \left( (E \setminus \Sigma) \cap B(x, \, r) \right) = o(r). \end{equation*}
\end{enumerate}
\end{exercise}

\begin{remark}According to the statement of \hyperref[th:dksdks]{Theorem \ref{th:dksdks}}, it might happen that the mass is distributed only on one side of the tangent plane (see \hyperref[fig:rec11]{Figure \ref{fig:rec11}}).

To prove that the statement can be improved (i.e., the situation described above cannot happen), we need to introduce a different definition of \textit{approximate tangent plane}, which is easier to deal with, and then prove that it is stronger than \hyperref[def:tp1]{Definition \ref{def:tp1}}.\end{remark}

\begin{figure}[h]
\centering
\includegraphics[width=13cm, height=10cm]{images/atpTGM1.png}
\caption{The mass is distributed only on one side of the approximate tangent plane.}
\label{fig:rec11}
\end{figure}

\begin{definition}[Approximate Tangent Plane] \label{def:tp2} Let $\psi_{x, \, r} : B(x, \, r) \longrightarrow B(0, \, 1)$ be the magnifying glass map, that is,
\begin{equation*} \psi_{x, \, r}(x^\prime) := \frac{x^\prime - x}{r}, \end{equation*}
and let $E_{x, \, r}$ be the image of $E$ via $\psi_{x, \, r}$. An element $V$ of the Grassmannian manifold $G(n, \, d)$ is an \textit{approximate tangent plane} to the set $E$ at the point $x$ if and only if
\begin{equation*} \mathbbm{1}_{E_{x, \, r}} \cdot \mathcal{H}^d \rightharpoonup \mathbbm{1}_V \cdot \mathcal{H}^d \end{equation*}
locally weakly-$\ast$, that is, 
\begin{equation*} \int_{E_{x, \, r}} \varphi(t) \, \mathrm{d}\mathcal{H}^d(t) \: \xrightarrow{r \to 0} \:  \int_{V}  \varphi(t) \, \mathrm{d} \mathcal{H}^d(t), \qquad \forall \, \varphi \in C_c^0(\R^d). \end{equation*}\end{definition}

\begin{remark} If $V$ is an approximate tangent plane to $E$ at $x$ in the sense of \hyperref[def:tp2]{Definition \ref{def:tp2}}, then $V$ is an approximate tangent plane to $E$ at $x$ in the sense of \hyperref[def:tp1]{Definition \ref{def:tp1}}. \end{remark}

\begin{proof} Let us consider the following measures:
\begin{equation*} \mu_{x, \, r} := \mathbbm{1}_{E_{x, \, r}} \cdot \mathcal{H}^d \qquad \text{and} \qquad \mu_x := \mathbbm{1}_{V} \cdot \mathcal{H}^d. \end{equation*}
The $d$-dimensional Hausdorff measure of $\partial B(0, \, 1) \cap V$ is equal to zero since $B(0, \, 1) \cap V$ is a $d$-dimensional set (which means that the boundary has dimension $d - 1$); hence
\begin{equation*} \mu_{x, \, r} \rightharpoonup \mu_x \implies \mu_{x, \, r} \left(B(0, \, 1) \right) \: \xrightarrow{r \to 0} \: \mu_x \left( B(0, \, 1) \right), \end{equation*} 
as a consequence of \hyperref[proposition:posme]{Proposition \ref{proposition:posme}}. By definition, we have $\mu_x \left(B(0, \, 1) \right) = \alpha_d$ and
\begin{equation*}\mu_{x, \, r} \left(B(0, \, 1) \right) = \mathcal{H}^d \left( E_{x, \, r} \cap B(0, \, 1) \right) = \frac{1}{r^d} \, \mathcal{H}^d \left(E \cap B(x, \, r) \right), \end{equation*}
which implies a condition weaker than \textbf{(b)}, that is
\begin{equation*}\mathcal{H}^d \left(E \cap B(x, \, r) \right) \sim \alpha_d r^d \quad \text{as $r \to 0^+$}. \end{equation*}
In a similar fashion, it turns out that
\begin{equation*}\mu_{x, \, r} \left(B(0, \, 1) \cap \mathcal{C}(0, \, V, \, \alpha) \right) \: \xrightarrow{r \to 0} \: \mu \left( B(0, \, 1) \cap \mathcal{C}(0, \, V, \, \alpha) \right) = \alpha_d, \end{equation*}
from which it follows that
\begin{equation*}\mathcal{H}^d \left(E \cap B(x, \, r) \cap \mathcal{C}(0, \, V, \, \alpha) \right) \sim \alpha_d r^d \quad \text{as $r \to 0^+$}, \end{equation*}
that is, the condition \textbf{(b)} holds. The reader may prove, in a similar way, that the condition \textbf{(a)} also holds true.
\end{proof}

\begin{theorem}\label{th:dksdks2} Let $E \subseteq \R^n$ be a Borel set. If $E$ is $d$-rectifiable and locally $\mathcal{H}^d$-finite, then $T(x)$ is the approximate tangent plane (\hyperref[def:tp2]{Definition \ref{def:tp2}}) to $E$ at $x$ for $\mathcal{H}^d$-almost every $x \in E$. 
\end{theorem}

\begin{proof}[Sketch of the Proof] We divide the argument into different steps to ease the notation.

\paragraph{Step 1.} First, observe that the thesis is equivalent to the following assertion: \textit{For every $i \geq 1$ the vector space $T_x \, \Sigma_i$ is the approximate tangent plane to $E$ at $x$ for $\mathcal{H}^d$-almost every $x \in E \cap \Sigma_i$}.

\paragraph{Step 2.} Fix $i \geq 1$ and let $\Sigma_{x, \, r}$ be the image of $\Sigma_i$ via $\psi_{x, \, r}$. We consider the measures
\begin{equation*} \begin{aligned} & \lambda_{x, \, r} := \mathbbm{1}_{\Sigma_{x, \, r}} \cdot \mathcal{H}^d, \\[1em]
& \mu_{x, \, r}^\prime := \mathbbm{1}_{\Sigma_{x, \, r} \setminus E_{x, \, r}} \cdot \mathcal{H}^d, \\[1em]
& \mu_{x, \, r}^{\prime\prime} := \mathbbm{1}_{E_{x, \, r} \setminus \Sigma_{x, \, r}} \cdot \mathcal{H}^d, \end{aligned} \end{equation*}
in such a way that the following decomposition holds:
\begin{equation*} \mu_{x, \, r} = \lambda_{x, \, r} - \mu_{x, \, r}^\prime + \mu_{x, \, r}^{\prime \prime}. \end{equation*}

\paragraph{Step 3.} The reader may prove that
\begin{equation*} \lambda_{x, \, r} \rightharpoonup \mathbbm{1}_V \cdot \mathcal{H}^d, \end{equation*}
locally weakly-$\ast$, using the fact that the projection $\pi_{x, \, r} : \sigma_{x, \, r} \longrightarrow V$ is an almost-isometry.

\paragraph{Step 4.} In a similar way to what we have done in the proof of \hyperref[th:dksdks]{Theorem \ref{th:dksdks}}, one can check that
\begin{equation*} \mu_{x, \, r}^\prime \rightharpoonup 0 \iff \mu_{x, \, r}^\prime \left (B_R \right) \xrightarrow{r \to 0} 0 \quad \text{for any fixed radius $R > 0$} \end{equation*}
as a consequence of \hyperref[proposition:posme]{Proposition \ref{proposition:posme}}. We now observe that
\begin{equation*} \mu_{x, \, r}^\prime \left( B(0, \, 1) \right) = \mathcal{H}^d \left( B(0, \,1) \cap \left( \Sigma_{x, \, r} \setminus E_{x, \, r} \right) \right) = \frac{1}{r^d} \mathcal{H}^d \left( B(x, \, r) \cap \left( \Sigma_i \setminus E \right) \right) \xrightarrow{ r \to 0} 0 \end{equation*}
since $ \mathcal{H}^d \left( B(x, \, r) \cap \left( \Sigma_i \setminus E \right) \right)$ is an element of the class $o(r^d)$, as proved in \hyperref[th:dksdks]{Theorem \ref{th:dksdks}}.

\paragraph{Step 5.} Similarly, by \hyperref[proposition:posme]{Proposition \ref{proposition:posme}} it follows that
\begin{equation*} \mu_{x, \, r}^{\prime \prime} \rightharpoonup 0 \iff \mu_{x, \, r}^{\prime \prime} \left (B_R \right) \xrightarrow{r \to 0} 0 \quad \text{for any fixed radius $R > 0$}. \end{equation*}
We now observe that
\begin{equation*} \mu_{x, \, r}^{\prime \prime} \left( B(0, \, 1) \right) = \mathcal{H}^d \left( B(0, \,1) \cap \left( E_{x, \, r} \setminus \Sigma_{x, \, r} \right) \right) = \frac{1}{r^d} \mathcal{H}^d \left( B(x, \, r) \cap \left( E\setminus \Sigma_i \right) \right), \end{equation*}
thus it suffices to prove that the right-hand side goes to zero as $r \to 0^+$. Indeed, the limit of the ratio is the Radon-Nikodym density, which is zero as a consequence of the fact that the two measures are orthogonal:
\begin{equation*} \frac{1}{r^d} \mathcal{H}^d \left( B(x, \, r) \cap \left( E\setminus \Sigma_i \right) \right) \xrightarrow{r \to 0} \frac{\mathrm{d} \left( \mathbbm{1}_{E \setminus \Sigma} \cdot \mathcal{H}^d \right)}{\mathrm{d} \left( \mathbbm{1}_{\Sigma} \cdot \mathcal{H}^d \right)} = 0. \end{equation*}
\end{proof}

\section{Rectifiability Criteria}

The primary goal of this section is to relate $d$-rectifiable set with properties of the approximate tangent plane. More precisely, in this section we denote by $V \in G(n, \, d)$ a $d$-dimensional plane, by $E \subseteq \R^n$ a Borel set, and we fix a point $x \in \R^n$.

\begin{definition}[Tangent Cone] \index{tangent cone} Fix $\alpha \in \left(0, \, \pi/2 \right)$. The cone $\mathcal{C}(x, \, V, \, \alpha)$ is \textit{tangent} to a set $E$ at $x$ if there exists a positive constant $r_0 > 0$ such that
\begin{equation*} B(x, \, r) \cap E \subseteq \mathcal{C}(x, \, V, \, \alpha), \qquad \forall \, r \leq r_0. \end{equation*} \end{definition}

\begin{theorem}Assume that $E$ admits a tangent cone at every point $x \in E$. Then there are (at most) countably many Lipschitz functions $f_i : \R^d \longrightarrow \R^n$ such that
\begin{equation*}E \subseteq \bigcup_{i \geq 1} f_i(\R^d).  \end{equation*}
In particular, the set $E$ is $d$-rectifiable. \end{theorem} %STEP 1

\begin{proof}We first prove a particular case, and then we reduce the general case to it using a standard argument.

\paragraph{Step 1.} Let us assume\footnote{We will get rid of these extra assumptions later. It is important to understand that the argument presented in the first step only proves a particular case.} that: \mbox{}
\begin{enumerate}[label=\alph*)]
\item The $d$-dimensional plane $V$, the angle $\alpha$ and the radius $r_0$ do not depend on the point $x \in E$.
\item The linear space $V$ is a straight $d$-dimensional plane.
\end{enumerate}
We divide the ambient space in strips of thickness $r_0 \sin \alpha$, that is,
\begin{equation*} \R^n = \bigcup_{i \in \Z} S_i, \quad \text{with $|S_i| = r_0 \sin \alpha$}. \end{equation*}
The set $E$ is thus given by the union
\begin{equation*}E = \bigcup_{i \in \Z} \left(E \cap S_i \right),\end{equation*}
which means that it is enough to prove that the intersection of $E$ with each strip $S_i$ is contained in the graph of a Lipschitz map, that is,
\begin{equation*} E \cap S_i \subset \Gamma(g_i),\end{equation*}
where $g_i : V \longrightarrow V^\perp$ is Lipschitz\footnote{The graph of the function $g_i$ is a subset of $\R^n$ since $g_i$ sends $V$ to its orthogonal, and $V \oplus V^\perp = \R^n$.}. By assumption, it turns out that (see \hyperref[fig:12222]{Figure \ref{fig:12222}}) the intersection of $E$ with each strip is contained in the cone centered at any $x \in E \cap S_i$, that is,
\begin{equation*} E \cap S_i \subset \mathcal{C}(x, \, V, \, \alpha) \qquad \forall \, x \in E \cap S_i. \end{equation*}
Therefore, if we denote by $(x, \, y)$ the coordinates associated with the decomposition $V \oplus V^\perp = \R^n$, then one can prove that
\begin{equation*} |y^\prime - y| \leq \tan (\alpha) \cdot |x^\prime - x|, \end{equation*}
which means that the $y$ coordinate furnishes a Lipschitz map (see \hyperref[fig:133]{Figure \ref{fig:133}}).

\paragraph{Step 2.} We are now ready to get rid of the extra assumptions, one by one. \mbox{}
\begin{enumerate}[label=\textbf{(\roman*)}]
\item Assume that $\alpha$ and $r_0$ do not depend on the point $x \in E$, and fix an angle $\alpha^\prime > \alpha$.

The cones of the form $\mathcal{C}(0, \, V, \, \alpha)$, as $V$ ranges in the set of all admissible $d$-dimensional planes, are contained in a finite family $C(0, \, V_i, \, \alpha^\prime)$ of slightly ampler cones. We can reduce to the first step by splitting $E$ in the finite union associated to the family above, that is,
\begin{equation*} E = \bigcup_{i = 1}^{N} E_i, \end{equation*}
with $V_i$ fixed $d$-dimensional straight plane for each $i = 1, \, \dots, \, N$.
\item Assume that the radius $r_0$ does not depend on the point $x \in E$.

For every $\alpha \in \Q$ there exists $\alpha^\prime(\alpha) > \alpha$ such that the cones of the form $\mathcal{C}(0, \, V, \, \alpha)$, as $V$ ranges in the set of all admissible $d$-dimensional planes, are contained in a given finite family $C(0, \, V_i, \, \alpha^\prime(\alpha))$ of slightly ampler cones. We can reduce to the first step by splitting $E$ in the (at most) countable union
\begin{equation*} E = \bigcup_{\alpha \in \Q} \left[ \, \bigcup_{i = 1}^{N_{\alpha^\prime(\alpha)}} E_i \right]. \end{equation*}
\item Finally, if no extra assumption holds, then it suffices to consider the collection
\begin{equation*} E_r := \left\{ x \in E \: \left| \: r_0(x) < r \right. \right\} \end{equation*}
and split $E$ as follows
\begin{equation*} E = \bigcup_{r, \, \alpha \in \Q} \left[ \, \bigcup_{i = 1}^{N_{r, \, \alpha^\prime(\alpha)}} E_{r, \, i} \right]. \end{equation*}
\end{enumerate}
\end{proof}

\begin{figure}[h]
\centering
\includegraphics[width=16cm, height=9cm]{images/atpTGM2.png}
\caption{Sketch of the first step.}
\label{fig:12222}
\end{figure}

\begin{figure}[h]
\centering
\includegraphics[width=16cm, height=9cm]{images/atpTGM3.png}
\label{fig:133}
\caption{The coordinates give a Lipschitz map.}
\end{figure}

\begin{definition}[Approximate Tangent Cone] \index{approximate tangent cone} Fix $\alpha \in \left(0, \, \pi/2 \right)$. The cone $\mathcal{C}(x, \, V, \, \alpha)$ is a \textit{approximately tangent} to a set $E$ at $x$ if and only if
\begin{equation*} \mathcal{H}^d \left( \left( B(x, \, r) \cap E \right) \setminus \mathcal{C}(x, \, V, \, \alpha) \right) = o(r^d). \end{equation*} \end{definition}

\begin{theorem}\label{th:91dsl,2peld}Assume that $E$ admits an approximate tangent cone at every point $x \in E$, and assume also that the lower density is bounded from below, that is,
\begin{equation*} \Theta_\ast(E, \, x) > 0 \qquad \forall \, x \in E. \end{equation*}
Then there are (at most) countably many Lipschitz functions $f_i : \R^d \longrightarrow \R^n$ such that
\begin{equation*}E \subseteq \bigcup_{i \geq 1} f_i(\R^d).  \end{equation*}
In particular, the set $E$ is $d$-rectifiable. \end{theorem}

\begin{corollary} \label{crthasjdkso} If $E$ admits an approximate tangent cone at $\mathcal{H}^d$-almost every point $x \in E$, and the lower density is bounded from below, i.e.,
\begin{equation*} \Theta_\ast(E, \, x) > 0 \qquad \text{for $\mathcal{H}^d$-almost every point $x \in E$}, \end{equation*}
then the set $E$ is $d$-rectifiable. \end{corollary}

\begin{proof}[Proof of Theorem \ref{th:91dsl,2peld}] We first prove a particular case, and then we reduce the general case to it using a standard argument.

\paragraph{Step 1.1.} Let us assume that: \mbox{}
\begin{enumerate}[label=\alph*)]
\item The $d$-dimensional plane $V$ and the angle $\alpha$ are the same for every points $x \in E$.
\item There exist $r_0 > 0$ and $\delta > 0$ such that for every $x \in E$ it turns out that
\begin{equation}\label{sc:1} \mathcal{H}^d \left(E \cap B(x, \, r) \right) \geq \delta \cdot r^d, \end{equation} 
for every $r \leq r_0$.
\end{enumerate}
First, we notice that the existence of an approximate tangent plane implies that for every $r \leq r_0$ it turns out that
\begin{equation} \label{sc:2}\mathcal{H}^d \left( \left(E \cap B(x, \, r) \right) \setminus \con \right) \leq \delta^\prime \cdot r^d, \end{equation}
where $\delta^\prime$ is a positive constant that can be chosen arbitrarily. To prove the particular case, we first need to show that there exists a radius $\overline{r} > 0$ such that
\begin{equation*} E \cap \left( V \times B\left(x_0, \, \overline{r} \right) \right) \subset \Gamma(g), \end{equation*}
where $x_0 \in V^\perp$ and $\Gamma(g)$ is the graph of a Lipschitz function $g : V \longrightarrow V^\perp$.

\paragraph{Step 1.2.} Fix an angle $\alpha^\prime \in (\alpha, \, \pi/2)$, and fix a point $x \in E$. We claim that
\begin{equation*} E \cap B\left(x_0, \, \frac{r_0}{2} \right) \subseteq \mathcal{C} \left(x, \, V, \, \alpha^\prime \right). \end{equation*}
We argue by contradiction. Suppose that there exists a point
\begin{equation*} x^\prime \in \left( E \cap B\left(x_0, \, \frac{r_0}{2} \right) \right) \setminus \mathcal{C} \left(x, \, V, \, \alpha^\prime \right), \end{equation*}
and let $r, \, r^\prime > 0$ be positive real numbers such that
\begin{equation*}B(x^\prime, \, r^\prime) \subset B(x, \, r) \qquad \text{and} \qquad B(x^\prime, \, r^\prime) \cap \mathcal{C}(x, \, V, \, \alpha) = \varnothing. \end{equation*}
More precisely, let us consider the following radii:
\begin{equation*} r := 2 \, |x - x^\prime| \qquad \text{and} \qquad r^\prime := |x - x^\prime| \, \sin(\alpha^\prime - \alpha). \end{equation*}
The assumptions \eqref{sc:1} and \eqref{sc:2} proves that
\begin{equation*} \delta^\prime \cdot r^d \geq \mathcal{H}^d \left( \left(E \cap B(x, \, r) \right) \setminus \con \right) \geq \mathcal{H}^d \left( E \cap B(x^\prime, \, r^\prime) \right) \geq \delta \cdot (r^\prime)^d, \end{equation*}
from which we can derive a contradiction by fixing $\delta^\prime$ such that
\begin{equation*} \delta^\prime < \delta \cdot \left( \frac{\sin(\alpha^\prime - \alpha)}{2} \right)^d. \end{equation*}

\paragraph{Step 1.3.} In this final step, the goal is to clean the proof of the particular case by fixing the values of $\delta^\prime$ and $\alpha^\prime$ in such a way that the contradiction above holds. In particular, let us consider
\begin{equation*} \alpha^\prime = \frac{2 \, \alpha + \pi}{4}, \end{equation*}
and
\begin{equation*} \delta^\prime = \frac{\delta}{2} \cdot \left( \frac{1}{2} \sin \left( \frac{\pi - 2 \, \alpha}{4} \right) \right)^d. \end{equation*}
In the previous step, we proved that
\begin{equation*} E \cap B\left(x_0, \, \frac{r_0}{2} \right) \subseteq \mathcal{C} \left(x, \, V, \, \alpha^\prime \right), \end{equation*}
and this is enough to infer that
\begin{equation*} E \cap \left(V \times B\left(x_0, \, \overline{r} \right)\right) \subseteq \Gamma(g),\end{equation*}
where $g : V \longrightarrow V^\perp$ is a Lipschitz function of constant $\tan (\alpha^\prime)$, provided that the thickness of the cylinder strips $2 \overline{r}$ is less or equal than $(r_0/2) \cdot \sin(\alpha^\prime)$, that is,
\begin{equation*} \overline{r} \leq \frac{r_0}{4} \sin \left( \frac{\pi + 2 \alpha}{4} \right). \end{equation*}

\paragraph{Step 2.} In the general case, it is enough to consider the countable covering of $E$ defined by setting
\begin{equation*} E_n := \left\{ x \in E \: \left| \: \text{\eqref{sc:1} and \eqref{sc:2} holds for $r = \delta = \frac{1}{n}$} \right. \right\}. \end{equation*}

\end{proof}

\begin{corollary}If $E$ admits a tangent plane at every point $x \in E$, then the set $E$ is $d$-rectifiable. \end{corollary}

\section{Area Formula for Rectifiable Sets}

\paragraph{Summary.} Let $\Sigma$ be a $d$-dimensional surface of class $C^1$, and let $f : \Omega \longrightarrow \Sigma$ be a Lipschitz function defined on an open subset $\Omega \subseteq \R^d$. In \hyperref[sec:aflm]{Section \ref{sec:aflm}} we proved that for every Borel set $E \subseteq \Omega$ it turns out that
\begin{equation} \label{areaformula:s1} \int_{y \in \Sigma} m_{f, \, E}(y) \, \mathrm{d} \mathcal{H}^d(y) = \int_{E} Jf(x) \, \mathrm{d}\mathcal{L}^d(x), \end{equation}
where $m_{f, \, E}(y) := \symbol{35} \left(f^{-1}(y) \cap E \right)$ and $Jf(x)$ is the Jacobian of $f$ at $x$.

The formula presented here may be extend with little efforts to a more general setting. More precisely, let $\Sigma^\prime$ be a $d$-dimensional surface of class $C^1$, and let $f : \Sigma^\prime \longrightarrow \Sigma$ be a Lipschitz function. The (area) formula
\begin{equation} \label{areaformula:s2}  \int_{\Sigma} \left[ \sum_{s \in f^{-1}(x)} h(s)\right] \, \mathrm{d} \mathcal{H}^d(x) = \int_{\Sigma^\prime} h(x) Jf(x) \, \mathrm{d}\mathcal{L}^d \end{equation}
holds, and the proof is very similar\footnote{The reader may try to fill in the details as an exercise. Indeed, it is enough to take a compatible atlas for the surface $\Sigma^\prime$ and use the previous result \eqref{areaformula:s1} on each chart $\varphi : U \xrightarrow{\: \sim \:} V$. To conclude, the reader may use a suitable partition of unity and prove that the formula glues as expected.} to the one presented in \hyperref[sec:aflm]{Section \ref{sec:aflm}}.

\paragraph{Rectifiable Sets.} Let $E \subset \R^n$ be a $d$-rectifiable set, and let $f : E \longrightarrow f(E)$ be a Lipschitz function\footnote{The codomain is not important since the image of $E$ via $f$ is always a $d$-rectifiable set.}. In this paragraph, we want to prove that for every Borel set $F \subseteq f(E)$ it turns out that
\begin{equation} \label{areaformula:rec} \int_{y \in F} m_{f, \, E}(y) \, \mathrm{d} \mathcal{H}^d(y) = \int_{f^{-1}(F)} Jf(x) \, \mathrm{d}\mathcal{L}^d(x), \end{equation}
where $Jf(x)$ denotes the Jacobian of $f$ at $x$, that is
\begin{equation} \label{jac:rec} Jf(x) := \sqrt{ \mathrm{det} \left(\nabla_t f(x) \right)^T \left(\nabla_t f(x) \right)}. \end{equation}

\begin{definition}[Tangent Differential] The function $f : E \longrightarrow X \subseteq \R^m$ is tangentially differentiable at $\mathcal{H}^d$-almost every $x \in E$ if and only if there exists a linear map $L : T_x E \longrightarrow \R^m$ such that
\begin{equation*} f(x^\prime) = f(x) + L \left( \pi (x - x^\prime) \right) + o(|x - x^\prime|) \end{equation*}
for every $x^\prime \in E$, where $\pi$ is the projection onto the tangent space $T_x E$.
\end{definition} 

\begin{remark}If we consider a Lipschitz extension $\widetilde{f} : \R^n \longrightarrow \R^m$ of $f$, then $f$ is tangentially differentiable at $x \in E$ if and only if there exists a linear map $L : \R^n \longrightarrow \R^m$ such that
\begin{equation*} \widetilde{f}(x + h) = f(x) + Lh + o(|h|) \qquad \forall \, h \in T_x E. \end{equation*}  \end{remark}

\paragraph{Proof of Area Formula.} Let $\{E_i\}_{i \geq 0}$ be given by the equivalent definition of rectifiable set (see \hyperref[prop:equivdlsd]{Proposition \ref{prop:equivdlsd}}), that is
\begin{equation*} E = \bigcup_{i = 0}^{+ \infty} E_i, \end{equation*}
and the collection satisfies the following properties: \mbox{}
\begin{enumerate}[label=\textbf{(\alph*)}]
\item The $d$-dimensional Hausdorff measure of $E_0$ is zero.
\item The set $E_i$ is contained in a surface of class $C^1$, denoted by $\Sigma_i$, for every $i \geq 1$.
\item The restriction $f \, \big|_{E_i}$ may be extended to a differentiable function $F_i : \Sigma_i \longrightarrow \R^m$.
\item The set $E_i$ is equal to the disjoint union of $E_i^\prime$ and $E_i^{\prime \prime}$, where $E_i^\prime$ is the set of points $x \in E_i$ where the rank of $\mathrm{d}f_x$ is not maximal, and $E_i^{\prime \prime}$ is the set of points where it is maximal.
\end{enumerate}
For every $i \geq 1$, the function $F_i : \Sigma_i \longrightarrow \Sigma_i^\prime$ is differentiable, and thus the area formula holds for every $x \in F_i^{-1}\left( \Sigma_i^\prime \right) \cap E_i^{\prime \prime}$:
\begin{equation} \label{areaformula:recs2}  \int_{\Sigma_i^\prime} \left[ \sum_{s \in F_i^{-1}(x) \cap S_i^{\prime \prime}} h(s)\right] \, \mathrm{d} \mathcal{H}^d(x) = \int_{S_i^{\prime\prime}} h(x)  Jf(x) \, \mathrm{d}\mathcal{L}^d(x). \end{equation}
In a similar fashion, the area formula also holds for every $x \in F_i^{-1}\left( \Sigma_i^\prime \right) \cap E_i^\prime$ since the Jacobian is zero at every point, and the image of a $\mathcal{H}^d$-null set via a differentiable function is also $\mathcal{H}^d$-null.

\begin{remark}Recall that, if two Lipschitz functions agree on a subset, then they have the same tangential gradient at almost every $x$ in the intersection (see \hyperref[lemma:sok1kdosd]{Lemma \ref{lemma:sok1kdosd}}).  \end{remark}

The area formula introduced in this final section can be slightly improved. Indeed, we do not need to consider a Lipschitz function $f$, but it is enough to require the following properties: \mbox{}
\begin{enumerate}[label=\textbf{(\roman*)}]
\item The function $f$ is differentiable almost everywhere.
\item The function $f$ sends $\mathcal{H}^d$-null sets into $\mathcal{H}^d$-null sets.
\end{enumerate}

\begin{exercise}Prove that for every $d \in (0, \, + \infty)$ there exists a set $E$ such that $\mathcal{H}^d(E)$ is finite and nonzero, but the lower density is zero:
\begin{equation*} \Theta_\ast(E, \, x)= 0 \qquad \text{for $\mathcal{H}^d$-almost every $x \in E$}. \end{equation*} \end{exercise}
\chapter{Bounded Variation Functions}

In this brief chapter, we introduce the notion of \textit{bounded variation} functions, and we investigate some of the main properties (extension, approximation, compactness, trace, etc.)

\begin{notation}In this chapter, we denote by $| \cdot |$ the Lebesgue measure $\mathcal{L}^n(\cdot)$, and we denote by $\mathrm{d}x$ the differential $\mathcal{d}\mathcal{L}^n(x)$, unless it is necessary to indicate the dimension. \end{notation}

\section{Definition and Elementary Properties}

\begin{definition}[Bounded Variation]\index{bounded variation}Let $\Omega \subseteq \R^n$ be an open set. A function $u : \Omega \longrightarrow \R$ is of bounded variation on $\Omega$, and we denote it by $u \in \mathrm{BV}(\Omega)$, if and only if the following properties hold:\mbox{}
\begin{enumerate}[label=\textbf{(\roman*)}]
\item The function $u$ belongs to $L^1(\Omega)$.
\item There exists a vector-valued measure $\mu =(\mu_1, \, \dots, \, \mu_n) \in \mathcal{M} \left( \Omega; \; \R^n \right)$ such that
\begin{equation} \label{bv:def1} \int_{\Omega} \frac{\partial \varphi}{\partial x_i}(x) u(x) \, \mathrm{d}x = - \int_{\Omega} \varphi(x) \, \mathrm{d}\mu_i(x), \qquad \forall \, \varphi \in C_c^\infty(\Omega)\end{equation}
for every $i = 1, \, \dots, \, n$.
\end{enumerate}  \end{definition}

\begin{remark}Let $u \in L^1(\Omega)$ be a summable function, and let us denote by $Du$ the weak derivative (in the sense of distributions). Then the condition \eqref{bv:def1} is equivalent to requiring that $u$ is weakly differentiable and $Du$ is $\R^n$-valued measure. \end{remark}

\begin{remark}In a similar fashion, the condition \eqref{bv:def1} is equivalent to the existence of a positive real-valued measure $\lambda$ and a vector-valued function $\tau : \Omega \longrightarrow \R^n$ such that
\begin{equation} \label{bv:def2} \int_{\Omega} \frac{\partial \varphi}{\partial x_i}(x) u(x) \, \mathrm{d}x = - \int_{\Omega} \varphi(x) \tau_i(x) \, \mathrm{d}\lambda(x), \qquad \forall \, \varphi \in C_c^\infty(\Omega)\end{equation}
for every $i = 1, \, \dots, \, n$.\end{remark}

There is an equivalent definition of the set $\mathrm{BV}(\Omega)$ that relies more on the vector-valued structure of $\R^n$ (that is, it is not equivalent in a generic metric space.)

\begin{definition}[Bounded Variation]\index{bounded variation}Let $\Omega \subseteq \R^n$ be an open set. A function $u : \Omega \longrightarrow \R$ is of bounded variation on $\Omega$, and we denote it by $u \in \mathrm{BV}(\Omega)$, if and only if the following properties hold:\mbox{}
\begin{enumerate}[label=\textbf{(\roman*)}]
\item The function $u$ belongs to $L^1(\Omega)$.
\item There exists a vector-valued measure $\mu \in \mathcal{M} \left( \Omega; \; \R^n \right)$ such that
\begin{equation} \label{bv:def3} \int_{\Omega} \mathrm{div} \left( \varphi \right)(x) u(x) \, \mathrm{d}x = - \int_{\Omega} \varphi(x) \, \mathrm{d}\mu(x), \qquad \forall \, \varphi \in C_c^\infty(\Omega)\end{equation}
for every $i = 1, \, \dots, \, n$.
\end{enumerate}  \end{definition}

\begin{remark}In a similar fashion, the condition \eqref{bv:def3} is equivalent to the existence of a positive real-valued measure $\lambda$ and a vector-valued function $\tau : \Omega \longrightarrow \R^n$ such that
\begin{equation} \label{bv:def4} \int_{\Omega} \mathrm{div} \left( \varphi \right) (x) u(x) \, \mathrm{d}x = - \int_{\Omega} \varphi(x) \cdot \tau(x) \, \mathrm{d} \lambda(x), \qquad \forall \, \varphi \in C_c^\infty(\Omega)\end{equation}
for every $i = 1, \, \dots, \, n$, where $\cdot$ denotes the scalar product between two $\R^n$ vectors.\end{remark}

\begin{exercise}Let $\Omega \subseteq \R^n$. A function $u$ satisfies \eqref{bv:def1} if and only if it satisfies \eqref{bv:def3}, i.e., the given definitions are actually equivalent.\end{exercise}

\begin{remark}The space $\mathrm{BV}(\Omega)$, endowed with the norm
\begin{equation} \label{bv:norm} \| u \|_{\mathrm{BV}(\Omega)} = \| u \|_{L^1(\Omega)} + \left|Du \right|(\Omega), \end{equation}
is a non-separable Banach space. \end{remark}

\begin{proof}The space $\left(\mathrm{BV}(\Omega), \, \| \cdot \|_{\mathrm{BV}(\Omega)} \right)$ is clearly a Banach space, as it can be checked quickly by the reader; thus it is enough to prove that it is not separable.

We prove that $\mathrm{BV}\left( [0, \, 1] \right)$ is not separable since the general case follows from a similar argument. For every $\alpha \in (0, \, 1)$ let us consider the characteristic function
\begin{equation*} \mathbbm{1}_\alpha := \mathbbm{1}_{[\alpha, \, 1]}. \end{equation*}
For every choice $\alpha \neq \beta \in [0, \, 1]$, it turns out that
\begin{equation*}\| \mathbbm{1}_\alpha - \mathbbm{1}_\beta \|_{\mathrm{BV}([0, \, 1])} = 2 + |\alpha - \beta|,  \end{equation*}
and hence we can consider the family of balls
\begin{equation*}B_\alpha := \left\{ \psi \in \mathrm{BV}\left( [0, \, 1] \right) \: \left| \: \| \psi - \mathbbm{1}_\alpha \|_{\mathrm{BV}\left( [0, \, 1] \right)} \leq 1 \right. \right\}. \end{equation*}
This collection of balls is indexed by the interval $[0, \, 1]$, which means that it is a family with the cardinality of the continuum. Therefore we can infer that $\mathrm{BV}\left( [0, \, 1] \right)$ is not separable since each dense subset must intersect every ball $B_\alpha$ in at least one point, which means that any dense subset has a cardinality bigger or equal than the continuum. \end{proof}

\begin{remark} The measure $\mu$ given by \eqref{bv:def1} is unique (since the weak derivative in the sense of distribution is unique, up to set of measure zero). We shall denote it by $Du$ through the entire chapter since the symbol $\nabla u$ is reserved for the classical/pointwise gradient. \end{remark}

\begin{lemma}Let $u \in L^1(\Omega)$ be a summable function. Then $u$ belongs to $\mathrm{BV}(\Omega)$ if and only if the induced linear functional, defined by
\begin{equation*} \Lambda_u : C_c^\infty(\Omega) \longrightarrow \R, \qquad \varphi \longmapsto \int_\Omega \mathrm{div} \left( \varphi \right)(x) u(x) \, \mathrm{d}x, \end{equation*}
is bounded with respect to the uniform norm $\| \cdot \|_\infty$. \end{lemma}

\begin{proof} \mbox{}

\paragraph{$"\implies"$} Assume that $u \in \mathrm{BV}(\Omega)$. The absolute value of the functional can be easily estimated using the identity \eqref{bv:def3} as follows:
\begin{equation*} \left| \int_\Omega \mathrm{div}\left(\varphi \right)(x) \, u(x) \, \mathrm{d}x \right| = \left| - \int_\Omega \varphi(x) \, \mathrm{d} \mu(x) \right| \leq \| \varphi \|_{\infty} |\mu|(\Omega), \end{equation*}
where $|\mu|(\Omega)$ denotes the total variation. In particular, the linear functional $\Lambda_u$ is bounded with respect to the uniform norm, and thus it can be extended up to the closure of the domain, that is,
\begin{equation*} \Lambda_u : \overline{C_c^\infty(\Omega)} = C_0^0(\Omega) \longrightarrow \R, \end{equation*}
where $C_0^0(\Omega)$ denotes the set of all continuous functions on $\Omega$ that are infinitesimal on the boundary $\partial \Omega$. Notice that the identity \eqref{bv:def3} holds for every function $\varphi \in C_0^1(\Omega)$ since the divergence operator is not defined on the set of all continuous functions.

\paragraph{$"\impliedby"$} Vice versa, assume that $u \in L^1(\Omega)$ and $\Lambda_u$ is a bounded functional with respect to the uniform norm. From the \hyperref[RieszTheorem]{Riesz Representation Theorem \ref{RieszTheorem}} one can find a measure $\mu \in \mathcal{M}\left(\Omega; \; \R^n \right)$ such that
\begin{equation*}\int_{\Omega} \mathrm{div}(\varphi)(x) \, u(x) \, \mathrm{d}x =: \Lambda_u(\varphi) = \int_{\Omega} \varphi(x) \, \mathrm{d}\mu(x). \end{equation*}
In particular, the function $u$ satisfies the property \eqref{bv:def3}, that is $u$ is weakly differentiable (in the sense of distributions) and its weak derivative is equal to $- \mathrm{d}\mu$, accordingly with the definition of bounded variation functions.
\end{proof}

\begin{example} \label{ex:bvfps} \mbox{}
\begin{enumerate}[label=\textbf{\arabic*)}]
\item The Sobolev space $W^{1, \, 1}(\Omega)$ is contained in $\mathrm{BV}(\Omega)$ for every bounded set $\Omega \subset \R^n$.

Indeed, let $f \in W^{1, \, 1}(\Omega)$ be a Sobolev function, and denote by $g \in L^1(\Omega)$ its weak derivative. It suffices to show that the distribution associated with $g$ is a vector-valued measure in $\mathcal{M} \left( \Omega; \; \R^n \right)$ satisfying \eqref{bv:def3} holds. By definition of weak derivative
\begin{equation*} \int_{\Omega} \mathrm{div} \left( \varphi \right)(x) u(x) \, \mathrm{d}  \mathcal{L}^n(x) = - \int_{\Omega} \varphi(x) g(x) \, \mathrm{d}\mathcal{L}^n(x), \qquad \forall \, \varphi \in C_c^\infty(\Omega),\end{equation*}
thus it is enough to set $\mu := g \cdot \mathcal{L}^n$, which is well defined because $g$ is a summable function.

\item If $E \subseteq \Omega$ is an open bounded set with differentiable ($C^1$ is enough) boundary, then the characteristic function $\mathbbm{1}_E$ belongs to $\mathrm{BV}\left( \R^n \right)$.

The distributional derivative is zero on both $\mathrm{Int}(E)$ and $\Omega \setminus \overline{E}$; hence it is a measure supported on the boundary $\partial E$. More precisely, one can check that
\begin{equation*} D \mathbbm{1}_E = \nu_{i} \, \mathbbm{1}_{\partial E} \cdot \mathcal{H}^{n-1}, \end{equation*}
where $\nu_{i}$ denotes the inner normal vector. Indeed, for any $\varphi \in C_c^\infty\left(\R^n; \; \R^n \right)$ it turns out that
\begin{equation*} \int_{\R^n} \mathrm{div}(\varphi)(x) \mathbbm{1}_E(x) \, \mathrm{d}x = \int_{E} \mathrm{div}(\varphi)(x) \, \mathrm{d}x \: {\color{red}=} \: - \int_{\partial E} \varphi(x) \nu_i(x) \, \mathrm{d}\mathcal{H}^{n-1}(x), \end{equation*}
where the {\color{red}red} identity follows from the divergence theorem.
\end{enumerate}
\end{example}

\begin{remark} The characteristic $\mathbbm{1}_E$ introduced above is the first example of a function of bounded variation on $\R^n$ which is not also Sobolev function. Moreover, it turns out that
\begin{equation*}\sup_{\|\varphi\| \leq 1} \left| \Lambda_{\mathbbm{1}_E}(\varphi) \right| \leq \mathcal{H}^{n-1}(\partial E). \end{equation*}
We will see (in the next chapter) that the inequality above is, for every smooth set $E$, an equality:
\begin{equation*} \left\| \Lambda_{\mathbbm{1}_E}(\varphi) \right\|_{\ast} = \mathcal{H}^{n-1}(\partial E). \end{equation*}
\end{remark}

\section{Alternative Definitions: Bounded Variation on the Real Line}

In this brief section, we want to discuss a different definition of the space $\mathrm{BV}\left([a, \, b]; \; \R \right)$, which is still used in some fields of mathematics (as partial differential equations evolution theory).

\begin{definition}[Bounded Variation, II] \index{bounded variation!alternative definition} A function $u$ is of bounded variation on $[a, \, b] \subseteq \R$ if and only if the \textit{total variation} is finite, i.e.
\begin{equation} \label{bv:defvar} T_v(u) := \sup_{P \in \mathcal{P}} \sum_{i = 0}^{n-1} \left| u(x_{i+1}) - u(x_i) \right| < + \infty, \end{equation} 
where $\mathcal{P}$ is the set of all finite partitions of the interval $[a, \, b]$.\end{definition}

\begin{definition}[Essential Variation] A function $u$ is of bounded essential variation on $[a, \, b]$, and we denote it by $u \in \mathrm{BV}_{\mathrm{ess}}\left([a, \, b] \right)$, if and only if 
\begin{equation} \label{bv:defvar2} T_{v_{ess}}(u) := \sup_{P \in \mathcal{P}_{ess}} \sum_{i = 0}^{n-1} \left| u(x_{i+1}) - u(x_i) \right| < + \infty, \end{equation} 
where $\mathcal{P}_{ess}$ is the set of all finite partitions of the interval $[a, \, b]$ with extremal points $x_i$ such that $u$ is $L^1$-approximate continuous at $x_i$ for every $i$.\end{definition}

\begin{remark}The essential total variation \eqref{bv:defvar2} is closely related to the total variation introduced in \eqref{bv:defvar} since, as one can easily check, it turns out that
\begin{equation*} T_{v_{ess}} (u) = \inf_{ \text{$w(x) = u(x)$ a.e.}} T_v(w). \end{equation*} \end{remark}

The next result shows that the definition of bounded variation introduced in this section is related to the one we will be dealing with in this course through the finiteness of the essential variation.

\begin{theorem}\label{th:bridge}A function $u$ belongs to $\mathrm{BV}\left((a, \, b) \right)$ (in the sense of definition \eqref{bv:def1}) if and only if $u \in L^1( (a, \, b) )$ and $T_{v_{ess}}(u) < + \infty$. \end{theorem}

\begin{lemma}\label{lemma:bridge} Let $u \in \mathrm{BV}\left((a, \, b) \right)$. There exists a constant\footnote{In the proof of \hyperref[th:bridge]{Theorem \ref{th:bridge}} is not necessary to know the exact value of $c$, but the reader may try to prove that it is equal to the trace.} $c \in \R$ such that
\begin{equation} \label{eq:charbv} u(x) = c + \int_{a}^{x} \, \mathrm{d}\mu(x) = c + \mu \left( (a, \, x) \right) \end{equation}
for $\mathcal{L}^1$-almost every $x \in (a, \, b)$.\end{lemma}

\begin{proof}It is enough to notice that the identity \eqref{eq:charbv} is equivalent to the following identity:
\begin{equation} \label{eq:charbveq} \left( u(x) - \int_{a}^{x} \, \mathrm{d}\mu(x) \right)^\prime = 0. \end{equation} \end{proof}

\begin{proof}[Proof of Theorem \ref{th:bridge}] \mbox{}

\paragraph{$"\implies"$} Assume that $u \in \mathrm{BV}\left((a, \, b) \right)$. By \eqref{eq:charbv} there exists a constant $c \in \R$ such that
\begin{equation*}u(x) = c + \mu \left( (a, \, x) \right). \end{equation*}
The function $u$ is summable by definition; thus it is enough to prove that the total essential variation is finite. The explicit expression for $u$ immediately yields to
\begin{equation*} T_{v_{ess}}(u) := \sup_{P \in \mathcal{P}_{ess}} \sum_{i = 0}^{n-1} \left| \mu\left( (a, \, x_{i + 1}) \right) - \mu\left( (a, \, x_{i}) \right) \right| = \sup_{P \in \mathcal{P}_{ess}} \sum_{i = 0}^{n-1} \left| \mu\left( (x_i, \, x_{i + 1}) \right) \right|.\end{equation*} 
In particular, we notice that the total essential variation is bounded by the mass of $\mu$, and thus it is finite by assumption:
\begin{equation*} T_{v_{ess}}(u) = \sup_{P \in \mathcal{P}_{ess}} \sum_{i = 0}^{n-1} \left| \mu\left( (x_i, \, x_{i + 1}) \right) \right| \leq \left| Du \right|\left( (a, \, b) \right) < + \infty.\end{equation*} 

\paragraph{$"\impliedby"$} This assertion is left as an exercise for the reader.

\end{proof}

\section{Functional Properties}

In this final section, we investigate some of the main functional properties of bounded variation functions (extension operator, trace operator, approximation results, etc.) which will be extensively used when we introduce the notion of finite perimeter set.

\begin{definition}[Lipschitz Boundary] \index{Lipschitz boundary} A domain $\Omega \subset \R^n$ with Lipschitz boundary is an open subset, which is locally the graph of a Lipschitz map with respect to some choice of orthogonal bases.\end{definition}

\begin{theorem}[Extension]\label{extendBV} Let $\Omega \subset \R^n$ be a bounded set with Lipschitz boundary $\partial \Omega$. Then there exists\footnote{The operator $E$ is, in general, not unique in any sense.} an extension operator
\begin{equation*}E : \mathrm{BV}(\Omega) \longrightarrow \mathrm{BV}(\R^n). \end{equation*} \end{theorem}

\begin{theorem}[Approximation, I]  \label{th:density1} The immersion
\begin{equation*}C^\infty \left(\R^n \right) \hookrightarrow \mathrm{BV}\left(\R^n\right) \end{equation*}
is dense, that is, for every $u \in \mathrm{BV}\left(\R^n\right)$ there exists a sequence $(u_n)_{n \in \N} \subset C^\infty \left( \R^n \right)$ such that
\begin{equation*} \begin{cases} u_n \longrightarrow u & \text{in $L^1(\R^n)$}, \\[0.5em] \nabla \, u_n \cdot \mathcal{L}^n \rightharpoonup Du & \text{weakly (in the sense of distributions)}, \\[0.5em] \| \nabla \, u_n \|_{L^1(\R^n)} \longrightarrow \left\| D u \right\|.\end{cases} \end{equation*} \end{theorem}

\begin{proof} \mbox{}

\paragraph{Step 1.} Let $\rho \in C_c^\infty(\R^n)$ be a mollifier and, for any $\epsilon > 0$, consider the scaling
\begin{equation*} \rho_\epsilon(x) := \frac{1}{\epsilon^n} \rho \left( \frac{x}{\epsilon} \right). \end{equation*}
The function $u_\epsilon := \rho_\epsilon \ast u$ belongs to $C_c^\infty(\R^n)$ and a standard approximation argument is enough to infer that
\begin{equation*}u_\epsilon \to u \quad \text{in $L^1(\R^n).$} \end{equation*}

\paragraph{Step 2.} The key point of the proof is the following inequality (from which the last two properties will follow immediately):
\begin{equation}\label{key:approx} \left\| \rho_\epsilon \ast Du \right\|_{L^1(\R^n)} = \left\| \nabla u_\epsilon \right\|_{L^1(\R^n)} \stackrel{?}{\leq} \left\| Du \right\|.\end{equation}
If we give this inequality for granted, then we can easily conclude that the family of functions $\left(\nabla u_\epsilon \right)_{\epsilon > 0}$ is uniformly bounded in $L^1(\R^n)$. In particular, $\left(\nabla u_\epsilon \cdot \mathcal{L}^d \right)_{\epsilon > 0}$ is a uniformly bounded sequence of measures, and therefore (by an Ascoli-Arzelà compactness theorem) it converges in sense of measures to a vector-valued measure $\mu \in \mathcal{M}(\R^n; \; \R^n)$.  For every $\varphi \in C_c^\infty(\R^n)$ it turns out that
\begin{equation*} \int_{\R^n} \nabla u_\epsilon(x) \varphi(x) \, \mathrm{d}x = - \int_{\R^n} u_\epsilon(x) \mathrm{div}(\varphi)(x) \, \mathrm{d}x, \end{equation*}
and the left-hand side converges to 
\begin{equation*} \int_{\R^n} \varphi(x) \, \mathrm{d} \mu(x), \end{equation*}
while the right-hand side converges to
\begin{equation*} - \int_{\R^n} u(x) \mathrm{div}(\varphi)(x) \, \mathrm{d}x \end{equation*}
as $\epsilon \to 0^+$. In particular, we obtain the identity
\begin{equation*} \int_{\R^n} \varphi(x) \, \mathrm{d}\mu(x) = - \int_{\R^n} u(x) \mathrm{div}(\varphi)(x) \, \mathrm{d}x  = \int_{\R^n} \varphi(x) \, \mathrm{d} (Du)(x), \, \qquad \forall \, \varphi \in C_c^\infty(\R^n), \end{equation*}
which implies that $\mu = Du$ (e.g., using the fundamental lemma in Calculus of Variations).

\paragraph{Step 3.} The third property follows easily from the fact that $\| \cdot \|_{L^1(\R^n)}$ is a lower semi-continuous function. Indeed, we have that
\begin{equation*}\liminf_{\epsilon \to 0^+} \left\| \nabla u_\epsilon \right\|_{L^1(\R^n)} \geq \left\| Du \right\|,\end{equation*}
which gives the sought convergence of the masses, as the opposite inequality holds by \eqref{key:approx}.
\end{proof}

\begin{theorem}[Approximation, II]\label{bv:th:approx2} Let $\Omega \subset \R^n$ be a bounded set with Lipschitz boundary $\partial \Omega$. The immersion
\begin{equation*}C^\infty \left( \overline{\Omega} \right) \hookrightarrow \mathrm{BV}\left(\Omega\right) \end{equation*}
is dense, that is, for every $u \in \mathrm{BV}\left(\Omega\right)$ there exists a sequence $(u_n)_{n \in \N} \subset C^\infty \left( \Omega \right)$ such that
\begin{equation*} \begin{cases} u_n \longrightarrow u & \text{in $L^1(\Omega)$}, \\[0.5em] \nabla \, u_n \cdot \mathcal{L}^n \rightharpoonup Du & \text{in the weak sense of distributions}, \\[0.5em] \| \nabla u_n \|_{L^1(\R^n)} \longrightarrow \left| D u \right|(\Omega).\end{cases} \end{equation*} \end{theorem}

\begin{proof} The key idea is to extend $u$ to $\R^n$ via an extension operator
\begin{equation*}E : \mathrm{BV}(\Omega) \longrightarrow \mathrm{BV}(\R^n),\end{equation*}
which exists as a consequence of \hyperref[extendBV]{Theorem \ref{extendBV}}, and apply the first approximation result to the function $E(u)$ (see \hyperref[th:density1]{Theorem \ref{th:density1}}).

We need to be careful though: It is in general false that every extension operator $E$ can be used here since, e.g., the sequence $\widetilde{u_\epsilon}$, which approximates $E(u)$, does not satisfy the third property (see \hyperref[fig:bv1]{Figure \ref{fig:bv1}}). More precisely, the norm is lower semi-continuous and hence
\begin{equation*}\liminf_{\epsilon \to 0^+} \left\| \nabla u_\epsilon \right\|_{L^1(\R^n)} \geq \left\| Du \right\|,\end{equation*}
but the opposite inequality needs not to hold. Indeed, we need to ask $E$ to be an extension operator satisfying the following key property:
\begin{equation}\label{key:approx2} \left| D E(u) \right|(\partial \Omega) = 0.\end{equation}
Such an extension operator exists, but it needs to be chosen carefully (we will not discuss this further, but the reader may try to do it as an exercise).
\end{proof}

\begin{remark}The immersion
\begin{equation*}C^\infty \left( \overline{\Omega} \right) \hookrightarrow \mathrm{BV}\left(\Omega\right) \end{equation*}
is not dense with respect to the norm \eqref{bv:norm}.\end{remark} 

\begin{exercise}Let $\Omega$ be a bounded set with Lipschitz boundary $\partial \Omega$, and let $u \in \mathrm{BV}(\Omega)$. Prove that the following is an extension operator
\begin{equation*} E : \mathrm{BV}(\Omega) \longrightarrow \mathrm{BV}(\R^n), \qquad u \longmapsto \widetilde{u}(x) := \begin{cases} u(x) & x \in \Omega \\[0.5em] 0 & x \notin \Omega, \end{cases} \end{equation*}
and prove also that it is not the right one for the proof of \hyperref[bv:th:approx2]{Theorem \ref{bv:th:approx2}}.\end{exercise}

\begin{figure}[h]
\centering
\includegraphics[width=12cm, height=8cm]{images/TGMFPS2.png}
\caption{What could go wrong in the proof?}
\label{fig:bv1}
\end{figure}

\begin{remark}\index{Sobolev critical exponent} Recall that the Sobolev critical exponent in $\R^n$ is defined by setting
\begin{equation*} 1^\ast := \frac{n}{n-1}. \end{equation*}\end{remark} 

\begin{theorem}[Sobolev Embedding] \index{Sobolev Embedding Theorem} \label{sobemb} Let $\Omega \subset \R^n$ be a bounded set with Lipschitz boundary $\partial \Omega$. The immersion
\begin{equation*} \mathrm{BV}\left(\Omega\right) \hookrightarrow L^p(\Omega) \end{equation*}
is continuous for every $1 \leq p \leq 1^\ast$ and compact for every $1 \leq p < 1^\ast$. \end{theorem}

\paragraph{Characterization of $\mathrm{BV}(\R^n)$.} In this brief paragraph, we want to state and prove a useful characterization of $\mathrm{BV}(\R^n)$, which is exactly the one that does not work for $L^1(\R^n)$.

\begin{proposition}Let $u \in L^1(\R^n)$. The following assertions are equivalent: \mbox{}
\begin{enumerate}[label=\textbf{(\arabic*)}]
\item The function $u$ belongs to $\mathrm{BV}(\R^n)$.
\item There is a constant $C > 0$ such that
\begin{equation*} \left\| \tau_h u - u \right\|_{L^1(\R^n)} \leq C |h|. \end{equation*}
\end{enumerate} \end{proposition}

\begin{proof} \mbox{} 

\paragraph{$"\implies"$} Assume that $u \in C^\infty(\R^n)$ in such a way that
\begin{equation*} \frac{u(x + h e_i) - u(x)}{h} \xrightarrow{h \to 0^+} \frac{\partial u}{\partial x_i}. \end{equation*}
First, we notice that it suffices to prove the thesis for an orthonormal basis of directions, e.g. $\{e_1, \, \dots, \, e_n\}$. A simple computation yields to
\begin{equation*} \begin{aligned} \left\| \tau_i u - u \right\|_{L^1(\R^n)} & = \int_{\R^n} \left| u(x + h e_i ) - u(x) \right| \, \mathrm{d}x \leq \\[1em] & \leq \left\| \frac{\partial u}{\partial x_i} \right\|_{L^1(\R^n)} |h|, \end{aligned} \end{equation*}
which is enough to conclude using the density of $C^\infty(\R^n)$ in $\mathrm{BV}(\R^n)$.

\paragraph{$"\impliedby"$} Vice versa, it suffices to prove that for every weakly differentiable summable function $u$ it turns out that
\begin{equation*} \frac{u(x + h e_i) - u(x)}{h} \xrightarrow{h \to 0^+} D_i u \end{equation*}
in the sense of distributions.
\end{proof}

\begin{theorem}Let $\Omega$ be either $\R^n$ or a bounded subset of $\R^n$ with Lipschitz boundary $\partial \Omega$. If $(u_n)_{n \in \N} \subset \mathrm{BV}(\Omega)$ is a bounded sequence, then there exists a subsequence $(n_k)_{k \in \N}$ such that
\begin{equation*} \begin{aligned} & u_{n_k} \to u \in \mathrm{BV}(\Omega) & \text{in $L^1(\Omega)$}, \\[1em] & D u_{n_k} \rightharpoonup D u & \text{weakly-$\ast$ (in the sense of measures)}. \end{aligned}\end{equation*} \end{theorem}

\begin{proof}The sequence of measures $\left( D u_n \right)_{n \in \N}$ is bounded by assumption; hence, up to subsequences, it converges in the weak sense of measures to a measure $\mu$ in $\mathcal{M}\left(\Omega; \; \R^n\right)$. By \hyperref[sobemb]{Sobolev Embedding Theorem \ref{sobemb}} the immersion
\begin{equation*} \mathrm{BV}(\Omega) \hookrightarrow L^1(\Omega) \end{equation*}
is compact, and thus the sequence $(u_n)_{n \in \N} \subset L^1(\Omega)$ is relatively compact, which means that - up to subsequences - it converges strongly (with respect to the $L^1$ norm) to an element $u$. To conclude the proof, it remains to prove that
\begin{equation*} \mu = D u. \end{equation*}
The same argument used in the proof of \hyperref[th:density1]{Theorem \ref{th:density1}} works here; therefore it is left to the reader to fill in the details. \end{proof}

\paragraph{Trace Operator.} In the last paragraph of this section, we investigate the trace operator, and we state the main theorem (which is somewhat equivalent to the one valid for Sobolev spaces), and we briefly explain why in the finite perimeter setting, we cannot introduce boundary condition using the trace operator.

\begin{theorem}[Trace Operator] \index{trace theorem} The trace operator, defined by
\begin{equation*} T : C^1\left( \overline{\Omega} \right) \longrightarrow L^1 \left( \partial \Omega \right), \qquad u \longmapsto u \, \big|_{\partial \Omega}, \end{equation*}
is bounded with respect to the $\| \cdot \|_{\mathrm{BV}(\Omega)}$ norm, and it can be uniquely extended to a linear operator
\begin{equation*} T : \mathrm{BV}(\Omega) \longrightarrow L^1 \left( \partial \Omega \right). \end{equation*}
\end{theorem}

\begin{remark}Recall that the space $C^1\left( \overline{\Omega} \right)$ is dense in $\mathrm{BV}(\Omega)$ in the weak sense (see \hyperref[th:density1]{Theorem \ref{th:density1}}). This embedding is the main difference with the Sobolev spaces setting we mentioned above and, as we will see later, it is also the reason why the trace operator is not used to assign boundary conditions. \end{remark}

\begin{remark}Recall that the trace operator defined on $W^{1, \, p}(\Omega)$ can be used to assign boundary conditions as follows. Let $(u_n)_{n \in \N} \subset W^{1, \, p}(\Omega)$ be a minimizing sequence for a functional $F(u) : W^{1, \, p}(\Omega) \longrightarrow \R$, and assume that
\begin{equation*} T u_n = g. \end{equation*} 
If $u_n \rightharpoonup u$ weakly in $W^{1, \, p}(\Omega)$, then the \textit{Sobolev trace theorem} implies that $T u = g$, which is exactly what we would like to happen when trying to find a solution for a minimum problem.

On the other hand, if $(u_n)_{n \in \N}$ is a bounded sequence in $\mathrm{BV}(\Omega)$, then - up to subsequences - $u_n$ converges to $u$ strongly in $L^1(\Omega)$. However, in general, it is not true that $T u_n$ converges (in any reasonable sense) to $T u$ (as we shall prove in the following examples).\end{remark}

\begin{example}Let $\Omega := (0, \, 1)$. Consider the sequence $(u_n)_{n \in \N} \subset \mathrm{BV}\left( (0, \, 1) \right)$ that is defined as it is portrayed in \hyperref[fig:ex1]{Figure \ref{fig:ex1}}. Clearly, for every $n \in \N$, it turns out that
\begin{equation*} T u_n(0) = 0 \qquad \text{and} \qquad T u_n(1) = 1, \end{equation*}
but $u_n$ converges pointwise to $u(x) \equiv 1$ on $\Omega$, which means that a boundary condition is not preserved under the limit ($T u(0) = 1$). The reader may check by herself that
\begin{equation*} u_n \rightharpoonup 1 \qquad \text{in $\mathrm{BV}(\Omega)$}, \end{equation*}
that is,
\begin{equation*} \begin{cases} u_n \to u & \text{in $L^1(\Omega)$}, \\[0.5em] \|u_n\|_{\mathrm{BV}(\Omega)} \leq M & \forall \, n \in \N.\end{cases}\end{equation*} \end{example}

\begin{figure}[h]
\centering
\includegraphics[width=12cm, height=9cm]{images/TGMFPS3.png}
\caption{First Counterexample}
\label{fig:ex1}
\end{figure}

\begin{example}Let $\Omega \subseteq \R^n$, and consider the sequence of subsets $(E_k)_{k \in \N} \subset \mathcal{P}\left( \Omega \right)$ defined as it is portrayed in \hyperref[fig:ex12]{Figure \ref{fig:ex12}}. Moreover, assume that for any $n \in \N$ \mbox{}
\begin{enumerate}[label=\textbf{(\alph*)}]
\item $E_k$ is smooth,
\item $\overline{E_k} \subset \Omega$, and
\item the measure $\mathcal{H}^{n-1}(\partial E_k)$ is of the same order of $\mathcal{H}^{n-1}(\partial \Omega)$.
\end{enumerate}
The sequence given by the characteristic functions $u_k := \mathbbm{1}_{E_k} \in \mathrm{BV}(\Omega)$ is uniformly bounded since the following properties hold:
\begin{enumerate}[label=\textbf{(\alph*)}]
\item $\| u_k \|_{L^1(\Omega)} \leq \mathcal{H}^n(\Omega)$;
\item $ \|D u_k \| = \mathcal{H}^{n-1} \left( \partial E_k \right) \to \mathcal{H}^{n-1}(\partial \Omega)$ as $n$ goes to $+ \infty$.
\end{enumerate}
Clearly, the limit of the sequence $u_k$ is the characteristic function of $\Omega$, that is, the constant function $1 \in \mathrm{BV}(\Omega)$. The reader may easily prove that
\begin{equation*} T u_n = 0 \qquad \text{but} \qquad T u = 1, \end{equation*}
which means that $T u_n$ does not converge in any reasonable sense to $T u$. \end{example}

\begin{figure}[h]
\centering
\includegraphics[width=14cm, height=9cm]{images/TGMFPS4.png}
\caption{Second Counterexample}
\label{fig:ex12}
\end{figure}
\chapter{Finite Perimeter Sets}

In this final chapter, we introduce the notion of \textit{finite perimeter set}, and we also present a variational setting that allows us to prove the existence of a solution to the Plateau's problem (for suitable boundary conditions) and the capillarity problem.

In the second half of the chapter, we introduce the essential and the reduced boundary, and we prove the \textit{Structure Theorem} due to De Giorgi and Federer.

\section{Main Definitions and Elementary Properties}

In this section, we present the definition of finite perimeter set, and we exploit the theory of bounded variation functions introduced in the previous chapter to derive approximation and compactness results.

\begin{definition}[Finite Perimeter Set] \index{finite perimeter set} Let $E \subseteq \R^n$. We say that $E$ has finite perimeter if and only if the characteristic function has bounded variation, that is,
\begin{equation*} \mathbbm{1}_E \in \mathrm{BV} \left( \R^n \right). \end{equation*}
Recall that the functional associated to $\mathbbm{1}_E$ is given by
\begin{equation*}\Lambda_{\mathbbm{1}_E}(\varphi) := \int_{E} \mathrm{div}(\varphi)(x) \, \mathrm{d}x,  \end{equation*}
and it is bounded with respect to the uniform norm. The perimeter of $E$ is defined by the operator norm of the functional $\Lambda_{\mathbbm{1}_E}(\varphi)$, that is,
\begin{equation*} \mathrm{Per} (E) := \left\| \Lambda_{\mathbbm{1}_E} \right\|_{\ast}. \end{equation*} \end{definition}

\begin{definition}[Relative F.P.S.] \index{finite perimeter set!relative} Let $\Omega \subseteq \R^n$ be an open set, and let $E \subseteq \Omega$ be a subset. We say that $E$ has finite perimeter in $\Omega$ if and only if
\begin{equation*} \mathbbm{1}_E \in \mathrm{BV} \left( \Omega \right). \end{equation*}
The perimeter of $E$ in $\Omega$ is defined by the operator norm of the functional associated to $\mathbbm{1}_E$, that is,
\begin{equation*} \mathrm{Per}_{\Omega}(E) = \left\| \Lambda_{\mathbbm{1}_E} \right\|_{\ast}, \end{equation*}
where $\Lambda_{\mathbbm{1}_E}$ denotes the restriction of the functional to the space $C_c^\infty(\Omega)$. \end{definition}

\begin{remark}Let $E \subseteq \Omega \subseteq \R^n$ be a smooth subset (e.g., assume that the boundary of $E$ is at least of class $C^1$). The reader may prove that the perimeter of $E$ in $\Omega$ is simply given by the following formula:
\begin{equation} \label{fps:gf} \mathrm{Per}_{\Omega}(E) = \mathcal{H}^{n-1} \left( \partial E \cap \Omega \right). \end{equation}
\textit{Hint.} A similar argument to the one used in the \hyperref[ex:bvfps]{Example \ref{ex:bvfps}} works here. In particular, we notice that the derivative of the characteristic function is given by
\begin{equation*}D \left( \mathbbm{1}_E \right) = \nu_i \mathbbm{1}_{\partial E \cap \Omega} \cdot \mathcal{H}^{n-1}. \end{equation*} \end{remark}

\begin{remark} It is important to notice that, in general, the formula
\begin{equation*} \mathrm{Per}_{\Omega}(E) = \mathcal{H}^{n-1} \left( \partial E \right)\end{equation*}
is not true, not even if $E$ is smooth. The intuitive idea  is clear: Since $\Omega$ is an open set, the boundary of $E$ may overlap with the boundary of $\Omega$ as it happens in \hyperref[fig:fpspr1]{Figure \ref{fig:fpspr1}}.  \end{remark}

\begin{figure}[h]
\centering
\includegraphics[width=14cm, height=8cm]{images/TGMFPS1.png}
\caption{The boundary of the set $E$ partially overlaps with the boundary of $\Omega$. Therefore only the {\color{magenta}magenta} part of $\partial E$ contributes to the perimeter of $E$ in $\Omega$.}
\label{fig:fpspr1}
\end{figure}

\paragraph{Compactness Results.} In this paragraph, we state two compactness results for finite perimeter sets, both of which derive from the functional properties of bounded variations functions.

\begin{notation}Let $\Omega$ be an open set, and let $E, \, F \subseteq \Omega$ be two subsets. The distance between $E$ and $F$ is the $n$-Lebesgue measure of the symmetric difference, that is,
\begin{equation*}d(E, \, F) := \| \mathbbm{1}_E - \mathbbm{1}_F \|_{L^1(\Omega)} = \mathcal{L}^n \left(E  \,\triangle \, F \right).  \end{equation*} \end{notation}

\begin{theorem}[Compactness, I] \label{th:cpfps}Let $(E_k)_{k \in \N} \subset \mathcal{P}(\R^n)$ be a uniformly bounded sequence of finite perimeter sets, that is,
\begin{equation*} \mathrm{Per} (E_k) < + \infty \qquad \text{and} \qquad E_k \subseteq \Omega^\prime \subset \subset \Omega, \end{equation*}
where $\Omega \subset \R^n$ is a bounded open subset. Then there exist a finite perimeter set $E \subseteq \Omega$ and an increasing subsequence $(n_k)_{k \in \N}$ such that
\begin{equation*} d(E_k, \, E) \xrightarrow{k \to + \infty} 0. \end{equation*}
Moreover, the perimeter is a lower semi-continuous functional, that is
\begin{equation*} \liminf_{k \to + \infty}\left( \mathrm{Per} (E_k) \right) \geq \mathrm{Per} (E). \end{equation*}\end{theorem}

\begin{remark}The assumption "$(E_k)_{k \in \N}$ is a uniformly bounded sequence" is crucial here, otherwise one can exhibit the easy counterexample
\begin{equation*} E_k := B(k, \frac{1}{2}), \end{equation*}
which is a collection given by integer translations of a ball centered at zero with radius one half. \end{remark}

\begin{theorem}[Compactness, II] Let $(E_k)_{k \in \N} \subset \mathcal{P}(\R^n)$ be a uniformly bounded sequence of finite perimeter sets, that is,
\begin{equation*} \mathcal{L}^n\left(E_k \right), \,  \mathrm{Per} (E_k) \leq c < + \infty. \end{equation*}
Then there exist a finite perimeter set $E \subseteq \Omega$ and an increasing subsequence $(n_k)_{k \in \N}$ such that
\begin{equation*} \left| \left( E_k \, \triangle \, E\right) \cap B(x, \, r) \right| \xrightarrow{k \to + \infty} 0, \qquad \forall \, B(x, \, r) \subseteq \R^n. \end{equation*}
Moreover, the perimeter is a lower semi-continuous functional, that is
\begin{equation*} \liminf_{k \to + \infty}\left( \mathrm{Per} (E_k) \right) \geq \mathrm{Per} (E). \end{equation*}\end{theorem}

\section{Approximation Theorem via Coarea Formula}

The primary goal of this section is to prove that any finite perimeter set can be approximated via a sequence of smooth (e.g. $C^\infty$ boundary) sets. In order to prove this assertion, we will need to use the \textit{coarea} formula presented in \hyperref[sec:coafor]{Section \ref{sec:coafor}}.

\begin{theorem} \label{fpsapprox} Let $E \subseteq \R^n$ be a finite perimeter set. Then there exists a sequence of smooth (boundary of class $C^\infty$) sets $(E_k)_{k \in \N}$ such that
\begin{equation*} \begin{cases} E_k \xrightarrow{d} E, \\[0.5em] \mathcal{H}^{n-1} \left( \partial E_k \right) = \mathrm{Per}(E_k) \xrightarrow{k \to + \infty} \mathrm{Per} (E),\end{cases} \end{equation*}
where $d$ denotes the symmetric distance introduced in the previous section. \end{theorem}

\begin{proof}Let $\rho$ be a mollifier kernel. Denote by $\rho_{\epsilon}$ the $\epsilon$-rescaling, which is defined in the usual way:
\begin{equation*} \rho_\epsilon(x) := \frac{1}{\epsilon^n} \, \rho \left( \frac{x}{\epsilon} \right). \end{equation*}
Let $u_\epsilon := \rho_\epsilon \ast \mathbbm{1}_E$ be the convolution. For every $s_\epsilon \in [0, \, 1]$ and $\epsilon > 0$ the set
\begin{equation*} E_{s_\epsilon, \, \epsilon} := \left\{ x \in \Omega \: \left| \: u_\epsilon(x) > s_\epsilon \right. \right\} \end{equation*}
is open and smooth, provided that we pick an $s_\epsilon$ such that it is not a critical value (which is always possible by Sard's Lemma).

\paragraph{Step 1.} We claim that, if $\delta \leq s_\epsilon \leq 1 - \delta$, then
\begin{equation} \label{claim:1:1} u_\epsilon \xrightarrow{L^1(\Omega)} u \implies E_{s_\epsilon, \, \epsilon} \xrightarrow{d} E. \end{equation}
Indeed, a straightforward application of the Fubini-Tonelli's theorem proves that
\begin{equation*}\| u_\epsilon - u \|_{L^1(\Omega)} = \int_{0}^1 d\left(E, \, E_{s, \, \epsilon} \right) \, \mathrm{d}s. \end{equation*}
On the other hand, for every $s^\prime < s$ it turns out that $E_{s^\prime, \, \epsilon} \supseteq E_{s, \, \epsilon}$; hence we can easily estimate the $L^1$-norm of the difference as follows:
\begin{equation*}\| u_\epsilon - u \|_{L^1(\Omega)} \geq \int_{s_\epsilon}^1 \left| E \setminus E_{s_\epsilon, \, \epsilon} \right| \, \mathrm{d}s \geq (1 - s_\epsilon) \cdot \left| E \setminus E_{s_\epsilon, \, \epsilon} \right|,\end{equation*}
where $\setminus$ denotes the usual difference between two sets.

In particular, if we take $s_\epsilon$ strictly less than $1$, then $\| u_\epsilon - u \|_{L^1(\Omega)} \to 0$ implies $\left| E \setminus E_{s_\epsilon, \, \epsilon} \right| \to 0$, which is exactly what we wanted to prove.

\paragraph{Step 2.} The function $u_\epsilon$ is smooth (by convolution). Hence we can apply the \textit{coarea formula} \eqref{coarea} with $h \equiv 1$ and $f = u_\epsilon$ to obtain the following relation:
\begin{equation*} \int_{\R^n} \left| \nabla u_\epsilon(x) \right|  \, \mathrm{d}x = \int_{0}^{1} \mathcal{H}^{n-1} \left( u_\epsilon^{-1}(s) \right) \, \mathrm{d}s,\end{equation*}
where $u_\epsilon^{-1}(s) = \partial E_{s, \, \epsilon}$ if $s$ is a regular value\footnote{In particular, the formula holds with $u_\epsilon^{-1}(s) = \partial E_{s, \, \epsilon}$ for $\mathcal{L}^n$-almost every $s \in [0, \, 1]$ since the set of singular values is negligible by Sard's Lemma.}. One can easily prove that
\begin{equation*} \frac{1}{1 - 2 \delta} \, \mathrm{Per} (E) \geq  \frac{1}{1 - 2 \delta} \int_{0}^{1} \mathrm{Per}(E_{s, \, \epsilon}) \, \mathrm{d}s \geq \dashint_\delta^{1 - \delta} \mathrm{Per} (E_{s, \, \epsilon}) \, \mathrm{d}s, \end{equation*}
for $\delta > 0$ small enough; hence there exists a non-null (with respect to the Lebesgue measure) set of regular values in the interval $[\delta, \, 1- \delta]$ such that
\begin{equation*}\frac{1}{1 - 2 \delta} \, \mathrm{Per} (E) \geq \mathrm{Per} (E_{s_\epsilon, \, \epsilon}) \implies \limsup_{\epsilon \to 0^+}\, \mathrm{Per} (E_{s_\epsilon, \, \epsilon}) \leq \frac{1}{1 - 2 \delta} \, \mathrm{Per} (E)\end{equation*}
for every regular value $s_\epsilon \in [\delta, \, 1 - \delta]$. Recall that the perimeter is a lower semi-continuous functional; hence we only need to refine the estimate and rule $\delta$ out of the equation.
\end{proof}

\begin{exercise} State the approximation theorem in a smooth bounded domain $\Omega \subset \R^n$.\end{exercise}

\section{Existence Results for Plateau's Problem}

\paragraph{Introduction.} Let $\Omega \subset \R^n$ be a bounded smooth subset of $\R^n$. Assume that $\Gamma$ is the boundary of a regular hypersurface $\mathcal{G} \subset \partial \Omega$, that is,
\begin{equation*} \Gamma = \partial \mathcal{G}. \end{equation*}
The Plateau's problem consists of finding the minimal hypersurface $\Sigma$ contained in $\Omega$ such that its boundary is given by $\Gamma$, that is, satisfying the boundary condition $\partial \Sigma = \Gamma$.

\paragraph{Finite Perimeter Setting.} Let $E$ be a subset of $\Omega$ satisfying the following condition:
\begin{equation*} \partial  E \cap \partial \Omega = \mathcal{G}. \end{equation*}
If $\Sigma$ is a regular hypersurface given by the intersection $\partial E \cap \Omega$, then the following relations holds:
\begin{equation} \label{eq:ddlslds} \mathrm{Per} (E) = \mathcal{H}^{n-1} \left( \partial  E \cap \partial \Omega \right) + \mathcal{H}^{n-1}(\Sigma). \end{equation}
This identity allows us to infer that minimizing the functional $\mathcal{H}^{n-1}(\Sigma)$ is equivalent to minimizing the functional $\mathrm{Per} (E)$ among all finite perimeter sets $E$ such that
\begin{equation*} T\left(\mathbbm{1}_E\right) = \mathbbm{1}_{\mathcal{G}}. \end{equation*}

\begin{figure}[h]
\centering
\includegraphics[width=7cm, height=10cm]{images/TGMFPS6.png}
\caption{Plateau's Problem in the F.P.S. formulation.}
\end{figure}

\begin{remark} In the previous sections, we proved that the perimeter is a nice functional, which is also semi-lower continuous. Hence the equivalent formulation in the f.p.s. setting should be enough to demonstrate the existence of a solution. But, as the problem is formulated now, many things could go wrong: \mbox{}
\begin{enumerate}[label=\textbf{\arabic*)}]
\item Assume that one can find a solution of the minimum problem \eqref{eq:ddlslds}. How can one recover a surface $\Sigma$?
\item The minimum of the functional \eqref{eq:ddlslds} actually exists?
\item If $(E_n)_{n \in \N}$ is a minimizing sequence, then the trace $T(\mathbbm{1}_{E_n})$ may not converge to $\mathbbm{1}_{\mathcal{G}}$, as proved in the previous chapter.
\item Topological issues. The hypersurface $\mathcal{G}$ with boundary $\Gamma$ may not even exist.
\end{enumerate} \end{remark}

\begin{figure}[h]
\centering
\includegraphics[width=14cm, height=10cm]{images/TGMFPS51.png}
\caption{In this Plateau's problem, the minimizing surface $\Sigma$ makes an extra contact with the boundary of $\Omega$, which means that $T( \mathbbm{1}_{E_n} )$ does not converge to $\mathbbm{1}_{\mathcal{G}}$.}
\end{figure}

\paragraph{Alternative Formulation.} The Plateau's problem needs to be reformulated differently (in the f.p.s. setting) to avoid the issues listed above (except the topological ones, which are way more delicate to deal with).

Let $\Omega \subset \R^n$ be a bounded smooth subset of $\R^n$. Assume that $\Gamma$ is the boundary of a regular hypersurface $\mathcal{G} \subset \partial \Omega$, that is,
\begin{equation*} \Gamma = \partial \mathcal{G}. \end{equation*}
Let $E_0$ be a fixed subset contained in the complement of $\Omega$, that is, $E_0 \subset \R^n \setminus \overline{\Omega}$. The Plateau's problem, again, constist of finding the minimal hypersurface $\Sigma$ contained in $\Omega$ such that
\begin{equation*} \partial \Sigma = \Gamma. \end{equation*}
In the finite perimeter setting, the idea is to look for the minimum point(s) of the functional
\begin{equation} \label{eq:ddlslds2} \mathrm{Per} (E) - \mathcal{H}^{n-1} \left( \partial E_0 \setminus \overline{\Omega} \right) \end{equation}
among all finite perimeter sets $E$ satisfying the following condition:
\begin{equation*} E \setminus \Omega = E_0 \qquad \text{up to $\mathcal{L}^n$-null sets}. \end{equation*}

\begin{figure}[h]
\centering
\includegraphics[width=14cm, height=10cm]{images/TGMFPS52.png}
\caption{In this picture, one can see how the trace issues are no longer a treat using this alternative formulation of the Plateau's problem. }
\end{figure}

\paragraph{Plateau's problem.} In this brief paragraph, we summarize the classical formulation of the Plateau's problem and we reinterpret it into the finite perimeter set framework.\mbox{}
\begin{enumerate}[label=\textbf{C\arabic*)}]
\item Given $\Omega$ be a bounded smooth subset of $\R^n$.
\item Given $\Gamma$ be a $(n-2)$-dimensional surface contained in $\partial \Omega$.
\item Goal: Find a $(n-1)$-dimensional surface $\Sigma$ such that $\partial \Sigma = \Gamma$ and $\Sigma$ minimizes the area functional $\mathcal{H}^{n-1}(\Sigma)$.
\end{enumerate}
In the finite perimeter set framework, the Plateau's problem can be formulated in the following way:
\mbox{}
\begin{enumerate}[label=\textbf{F\arabic*)}]
\item Given $\Omega$ be a bounded smooth subset of $\R^n$.
\item Given $\mathcal{G} \subset \partial \Omega$ such that there exists a finite perimeter set $E_0$ in $\R^n$ contained in the complement of $\Omega$, that is,
\begin{equation*} E_0 \subseteq \R^n \setminus \overline{\Omega} \qquad \text{and} \qquad  T(\mathbbm{1}_{E_0}) = \mathbbm{1}_{\mathcal{G}} \end{equation*}
up to $\mathcal{L}^n$-null sets.
\item Goal: Find a finite perimeter set $E$ in $\R^n$ such that $E \setminus \overline{\Omega}$ is - up to $\mathcal{L}^n$-null sets - equal to $E_0$, such that $E$ is a minimum point of the perimeter functional.
\end{enumerate}

In conclusion, a natural question arises: For which $\mathcal{G}$ there exists $E_0$ satisfying the property \textbf{F2)}? Surprisingly, it is enough to ask that $\mathcal{G}$ is any Borel set as a consequence of the following theorems.

\begin{theorem}The trace operator
\begin{equation*} T : C^1\left( \overline{\Omega} \right) \longrightarrow L^1 \left( \partial \Omega \right), \qquad u \longmapsto u \, \big|_{\partial \Omega} \end{equation*}
is surjective. \end{theorem}

\begin{remark}In general, the trace operator introduced above does not admit a linear right inverse. \end{remark}

\begin{theorem}Let $u$ be a characteristic function in $L^1 \left(\partial \Omega \right)$. Then there exists a finite perimeter set $E \subseteq \R^n$ such that
\begin{equation*} T (\mathbbm{1}_E) = u. \end{equation*}
\end{theorem}

In particular, the only problems one needs to deal with in Plateau's problem are of topological nature (see, e.g., \hyperref[fig:oewoe]{Figure \ref{fig:oewoe}}).

\begin{figure}[h]
\centering
\includegraphics[width=14cm, height=10cm]{images/TGMFPS53.png}
\caption{The surface $\Gamma$ is not the boundary of any hypersurface $\mathcal{G}$ contained in the torus $\Omega$. }
\label{fig:oewoe}
\end{figure}

\paragraph{Conclusion.} To conclude this chapter, we sketch the framework and the solution of a Plateau's problem. The reader may refer to \hyperref[fig:211]{Figure \ref{fig:211}} for a better understanding of what is going on, but the main idea is to take $\Gamma$ equal to the disjoint union of two circumferences (lying on the planes $\pi_1$ and $\pi_2$ respectively), and $\mathcal{G}$ equal to the disjoint union of the two closed disks.

\begin{figure}[h]
\centering
\includegraphics[width=14cm, height=10cm]{images/TGMFPS211.png}
\caption{Example of Plateau's problem}
\label{fig:211}
\end{figure}

The set $\Omega$ is convex, but once again the condition on the trace is \textbf{not} closed. Therefore this example shows how important is the introduction of a different formulation where the operator $T$ does not appear.

The solution of the minimum problem depends on the distance $d$ between the planes $\pi_1$ and $\pi_2$, and it is either a catenoid or a disjoint union of disks.

\section{Capillarity Problem}

\paragraph{Introduction.} Let $\Omega \subset \R^n$ be an open bounded subset of $\R^n$. In this section, we shall discuss the capillarity problem equilibrium when the container ($\Omega$) is fixed and the volume of the liquid is fixed. More precisely, the drop of liquid (denoted by $E$), contained in $\Omega$, is at equilibrium if and only if it is a critical point (local minima) of the capillarity energy functional
\begin{equation} \label{funccap} \mathscr{F}(E) := \mathcal{H}^{n-1} \left( \partial E \cap \Omega \right) + \sigma \mathcal{H}^{n-1} \left(\partial E \cap \partial \Omega \right), \end{equation}
where $\sigma \in \R$ is a constant, subject to the constraint that the volume is fixed, e.g. $|E| = m$.

\begin{figure}[h]
\centering
\includegraphics[width=14cm, height=8cm]{images/TGMFPS54.png}
\caption{Capillarity Problem}
\label{fig:oewoe12}
\end{figure}

In the Appendix, we shall prove that, if we set the first variation of the functional $\mathscr{F}$ equal to zero, then we find that the mean curvature $H$ is constant. More precisely, it is equal to the Lagrange multiplier associated with the constraint on\footnote{We denote by $\Sigma^f$ the free portion of the surface, that is, the portion of the surface that does not lie on $\partial \Omega$.} $\Sigma^f$, and $\theta := \theta_Y$ is constant on $\Gamma$ and equal to the so-called \textit{Young angle}, i.e. the solution to the following trigonometric equation:
\begin{equation*} \cos( \theta_Y ) = - \sigma. \end{equation*}

\begin{remark} Suppose that the drop of liquid is subject to the action of the gravitational force. Then one needs to slightly modify the functional $\mathscr{F}$ to take into account the gravity in the following way:
\begin{equation} \label{funccapg} \mathscr{G}(E) := \mathcal{H}^{n-1} \left( \partial E \cap \Omega \right) + \sigma  \mathcal{H}^{n-1} \left(\partial  E \cap \partial \Omega \right) + \int_{E} \vec{g} \cdot x. \end{equation}
The reader may compute the first variation of the functional $\mathscr{G}$ and set it equal to zero. It is not hard to check that the mean curvature $H$ is not constant anymore. \end{remark}

\paragraph{Finite Perimeter Formulation.} In the f.p.s. framework, we can translate the capillarity problem associated to the functional $\mathscr{F}$ (subject to the volume constraint) as follows:
\mbox{}
\begin{enumerate}[label=\textbf{F\arabic*)}]
\item Given $\Omega$ bounded smooth subset of $\R^n$.
\item Goal: Find a finite perimeter set $E \subseteq \Omega$ in $\R^n$, with fixed volume $|E| = m$, which minimizes the functional
\begin{equation} \label{funccacfps} \mathscr{F}(E) = \mathrm{Per}_\Omega (E) + \sigma \mathcal{H}^{n-1} \left( \Sigma^c \right), \end{equation}
where $\mathbbm{1}_{\Sigma^c} = T( \mathbbm{1}_E )$ denotes the portion of the surface $\Sigma$ that lies on the boundary of $\Omega$.
\end{enumerate}

\begin{proposition} Suppose that $\sigma \notin [-1, \, 1]$. Then the minimization problem \eqref{funccacfps}, with fixed volume, is ill-posed. \end{proposition}

\begin{remark}More precisely, we shall prove that the functional \eqref{funccacfps} is not lower semi-continuous with respect to the distance $d$ whenever $|\sigma| > 1$. \end{remark}

\begin{proof} Let us consider the sequence of subsets $(E_{1/n})_{n \in \N}$ portrayed in \hyperref[fig:oewoe124]{Figure \ref{fig:oewoe124}}, and suppose that $E_{1/n}$ is smooth for every $n \in \N$. If $|\sigma| > 1$, then one can prove that the functional $\mathscr{F}$ is not lower semi-continuous using this sequence, that is,
\begin{equation*}\mathscr{F}(E_{1/n}) = \left| \Sigma^f \right| + \left| \Sigma^c \right| + o(1) \xrightarrow{\epsilon \to 0^+} \left| \Sigma^f \right| \pm \left| \Sigma^c \right| < \mathscr{F}(E). \end{equation*}
In other words, the relaxation of the functional $\mathscr{F}$ is given by
\begin{equation*}\widetilde{\mathscr{F}}(E) = \begin{cases} \left| \Sigma^f \right| + \left| \Sigma^c \right|, & \text{if $\sigma > 1$}, \\[0.5em] \left| \Sigma^f \right| - \left| \Sigma^c \right| & \text{if $\sigma < - 1$}, \end{cases} \end{equation*}
and this is enough to prove the claim above since
\begin{equation*} \sigma > 1 \implies \widetilde{\mathscr{F}}(E) < \mathscr{F}(E) =  \left| \Sigma^f \right| + \sigma \left| \Sigma^c \right|, \end{equation*} 
and similarly
\begin{equation*} \sigma < - 1 \implies \widetilde{\mathscr{F}}(E) > \mathscr{F}(E) =  \left| \Sigma^f \right| + \sigma \left| \Sigma^c \right|. \end{equation*} \end{proof}

\begin{figure}[h]
\centering
\includegraphics[width=12cm, height=8cm]{images/TGMFPS55.png}
\caption{Left: Sequence when $\sigma > 1$. Right: Sequence when $\sigma < - 1$.  }
\label{fig:oewoe124}
\end{figure}

\begin{claim}If $\sigma \in [-1, \, 1]$, then the functional \eqref{funccacfps} is lower semi-continuous. \end{claim}

The proof of this claim cannot be tackled with the tools introduced in this section since we need to use more sophisticated methods. We will come back to this result in the next chapter.

\section{Finite Perimeter Sets: Structure Theorem}

The primary goal of this section is to state and prove the structure theorem for finite perimeter set, due to both De Giorgi and Federer (it appeared in different papers, but we shall merge them to obtain a very reasonable result.)

\begin{lemma} Let $E \subset \R^n$ be a bounded subset of $\R^n$ such that $\partial E$ is a Lipschitz boundary. If $\nu_e$ is the external normal to $E$, then
\begin{equation*}D (\mathbbm{1}_E) = - \nu_e \, \mathbbm{1}_{\partial E} \cdot \mathcal{H}^{n-1} \end{equation*}
and, in particular, it turns out that
\begin{equation} \label{perform} \mathrm{Per}(E) = \mathcal{H}^{n-1} \left( \partial E \right).\end{equation} \end{lemma}

\begin{proof} The identity follows immediately from the divergence theorem for Lipschitz boundaries. \end{proof}

\begin{example} In general, the $\mathcal{H}^{n-1}$ measure of the boundary $\partial E$ does not coincide with the perimeter of the set $\mathrm{Per} (E)$. For example, if $E$ is any $\mathcal{L}^n$-null set, then it is easy to prove that
\begin{equation*} \mathrm{Per} (E) = \mathrm{Per} (\varnothing) = 0. \end{equation*}
On the other hand, the interior of $E$ is empty and hence
\begin{equation*} \mathcal{H}^{n-1} \left( \partial E \right) = \mathcal{H}^{n-1} \left( \overline{E} \right), \end{equation*}
and this last quantity is, in general, strictly positive. \end{example}

There is a result which links the notion of the perimeter, and the Hausdorff measure of the boundary for every Borel set $E$ Borel, but the proof is rather involved. Here we state the result and sketch the proof of a slightly weaker statement.

\begin{theorem}For every $E \subset \R^n$ Borel, it turns out that
\begin{equation} \label{perform2} \mathrm{Per} (E) \leq \mathcal{H}^{n-1} \left( \partial E \right).\end{equation}  \end{theorem}

\begin{theorem}There is a constant $c > 1$ such that
\begin{equation} \label{perform3} \mathrm{Per} (E) \leq c \cdot  \mathcal{H}^{n-1} \left( \partial E \right)\end{equation}
for every $E \subset \R^n$ Borel. \end{theorem}

\begin{proof}[Sketch of the Proof] We can assume without loss of generality that $E$ is a closed and bounded Borel set with finite Hausdorff measure of the boundary, that is,
\begin{equation*}\mathcal{H}^{n-1} \left( \partial E \right) < + \infty. \end{equation*}

\paragraph{Step 1.} Let $\delta > 0$  be a positive number, and let $\{B_i^\delta\}_{i \in \N}$ be a collection of open balls of uniformly bounded radii $r_i \leq \delta$, which is an optimal cover of the boundary $\partial E$ with respect to the spherical Hausdorff measure. By compactness of $\partial E$, it turns out that
\begin{equation*} \partial E \subseteq \bigcup_{i = 1}^{N(\delta)} B_i^\delta, \end{equation*}
where $N(\delta)$ is a positive integer. In particular, we have the estimate
\begin{equation*} \sum_{i = 1}^{N(\delta)} (2 r_i)^{n-1} \leq \mathcal{H}_{s}^{n - 1} \left( \partial E \right) \leq C \cdot \mathcal{H}^{n-1}(\partial E), \end{equation*}
as a consequence of the fact that the Hausdorff measure is equivalent to the Hausdorff spherical measure. Let
\begin{equation*}E_\delta := E \cup \bigcup_{i = 1}^{N(\delta)} B_i^\delta, \end{equation*}
and notice that $E_\delta \xrightarrow{d} E$ as $\delta \to 0^+$, that is,
\begin{equation*}|E_\delta \, \triangle \, E| = |E_\delta \setminus E| \xrightarrow{\delta \to 0^+} 0, \end{equation*}
since we assumed $E$ to be a closed set. 

\paragraph{Step 2.} Suppose that the balls intersect transversally. Then the boundary $\partial E_\delta$ is Lipschitz, which means that the perimeter formula \eqref{perform} holds. Moreover, the perimeter functional is a lower semi-continuous function, and therefore
\begin{equation*} \mathrm{Per} (E) \leq \liminf_{\delta \to 0^+} \mathrm{Per} (E_\delta) = \liminf_{\delta \to 0^+} \mathcal{H}^{n-1} \left( \partial E_\delta \right). \end{equation*}
On the other hand, the boundary of $E_\delta$ is covered by the finite family of $\delta$-balls, which means that
\begin{equation*} \begin{aligned} \mathcal{H}^{n-1} \left( \partial E_\delta \right) & \leq \mathcal{H}^{n-1} \left( \bigcup_{i = 1}^N \partial B_i^\delta \right) \leq \\[1em] & \leq \sum_{i = 1}^N \mathcal{H}^{n-1} \left( \partial B_i^\delta \right) \leq \\[1em]& \leq C^\prime \sum_{i = 1}^N (2r_i)^{n-1} \leq c \cdot \mathcal{H}^{n-1} \left( \partial E \right), \end{aligned} \end{equation*}
where $c := C \cdot C^\prime$ is the sought constant.
\end{proof}

\begin{exercise}Let $E \subset \R$ be a finite perimeter set. Prove that $E$ is, up to $\mathcal{L}^1$-null sets, the union of finitely many intervals. \label{ex:finasmdsd} \end{exercise}

\begin{lemma}Let $E \subseteq \R$ be a finite perimeter set. There exists a representative of $E$, also denoted by $E$, such that formula \eqref{perform} holds.\end{lemma}

\begin{proof}The function $\mathbbm{1}_E$ belongs to $\mathrm{BV}(\R)$ by assumption. Therefore, by \hyperref[lemma:bridge]{Lemma \ref{lemma:bridge}} there exists a constant $c \in \R$ such that
\begin{equation*}\mathbbm{1}_E(x) = c + \mu \left([- \infty, \, x] \right), \end{equation*}
where $\mu$ denotes the weak derivative $D (\mathbbm{1}_E)$. The reader may prove that, if $\mu( \{x\} ) = 0$ for every $x \in \R$, then this is a continuous representative of $\mathbbm{1}_E$ that satisfies the formula \eqref{perform}.\end{proof}

\begin{proposition}Let $E \subset \R^n$ and $n \geq 2$. Then there exists a Borel set $E$ such that no representative of $E$ satisfies the perimeter formula \eqref{perform}, that is, the inequality is strict for every representative $E^\prime$ of $E$.\end{proposition}

\begin{proof}Let $\{x_n\}_{n \in \N} \subset \R^2$ be a dense sequence of points, and let $(r_n)_{n \in \N}$ be a sequence of positive real numbers satisfying the condition
\begin{equation} \label{ex.fps.1} \sum_{n \in \N} n \cdot r_n < + \infty. \end{equation}

\paragraph{Step 1.} Let
\begin{equation*} L := \sum_{n \in \N} r_n < + \infty\end{equation*}
and set $B_n := B(x_n, \, r_n)$. We define inductively a sequence of finite perimeter sets given by the symmetric differences, that is,
\begin{equation*} \begin{cases} E_0 = B_0, \\[0.5em] E_n = E_{n-1} \: \triangle \: B_n. \end{cases} \end{equation*}
We claim that $\left(\mathbbm{1}_{E_n}\right)_{n \in \N} \subset \mathrm{BV}(\R^2)$ is a Cauchy sequence with respect to the $\| \cdot \|_{\mathrm{BV}}$-norm. If the claim holds, then we can infer that
\begin{equation*} \mathbbm{1}_{E_n} \xrightarrow{L^1} \mathbbm{1}_E, \end{equation*}
and also that
\begin{equation*} D (\mathbbm{1}_E) = \pm \, \nu \, \mathbbm{1}_\Sigma \cdot \mathcal{H}^1, \end{equation*}
where 
\begin{equation*} \Sigma = \bigcup_{n \in \N} \partial B(x_n, \, r_n) \end{equation*}
is a $1$-rectifiable surface. In fact, it is enough to notice that
\begin{equation*} \partial E_n = \bigcup_{i = 0}^n \partial B_i \implies \mathcal{H}^1 \left( \partial E_n \right) \leq 2\pi L, \end{equation*}
and
\begin{equation*}D \mathbbm{1}_{E_n} = \pm \, \nu_n \, \mathbbm{1}_{\partial E_n} \cdot \mathcal{H}^1.\end{equation*}

\paragraph{Step 2.} We now prove the claim. By definition of symmetric difference it turns out that
\begin{equation*} \mathbbm{1}_{E_n} = \mathbbm{1}_{B_0} + \left( \mathbbm{1}_{B_1 \setminus E_0} - \mathbbm{1}_{B_1 \cap E_0} \right) + \dots =: \sum_{i = 0}^n u_i, \end{equation*}
and therefore it is equivalent to show that
\begin{equation*} \sum_{i = 0}^{+ \infty} \|u_i\|_{\mathrm{BV}(\R^n)} < + \infty. \end{equation*}
The $L^1$-norm of $u_i$ may be roughly estimated by the area of the ball $B_i$, which means that
\begin{equation*} \|u_i\|_{L^1(\R^n)} = \pi \cdot r_i^2. \end{equation*}
In a similar way, the total variation of the derivative may be estimated by
\begin{equation*} \|D u_i\| = 2 \pi \cdot r_i + 2 \mathcal{H}^1 \left( \partial E_{i-1} \cap B_i \right) \leq 2 \pi (i + 1) r_i,\end{equation*}
and therefore the assumption \eqref{ex.fps.1} proves the claim.

In particular, the support of the measure $D \mathbbm{1}_E$ is the whole plane $\R^2$, and hence every representative of $E$ has boundary containing the support of $\mathbbm{1}_E$, that is,
\begin{equation*} " \partial \widetilde{E} = \R^2. " \end{equation*}
We finally infer that no representative $\widetilde{E}$ of $E$ has $\mathcal{H}^1$-finite boundary.\end{proof}

\paragraph{Main Result.} In this final paragraph, we are finally ready to state and prove the so-called \textit{structure theorem} due to De Giorgi and Federer. The proof is quite involved. Therefore we will need a lot of work to get through it.

\begin{definition}[Essential Boundary] \index{essential boundary} Let $E$ be a Borel set in $\R^n$. The \textit{essential boundary} of $E$ in $\R^n$ is defined by
\begin{equation*} \partial_\ast E := \left\{ x \in \partial E \: \left| \: \text{$\Theta_n \left(E, \, x \right) \neq 0, \, 1$ or $\Theta_n \left(E, \, x \right)$ does not exist} \right. \right\}, \end{equation*}
where $\Theta_n$ denotes the $n$-dimensional Hausdorff density.\end{definition}

\begin{remark}The essential boundary is, in general, strictly contained in the boundary. For example, if $\partial E$ is locally the graph of a function $f$ that admits a cuspid $p$, then $p \in \partial E \setminus \partial_\ast E$. \end{remark}

\begin{definition}[Normal Vector Field] If $\Sigma$ is a $(n - 1)$-rectifiable set, then $\eta : \Sigma \longrightarrow \R^n$ is a \textit{normal vector field} if $|\eta(x)| = 1$ for every $x \in \Sigma$ and
\begin{equation*}\eta(x) \perp \mathrm{Tan}(\Sigma, \, x) \quad \text{for $\mathcal{H}^{n-1}$-almost every $x \in \Sigma$}. \end{equation*} \end{definition}

\begin{theorem}[De Giorgi-Federer] \label{th:dgf} Let $E$ be a finite perimeter set in $\R^n$ (or in $\Omega$). Then the following assertions hold: \mbox{}
\begin{enumerate}[label=\textbf{(\arabic*)}]
\item The essential boundary $\partial_\ast E$ is $\mathcal{H}^{n - 1}$-finite.
\item The essential boundary $\partial_\ast E$ is $(n-1)$-rectifiable.
\item There exists a normal vector field $\eta$ which is "inner normal" in the following sense:
\begin{equation} \label{degiorgifederer1} \mathcal{L}^n \left(  \left( E \, \triangle \, (x + M_{\eta(x)}^+) \right) \cap B_{x, \, r} \right) \ll r^n \quad \text{as $r \to 0^+$}\end{equation}
for $\mathcal{H}^{n-1}$-almost every $x \in \partial_\ast E$.
\item The derivative of the characteristic function of $E$ is given by
\begin{equation} \label{degiorgifederer2} D (\mathbbm{1}_E) = \mathbbm{1}_{\partial_\ast E} \, \eta \cdot \mathcal{H}^{n-1} \end{equation} 
\end{enumerate}
\end{theorem}

\begin{remark}The assertion \eqref{degiorgifederer1} is completely equivalent to the following one:
\begin{equation} \label{degiorgifederer3} \frac{1}{r} (E - x) \xrightarrow{L_{loc}^1(\R^n)}_{r \to 0^+} M_{\eta(x)}^+.\end{equation}\end{remark}

\begin{corollary}Under the assumptions of the \hyperref[th:dgf]{Theorem \ref{th:dgf}} it turns out that
\begin{equation*} \Theta_n(E, \, x) = \frac{1}{2} \end{equation*}
for $\mathcal{H}^{n - 1}$-almost every $x \in \partial_\ast E$. \end{corollary}

\begin{corollary}Under the assumptions of the \hyperref[th:dgf]{Theorem \ref{th:dgf}} it turns out that
\begin{equation*}  \mathrm{Per} (E) = \mathcal{H}^{n-1} \left( \partial_\ast E \right) < \infty. \end{equation*}\end{corollary}

\begin{figure}[h]
\centering
\includegraphics[width=14cm, height=8cm]{images/TGMFPS215.png}
\caption{Intuitive idea of the statement of the De Giorgi-Federer structure theorem. The picture is slightly misleading: the vector space $M_\eta(x)^+$ is the whole space "under" the tangent to $\partial_\ast E$ at $x$.}
\end{figure}

In a different paper, Federer proved that also a sort of converse of this theorem holds true, but we will only state it here since the proof is far beyond the reach of this course.

\begin{theorem}[Federer] Let $E \subseteq \R^n$ be a Borel set such that $\mathcal{H}^{n - 1}(\partial_\ast E) < + \infty$. Then $E$ is a finite perimeter set. \end{theorem}

\begin{exercise}Let $E \subseteq \R^n$ be a set with empty essential boundary, that is $\partial_\ast E = \varnothing$. Prove that either $|E| = 0$ or $|\R \setminus E| = 0$. \end{exercise}

\begin{definition}[Reduced Boundary] \index{reduced boundary} Let $E$ be a finite perimeter set in $\R^n$. The \textit{reduced boundary} of $E$ in $\R^n$, denoted by $\partial^\ast E$, is the set of all $x \in \R^n$ such that the Radon-Nikodym density
\begin{equation*}\eta(x) := \frac{\mathrm{d} (D \mathbbm{1}_E)}{\mathrm{d} |D \mathbbm{1}_E|}(x) \end{equation*}
exists and has norm equal to one.\end{definition}

\begin{remark}Equivalently, if we denote by $\tau \cdot \mu$ the weak derivative of $\mathbbm{1}_E$, then $\eta(x)$ is the $L^1$-approximate limit, with $|\tau| = 1$, of $\tau$ at $\mathcal{H}^{n-1}$-almost every $x$, that is,
\begin{equation*}\dashint_{B(x, \, r)} |\tau(y) - \eta(x)| \, \mathrm{d}\mu(y) \xrightarrow{r \to 0^+} 0. \end{equation*} \end{remark}

\begin{example}In general, the reduced boundary is a proper subset of the boundary. For example, if $E$ is a rhombus, then the four vertexes are not in the reduced boundary $\partial^\ast E$.\end{example}

\begin{theorem}[De Giorgi] \label{th:dg} Let $E \subseteq \R^n$ be a finite perimeter set in $\R^n$, and set $D( \mathbbm{1}_E) := \tau \cdot \mu$. Then the following assertions hold true: \mbox{}
\begin{enumerate}[label=\textbf{(\arabic*)}]
\item The Radon-Nikodym density $\eta(x)$ coincides $\mu$-almost everywhere with $\tau(x)$. Moreover, the measure $\mu$ is supported in the reduced boundary, that is,
\begin{equation} \label{degiorgi0} \mu \left(\R^n \setminus \partial^\ast E \right) = 0. \end{equation}
\item The reduced boundary $\partial^\ast E$ is $(n-1)$-rectifiable.
\item The measure $\mu$ is the restriction of the $(n-1)$-dimensional Hausdorff measure to the reduced boundary, that is,
\begin{equation} \label{degiorgi1} \mu = \mathcal{H}^{n - 1} \restr \partial^\ast E.\end{equation}
\item The Radon-Nikodym density $\eta(x)$ is normal to the reduced boundary at $x$ for every $x \in \partial^\ast E$, that is,
\begin{equation*} \eta(x)^\perp = \mathrm{Tan}_\ast \left(\partial^\ast E, \, x \right) \end{equation*}
where $\mathrm{Tan}_\ast \left(-, \, \cdot \right)$ denotes the approximate tangent plane.
\item The normal vector field $\eta$ satisfies the following property:
\begin{equation} \label{degiorgi2} \mathcal{L}^n \left(  \left( E \, \triangle \, (x + M_{\eta(x)}^+) \right) \cap B_{x, \, r} \right) \ll r^n \quad \text{as $r \to 0^+$}\end{equation}
for every $x \in \partial^\ast E$. Equivalently,
\begin{equation} \label{degiorgi3} \frac{1}{r} (E - x) \xrightarrow{L_{loc}^1(\R^n)}_{r \to 0^+} M_{\eta(x)}^+.\end{equation}
\end{enumerate}
\end{theorem}

\begin{corollary}Under the assumptions of the \hyperref[th:dg]{Theorem \ref{th:dg}} it turns out that
\begin{equation*} \Theta_n(E, \, x) = \frac{1}{2} \end{equation*}
for every $x \in \partial^\ast E$. \end{corollary}

The proof of \hyperref[th:dg]{De Giorgi's Theorem \ref{th:dg}} will follow reasonably quickly from a sequence of many technical lemmas that we will patently state and prove here.

In order to ease the statements of the following results, we will fix a point $x \in \partial^\ast E$ and denote it by $0$ unless otherwise stated.

\begin{lemma} \label{lemma:dg1} There exists an universal constant $c := c(n) > 0$ such that
\begin{equation} \label{eq:dg1} \mu(B_r) \leq c \cdot r^{n - 1}. \end{equation}
In particular, it turns out that there exists an universal constant $c^\prime := c^\prime(n) > 0$ such that
\begin{equation} \label{eq:dg2} \mu(B_r) \leq c^\prime \cdot \mathcal{H}^{n - 1} \left( \partial B_r \cap E \right). \end{equation} \end{lemma}

\begin{proof}Set $E_r := E \cap B_r$.

\paragraph{Step 1.} We claim that, for almost every $r > 0$, the weak derivative of the characteristic function satisfies an additive formula:
\begin{equation} \label{fps.st.claim1} D (\mathbbm{1}_{E_r}) = \mathbbm{1}_{B_r} \cdot D (\mathbbm{1}_E) + \mathbbm{1}_{E} \cdot D (\mathbbm{1}_{B_r}). \end{equation}
We notice that the integral of $\eta(0)$ with respect to the measure $D (\mathbbm{1}_{E_r})$ is given by
\begin{equation*} \int_{\R^n} \eta(0) \, \mathrm{d}D (\mathbbm{1}_{E_r}) = 0,\end{equation*}
since $\eta(0)$ is constant. On the other hand, it follows from \eqref{fps.st.claim1} that
\begin{equation*} \begin{aligned} \int_{\R^n} \eta(0) \, \mathrm{d}D(\mathbbm{1}_{E_r}) & = \int_{B_r} \eta(0) \eta(x) \, \mathrm{d}\mu(x) + \int_{E} \eta(0) \mathrm{d}D( \mathbbm{1}_{B_r} )(x) =
\\[1em] & = \int_{B_r} \eta(0) \eta(x) \, \mathrm{d}\mu(x) + \int_{\partial B_r \cap E} \eta(0) \nu_{inn}(x) \, \mathrm{d}\mathcal{H}^{n - 1}(x), \end{aligned}  \end{equation*}
where $\nu_{inn}(x)$ denotes the inner normal vector to the sphere at $x$. In particular, it turns out that
\begin{equation*} \begin{aligned} \int_{B_r} \eta(0) \eta(x) \, \mathrm{d}\mu(x) & = - \int_{\partial B_r \cap E} \eta(0) \nu_{inn}(x) \, \mathrm{d}\mathcal{H}^{n - 1}(x) \leq \\[1em] & \leq \mathcal{H}^{n - 1}(\partial B_r) \sim \alpha_n r^{n - 1}, \end{aligned} \end{equation*}
where the inequality follows from the fact that both $\nu_{inn}$ and $\eta$ are vectors of norm less or equal than $1$.

In conclusion, we notice that for $r \to 0^+$ the left-hand side of the inequality above is asymptotically equivalent to $\mu(B_r)$, which means that there exists a $r_0 > 0$ small enough such that
\begin{equation*} \frac{1}{2} \, \mu(B_r) \leq \int_{B_r} \eta(0) \eta(x) \, \mathrm{d}\mu(x)  \leq \mathcal{H}^{n - 1}(\partial B_r) \sim c(n) \cdot r^{n - 1},\end{equation*}
for almost every $0 < r < r_0$.

\paragraph{Step 2.} We now sketch the proof of the claim \eqref{fps.st.claim1}. More precisely, we show that the formula holds for every f.p.s. $E$ and almost every $r  > 0$ - even when $0$ does not belong to the reduced boundary -.

\paragraph{Step 2.1.} First, we notice that \eqref{fps.st.claim1} holds for every smooth f.p.s. $E$ that is also transversal to the boundary of the ball $B_r$. Indeed, one can easily prove that
\begin{equation*} \partial E_r = \left(\partial B_r \cap \overline{E} \right) \cup \left( \overline{B_r} \cap \partial E \right).\end{equation*}

\paragraph{Step 2.2.} Let $E^n$ be an approximation of $E$ by smooth sets, that is
\begin{equation*}E^n \xrightarrow{L_{loc}^1} E. \end{equation*}
Choose $r > 0$ in such a way that $\partial E_r^n$ is transversal to the sphere $\partial B_r$ for every $n \in \N$. The previous step proves that $E^n$ satisfies \eqref{fps.st.claim1} for every $n \in \N$, which means that we can pass the identity to the limit\footnote{We will not give any detail, but one needs to be careful to the notion of convergence that comes into play here.}.  \end{proof}

\begin{lemma} \label{lemma:dg2} There are two universal constants $c := c(n), \, c^\prime := c^\prime(n) > 0$ such that for every $r > 0$ small enough
\begin{equation} \label{eq:dg3} c \cdot r^n \leq |E \cap B_r| \leq c^\prime \cdot r^n, \end{equation}
and
\begin{equation} \label{eq:dg4} c \cdot r^n \leq |E^c \cap B_r| \leq c^\prime \cdot r^n. \end{equation}\end{lemma}

\begin{remark}It is easy to prove that the complement of a f.p.s. is also a f.p.s. with the same weak derivative up to null sets. In particular, the inequality \eqref{eq:dg3} implies the inequality \eqref{eq:dg4} by passing to the complement. \end{remark}

\begin{proof} Set $E_r := E \cap B_r$.

\paragraph{Step 1.} We claim that the following \textit{isoperimetrical inequality} holds: For every $r > 0$ there exists an universal constant $c := c(n) > 0$ such that
\begin{equation} \label{claim.iso} v(r)^{ \frac{n - 1}{n} } \leq c \cdot \mathrm{Per} (E_r), \end{equation}
where $v(r)$ denotes the Lebesgue measure of $E_r$. If the inequality holds, then it is easy to see that
\begin{equation*} \begin{aligned} v(r)^{ \frac{n-1}{n} } & \leq c_1 \cdot \left\| D( \mathbbm{1}_{E_r}) \right\| \leq
\\[1em] & \leq c_2 \left( \mathrm{Per}_{B_r}(E) + \mathcal{H}^{n -  1}\left(E \cap \partial B_r \right) \right) \leq
\\[1em] & \leq c_3 \cdot \mathcal{H}^{n - 1}\left(E \cap \partial B_r \right) = c_3 \cdot \dot{v}(r), \end{aligned} \end{equation*}
since
\begin{equation*}\mathrm{Per}_{B_r}(E) = |D(\mathbbm{1}_E)|(B_r) = \mu(B_r), \end{equation*}
and clearly $\mu(B_r)$ and $\mathcal{H}^{n - 1}\left(E \cap \partial B_r \right)$ are comparable quantities as a consequence of the previous result.

\paragraph{Step 2.} The symbol $\dot{v}$ denotes the classic derivative of $v$ since a Lipschitz function is almost everywhere differentiable (by \hyperref[th:rad]{Rademacher Theorem \ref{th:rad}}). The reader may prove, as an exercise, that the formula used above holds:
\begin{equation} \label{claim.iso2} \dot{v}(r) = \mathcal{H}^{n - 1} \left(E \cap \partial B_r \right) \quad \text{for almost every $r > 0$.} \end{equation}

\paragraph{Step 3.} In conclusion, we notice that $v(r)$ is an almost everywhere differentiable function which satisfies the following inequality:
\begin{equation*} v(r)^{ \frac{n - 1}{n} } \leq c_3 \cdot \dot{v}(r). \end{equation*}
A straightforward computation yields to
\begin{equation*} v(r) \geq c(n) \cdot r^n, \end{equation*}
which is exactly the nontrivial part of \eqref{eq:dg3}. The trivial one, on the other hand, follows immediately from the worst estimate possible:
\begin{equation*} v(r) \leq |B_r| \sim c^\prime(n) \cdot r^n. \end{equation*}

\paragraph{Step 4.} It remains to prove the the claimed isoperimetrical inequality \eqref{claim.iso}; surprisingly, it follows easily from a far more general theorem.

Precisely, one can prove that there exists a universal constant $c := c(n)$ such that for any given $F$ bounded f.p.s., it turns out that
\begin{equation*} |F|^{\frac{n - 1}{n}} \leq c(n) \cdot \mathrm{Per}(F). \end{equation*}
Recall that the immersion
\begin{equation*} \mathrm{BV}(\Omega) \hookrightarrow L^{1^\ast}(\Omega), \end{equation*}
where $1^\ast = \frac{n }{n - 1}$, is continuous (and compact). Therefore, there exists a universal constant $c^\prime := c^\prime(n) > 0$ such that
\begin{equation*} \| u \|_{L^{ \frac{n}{n- 1} }(\Omega)} \leq c^\prime \cdot \| u \|_{\mathrm{BV}(\Omega)}. \end{equation*}
The reader may prove that one can always replace the $\mathrm{BV}(\Omega)$-norm with the total variation $ \|Du \|$ if $u$ is, for example, compactly supported. The isoperimetrical inequality follows immediately by taking $u := \mathbbm{1}_F$ for every \textbf{bounded} f.p.s. $F$.\end{proof}

\begin{lemma} \label{lemma:dg3} Let $E_r := \frac{1}{r} (E - x) = \frac{1}{r} E$. Then
\begin{equation} \label{eq:dg6} E_r \xrightarrow{L_{loc}^1} M_{\eta(0)}^+, \end{equation}
or, equivalently, it turns out that
\begin{equation} \label{eq:dg7} \left| (E \, \triangle \, M_{\eta(0)}^+) \cap B_r \right| \ll r^n \quad \text{as $r \to 0^+$}. \end{equation}\end{lemma}

\begin{proof} Denote by $\eta_r$ and $\mu_r$ the rescaling of $\eta$ and $\mu$ respectively, in such a way that
\begin{equation*} D ( \mathbbm{1}_{E_r} ) = \eta_r \cdot \mu_r. \end{equation*}

\paragraph{Step 1.} By \hyperref[lemma:dg1]{Lemma \ref{lemma:dg1}} it turns out that
\begin{equation} \label{eq.1.1} \mu_r(B_1) = \frac{\mu(B_r)}{r^{n - 1}} \leq c(n) \implies \mu_r(B_R) \leq c(n) \cdot R^{n - 1} \end{equation}
for every $R > 0$. Similarly, by \hyperref[lemma:dg2]{Lemma \ref{lemma:dg2}} it turns out that
\begin{equation*} c_1(n) \leq |E_r \cap B_1| \leq c_1^\prime(n), \end{equation*}
from which it follows that
\begin{equation} \label{eq.1.2}c_1(n) \cdot R^n \leq |E_r \cap B_R| \leq c_1^\prime(n) \cdot R^{n} \end{equation}
for every $R > 0$. The compactness property of the f.p.s. (see \hyperref[th:cpfps]{Theorem \ref{th:cpfps}}), together with the inequality \eqref{eq.1.1}, imply that, up to subsequences,
\begin{equation*}\begin{aligned} & \mathbbm{1}_{E_r} \xrightarrow{L_{loc}^1} \mathbbm{1}_{E_0} \\[1em] & D \left( \mathbbm{1}_{E_r} \right) \xrightarrow{\text{locally}} D \left( \mathbbm{1}_{E_0} \right)\end{aligned} \end{equation*} 
for some f.p.s. $E_0$. Since $0$ belongs to the reduced boundary $\partial^\ast E$, we have that
\begin{equation*} \dashint_{B_r} | \eta(x) - \eta(0) | \, \mathrm{d}\mu(x) \xrightarrow{r \to 0^+} 0, \end{equation*}
which means that for every $R > 0$
\begin{equation*} \dashint_{B_R} | \eta_r(x) - \eta(0) | \, \mathrm{d}\mu_r(x) \xrightarrow{r \to 0^+} 0. \end{equation*}
We may assume without loss of generality that $\eta(0) = e_1$, where $\{e_1, \, \dots, \, e_n\}$ is the standard orthonormal basis of $\R^n$. It turns out that
\begin{equation*} D \left( \mathbbm{1}_{E_0} \right) = \eta(0) \cdot \mu_0, \end{equation*}
from which we infer that $D^i \left( \mathbbm{1}_{E_0} \right)$ is zero for every $i = 2, \, \dots, \, n$.

\paragraph{Step 2.} By \hyperref[lemma:dgex1]{Lemma \ref{lemma:dgex1}} it turns out that $\mathbbm{1}_{E_0}$ has a representative which depends only on the first variable $x_1$, and thus we can infer that
\begin{equation*} E_0 = x_0 + M_{\eta(0)}^+ \end{equation*}
for some $x_0 \in \R^n$. It remains to prove that $x_0$ is necessarily equal to zero.

We argue by contradiction: suppose that $x_0$ is not $0$. Then one can find a radius $R > 0$ small enough such that the Lebesgue measure of the intersection $E_0 \cap B_R$ is zero, in contradiction with the inequality \eqref{eq.1.2}. Therefore
\begin{equation*}E_0 = M_{\eta(0)}^+ \qquad \text{and} \qquad E_r \xrightarrow{L_{loc}^1} E_0 = M_{\eta(0)}^+, \end{equation*}
which is exactly what we wanted to prove.\end{proof}

\begin{lemma} \label{lemma:dgex1} Let $f \in L_{loc}^1(\R^n)$ be a function such that
\begin{equation*} D^i f \equiv 0 \qquad \forall \, i = 2, \, \dots, \, n. \end{equation*}
Then there exists $\widetilde{f} \in L_{loc}^1(\R)$ such that
\begin{equation*} \widetilde{f}(x_1) = f(x_1, \, \dots, \, x_n) \quad \text{for $\mathcal{L}^n$-almost every $x = (x_1, \, x_2, \, \dots, \, x_n) \in \R^n$.} \end{equation*} \end{lemma}

\begin{lemma} \label{lemma:dg4} Let $E_r := \frac{1}{r} (E - x) = \frac{1}{r} E$. Then
\begin{equation} \label{eq:dg8} D(\mathbbm{1}_{E_r}) \xrightarrow{\text{locally}} D( \mathbbm{1}_{M_{\eta(0)}^+}) = \eta(0) \, \mathbbm{1}_{\eta(0)^\perp} \cdot \mathcal{H}^{n - 1}.\end{equation}\end{lemma}

\begin{lemma} \label{lemma:dg5} For every $r > 0$ it turns out that
\begin{equation} \label{eq:dg10} \mu(B_r) \sim \alpha_{n - 1} r^{n - 1}.\end{equation}\end{lemma}

\begin{proof} We divide the argument into two steps to ease the notation.

\paragraph{Step 1.} First, we deduce from \hyperref[lemma:dg4]{Lemma \ref{lemma:dg4}} that
\begin{equation} \label{eq:dg11} \left| D(\mathbbm{1}_{E_r}) \right| \xrightarrow{\text{locally} } \left| D( \mathbbm{1}_{M_{\eta(0)}^+}) \right|.\end{equation}
Indeed, if $\eta(0) = e_1$, then
\begin{equation*}\dashint_{B_R} |\eta_r(x) - \eta(0)| \, \mathrm{d}\mu_r(x) \xrightarrow{r \to 0^+} 0, \qquad \forall \, R > 0 \end{equation*}
implies that
\begin{equation*}\left| \frac{\partial \, \mathbbm{1}_{E_r}}{\partial x_1} \right| \xrightarrow{r \to 0^+} \left| \frac{\partial \, \mathbbm{1}_{E_0}}{\partial x_1} \right|,\end{equation*}
which is exactly what we wanted to prove since $E_0 = M_{\eta(0)}^+$.

\paragraph{Step 2.} From \eqref{eq:dg11} it follows that\footnote{Here the reader needs to be careful. The convergence is not true in general, but one can prove that it is enough to ask that $\mathcal{H}^{n-1}( \eta(0)^\perp \cap B_1) = 0$.}
\begin{equation*} \frac{\mu(B_r)}{r^{n-1}} = \mu_r(B_1) \xrightarrow{ \qquad } \mathcal{H}^{n - 1}( \eta(0)^\perp \cap B_1 ) = \alpha_{n - 1} \end{equation*}
for almost every $r > 0$. To conclude the proof, we need the following auxiliary lemma.

\begin{lemma} Let $(\lambda_n)_{n \in \N}$ be a sequence of vector measures on $X$. Assume that $\lambda_n := \tau_n \cdot \mu_n$ converges in the sense of measures (weakly-$\ast$) to a measue $\lambda :=  \tau \cdot \mu$, and assume that there exists a constant vector $v$ such that
\begin{equation*} \int_X |\tau_n - v|(x) \, \mathrm{d}\mu_n(x) \xrightarrow{n \to + \infty} 0. \end{equation*}
Then $\mu_n$ converges to $\mu$ and $\lambda = v \cdot \mu$ almost everywhere. \end{lemma}
\end{proof}

We have finally introduced all the technical tools to prove the statements of the De Giorgi's theorem presented above. 

\begin{proof}[Proof of Theorem \ref{th:dg}] \mbox{}
\begin{enumerate}[label=\textbf{(\arabic*)}]
\item The first statement about the density and the support of $\mu$ is an immediate consequence of the argument used in the next point.
\item We sketch two different proofs of the fact that the reduced boundary is $(n - 1)$-rectifiable. The first proof is due to De Giorgi, while the second one relies more on the sequence of technical lemmas above.
\begin{enumerate}[label=\textbf{(\alph*)}]
\item For every $r_0 > 0$ and $\delta_0 > 0$ we consider
\begin{equation*} S_{r_0, \, \delta_0} := \left\{ x \in \partial^\ast E \: \left| \: \text{$\left| ( E \, \triangle (x + M_{\eta(x)}^+) ) \cap B(x, \, r) \right| \leq \delta_0 r^n$ for every $r \leq r_0$}. \right. \right\}. \end{equation*}
The idea is to prove that each $S_{r_0, \, \delta_0}$ is $(n - 1)$-rectifiable. In fact, we know that the countable union of $d$-rectifiable set is also $d$-rectifiable, and hence the thesis follows from the fact that
\begin{equation*} \partial^\ast E = \bigcup_{(n, \, m) \in \N^2} S_{\frac{1}{n}, \, m}. \end{equation*}
\item The support of the measure $\mu$ is contained in the reduced boundary $\partial^\ast E$; hence by \hyperref[lemma:dg5]{Lemma \ref{lemma:dg5}} we can infer that
\begin{equation*} \mu \left( B(x, \, r) \right) \sim \alpha_{n - 1} r^{n - 1} \quad \text{for $\mu$-almost every $x \in \partial^\ast E$}. \end{equation*}
We claim that there exists $F \subseteq \mathrm{spt}(\mu)$ such that
\begin{equation*}c_1 \, \mathbbm{1}_F \cdot \mathcal{H}^{n - 1} \leq \mu \leq c_2 \, \mathbbm{1}_F \cdot \mathcal{H}^{n - 1} \qquad \text{and} \qquad \mu \left( \partial^\ast E \setminus F \right) = 0. \end{equation*}
Indeed, it is enough to notice that $\mu \left(E \setminus \partial^\ast E \right) = 0$ as the support of $\mu$ is contained in the reduced boundary. Moreover, the $(n-1)$-dimensional upper density is zero at almost every point of $F$, that is,
\begin{equation*} \Theta^\ast(\mu, \, x) = 0 \quad \text{for $\mathcal{H}^{n - 1}$-almost every $x \in F$}, \end{equation*}
and this implies that
\begin{equation*} \begin{cases} \mathcal{H}^{n - 1}\left(\partial^\ast E \setminus F \right) = 0, \\[0.5em] \mathcal{H}^{n - 1}\left(\partial^\ast E \, \triangle \, F \right) = 0. \end{cases} \end{equation*}
The measure $\mu$ is thus equivalent to the restriction of the $\mathcal{H}^{n-1}$ measure to the reduced boundary, that is, there are two constants $c_1, \, c_2 > 0$ such that
\begin{equation*}c_1^\prime \, \mathbbm{1}_{\partial^\ast E} \cdot \mathcal{H}^{n - 1} \leq \mu \leq c_2^\prime \, \mathbbm{1}_{\partial^\ast E} \cdot \mathcal{H}^{n - 1} \implies \mu \sim \mathcal{H}^{n - 1} \restr \partial^\ast E. \end{equation*}
We deduce that the blowup of $\mu$ at $x$ is given by
\begin{equation*} \mathbbm{1}_{\eta(x)^\perp} \cdot \mathcal{H}^{n - 1}, \end{equation*}
which means that $\eta(x)^\perp$ is the approximate tangent plane of $\partial^\ast E$ at $x$.

Moreover, the lower density is bounded from below; therefore, by \hyperref[crthasjdkso]{Corollary \ref{crthasjdkso}}, we can finally infer that $\partial^\ast E$ is $(n - 1)$-rectifiable.
\end{enumerate}
\item In the previous point we have proved that $\mu$ is equivalent to $\mathbbm{1}_{\partial^\ast E} \cdot \mathcal{H}^{n - 1}$, which means that there exists a function $g$ such that
\begin{equation*}\mu = \mathbbm{1}_{\partial^\ast E} \, g \cdot \mathcal{H}^{n - 1}. \end{equation*}
Therefore it is enough to show that $g$ is equal to $1$ $\mathcal{H}^{n-1}$-almost everywhere. The $(n-1)$-dimensional upper density of the Hausdorff measure is given by
\begin{equation*} \Theta_{n - 1}^\ast(\partial^\ast E, \, x) = 1 \quad \text{for $\mathcal{H}^{n - 1}$-almost every $x \in \partial^\ast E$}, \end{equation*}
while the $(n- 1)$-dimensional upper density of $\mu$ is given by
\begin{equation*} \Theta_{n - 1}^\ast(\mu, \, x) = 1 \quad \text{for $\mu$-almost every $x \in \partial^\ast E$}. \end{equation*}
In particular, it follows that $g = 1$ almost everywhere (with respect to either $\mu$ or $\mathcal{H}^{n - 1}$).
\item This assertion also follows immediately from the second proof of \textbf{(2)}.
\item This assertion simply summarize the content of \hyperref[lemma:dg3]{Lemma \ref{lemma:dg3}}.
\end{enumerate} \end{proof}

We are finally ready to prove the \hyperref[th:dgf]{De Giorgi-Federer Theorem \ref{th:dgf}}. The reduced boundary is a subset of the essential boundary, therefore everything follows from the previous result if we are able to prove that
\begin{equation*} \mathcal{H}^{n - 1} \left( \partial_\ast E \setminus \partial^\ast E \right) = 0. \end{equation*}

\begin{proposition}[Isoperimetrical Inequality] \label{prop:dgiso}Let $E \subseteq \R^n$ be a finite perimeter set in $B(x, \, r)$. Then there exists an universal constant $c := c(n) > 0$ such that\footnote{The inequality presented here is \textbf{not} sharp! The result can be refined, but it is not necessary for our purposes.}
\begin{equation} \label{prop:dgisoeq} \left( |E| \wedge |B(x, \, r) \setminus E| \right)^{\frac{n - 1}{n}} \leq c \cdot \mathrm{Per}_{B(x, \, r)}(E). \end{equation} \end{proposition}

\begin{proof} The Sobolev Embedding Theorem implies that the immersion
\begin{equation*} \mathrm{BV}(B(x, \, r)) \hookrightarrow L^{1^\ast}(B(x, \, r)), \end{equation*}
where $1^\ast = \frac{n}{n - 1}$, is continuous. Therefore, there exists a universal constant $c := c(n) > 0$ (which does not depend on the radius and the center of the ball) such that
\begin{equation*} \| u \|_{L^{ \frac{n}{n - 1} }(B(x, \, r))} \leq \| u \|_{\mathrm{BV}(B(x, \, r))}. \end{equation*}
We replace the $\mathrm{BV}(B(x, \, r))$ norm with the equivalent one
\begin{equation*} \| Du \| + \left| \dashint_{B(x, \, r)} u \right|. \end{equation*}
It turns out that
\begin{equation*} \| u - \dashint_{B(x, \, r)}u \|_{L^{ \frac{n}{n - 1} }(B(x, \, r))} \leq c(n) \, \|Du \|. \end{equation*}
By symmetry, we may assume without loss of generality that $|E| < |B(x, \, r) \setminus E|$. Then it is easy to prove that
\begin{equation*} \mathbbm{1}_E(x) - \frac{|E|}{|B(x, \, r)|} > \frac{1}{2}, \qquad \text{if $x \in E$}, \end{equation*}
and the thesis follows from the inequality above since
\begin{equation*}  c(n) \, \|Du \| \geq  \| u - \dashint_{B} u\|_{L^{ \frac{n}{n - 1} }(B(x, \, r))} \geq \frac{1}{2} |E|^{ \frac{n - 1}{n}}. \end{equation*}
\end{proof}

\begin{lemma} \label{lemma:dg10}Under the assumptions of the \hyperref[th:dgf]{Theorem \ref{th:dgf}}, it turns out that
\begin{equation*} \mu \left(B(x, \, r) \right) \ll r^{n - 1} \implies \Theta_n^\ast(E, \, x) \in \{0, \, 1\}. \end{equation*}
In particular, the point $x$ does not belong to the essential boundary $\partial_\ast E$. \end{lemma}

\begin{proof} Let $x \in \R^n$ be a point such that $\Theta_n(E, \, x) \neq 0, \, 1$ (i.e., $x$ belongs to the essential boundary). Then there exists $\delta > 0$ and a sequence $(r_n)_{n \in \N}$ converging to $0$ such that either
\begin{equation*} \exists \: (r_{n_k})_{k \in \N} \: : \: r_k \xrightarrow{k \to + \infty} 0 \quad \text{and} \quad \frac{|E \cap B(x, \, r_{n_k})|}{r_{n_k}^n} \geq \delta, \end{equation*}
or
\begin{equation*} \exists \: (r_{n_k})_{k \in \N} \: : \: r_k \xrightarrow{k \to + \infty} 0 \quad \text{and} \quad \frac{|E \cap B(x, \, r_{n_k})|}{r_{n_k}^n} \leq \alpha_n - \delta, \end{equation*}
Either way the function
\begin{equation*} \R \ni r \longmapsto \frac{|E \cap B(x, \, r)|}{r^n} \end{equation*}
is continuous; hence there exists $\lambda^\prime < \lambda$ real numbers such that
\begin{equation*}\lambda^\prime \leq \frac{|E \cap B(x, \, r_{n_k})|}{r_{n_k}^n} < \lambda \end{equation*}
for $r_{n_k} \xrightarrow{k \to + \infty} 0$. On the other hand, the isoperimetrical inequality \eqref{prop:dgisoeq} implies that
\begin{equation*}|E \cap B(x, \, r) \setminus E|^{\frac{n-1}{n}} \leq c \cdot \mathrm{Per}_{B(x, \, r)}(E) = \mu \left(B(x, \, r) \right), \end{equation*}
from which we derive a contradiction with the assumption $\mu(B(x, \, r)) \ll r^{n-1}$ by noticing that
\begin{equation*}\lambda^\prime r^{n-1} \leq |E \cap B(x, \, r) \setminus E|^{\frac{n-1}{n}} \leq c \cdot \mathrm{Per}_{B(x, \, r)}(E) = \mu \left(B(x, \, r) \right) \end{equation*}
for every $r > 0$ small enough.\end{proof}

\begin{lemma} Under the assumptions of the \hyperref[th:dgf]{Theorem \ref{th:dgf}}, it turns out that
\begin{equation*} \mathcal{H}^{n - 1} \left( \partial_\ast E \setminus \partial^\ast E \right) = 0. \end{equation*} \end{lemma}

\begin{proof} Equivalently, we may prove that for $\mathcal{H}^{n-1}$-almost every point $x \notin \partial^\ast E$, it turns out that $x$ does not belong to $\partial_\ast E$ as well. By \hyperref[lemma:dg10]{Lemma \ref{lemma:dg10}}, it suffices to prove that
\begin{equation*} \mu \left( B(x, \, r) \right) \ll r^{n-1},\end{equation*}
but this follows easily from the fact that
\begin{equation*} \mu = \mathcal{H}^{n-1} \restr \partial^\ast E. \end{equation*} \end{proof}

\section{Back to the Capillarity Problem}

We are finally ready to prove the existence of a solution to the capillarity problem for any $\sigma \in [- 1, \, 1]$ using the new notion of essential boundary.

\paragraph{Framework.} Fix $\Omega \subset \R^n$ regular bounded open set, and fix a volume
\begin{equation*} 0 < m < |\Omega|. \end{equation*}
The capillarity energy is given by
\begin{equation} \label{cap:fun} \mathscr{F}(E) := \mathcal{H}^{n - 1}(\Sigma^f) + \sigma \, \mathcal{H}^{n - 1}(\Sigma^c), \end{equation}
where we define
\begin{equation*} \begin{cases} \Sigma^f := \partial_\ast E \cap \Omega, \\[0.5em] \Sigma^c := \partial_\ast E \cap \partial \Omega. \end{cases} \end{equation*}

\begin{theorem} The functional \eqref{cap:fun} is lower semi-continuous on the set of all finite perimeter set. Moreover, there exists (at least) a minimum point $E$ for $\mathscr{F}$ for every $\sigma \in [-1, \, 1]$. \end{theorem}

\begin{proof}We split the proof into two cases: positive and negative $\sigma$.

\paragraph{Case "$1 \leq \sigma \geq 0$":} The functional \eqref{cap:fun} can be equivalently written as
\begin{equation} \label{cap:fun1} \begin{aligned} \mathscr{F}(E) & = \sigma \, \mathcal{H}^{n - 1}(\partial_\ast E) + (1 - \sigma) \, \mathcal{H}^{n - 1} \left(\Sigma^f \right) = \\[1em] & = \sigma \cdot \mathrm{Per}(E) + (1 - \sigma) \cdot \mathrm{Per}_\Omega(E). \end{aligned} \end{equation}
In particular, the functional \eqref{cap:fun1} is equal to the convex sum of two lower semi-continuous functionals, which means that it is also lower semi-continuous.

The functional $\mathscr{F}$ is also coercive. Indeed, let $(E_n)_{n \in \N} \subset \mathcal{X}$ be a uniformly bounded sequence, that is, there exists a constant $M \in \R$ such that
\begin{equation*} \mathscr{F}(E_n) \leq M, \qquad \forall \, n \in \N. \end{equation*}
Then, up to subsequences, there exists a finite perimeter set $E$ such that $E_n \xrightarrow{ \mathcal{X} } E$. If $\sigma$ is strictly greater than $0$, then the perimeters are equibounded
\begin{equation*} \sigma \cdot \mathrm{Per}(E_n) \leq \mathscr{F}(E_n) \leq M, \end{equation*}
which means that we can apply a compactness theorem for f.p.s. and conclude that the functional is coercive. On the other hand, if $\sigma = 0$, then the coerciveness follows immediately from the following estimate:
\begin{equation*} \mathrm{Per}(E_n) \leq \mathcal{H}^{n - 1}(\partial \Omega) + \mathrm{Per}_\Omega(E_n) \leq \mathcal{H}^{n - 1}(\partial \Omega) + M < + \infty. \end{equation*}

\paragraph{Case "$-1 \geq \sigma \leq 0$":} The idea is to pass everything to the complement, that is, we consider as a variable for the functional \eqref{cap:fun} the set $E^c := \Omega \setminus E$ instead of $E$. In particular, we notice that there are nice relations between the surfaces, that is,
\begin{equation*} \begin{cases} \Sigma_c^f = \Sigma^f, \\[0.5em] \Sigma_c^c = \partial \Omega \setminus \Sigma^c. \end{cases} \end{equation*}
It turns out that
\begin{equation} \label{cap:fun2} \begin{aligned} \mathscr{F}(E) & = \sigma \, \mathcal{H}^{n - 1}(\partial \Omega) - \sigma \, \mathcal{H}^{n - 1}(\partial \Omega) + \mathscr{F}(E) = \\[1em] & = \sigma \, \mathcal{H}^{n-1}(\partial \Omega) + \mathcal{H}^{n - 1}(\Sigma_c^f) + |\sigma| \, \mathcal{H}^{n - 1}(\Sigma_c^c) =
\\[1em] & = \mathscr{F}_{|\sigma|}(E^c) + \sigma \, \mathcal{H}^{n - 1}(\partial \Omega). \end{aligned} \end{equation}
Since $\sigma \, \mathcal{H}^{n - 1}(\partial \Omega)$ is a constant, we can immediately reduce to the previous case $1 \geq |\sigma| \geq 0$.
\end{proof}

\begin{remark}The restriction of \eqref{cap:fun} to any smooth subset $\Omega$ in $\mathcal{X}$ agrees with the classical capillarity energy presented in the previous section, and the relaxation of $\mathscr{F}$ to $\mathcal{X}$ is given by $\mathscr{F}$ itself. Therefore every minimizing sequence, made up of smooth sets, converges to the minimum of $\mathscr{F}$ on $\mathcal{X}$. \end{remark}

\begin{remark}Up to dimension $8$, the inner regularity theory of the capillarity problem works just fine; the boundary regularity, on the other hand, is quite messy. \end{remark}

\begin{remark}A similar problem arises when dealing with Plateau's type problems. More precisely, the surface
\begin{equation*} \Sigma = \left(\partial_\ast E_{min} \right) \cap \overline{\Omega} \end{equation*}
is closed (in $\Omega)$, and it is also analytic up to dimension $7$. As we shall prove at the end of the course, starting from dimension $8$, one can find a counterexample to this statement (actually, it is possible to find a minimal surface such that it is analytic outside of a closed singular set of dimension $n - 8$).  \end{remark}
\chapter{Appendix}

\section{First Variation of a Functional}

\paragraph{Framework.} Let $\Sigma$ be a $d$-rectifiable and $\mathcal{H}^d$-finite set in $\R^n$, and let $(\Phi_t)_{t \in \R}$ be a one parameter family of diffeomorphisms of  $\R^n$ such that $\Phi_0 = \mathrm{id}_{\R^n}$. Let
\begin{equation*} v(x) := \frac{\mathrm{d} \Phi_t(x)}{\mathrm{d}t} \, \big|_{t = 0}, \end{equation*}
and suppose that $\Sigma_t := \Phi_t(\Sigma)$ is the collection of all the competitors (of $\Sigma$) to a certain minimum problem.

\begin{lemma}Let $x \in \Sigma$ be a point, and let $\{e_1, \, \dots, \, e_d\}$ be an orthonormal basis of the tangent space $\mathrm{Tan}(\Sigma, \, x)$. Then it turns out that
\begin{equation} \label{app:eq1} \frac{\mathrm{d} \mathcal{H}^d(\Sigma_t)}{\mathrm{d}t} \, \big|_{t = 0} = \int_\Sigma \mathrm{div}_T(v)(x) \, \mathrm{d}\mathcal{H}^d(x), \end{equation}
where $\mathrm{div}_T(-)$ denotes the divergence along the tangent plane, that is,
\begin{equation*} \mathrm{div}_T(v)(x) = \sum_{i = 1}^d \frac{\partial \left( \langle v, \, e_i \rangle \right)}{\partial e_i}(x). \end{equation*} \end{lemma}

\begin{proof}The map $\Phi_t$ is a diffeomorphism for every $t$; hence we can apply the area formula \eqref{areaformula2} to $\Sigma_t$, and obtain the following identity:
\begin{equation}\label{app:eq2} \mathcal{H}^{d} \left( \Sigma_t \right) = \int_\Sigma J_T (\Phi_t) (x) \, \mathrm{d} \mathcal{H}^d(x), \end{equation}
where $J_T$ denotes the tangential Jacobian, that is,
\begin{equation*}J_T (\Phi_t) = \sqrt{ \mathrm{det} \left( (\nabla_T \Phi_t)^t (\nabla_T \Phi_t) \right)}. \end{equation*}
Fix $x \in \Sigma$. By Taylor's expansion theorem one can prove that
\begin{equation*} \begin{cases} \Phi_t(x) = x + t \, v(x) + \mathcal{O}_x(t), \\[0.5em] \mathrm{d} \Phi_t(x) = \mathrm{Id} + t \, \mathrm{d}v(x) + \mathcal{O}_x(t). \end{cases} \end{equation*}
Let $\{e_1, \, \dots, \, e_n\}$ be an orthonormal basis of $\R^n$ such that $\{e_1, \, \dots, \, e_d\}$ is an orthonormal basis of the tangent space $\mathrm{Tan}(\Sigma, \, x)$. Then we can write the relations above as follows:
\begin{equation*} \nabla_T \Phi_t = \begin{bmatrix}[c|c] \mathrm{Id}_{d \times d} & \underline{0} \\[0.3em]   \underline{0} & \underline{0} \end{bmatrix} + t \begin{pmatrix} \nabla_T v \\[0.3em] \underline{0} \end{pmatrix} + \mathcal{O}_x(t). \end{equation*}
If we refer to the first matrix on the right-hand side by $J$, then it is easy to prove that
\begin{equation*} (\nabla_T \Phi_t)^t (\nabla_T \Phi_t)(x) = \mathrm{Id}_{d \times d}(x) + t \left(J^t \, \nabla_T v(x) + \left(\nabla_T v(x) \right)^t J \right) + \mathcal{O}_x(t),\end{equation*}
from which it follows that
\begin{equation*} \begin{aligned} J_T (\Phi_t)(x) & = \sqrt{ \mathrm{det} \left( (\nabla_T \Phi_t)^t (\nabla_T \Phi_t) \right)} \simeq
\\[1em] & \simeq \sqrt{1 + t \cdot \mathrm{tr}\left(J^t \, \nabla_T v(x) + \left(\nabla_T v(x) \right)^t J \right) + \mathcal{O}_x(t)} =
\\[1em] & = \sqrt{1 + 2t \, \mathrm{div}_T (v)(x) + \mathcal{O}_x(t)} \simeq
\\[1em] & \simeq 1 + t \, \mathrm{div}_T (v)(x) + \mathcal{O}_x(t),\end{aligned} \end{equation*}
where the first approximation $\simeq$ follows from the Taylor's expansion
\begin{equation*} \mathrm{det}( \mathrm{Id}_{n \times n} + t A ) \simeq 1 + t \, \mathrm{Tr}(A) + o(t) \quad \text{as $t \to 0$}, \end{equation*}
and the last approximation follows from the Taylor's expansion of $\sqrt{1 + x}$. In particular, it turns out that
\begin{equation*}\begin{aligned} \int_\Sigma J_T (\Phi_t) (x) \, \mathrm{d}\mathcal{H}^d(x) & = \int_\Sigma \left(1 + t \, \mathrm{div}_T (v)(x) + \mathcal{O}_x(t) \right) \, \mathrm{d}\mathcal{H}^d(x) =
\\[1em] & = \mathcal{H}^d(\Sigma) + t \int_\Sigma \mathrm{div}_T(v)(x) \, \mathrm{d}\mathcal{H}^{d}(x) + o(t). \end{aligned} \end{equation*}
Notice that the remainder needs a reasonable assumption to behave properly: For example, we can require either $\Sigma$ bounded or $\Phi_t$ is the identity outside of a compact set $K$ for every $t \in \R$.

In conclusion, it is enough to take the derivative with respect to the time of \eqref{app:eq2}, and evaluate it at $t = 0$, to obtain exactly the sought formula \eqref{app:eq1}.
\end{proof}

\begin{theorem}[Divergence Theorem] Let $\Sigma$ be a compact $(d- 1)$-dimensional surface with a boundary $\partial \Sigma$ of class $C^2$. Then
\begin{equation}\label{divth} \frac{\mathrm{d}}{\mathrm{d}t} \mathcal{H}^{d - 1}(\Sigma_t) \, \big|_{t = 0} = - \int_{\Sigma} \vec{H}(x) v(x) \, \mathrm{d}\mathcal{H}^{d - 1}(x) + \int_{\partial \Sigma} \eta_{\partial \Sigma}(x) v(x) \, \mathrm{d}\mathcal{H}^{d - 2}(x), \end{equation}
where $\vec{H}(x)$ is the mean curvature vector of $\Sigma$ at $x$ and $\eta_{\partial \Sigma}$ is the outward normal of $\partial \Sigma$ at $x$. \end{theorem}

\begin{lemma} Let $\Sigma_0 \subset \R^n$ be a hypersurface which minimizes the area (the $(n - 1)$-dimensional volume) among all $\Sigma$ such that
\begin{equation*} \Sigma \, \triangle \, \Sigma_0 \subset \subset \Omega, \end{equation*}
where $\Omega$ is a fixed open set in $\R^n$. Then
\begin{equation*} \vec{H}_{\Sigma_0}(x) = 0, \qquad \forall \, x \in \Omega \cap \Sigma_0. \end{equation*} \end{lemma}

\begin{proof} Let $v$ be a compactly supported smooth vector field on $\Omega$. Then there exists a family $\left( \Phi_t \right)_{t \in I}$ of diffeomorphisms with $v$ as initial speed\footnote{For example, the reader may consider the family of diffeomorphisms $\Phi_t(x) := x + t \, v(x)$, which is well-defined in a small neighborhood of $0$.} and $\Phi_0 = \mathrm{id}_{\R^n}$. It turns out that the hypersurfaces $(\Sigma_t)_{t \in I} := \left( \Phi_t(\Sigma_0) \right)_{t \in I}$ are all competitors for the area problem; hence the function
\begin{equation*} t \longmapsto \mathcal{H}^{n - 1}( \Sigma_t ) \end{equation*}
has a (local) minimum at $t = 0$. From the divergence formula \eqref{divth} we obtain the relation
\begin{equation*} \frac{\mathrm{d}}{\mathrm{d}t} \mathcal{H}^{n - 1}(\Sigma_t) \, \big|_{t = 0} = 0 \implies  \int_{\Sigma} \vec{H}_{\Sigma_0}(x) v(x) \, \mathrm{d}\mathcal{H}^{n - 1}(x) = 0, \end{equation*}
since the boundary term is equal to zero. Therefore, a straightforward application of the fundamental lemma in calculus of variation allows us to infer that
\begin{equation*} \vec{H}_{\Sigma_0}(x) = 0, \qquad \forall \, x \in \Omega \cap \Sigma_0. \end{equation*}\end{proof}

\begin{remark} If $\Sigma$ is a finite perimeter set, then a similar computation shows that
\begin{equation*} \int_{\partial_\ast \Sigma} \mathrm{div}_T (v)= 0, \end{equation*}
which means that we cannot choose any smooth vector field $v$ when dealing with it.\end{remark}

We can summarize the results obtained in this section with a slightly different version of the divergence formula \eqref{divth}, which holds for every smooth vector field $v$.

\begin{proposition}Let $\Sigma$ be a compact $(n- 1)$-dimensional (hyper)surface with a boundary $\partial \Sigma$ of class $C^2$. Then
\begin{equation}\label{divth2} \int_\Sigma \mathrm{div}_T(v)(x) \, \mathrm{d}\mathcal{H}^{n-1}(x) = - \int_{\Sigma} \vec{H}(x) v(x) \, \mathrm{d}\mathcal{H}^{n - 1}(x) + \int_{\partial \Sigma} \eta_{\partial \Sigma}(x v(x) \, \mathrm{d}\mathcal{H}^{n - 2}(x) \end{equation}
for every\footnote{Here we can take any smooth vector field $v$ because $\Sigma$ is compact by assumption. If not, then $v$ needs to be a compactly supported vector field.} $v \in C^\infty(\R^n)$. \end{proposition}

\begin{proof}We can decompose the vector field as follows:
\begin{equation*}v = v_N \cdot \eta_{\partial \Sigma} + v_T, \end{equation*}
where $v_N$ is the normal component, and $v_T$ is the tangential component. Then one can easily check that the following identity holds:
\begin{equation*}\mathrm{div}_T (v) = \underbracket{\mathrm{div}_T(v_N)}_{= 0} \cdot \eta_{\partial \Sigma} + v_N \cdot \mathrm{div}_T (\eta_{\partial \Sigma}) + \mathrm{div}_T(v_T). \end{equation*}
We notice that $\mathrm{div}_T(\eta_{\partial \Sigma})$ is the trace of the second fundamental form, which means that
\begin{equation*}\mathrm{div}_T(\eta_{\partial \Sigma}) = |\vec{H}|(x) =: H(x). \end{equation*}
The usual divergence theorem for a smooth vector field gives us the identity
\begin{equation*}\int_\Sigma \mathrm{div}_T(v)\, \mathrm{d}\mathcal{H}^{n - 1}(x) = \int_{\partial \Sigma} v(x)\eta_{\partial \Sigma}(x)\, \mathrm{d}\mathcal{H}^{n - 2}(x). \end{equation*}
If we plug the first relation (for the divergence) into the second one (divergence theorem for smooth functions), then it turns out that
\begin{equation*} \int_\Sigma \mathrm{div}_T(v)(x) \, \mathrm{d}\mathcal{H}^{n-1}(x) = - \int_{\Sigma} \vec{H}(x) v(x) \, \mathrm{d}\mathcal{H}^{n - 1}(x) + \int_{\partial \Sigma} \eta_{\partial \Sigma}(x) v(x) \, \mathrm{d}\mathcal{H}^{n - 2}(x),\end{equation*}
which is exactly what we wanted to prove.
\end{proof}

\begin{remark}Let $\Sigma \subset \R^n$ be a compact $(n- 1)$-dimensional surface with a boundary $\partial \Sigma$ of class $C^2$. If $\Sigma$ minimizes the $(n-1)$-dimensional volume among all $\widetilde{\Sigma}$ such that
\begin{equation*} \left( \widetilde{\Sigma} \, \triangle \, \Sigma \right) \cap \partial \Sigma = \varnothing, \end{equation*}
then the mean curvature $\vec{H}_{\Sigma}$ is identically equal to $0$. \end{remark}

We can state and prove a similar result for finite perimeter sets, but first we need a technical result concerning the closure of $\mathrm{BV}(\Omega)$ with respect to the action of a diffeomorphism.

\begin{lemma}Let $\Omega$ be a fixed open set in $\R^n$, and let $\Phi$ be a diffeomorphism of $\Omega$. If $u \in \mathrm{BV}(\Omega)$, then the composition $u \circ \Phi^{-1}$ also belongs to $\mathrm{BV}(\Omega)$. \end{lemma}

\begin{lemma}\label{bvdiff} Let $\Omega$ be a fixed open set in $\R^n$, and let $E$ be a finite perimeter set which minimizes the perimeter inside $\Omega$ among all $\widetilde{E}$ f.p.s. in $\Omega$ such that
\begin{equation*} E \, \triangle \, \widetilde{E} \subset \subset \Omega \end{equation*}
up to null sets. Then it turns out that
\begin{equation*} \int_{\partial_\ast E} \mathrm{div}_T(v) \, \mathrm{d} \mathcal{H}^{n - 1} = 0, \qquad \forall \, v \in C_c^\infty(\Omega). \end{equation*} \end{lemma}

\begin{proof}Fix $v$ compactly supported smooth vector field on $\Omega$. Then there exists a family $\left( \Phi_t \right)_{t \in I}$ of diffeomorphisms with $v$ as initial speed\footnote{For example, the reader may consider the family of diffeomorphisms $\Phi_t(x) := x + t \, v(x)$, which is well-defined in a small neighborhood of $0$.} and $\Phi_0 = \mathrm{id}_{\R^n}$. Since $\Phi_t(x) = x$ for every $x$ which does not belong to the support of $v$, \hyperref[bvdiff]{Lemma \ref{bvdiff}} proves that $(E_t)_{t \in I} := \left( \Phi_t(E_0) \right)_{t \in I}$ is a collection of finite perimeter set such that
\begin{equation*} \partial_\ast E_t = \Phi_t( \partial_\ast E ). \end{equation*}
In particular, the f.p.s. $E_t$ are all competitors for the perimeter problem; hence the function
\begin{equation*}t \longmapsto \mathrm{Per}_\Omega(E_t) \end{equation*}
admits a (local) minimum at $t = 0$. The thesis follows from a simple identity which is left to the reader:
\begin{equation*} \frac{\mathrm{d} \left( \mathrm{Per}_\Omega(E_t) \right)}{\mathrm{d}t} \, \big|_{t = 0} = \int_{\partial_\ast E} \mathrm{div}_T(v) \, \mathrm{d} \mathcal{H}^{n - 1}. \end{equation*}\end{proof}
 
\begin{definition}[Weak Mean Curvature] \index{weak mean curvature} Let $\Sigma$ be a $d$-rectifiable set in $\R^n$. We say that $\Sigma$ has mean curvature in the weak sense if and only if there exists $\vec{g} \in L^p \left(\R^n, \, \mathcal{H}^d \restr \Sigma \right)$ such that
\begin{equation*} \int_\Sigma \mathrm{div}_T(v)(x) \, \mathrm{d}\mathcal{H}^d(x) = \int_\Sigma \vec{g}(x) v(x) \, \mathrm{d}\mathcal{H}^d(x), \qquad \forall \, v \in C_c^\infty(\R^n). \end{equation*}  \end{definition}

\begin{remark} Let $\Sigma$ be a compact $(n- 1)$-dimensional (hyper)surface with a boundary $\partial \Sigma$ of class $C^2$. If $\Sigma$ admits a weak mean curvature $\vec{g}$, then the following properties hold: \mbox{}
\begin{enumerate}[label=(\roman*)]
\item The weak mean curvature $\vec{g}$ coincides with the mean curvature $\vec{H}$ almost everywhere.
\item The boundary of $\Sigma$ is empty\footnote{This follows fairly easily from the integration by parts formula.}.
\end{enumerate} \end{remark}

\begin{definition}[$k$-Varifold] \index{varifold} An \textit{integral $k$-dimensional varifold} is a couple $(\Sigma, \, \theta)$, where $\Sigma$ is a $k$-rectifiable set in $\R^n$, and $\theta : \Sigma \longrightarrow \N$ is a function in $L^1 \left(\R^n, \, \mathcal{H}^k \restr \Sigma \right)$. \end{definition}

\begin{definition}[Varifold W.M.C.] \index{weak mean curvature!varifold} Let $(\Sigma, \, \theta)$ be a $k$-dimensional varifold in $\R^n$. We say that $\Sigma$ has mean curvature in the weak sense if and only if there exists $\vec{g} \in L^p \left(\R^n, \, \mathcal{H}^d \restr \Sigma \right)$ such that
\begin{equation*} \int_\Sigma \theta(x) \mathrm{div}_T(v)(x) \, \mathrm{d}\mathcal{H}^k(x) = \int_\Sigma \theta(x) \vec{g}(x) v(x)\, \mathrm{d}\mathcal{H}^k(x), \qquad \forall \, v \in C_c^\infty(\R^n). \end{equation*}  \end{definition}

\begin{exercise} Let $\Sigma$ be a compact $(n- 1)$-dimensional (hyper)surface with a boundary $\partial \Sigma$ of class $C^2$, and let $\theta \in \mathrm{BV}(\Sigma; \; \N)$. Suppose that
\begin{equation*} \int_\Sigma \theta(x) \, \mathrm{div}(v)(x) \, \mathrm{d}\mathcal{H}^k(x) = - \int_\Sigma v(x) \, \mathrm{d}\theta(x) \end{equation*}
for every tangent vector field $v \in C_c^\infty(\Sigma)$. Prove that:
\begin{enumerate}[label=(\roman*)]
\item The weak mean curvature $\vec{g}$ coincides with the mean curvature $\vec{H}$ almost everywhere.
\item The boundary of $\Sigma$ is empty.
\item The function $\theta$ is locally constant.
\end{enumerate} 
\end{exercise}

\begin{exercise} Let $\Sigma$ be a compact $(n- 1)$-dimensional (hyper)surface with a boundary $\partial \Sigma$ of class $C^2$, and let $\theta \in C^\infty(\Sigma; \; \R)$. Suppose that
\begin{equation*} \int_\Sigma \theta(x) \, \mathrm{div}(v)(x) \, \mathrm{d}\mathcal{H}^k(x) = - \int_\Sigma v(x) \, \mathrm{d}\theta(x) \end{equation*}
for every tangent vector field $v \in C_c^\infty(\Sigma)$. Prove or disprove the following formula:
\begin{equation*}\vec{g} = \vec{H} - \frac{\nabla_T \theta}{\theta}. \end{equation*}
\end{exercise}

\section{Regularity in Capillarity Problems}

\paragraph{Framework.} Let $\Omega \subset \R^n$ be an open bounded subset of $\R^n$. In this section, we shall discuss the capillarity problem equilibrium when the container ($\Omega$) is fixed and the volume of the liquid is fixed. More precisely, the drop of liquid (denoted by $E$), contained in $\Omega$, is at equilibrium if and only if it is a critical point (local minima) of the capillarity energy functional
\begin{equation} \label{funccap:app} \mathscr{F}(E) := \mathcal{H}^{n-1} \left( \partial E \cap \Omega \right) + \sigma \mathcal{H}^{n-1} \left(\partial E \cap \partial \Omega \right), \end{equation}
where $\sigma \in \R$ is a constant, subject to the constraint that the volume is fixed, e.g. $|E| = m$.

\paragraph{Mean Curvature.} Assume that $E$ is a f.p.s. which minimizes the capillarity energy \eqref{funccap:app} and satisfies the constraint on the volume. If $E$ is regular enough, then one can prove that
\begin{equation*} \begin{cases} \text{$H_f$ is constant on the free portion of the surface $\Sigma^f$}, \\[0.5em] \text{$\theta_Y$ is constant on the boundary $\Gamma$}. \end{cases} \end{equation*}
More precisely, the free mean curvature $H_f$ is the Lagrange multiplier associated to the volume constraint, while $\theta_Y$ is the solution of the trigonometric equation
\begin{equation} \label{youngang} \cos \theta_Y = - \sigma. \end{equation}
The first assertion is easy to prove. Let $(\Phi_t)_{t \in I}$ be a family of diffeomorphisms fixing the boundary, and choose a smooth vector field $v$ in such a way that the volume constraint is not violated, that is,
\begin{equation*}\text{the constraint is preserved} \iff \int_{\Sigma^f} v \cdot \eta = 0,\end{equation*}
which means that the first variation of the volume is zero. As a result of the technical tools introduced in the previous section, it turns out that if we set the first variation equal to zero, then we obtain
\begin{equation*} \int_{\Sigma^f} H (v \cdot \eta) = 0, \qquad \forall \, v \in C_c^\infty \: : \: \int_{\Sigma^f} v \cdot \eta = 0. \end{equation*}
The Du Bois-Reymond's fundamental lemma in calculus of variation allows us to infer that
\begin{equation*} H \, \big|_{\Sigma^f} = c \implies H_f = c ,\end{equation*}
that is, the free mean curvature is constant. The proof of \eqref{youngang} is very similar, but we need to consider a family of transformations which moves the boundary.

\begin{remark}If $E$ is not regular enough, then one might translate the problem in the setting of $k$-varifolds, and prove that the weak mean curvature is constant on $\Sigma^f$. Unfortunately, it is not easy at all to give a meaning to $\theta$, and recover something like formula \eqref{youngang}. \end{remark}

\section{Regularity in the F.P.S. Setting}

\paragraph{Introduction.} Let $\Omega \subset \R^n$ be an open bounded subset of $\R^n$, and let $E$ be a finite perimeter set in $\Omega$ minimizing the perimeter $\mathrm{Per}_\Omega(-)$ among all f.p.s. $\widetilde{E}$ such that
\begin{equation*} E \, \triangle \, \widetilde{E} \subset \subset \Omega. \end{equation*}

\begin{theorem} \label{th:regper} The essential boundary $\partial_\ast E$ is closed in $\Omega$, and there is a representative of $E$ in its equivalence class which satisfies the following properties: \mbox{}
\begin{enumerate}[label=\textbf{(\alph*)}]
\item The topological boundary $\partial E$ coincides with the essential boundary $\partial_\ast E$.
\item The perimeter formula is satisfies, that is,
\begin{equation*} \mathrm{Per}_\Omega(E) = \mathcal{H}^{n - 1}(\partial E). \end{equation*}
\end{enumerate}
Moreover, the representative above can be uniquely characterized as follows:
\begin{equation*} E = \left\{ x \in E \: \left| \: \Theta(E, \, x) = 1 \right. \right\}. \end{equation*} \end{theorem}

\begin{lemma} \label{lemma:regper} There exists a universal constant $\delta_0 := \delta_0(n) > 0$ such that for every $(x_0, \, r_0) \in \Omega \times \R$ satisfying $B(x_0, \, r_0) \subset \Omega$, it turns out that
\begin{equation*} \begin{aligned} & \frac{\left| E \cap B(x_0, \, r_0) \right|}{\alpha_n r_0^n} \leq \delta_0 \implies \left|E \cap B \left(x_0, \, \frac{r_0}{2} \right) \right| = 0,
\\[1em] & \frac{\left| E \cap B(x_0, \, r_0) \right|}{\alpha_n r_0^n} \geq 1- \delta_0 \implies \left|E \cap B \left(x_0, \, \frac{r_0}{2} \right) \right| = \left| B(x_0, \, r_0) \right|, \end{aligned} \end{equation*}
almost everywhere. \end{lemma}

\begin{proof}[Truncation Argument] Fix $(x_0, \, r_0) \in \Omega \times \R$. For every $r \in [\frac{r_0}{2}, \, r_0]$, set
\begin{equation*} E_r := E \setminus B(x_0, \, r). \end{equation*}

\paragraph{Step 1.} The set $E$ minimizes the perimeter; hence
\begin{equation*} \mathrm{Per}_\Omega(E_r) \geq \mathrm{Per}_\Omega(E). \end{equation*}
The usual isoperimetrical inequality yields to\footnote{The last inequality needs to be justified. The reader should check very carefully the details since a lot of issues could arise here.}
\begin{equation*} \begin{aligned} c(n) \left| E \cap B(x_0, \, r) \right|^{1 - \frac{1}{n}} & \leq \mathrm{Per}_{B(x_0, \, r)}(E) =
\\[1em] & = \mathcal{H}^{n - 1}\left( \partial_\ast E \cap B(x_0, \, r) \right) \leq
\\[1em] & \leq \mathcal{H}^{n - 1}\left(E \cap \partial B(x_0, \, r) \right)\end{aligned} \end{equation*}
for almost every $r \in [\frac{r_0}{2}, \, r_0]$. Recall that
\begin{equation*} D( \mathbbm{1}_{E_r} ) = \mathbbm{1}_{\Omega \setminus B(x_0, \, r)} \, D(\mathbbm{1}_E) + \nu \, \mathbbm{1}_{E \cap \partial B(x_0, \, r)} \cdot \mathcal{H}^{n - 1}, \end{equation*}
from which it follows that
\begin{equation*} \partial_\ast E_r = \left( \partial_\ast E \setminus \overline{B(x_0, \, r)} \right) \cup \left(E \cap \partial B(x_0, \, r) \right) \end{equation*}
for almost every admissible $r$.

\paragraph{Step 2.} Let $v(r) := \left| E \cap B(x_0, \, r) \right|$ be the $n$-dimensional Lebesgue measure. Then\footnote{Here the reader needs to be extremely careful. A priori $v$ is simply differentiable in the weak sense of distributions, while in the computation above we need to deal with the classical derivative of $v$.}
\begin{equation*} \dot{v}(r) = \mathcal{H}^{n - 1} \left( E \cap \partial B(x_0, \, r) \right) \qquad \text{for a.e. admissible $r$.} \end{equation*}
It follows from the isoperimetrical inequality that
\begin{equation*}c(n) \, v(r)^{1 - \frac{1}{n}} \leq \dot{v}(r), \end{equation*}
from which it follows that\footnote{To solve the differential inequality one needs to divide by $v$. If $v(r) \neq 0$ for a.e. admissible $r$, then everything is fine. On the other hand, if $v(r) = 0$ in the middle of the interval, then one needs to check that $v$ is strictly increasing, and thus $\{r \: : \: v(r) = 0\}$ is a null set. }
\begin{equation*} v\left(r_0 \right)^{\frac{1}{n}} - v\left(\frac{r_0}{2} \right)^{\frac{1}{n}} \geq \frac{c(n)}{n} \, \frac{r_0}{2}. \end{equation*}
If we rearrange the inequality, then we obtain the following estimate:
\begin{equation*}v\left(\frac{r_0}{2} \right)^{\frac{1}{n}} \leq \left[ \alpha_n^{\frac{1}{n}} \, \left(\frac{v(r_0)}{\alpha_n r_0^n}\right)^{\frac{1}{n}} - \frac{c(n)}{2n} \right] r_0.\end{equation*}
In particular, the thesis follows by choosing the universal constant $\delta_0$ in such a way that $v(r_0/2)$ is either $0$ or equal to the volume of the ball, that is,
\begin{equation*} \delta_0 = \frac{1}{\alpha_n} \left( \frac{c(n)}{2n} \right)^n. \end{equation*} \end{proof}

\begin{proof}[Proof of Theorem \ref{th:regper}] Everything is obvious except the fact that $\partial_\ast E$ is a closed set; we shall show that the complement is open.  \mbox{}

\paragraph{Case 1.} Fix a point $x_0$ with density $\Theta(E, \, x_0)$ equal to $0$. Then there exists $r_0 > 0$ small enough such that
\begin{equation*} \frac{\left| E \cap B(x_0, \, r_0) \right|}{\alpha_n r_0^n} \leq \delta_0, \end{equation*}
and thus by \hyperref[lemma:regper]{Lemma \ref{lemma:regper}} it turns out that
\begin{equation*} \left|E \cap B \left(x_0, \, \frac{r_0}{2} \right) \right| = 0. \end{equation*}
In particular, it turns out that
\begin{equation*} \Theta(E, \, x) = 0 \qquad \forall \, x \in E \cap B \left(x_0, \, \frac{r_0}{2} \right). \end{equation*}

\paragraph{Case 2.} Fix a point $x_0$ with density $\Theta(E, \, x_0)$ equal to $1$. Then there exists $r_0 > 0$ small enough such that
\begin{equation*} \frac{\left| E \cap B(x_0, \, r_0) \right|}{\alpha_n r_0^n} \geq 1 - \delta_0, \end{equation*}
and thus by \hyperref[lemma:regper]{Lemma \ref{lemma:regper}} it turns out that
\begin{equation*} \left|E \cap B \left(x_0, \, \frac{r_0}{2} \right) \right| = \left|B \left(x_0, \, \frac{r_0}{2} \right) \right|. \end{equation*}
In particular
\begin{equation*} \Theta(E, \, x) = 1 \qquad \forall \, x \in E \cap B \left(x_0, \, \frac{r_0}{2} \right), \end{equation*}
which means that the complement of $\partial_\ast E$ is open (i.e., $\partial_\ast E$ is closed).
\end{proof}

\begin{exercise} What happens if the f.p.s. $E$ minimizes the perimeter $\mathrm{Per}_\Omega(-)$ among all f.p.s. $\widetilde{E}$ such that
\begin{equation*} E \, \triangle \, \widetilde{E} \subset \subset \Omega, \end{equation*}
subject to a volume constraint? Is $\partial_\ast E$ still closed?\end{exercise}

\section{Structure of Finite Length Continua in $\R^n$}

The main goal of this section is to prove that a continua (= compact and connected) set $K$ in $\R^n$ with finite length is the image of a surjective path $\gamma$ such that
\begin{equation*} \mathrm{Length}(\gamma) \leq 2 \cdot \mathcal{H}^1(K) < + \infty. \end{equation*}

\begin{definition}Let $I := [a, \, b]$. A continuous path $\gamma: I \longrightarrow \R^n$ of finite length has constant speed $c$ if and only if
\begin{equation*}\mathrm{Length}(\gamma, \, J) < c \cdot \mathrm{Length}(\gamma, \, I) \end{equation*}
for every $J \subset I$. \end{definition}

\begin{remark}[Arc Length] Let $\gamma: I \longrightarrow \R^n$ be a continuous path of finite length $L$. If we set
\begin{equation*} \sigma(s) := \inf \left\{ t \in [a, \, b] \: \left| \: \mathrm{Length}\left(\gamma, \, [a, \, t] \right) \geq Ls \right. \right\}, \end{equation*}
then the riparametrization $\gamma \circ \sigma : [0, \, 1] \longrightarrow \R^n$ is a continuous path with constant speed $L$. \end{remark}

\begin{remark}[Tangent Plane] Let $K$ be a continua set in $\R^n$. Then the tangent plane $\mathrm{Tan}(K, \, x)$ is the vector space in the classical sense for $\mathcal{H}^1$-almost every $x$.\end{remark}

\begin{lemma} \label{conle:1} Let $K$ be a continua set in $\R^n$ with finite length. Then $K$ is connected by arcs and, more precisely, it is connected by injective paths of finite length. \end{lemma}

\begin{proof}[Sketch of the Proof] The main idea is to obtain the injective path $\gamma$ as the uniform limit (with respect to the Hausdorff distance) of a sequence of finite length paths $\gamma^\delta$ satisfying certain properties.

\paragraph{Step 1.} Fix $x_0, \, x \in K$. For every $\delta > 0$ we may consider the (almost) shortest $\delta$-chain of points $x_0^\delta, \, \dots, \, x_n^\delta \in K$ and times $0 =: t_0^\delta \leq \dots \leq t_n^\delta := 1$, where $n$ depends on $\delta$, such that the following properties hold: \mbox{}
\begin{enumerate}[label=\textbf{\roman*)}]
\item The first point is $x_0 := x_0^\delta$, and the final point is $x := x_n^\delta$ for every $\delta > 0$.
\item The distance between two consecutive points is less than $\delta$, that is,
\begin{equation*} |x_i^\delta - x_{i - 1}^\delta| < \delta, \end{equation*}
and the total length is less than the length of $K$, that is,
\begin{equation*} \sum_{i = 1}^{n} |x_i^\delta - x_{i - 1}^\delta| < \mathcal{H}^1(K). \end{equation*}
\item For every $i = 1, \, \dots, \, n$ it turns out that $\gamma^\delta$ reaches the point $x_i^\delta$ at the time $t_i^\delta$.
\item The Lipschitz constant of $\gamma^\delta$ is less or equal than $4$ times its length.
\end{enumerate}
The reader may prove that the curve $\gamma$ can be obtained as the uniform limit of the sequence $(\gamma^\delta)_{\delta > 0}$. \end{proof}

\begin{theorem}\label{strlsds}Let $K$ be a continua set in $\R^n$ with finite length. Then there exists a surjective path $\gamma : [0, \, 1] \longrightarrow K$ such that
\begin{equation*} \mathrm{Length}(\gamma) \leq 2 \cdot \mathcal{H}^1(K) < + \infty. \end{equation*}
In particular, every continua $\mathcal{H}^1$-finite set is rectifiable. \end{theorem}

\begin{proof}The assertion is an immediate consequence of \hyperref[conle:1]{Lemma \ref{conle:1}}. \end{proof}

%\begin{proof}[Sketch of the Proof] Let $x_0, \, x_1 \in K$ be the points maximizing the distance, that is, 
%\begin{equation*}(x_0, \, x_1) \in \mathrm{argmax} \left\{ d(y_0, \, y_1) \: : \: (y_0, \, y_1) \in K \times K \right\}. \end{equation*}
%Let $\gamma$ be the injective finite length path connecting $x_0$ and $x_1$ (which exists by \hyperref[conle:1]{Lemma \ref{conle:1}}), and let us consider the following additional paths: \mbox{}
%\begin{enumerate}[label=\textbf{(\alph*)}]
%\item A closed curve $\gamma_1$ which is obtained as follows. Start from $x_0$ and go to $x_1$ along $\gamma$, and then come back to $x_0$ in the same way.
%\item A curve $\gamma_2$ which maximizes the distance between $x_2$ and $\gamma \, \big|_{[0, \, 1]}$.
%\end{enumerate}
%... \end{proof} %DA FINIRE

\begin{remark}[Degree] \label{rmk:degree} The degree of $\gamma$ at $x$, which will be denoted by $\mathrm{deg}(\gamma, \, x)$, is well-defined for $\mathcal{H}^{1}$-almost every $x$. In particular, for every vector field $v \in C_c^\infty(\R^n; \; \R^n)$, it turns out that
\begin{equation*} \int v(\gamma_i(s)) \, \dot{\gamma}_i(s) \, \mathrm{d}s = 0 \end{equation*}
for every $i \in \N$, which means that
\begin{equation*} \int v(\gamma(s)) \, \dot{\gamma}(s) \, \mathrm{d}s = 0, \qquad \forall \, v \in C_c^\infty(\R^n; \; \R^n). \end{equation*}
Since $v(\gamma_n)$ converges uniformly to $v(\gamma)$ and $\dot{\gamma}_n$ converges to $\dot{\gamma}$ in $L_{w^\ast}^\infty$ (i.e., in the weak-$\ast$ topology of $L^\infty$), then the area formula implies that
\begin{equation*}0 = \int v(\gamma(s)) \frac{\dot{\gamma}(s)}{|\dot{\gamma}(s)|}\, J(\gamma)(s) \mathrm{d}s = \int_K v(x) \sum_{s \in \gamma^{-1}(x)}\tau(s) \cdot \mathrm{deg}(\gamma, \, x) \, \mathrm{d}\mathcal{H}^1(x), \end{equation*}
where
\begin{equation*} \tau(s) := \frac{\dot{\gamma}(s)}{|\dot{\gamma}(s)|}  \end{equation*}
indicates the orientation of the curve. In particular, it turns out that
\begin{equation*} \tau(x) \cdot \mathrm{deg}(\gamma, \, x) = 0 \end{equation*}
for $\mathcal{H}^1$-almost every $x$, which means that the cardinality of the set $\gamma^{-1}(x)$ is even for $\mathcal{H}^1$-almost every $x$. \end{remark}

\section{Lower Semi-Continuity: Golab Theorem}

In this brief section, we use the structure theorem of a continua set to prove the famous Golab compactness result.

\begin{theorem}[Golab] Let $(K_i)_{i \in \N}$ be a sequence of continua sets in $\R^n$. If $K_i$ converges to a continua set $K$ with respect to the Hausdorff distance, then
\begin{equation*} \liminf_{i \to + \infty} \mathcal{H}^1(K_i) \geq \mathcal{H}^1(K). \end{equation*} \end{theorem}

\begin{proof}By \hyperref[strlsds]{Theorem \ref{strlsds}} for every $i \in \N$ there exists a surjective path $\gamma_i$ of finite length that parametrizes the corresponding continua set $K_i$. Up to subsequences, it turns out that $\gamma_i$ converges to some path $\gamma$, which parametrizes $K$. 

The preimage $\gamma^{-1}(x)$ is nonempty, and its cardinality is even (see \hyperref[rmk:degree]{Remark \ref{rmk:degree}}). Therefore, one can easily infer that
\begin{equation*} \left|\gamma^{-1}(x) \right| \geq 2 \quad \text{at $\mathcal{H}^1$-almost every $x \in K$}, \end{equation*}
which means that
\begin{equation*} \mathcal{H}^1(K) \leq \frac{1}{2} \cdot \mathrm{Length}(\gamma). \end{equation*}
The length is a semi-lower continuous functional
\begin{equation*} \mathrm{Length}(\gamma) \leq \liminf_{i \to + \infty} \mathrm{Length}(\gamma_i), \end{equation*}
and by \hyperref[strlsds]{Theorem \ref{strlsds}} it turns out that
\begin{equation*} \mathrm{Length}(\gamma) \leq \liminf_{i \to + \infty} \mathrm{Length}(\gamma_i) \leq 2 \liminf_{i \to + \infty} \mathcal{H}^1(K_i). \end{equation*}
If we combine the two inequalities, then we obtain the thesis:
\begin{equation*} \mathcal{H}^1(K) \leq \frac{1}{2} \cdot \mathrm{Length}(\gamma) \leq \liminf_{i \to + \infty} \mathcal{H}^1(K_i). \end{equation*}
\end{proof}

\section{Simons' Cone}

In this final section, we show that the minimality of the functional $\mathcal{H}^{n-1}(- \cap K)$ may be achieved via a calibration. We shall apply this result to prove the minimality of the Simons' Cone, which is a singular surface, in dimension $n = 8$.

\begin{theorem}[Minimality through Calibration] Let $\Sigma_0$ be a compact $(n-1)$-dimensional hypersurface in $\R^n$ with boundary $\partial \Sigma_0$ oriented by a continuous unit normal $\eta_0$. Assume that there exists a calibration for $\Sigma_0$, that is, a vector field $v$, defined on the space $\R^n$, with the following properties: \mbox{}
\begin{enumerate}[label=\textbf{(\roman*)}]
\item The vector field $v$ coincides with $\eta_0$ on $\Sigma_0$.
\item The norm of $v$ is less than or equal to $1$.
\item The divergence of $v$ is identically zero.
\end{enumerate}
Then $\Sigma_0$ is the hypersurface minimizing the functional $\mathcal{H}^{n-1}(-)$ among all the $(n - 1)$-dimensional hypersurfaces $\Sigma$ with the same boundary, oriented in the same way.\end{theorem}

\begin{proof} A straightforward computation shows that
\begin{equation*} \begin{aligned} \mathcal{H}^{n - 1}(\Sigma_0) & \: {\color{blue}=} \: \int_{\Sigma_0} v \cdot \eta_0 \, \mathrm{d}\mathcal{H}^{n - 1} \: {\color{red}=}
\\[1em] & \: {\color{red}=}\: \int_\Sigma v \cdot \eta_0 \, \mathrm{d}\mathcal{H}^{n - 1} \leq
\\[1em] & \leq \mathcal{H}^{ n -1}(\Sigma),
\end{aligned} \end{equation*}
where the {\color{blue} blue} equality follows from \textbf{(i)}, and the {\color{red} red} follows from \textbf{(ii)}, \textbf{(iii)}, and the divergence theorem. \end{proof} 

\begin{theorem}[Minimality through Calibration, II] Let $\Sigma_0$ be a complete $(n-1)$-dimensional hypersurface in $\R^n$ without boundary. Assume that there exist a Borel set $E$ and a calibration for $\Sigma_0$, that is, a vector field $v$, defined on the space $\R^n$, with the following properties: \mbox{}
\begin{enumerate}[label=\textbf{(\roman*)}]
\item The vector field $v$ coincides with $\eta_0$ on $\Sigma_0$.
\item The norm of $v$ is less than or equal to $1$.
\item The divergence of $v$ is less than or equal to zero in $E$, and greater than or equal to zero in the complement $\R^n \setminus E$.
\item The surface $\Sigma_0$ is the boundary of $E$, and $\eta_0$ is the outwards normal.
\end{enumerate}
Then $\Sigma_0$ is the hypersurface minimizing the functional $\mathcal{H}^{n-1}(- \cap K)$ among all the $(n - 1)$-dimensional hypersurfaces $\Sigma$ which agrees with $\Sigma$ outside of a compact set $K$.\end{theorem}

\paragraph{Simons' Cone.} The surface
\begin{equation*}\Sigma_S := \left\{ (x, \, y) \in \R^4 \times \R^4 \: \left| \: |x| = |y| \right. \right\} \end{equation*}
admits a calibration $v$. In particular, the Simons' cone is the minimal surface (with respect to the $(n-1)$-dimensional Hausdorff measure) in dimension $n = 8$, and it is easy to prove that the origin $(0, \, 0)$ is a singular point (as stated in the final remark of the previous chapter). We will not present a proof here, but the calibration is given by
\begin{equation*} v(x, \, y) := \frac{\nabla f(x, \, y)}{f(x, \, y)}, \quad \text{where $f(x, \, y) = \frac{|x|^4 - |y|^4}{4}$}, \end{equation*}
and the Borel set $E$ is given by
\begin{equation*}E := \left\{ (x, \, y) \in \R^4 \times \R^4 \: \left| \: |x| < |y| \right. \right\}. \end{equation*}

\printindex

\bibliography{Bibliografia}{} % BIBLIOGRAFIA
\bibliographystyle{plain}
\end{document}
