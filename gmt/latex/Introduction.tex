\chapter{Introduction}

In this chapter, we introduce the main topics of the course and give a brief overview of what we will see and what we will be able to prove by the end of the course.

\section{Plateau's Problem}

The primary goal and the motivating example of this course is the \textbf{Plateau's problem}, that is, the problem to find the $d$-dimensional surface $\Sigma$ of the minimal area with prescribed $(d-1)$-dimensional boundary $\Gamma$.

By the end, we will be able to prove that a solution indeed exists, but we will not find it explicitly since it is a $NP$ (hard) numerical problem.

As of now, the problem is not well defined. In fact, the notions of \textit{surface}, \textit{area}, and \textit{boundary} make sense in the smooth setting but, as the examples below show, we need to work in a less regular setting.

More precisely, requiring the surface to be smooth is not enough for modeling reasons (e.g., dip a wire frame into a soap solution, form a soap film, and look for the minimal surface whose boundary is the wire frame), and also for existence reasons.

\begin{example} Here we give a list of Plateau's problems with prescribed boundary conditions, and we write down the correct solutions, without proving anything. \mbox{}
\begin{enumerate}[label=\textbf{(\alph*)}]
\item Let us identify $\R^4 \cong \C \times \C$ and, if $d = 2$, let us consider the smooth boundary given by
\begin{equation*} \Gamma_1 := \left( S^1 \times \{0\} \right) \cup \left( \{0\} \times S^1 \right). \end{equation*}
Surprisingly, every minimizing sequence of smooth surfaces converges to a surface which is not smooth at all. Indeed, the solution of the problem is
\begin{equation*} \Sigma_1 := \left( D^2 \times \{0\} \right) \cup \left( \{0\} \times D^2 \right). \end{equation*}
The surface $\Sigma_1$ is clearly singular at the origin, but the singularity may be removed (by factorizing it into two nonsingular surfaces).
\item Let us identify $\R^4 \cong \C \times \C$ and, if $d = 2$, let us consider the smooth boundary given by
\begin{equation*} \Gamma_2 := \left\{ (z^2, \, z^3) \: : \: z \in S^1 \right\}. \end{equation*}
The solution to the Plateau's problem is
\begin{equation*} \Sigma_2 := \left\{ (z^2, \, z^3) \: : \: z \in D^2 \right\}, \end{equation*}
which is a non-smooth surface, whose singularity cannot be removed (since the polynomial $z_1^3 = z_2^2$ cannot be factorized).
\item Let us identify $\R^8 \cong \R^4 \times \R^4$ and, if $d = 7$, let us consider the smooth boundary given by
\begin{equation*} \Gamma_3 := S^3 \times S^3. \end{equation*}
The minimal surface of prescribed boundary $\Gamma_3$ is
\begin{equation*} \Sigma_3 := \left\{ (x_1, \, x_2) \in \R^4 \times \R^4 \: : \: |x_1| = |x_2| \leq 1 \right\}. \end{equation*}
\end{enumerate}
\end{example}

To conclude this introductive chapter, we give a brief overview of the main approaches (studied in this course) to the Plateau's problem, as $d$ ranges between $1$ and $\infty$.

\section{Geodesics problem ($d=1$)}

The geodesics problem (that is, find the shortest curve connecting two points) is, surprisingly, still an open in the non-Riemannian setting. However, in the Riemannian setting, the geodesics problem is completely solved.

Indeed, if we consider the curves parametrized by paths, the \textit{length} is a well-defined notion, and the associated functional is lower semi-continuous and coercive; hence the compactness is easy to prove.

There are many possible approaches to the geodesics problem, e.g., the Steiner approach and the set theoretical approach, which we describe briefly in the remainder of the section.

\paragraph{Steiner Problem.} It is also called networks approach, and it is used to prove the existence of the geodesics and find the explicit expression for it. The reader may consult \cite{steinerpro} for a detailed dissertation on the topic.

\paragraph{Set Theoretical Approach.} The main idea is to find a closed and connected set $\Sigma$ of minimum \textit{length}, containing a given finite set $\Gamma$. As we shall see later in the course, in this case the length is a well defined concept: the \textit{Hausdorff distance}.

In fact, if $X$ is a suitable space (metric, endowed with Hausdorff distance, etc...), then the class defined by
\begin{equation*}\mathcal{X} := \left\{ K \subseteq X \: : \: K \, \, \text{compact and connected} \right\} \end{equation*}
is compact and, by Gotab theorem\footnote{\cite{falconer} Let $\mathscr{C}$ be an infinite collection of non-empty compact sets all lying in a bounded portion $B$ of $\R^n$. Then there exists a sequence $\{E_j\}$ of distinct sets of $\mathfrak{C}$ convergent in the Hausdorff metric to a non-empty compact set $E$.}, $\mathcal{H}^1$ is lower semi-continuous on $X$. 

\section{"Surface" Problem ($d>1$)}

\paragraph{3.} The Plateau's problem is much harder when $d = 2$, but there are still many approaches possible some of which relying, in a certain sense, on the work already done in the geodesics case.

\paragraph{Set Theoretical Approach.} This approach is highly nontrivial. For example, one may ask what does it mean that a compact set $\Sigma$ spans a boundary $\Gamma$? Moreover, there is another problem one should deal with: the $2$-dimensional Hausdorff measure $\mathcal{H}^2$ is, generally, not lower semi-continuous. The reader may consult \cite{Reifenberg1960} for a complete treatise of the topic.

\begin{remark}Suppose that $d = 2$, $n = 3$ and that $\Sigma$ is a surface with boundary $\Gamma$. If $\gamma$ is another closed curve, linked to $\Gamma$ (by a nonzero linking number), then $\gamma \cap \Sigma \neq \emptyset$. \end{remark}

\paragraph{Parametric Approach.} This method is essentially due to Douglas \cite{douglas}. The main idea is the following: since a parametrization $\phi : D^2 \to \R^n$ defines surfaces in $\R^n$, the area functional is well-defined and given by the formula
\begin{equation*}A(\phi) := \int_{D^2} \left| \frac{\partial \, \phi}{\partial \, s_1} \wedge \frac{\partial \, \phi}{\partial \, s_2} \right| \, \mathrm{d}s_1 \, \mathrm{d}s_2. \end{equation*}
On the other hand, the existence through lower semi-continuity and the compactness are a delicate matter, since coercivity is not an easy property to obtain (the integrand is similar to a determinant).

There is a trick which is similar to the one we can use to find geodesics in the differential geometry setting. More precisely, we consider the functional
\begin{equation*}E(\phi) := \frac{1}{2} \, \int_{D^2} \left| \nabla \phi \right|^2 \, \mathrm{d}s_1 \, \mathrm{d}s_2. \end{equation*}
If we find a minimal point $\phi$ for $E$, then $\phi$ will be a \textbf{conformal parametrized} minimum for $A$. This trick, on the other hand, heavily depends on a nontrivial theorem: every such $\Sigma$ admits a conformal re-parametrization.

The lack of conformal parametrization, though, is what stop us from extending the same trick to dimension $d$ strictly bigger than $2$.

\paragraph{Higher Dimension.} If the codimension of $\Sigma$ is equal to $1$ (that is, $n = d+1$), then finite perimeter sets generalize the notion of open $(d+1)$-dimensional sets with smooth boundary in $\R^n$.

The class of finite perimeter sets has excellent compactness properties and a notion of area lower semi-continuous. 

This approach is called "weak" surfaces approach, and it is essentially due to Caccioppoli \cite{cacio} and De Giorgi \cite{degi}. A different approach, working for any $d$ and $n$, referred to as \textit{integral currents}, was introduced by Federer and Flaming in their joint paper \cite{fed}.