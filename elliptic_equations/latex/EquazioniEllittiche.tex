
\documentclass[a4paper,10 pt, twoside]{report}

\usepackage{graphicx}
\usepackage[dvipsnames]{xcolor}
\usepackage{amsfonts}
\usepackage[labelfont=bf]{caption}
\usepackage[pass]{geometry}
\usepackage{amsthm}
\usepackage{amsmath, amssymb}
\usepackage{setspace}
\usepackage[english]{babel}
\usepackage{tikz-cd}
\usepackage{makeidx}         % permette di generare l'indice
\usepackage{fancyhdr}
\usepackage[utf8]{inputenc}
\usepackage{mathtools}
\usepackage{enumitem}
\usepackage{accents}
\usepackage{faktor}
\usepackage{mathrsfs}  
\usepackage{pifont}
\usepackage{mwe}
\usepackage{young}
\usepackage[vcentermath]{youngtab}
\usepackage{hyperref}
\usepackage{scalerel}[2014/03/10]
\usepackage[usestackEOL]{stackengine}

\usepackage{bbm}
\usepackage{PTSansNarrow}
\usepackage[T1]{fontenc}

\usepackage{ragged2e}
\usepackage[framemethod=tikz]{mdframed}
\usepackage{marginnote}
\usepackage{xparse}
\usepackage{pgfplots}
\usepackage{tikzpagenodes}


\usetikzlibrary{calc}
\usetikzlibrary{matrix}
\usetikzlibrary{plotmarks}
\usetikzlibrary{datavisualization}
\usetikzlibrary{datavisualization.formats.functions}
\pgfplotsset{soldot/.style={color=blue,only marks,mark=*}} \pgfplotsset{holdot/.style={color=blue,fill=white,only marks,mark=*}}
\pgfplotsset{compat=1.6}

%%CAUTION
\newcounter{mycaution}
\newcommand\pointeranchor{}
\newcommand\boxanchor{}
\newlength\boxvshift
\newlength\uppertrianglecorner

\newcommand\tikzmark[1]{%
  \tikz[remember picture,overlay]\node[inner xsep=0pt,outer sep=0pt] (#1) {};}

\NewDocumentCommand{\caution}{O{c}O{BrickRed}O{Caution!}m}{%
\stepcounter{mycaution}%
\tikzmark{\themycaution}%
\if#1b\relax
\renewcommand\pointeranchor{mybox\themycaution.south east}%
\renewcommand\boxanchor{south east}%
\setlength\boxvshift{-10pt}%
\setlength\uppertrianglecorner{13pt}%
\else
\if#1t\relax
\renewcommand\pointeranchor{mybox\themycaution.north east}%
\renewcommand\boxanchor{north east}%
\setlength\boxvshift{10pt}%
\setlength\uppertrianglecorner{-7pt}%
\else
\if#1c\relax
\renewcommand\pointeranchor{mybox\themycaution.east}%
\renewcommand\boxanchor{east}%
\setlength\boxvshift{0pt}%
\setlength\uppertrianglecorner{3pt}%
\fi\fi\fi%
\begin{tikzpicture}[remember picture,overlay]
\node[draw=#2,anchor=\boxanchor,xshift=-\marginparsep,yshift=\boxvshift]   
  (mybox\themycaution)
  at ([yshift=3pt]current page text area.west|-\themycaution) 
  {\parbox{\marginparwidth}{\vskip10pt\RaggedRight\small#4}};
\node[fill=white,font=\color{#2}\sffamily,anchor=west,xshift=7pt]
  at (mybox\themycaution.north west) {\ #3\ };
\fill[#2]
  ([yshift=\uppertrianglecorner]\pointeranchor) --
  ([yshift=\uppertrianglecorner-3pt,xshift=3pt]\pointeranchor) --
  ([yshift=\uppertrianglecorner-6pt]\pointeranchor) -- cycle;
\end{tikzpicture}%
}

%%

\makeindex %DDD

\definecolor{coral}{rgb}{1.0, 0.5, 0.31}

\hypersetup{
    colorlinks=true,
    linkcolor=magenta,
    filecolor=magenta,      
    urlcolor=cyan,
}

\definecolor{amaranth}{rgb}{0.9, 0.17, 0.31}

\DeclareRobustCommand\longtwoheadrightarrow
     {\relbar\joinrel\twoheadrightarrow}

\newcommand{\notimplies}{%
  \mathrel{{\ooalign{\hidewidth$\not\phantom{=}$\hidewidth\cr$\implies$}}}}

\pagestyle{plain}
\setlength{\topmargin}{0.0in}
\setlength{\headheight}{0.2in}
\setlength{\headsep}{0.2in}
\setlength{\footskip}{0.5in}
\setlength{\footnotesep}{0.15in}
\setlength{\textheight}{8.3in}
\setlength{\textwidth}{5.5in} % 6
\setlength{\oddsidemargin}{0.5in}
\setlength{\evensidemargin}{0.5in}
\setlength{\parindent}{0.2 in}
\setlength{\parskip}{0.1 in}
\setlength{\marginparwidth}{1.2 in}

\newtheorem{theorem}{Theorem}[chapter]
\newtheorem{lemma}[theorem]{Lemma}
\newtheorem{proposition}[theorem]{Proposition}
\newtheorem{corollary}[theorem]{Corollary}
\theoremstyle{definition}
\newtheorem{definition}[theorem]{Definition}

\newtheorem{remark}{Remark}[chapter]
\newtheorem{example}{Example}[chapter]
\newtheorem*{notation}{Notation}
\newtheorem*{claim}{Claim}
\newtheorem{exercise}{Exercise}[chapter]

\newcommand{\smallO}[1]{\scriptstyle\mathcal{O}}
\DeclarePairedDelimiter\floor{\lfloor}{\rfloor}
\newcommand*\conj[1]{\overline{#1}}
\newcommand{\R}{\mathbb R}
\newcommand{\C}{\mathbb C}
\newcommand{\T}{\mathbb T}
\newcommand{\N}{\mathbb N}
\newcommand{\F}{\mathcal F}
\newcommand{\G}{\mathcal G}
\newcommand{\Z}{\mathbb Z}
\newcommand{\Q}{\mathbb Q}
\newcommand{\p}{\mathbb P}
\newcommand{\sumij}{\sum_{i, \, j=1}^n}
\newcommand{\aij}{a_{i, \, j}}
\newcommand{\Dij}{D^{i, \, j}}
\newcommand{\mfX}{\mathfrak X}
\newcommand{\vertiii}[1]{{\left\vert\kern-0.25ex\left\vert\kern-0.25ex\left\vert #1 
    \right\vert\kern-0.25ex\right\vert\kern-0.25ex\right\vert}}
    
    \newcommand{\vertiiii}[1]{{\left\vert\kern-0.25ex\left\vert\kern-0.25ex\left\vert\kern-0.25ex\left\vert #1 
    \right\vert\kern-0.25ex\right\vert\kern-0.25ex\right\vert\kern-0.25ex\right\vert}}

%SPACES
\newcommand{\Sc}{\mathscr{S}}
\newcommand{\Scp}{\mathscr{S}^\prime}
\newcommand{\distr}{\mathcal{D}^\prime(\Omega)}
\newcommand{\ccs}{\mathcal{D}(\Omega)}

\newcommand{\spt}{\mathrm{spt}}
\newcommand{\D}{\mathcal D}
\newcommand{\s}{\mathcal S}

\newcommand{\weak}{\stackrel{\ast}{\rightharpoonup}}

\DeclareFontFamily{U}{mathx}{\hyphenchar\font45}
\DeclareFontShape{U}{mathx}{m}{n}{
      <5> <6> <7> <8> <9> <10>
      <10.95> <12> <14.4> <17.28> <20.74> <24.88>
      mathx10
      }{}
\DeclareSymbolFont{mathx}{U}{mathx}{m}{n}
\DeclareFontSubstitution{U}{mathx}{m}{n}
\DeclareMathAccent{\widecheck}{0}{mathx}{"71}
\DeclareMathAccent{\wideparen}{0}{mathx}{"75}

\def\cs#1{\texttt{\char`\\#1}}

\newcommand{\can}{\symbol{35}}
\newcommand*{\double}[2][.1ex]{%
  \mathrel{\vcenter{\offinterlineskip%
  \hbox{$#2$}\vskip#1\hbox{$#2$}}}}
\newcommand*{\doublerightarrow}{\double{\longrightarrow}}

\newcommand{\restr}{%
  \,\raisebox{-.127ex}{\reflectbox{\rotatebox[origin=br]{-90}{$\lnot$}}}\,%
}

\newcommand\subgroup{\leqslant}
\newcommand\normsubgroup{\leq}

\makeatletter
\renewcommand*\env@matrix[1][*\c@MaxMatrixCols c]{%
  \hskip -\arraycolsep
  \let\@ifnextchar\new@ifnextchar
  \array{#1}}
\makeatother

\definecolor{darkgreen}{rgb}{0.0, 0.2, 0.13}

\newcommand{\bsquare}{\item[\color{magenta}\ding{110}]} 
\newcommand{\barrow}{\item[\color{blue}\ding{228}]}
\newcommand{\bwarrow}{\item[\color{gray}\ding{227}]}

\def\dashint{\,\ThisStyle{\ensurestackMath{%
  \stackinset{c}{.2\LMpt}{c}{.5\LMpt}{\SavedStyle-}{\SavedStyle\phantom{\int}}}%
  \setbox0=\hbox{$\SavedStyle\int\,$}\kern-\wd0}\int}
\def\ddashint{\,\ThisStyle{\ensurestackMath{%
  \stackinset{c}{.2\LMpt}{c}{.5\LMpt+.2\LMex}{\SavedStyle-}{%
    \stackinset{c}{.2\LMpt}{c}{.5\LMpt-.2\LMex}{\SavedStyle-}{%
      \SavedStyle\phantom{\int}}}}\setbox0=\hbox{$\SavedStyle\int\,$}\kern-\wd0}\int}
      
      
\fancyhf{}
% Put the page number at the right edge of odd pages, and left edge of even pages.
\fancyhead[LO,RE]{\textbf \thepage}
% Custom text at the left edge of odd pages, and right edge of odd pages.
\fancyhead[RO]{ \rightmark}
\fancyhead[LE]{ \leftmark}

% Repeat for \fancyfoot if needed, e.g.
% Some decorative symbol at the centre of both odd and even pages
\fancyfoot[C]{ }

% Set this length to 0pt if you don't want any lines!
\renewcommand{\headrulewidth}{1pt}

% Apply the fancy header style
\pagestyle{fancy}


\begin{document}
\newpage \thispagestyle{empty}

\begin{center}

\begin{spacing}{1.5}
{\huge  \sf Lecture Notes}\\
\vspace*{\fill}
\end{spacing}
\begin{spacing}{2.5}
\textbf{\huge Elliptic Equations}\\[0.5cm]
\vspace*{\fill}
\begin{minipage}{5cm}
\centering {\textit{Course held by}}
\end{minipage}
\hspace*{\fill}
\begin{minipage}{5cm}
\centering {\textit{Notes written by}} \\
\end{minipage}
\end{spacing}

\begin{spacing}{1.3}

\begin{minipage}{5cm}
\centering {\textbf{\large Prof. Antonio Tarsia}}
\end{minipage}
\hspace*{\fill}
\begin{minipage}{5cm}
\centering {\textbf{\large Francesco Paolo Maiale}}
\end{minipage}

\vspace*{\fill}

\textnormal{\large Department of Mathematics \\[0.4em] Pisa University \\[0.4em] \today}

\end{spacing}
\end{center}

\newpage \thispagestyle{empty}
\begin{center}
\vspace*{1cm}
\begin{spacing}{2.5}
 {
\textbf{\huge Disclaimer}
}
\end{spacing} \end{center}

\begin{spacing}{1.2}
These \textbf{incomplete} notes came out of the \textit{Elliptic Equations} course, held by Professor Antonio Tarsia in the second semester of the academic year 2016/2017.

I have used them to study for the exam; hence they have been reviewed thoroughly. Unfortunately, there may still be many mistakes and oversights; to report them, send me an email at \textbf{francescopaolo (dot) maiale (at) gmail (dot) com}.
\end{spacing}


{ \setlength{\parskip}{0.05 in}

\clearpage                       % Otherwise \pagestyle affects the previous page.
{                                % Enclosed in braces so that re-definition is temporary.
  \pagestyle{empty}              % Removes numbers from middle pages.
  \fancypagestyle{plain}         % Re-definition removes numbers from first page.
  {
    \fancyhf{}%                       % Clear all header and footer fields.
    \renewcommand{\headrulewidth}{0pt}% Clear rules (remove these two lines if not desired).
    \renewcommand{\footrulewidth}{0pt}%
  }
  \tableofcontents
  \thispagestyle{empty}          % Removes numbers from last page.
}}


\chapter{Introduction}

In this chapter, we introduce the main topics of the course and give a brief overview of what we will see and what we will be able to prove by the end of the course.

\section{Plateau's Problem}

The primary goal and the motivating example of this course is the \textbf{Plateau's problem}, that is, the problem to find the $d$-dimensional surface $\Sigma$ of the minimal area with prescribed $(d-1)$-dimensional boundary $\Gamma$.

By the end, we will be able to prove that a solution indeed exists, but we will not find it explicitly since it is a $NP$ (hard) numerical problem.

As of now, the problem is not well defined. In fact, the notions of \textit{surface}, \textit{area}, and \textit{boundary} make sense in the smooth setting but, as the examples below show, we need to work in a less regular setting.

More precisely, requiring the surface to be smooth is not enough for modeling reasons (e.g., dip a wire frame into a soap solution, form a soap film, and look for the minimal surface whose boundary is the wire frame), and also for existence reasons.

\begin{example} Here we give a list of Plateau's problems with prescribed boundary conditions, and we write down the correct solutions, without proving anything. \mbox{}
\begin{enumerate}[label=\textbf{(\alph*)}]
\item Let us identify $\R^4 \cong \C \times \C$ and, if $d = 2$, let us consider the smooth boundary given by
\begin{equation*} \Gamma_1 := \left( S^1 \times \{0\} \right) \cup \left( \{0\} \times S^1 \right). \end{equation*}
Surprisingly, every minimizing sequence of smooth surfaces converges to a surface which is not smooth at all. Indeed, the solution of the problem is
\begin{equation*} \Sigma_1 := \left( D^2 \times \{0\} \right) \cup \left( \{0\} \times D^2 \right). \end{equation*}
The surface $\Sigma_1$ is clearly singular at the origin, but the singularity may be removed (by factorizing it into two nonsingular surfaces).
\item Let us identify $\R^4 \cong \C \times \C$ and, if $d = 2$, let us consider the smooth boundary given by
\begin{equation*} \Gamma_2 := \left\{ (z^2, \, z^3) \: : \: z \in S^1 \right\}. \end{equation*}
The solution to the Plateau's problem is
\begin{equation*} \Sigma_2 := \left\{ (z^2, \, z^3) \: : \: z \in D^2 \right\}, \end{equation*}
which is a non-smooth surface, whose singularity cannot be removed (since the polynomial $z_1^3 = z_2^2$ cannot be factorized).
\item Let us identify $\R^8 \cong \R^4 \times \R^4$ and, if $d = 7$, let us consider the smooth boundary given by
\begin{equation*} \Gamma_3 := S^3 \times S^3. \end{equation*}
The minimal surface of prescribed boundary $\Gamma_3$ is
\begin{equation*} \Sigma_3 := \left\{ (x_1, \, x_2) \in \R^4 \times \R^4 \: : \: |x_1| = |x_2| \leq 1 \right\}. \end{equation*}
\end{enumerate}
\end{example}

To conclude this introductive chapter, we give a brief overview of the main approaches (studied in this course) to the Plateau's problem, as $d$ ranges between $1$ and $\infty$.

\section{Geodesics problem ($d=1$)}

The geodesics problem (that is, find the shortest curve connecting two points) is, surprisingly, still an open in the non-Riemannian setting. However, in the Riemannian setting, the geodesics problem is completely solved.

Indeed, if we consider the curves parametrized by paths, the \textit{length} is a well-defined notion, and the associated functional is lower semi-continuous and coercive; hence the compactness is easy to prove.

There are many possible approaches to the geodesics problem, e.g., the Steiner approach and the set theoretical approach, which we describe briefly in the remainder of the section.

\paragraph{Steiner Problem.} It is also called networks approach, and it is used to prove the existence of the geodesics and find the explicit expression for it. The reader may consult \cite{steinerpro} for a detailed dissertation on the topic.

\paragraph{Set Theoretical Approach.} The main idea is to find a closed and connected set $\Sigma$ of minimum \textit{length}, containing a given finite set $\Gamma$. As we shall see later in the course, in this case the length is a well defined concept: the \textit{Hausdorff distance}.

In fact, if $X$ is a suitable space (metric, endowed with Hausdorff distance, etc...), then the class defined by
\begin{equation*}\mathcal{X} := \left\{ K \subseteq X \: : \: K \, \, \text{compact and connected} \right\} \end{equation*}
is compact and, by Gotab theorem\footnote{\cite{falconer} Let $\mathscr{C}$ be an infinite collection of non-empty compact sets all lying in a bounded portion $B$ of $\R^n$. Then there exists a sequence $\{E_j\}$ of distinct sets of $\mathfrak{C}$ convergent in the Hausdorff metric to a non-empty compact set $E$.}, $\mathcal{H}^1$ is lower semi-continuous on $X$. 

\section{"Surface" Problem ($d>1$)}

\paragraph{3.} The Plateau's problem is much harder when $d = 2$, but there are still many approaches possible some of which relying, in a certain sense, on the work already done in the geodesics case.

\paragraph{Set Theoretical Approach.} This approach is highly nontrivial. For example, one may ask what does it mean that a compact set $\Sigma$ spans a boundary $\Gamma$? Moreover, there is another problem one should deal with: the $2$-dimensional Hausdorff measure $\mathcal{H}^2$ is, generally, not lower semi-continuous. The reader may consult \cite{Reifenberg1960} for a complete treatise of the topic.

\begin{remark}Suppose that $d = 2$, $n = 3$ and that $\Sigma$ is a surface with boundary $\Gamma$. If $\gamma$ is another closed curve, linked to $\Gamma$ (by a nonzero linking number), then $\gamma \cap \Sigma \neq \emptyset$. \end{remark}

\paragraph{Parametric Approach.} This method is essentially due to Douglas \cite{douglas}. The main idea is the following: since a parametrization $\phi : D^2 \to \R^n$ defines surfaces in $\R^n$, the area functional is well-defined and given by the formula
\begin{equation*}A(\phi) := \int_{D^2} \left| \frac{\partial \, \phi}{\partial \, s_1} \wedge \frac{\partial \, \phi}{\partial \, s_2} \right| \, \mathrm{d}s_1 \, \mathrm{d}s_2. \end{equation*}
On the other hand, the existence through lower semi-continuity and the compactness are a delicate matter, since coercivity is not an easy property to obtain (the integrand is similar to a determinant).

There is a trick which is similar to the one we can use to find geodesics in the differential geometry setting. More precisely, we consider the functional
\begin{equation*}E(\phi) := \frac{1}{2} \, \int_{D^2} \left| \nabla \phi \right|^2 \, \mathrm{d}s_1 \, \mathrm{d}s_2. \end{equation*}
If we find a minimal point $\phi$ for $E$, then $\phi$ will be a \textbf{conformal parametrized} minimum for $A$. This trick, on the other hand, heavily depends on a nontrivial theorem: every such $\Sigma$ admits a conformal re-parametrization.

The lack of conformal parametrization, though, is what stop us from extending the same trick to dimension $d$ strictly bigger than $2$.

\paragraph{Higher Dimension.} If the codimension of $\Sigma$ is equal to $1$ (that is, $n = d+1$), then finite perimeter sets generalize the notion of open $(d+1)$-dimensional sets with smooth boundary in $\R^n$.

The class of finite perimeter sets has excellent compactness properties and a notion of area lower semi-continuous. 

This approach is called "weak" surfaces approach, and it is essentially due to Caccioppoli \cite{cacio} and De Giorgi \cite{degi}. A different approach, working for any $d$ and $n$, referred to as \textit{integral currents}, was introduced by Federer and Flaming in their joint paper \cite{fed}.
\chapter{Existence Results for Elliptic Problems}

In this chapter we are concerned with the existence of solutions  for elliptic problems, both in variational and in nonvariational form.

\section{Near Operators Theory}

\begin{definition}[Nearness] Let $\mathfrak{X}$ be any set, let $\mathcal{B}$ be a Banach space and let $A, \, B : \mathfrak{X} \to \mathcal{B}$ be operators. We say that $A$ is \textit{near} to $B$ if there exist a constant $\alpha > 0$ and $k \in (0, \, 1)$ such that
\begin{equation} \label{eq:nearness} \left\| B(x_1) - B(x_2) - \alpha \left[ A(x_1) - A(x_2) \right] \right\| \leq k \, \left\| B(x_1) - B(x_2) \right\|, \qquad \forall \, x_1, \, x_2 \in \mathfrak{X}. \end{equation}\end{definition}

\begin{remark} The condition is in general \textbf{not} symmetric: if $A$ is near $B$, then it's not necessarily true that $B$ is near $A$. On the other hand, if $\mathcal{B}$ is a Hilbert space, then the condition is symmetric. %DIM.
\end{remark}

\begin{lemma}Let $\mathfrak{X}$ be any set, let $\mathcal{B}$ be a Banach space, let $A, \, B : \mathfrak{X} \to \mathcal{B}$ be operators and assume that $A$ is near to $B$. Then the following inequalities are satisfied for any $x_1, \, x_2 \in \mathfrak{X}$:
\begin{equation} \label{ineq1} \left\| B(x_1) - B(x_2) \right\| \leq \frac{\alpha}{1 - k} \, \left\| A(x_1) - A(x_2) \right\|,\end{equation} 
\begin{equation} \label{ineq2} \left\| A(x_1) - A(x_2) \right\| \leq \frac{k+1}{\alpha} \, \left\| B(x_1) - B(x_2) \right\|. \end{equation}\end{lemma}

\begin{proof} We only prove the first inequality, and leave the second one to the reader. Since $A$ is near to $B$, we have that
\begin{equation*} \begin{aligned} \left\| B(x_1) - B(x_2) \right\| & \leq \left\| B(x_1) - B(x_2) - \alpha \left[ A(x_1) - A(x_2) \right] \right\| + \alpha \, \left\| A(x_1) - A(x_2) \right\| \leq \\ & \leq k \, \|B(x_1) - B(x_2)\|  + \alpha \, \left\| A(x_1) - A(x_2) \right\| , \end{aligned}\end{equation*} 
from which it follows easily that
\begin{equation*}\left(1 - k\right) \cdot \left\| B(x_1) - B(x_2) \right\| \leq \alpha \, \left\| A(x_1) - A(x_2) \right\|.\end{equation*} \end{proof}

\begin{lemma}\label{lemma:infjd}Let $\mathfrak{X}$ be any set, let $\mathcal{B}$ be a Banach space, let $A, \, B : \mathfrak{X} \to \mathcal{B}$ be operators and assume that $A$ is near to $B$. Then $A$ is injective if and only if $B$ is injective. \end{lemma}

\begin{proof}This is a immediate consequence of the previous Lemma, since \eqref{ineq1} and \eqref{ineq2} easily implies that $A(x_1) = A(x_2)$ if and only if $B(x_1) = B(x_2)$. \end{proof}

\begin{lemma}Let $B : \mathcal{X} \to \mathcal{B}$ be an injective operator. Then $(\mathcal{X}, \, d)$ is a metric space, where
\begin{equation*}d(x, \, y)  = \| B(x) - B(y) \|_{\mathcal{B}}, \qquad \forall \, x, \, y \in \mathcal{X}.\end{equation*} 
Moreover, if $B$ is also surjective, then the metric space $(\mathcal{X}, \, d)$ is complete. \end{lemma}

\begin{proof} We want to prove that $d$ is a distance on $\mathcal{X}$, that is, the characterizing properties hold true. By definition it is greater or equal to zero and, since $B$ is injective, it follows that
\begin{equation*}d(x, \, y)  = 0 \iff B(x) = B(y) \iff x = y. \end{equation*}
The symmetric property and the triangle inequality are straightforward consequences of the definition, therefore we will not give more details.

Assume that $B$ is bijective and let $(x_n)_{n \in \N} \subset \mathcal{X}$ be a Cauchy sequence. Then $\left(B(x_n) \right)_{n \in \N} \subset \mathcal{B}$ is a Cauchy sequence and, by completeness of $\mathcal{B}$, it converges to $y \in \mathcal{B}$. Since $B$ is surjective, there is $x \in \mathcal{X}$ such that $B(x) = y$; it follows that
\begin{equation*} d(x_n, \, x) = \| B(x_n) - y \|_{\mathcal{B}} \xrightarrow{n \to + \infty} 0, \end{equation*}
and this concludes the proof.  \end{proof}

\begin{theorem}\label{bijectle}Let $\mathfrak{X}$ be any set, let $\mathcal{B}$ be a Banach space, let $A, \, B : \mathfrak{X} \to \mathcal{B}$ be operators and assume that $A$ is near to $B$. If $B$ is a bijection, then also $A$ is a bijection. \end{theorem}

\begin{proof} It suffices to prove that $A$ is a \textit{surjective} operator. In fact, it is \textit{injective} as a consequence of \hyperref[lemma:infjd]{Lemma \ref{lemma:infjd}} which was proved earlier.

Let $f \in \mathcal{B}$ be any element. We want to prove that there exists a (unique) $u \in \mathcal{X}$ such that $A(u) = f$ or, equivalently, that
\begin{equation*} B(u) = B(u) - \alpha \, A(u) + \alpha \, f =:F(u). \end{equation*}

For any $u \in \mathcal{X}$, $F(u)$ belongs to $\mathcal{B}$, hence there is one and only one $v \in \mathcal{X}$ such that $B(v) = F(u)$, as $B$ is a bijective operator. 

Let us denote by $T : \mathcal{X} \to \mathcal{X}$ the application that sends $u$ in the unique solution $v$. If $v = T(u)$ and $w = T(z)$, it turns out that
\begin{equation*} \begin{aligned} d(v, \, w) & = \left\| B(v) - B(w) \right\|_{\mathcal{B}} = \left\| F(u) - F(z) \right\|_{\mathcal{B}} = \\ &  = \left\| B(v) - B(w) - \alpha \left[ A(v) - A(w) \right] \right\| \leq k \, \left\| B(v) - B(w) \right\|, \end{aligned} \end{equation*}
i.e., $T$ is a contraction (since $k$ is strictly less than one).

Finally $\mathcal{X}$ is a complete metric space, thus there is one and only one fixed point $u \in \mathcal{X}$ of $T$, that is, $A$ is surjective:
\begin{equation*} T(u) = u \implies B(u) = F(u) \implies A(u) = f. \end{equation*}\end{proof}

\paragraph{Equivalence Relation.} Let us endow the set $\mathcal{X}$ with the following equivalence relation:
\begin{equation*} u \sim v \iff B(u) = B(v). \end{equation*}
Let us denote by $[u]_\mathcal{X}$ the equivalence class of $u$ and let $X = \faktor{\mathcal{X}}{\sim}$. We can define two operators $A^\ast, \, B^\ast : X \to \mathcal{B}$ by setting
\begin{equation*} A^\ast( [u]_{\mathcal{X}} ) = A(u) \qquad \text{and} B^\ast( [u]_{\mathcal{X}} ) = B(u), \end{equation*}
then, if $A$ is near $B$, also $A^\ast$ is near $B^\ast$ (with the same constants). 

If we assume that $B$ is surjective, then $B^\ast$ is bijective (by definition). It follows from \hyperref[bijectle]{Theorem \ref{bijectle}} that $A^\ast$ is bijective, and this directly implies that $A$ is surjective.

\begin{theorem}Let $\mathfrak{X}$ be any set, let $\mathcal{B}$ be a Banach space, let $A, \, B : \mathfrak{X} \to \mathcal{B}$ be operators and assume that $A$ is near to $B$. If $B$ is a surjective operator, then also $A$ is a surjective operator. \end{theorem}

\begin{theorem}\label{th:conm} Let $\mathfrak{X}$ be any set, let $\mathcal{B}$ be a Banach space and let $\{A_t : \mathcal{X} \to \mathcal{B}\}_{t \in [0, \, 1]}$ be a family of operators such that: \mbox{}
\begin{enumerate}[label=\textbf{(\alph*)}]
\item There exists $t_0 \in [0, \, 1]$ such that $A_{t_0}$ is a bijection.
\item There exists $c > 0$ such that, for any $s, \, t \in [0, \, 1]$ and any $x, \, y \in \mathcal{X}$,
\begin{equation*} \left\| A_t(x) - A_t(y) - \left[ A_s(x) - A_s(y) \right] \right\|_{\mathcal{B}} \leq c \, |t - s| \, \|A_t(x) - A_t(y) \|_{\mathcal{B}}. \end{equation*}
\end{enumerate}
Then, for any $s \in [0, \, 1]$, it turns out that $A_s$ is a bijection.\end{theorem}

\begin{proof}Set $I := \left\{ t \in [0, \, 1] \: \left| \: \text{$A_t$ is a bijection} \right. \right\}$. By connectedness of $[0, \, 1]$, it suffices to prove that $I$ is nonempty, open and closed in the subspace topology.

The assumption \textbf{(a)} implies that $I$ is nonempty. Let $t \in I$ and let $\delta > 0$ be such that, for any $s \in [t - \delta, \, t + \delta] \cap [0, \, 1]$, it turns out that $c \, |t - s| < 1$. Then the inequality \textbf{(b)} implies that $A_s$ near $A_t$, and thus $A_s$ is bijective (i.e. $s \in I$) by \hyperref[bijectle]{Theorem \ref{bijectle}}.

Let $(s_n)_{n \in \N} \subset I$ be a converging sequence and let $s \in [0, \, 1]$ be its limit. There exists $N \in \N$ such that $c \, |s_n - s| < 1$, for any $n \geq N$, therefore $A_s$ is near $A_{s_N}$ and, again, it is bijective.\end{proof}

\paragraph{Fréchet Differential} Let $\mathcal{B}_1$ and $\mathcal{B}_2$ be Banach spaces and let $F : U \subset \mathcal{B}_1 \to \mathcal{B}_2$. The function $F$ is differentiable in the sense of Fréchet at $u_0 \in U$ if there is a linear continuous application $L : \mathcal{B}_1 \to \mathcal{B}_2$ such that
\begin{equation} \lim_{h \to 0} \frac{ \| F(u_0 + h) - F(u_0) - L \, h \|_{\mathcal{B}_2}}{\|h\|_{\mathcal{B}_1}} = 0, \end{equation}
and we denote it by $\mathrm{d}F_{u_0}$.

We say that $F$ is continuously differentiable at $u_0$ if it is differentiable in a neighborhood $W$ of $u_0$ and the mapping $W \ni u \mapsto \mathrm{d}F_u \in \mathcal{L}(\mathcal{B}_1, \, \mathcal{B}_2)$ is continuous at $u_0$.

\begin{theorem} Let $(Y, \, \| \cdot \|_Y)$ and $(Z, \, \| \cdot \|_Z)$ be two Banach spaces, let $y_0 \in Y$ be any point and let $F : U \to Z$ be a function defined on a neighborhood of $y_0$. Assume that \mbox{}
\begin{enumerate}[label=\textbf{(\alph*)}]
\item $F \in C^1(U)$;
\item the Fréchet differential $\mathrm{d}F_{y_0} : Y \to Z$ is invertible.
\end{enumerate}
Then there exist $k \in (0, \, 1)$ and a neighborhood $W \subset U$ such that for any $y_1, \, y_2 \in W$
\begin{equation*} \left\| \mathrm{d}F_{y_0}[y_1-y_2] - \left[F(y_1) - F(y_2)\right] \right\|_Z \leq \left\| \mathrm{d}F_{y_0}[y_1 - y_2] \right\|_Z. \end{equation*} \end{theorem}
 
\section{Historical Digression on Near Operators}

The idea of introducing the notion of \textit{nearness} between operators originates from the existence and uniqueness problem of nonvariational elliptic equations of the following kind:
\begin{equation} \label{problema} \begin{cases} u \in H^2(\Omega) \cap H_0^1(\Omega) \\ \\ \displaystyle\sum_{i, \, j = 1}^n a_{i, \, j}(x) \, D^{i, \, j} \, u(x) = f(x) & \text{for $x \in \Omega$}, \end{cases} \end{equation}
where $\Omega \subset \R^n$ is bounded and - for simplicity only - convex, $f \in L^2(\Omega)$, the coefficients $a_{i, \, j} \in L^\infty(\Omega)$ and the matrix $A = \left(a_{i, \, j} \right)_{i, \, j = 1, \, \dots, \, n}$ is uniformly elliptic and symmetric.

\vspace{1.8mm}
If $n > 2$ the problem \eqref{problema} is not well-defined under the solely assumption of uniformly ellipticity (see \hyperref[talenti]{Example \ref{talenti}}). Therefore, it is necessary to introduce more restrictive condition on the coefficients of the matrix $A$, e.g., regularity ($a_{i, \, j} \in C^0(\Omega)$) or algebraic conditions.

\begin{definition}[Corder Condition] Let $A = \left(a_{i, \, j} \right)_{i, \, j = 1, \, \dots, \, n}$ be a matrix such that $\|A(x)\| \neq 0$ for almost every $x \in \Omega$. We say that $A(x)$ satisfies the \textit{Corder condition} if there exists $\epsilon > 0$ such that
\begin{equation} \label{concord} \frac{ \left( \sum_{i = 1}^{n} a_{i, \, i}(x) \right)^2 }{\sum_{i, \, j = 1}^n a_{i, \, j}^2(x) } \geq n - 1 + \epsilon \qquad \text{for almost every $x \in \Omega$}. \end{equation} \end{definition}

\begin{definition}[Campanato Condition] Let $A = \left(a_{i, \, j} \right)_{i, \, j = 1, \, \dots, \, n}$ be a matrix. We say that $A(x)$ satisfies the \textit{$A_x$ condition} if there are real constants $\sigma, \, \gamma$ strictly positive, $\delta \geq 0$ and a function $a(x) \in L^\infty(\Omega)$ such that $\gamma + \delta < 1$, $a(x) \geq \sigma > 0$ and
\begin{equation} \label{condax} \left| \sum_{i = 1}^n \xi_{i, \, i} - a(x) \cdot \sum_{i, \, j = 1}^n a_{i, \, j}(x) \, \xi_{i, \, j} \right| \leq \gamma \cdot \left( \sum_{i, \, j = 1}^n \xi_{i, \, j}^2 \right)^{\frac{1}{2}} + \delta \cdot \left| \sum_{i = 1}^n \xi_{i, \, i} \right|, \end{equation}
for any matrix $\xi \in M_n(\R)$ and for almost every $x \in \Omega$. \end{definition}

\begin{remark}The Campanato condition implies the uniform ellipticity of $A$ (but, for $n > 2$, it is not equivalent) over $\Omega$. Indeed, let $\xi = \left( \eta_i \, \eta_j \right)_{i, \, j = 1, \, \dots, \, n}$ and substitute this in \eqref{condax}:
\begin{equation*} \left| \sum_{i = 1}^n \eta_i^2 - a(x) \cdot \sum_{i, \, j = 1}^n a_{i, \, j}(x) \, \eta_i \, \eta_j \right| \leq \gamma \cdot \left( \sum_{i, \, j = 1}^n \eta_i^2 \, \eta_j^2 \right)^{\frac{1}{2}} + \delta \cdot \left| \sum_{i = 1}^n \eta_i^2 \right|. \end{equation*}
It follows that
\begin{equation*}\left[ 1 - (\gamma + \delta) \right] \cdot \sum_{i = 1}^n \eta_i^2 \leq a(x) \cdot \sum_{i, \, j = 1}^n a_{i, \, j}(x) \, \eta_i \, \eta_j,\end{equation*}
and, if we let $\mu = \sup_{\Omega}a(x)$, it turns out that
\begin{equation*}\frac{\left[ 1 - (\gamma + \delta) \right]}{\mu} \cdot \sum_{i = 1}^n \eta_i^2 \leq \sum_{i, \, j = 1}^n a_{i, \, j}(x) \, \eta_i \, \eta_j ,\end{equation*}
which is the uniform ellipticity condition. \end{remark}

Observe that the Corder condition and the $A_x$ condition are actually \textbf{equivalent}. The proof is rather tedious, but simple, thus it is left to the reader.

\paragraph{Condition $A_x$.} In this brief paragraph, we want to motivate the Campanato condition and prove that the problem \eqref{problema} admits one and only one solution in that case.

\begin{theorem}\label{th:unicasol}Assume that the matrix $A = \left(a_{i, \, j} \right)_{i, \, j = 1, \, \dots, \, n}$ satisfies the Campanato condition. Then there exists one and only one solution of \eqref{problema}. \end{theorem}

\begin{proof}[Proof 1]Let us consider
\begin{equation} \Delta u(x) = \alpha \, f(x) + \Delta w(x) - \alpha \sumij \aij(x) \, D^{i, \, j} \, w(x), \label{problemapart} \end{equation}
and let us define the application $\mathcal{T} : H^2(\Omega) \cap H_0^1(\Omega) \to H^2(\Omega) \cap H_0^1(\Omega)$, which sends $w$ to the solution $u$ of the problem \eqref{problemapart}.

We now prove that $\mathcal{T}$ is a contraction of the space $H^2(\Omega) \cap H_0^1(\Omega)$, provided that the $A_x$ condition is satisfied. In order to ease the notation, we set
\begin{equation*} \Gamma_{i,  \, j}^{n} \, w(x) := \sumij \aij(x) \, D^{i, \, j} \, w(x). \end{equation*}
By \hyperref[rem:normeequ]{Remark \ref{rem:normeequ}} it follows that the $H^2(\Omega) \cap H_0^1(\Omega)$-norm is equivalent to
\begin{equation*} \| \Delta \, k(x) \|_{L^2(\Omega)} = \int_{\Omega} \left| \Delta \, k(x) \right|^2 \, \mathrm{d}x, \end{equation*}
therefore we can easily estimate the norm of the difference between two points
\begin{equation*} \begin{aligned} \| \mathcal{T}(w_1) - \mathcal{T}(w_2) \|_{H^{2, \, 2}(\Omega)}^2 & \leq \int_\Omega \left| \Delta u_1(x) - \Delta u_2(x) \right|^2 \, \mathrm{d}x = \\ & = \int_\Omega \left| \Delta w_1(x) - \alpha \, \Gamma_{i, \, j}^{n} \, w_1(x) - \left[ \Delta w_2(x) - \alpha \, \Gamma_{i, \, j}^{n} \, w_2(x) \right] \right|^2 \, \mathrm{d}x  = \\ & = \int_\Omega \left| \Delta \left(w_1(x) - w_2(x) \right) - \alpha \, \Gamma_{i, \, j}^n \, \left(w_1(x) - w_2(x) \right) \right|^2 \, \mathrm{d}x \stackrel{{\color{red}(\ast)}}{\leq} \\ & \stackrel{{\color{red}(\ast)}}{\leq} \int_\Omega \left\{ \gamma \cdot \left[ \Dij \left(w_1(x) - w_2(x) \right)^2 \right]^{\frac{1}{2}} + \delta \cdot \left| \Delta \left(w_1(x) - w_2(x) \right) \right| \right\}^2 \, \mathrm{d}x, \end{aligned} \end{equation*}
where the inequality ${\color{red}(\ast)}$ follows from a straightforward application of the Campanato condition. Notice now that for any $a, \, b \in \R$, it follows that
\begin{equation*} (\gamma \, a + \delta \, b)^2 \leq \gamma \, (\gamma + \delta) \, a^2 + \delta \, (\gamma + \delta) \, b^2, \end{equation*}
therefore
\begin{equation*} \begin{aligned}\dots &  \leq \int_\Omega \left[ \gamma \, (\gamma + \delta) \, D_{i, \, j} \, \left(w_1(x) - w_2(x) \right)^2 + \delta \, (\gamma + \delta) \, \left| \Delta \left(w_1(x) - w_2(x) \right) \right| \right]^2 \, \mathrm{d}x \stackrel{{\color{blue}(\ast)}}{\leq} \\ & \stackrel{{\color{blue}(\ast)}}{\leq} (\gamma + \delta)^2 \cdot \left\| \Delta \left(w_1(x) - w_2(x) \right) \right\|_{L^2(\Omega)}^2, \end{aligned} \end{equation*}
where ${\color{blue}(\ast)}$ is the \textbf{Miranda-Talenti} inequality (see \hyperref[MtLemma]{Lemma \ref{MtLemma}}).

We conclude that $\mathcal{T}$ is a contraction of the space $H^2(\Omega) \cap H^0(\Omega)$ into itself (again, by \hyperref[rem:normeequ]{Remark \ref{rem:normeequ}}), therefore it admits one and only one fixed point (i.e. the solution).
\end{proof}

\begin{lemma}[Miranda-Talenti]  \label{MtLemma} Let $\Omega$ be a convex subset of $\R^n$, and let $u \in H^2(\Omega) \cap H_0^1(\Omega)$ be a function. Then we have the following inequality:
\begin{equation*} \left\| u \right\|_{H^2(\Omega)} \leq \left\| \Delta \, u \right\|_{L^2(\Omega)}. \end{equation*}
\end{lemma}

\begin{proof} The inequality follows easily from the identity
\begin{equation*} \sum_{i, \, j = 1}^{n} (D_{i, \, j})^2 + \sum_{i, \, j= 1}^{n} \left[ D_{i, \, i}u \, D_{j, \, j} u - (D_{i, \, j})^2 \right] = \left(\Delta \, u \right)^2, \end{equation*}
which can be easily derived if one knows that the average curvature of $\partial \, \Omega$ is strictly negative (by convexity). \end{proof}

\begin{remark}\label{rem:normeequ} The Miranda-Talenti inequality implies that the $H^2(\Omega)$-norm and the $L^2(\Omega)$-norm of the laplacian are equivalents, but this is a more general fact holding also on non-convex subsets.

Indeed, it is obvious (by definition) that for any $u \in H^2(\Omega) \cap H_0^1(\Omega)$ there exists a constant $c > 0$ such that
\begin{equation*} \left\| \Delta \, u \right\|_{L^2(\Omega)} \leq c \cdot \left\| u \right\|_{H^2(\Omega)}. \end{equation*}
On the other hand, the divergence theorem and the Poincarè inequality immediately imply the opposite inequality, that is,
\begin{equation*} \begin{aligned} \int_{\Omega} \left| \nabla \, u(x) \right|^2 \, \mathrm{d}x & = \int_{\Omega} \nabla \, u(x) \cdot \nabla \, u(x) \, \mathrm{d}x \leq \\ & \leq \left( \int_{\Omega} \left| \Delta \, u (x) \right|^2 \, \mathrm{d}x \right)^{\frac{1}{2}} \cdot \left( \int_{\Omega} \left| u (x) \right|^2 \, \mathrm{d}x \right)^{\frac{1}{2}} \leq \\ & \leq c(\Omega) \cdot \left\| \Delta \, u \right\|_{L^2(\Omega)} \cdot \left\| u \right\|_{H^2(\Omega)}.\end{aligned} \end{equation*}
\end{remark}

We now give another proof of \hyperref[th:unicasol]{Theorem \ref{th:unicasol}} which relies on the near operator theory we have developed in the first section.

\begin{proof}[Proof 2] Let us set
\begin{equation*} B \, u(x) := \Delta \, u(x) \qquad \text{and} \qquad A \, u(x) := \sumij a_{i, \, j}(x) \, D^{i, \, j} \, u(x). \end{equation*}
The idea is to prove that the operator $A$ is a bijection from $H^2(\Omega) \cap H_0^1(\Omega)$ to $L^2(\Omega)$. We first notice that the following two facts hold true: \mbox{}
\begin{enumerate}[label=\textbf{(\arabic*)}]
\item There is an algebraic relation between $A$ and $B$.
\item The laplacian $\Delta$ is a bijection between $H^2(\Omega) \cap H_0^1(\Omega)$ to $L^2(\Omega)$.
\end{enumerate}
The proof of the first point is easy, and it is left to the reader. The second fact, on the other hand, will be proved later in the course.

The Campanato condition implies that $A$ is near $B$, therefore by \hyperref[bijectle]{Theorem \ref{bijectle}} we conclude that also $A$ is a bijection, which is exactly what we wanted to prove.\end{proof}

\begin{example}[Talenti] \label{talenti}We want to prove that, for $n > 2$, the problem \eqref{problema} is in general not well posed in $H^2(\Omega) \cap H_0^1(\Omega)$, provided that $f \in L^2(\Omega)$ and $a_{i, \, j} \in L^\infty(\Omega)$.

\vspace{2mm}
Let $\Omega = S(0, \, r) = \partial \, B(0, \, r)$ be the $n$-dimensional sphere, let $\lambda \in (0, \, 1)$ be a real number and consider the problem associated to the equation
\begin{equation*} \mathcal{A}(u) := \sumij a_{i, \, j}(x) \, D^{i, \, j} \, u(x) = 0. \end{equation*}
The coefficients of the matrix $A$ are given by
\begin{equation*} a_{i, \, j}(x) = \delta_{i, \, j} + b \, \frac{x_i \, x_j}{\|x\|^2}, \qquad \text{where} \qquad b = - 1 + \frac{n - 1}{1 - \lambda} \end{equation*}
for any $i, \, j = 1, \, \dots, \, n$.

We first prove that the matrix $A = \left(a_{i, \, j} \right)_{i, \, j = 1, \, \dots, \, n}$ is uniformly elliptic on $\Omega$. Indeed, it turns out that
\begin{equation*} \begin{aligned} \sumij \left( \delta_{i, \, j} + b \, \frac{x_i \, x_j}{\|x\|^2} \right) \, \xi_i \, \xi_j & = \sum_{i = 1}^n \xi_i^2 + \sumij b \, \frac{x_i \, x_j \, \xi_i \, \xi_j}{\|x\|^2} = \\ & = \|\xi\|^2 \, \left( 1 + b \, \sumij \frac{x_i \, x_j \, \xi_i \, \xi_j}{\|x\|^2 \, \|\xi\|^2} \right) \leq 1,\end{aligned} \end{equation*} 
since $b > - 1$ and by Cauchy-Schwartz inequality also
\begin{equation*}  \sumij \frac{x_i \, x_j \, \xi_i \, \xi_j}{\|x\|^2 \, \|\xi\|^2} = \frac{ (x, \, \xi)^4 }{\|x\|^2 \, \|\xi\|^2} \leq 1.\end{equation*}
The function $u(x) = \|x\|^\lambda$ is a solution of \eqref{problema} and its value at the boundary is exactly equal to $r^\lambda$. Similarly, the constant function $v(x) = r^\lambda$ also solves the problem \eqref{problema}, and the value at the boundary is the same.

We conclude that the problem associated to the operator $\mathcal{A}$ is \textbf{ill posed}. Moreover, it is not hard to check that the Cordes condition does not hold, coherently with the theory.
\end{example}

\paragraph{The $2$-dimensional case.} If $n = 2$, then the near operator theory allows us to prove the existence and the uniqueness of a solution of \eqref{problema}, provided that the coefficients are $L^\infty(\Omega)$ functions.

More precisely, the idea is to prove that the Campanato condition is, in fact, equivalent to the uniform ellipticity of the operator $\mathcal{A}$.

Suppose that $A(x)$ is uniformly elliptic and symmetric on $\Omega$. There are $\lambda_1(x), \, \lambda_2(x) \in \R$ eigenvalues such that $A(x)$ is similar to the diagonal matrix $\Gamma(x) := \mathrm{diag}\left(\lambda_1(x), \, \lambda_2(x) \right)$, and there is also a $\nu > 0$ such that $\lambda_1(x) \geq \nu$ and $\lambda_2(x) \geq \nu$ for almost every $x \in \Omega$.

It suffices to prove that there exists a function $a(x) \in L^\infty \left(\Omega \right)$ such that $a(x) \geq \omega > 0$ for almost every $x \in \Omega$ and, for any matrix $\xi := \{ \xi_{i, \, j} \}_{i, \, j = 1, \, 2}$, it turns out that
\begin{equation*} \begin{aligned} \left| \sum_{i = 1}^{2} \xi_{i, \, i} - a(x) \, \sum_{i, \, j = 1}^{2}  a_{i, \, j}(x) \, \xi_{i, \, j} \right| & = \left< \mathrm{I} - a(x) \, A(x), \, \xi \right> \leq \\ & \leq \left \| \mathrm{I} - a(x) \, A(x) \right\| \cdot \left\| \xi \right\| \leq \rho \, \|\xi\|, \end{aligned}\end{equation*}
for some $\rho \in (0, \, 1)$. Clearly
\begin{equation*} \left\| \mathrm{I} - a(x) \, A(x) \right\| \leq \rho \iff a^2(x) \, \left[ \lambda_1^2(x) + \lambda_2^2(x) \right] - 2 \, a(x) \, \left[ \lambda_1(x) + \lambda_2(x) \right] + 2 - \rho^2 \leq 0,\end{equation*}
and the right-hand side admits a real solution $a(x)$ if and only if the determinant is greater or equal than $0$, which  is possible if and only if
\begin{equation*} \frac{2 \, \lambda_1(x) \, \lambda_2(x)}{ \lambda_1^2(x) + \lambda_2^2(x) } \geq 1 - \rho^2. \end{equation*}
If we set $M = \max_{i = 1, \, 2} \, \sup_{x \in \Omega} \lambda_i(x)$, then we deduce that
\begin{equation*} \frac{2 \, \lambda_1(x) \, \lambda_2(x)}{ \lambda_1^2(x) + \lambda_2^2(x) } \geq \frac{\nu^2}{M^2}, \end{equation*}
therefore it is enough to choose $\rho$ and $a(x)$ in such a way that
\begin{equation*} \left[1 - \frac{\nu^2}{M^2} \right]^\frac{1}{2} \leq \rho < 1 \qquad \text{and} \qquad a(x) = \frac{ \lambda_1(x) + \lambda_2(x)}{ \lambda_1^2(x) + \lambda_2^2(x) }. \end{equation*}

\section{Regularity Conditions} 

In this section we want to prove that the problem \eqref{problema} is well-posed if we assume that the matrix $A(x) := \left\{ a_{i, \, j}(x) \right\}_{i, \, j = 1, \, \dots, \, n}$ is uniformly elliptic, and also that the coefficients $a_{i, \, j}(x)$ are $\alpha$-Hölder continuous, that is,
\begin{equation*} a_{i, \, j}(x) \in C^{0, \, \alpha}(\Omega). \end{equation*}

\begin{definition}[Sub(Super)-Solution] Let $u \in H^2(\Omega)$ be a function such that
\begin{equation} \label{subsup} \sum_{i, \, j= 1}^{n} a_{i, \, j}(x) \, D^{i, \, j} \, u(x) \geq 0 \qquad (\text{respectively, $\leq 0$}). \end{equation}
We say that $u$ is a \textit{sub-solution} (respectively, \textit{super-solution}) of the problem \eqref{problema} with no boundary condition.
\end{definition}

\begin{theorem}[Max. Principle] \label{th:mp} Let $u \in C^2 \left(\Omega \right) \cap C^0 \left( \overline{\Omega} \right)$ be a sub-solution of the problem \eqref{problema} with no boundary condition. If $A(x)$ is a uniformly elliptic matrix on $\Omega$, and the coefficients $a_{i, \, j}(x)$ are functions of class $C^0 \left( \overline{\Omega} \right)$, then
\begin{equation} \label{maxprc} \max_{x \in \overline{\Omega}} u(x) =  \max_{x \in \partial \, \Omega} u(x). \end{equation}
\end{theorem}

\begin{theorem}[Min. Principle] \label{th:minp} Let $u \in C^2 \left(\Omega \right) \cap C^0 \left( \overline{\Omega} \right)$ be a super-solution of the problem \eqref{problema} with no boundary condition. If $A(x)$ is a uniformly elliptic matrix on $\Omega$, and the coefficients $a_{i, \, j}(x)$ are functions of class $C^0 \left( \overline{\Omega} \right)$, then
\begin{equation} \label{minprc} \min_{x \in \overline{\Omega}} u(x) =  \min_{x \in \partial \, \Omega} u(x). \end{equation}
\end{theorem}

\begin{corollary}\label{ch:mp} Let $u$ be a solution of the problem
\begin{equation*} \sum_{i, \, j= 1}^{n} a_{i, \, j}(x) \, D^{i, \, j} \, v(x) = 0, \end{equation*}
and suppose that the same assumptions of \hyperref[th:mp]{Theorem \ref{th:mp}} are met. Then
\begin{equation*} \max_{x \in \overline{\Omega}} u(x) =  \max_{x \in \partial \, \Omega} u(x) \quad \text{and} \quad \min_{x \in \overline{\Omega}} u(x) =  \min_{x \in \partial \, \Omega} u(x). \end{equation*} \end{corollary}

\begin{corollary} \label{ch:uniqueness} If the same assumptions of \hyperref[th:mp]{Theorem \ref{th:mp}} are met, then the Dirichlet problem 
\begin{equation} \label{problema1} \begin{cases} \displaystyle\sum_{i, \, j = 1}^n a_{i, \, j}(x) \, D^{i, \, j} \, u(x) = f(x) & \text{for $x \in \Omega$} \\ \\ u(x) = g(x) & \text{for $x \in \partial \, \Omega$}. \end{cases} \end{equation}
admits at most one solution.\end{corollary}

\begin{proof} Suppose that $u_1(x)$ and $u_2(x)$ are both solution of class $C^2 \left(\Omega \right) \cap C^0 \left( \overline{\Omega} \right)$ of the problem \eqref{problema1}. If we set
\begin{equation*} W(x) := u_1(x) - u_2(x), \end{equation*}
then it's straightforward to prove that $W(x)$ is a solution of the Dirichlet problem
\begin{equation} \label{problema2} \begin{cases} \displaystyle\sum_{i, \, j = 1}^n a_{i, \, j}(x) \, D^{i, \, j} \, u(x) = 0 & \text{for $x \in \Omega$} \\ \\ u(x) = 0 & \text{for $x \in \partial \, \Omega$}. \end{cases} \end{equation}
Finally, \hyperref[ch:uniqueness]{Corollary \ref{ch:uniqueness}} proves that $W(x) \equiv 0$ is the unique solution of \eqref{problema2}, hence $u_1(x) \equiv u_2(x)$. \end{proof}

\begin{proof}[Proof of Maximum Principle] We divide the proof into two steps since the first one can be easily proved, and the second one follows from a simple approximation argument.

\paragraph{Step 1.} Suppose that for any $x \in \Omega$ it turns out that
\begin{equation} \label{st11} \sum_{i, \, j= 1}^n a_{i, \, j}(x) \, D^{i, \, j} \, u(x) > 0. \end{equation}
We argue by contradiction. Suppose that there exists a maximal point $x_0 \in \accentset{\circ}\Omega$ for $u$; then for any vector $v \in \R^n$, it turns out that
\begin{equation} \label{st12} \left( H_u(x_0) \, v, \, v \right)_{\R^n} \leq 0, \end{equation}
where $H_u(x_0)$ is the Hessian matrix of $u$ computed at the point $x_0$. By assumption, the matrix $A(x_0)$ is uniformly elliptic, hence there exists a real number $\nu > 0$ such that
\begin{equation} \label{st13} \left( A(x_0) \, v, \, v \right)_{\R^n} \geq \nu \, \|v\|_{\R^n}^2, \qquad \forall \, v \in \R^n. \end{equation}
It remains to prove that these relations yield to a contradiction. Indeed, the condition \eqref{st11} is equivalent to requiring that
\begin{equation*} \left( A(x), \, H(x) \right) > 0 \end{equation*}
at any point $x \in \R^n$ and, in particular, at $x = x_0$. On the other hand, both $A(x_0)$ and $H(x_0)$ are symmetric, thus there are $U$ and $V$ unitary matrices such that
\begin{equation*} U \, A(x_0) \, U^\ast =: \Lambda_A \quad \text{and} \quad V \, H(x_0) \, V^\ast =: \Lambda_H \end{equation*} 
are diagonal. Therefore, if we set $Q := U^\ast \, V$, then it is simple to prove that
\begin{equation*} \left( A(x), \, H(x) \right) = \left(\Lambda_A \, Q, \, Q \, \Lambda_H \right),\end{equation*}
and also that the latter scalar product is less or equal than zero, as a consequence of the fact that $A(x_0)$ is defined positive \eqref{st12} and $H(x_0)$ is semi-definite negative \eqref{st13}.

\paragraph{Step 2.} Suppose now that for any $x \in \Omega$ it turns out that
\begin{equation} \label{st21} \sum_{i, \, j= 1}^n a_{i, \, j}(x) \, D^{i, \, j} \, u(x) \geq 0. \end{equation}
We can define a small perturbation of the function $u(x)$, that is,
\begin{equation*} u_\epsilon(x) := u(x) + \epsilon \, \|x\|^2, \end{equation*}
so that $u_\epsilon(x)$ satisfies the condition \eqref{st11} for any $\epsilon > 0$. The first step proves that
\begin{equation*} \max_{x \in \overline{\Omega}} u_\epsilon(x) =  \max_{x \in \partial \, \Omega} u_\epsilon(x), \end{equation*}
hence the general case follows easily by taking the limit $\epsilon \to 0^+$.
\end{proof}

\begin{theorem}[Aleksandrov-Bakel'man-Pucci] Let
\begin{equation*}A(x, \, D) \, u := \sum_{i, \, j=1}^{n} a_{i, \, j}(x) \, D^{i, \, j} \, u(x) + \sum_{i = 1}^{n} b_i(x) \, D^i \, u(x) + c(x) \, u(x) \end{equation*}
be a differential operator such that the matrix $A(x)$ is uniformly elliptic on $\Omega$, and the coefficients $a_{i, \, j}(x)$, $b_i(x)$ and $c(x) \leq 0 $ are all of class $L^\infty(\Omega)$.

\vspace{1.8mm}
Let $u \in C^0 \left(\overline{\Omega}\right) \cap W_{\mathrm{loc}}^{2, \, n}(\Omega)$ be a function such that $A(x, \, D) \, u \geq f(x)$ for almost every $x \in \Omega$, and let
\begin{equation*} D(x) := \mathrm{det}\left(A(x)\right) \quad \text{and} \quad D^\ast(x) := \left[ D(x) \right]^{\frac{1}{n}}. \end{equation*}
Moreover, suppose that
\begin{equation*} \frac{f(x)}{D^\ast(x)} \in L^n(\Omega) \quad \text{and} \quad \frac{b(x)}{D^\ast(x)} \in L^n(\Omega). \end{equation*}
Then it turns out that
\begin{equation} \label{maxprch} \sup_{x \in \Omega} u(x) =  \sup_{x \in \partial \, \Omega} u^+(x) + c \, \left\| \frac{f}{D^\ast} \right\|_{L^n(\Omega)}, \end{equation}
where $u^+(x) := \max \{u(x), \, 0\}$ and the constant $c$ depends only on the dimension $n$ and on the following quantity: 
\begin{equation*} \left\| \frac{b(x)}{D^\ast(x)} \right\|_{L^n(\Omega)}. \end{equation*}\end{theorem}

The second step needed to prove that \eqref{problema} is well-posed, is the following a priori estimate.

\begin{theorem} \label{th:apriori} Let $u(x) \in H^{2, \, 2}(\Omega) \cap H_0^{1, \, 2}(\Omega)$ be a solution of the Dirichlet problem \eqref{problema}. Assume that $\partial \, \Omega$ is a manifold of class $C^3$, $f$ and $a_{i, \, j}$ belong to $C^{0, \, \alpha} \left(\overline{\Omega}\right)$. Then $D^{i, \, j} \, u \in C^{0, \, \alpha} \left(\overline{\Omega}\right)$, and there exists a positive constant $c > 0$ such that
\begin{equation} \label{eq:apriori} \sum_{i, \, j = 1}^n \left\| D^{i, \, j} \, u \right\|_{C^{0, \, \alpha} \left(\overline{\Omega}\right)}^2 \leq c \cdot \left( \|f\|_{C^{0, \, \alpha} \left(\overline{\Omega}\right)}^2 + \sum_{i, \, j = 1}^{n} \left\| D^{i, \, j} \, u \right\|_{\infty, \, \Omega}^2 \right). \end{equation}\end{theorem}

\begin{theorem} \label{th:aprioricho} Let $\Omega \subset \R^n$ be an open bounded subset, whose boundary $\partial \, \Omega$ is of class $C^3$. If $f \in C^{0, \, \alpha} \left(\overline{\Omega}\right)$, then there exists a positive constant
\begin{equation*} C := C \left(\Omega, \, \nu, \, \|a_{i, \, j}\|_{C^{0, \, \alpha} \left(\overline{\Omega}\right)} \right) > 0 \end{equation*}
such that, if $u \in H^{2, \, 2}(\Omega)$ is a solution of the Dirichlet problem \eqref{problema}, then
\begin{equation} \label{eq:aprioricho} \sum_{i, \, j = 1}^{n} \left\| D^{i, \, j} \, u \right\|_{0, \, \Omega}^2 \leq C \cdot \left( \|f\|_{C^{0, \, \alpha} \left(\overline{\Omega}\right)}^2 \right). \end{equation} \end{theorem}

\begin{proof}We argue by contradiction. If \eqref{eq:aprioricho} doesn't hold, then there exist: \mbox{}
\begin{enumerate}[label=\textbf{(\arabic*)}]
\item A uniformly bounded (by a constant $M > 0$) sequence of coefficients $\left(a_{i, \, j}^{(k)}(x) \right)_{k \in \N} \subset C^{0, \, \alpha} \left(\overline{\Omega}\right)$ such that the matrix
\begin{equation*} A^{(k)}(x) := \left\{ a_{i, \, j}^{(k)}(x) \right\}_{i, \, j = 1, \, \dots, \, n}\end{equation*}
is uniformly elliptic for any $k \in \N$, with the same constant $\nu>0$.
\item A sequence of functions $ \left( f_k(x) \right)_{k \in \N} \subset C^{0, \, \alpha} \left(\overline{\Omega}\right)$ such that
\begin{equation*} \| f_k \|_{C^{0, \, \alpha} \left(\overline{\Omega}\right)} \xrightarrow{k \to + \infty} 0. \end{equation*}
\item A sequence of solutions of the Dirichlet problems
\begin{equation} \label{problemaseq} \begin{cases} \displaystyle\sum_{i, \, j = 1}^n a_{i, \, j}(x) \, D^{i, \, j} \, u_k(x) = f_k(x) & \text{for $x \in \Omega$} \\ \\ u_k(x) = 0 & \text{for $x \in \partial \, \Omega$} \end{cases} \end{equation}
such that $\|u_k\|_{H^{2, \, 2}(\Omega)} = 1$ for all $k \in \N$.
\end{enumerate}
It follows from the Ascoli-Arzelà Theorem that there exists a subsequence $(k_n)_{n \in \N} \subset (k)_{k \in \N}$ such that $a_{i, \, j}^{k_n}(x)$ converges uniformly to $a_{i, \, j}(x)$ in $\overline{\Omega}$. From the a priori estimate \eqref{eq:apriori}, it turns out that for $k \in \N$ big enough
\begin{equation*} \sum_{i, \, j = 1}^n \left\| D^{i, \, j} \, u_k \right\|_{C^{0, \, \alpha} \left(\overline{\Omega}\right)}^2 \leq c \cdot \left( \|f_k\|_{C^{0, \, \alpha} \left(\overline{\Omega}\right)}^2 + \sum_{i, \, j = 1}^{n} \left\| D^{i, \, j} \, u_k \right\|_{\infty, \, \Omega}^2 \right) \leq c_1, \end{equation*}
since the norm of $f_k$ converges to zero, and the $H^{2, \, 2}(\Omega)$-norm of $u_k$ is constantly equal to one.

The sequence $(u_k)_{k \in \N}$ is equibounded in $C^{2, \, \alpha} \left(\overline{\Omega}\right)$, hence we may always extract a subsequence uniformly converging to $u$ (by Ascoli-Arzelà), and this implies the strong convergence in $H^{2, \, 2}(\Omega)$. Taking the limit for $k \to + \infty$ of the Dirichlet problem \eqref{problemaseq}, it turns out that $u$ is a solution of the problem \eqref{problema2}, hence $u \equiv 0$ on $\Omega$, that is, a contradiction since the $H^{2, \, 2}(\Omega)$-norm of $u$ is equal to $1$ (as a consequence of the uniform convergence).
\end{proof}

\begin{corollary} \label{apriori:cho} Under the assumptions of \hyperref[th:aprioricho]{Theorem \ref{th:aprioricho}}, the solution $u$ of the Dirichlet problem \eqref{problema} satisfies the following estimate:
\begin{equation}\label{eq:apriori:cho} \|u\|_{C^{2, \, \alpha} \left(\overline{\Omega} \right)} \leq c \, \|f\|_{C^{0, \, \alpha} \left( \overline{\Omega} \right)}. \end{equation} \end{corollary}

\begin{theorem}\label{th:aprioricho2} Let $\Omega \subset \R^n$ be an open bounded subset, whose boundary $\partial \, \Omega$ is of class $C^3$. Suppose that the coefficients $a_{i, \, j}(x)$ are of class $C^{0, \, \alpha}\left(\overline{\Omega}\right)$, and suppose that the matrix $A(x)$ is uniformly elliptic on $\Omega$.

Then, for any $f \in C^{0, \, \alpha} \left(\overline{\Omega}\right)$, the Dirichlet problem \eqref{problema} admits one and only one solution $u \in C^{2, \, \alpha} \left(\overline{\Omega}\right)$ satisfying the estimate \eqref{eq:apriori:cho}. \end{theorem}

\begin{proof}In order to prove this theorem, we use the continuity method introduced in the previous section. More precisely, let us consider the family of operators
\begin{equation*} A_t \, u := (1 - t) \, \nu \, \Delta \, u + t \, \sum_{i, \, j = 1}^n a_{i, \, j} \, D^{i, \, j} \, u, \qquad t \in [0, \, 1], \end{equation*}
and we notice that the coefficients $a_{i, \, j}^{(t)}(x) := (1-t) \, \nu \, \delta_{i, \, j} + t \, a_{i, \, j}(x)$ verifies the uniform ellipticity property, that is, the matrices
\begin{equation*} A^{(t)}(x) := \left\{ (1-t) \, \nu \, \delta_{i, \, j} + t \, a_{i, \, j}(x) \right\}_{i, \, j=1}^n\end{equation*}
are uniformly elliptic on $\Omega$. If we set $f_t := A_t \, u$, then it follows from \hyperref[apriori:cho]{Corollary \ref{apriori:cho}} that for any $u \in C^{2, \, \alpha} \left(\overline{\Omega} \right)$
\begin{equation*} \|u\|_{C^{2, \, \alpha} \left(\overline{\Omega} \right)}  \leq c \cdot \left\| A_t \, u \right\|_{C^{0, \, \alpha} \left(\overline{\Omega} \right)}\end{equation*}
since we can apply it to the Dirichlet problem
\begin{equation*} \begin{cases} \displaystyle\sum_{i, \, j = 1}^n (1-t) \, \nu \, \Delta \, u(x) + t \, \displaystyle\sum_{i, \, j = 1}^n a_{i, \, j}(x) \, D^{i, \, j} \, u(x) = f_t(x) & \text{for $x \in \Omega$} \\ \\ u(x) = 0 & \text{for $x \in \partial \, \Omega$}. \end{cases} \end{equation*}
It remains to check if the assumption of \hyperref[th:conm]{Theorem \ref{th:conm}} are met, where $\mfX = C^{2, \, \alpha} \left( \overline{\Omega} \right)$ and $\mathcal{B} = C^{0, \, \alpha} \left( \overline{\Omega} \right)$.

The first assumption holds true for $t = 0$ since the operator $\Delta$ is an isomorphism between $\mfX$ and $\mathcal{B}$; to check the second assumption, it suffices to observe that
\begin{equation*} \begin{aligned}\left\|A_t \, u - A_s \, u \right\|_{C^{0, \, \alpha} \left( \overline{\Omega} \right)} & = |t - s| \, \left\| \nu \, \Delta u - \sum_{i, \, j = 1}^n a_{i, \, j} \, D^{i, \, j} \, u \right\|_{C^{0, \, \alpha} \left( \overline{\Omega} \right)} \leq \\ & \leq c \, |t - s| \, \|u\|_{C^{2, \, \alpha} \left( \overline{\Omega} \right)}\,  {\color{red}\leq} \\ & {\color{red}\leq} \, c_1 \, c \, |t - s| \, \|A_t \, u\|_{C^{0, \, \alpha} \left( \overline{\Omega} \right)},\end{aligned} \end{equation*}
where the inequality ${\color{red}\leq}$ follows from \hyperref[apriori:cho]{Corollary \ref{apriori:cho}}.
\end{proof}

\section{Lax-Milgram Theorem}

\begin{theorem}[Lax-Milgram]\label{lax-milgram} Let $H$ be a Hilbert space, and let $a : H \times H \to \R$ be a function satisfying the following properties: \mbox{}
\begin{enumerate}[label=\textbf{(\arabic*)}]
\item $a(0, \, v) = 0$ for any $v \in H$.
\item $v \longmapsto a(u, \, v)$ is linear for any $u \in H$.
\item For any $u_1, \, u_2, \, v \in H$ it turns out that
\begin{equation*} \left| a(u_1, \, v) - a(u_2, \, v) \right| \leq M \, \|u_1 - u_2\|_H \, \|v\|_H. \end{equation*}
\item There exists $\nu > 0$ such that, for any $u_1, \, u_2 \in H$, the following inequality holds true:
\begin{equation*} a(u_1, \, u_1 - u_2) - a(u_2, \, u_1 - u_2) \geq \nu \, \|u_1 - u_2 \|_H^2. \end{equation*}
\end{enumerate}
Then for any $F \in H^\ast$ there exists one and only one $u \in H$ such that
\begin{equation*} a(u, \, v) = F(v), \qquad \forall \, v \in H. \end{equation*}
Moreover, the following estimate holds true:
\begin{equation*} c(\nu) \, \|u\|_H \leq \|F\|_{H^\ast}. \end{equation*}
\end{theorem}

\begin{proof} Let us consider the application
\begin{equation*} \mathcal{A} : H \to H^\ast, \qquad u \longmapsto \varphi_u \: : \: \varphi_u(v) := a(u, \, v). \end{equation*}
We want to prove that $\mathcal{A}$ is a bijection between $H$ and $H^\ast$, that is, for every $F \in H^\ast$ there is a unique $u \in H$ such that
\begin{equation*} \mathcal{A}(u)(v) = F(v) \qquad \forall \, v \in H \iff a(u,\, v) = F(v) \qquad \forall \, v \in H. \end{equation*}
By \hyperref[bijectle]{Theorem \ref{bijectle}} it suffices to prove that $A$ is near $\mathcal{J} : H \to H^\ast$ which is defined by
\begin{equation*} u \longmapsto \mathcal{J}_u \: : \: \mathcal{J}_u(v) := (u, \, v)_H, \end{equation*}
where $(\cdot, \, \cdot)_H$ is the scalar product. 

\paragraph{Step 1.} The operator $\mathcal{J}$ preserves the $H$-norm, that is, for any $u \in H$ it turns out that
\begin{equation*} \left\| \mathcal{J}(u) \right\|_{H^\ast} = \| u\|_H. \end{equation*}
The Riesz operator $\mathcal{R} : H^\ast \to H$ is defined by sending $F$ to the unique element $w_F \in H$ representing $F$, that is,
\begin{equation*} \mathcal{R}(F) = w_F \iff F(v) = \left(w, \, v \right)_H \qquad \forall \, v \in H. \end{equation*}
The Riesz operator $\mathcal{R}$ is an isometry (as it was proved in Riesz theorem), hence
\begin{equation*}\left(\mathcal{R} \left( \mathcal{A}(u) \right), \, v \right)_{H} = a(u, \, v) \qquad \text{and} \qquad \mathcal{R} = \mathcal{J}^{-1}, \end{equation*}
that is, $\mathcal{J}$ is an invertible operator.

\paragraph{Step 2.} The thesis of the theorem follows easily if we can prove that the nearness condition \eqref{eq:nearness} holds. By definition of the operators $\mathcal{A}$ and $\mathcal{J}$ it turns out that
\begin{equation*} \begin{aligned} \left\| \mathcal{J}(u_1) - \mathcal{J}(u_2) - \alpha \, \left[ \mathcal{A}(u_1) - \mathcal{A}(u_2) \right] \right\|_{H^\ast}^2 & = \left\| u_1 - u_2 - \alpha \left[ \mathcal{R} \left(\mathcal{A}(u_1) \right) - \mathcal{R} \left(\mathcal{A}(u_2) \right) \right] \right\|_{H}^2 = \\ & = \|u_1 - u_2\|_{H}^2 + \alpha^2 \, \left\|  \mathcal{R} \left(\mathcal{A}(u_1) \right) - \mathcal{R} \left(\mathcal{A}(u_2) \right)  \right\|_{H}^2 - \dots \\ & \dots - 2 \, \alpha \, \left(  \mathcal{R} \left(\mathcal{A}(u_1) \right) - \mathcal{R} \left(\mathcal{A}(u_2) \right), \, u_1 - u_2 \right)_H = \\ & = \|u_1 - u_2\|_{H}^2 + \alpha^2 \, \left\|  \mathcal{R} \left(\mathcal{A}(u_1) \right) - \mathcal{R} \left(\mathcal{A}(u_2) \right)  \right\|_{H}^2 - \dots \\ & \dots - 2 \, \alpha \, \left[ a(u_1, \, u_1 - u_2) - a(u_2, \, u_1 - u_2) \right] \, \, {\color{red} \leq}  \\ & {\color{red} \leq} \, \, \| u_1 - u_2 \|_H^2 + \alpha^2 \, M^2 \, \|u_1 - u_2\|_H^2 - 2 \, \alpha \, \nu \, \|u_1 - u_2\|_H^2 = \\ & = \left[1 + \alpha^2 \, M^2 - 2 \, \alpha \, \nu \right] \, \|u_1 - u_2\|_H^2 = \\ & = k \, \left\| \mathcal{J}(u_1) - \mathcal{J}(u_2) \right\|_{H^\ast}^2, \end{aligned}\end{equation*}
where the inequality ${\color{red} \leq}$ follows easily from properties \textbf{(2)} and \textbf{(3)} 
\end{proof}

\section{Garding Inequality}

Let us consider a differential elliptic operator in divergence form, that is,
\begin{equation*} A(x, \, D) \, u = \sum_{|\alpha| \leq m} \sum_{|\beta| \leq m} (-1)^{|\alpha|} \, D^\alpha \, \left(A_{\alpha, \, \beta}(x) \, D^\beta \, u \right). \end{equation*}
We denote by $A_0(x, \, D)$ the principal part of the operator, and we denote by $\left(\cdot, \, \cdot \right)_{\C^N}$ and $\| \cdot \|_{\C^N}$ respectively the scalar product and the norm in $\C^N$.

\begin{lemma} \label{lemma:gar1}  Let $A_0(D)$ be a differential operator in divergence form satisfying the weak Legendre-Hadamard condition \eqref{eq:l22}, and assume that the coefficients of $A$ are constants. Then for every $u \in H_0^m \left( \Omega; \; \R^N \right)$ it turns out that
\begin{equation} \label{eq:gar1} \int_{\Omega} \left[\sum_{|\alpha| = m} \sum_{|\beta| = m}  \left( A_{\alpha, \, \beta} D^\alpha \, u(x), \, D^\beta \, u(x) \right)  \right] \, \mathrm{d}x \geq c(\nu) \, \|u\|_{H^m \left(\Omega; \; \R^N \right)}. \end{equation}\end{lemma}

\begin{proof}We may always assume that $A_{\alpha, \, \beta} = A_{\alpha, \, \beta}^\ast$ since we can prove the thesis separately for the self-adjoint part and the anti self-adjoint part.

For any $u \in C_0^\infty(\R^n, \, \R^N)$ it turns out that
\begin{equation*} \begin{aligned}\int_{\R^n} \left[\sum_{|\alpha| = m} \sum_{|\beta| = m}  \left( A_{\alpha, \, \beta} D^\alpha \, u(x), \, D^\beta \, u(x) \right)  \right] \, \mathrm{d}x & =\sum_{|\alpha| = m} \sum_{|\beta| = m} \int_{\R^n}  \left( A_{\alpha, \, \beta} D^\alpha \, u(x), \, D^\beta \, u(x) \right)\, \mathrm{d}x  = \\ & = \int_{\R^n} \left[\sum_{|\alpha| = m} \sum_{|\beta| = m}  \left( A_{\alpha, \, \beta} \, \hat{u}(\xi), \, \overline{\hat{u}(\xi)} \right)_{\C^N}  \right] \, \xi^{\alpha + \beta} \, \mathrm{d}\xi \geq \\ & \geq \nu \, \int_{\R^n} \left( \|\mathfrak{Re}(\hat{u})(xi)\|^2 + \|\mathfrak{Im}(\hat{u})(xi)\|^2 \right) \, \|\xi\|^{2m} \, \mathrm{d}\xi = \\ & = \nu \, \int_{\R^n} \left(\hat{u}(\xi), \, \overline{\hat{u}(\xi)} \right)_{\C^N} \, \|\xi\|^{2m} \, \mathrm{d}\xi \geq \\ & \geq c(\nu) \, \int_{\R^n} \left[ \left( \hat{u}(\xi), \, \overline{\hat{u}(\xi)} \right)_{\C^N} \, \sum_{|\alpha| = m} \xi^{2 \alpha} \right] \, \mathrm{d}\xi = \\ & = c(\nu) \, \|u\|_{H^m(\R^n, \, R^N)}^2, \end{aligned} \end{equation*}
and this concludes the proof by density of the inclusion $C_0^\infty(\Omega, \, \R^N) \subset H_0^1(\Omega, \, \R^N)$.  \end{proof}

\begin{lemma} \label{lemma:gar2}  Let $A_0(D)$ be a differential operator in divergence form satisfying the weak Legendre-Hadamard condition \eqref{eq:l22}, and assume that the coefficients of $A$ are continuous on $\overline{\Omega}$. Then for every $u \in H_0^m \left( B(x_0, \, r); \; \R^N \right)$, with $x_0 \in \Omega$ and $r > 0$ small enough, it turns out that
\begin{equation} \label{eq:gar2} \int_{B(x_0, \, r)} \left[\sum_{|\alpha| = m} \sum_{|\beta| = m}  \left( A_{\alpha, \, \beta} D^\alpha \, u(x), \, D^\beta \, u(x) \right)  \right] \, \mathrm{d}x \geq \left[c(\nu) - \omega(r) \right] \, \|u\|_{H^m \left(\Omega; \; \R^N \right)}, \end{equation}
where $\omega(r)$ is the modulus of continuity defined by
\begin{equation*} \omega(r) := \sup \left\{ \left\|A_{\alpha, \, \beta}(x) - A_{\alpha, \, \beta}(y) \right\| \: \left| \: x, \, y \in \overline{\Omega}, \, \, \|x-y\| \leq r, \, \, |\alpha| = |\beta| = m \right. \right\}. \end{equation*}\end{lemma}

\begin{proof}Since
\begin{equation*} \begin{gathered}  \left| \int_{B(x_0, \, r)} \left[\sum_{|\alpha| = m} \sum_{|\beta| = m}  \left( \left[A_{\alpha, \, \beta}(x) - A_{\alpha, \, \beta}(x_0) \right] \, D^\alpha \, u(x), \, D^\beta \, u(x) \right)  \right]  \, \mathrm{d}x \right| \geq \dots \\ \dots \geq \omega(r) \, \int_{B(x_0, \, r)} \sum_{|\alpha = m} \|D^\alpha \, u(x) \|^2 \, \mathrm{d}x, \end{gathered} \end{equation*}
a simple application of \hyperref[lemma:gar1]{Lemma \ref{lemma:gar1}} gives us the estimate \eqref{eq:gar2}. \end{proof}

\begin{lemma} \label{lemma:gar3}  Let $\Omega \subset \R^n$ be a subset with a boundary locally Lipschitz. Let $A_0(D)$ be a differential operator in divergence form satisfying the weak Legendre-Hadamard condition \eqref{eq:l22}, and assume that the coefficients of $A$ are continuous on $\overline{\Omega}$. Then for every $u \in H_0^m \left( \Omega; \; \R^N \right)$ it turns out that
\begin{equation} \label{eq:gar3} \int_{\Omega} \left[\sum_{|\alpha| = m} \sum_{|\beta| = m}  \left( A_{\alpha, \, \beta} D^\alpha \, u(x), \, D^\beta \, u(x) \right)  \right] \, \mathrm{d}x \geq c(\nu) \, \left[ \|u\|_{H^m \left(\Omega; \; \R^N \right)} - \|u\|_{L^2 \left(\Omega; \; \R^N \right)} \right]. \end{equation}
\end{lemma}

\begin{proof} We divide the argument into many steps, in order to ease the notation.

\paragraph{Step 1.} ... \end{proof} %PRIMA O POI

\begin{lemma}\label{lemma:gar4} Let $A(x, \, D)$ be a differential operator in divergence form satisfying the weak Legendre-Hadamard condition \eqref{eq:l22}, and assume that: \mbox{}
\begin{enumerate}[label=\textbf{(\arabic*)}]
\item The coefficients of order $m$ are continuous on $\overline{\Omega}$.
\item The coefficients of order $<m$ are essentially bounded.
\end{enumerate}
Then for every $u \in H_0^m \left( \Omega; \; \R^N \right)$ it turns out that
\begin{equation} \label{eq:gar4} \int_{\Omega} \left[\sum_{|\alpha|\leq m} \sum_{|\beta| \leq m}  \left( A_{\alpha, \, \beta} D^\alpha \, u(x), \, D^\beta \, u(x) \right)  \right] \, \mathrm{d}x \geq c(\nu) \, \left[ \|u\|_{H^m \left(\Omega; \; \R^N \right)} - \|u\|_{L^2 \left(\Omega; \; \R^N \right)} \right]. \end{equation}
\end{lemma}

\begin{proof} The estimate \eqref{eq:gar4} is a straightforward consequence of the lemmas we have proved so far. More precisely, the terms of maximal order can be estimated via \hyperref[lemma:gar3]{Lemma \ref{lemma:gar3}}, while the other terms can be easily estimated with the interpolation inequality \eqref{eq:interp4}.  \end{proof}

\begin{theorem}[Lions] Let $X \subset Y \subset Z$ be Banach spaces such that the inclusion
\begin{equation*} X \hookrightarrow Y \end{equation*}
is continuous and compact, while the inclusion
\begin{equation*} Y \subset Z \end{equation*}
is continuous. For any $\epsilon > 0$ there exists a constant $c(\epsilon)>0$ such that for every $u \in X$ it turns out that
\begin{equation} \label{eq:interp}\|u\|_Y \leq \epsilon \, \|u\|_X + c(\epsilon) \, \|u\|_z. \end{equation} \end{theorem}

\begin{proof}We argue by contradiction. If \eqref{eq:interp} is not satisfied, then there exists $\epsilon > 0$ such that for any divergent sequence of real numbers $(c_n)_{n \in \N}$ there is a sequence $(u_n)_{n \in \N} \subset X$ such that
\begin{equation} \label{eq:interpfals}\|u_n\|_Y \geq \epsilon \, \|u_n\|_X + c_n \, \|u_n\|_z. \end{equation}
Let $v_n := u_n / \|u_n\|_X$. Then \eqref{eq:interpfals} can be written as follows
\begin{equation} \label{eq:interpfals2}\|v_n\|_Y \geq \epsilon + c_n \, \|v_n\|_z. \end{equation}
The $X$-norm of $v_n$ is equal to $1$ for each $n \in \N$, hence there exists a subsequence $\left(v_{n_k}\right)_{k \in \N} \subset Y$ (by compactness of the immersion) such that
\begin{equation*} v_{n_k} \xrightarrow{k \to + \infty} v_\infty \qquad \text{strongly in $Y$}. \end{equation*}
On the one hand, it follows from the continuity of the inclusion and \eqref{eq:interpfals2} that
\begin{equation*} \frac{c}{c_n} = \frac{c \, \|v_n\|_X}{c_n} \geq \frac{\|v_n\|_Y}{c_n} \geq \| v_n \|_Z, \end{equation*}
and by taking the limit as $n \to + \infty$ we infer that $\|v_n\|_{Z} \to 0$. This is absurd since by continuity of the inclusion $Y \subset Z$ it turns out that $v_\infty \equiv 0$, but \eqref{eq:interpfals2} implies also that $\|v_{n_k}\|_Y \geq \epsilon$ (in contradiction with the fact that $v_\infty$ is zero). \end{proof}

\begin{corollary}For any $\epsilon > 0$ and for any $u \in H_0^m \left( \Omega; \; \R^N \right)$ it turns out that
\begin{equation} \label{eq:interp4}\|u\|_{H^{m-1}\left( \Omega; \; \R^N \right)} \leq \epsilon \, \|u\|_{H^m\left( \Omega; \; \R^N \right)} + c(\epsilon) \, \|u\|_{L^2\left( \Omega; \; \R^N \right)}. \end{equation} \end{corollary}

\begin{proof}It is a straightforward consequence of the interpolation inequality \eqref{eq:interp} since the immersion
\begin{equation*} H^{m}\left( \Omega; \; \R^N \right) \hookrightarrow H^{m-1}\left( \Omega; \; \R^N \right) \end{equation*}
is compact by Rellich Theorem\footnote{Let $\Omega \subseteq \R^n$ be an open, bounded Lipschitz domain, and let $1 \leq p < n$. Set
\begin{equation*} p^\ast := \frac{np}{n - p}. \end{equation*}
Then the Sobolev space $W^{1, \, p} \left(\Omega; \; \R \right)$ is continuously embedded in the $L^p$-space $L^{p^\ast} \left(\Omega; \; \R \right)$ and is compactly embedded in $L^q\left(\Omega; \; \R \right)$ for every $1 \leq q < p^\ast$. In symbols,
\begin{equation*} W^{1, \, p}(\Omega) \hookrightarrow L^{p^\ast}(\Omega),\end{equation*}
and
\begin{equation*} W^{1, \, p}(\Omega) \subset \subset L^q(\Omega), \qquad \forall \, 1 \leq q < p^\ast. \end{equation*}}.\end{proof}

\section{Dirichlet problem for linear systems}

In this section, the primary goal is proving that the Dirichlet problem is well-posed in the linear systems setting.

\paragraph{Sobolev Spaces.} For any real number $s > 0$ the $-s$-Sobolev space is defined by setting
\begin{equation*} H^{-s}(\Omega) := \left( H_0^{s} (\Omega) \right)^\ast. \end{equation*}
The space of compactly supported infinitely derivable functions $\mathcal{D}(\Omega)$ is dense in $H_0^s(\Omega)$, hence
\begin{equation*} \mathcal{D}^\prime(\Omega) \supset H^{-s}(\Omega). \end{equation*}

\begin{theorem}\label{sob:char} Let $m > 0$ be an integer positive number. Every functional $f \in H^{-m}(\Omega)$ admits a non-unique representation as follows:
\begin{equation} \label{eq:repsob} f = \sum_{|\alpha| \leq m} D^\alpha \, f_\alpha, \qquad f_\alpha \in L^2(\Omega). \end{equation} \end{theorem}

\begin{proof}Let us consider, for $|\alpha| \leq m$, the linear application
\begin{equation*} \Psi_\alpha : H^m(\Omega) \ni u \longmapsto D^\alpha \, u \in L^2(\Omega). \end{equation*}
Clearly $\left(\Psi_\alpha \right)_{|\alpha| \leq m}$ establish an isomorphism between $H^m(\Omega)$ and a linear submanifold $V \subset \left[L^2(\Omega) \right]^h$, where $h$ is the number of all the derivatives $D^\alpha$ with $|\alpha| \leq m$.

Therefore to a linear continuous functional defined on $H^m(\Omega)$, corresponds a linear continuous functional defined on $V$. By Hahn-Banach theorem $L$ may be isometrically extended to a linear continuous functional $\tilde{L}$ defined over $\left[L^2(\Omega) \right]^h$. Moreover, it is easy to check that
\begin{equation*}\tilde{L} : \left[L^2(\Omega) \right]^h \to \R \implies \tilde{L} = \sum_{i = 1}^h L_i, \end{equation*}
where $L_i \in \left(L^2(\Omega) \right)^\ast$ for every index $i = 1, \, \dots, \, h$. By Riesz theorem, it turns out that
\begin{equation*} \tilde{L}(u) = \sum_{|\alpha| \leq m} \int_{\Omega} g_\alpha \, D^\alpha \, u, \qquad g_\alpha \in L^2(\Omega). \end{equation*}
In conclusion, since $L \in \left(H_0(\Omega) \right)^\ast$, the formula above is uniquely represented by the functional
\begin{equation*} L(\varphi) := \left< f, \, \varphi \right> \end{equation*}
defined on $\mathcal{D}(\Omega)$, and hence
\begin{equation*} f = \sum_{|\alpha| \leq m} D^\alpha \, f_\alpha, \qquad \text{where} \quad f_\alpha = (-1)^{|\alpha|} \, g_\alpha. \end{equation*} \end{proof}

\begin{theorem}[Global Existence, I] Let $A_0(D)$ be a differential operator in divergence form satisfying the weak Legendre-Hadamard condition \eqref{eq:l22}, and assume that the coefficients of $A$ are constants. Then for every $F \in H^{-m}(\Omega; \; \R^N)$ there exists one and only one solution in $H_0^m(\Omega; \; \R^N)$ of the system
\begin{equation*}A_0(D) \, u = F \end{equation*}
and the following estimate holds true:
\begin{equation} \label{eq:estisb} \|u\|_{H_0^m\left( \Omega; \; \R^N \right)} \leq c(\nu) \, \|F\|_{H^{-m}(\Omega; \; \R^N)}. \end{equation}\end{theorem}

\begin{remark}If we set
\begin{equation} \label{eq:sobn1} \|F\|_{-m, \, \Omega}^\ast := \inf \left\{ \sum_{j = 0}^m d_{\Omega}^{m-j} \, \left[ \int_\Omega \left( \sum_{|\alpha|=j} \|f_\alpha\|^2 \right) \, \mathrm{d}x \right]^{\frac{1}{2}} \right\}, \end{equation}
where the infimum is taken over the possible representations of $F$ of the form \eqref{eq:repsob}. One can easily show that \eqref{eq:sobn1} is a norm on $H^{-m}(\Omega)$, which is equivalent to the usual one:
\begin{equation} \label{eq:sobn2} \|F\|_{-m, \, \Omega} = \sup \left\{ \left| \left< F, \, \varphi \right> \right| \: \left| \: \varphi \in H_0^m(\Omega; \; \R^N), \, \, \|\varphi\|_{H_0^m(\Omega; \; \R^N)} = 1 \right. \right\}. \end{equation}\end{remark}

\begin{proof}Let us consider the bilinear form $a : H_0^m \left( \Omega; \; \R^N \right) \times H_0^m \left( \Omega; \; \R^N \right) \to \R$ defined by setting
\begin{equation*} a(u, \, v) := \int_{\Omega} \left[ \sum_{|\alpha| = |\beta| = m|} \left(A_{\alpha, \, \beta}(x) \, D^\alpha\, u(x), \, D^\beta \, v(x) \right) \right] \, \mathrm{d}x. \end{equation*}
The functional
\begin{equation*} L(v) := F(v), \qquad v \in H_0^m \left(\Omega; \; \R^N \right) \end{equation*}
is continuous, while $a$ is coercive on the product as a consequence of \hyperref[lemma:gar1]{Lemma \ref{lemma:gar1}}. By \hyperref[lax-milgram]{Lax-Milgram Theorem \ref{lax-milgram}} it turns out that there exists one and only one solution of the system
\begin{equation*} \int_{\Omega} \left[ \sum_{|\alpha| = |\beta| = m|} \left(A_{\alpha, \, \beta}(x) \, D^\alpha\, u(x), \, D^\beta \, v(x) \right) \right] \, \mathrm{d}x = F(v), \qquad \forall \, v \in H_0^m\left(\Omega; \; \R^N \right) \end{equation*}
along with the estimate \eqref{eq:estisb}. \end{proof}

\begin{theorem}[Global Existence, II] Let $A(D)$ be a differential operator in divergence form satisfying the weak Legendre-Hadamard condition \eqref{eq:l22}, and assume that: \mbox{}
\begin{enumerate}[label=\textbf{(\arabic*)}]
\item The coefficients of order $m$ are continuous on $\overline{\Omega}$.
\item The coefficients of order $<m$ are essentially bounded.
\end{enumerate}
Let $\gamma > 0$ be a big enough real number. Then for every $F \in H^{-m}(\Omega; \; \R^N)$ there exists one and only one solution in $H_0^m(\Omega; \; \R^N)$ of the system
\begin{equation*}A(x, \, D) \, u + \gamma \, u = F, \end{equation*}
and the following estimate holds true:
\begin{equation} \label{eq:estisb2} \|u\|_{H_0^m\left( \Omega; \; \R^N \right)} \leq c(\nu) \, \|F\|_{H^{-m}(\Omega; \; \R^N)}. \end{equation}
The same assertion holds true for any $\gamma > 0$, provided that the diameter $d_\Omega$ of the subset $\Omega$ is small enough.\end{theorem}

\begin{proof}Let us consider the bilinear form $a : H_0^m \left( \Omega; \; \R^N \right) \times H_0^m \left( \Omega; \; \R^N \right) \to \R$ defined by setting
\begin{equation*} a(u, \, v) := \int_{\Omega} \left[ \sum_{|\alpha| = |\beta| = m|} \left(A_{\alpha, \, \beta}(x) \, D^\alpha\, u(x), \, D^\beta \, v(x) \right) + \gamma \, u(x) \right] \, \mathrm{d}x. \end{equation*}
The functional
\begin{equation*} L(v) := F(v), \qquad v \in H_0^m \left(\Omega; \; \R^N \right) \end{equation*}
is continuous, hence the conclusion follows from \hyperref[lemma:gar4]{Lemma \ref{lemma:gar4}} for any $\gamma \geq c(\nu)$. Indeed, it suffices to observe that
\begin{equation*} \begin{aligned} a(u, \, v) & \geq c(\nu) \, \left[ \|u\|_{H^m(\Omega; \; \R^N)}^2 - \|u\|_{L^2(\Omega; \; \R^N)}^2 \right] + \gamma \, \|u\|_{L^2(\Omega; \; \R^N)}^{2} = \\ & = c(\nu) \|u\|_{H^m(\Omega; \; \R^N)}^2 + \left[\gamma - c(\nu) \right] \, \|u\|_{L^2(\Omega; \; \R^N)}^2 \geq \\ & \geq c(\nu) \, \|u\|_{H^m(\Omega; \; \R^N)}^2. \end{aligned} \end{equation*}
On the other hand, if $\gamma$ is any positive real number, it turns out that
\begin{equation*} \begin{aligned} a(u, \, v) & \geq c(\nu) \, \left[ \|u\|_{H^m(\Omega; \; \R^N)}^2 - \|u\|_{L^2(\Omega; \; \R^N)}^2 \right] + \gamma \, \|u\|_{L^2(\Omega; \; \R^N)}^{2} = \\ & = c(\nu) \|u\|_{H^m(\Omega; \; \R^N)}^2 + \left[\gamma - c(\nu) \right] \, \|u\|_{L^2(\Omega; \; \R^N)}^2 \geq \\ & \geq \left[ c(\nu) - d_\Omega \, \left(c(\nu) - \gamma \right) \right] \, \|u\|_{H^m \left( \Omega; \; \R^N \right)}^2 \, {\color{red} \geq} \\ & \, {\color{red} \geq} \,  c(\nu, \, \gamma, \, d_\Omega) \, \|u\|_{H^m \left( \Omega; \; \R^N \right)}^2, \end{aligned} \end{equation*}
where the {\color{red}red} inequality follows from the Poincaré inequality when $\gamma \leq c(\nu)$ and $d_\Omega$ is small enough.\end{proof}

\section{Global Existence via Special Operators}

Let us consider the Dirichlet problem
\begin{equation} \label{fpr} \begin{cases} u \in H_0^m\left(\Omega, \, \R^N \right) \\ \\ A(x, \, D) \, u(x) = F(x). \end{cases} \end{equation}
In the previous section, we have proved that for any $F \in H^{-m}\left(\Omega, \, \R^N \right)$ the problem \eqref{fpr} admits one and only one solution (i.e., it is well-posed) if the diameter of $\Omega$ is small enough. 

In this section, we show, via apriori estimates, that the same result holds true even when the diameter of $\Omega$ is small enough for the Poincaré inequality to hold. Let us set
\begin{equation*} \mathcal{P} \, u(x) = A(x, \, D) \, u(x). \end{equation*}

\begin{theorem}\label{th:dksdkksd}Let $\Omega \subset \R^n$ be an open bounded subset, and assume that $\partial \, \Omega$ is a boundary of class $C^m$. Suppose that $A_0(x, \, D)$ is an elliptic operator satisfying the Legendre-Hadamard condition \eqref{eq:l22}, with coefficients of class $C^1\left(\overline{\Omega} \right)$. Then the linear application
\begin{equation*} \mathcal{P} : H^{m+1}(\Omega, \, \R^N) \cap H_0^1(\Omega, \, \R^N) \to H^{1 - m} \left(\Omega, \, \R^N \right) \end{equation*}
has finite-dimensional kernel and closed rank. \end{theorem}

\begin{lemma}[Peetre] \label{lemma:peetre}Let $E, \, \Phi, \, G$ be three Banach spaces such that $E \subset \Phi$ is a compact immersion, and let $L$ be a continuous linear operator from $E$ to $G$. The following properties are equivalent: \mbox{}
\begin{enumerate}[label=\textbf{(\alph*)}]
\item The rank of $C$ is closed, and the kernel of $C$ is finite-dimensional.
\item There exists a positive constant $c > 0$ such that
\begin{equation} \label{eq:peetr} \|u\|_E \leq c \left( \|C \, u \|_G + \|u\|_{\Phi} \right), \qquad \forall \, u \in E. \end{equation}
\end{enumerate} \end{lemma}

\begin{proof}We divide the proof into two steps: second assertion implies the first assertion, and vice versa.

\paragraph{Step 1.} Let $E_0 := \mathrm{Ker}(C)$. The unitary ball in $E_0$ is compact in $\Phi$, thus, by \eqref{eq:peetr}, it is also compact in $E$. We conclude that $E_0$ is a finite-dimensional subspace of $E$\footnote{A Banach space such that every bounded subset is relatively sequentially compact, is necessarily a finite-dimensional space.}.

There exists $E_1 \subset E$ such that $E = E_0 \oplus E_1$, and the restriction of the operator $C$ on $E_1$ is injective by construction. We claim now that for any $u \in E_1$ it turns out that
\begin{equation} \label{eq:peetr1} \|u\|_E \leq c^\prime \, \|C \, u \|_G, \qquad \forall \, u \in E_1. \end{equation}
We argue by contradiction. If \eqref{eq:peetr1} does not hold true, then there exist a sequence $c_n^\prime \nearrow + \infty$ and a sequence $(u_n)_{n \in \N} \subset E_1$ such that
\begin{equation*} \|u_n\|_E > c_n^\prime \, \|C \, u_n \|_G. \end{equation*}
In particular, if we set $v_n := u_n / \|u_n\|_E$, the above inequality is equivalent to
\begin{equation} \label{eq:peetr2} \frac{1}{c_n^\prime} > \|C \, v_n \|_G. \end{equation}
The sequence $(v_n)_{n \in \N} \subset E$ is bounded, hence there exists a converging (in $\Phi$) subsequence $(v_{n_k})_{k \in \N}$ such that
\begin{equation*} v_{n_k} \xrightarrow{k \to + \infty}_{\Phi} v \in \Phi. \end{equation*}
It follows from \eqref{eq:peetr} and the above inequality that $(v_{n})_{n \in \N}$ is a Cauchy sequence in $E_1$, and thus it converges necessarily to the same $v$. Passing to the limit as $n \to + \infty$ in \eqref{eq:peetr2} immediately implies $v \equiv 0$, but this is absurd since by \eqref{eq:peetr} it turns out that
\begin{equation}1 \leq c \left( \|C \, v_n \|_G + \|v_n\|_{\Phi} \right), \qquad \forall \, n \in \N. \end{equation}
The claim \eqref{eq:peetr1} is now proved.

Let $(w_n)_{n \in \N} \subset \mathrm{Ran}(C)$ be a converging sequence, and let $w \in G$ be its limit. There exists a sequence $(u_n)_{n \in \N} \subset E$ such that $C \, u_n = w_n$ for every $n \in \N$; using the decomposition of $E$, it turns out that
\begin{equation*} u_n = v_n + z_n \implies C(u_n) = C(z_n) = w_n, \end{equation*}
and thus it follows from \eqref{eq:peetr1} that
\begin{equation*} \|z_n\|_E \leq c^\prime \, \|C \, z_n \|_G = c^\prime \, \|w_n\|_G.\end{equation*}
Consequently $(z_n)_{n \in \N} \subset E_1$ is a Cauchy sequence in $E$, hence it converges to $z \in E$. By the continuity of the operator $C$, we conclude that $C \, z = w$ (which is exactly what we wanted to prove).

\paragraph{Step 2.} Let $E = E_0 \oplus E_1$ as above. The restriction $C \, \big|_{E_1}$ is a closed map, thus by the closed graph theorem it follows that
\begin{equation}\label{eq:1111} \|v\|_E \leq c_1 \, \| C \, v \|_G, \qquad \forall \, v \in E_1. \end{equation}
On the other hand, for any $w \in E_0$ one can prove the inequality
\begin{equation} \label{eq:1112} \|w\|_E \leq c_2 \, \| w \|_\Phi. \end{equation}
We argue by contradiction. If \eqref{eq:1112} does not hold true, then there exist a sequence $(d_n) \subset \R$ increasingly converging to $+ \infty$, and a sequence $(w_n)_{n \in \N} \subset E_0$ such that
\begin{equation*} \|w_n\|_E \geq d_n \, \|w_n\|_{\Phi}. \end{equation*}
If we set $y_n := w_n / \|w_n\|_E$, then it turns out that
\begin{equation*} \frac{1}{d_n} \geq \|y_n\|_{\Phi}, \end{equation*}
hence $y_n \to 0$ in $\Phi$. This is absurd since the sequence $(y_n)_{n \in \N}$ belongs to a finite-dimensional subspace, hence it admits a converging subsequence to an  element $y$ of norm $1$. In conclusion, the inequality \eqref{eq:peetr} follows immediately from \eqref{eq:1111} and \eqref{eq:1112}.
\end{proof}

\begin{proof}[Proof of Theorem \ref{th:dksdkksd}] The thesis is an immediate corollary of \hyperref[lemma:peetre]{Peetre's Lemma \ref{lemma:peetre}}, where
\begin{equation*} \begin{aligned} & E = H^{m+1}(\Omega, \, \R^N) \cap H_0^1(\Omega, \, \R^N) \\ & \Phi = H_0^m \left(\Omega, \, \R^N \right) \\ & G = H^{1-m} \left(\Omega, \, \R^N \right) \\ & C \, u = \mathcal{P} \, u, \end{aligned} \end{equation*}
by noticing that \mbox{}
\begin{enumerate}[label=\textbf{(\arabic*)}]
\item the immersion $H^{m+1}\left(\Omega, \, \R^N \right)$ into $H^m\left(\Omega, \, \R^N \right)$ is compact by Rellich Theorem\footnote{Let $\Omega \subseteq \R^n$ be an open, bounded Lipschitz domain, and let $1 \leq p < n$. Set
\begin{equation*} p^\ast := \frac{np}{n - p}. \end{equation*}
Then the Sobolev space $W^{1, \, p} \left(\Omega; \; \R \right)$ is continuously embedded in the $L^p$-space $L^{p^\ast} \left(\Omega; \; \R \right)$ and is compactly embedded in $L^q\left(\Omega; \; \R \right)$ for every $1 \leq q < p^\ast$. In symbols,
\begin{equation*} W^{1, \, p}(\Omega) \hookrightarrow L^{p^\ast}(\Omega),\end{equation*}
and
\begin{equation*} W^{1, \, p}(\Omega) \subset \subset L^q(\Omega), \qquad \forall \, 1 \leq q < p^\ast. \end{equation*}};
\item the following estimate holds true:
\begin{equation*} \|u\|_{H^{m+1}(\Omega, \, \R^N)} \leq c \, \left( \|u\|_{H^m\left(\Omega, \, \R^N \right)} + \|F\|_{H^{1-m}\left(\Omega, \, \R^N \right)} \right). \end{equation*}
\end{enumerate}
\end{proof}
\chapter{Sobolev Spaces} \thispagestyle{empty}

In this final chapter, we use all the theory we have developed so far to introduce and study the main properties of the Sobolev spaces $W^{m, \, p}(\Omega)$, whose importance is well-known in the partial differential equations field.

\section{Introduction and Elementary Properties}

\paragraph{Introduction.} In this paragraph, we denote by $I$ an open interval $(a, \, b)$ of the real line $\R$ (eventually unbounded), unless stated otherwise.

\begin{definition}[Weak Derivative] \index{weak derivative} Let $f \in \Ll(I)$ be a locally summable function. A function $g \in \Ll(I)$ is the \textit{weak derivative} of $f$ if and only if
\begin{equation} \label{wd} \int_I f(x) \varphi^\prime(x) \, \mathrm{d}x = - \int_I g(x) \varphi(x) \, \mathrm{d}x \quad \text{for every $\varphi \in \Cc(I)$}. \end{equation}  \end{definition}

\begin{proposition}\label{prop:312}Let $f \in \Ll(I)$. The following properties hold: \mbox{}
\begin{enumerate}[label=\textbf{(\alph*)}]
\item If $g \in \Ll(I)$ is the weak derivative of $f$, then it is unique up to the a.e. equivalence relation. More precisely, if $g_1 \in \Ll(I)$ and $g_2 \in \Ll(I)$ are both weak derivatives of $f$, then
\begin{equation*} g_1(x) = g_2(x) \quad \text{for almost every $x \in I$}. \end{equation*}
\item If $f \in C^1(I)$, then a representative of the weak derivative equivalence class coincides with the usual derivative $f^\prime$.
\item Let $f \in C^0(I)$ be a continuous function such that the usual derivative $f^\prime(x)$ exists at all $x \in I \setminus D$, where $D$ is at most countable. Then the weak derivative of $f$ exists, and a representative of the equivalence class is given by $f^\prime$.
\end{enumerate}\end{proposition}

\begin{proof}\mbox{}
\begin{enumerate}[label=\textbf{(\alph*)}]
\item Suppose that $g_1$ and $g_2$ are both weak derivatives of $f$. It follows from \eqref{wd} that
\begin{equation*} \int_I \left[g_1(x) - g_2(x) \right] \varphi(x) \, \mathrm{d}x = 0 \quad \text{for all $\varphi \in \Cc(I)$}, \end{equation*}
and hence $g_1(x) = g_2(x)$ for almost every $x \in I$ as a consequence of the fundamental lemma in calculus of variations\footnote{\textbf{Lemma.} Let $f \in \Ll(I)$. If
\begin{equation*} \int_{I} f(x) \varphi(x) \, \mathrm{d}x = 0 \quad \text{for all $\varphi \in \Cc(I)$} \implies \text{$f(x) = 0$ for almost every $x \in I$}. \end{equation*}}.
\item It suffices to integrate by parts
\begin{equation*} \int_I f(x) \varphi^\prime(x) \, \mathrm{d}x = 0, \end{equation*}
and apply the same argument of the previous point.
\end{enumerate}\end{proof}

\begin{remark}Let $f \in C^0(\Omega)$ be a continuous function, and suppose that $f^\prime(x)$ exists at almost every $x \in I$. Then
\begin{equation*} g(x) := f^\prime(x) \quad \text{a.e. $x \in I$} \notimplies \text{$g$ is the weak derivative of $f$}. \end{equation*} \end{remark}

\begin{proof}Recall that the Cantor function $f_C : [0, \, 1] \to \R$ is the unique continuous increasing function satisfying the following relations:
\begin{equation*} f_C(x) + f_C(1-x) = 1 \quad \text{and} \quad f_C \left( \frac{x}{3} \right) = \frac{1}{2} f_C(x), \end{equation*}
for every $x \in [0, \, 1]$. The Cantor function is locally constant on the complement of the Cantor set $C$ since
\begin{equation*} \text{$f$ constant on the interval $J$} \implies \text{$f$ constant on $\frac{1}{3} J$ and $1 - \frac{1}{3} J$}. \end{equation*}
Therefore $f_C$ is locally constant on a set of full measure (note that the Cantor set $C$ is uncountable and has measure zero).

\paragraph{Conclusion.} The usual derivative $f_C^\prime$ exists at almost all $x \in [0, \, 1]$, and it must be equal to $0$ a.e. because $f_C$ is locally constant almost everywhere. If the function identically equal to zero were the weak derivative of $f_C$, then a variation of the fundamental lemma in the calculus of variation\footnote{\textbf{Lemma.} (Paul du Bois-Reymond.) Let $f \in \Ll(I)$. If
\begin{equation*} \int_{I} f(x) \, \varphi(x) \, \mathrm{d}x = 0 \qquad \forall \, \varphi \in \Cc(I) \: : \: \int_{I} \varphi(x) \, \mathrm{d}x = 0, \end{equation*}
then $f(x) = c$ for almost every $x \in I$.} would imply $f_C$ constant, which gives the sought contradiction.\end{proof}

\begin{proof}[Alternative Proof] We argue by contradiction. If $f_C \in \Ll \left( [0, \, 1] \right)$ is a weakly differentiable function, then \hyperref[wabs]{Proposition \ref{wabs}} proves that the Cantor function $f_C$ admits an absolutely continuous representative, and thus it maps null-sets to null-sets.

But $f_C$ maps the Cantor set (a set of measure zero) to its complement (a set of strictly positive measure), and this is yields to a contradiction.\end{proof}

\begin{definition}[Absolutely Continuous] \index{absolutely continuous} A function $f : I \longrightarrow \R$ is \textit{absolutely continuous} if and only if for all $\epsilon > 0$ there exists $\delta(\epsilon) := \delta > 0$ such that, for any finite \textbf{disjoint} family $J_1, \, \dots, \, J_n \subset I$ of open intervals $(a_i, \, b_i)$ satisfying the property
\begin{equation*} \sum_{i = 1}^n |J_i| = \sum_{i = 1}^n |b_i - a_i| \leq \delta, \end{equation*}
it turns out that
\begin{equation*} \sum_{i=1}^n \left| f(b_i) - f(a_i)\right| \leq \epsilon. \end{equation*} \end{definition}

\begin{definition}[Oscillation] \index{oscillation} Let $f : I \longrightarrow \R$ be a continuous function, and let $J \subset I$ be a subset. The \textit{oscillation} of $f$ in $J$ is defined by setting
\begin{equation*}\mathrm{osc} \left( f, \, J \right) := \sup \left\{|f(x) - f(y)| \: : \: x, \, y \in J \right\}. \end{equation*} \end{definition}

\begin{remark} There are many equivalent definitions of absolute continuity of a function, whose proof is left to the reader. \mbox{}
\begin{enumerate}[label=\textbf{(\arabic*)}]
\item A function $f : I \longrightarrow \R$ is \textit{absolutely continuous} if and only if for all $\epsilon > 0$ there exists $\delta(\epsilon) := \delta > 0$ such that, for any finite \textbf{disjoint} family $J_1, \, \dots, \, J_n \subset I$ of open intervals $(a_i, \, b_i)$ satisfying the property
\begin{equation*} \sum_{i = 1}^n |J_i| = \sum_{i = 1}^n |b_i - a_i| \leq \delta, \end{equation*}
it turns out that
\begin{equation*} \sum_{i=1}^n \mathrm{osc} \left( f, \, J_i \right) \leq \epsilon. \end{equation*} 
\item A function $f : I \longrightarrow \R$ is \textit{absolutely continuous} if and only if for all $\epsilon > 0$ there exists $\delta(\epsilon) := \delta > 0$ such that, for any countable \textbf{disjoint} family $J_1, \, \dots, \, J_n, \, \dots \subset I$ of open intervals $(a_i, \, b_i)$ satisfying the property
\begin{equation*} \sum_{i = 1}^{+ \infty} |J_i| = \sum_{i = 1}^{+ \infty} |b_i - a_i| \leq \delta, \end{equation*}
it turns out that
\begin{equation*} \sum_{i=1}^{+\infty} \mathrm{osc} \left( f, \, J_i \right) \leq \epsilon. \end{equation*} 
\end{enumerate}\end{remark}

\begin{proposition}\label{prop:12od} A finite measure $\mu$ defined on the Borel $\sigma$-algebra $\mathcal{B}(\R)$ is absolutely continuous with respect to the Lebesgue measure if and only if the cumulative distribution function\index{cumulative distribution function}
\begin{equation*} F_\mu(x) = \mu \left( (- \infty, \, x] \right) \end{equation*}
is absolutely continuous. \end{proposition}

\begin{theorem}\label{th:sob2ed} A function $f \in \Ll(I)$ is (locally) absolutely continuous if and only if there exists $g \in \Ll(I)$ such that
\begin{equation} \label{eq:sob2ed} f(x) - f(y) = \int_{x}^{y} g(t) \, \mathrm{d}t. \end{equation}
More precisely, any absolutely continuous function is a.e. differentiable, and the usual derivative is a representative of the weak derivative equivalence class. \end{theorem}

\begin{proof} Let $J := [a, \, b] \subset I$ be a closed interval. The function $f$ is absolutely continuous in $J$, and hence we can define a measure $\mu$ by setting
\begin{equation*} \mu \left([x, \, y) \right) := f(y) - f(x). \end{equation*}
We can easily prove that the cumulative distribution function is absolutely continuous, and therefore $\mu$ is absolutely continuous with respect to the Lebesgue measure, as a consequence of \hyperref[prop:12od]{Proposition \ref{prop:12od}}. It follows from the \hyperref[th:rnnn]{Radon-Nikodym Theorem} that there exists $g \in L^1(J)$ such that
\begin{equation*} f(y) - f(x) = \int_{x}^{y} g(t) \, \mathrm{d}t \quad \text{for all $x, \, y \in J$}. \end{equation*}
Therefore, $f$ is differentiable at almost every $x \in J$, and we can easily prove that $g$ is a representative of the weak derivative equivalence class (using the integration by parts formula). \end{proof}

\begin{proposition} \label{wabs} Let $f \in \Ll(I)$ be a weakly differentiable function. There exists an absolutely continuous representative in the equivalence class of $f$. \end{proposition}

\begin{proof} Suppose that $f \in \Ll(I)$ is a weakly differentiable function, and let $g \in \Ll(I)$ be the weak derivative. It is easy to prove that
\begin{equation*} \eqref{wd} \iff \int_I f(x) \varphi^\prime(x) \, \mathrm{d}x = - \int_I g(x) \varphi(x) \, \mathrm{d}x \quad \text{for all $\varphi \in \mathrm{PW} C_c^1(I)$}, \end{equation*} 
where $\mathrm{PW} C_c^1(I)$ denotes the space of all piecewise continuously differentiable functions with compact support in $I$.

\paragraph{Conclusion.} Let $x < y \in I$, and let us consider the function
\begin{equation*} \Phi_{\eta}^{\epsilon}(t) := \begin{cases} 0 & t \leq x - \epsilon, \\ 1 & x + \epsilon < t < y - \eta, \\ 0 & t \geq y + \eta. \end{cases} \end{equation*}
Denote by $\Phi_{\eta}^{\epsilon}$ the obvious piecewise differentiable extension to the whole interval $I$. It is easy to compute the left-hand side of \eqref{wd} since
\begin{equation*} \int_I f(t)  \left( \Phi_{\eta}^{\epsilon} \right)^\prime(t) \, \mathrm{d}t = \frac{1}{2 \epsilon} \int_{ x-\epsilon}^{x + \epsilon} f(t) \, \mathrm{d}t - \frac{1}{2 \eta} \int_{y - \eta}^{y + \eta} f(t) \, \mathrm{d}t, \end{equation*}
and, similarly, the right-hand side is given by
\begin{equation*} \int_I g(t) \Phi_{\eta}^{\epsilon}(t) \, \mathrm{d}t = \int_{x}^{y} g(t) \, \mathrm{d}t + o(1) \quad \text{for $\epsilon, \, \eta \to 0^+$}. \end{equation*}
If we let
\begin{equation*} \chi_\epsilon(t) := \frac{1}{2 \epsilon}  \chi_{[- \epsilon, \, \epsilon]} \quad \text{and} \quad  \chi_\eta(t) := \frac{1}{2  \eta} \chi_{[- \eta, \, \eta]} \end{equation*}
then the left-hand side may be rewritten as follows:
\begin{equation*} \int_I f(t)  \left( \Phi_{\eta}^{\epsilon} \right)^\prime(t) \, \mathrm{d}t = f \ast \chi_\epsilon(x) - f \ast \chi_\eta(y). \end{equation*}
If we take the limit as $\epsilon \to 0^+$, and then we take the limit as $\eta \to 0^+$, it turns out that
\begin{equation*} f(x) - f(y) = \int_{x}^{y} g(t) \, \mathrm{d}t \implies f(y) - f(x) = \int_{x}^{y} (-g)(t) \, \mathrm{d}t, \end{equation*}
and \hyperref[th:sob2ed]{Theorem \ref{th:sob2ed}} concludes the proof.
\end{proof}

\section{$W^{1, \, p}$ Spaces}

\paragraph{Introduction.} In this section, we denote by $I$ an open interval $(a, \, b)$ of the real line $\R$ (eventually unbounded), unless stated otherwise.

\begin{definition}[Sobolev Space] \index{Sobolev space} Let $p \in [1, \, + \infty]$. The $(1, \, p)$-Sobolev space, denoted by $W^{1, \, p}(I)$, is the space of all the $L^p(I)$ functions with weak derivative in $L^p(I)$, that is,
\begin{equation*} W^{1, \, p}(I) := \left\{ f \in L^p(I) \: \left| \: D f \in L^p(I) \right. \right\}. \end{equation*} \end{definition}

\paragraph{Normed Space.} In this paragraph, we briefly discuss the main idea that allows one to define a norm on $W^{1, \, p}(I)$ that makes it a Banach space.

\begin{remark}The Sobolev space $W^{1, \, p}(I)$ is a vector space, and the derivative map
\begin{equation*} D : W^{1, \, p}(I) \longrightarrow L^p(I), \qquad f \longmapsto D f \end{equation*}
is well-defined and linear. \end{remark}

\begin{lemma} The graph of the operator $D : W^{1, \, p}(I) \longrightarrow L^p(I)$ is closed, that is,
\begin{equation*} \Gamma(D) := \left\{ \left(f, \, D f \right) \: \left| \: f \in W^{1, \, p}(I) \right. \right\} \subset L^p(I) \times L^p(I) \end{equation*}
is closed with respect to the subspace topology.
\end{lemma}

\begin{proof} Let $\left(f_n, \, D f_n \right)_{n \in \N} \subset \Gamma(D)$ be a converging sequence, and let $(f, \, g) \in L^p \times L^p$ be its limit in the product norm, that is,
\begin{equation*} f_n \xrightarrow{L^p} f \quad \text{and} \quad D f_n \xrightarrow{L^p} g. \end{equation*}
The identity
\begin{equation*} \int_I f_n(t) \varphi^\prime(t) \, \mathrm{d}t = - \int_I D f_n(t) \varphi(t) \, \mathrm{d}t \end{equation*}
holds for every $n \in \N$ and for every $\varphi \in C_c^\infty(I)$. Recall that the $L^p$ convergence implies the pointwise convergence of a subsequence, and therefore we can take the limit as $n$ goes to $+ \infty$ of the identity above to obtain
\begin{equation*} \int_I f(t) \varphi^\prime(t) \, \mathrm{d}t = - \int_I g(t) \varphi(t) \, \mathrm{d}t, \end{equation*}
which means that $g = Df$, i.e., the graph is closed. \end{proof}

\begin{corollary} The mapping
\begin{equation*} W^{1, \, p}(I) \longrightarrow L^p(I) \times L^p(I), \qquad f \longmapsto \left(f, \, D f  \right)\end{equation*}
is linear, injective and onto the rank. More precisely, it turns out that
\begin{equation*} W^{1, \, p}(I) \longrightarrow \Gamma(D) \end{equation*}
is linear and bijective. \end{corollary}

In conclusion, the product $L^p \times L^p$ induces on $\Gamma(D)$ the subspace topology by taking the restriction of the product norm $\| \cdot \|_p + \| - \|_p$. More precisely, if we endow $W^{1, \, p}$ with the topology generated by the norm
\begin{equation} \label{sobnorm} \|f\|_{W^{1, \, p}(I)} := \|f\|_{L^p(I)} + \| D f \|_{L^p(I)}, \end{equation}
then one can easily prove that $W^{1, \, p}$ is complete (and thus a Banach space).

\begin{remark} Clearly \eqref{sobnorm} may be replaced by any equivalent norm in the product $L^p \times L^p$. In particular, when $p = 2$, it is particularly useful to consider the equivalent norm
\begin{equation} \label{sobnorm2} \|f\|_{H^1(I)} := \left( \|f\|_{L^2(I)}^2 + \| D f \|_{L^p(I)}^2 \right)^{\frac{1}{2}} \end{equation}
since it makes $H^1(I) := W^{1, \, 2}(I)$ a Hilbert space, with scalar product given by the formula
\begin{equation} \label{sobsp2} (u, \, v)_{H^1(I)} = (u, \, v)_{L^2(I)} + (D u, \, D v)_{L^2(I)}. \end{equation} \end{remark}

\begin{remark} The method employed in this paragraph to define a complete norm can be easily extended to any linear subspace of a Banach space. Indeed, let
\begin{equation*} T : \mathfrak{Y} \longrightarrow \mathcal{B} \end{equation*}
be a linear operator between a linear space $\mathfrak{Y} \subseteq \mathcal{B}$ and a Banach space $\left(\mathcal{B}, \, \|\cdot\|_{\mathcal{B}} \right)$. If $T$ is a closed operator, then the operator
\begin{equation*} \mathfrak{Y} \longrightarrow \mathrm{Grap}(D) \subset \mathcal{B} \times \mathcal{B}, \qquad f \longmapsto (f, \, Tf) \end{equation*}
is linear and bijective. In particular, it turns out that $\mathfrak{Y}$ is a Banach space endowed with the norm
\begin{equation*} \|f\|_{\mathfrak{Y}} := \|f\|_{\mathcal{B}} + \| T \, f \|_{\mathcal{B}}. \end{equation*}\end{remark}

\begin{proposition} Let $p \in [1, \, + \infty]$. Then for every $f \in W^{1, \, p}(I)$ there is an absolutely continuous representative of the equivalence class. In other words, the inclusion
\begin{equation*} W^{1, \, p}(I) \subseteq C^0(I) \end{equation*}
is a linear and continuous, i.e., there exists a positive constant $C$ such that
\begin{equation*} \|f\|_{C^0(I)} \leq C \|f\|_{W^{1, \, p}(I)}. \end{equation*}  \end{proposition}

\begin{proof} This follows directly from \hyperref[wabs]{Proposition \ref{wabs}}. \end{proof}

\begin{proposition} \label{prop:alphsd} \mbox{}
\begin{enumerate}[label=\textbf{(\alph*)}]
\item A function $f : I \longrightarrow \R$ is Lipschitz if and only if it belongs to $W^{1, \, \infty}(I)$.
\item If $p \in (1, \, + \infty)$, then
\begin{equation*} f \in W^{1, \, p}(I) \implies f \in C^{0, \, \frac{1}{p^\prime}}(I), \end{equation*}
where $p^\prime$ is the conjugate of $p$. Vice versa, 
\begin{equation*} f \in C^{0, \, \alpha}(I) \notimplies f \in W^{1, \, p}(I). \end{equation*}
\item The Sobolev space $W^{1, \, p}(I)$ is isomorphic to a closed subset of $L^p(I) \times L^p(I)$. In particular, it is separable for $p \in [1, \, + \infty)$ and it is reflexive for $p \in (1, \, + \infty)$.
\end{enumerate} \end{proposition}

\begin{proof}\mbox{}
\begin{enumerate}[label=\textbf{(\alph*)}]
\item This assertion is left as a simple exercise. For the solution, the reader may refer to \href{https://math.stackexchange.com/questions/874982/relation-between-sobolev-space-w1-infty-and-the-lipschitz-class}{this post}.
\item Let $f \in W^{1, \, p}(I)$ be a $p$-summable function, for some $p \in (1, \, + \infty)$. By \hyperref[wabs]{Proposition \ref{wabs}} there is an absolutely continuous representative, which we still denote by $f$, such that
\begin{equation*}f(x) - f(y) = \int_{y}^{x} D f(t) \, \mathrm{d}t. \end{equation*}
If we take the absolute values and apply the Hölder inequality, we find that
\begin{equation*} \left| f(x) - f(y) \right| \leq \|D f\|_{L^p(I)} \cdot |x - y|^{\frac{1}{p^\prime}}, \end{equation*}
which means that $f$ is $1/p^\prime$-Hölder continuous.

Vice versa, the reader may check for herself that the Weierstrass function is Hölder continuous, but is not absolutely continuous, and is not of bounded variation either.
\end{enumerate} \end{proof}

\paragraph{Dual Space.} Let $p \in [1, \, + \infty)$. The dual space of $W^{1, \, p}(I)$ can be easily represented as
\begin{equation*} W^{1, \, p}(I) \ni u \longmapsto \int_I \left[ u(x) f(x) + \left(D u(x)\right) g(x) \right]\, \mathrm{d}x, \end{equation*}
as $f$ and $g$ range in $L^{p^\prime}$ (i.e., the dual space $L^p$ for $p \neq + \infty$). This is an easy consequence of the fact that the dual of a product is the product of the duals
\begin{equation*} \left( L^p \times L^p \right)^\ast = L^{p^\prime} \times L^{p^\prime}, \end{equation*}
endowed with a suitable norm (see \hyperref[es:dualsi]{Exercise \ref{es:dualsi}}).

\begin{remark} The representation is \textbf{not} unique. Indeed, the following \hyperref[prop:repvla]{Proposition \ref{prop:repvla}} explains partially why (since $\phi$ can be chosen almost arbitrarily). \end{remark}

\begin{proposition}\label{prop:repvla} Let $x \in I$, and let $p \in [1, \, + \infty)$. Then there exists a function $\Psi_x \in L^{p^\prime}$ such that the valuation functional $j_x \in \left(W^{1, \, p}(I) \right)^\ast$ can be represented as
\begin{equation*} j_x(u) = \int_{I} u(t) \phi(t) \, \mathrm{d}t + \int_I D u(t) \Psi_x(t) \, \mathrm{d}t, \end{equation*}
where $\phi : I \longrightarrow \R$ is a continuous function with compact support.\end{proposition}

\begin{proof} Let $\phi : I \longrightarrow \R$ be a continuous function with compact support and unitary mass
\begin{equation*} \int_I \phi(t) \, \mathrm{d}t = 1. \end{equation*}
Let $u$ be the absolutely continuous representative of the equivalence class $u \in W^{1, \, p}(I)$. In particular, we have that
\begin{equation*} u(x) = u(y) + \int_{y}^x D u(t) \, \mathrm{d}t \quad \text{for every $y < x \in I$}, \end{equation*}
and thus, if one multiplies by $\phi(y)$ and integrate in $\mathrm{d}y$ both members, then it turns out that
\begin{equation*} u(x) = \int_{I} u(y) \phi(y) \, \mathrm{d}y + \int_{I}\left( \int_{y}^x D u(t) \, \mathrm{d}t \right) \phi(y) \, \mathrm{d}y. \end{equation*} 
By Fubini-Tonelli theorem
\begin{equation*} u(x) = \int_{I} u(y) \phi(y) \, \mathrm{d}y + \int_{I} \left[ \mathbbm{1}_{(- \infty, \, x]}(t) \int_I \mathbbm{1}_{(- \infty, \, t]}(y)  \phi(y) \, \mathrm{d}y \right] D u(t) \, \mathrm{d}t, \end{equation*}
and this implies that
\begin{equation*} u(x) = \int_{I} u(y) \phi(y) \, \mathrm{d}y + \int_{I} D  u(y) \Psi_x(y) \mathrm{d}y, \end{equation*} 
where
\begin{equation*} \Psi_x(y) := \mathbbm{1}_{(- \infty, \, x)}(y) \int_{- \infty}^{y} \phi(t) \, \mathrm{d}t. \end{equation*}\end{proof}

\paragraph{Inclusion in Bounded Functions.} It follows from \hyperref[prop:repvla]{Proposition \ref{prop:repvla}} that, given $p \in [1, \, + \infty)$ and $u \in W^{1, \, p}(I)$, we have the estimate
\begin{equation*} \left| u(x) \right| \leq \|u\|_{L^p(I)} \, \|\phi\|_{L^{p^\prime}(I)} + \| D \, u \|_{L^p(I)} \, \| \Psi_x \|_{L^{p^\prime}(I)}. \end{equation*}
Moreover, the $L^{p^\prime}$-norm of $\Psi_x$ does not depend on $x$, and hence there exists a positive constant $C > 0$ such that
\begin{equation*} \left| u(x) \right| \leq C \|u\|_{W^{1, \,p}(I)} ,\end{equation*}
which means that the inclusion
\begin{equation*} W^{1, \, p}(I) \hookrightarrow \left( C_b^0(I), \, \|\cdot\|_{\infty, \, I} \right) \end{equation*}
is continuous with respect to the uniform norm.

\begin{proposition}Let $I$ be a \textbf{bounded} interval of $\R$, and let $p \in (1, \, + \infty]$. Then the inclusion
\begin{equation*} W^{1, \, p}(I) \hookrightarrow C^0\left( \overline{I} \right) \end{equation*}
is continuous and compact.
\end{proposition}

\begin{proof}We have already proved in \hyperref[prop:alphsd]{Proposition \ref{prop:alphsd}} that the inclusion is continuous, and
\begin{equation*} \left| u(x) - u(y) \right| \leq \| D u \|_{L^p(I)} \cdot |x - y|^{ \frac{1}{p^\prime}}. \end{equation*} 
Namely, the elements of the unitary ball in $W^{1, \, p}(I)$ are equicontinuous (as they are also equi-Hölder of parameter $1/p^\prime$). Moreover, the $L^p$-norm is bounded by $1$, and thus it follows from the main-value theorem that for all $u \in B_{W^{1, \, p}(I)}(0, \, 1)$ there exists a point $x_0 := x_0(u) \in I$ such that
\begin{equation*} |u(x_0)| \leq C < + \infty, \end{equation*}
which, in turn, implies that
\begin{equation*} |u(x)| \leq |u(x_0)| + \|D u \|_{L^p(I)} \cdot |x - x_0|^{ \frac{1}{p^\prime}}. \end{equation*}
In particular, every sequence $(u_n)_{n \in \N} \in B_{W^{1, \, p}(I)}(0, \, 1)$ is equibounded and equicontinuous. The interval $I$ is bounded; hence the closure is compact and we conclude that the inclusion
\begin{equation*} W^{1, \, p}(I) \hookrightarrow C^0\left( \overline{I} \right) \end{equation*}
is compact as a straightforward corollary of the Ascoli-Arzelà theorem.\end{proof}

\begin{remark}The inclusion
\begin{equation*} W^{1, \, p}(\R) \hookrightarrow C_b^0\left( \R\right) \end{equation*}
is not compact. Indeed, the action of the group of translations $\mathcal{T}_\R$ preserves the norm, that is,
\begin{equation*} u \in W^{1, \, p}(\R) \implies \left\| T_h u \right\|_{W^{1, \, p}(\R)} = \|u\|_{W^{1, \, p}(\R)}. \end{equation*}
Let $u \in W^{1, \, p}(\R)$ be any function of norm $1$, and let $\left( \tau_n u\right)_{n \in \N} \subset W^{1, \, p}(\R)$ be a sequence. The weak limit is clearly the function identically zero, but
\begin{equation*}\| \tau_n u \|_{W^{1, \, p}(\R^n)} = 1 \quad \text{for all $n \in \N$} \implies \text{$u_n \not \to 0$ strongly}. \end{equation*} \end{remark} 

\subsection{Extension Operator}

Let $r : W^{1, \, p}(\R) \longrightarrow W^{1, \, p}(I)$ be the restriction operator defined by setting
\begin{equation*} r(f) := f \, \big|_{I}. \end{equation*}

\begin{definition}[Extension Operator] \index{extension operator} A linear continuous operator
\begin{equation*} E : W^{1, \, p}(I) \longrightarrow W^{1, \, p}(\R) \end{equation*}
is called \textit{extension operator} if it is a right inverse of $r$, i.e., if
\begin{equation*} r(E(f)) = f. \end{equation*} \end{definition}

\begin{example}Let $u \in W^{1, \, p}(I)$, and assume that $I$ is an open interval of the form $(a, \, b)$. We have proved earlier that there always is an absolutely continuous function in the class of equivalence $u$, which we still denote by $u$. It turns out that the value at the extremal points of the interval can be computed as follows:
\begin{equation*} u(a) = \lim_{n \to + \infty} u \left( a + \frac{1}{n} \right) = \lim_{x \to a^+} u(x) \quad \text{and} \quad u(b) = \lim_{n \to + \infty} u \left( b - \frac{1}{n} \right) = \lim_{x \to b^-} u(x). \end{equation*}
Let $\delta > 0$ be a real number, and consider the following extension of $u$, given by
\begin{equation*} \tilde{u}(x) := \begin{cases} 0 & x \in (- \infty, \, a - \delta], \\ \text{linear interpolation} & x \in (a-\delta, \, a], \\ u(x) & x \in (a, \, b] \\ \text{linear interpolation}, & x \in (b, \, b + \delta], \\ 0 & x \in (b + \delta, \, + \infty).\end{cases} \end{equation*}
The reader may check by herself, as an exercise, that $\tilde{u}$ belongs to $W^{1, \, p}(\R)$. More precisely, it is enough to check that \mbox{}
\begin{enumerate}[label=\textbf{(\arabic*)}]
\item $\tilde{u}$ belongs to $L^p(\R)$, and
\item $\tilde{u}$ is the primitive (in the sense of formula \eqref{eq:sob2ed}) of some $\tilde{v} \in L^p(\R)$.
\end{enumerate}  \end{example}

\begin{example}Let $u \in W^{1, \, p}(I)$ be any function, and assume that $I = [0, \, a)$. We can always consider the extension by reflection, that is,
\begin{equation*} \tilde{u}(x) := \begin{cases} u(x) & x \in (-a, \, 0] \\ u(x) & x \in [0, \, a).\end{cases} \end{equation*}
Again, the reader may verify, as an exercise, that
\begin{equation*} \tilde{u} \in W^{1, \, p}\left( (-a, \, a) \right). \end{equation*} \end{example}

The following theorem asserts that for all $p \in [1, \, + \infty]$, one can always find an extension operator $P$ satisfying the additional property that the inclusions
\begin{equation*} P : W^{1, \, p}(I) \hookrightarrow L^p(\R) \quad \text{and} \quad P : W^{1, \, p}(I) \hookrightarrow W^{1, \, p}(\R) \end{equation*}
are both continuous.

\begin{theorem}[\cite{brezis}]\label{th:ext} Let $1 \leq p \leq \infty$. There exists a bounded linear operator $P : W^{1, \, p}(I) \longrightarrow W^{1,\, p}(\R)$, called an extension operator, satisfying the following properties: \mbox{}
\begin{enumerate}[label=\textbf{(\alph*)}]
\item The map $P$ is an actual extension, that is,
\begin{equation*} P u \, \big|_{I} = u \quad \text{for every $u \in W^{1, \, p}(I)$}. \end{equation*}
\item The inclusion $P : W^{1, \, p}(I) \hookrightarrow L^p(\R)$ is continuous with respect to the $L^p$-norm, that is,
\begin{equation*} \|P  u \|_{L^p(\R)} \leq C  \|u \|_{L^p(I)} \quad \text{for every $u \in W^{1, \, p}(I)$}. \end{equation*}
\item The inclusion $P : W^{1, \, p}(I) \hookrightarrow W^{1, \, p}(\R)$ is continuous, that is,
\begin{equation*} \|P  u \|_{W^{1, \, p}(\R)} \leq C \|u \|_{L^p(I)} \quad \text{for every $u \in W^{1, \, p}(I)$}. \end{equation*}
\end{enumerate}
Furthermore, the constant $C$ depends only on the length of the interval $I$. \end{theorem}

\subsection{Characterizations of $W^{1, \, p}(I)$}

In this section, we investigate some possible characterization of the space $W^{1, \, p}(I)$, for a suitable range of $p$'s, in terms of the $L^p$-continuity property.

\begin{proposition}\label{char:ppp1} Let $p \in (1, \, + \infty]$. If $u \in L^p(I)$, then the following assertions are equivalent: \mbox{}
\begin{enumerate}[label=\textbf{(\alph*)}]
\item The function $u$ belongs to $W^{1, \, p}(I)$.
\item For every $\varphi \in \Cc(I)$ it turns out that
\begin{equation*} \left| \int_I u(t)  \varphi^\prime(t) \, \mathrm{d}t \right| \leq C(u) \|\varphi\|_{L^{p^\prime}(I)}. \end{equation*}
\end{enumerate}
\end{proposition}

\begin{proof} We divide the proof into two steps.

\paragraph{Step 1.} Assume that $u \in W^{1, \, p}(I)$ is a Sobolev function, and let $\varphi \in \Cc(I)$ be a test function. By definition, we have the identity
\begin{equation*}\left| \int_I u(t) \varphi^\prime(t) \, \mathrm{d}t \right| =  \left| -\int_I D  u(t) \varphi(t) \, \mathrm{d}t \right|, \end{equation*}
and thus by the Hölder inequality it turns out that
\begin{equation*}\left| \int_I u(t) \varphi^\prime(t) \, \mathrm{d}t \right| \leq \left\| D \, u \right\|_{L^p(I)} \cdot \|\varphi\|_{L^{p^\prime}(I)}, \end{equation*}
which is exactly what we wanted to prove.

\paragraph{Step 2.} The functional
\begin{equation*} \Cc(I) \ni \varphi \longmapsto \int_I u(t) \varphi^\prime(t) \, \mathrm{d}t \in \R \end{equation*}
is linear and, by assumption, continuous with respect to the $L^{p^\prime}$ norm. The inclusion
\begin{equation*} \Cc(I) \subset L^{p^\prime}(I) \end{equation*}
is dense, which means that the functional can be extended to the whole $L^{p^\prime}$ isometrically. By the \hyperref[theorem:riesz]{Riesz Representation Theorem \ref{theorem:riesz}} there exists\footnote{Here we need $p \neq 1$ since $L^p$ is the dual space of $L^{p^\prime}$ whenever $p^\prime \neq + \infty$.} an element $f \in L^p(I)$ such that
\begin{equation*} \int_I f(t) \varphi(t) \, \mathrm{d}t = \left\langle f, \, \varphi \right\rangle_2 = \int_{I} u(t) \varphi^\prime(t) \, \mathrm{d}t \quad \text{for all $\varphi \in \Cc(I)$}, \end{equation*}
and therefore $- f \in L^p(I)$ is the weak derivative of $u$.
\end{proof}

\begin{proposition}[$L^p$-continuity] Let $p \in [1, \, + \infty)$, and let $u \in L^p(\R)$. Then there exists a modulus of $L^p$-continuity $\omega : [0, \, + \infty] \longrightarrow [0, \, + \infty]$ such taht
\begin{equation} \label{eq,sdsld} \left\| \tau_h f - f \right\|_{L^p(\R)} = \omega(|h|) \searrow 0.\end{equation} 
\end{proposition}

\begin{proof} We divide the proof into two steps.

\paragraph{Step 1.} The property \eqref{eq,sdsld} clearly holds true for all compactly supported continuous functions.

\paragraph{Step 2.} Let us consider the subspace
\begin{equation*}\mathcal{G} := \left\{ g \in L^p(\R) \: \left| \: \text{\eqref{eq,sdsld} is satisfied by $g$} \right. \right\} \subset L^p(\R). \end{equation*}
The first step proves that $C_c^0(\R) \subseteq \mathcal{G}$, which means that if we can prove \eqref{eq,sdsld} for any function on the closure with respect to the $L^p$ norm of $\mathcal{G}$, then \eqref{eq,sdsld} will be true for the closure of $C_c^0(\R)$ in $L^p(\R)$, which coincides with $L^p(\R)$ itself.

\paragraph{Step 3.} Let $f \in \overline{\mathcal{G}}$ be any element of the closure so that for any positive $\epsilon > 0$ there exists a function $g_\epsilon \in \mathcal{G}$ such that
\begin{equation*} \| f - g_\epsilon \|_{L^p(\R)} \leq \epsilon. \end{equation*}
It follows that
\begin{equation*}\begin{aligned} \| \tau_h  f - f \|_{L^p(\R)} & \leq  \| \tau_h f - \tau_h g_\epsilon \|_{L^p(\R)} +  \| \tau_h g_\epsilon - g_\epsilon \|_{L^p(\R)} +  \| f - g_\epsilon \|_{L^p(\R)} = \\[1em] & = 2  \| f - g_\epsilon \|_{L^p(\R)} + \| \tau_h  g_\epsilon - g_\epsilon \|_{L^p(\R)} \leq \\[1em] & \leq 2  \epsilon + \| \tau_h g_\epsilon - g_\epsilon \|_{L^p(\R)}.\end{aligned} \end{equation*}
By taking the limit as $|h| \to 0$, it turns out that 
\begin{equation*}\limsup_{|h| \to 0} \| \tau_h f - f \|_{L^p(\R)} \leq 2 \, \epsilon, \end{equation*}
and this is enough to conclude the proof since $\epsilon$ was chosen to be arbitrarily small.
\end{proof}

We are finally ready to state the main result of this section, namely the characterization of $W^{1, \, p}(I)$ in terms of the $L^p$-continuity property stated above.

\begin{proposition} \label{prop:rllsdlsdl} Let $p \in (1, \, + \infty)$. If $u \in L^p(I)$, then the following assertions are equivalent: \mbox{}
\begin{enumerate}[label=\textbf{(\alph*)}]
\item The function $u$ belongs to $W^{1, \, p}(\R)$.
\item For every translation $\tau_h \in \mathcal{T}_\R$, it turns out that
\begin{equation*} \left\| \tau_h u - u \right\|_{L^p(\R)} \lesssim |h|. \end{equation*}
\end{enumerate}
\end{proposition}

\begin{remark} Let $f \in L^p(\R^n)$ be any $p$-summable function. As we mentioned above $f$ satisfies, for all $p \neq + \infty$, the $L^p$-continuity property, that is,
\begin{equation*} \| \tau_h f - f \|_{L^p(\R^n)} = o(1) \quad \text{as $|h| \to 0$}. \end{equation*}
Equivalently, $f$ admits a \textit{modulus of $L^p$-continuity}, i.e., there exists an increasing continuous function $\omega : [0, \, + \infty] \longrightarrow [0, \, + \infty]$ satisfying the properties
\begin{equation*} \omega(0) = 0 \quad \text{and} \quad \text{$\omega(t) = o(1)$ for $t \to 0$}, \end{equation*}
such that
\begin{equation} \label{mccc} \| \tau_h \, f - f \|_{L^p(\R^n)} \leq \omega \left( |h| \right). \end{equation}\end{remark}

\begin{proof}[Proof of Proposition \ref{prop:rllsdlsdl}] We divide the proof into two steps.

\paragraph{Step 1.} Assume that $u \in W^{1, \, p}(\R)$, and let $u$ be its absolutely continuous representative. For any positive number $h > 0$, we have that
\begin{equation*} u(x + h) - u(x) = \int_{x}^{x + h} D u(t) \, \mathrm{d}t. \end{equation*}
Hence, we can easily estimate the absolute value of the left-hand side using the Hölder inequality
\begin{equation*}  \left| u(x + h) - u(x) \right| \leq \int_x^{x+h} \left| D  u(t) \right| \, \mathrm{d}t \leq \left\| D u \right\|_{L^p(\R)} \, |h|^{1/p^\prime}, \end{equation*}
where $p^\prime$ is the conjugate of $p$. Consequently, we have
\begin{equation*} \begin{aligned}  \left\| u - \tau_h u \right\|_{L^p(\R)}^p & \leq \overbrace{|h|^{p^\prime \cdot p}}^{= h^{p - 1}} \iint_{\R \times \R} \mathbbm{1}_{[x, \, x+ h]}(t)  \left| D u(t) \right| \, \mathrm{d}t \mathrm{d}x \: {\color{red}=} \\[1em] & \: {\color{red}=} \: h \cdot |h|^{p-1} \int_{\R} \left| D u(t) \right|^p \, \mathrm{d}t = \\[1em] & = \left\| D u\right\|_{L^p(\R)}^{p} \cdot |h|^p, \end{aligned}\end{equation*}
where the {\color{red}red} identity follows from the Fubini-Tonelli theorem. Taking the $p$th root of both the left-hand side and the right-hand side, it turns out that
\begin{equation*} \left\| u - \tau_h \, u \right\|_{L^p(\R)} \lesssim |h|,\end{equation*}
where the constant depends on $u$ only $c := \left\| D u \right\|_{L^p(\R)}$, and is finite by assumption.

\paragraph{Step 2.} Vice versa, we use the characterization given by \hyperref[char:ppp1]{Proposition \ref{char:ppp1}} and prove instead that
\begin{equation*} \left\| u - \tau_h u \right\|_{L^p(\R)} \lesssim |h| \implies \left| \int_{\R} u(t) \varphi^\prime(t) \, \mathrm{d}t \right| \leq C(u)  \|\varphi\|_{L^{p^\prime}(\R)} \end{equation*}
for every $\varphi \in C_c^\infty(\R)$. Recall that the translation is a self-adjoint operator, that is,
\begin{equation*} \int_{\R} \tau_h u(t) \varphi(t) \, \mathrm{d}t = \int_{\R}  u(t) \tau_{-h} \varphi(t) \, \mathrm{d}x, \end{equation*}
and hence
\begin{equation*}\int_\R u(t) \frac{\varphi(t) - \varphi(t - h)}{h} \, \mathrm{d}t = - \frac{1}{h} \int_{\R} \left[ \tau_h u(t) - u(t) \right] \varphi(t) \, \mathrm{d}t. \end{equation*}
If we take the absolute value, then it turns out that
\begin{equation*} \begin{aligned} \left| \int_\R u(t) \frac{\varphi(t) - \varphi(t - h)}{h} \, \mathrm{d}t \right| & \leq \frac{1}{|h|} \left| \int_{\R} \left[\tau_h u(t) - u(t) \right] \varphi(t) \, \mathrm{d}t \right| \leq \\[1em] & \leq \frac{1}{|h|}  \| \tau_h  u - u \|_{L^p(\R)} \cdot \| \varphi \|_{L^{p^\prime}(\R)} \lesssim \\[1em] & \lesssim \frac{1}{|h|} |h| \|\varphi\|_{L^{p^\prime}(\R)} \simeq \| \varphi \|_{L^{p^\prime}(\R)}, \end{aligned} \end{equation*}
and we conclude by taking the limit as $h \to 0$ since the right-hand side of the inequality does not depend on $h$ anymore.
\end{proof}

\section{Compactness in $L^p(\R^n)$}

First, we recall a definition that will be extremely useful in what follows in this section.

\begin{definition}[Modulus of Continuity] \index{modulus of continuity} A modulus of continuity is any real-extended valued function 
\begin{equation*} \omega : [0, \, \infty] \longrightarrow [0, \, \infty], \end{equation*}
vanishing at $0$ and continuous at $0$, that is,
\begin{equation*} \lim_{|h| \to 0} \omega(|h|) = 0. \end{equation*}\end{definition}

Let $f \in L^p(\R^n)$ be any $p$-summable function defined on the whole real line. The following properties are trivial:\mbox{}
\begin{enumerate}[label=\textbf{(\arabic*)}]
\item The $L^p$-norm of $f$ is finite, i.e., $\|f\|_{L^p(\R^n)} < + \infty$.
\item There is a modulus of $L^p$-continuity $\omega : [0, \, + \infty] \to [0, \, + \infty]$ such that
\begin{equation} \label{modcontbas} \| \tau_h f - f \|_{L^p(\R^n)} \leq \omega( |h| ). \end{equation}
\item There is a modulus of $L^p$-continuity $\alpha : [0, \, + \infty] \to [0, \, + \infty]$ such that
\begin{equation} \label{modcontalt} \| f \|_{L^p\left(B_R^\complement \right)} \leq \alpha \left( \frac{1}{R} \right). \end{equation}
\end{enumerate}

\begin{remark}If $\mathcal{F} = \{f_1, \, \dots, \, f_n\} \subset L^p(\R^n)$ is a finite family of functions, then the properties \textbf{(1)}, \textbf{(2)} and \textbf{(3)} hold uniformly. More precisely, we have that:
\begin{enumerate}[label=\textbf{(\arabic*)}]
\item The $L^p$-norm of the $f_i$'s is bounded by a uniform constant. Namely, it turns out that
\begin{equation*} \|f_i\|_{L^p(\R^n)} \leq \max_{j = 1, \, \dots, \, n} \|f_j\|_{L^p(\R^n)} =: C < + \infty \end{equation*}
for all $i = 1, \, \dots, \, n$.
\item There is a modulus of $L^p$-continuity $\omega : [0, \, + \infty] \to [0, \, + \infty]$ such that
\begin{equation*} \| \tau_h f_i - f_i \|_{L^p(\R^n)} \leq \omega( |h| )\quad \text{for all $i \in  \{1, \, \dots, \, n\}$}, \end{equation*}
and it is given by the maximum of the moduli of continuity of the $f_i$'s, that is,
\begin{equation*} \omega(t) := \max_{j = 1, \, \dots, \, n} \omega_j(t). \end{equation*}
\item There is a modulus of $L^p$-continuity $\omega : [0, \, + \infty] \to [0, \, + \infty]$ such that
\begin{equation*} \| f_i \|_{L^p(B_R^\complement)} \leq \alpha \left( \frac{1}{R} \right) \quad \text{for all $i \in  \{1, \, \dots, \, n\}$},\end{equation*}
and it is given by the maximum of the moduli of continuity of the $f_i$'s, that is,
\begin{equation*} \alpha(r) := \max_{j = 1, \, \dots, \, n} \alpha_j(r). \end{equation*}
\end{enumerate}\end{remark}

\begin{remark}In a similar fashion, one could prove that any compact family $\mathcal{F} \subset L^p(\R^n)$ satisfies the properties \textbf{(1)}, \textbf{(2)} and \textbf{(3)} uniformly. \end{remark}

We are now ready to state a complete characterization of relatively compact sets (and thus compact sets) in the $L^p$ space for all $p \in [1, \, + \infty)$.

\begin{theorem}[Fréchet-Kolmogorov] \label{frekol} \index{Fréchet-Kolmogorov Theorem}\index{Compactness Theorem in $L^p$} A set $E \subset L^p(\R^n)$, for $p \in [1, \, + \infty)$, is relatively compact if and only if $E$ satisfies the following properties: \mbox{}
\begin{enumerate}[label=\textbf{(\arabic*)}]
\item The set $E$ is equibounded in $L^p(\R^n)$, that is, there exists a constant $c > 0$ such that
\begin{equation*} \|f\|_{L^p(\R^n)} \leq c \quad \text{for all $f \in E$}. \end{equation*}
\item The set $E$ is equicontinuous in $L^p(\R^n)$, that is, there exists a modulus of $L^p$-continuity $\omega : [0, \, + \infty] \longrightarrow [0, \, + \infty]$ such that
\begin{equation*} \| \tau_h f - f \|_{L^p(\R^n)} \leq \omega( |h| )  \quad \text{for all $f \in E$}. \end{equation*}
\item The set $E$ is equiconcentrated in  $L^p(\R^n)$, that is, there exists a modulus of $L^p$-continuity $\alpha : [0, \, + \infty] \longrightarrow [0, \, + \infty]$ such that
\begin{equation*} \| f \|_{L^p\left(B_R^\complement\right)} \leq \alpha\left( \frac{1}{R} \right) \quad \text{for all $f \in E$}. \end{equation*}
\end{enumerate}\end{theorem}

Before we give the complete proof of this result, we discuss briefly the case $p = + \infty$ and why the same statement does not hold.

\begin{proposition}A set $E \subseteq C_0^0(\R^n)$ is relatively compact with respect to the uniform topology if and only if $E$ satisfies the following properties: \mbox{}
\begin{enumerate}[label=\textbf{(\alph*)}]
\item The set $E$ is equibounded in $L^\infty(\R^n)$, that is, there exists a constant $c > 0$ such that
\begin{equation*} \|f\|_{L^\infty(\R^n)} \leq c \quad \text{for all $f \in E$}. \end{equation*}
\item The set $E$ is equicontinuous in $L^\infty(\R^n)$, that is, there exists a modulus of $L^\infty$-continuity $\omega : [0, \, + \infty] \longrightarrow [0, \, + \infty]$ such that
\begin{equation*} \| \tau_h f - f \|_{L^\infty(\R^n)} \leq \omega( |h| )  \quad \text{for all $f \in E$}. \end{equation*}
\item The set $E$ is equiconcentrated in  $L^\infty(\R^n)$, that is, there exists a modulus of $L^\infty$-continuity $\alpha : [0, \, + \infty] \longrightarrow [0, \, + \infty]$ such that
\begin{equation*} \| f \|_{L^\infty\left(B_R^\complement\right)} \leq \alpha\left( \frac{1}{R} \right) \quad \text{for all $f \in E$}. \end{equation*}
\end{enumerate}
\end{proposition}

\begin{proof}The idea is to first prove uniform convergence on compact sets (closed balls), and then generalize it to the whole space.

\paragraph{Step 1.} Let $(f_n)_{n \in \N} \subset E$ be an arbitrary sequence, and fix $R > 0$. The properties $\mathbf{(a)}$ and $\mathbf{(b)}$ allows us to apply the Ascoli-Arzelà theorem and find a subsequence $(f_{n_k, \, R})_{k \in \N}$ uniformly converging to some $f_R$ in the closed ball of center $0$ and radius $R$.

The argument does not depend on the particular value of $R$, therefore we can find such a subsequence for all $R > 0$. The diagonal method allows us to extract a sub-subsequence, still denoted by $(f_{n_k})_{k \in \N} \subset E$, such that
\begin{equation*}f_{n_k} \doublerightarrow f \, \big|_{\overline{B_R}} \quad \text{for all $R > 0$, uniformly in $\overline{B_R}$}. \end{equation*}

\paragraph{Step 2.} In conclusion, the property $\mathbf{(c)}$ implies that $E$ is relatively compact with respect to the uniform norm since one can always write
\begin{equation*} \| f_n - f \|_{\infty, \, \R^n} \leq \|f_n - f\|_{\infty, \, \overline{B_R}} + 2 \, \alpha(R) = \|f_n - f\|_{\infty, \, \overline{B_R}} + o(1) \quad \text{as $R \to + \infty$}. \end{equation*} \end{proof}

\begin{proof}[Proof of Theorem \ref{frekol}] The proof presented here is rather involved, and requires a lot of work. Thus we divide it into five steps to ease the notation a little.

\paragraph{Step 1.} First, observe that, for any $R > 0$, the property $\mathbf{(3)}$ implies that
\begin{equation} \label{eq:frekol1} \| f - \chi_{B_R} f \|_{L^p(\R^n)} = \| f \|_{L^p(B_R^\complement)} \leq \alpha\left(\frac{1}{R} \right), \end{equation}
where $\chi_{B_R}$ is the indicator function of the ball of radius $R$.

\paragraph{Step 2.} In a similar fashion, for every positive mollifier $\varphi \in C_c^0(\R^n)$ with mass equal to $1$ and diameter of the support $d > 0$, it turns out that
\begin{equation} \label{eq:frekol2} \| f \ast \varphi - f \|_{L^p(\R^n)} \leq \sup_{|h| \leq d} \, \| \tau_h f - f \|_{L^p(\R^n)} \leq \omega(d). \end{equation}
The estimate \eqref{eq:frekol2} requires a little bit of work to be justified completely. Indeed, for every such mollifier $\varphi \in C_c^0(\R^n)$, we have
\begin{equation*} \begin{aligned} \left| f \ast \varphi(x) - f(x) \right| & = \left| \int_{\R^n} \varphi(y) \left[ f(x) - f(x - y) \right] \, \mathrm{d}y \right| = \\[1em] & = \left| \int_{\R^n} \varphi^{1/p}(y) \left[ f(x) - f(x - y) \right] \varphi^{1/p^\prime}(y) \,  \mathrm{d}y \right| \leq \\[1em] & \leq \left( \int_{\mathrm{spt}(\varphi)} \varphi(y) \left| f(x) - f(x - y) \right|^p \, \mathrm{d}y \right)^{1/p} \, \left( \underbrace{\int_{\R^n} |\varphi(y)| \, \mathrm{d}y}_{=1} \right)^{1/p^\prime}. \end{aligned} \end{equation*}
It follows that
\begin{equation*} \begin{aligned} \left\| f \ast \varphi - f \right\|_{L^p(\R^n)}^p & \leq \int_{\R^n} \left[ \int_{\mathrm{spt}(\varphi)} \varphi(y) \left| f(x) - f(x - y) \right|^p \, \mathrm{d}y \right] \, \mathrm{d}x  = \\[1em] & = \int_{\mathrm{spt}(\varphi)} \left[ \int_{\R^n} \left| \tau_y  f(x) - f(x) \right|^p \, \mathrm{d}x  \right] \, \varphi(y) \, \mathrm{d}y \leq \\[1em] & \leq \sup_{|y| \leq d} \| \tau_y \, f - f \|_{L^p(\R^n)}^p, \end{aligned} \end{equation*}
where the last inequality follows easily from the fact that the support of $\varphi$ has diameter $d$.

\paragraph{Step 3.} The space $L^p(\R^n)$ is a complete normed space, which means that, as a consequence of \hyperref[totbound]{Lemma \ref{totbound}}, it is enough to prove that
\begin{equation*} \mathbf{(1)} + \mathbf{(2)} + \mathbf{(3)} \implies \text{$E$ is totally bounded.}\end{equation*}
Let $B$ denotes the unitary ball $B_{L^p(\R^n)}(0, \, 1)$. Note that the estimates \eqref{eq:frekol1} and \eqref{eq:frekol2} respectively prove the following inclusions:
\begin{equation*} \begin{cases} E \subseteq \chi_{B_R} \cdot E + \alpha\left( \frac{1}{R} \right) \cdot B, \\[0.6em] E \subseteq \varphi \ast E + \omega(d) \cdot B. \end{cases} \end{equation*}
Moreover, if $E$ is equicontinuous and $R > 0$, then the family of functions $\chi_{B_R}\cdot E$ is also equicontinuous (although, with a different modulus of $L^p$-continuity, which we denote by $\omega^R$). More precisely,
\begin{equation*}\begin{aligned} \left\| \varphi \ast \left( \chi_{B_R} f \right) - f \right\|_{L^p(\R^n)} & \leq \left\| \varphi \ast \left( \chi_{B_R} f \right) - \chi_{B_R} f \right\|_{L^p(\R^n)} + \left\|  \chi_{B_R} f -   f \right\|_{L^p(\R^n)} \leq \\[1em] & \leq \omega^R(d) + \alpha\left(\frac{1}{R}\right),\end{aligned} \end{equation*} 
which means that
\begin{equation*} E \subseteq \chi_{B_R} \cdot E + \alpha\left( \frac{1}{R} \right) \cdot B \subseteq \varphi \ast \left( \chi_{B_R} \cdot E \right) + \left( \omega^R(d) + \alpha\left(\frac{1}{R}\right) \right) \cdot B. \end{equation*}

\paragraph{Step 4.} Fix $\epsilon >0$. One can always find real numbers $R > 0$ and $d > 0$ such that the constant on the right-hand side can be estimated by $2 \epsilon$, that is,
\begin{equation*} E \subseteq \varphi \ast \left( \chi_{B_R} \cdot E \right) + 2 \, \epsilon \cdot B. \end{equation*}
Therefore, we can equivalently prove that the set $\varphi \ast \left( \chi_{B_R} \cdot E \right)$ is totally bounded in $C_c^0(B_{R + d})$ with respect to the uniform norm. In fact, this is enough to come to the same conclusion in the $L^p$ topology (since it is weaker, and thus there are less open sets to check).

\paragraph{Step 5.} For any $f \in L^p(\R^n)$ and any $\varphi \in C_c^0(\R^n)$, it turns out that
\begin{equation*} \begin{aligned} \left\| \tau_h \left( \varphi \ast f \right) - \varphi \ast f \right\|_{\infty} & = \left\| \varphi \ast \left( \tau_h f - f \right) \right\|_{\infty} \; {\color{red} \leq} \\[1em] & \; {\color{red} \leq} \; \| \varphi\|_{L^{p^\prime}(\R^n)} \cdot \| \tau_h f - f \|_{L^{p}(\R^n)}, \end{aligned} \end{equation*}
where the {\color{red}red} inequality follows from a straightforward application of Young inequality\footnote{\textbf{Young Inequality.} Let $f \in L^p(\R^n)$ and $g \in L^q(\R^n)$, and assume that
\begin{equation*}1 + \frac{1}{r} = \frac{1}{p} + \frac{1}{q}.\end{equation*}
The convolution $f \ast g$ belongs to $L^r(\R^n)$, and the following inequality holds:
\begin{equation} \label{eq:youngcont} \|f \ast g \|_{L^r(\R^n)} \leq \| f \|_{L^p(\R^n)} \| g \|_{L^q(\R^n)}. \end{equation}}. In particular, for every $f \in E$, it turns out that
\begin{equation*} \left\| \tau_h \left( \varphi \ast f \right) - \varphi \ast f \right\|_{\infty} \leq \| \varphi\|_{L^{p^\prime}(\R^n)} \cdot \| \tau_h f - f \|_{L^{p}(\R^n)} \leq \| \varphi\|_{L^{p^\prime}(\R^n)} \cdot \omega(|h|),\end{equation*}
from which it also follows that
\begin{equation*} \left\|  \varphi \ast f  \right\|_{\infty} \leq \| \varphi \|_{L^{p^\prime}(\R^n)} \cdot \| f\|_{L^p(\R^n)},\end{equation*}
i.e., the set $\varphi \ast \left( \chi_{B_R} \cdot E \right)$ is equibounded and equicontinuous with respect to the uniform topology (=uniform norm). A straightforward application of the Ascoli-Arzelà theorem proves that the set $\varphi \ast \left( \chi_{B_R} \cdot E \right)$ is relatively compact in $C_c^0(B_{R + d})$ with respect to the uniform norm, and hence it is relatively compact in $L^p(\R^n)$.
\end{proof}

There is a different way to prove this theorem using the regularization by convolution only, but we will not give the details here. The interested reader can try to fill in the missing information in the following road-map.

\begin{proof}[Alternative Proof] Consider the inclusion
\begin{equation*} E \subseteq \varphi \ast E + \omega(d) \cdot B \end{equation*}
proved in the third step of the previous proof. The set $\varphi \ast E$ is compact in $C_0^0(\R^n)$ since one can easily prove that the following properties hold true: \mbox{}
\begin{enumerate}[label=\textbf{(\alph*)}]
\item For every $f \in E$ it turns out that
\begin{equation*} \| \varphi \ast f \|_\infty \leq \|\varphi\|_{L^{p^\prime}(\R^n)} \cdot \|f\|_{L^p(\R^n)}. \end{equation*}
\item For every $f \in E$ and for every $h \in \R$ it turns out that
\begin{equation*} \| \tau_h \left(\varphi \ast f  \right) - \varphi \ast f\|_\infty \leq \|\varphi\|_{L^{p^\prime}(\R^n)} \cdot \omega(|h|). \end{equation*}
\item For every $f \in E$ and for every $R$ sufficiently large it turns out that
\begin{equation*} \|  \varphi \ast f\|_{B_R^\complement, \, \infty} \leq \alpha\left( \frac{1}{R-d} \right). \end{equation*}
\end{enumerate}
In particular, the set $\varphi \ast E$ is compact in $L_{\mathrm{loc}}^p(\R^n)$, and we can conclude that it is also continuously included in $L^p(\R^n)$ by using the property $\mathbf{(3)}$.\end{proof}

\section{Sobolev Space $W^{m, \, p}(\Omega)$}

The primary goal of this section is to generalize the notion of Sobolev space from the interval $I \subseteq \R$ to an arbitrary (and eventually unbounded) open set $\Omega \subseteq \R^n$ for $n \geq 2$.

\begin{remark} \mbox{}
\begin{enumerate}[label=\textbf{(\arabic*)}]
\item Let $\alpha \in \N^n$ be a multi-index. The $\alpha$th derivative operator, denoted by $D^\alpha$, is defined on the space $C_c^\infty(\Omega)$ as follows:
\begin{equation*} D^\alpha u(x_1, \, \dots, \, x_n) := \frac{\partial^{\alpha_1}}{\partial x_1^{\alpha_1}} \dots  \frac{\partial^{\alpha_n}}{\partial x_n^{\alpha_n}} u(x_1, \, \dots, \, x_n). \end{equation*}
\item For every $f \in C^1(\Omega)$ and for every $g \in C_c^\infty(\Omega)$, it turns out that
\begin{equation*} \int_{\Omega} \frac{\partial}{\partial x_i} f(x) g(x) \, \mathrm{d}x = - \int_{\Omega} f(x) \frac{\partial}{\partial x_i} g(x) \, \mathrm{d}x. \end{equation*}
In particular, for any multi-index of length $|\alpha| \leq k$ and any $f \in C^{k}(\Omega)$, the formula above can be generalized as follows:
\begin{equation} \label{sobgen1} \int_{\Omega} D^\alpha f(x) g(x) \, \mathrm{d}x = (-1)^{|\alpha|} \int_{\Omega} f(x)  D^\alpha g(x) \, \mathrm{d}x. \end{equation}
\end{enumerate}\end{remark}

\begin{definition}[Weak Derivative] \index{weak derivative!higher order} Let $f \in \Ll(\Omega)$ be a locally summable function. A function $g \in \Ll(\Omega)$ is the $\alpha$th \textit{weak derivative} of $f$, and we denote $g$ by $D^\alpha f$, if and only if
\begin{equation} \label{wdgen} \int_{\Omega} f(x) \partial^\alpha \varphi(x) \, \mathrm{d}x = (-1)^{|\alpha|} \int_{\Omega} g(x)  \varphi(x) \, \mathrm{d}x, \qquad \forall \, \varphi \in \Cc(\Omega). \end{equation}  \end{definition}

\begin{proposition}\label{prop:312gen}Let $f \in \Ll(\Omega)$. Then the following properties hold: \mbox{}
\begin{enumerate}[label=\textbf{(\alph*)}]
\item If $g \in \Ll(\Omega)$ is the $\alpha$th weak derivative of $f$, then it is unique up to the almost everywhere equivalence relation.
\item The definition \eqref{wdgen} is local. Namely, if there exists a neighborhood $U_x \subset \Omega$ where $f$ is $\alpha$-weakly differentiable for every $x \in \Omega$, then $f$ admits a global $\alpha$-weak derivative $g$ in $\Omega$.
\end{enumerate}\end{proposition}

\begin{proof} \mbox{}
\begin{enumerate}[label=\textbf{(\alph*)}]
\item Suppose that $g_1$ and $g_2$ are both $\alpha$-th weak derivatives of $f$. It follows from the definition formula \eqref{wdgen} that
\begin{equation*} \int_{\Omega} \left[g_1(x) - g_2(x) \right] \varphi(x) \, \mathrm{d}x = 0 \qquad \forall \, \varphi \in \Cc(\Omega), \end{equation*}
and therefore $g_1(x) = g_2(x)$ for almost every $x \in \Omega$ as a consequence of the fundamental lemma in calculus of variations\footnote{\textbf{Lemma.} Let $f \in \Ll(\Omega)$. If
\begin{equation*} \int_{\Omega} f(x) \, \varphi(x) \, \mathrm{d}x = 0 \qquad \forall \, \varphi \in \Cc(\Omega), \end{equation*}
then $f(x) = 0$ for almost every $x \in \Omega$.}.
\item Let us consider an open covering
\begin{equation*} \mathcal{U} := \left\{ U_i \right\}_{i \in \N} \end{equation*}
of $\Omega$, and the collection $g_i \in \Ll(U_i)$ of weak derivatives of $f \, \big|_{U_i}$. Let us consider also the partition of unity
\begin{equation*} \left\{ \rho_i : \Omega \longrightarrow \R_{\geq 0} \right\}_{i \in \N}\end{equation*}
subordinated to the covering $\mathcal{U}$. We claim that the function
\begin{equation*} g(x) := \sum_{i = 0}^{+ \infty} \rho_i(x) \cdot g_i(x) \end{equation*}
is the (global) weak derivative of $f$, that is, the following properties are satisfied: \mbox{}
\begin{enumerate}[label=\textbf{(\arabic*)}]
\item The function is locally summable, that is, $g \in \Ll(\Omega)$.
\item The function $g$ is the weak derivative of $f$, that is, it satisfies \eqref{wdgen}.
\end{enumerate}
Let $K \subset \Omega$ be a compact subset. The partition is locally finite; hence the first assertion follows easily from the estimate
\begin{equation*} \| g \|_{L^1(K)} = \left\| \sum_{i = 1}^\ell \rho_i \cdot g_i \right\|_{L^1(K)} \leq \sum_{i = 1}^{\ell} \| g_i \|_{L^1(K)} < + \infty, \end{equation*}
since the $g_i$'s are locally summable by assumption. 

Let $\varphi \in \Cc(\Omega)$, and let $K \subset \Omega$ be a compact set containing the support of $\varphi$. Then
\begin{equation*}\begin{aligned} \int_{\Omega} f(x) D^\alpha \varphi(x) \, \mathrm{d}x & = \int_{K} f(x)  D^\alpha \varphi(x) \, \mathrm{d}x = \\[1em] & = \int_{K} \left(\sum_{i = 1}^{\ell} \rho_i(x) \right) f(x) D^\alpha  \varphi(x) \, \mathrm{d}x = \\[1em] & = \sum_{i = 1}^{\ell} \int_{K} \left( \rho_i(x)  f(x) \right)  D^\alpha  \varphi(x) \,\mathrm{d}x = \\[1em] & = \sum_{i = 1}^{\ell} \int_{U_i} \left( \rho_i(x) f(x) \right) D^\alpha \varphi(x) \,\mathrm{d}x = \\[1em] & = (-1)^{|\alpha|} \sum_{i = 1}^{\ell} \int_{U_i} \left( \rho_i(x) g_i(x) \right)  \varphi(x) \,\mathrm{d}x = \\[1em] & = (-1)^{|\alpha|} \int_{K} g(x) \varphi(x) \, \mathrm{d}x,\end{aligned}\end{equation*}
which is exactly what we wanted to prove.
\end{enumerate}\end{proof}

\begin{definition}[Sobolev Space] \index{Sobolev space!$W^{m, \, p}$}Let $p \in [1, \, + \infty]$, and let $m \in \N$. The $(m, \, p)$-Sobolev space, denoted by $W^{m, \, p}(\Omega)$, is the space of all $L^p(\Omega)$ functions with $\alpha$ weak derivatives in $L^p(\Omega)$ for every $|\alpha| \leq m$, that is,
\begin{equation*} W^{k, \, p}(I) := \left\{ f \in L^p(\Omega) \: \left| \: \text{$D^\alpha f \in L^p(\Omega)$ for all $|\alpha| \leq m$}  \right. \right\}. \end{equation*} \end{definition}

\paragraph{Normed Space.} In this paragraph, we discuss briefly the main idea that allows us to define a norm on $W^{m, \, p}(\Omega)$ that makes it a Banach space.

\begin{remark}The Sobolev space $W^{m, \, p}(\Omega)$ is a vector space, and the application
\begin{equation*} D^\alpha : W^{m, \, p}(\Omega) \to L^p(\Omega), \qquad f \longmapsto D^\alpha f \end{equation*}
is well-defined and linear for all multi-indices $\alpha \in \N^n$ such that $|\alpha| \leq m$.\end{remark}

\begin{lemma} The graph of the operator $D^\alpha : W^{k, \, p}(\Omega) \longrightarrow L^p(\Omega)$ is closed for every $\alpha \in \N^n$ of length $|\alpha| \leq m$, that is,
\begin{equation*}\Gamma(D^\alpha) := \left\{ \left(f, \, D^\alpha f \right) \: \left| \: f \in W^{k, \, p}(\Omega) \right. \right\} \subset L^p(\Omega) \times L^p(\Omega) \end{equation*}
is closed with respect to the subspace topology.
\end{lemma}

\begin{lemma}Let $N$ denote the cardinality of the set $\mathfrak{N} := \left\{ \alpha \in \N^n \: : \: |\alpha| \leq m \right\}$. The graph of the operator $T$, defined by setting
\begin{equation*} T : W^{k, \, p}(\Omega) \longrightarrow \left( L^p(\Omega) \right)^{N}, \qquad f \longmapsto \left(D^\alpha f\right)_{\alpha \in \mathfrak{N}}, \end{equation*}
is closed with respect to the subspace topology.\end{lemma}

\begin{proof} Let $\left(f_j \right)_{j \in \N} \subset W^{m, \, p}(\Omega)$ be a converging sequence, i.e.,
\begin{equation*}\begin{cases} f_j \xrightarrow{L^p} f \in L^p(\Omega), \\[1em] D^\alpha  f_j \xrightarrow{L^p} f_\alpha \in L^p(\Omega) & \text{for all $\alpha \in \mathfrak{N}$.} \end{cases} \end{equation*}
If we are able to prove that $f_\alpha$ is nothing else than the $\alpha$th weak derivative of $f$ for all $\alpha \in \mathfrak{N}$, then we can easily infer that $f$ belongs to $W^{m, \, p}(\Omega)$. By definition of weak derivative, it turns out that
\begin{equation*} \int_{\Omega} f_j(x) D^\alpha \varphi(x) \, \mathrm{d}x = (-1)^{|\alpha|} \int_{\Omega} D^\alpha  f_j(x) \varphi(x) \, \mathrm{d}x, \qquad \forall \varphi \in \Cc(\Omega) \end{equation*}
holds for every $j \in \N$ and for every admissible $\alpha$.

Therefore, we can pass the identity to the limit using the Lebesgue dominated convergence theorem. This is possible because the $L^p$ convergence implies (up to a subsequence) the a.e. pointwise convergence of the sequence, i.e.
\begin{equation*}\begin{cases} f_{j_k}(x) \xrightarrow{k \to + \infty} f(x) & \text{for almost every $x \in \Omega$}, \\[1em] D^\alpha  f_{j_k}(x) \xrightarrow{k \to + \infty} f_\alpha(x) & \text{for almost every $x \in \Omega$ and for all $\alpha \in \mathfrak{N}$.} \end{cases} \end{equation*} \end{proof}

\begin{corollary} The operator
\begin{equation*} T : W^{m, \, p}(\Omega) \longrightarrow \left( L^p(\Omega) \right)^N\end{equation*}
is linear, injective and onto its rank. More precisely, it turns out that
\begin{equation*} T : W^{m, \, p}(\Omega) \longrightarrow \Gamma(T) \end{equation*}
is linear and bijective. \end{corollary}

In particular, the product $(L^p(\Omega))^N = L^p(\Omega) \times  \dots \times L^p(\Omega)$ induces on the graph $\Gamma(T)$ the subspace topology, that is, the topology generated by the restriction of the product norm
\begin{equation*} \| -_1 \|_p + \dots + \| -_N \|_p. \end{equation*}
More precisely, for $p \neq + \infty$, if we endow $W^{m, \, p}(\Omega)$ with the topology generated by the norm
\begin{equation} \label{sobnormgen} \|f\|_{m, \, p, \, \Omega} := \left( \sum_{|\alpha| \leq m} \left\| D^\alpha f \right\|_{L^p(\Omega)}^p \right)^{\frac{1}{p}}, \end{equation}
then $W^{m, \, p}(\Omega)$ is a Banach space. Similarly, if $p = + \infty$, then the norm
\begin{equation} \label{sobnormgeninfty} \|f\|_{m, \, \infty, \, \Omega} := \sup_{\substack{|\alpha| \leq m \\[0.2em] x \in \Omega}} \left| D^\alpha f(x) \right|, \end{equation}
gives to $W^{m, \, \infty}(\Omega)$ the structure of a Banach space. In any case, there is an isometry
\begin{equation*} \left( W^{m, \, p}(\Omega), \, \|\cdot\|_{m, \, p, \, \Omega} \right) \xrightarrow{\sim} \left( \Gamma(T), \, \|\cdot\|_{(L^p(\Omega))^N} \, \big|_{\Gamma(T)} \right) \end{equation*}
which makes the following properties trivially true: \mbox{}
\begin{enumerate}[label=\textbf{(\alph*)}]
\item The Sobolev space $W^{m, \, p}(\Omega)$ is isomorphic to a closed subset of $\left(  L^p(\Omega) \right)^N$.
\item If $1 \leq p < + \infty$, then $W^{m, \, p}(\Omega)$ is a separable Banach space. If $p = + \infty$, then $W^{m, \, \infty}(\Omega)$ is not separable.
\item If $1 < p < + \infty$, then $W^{m, \, p}(\Omega)$ is a reflexive Banach space.
\end{enumerate}

\paragraph{Dual Space.} \index{Sobolev space!dual}The elements $g$ of the dual space $\left(W^{m, \, p}(\Omega) \right)^\ast$ can be easily represented by
\begin{equation*} g \: : \: W^{m, \, p}(\Omega) \ni u \longmapsto \sum_{|\alpha| \leq m}  \int_{\Omega} g_\alpha(x)  D^\alpha f(x) \, \mathrm{d}x, \end{equation*}
where $g_\alpha \in L^{p^\prime}(\Omega)$, and $p^\prime$ is the conjugate of $p$.

\begin{remark} Contrarily to the one-variable case, in general, it is not true that
\begin{equation*} W^{m, \, p}(\Omega) \subset C^0(\Omega). \end{equation*}
For example, the space $W^{0, \, p}(\Omega) = L^p(\Omega)$ is not contained in the space of all continuous functions on $\Omega$ since a $L^p$ function need not be continuous (nor there is a continuous representative in each class) when the dimension is at least $n \geq 2$. In a similar fashion (see \hyperref[ex:unsdksd]{Exercise \ref{ex:unsdksd}}), one can show that
\begin{equation*} W^{m, \, p}(\Omega) \not \subset L^\infty(\Omega). \end{equation*}\end{remark}

\begin{theorem}[Sobolev Embedding Theorem, \cite{lawrencepde}] \index{Sobolev Embedding Theorem} Let $\Omega$ be a bounded open subset of $\R^n$, with a $C^1$ boundary. Let $f \in W^{m, \, p}(\Omega)$. \mbox{}
\begin{enumerate}[label=\textbf{(\alph*)}]
\item If
\begin{equation*} m < \frac{n}{p}, \end{equation*}
then $f$ belongs to $L^{q}(\Omega)$, where
\begin{equation*} \frac{1}{q} = \frac{1}{p} - \frac{k}{n}. \end{equation*}
We have in addition the estimate
\begin{equation*} \| f \|_{L^q(\Omega)} \leq C(m, \, p, \, n, \, \Omega) \, \| f\|_{W^{m, \, p}(\Omega)}. \end{equation*}
\item If
\begin{equation*} m > \frac{n}{p}, \end{equation*}
then $f \in C^{k - \floor{n/p} - 1, \, \gamma}(\bar{\Omega})$, where
\begin{equation*} \gamma = \begin{cases} \floor{\frac{n}{p}} + 1 - \frac{n}{p}, & \text{if $\frac{n}{p}$ is not an integer} \\[1em] \text{any positive number $< 1$,} & \text{if $\frac{n}{p}$ is an integer}. \end{cases} \end{equation*}
We have in addition the estimate
\begin{equation*} \| f \|_{C^{k - \floor{n/p} - 1, \, \gamma}(\bar{\Omega})} \leq C(m, \, \gamma, \, p, \, n, \, \Omega) \, \| f\|_{W^{m, \, p}(\Omega)}. \end{equation*}
\end{enumerate}\end{theorem}

\begin{remark}The space $W^{m, \, 2}(\Omega)$ is a Hilbert space for any $m \in \N$. \end{remark}

\subsection{Operations on Sobolev Space $W^{m, \, p}(\Omega)$}

\paragraph{Product.} In this brief paragraph, we introduce the main properties of the multiplication operator in Sobolev spaces.\index{Sobolev space!product closure}

\begin{lemma}\label{ex:product} Let $f \in W^{m, \, p}(\Omega)$, and let $\varphi \in C_c^\infty(\Omega)$. \mbox{}
\begin{enumerate}[label=\textbf{(\alph*)}]
\item The support of the product is smaller than the support of the test function, that is,
\begin{equation*}\mathrm{spt}( f  \varphi ) \subseteq \mathrm{spt}(\varphi). \end{equation*}
\item The product is a closed operator, that is,
\begin{equation*}f \varphi \in W^{m, \, p}(\Omega). \end{equation*}
Moreover, the Leibniz rule holds true for any (integer) derivative, that is,
\begin{equation} \label{leibniz100} D^\alpha \left( f \cdot \varphi \right) = \sum_{\beta \leq \alpha} \binom{\alpha}{\beta} \, D^\beta \, f\, D^{\alpha - \beta} \, \varphi. \end{equation}
\end{enumerate}\end{lemma}

\begin{proof} \mbox{}
\begin{enumerate}[label=\textbf{(\alph*)}]
\item Obvious.
\item First, we check the two assertions for the derivatives of order one, and then we generalize it by induction. Indeed, for every test function $\psi$ it turns out that
\begin{equation*} \begin{aligned} \int_\Omega & \left[ f(x) \frac{\partial \varphi(x)}{\partial x_j} + \varphi(x) D^j \left( f(x) \right) \right] \psi(x) \, \mathrm{d}x =  \\[1em] & = \int_\Omega f(x) \frac{\partial \varphi(x)}{\partial x_j}  \psi(x) \, \mathrm{d}x + \int_{\Omega} \varphi(x) D^j \left( f(x) \right) \psi(x) \, \mathrm{d}x = \\[1em] & = \int_\Omega f(x) \left( \frac{\partial \varphi(x)}{\partial x_j} \psi(x) \right) \, \mathrm{d}x + \int_{\Omega} \varphi(x)  D^j \left( f(x) \right)  \psi(x) \, \mathrm{d}x \: {\color{red}=} \\[1em] & \: {\color{red}=} \: - \int_{\Omega} \left( f(x) \varphi(x) \right)  \frac{\partial \psi(x)}{\partial x_j} \, \mathrm{d}x,  \end{aligned} \end{equation*}
where the {\color{red}red} equality follows from the formula
\begin{equation*} \frac{\partial \varphi(x)}{\partial  x_j}  \psi(x) =\frac{\partial  \left( \varphi \psi \right)(x)}{\partial  x_j} - \varphi(x)  \frac{\partial  \psi(x)}{\partial  x_j}. \end{equation*}
The reader may check by herself that Sobolev spaces can be equivalently defined as
\begin{equation*} f \in W^{m, \, p}(\Omega) \iff \begin{cases} D^j f \in W^{m-1, \, p}(\Omega) & \forall \, j =1, \, \dots, \, n \\[1em] f \in L^p(\Omega), \end{cases} \end{equation*}
and hence the thesis follows by induction on the length of the multi-indices $\alpha \in \N^n$.
\end{enumerate}\end{proof}

\paragraph{Regularization by Convolution $(1 \leq p < + \infty)$.} Let $f \in W_c^{m, \, p}(\Omega)$ be a Sobolev function with compact support, and let $\varphi \in C_c^\infty(\Omega)$ be a function such that
\begin{equation*} \mathrm{diam} \left( \mathrm{spt}(\varphi) \right) < d \left( \mathrm{spt}(f), \, \Omega^c \right). \end{equation*}
It is a well-known fact that the support of the convolution is contained in the sum of the supports. As a consequence of the condition above, we have that
\begin{equation*} \mathrm{spt} \left( f \ast \varphi \right) \subset \subset \Omega, \end{equation*}
and the inclusion is compact. Furthermore, one can easily prove that
\begin{equation*} f \ast \varphi \in C_c^\infty(\Omega), \end{equation*}
that is, the usual informal assertion \textit{"the regularity of the convolution is the maximum between the regularity of the elements"} holds true.

\vspace{1.3mm}
Suppose that $\Phi \in C_c^\infty(B(0, \, 1))$ is a mollifier (that is, a positive function with total mass equal to $1$), and let $\Phi_\epsilon$ be the rescaled associated function defined by setting
\begin{equation*} \Phi_\epsilon(x) := \frac{1}{\epsilon^n} \, \Phi \left( \frac{x}{\epsilon} \right). \end{equation*}

A natural question now arises: \textit{The convolution product $\Phi_\epsilon \ast f$ converges to $f$ in the $W^{m, \, p}(\Omega)$ topology (=with respect to the $\| \cdot \|_{m, \,p, \, \Omega}$ norm) when $\epsilon \to 0^+$?}

\begin{lemma}Let $\Phi \in L^1(\R^n)$, and let $f \in W^{m, \, p}(\R^n)$. The convolution product belongs to $W^{m, \, p}(\R^n)$ and, for every $|\alpha| \leq m$, it turns out that
\begin{equation} \label{es:eqq1} D^\alpha \left( f \ast \Phi \right) = D^\alpha f \ast \Phi = f \ast D^\alpha \Phi. \end{equation} \end{lemma}

\begin{proof}Suppose that \eqref{es:eqq1} holds true. By Young's inequality \eqref{eq:youngcont} it turns out that
\begin{equation*}\left\| D^\alpha \left(f \ast \Phi \right) \right\|_{L^p(\R^n)} = \left\| D^\alpha f \ast \Phi \right\|_{L^p(\R^n)} \leq \| \Phi \|_{L^1(\R^n)} \cdot \|D^\alpha f\|_{L^p(\R^n)}  \end{equation*}
for every $|\alpha| \leq m$. Therefore the convolution product belongs to $W^{m, \, p}(\R^n)$.\end{proof}

The answer to the question raised above is thus affirmative. Indeed, it follows from \eqref{es:eqq1} that
\begin{equation*} D^\alpha \left( \Phi_\epsilon \ast f \right) = \Phi_\epsilon \ast D^\alpha f, \end{equation*}
and this converges to $D^\alpha f$ in the $L^p(\Omega)$ topology (= with respect to the $L^p$ norm) for every $|\alpha| \leq m$, which means that
\begin{equation*}\left\|  \Phi_\epsilon \ast f - f \right\|_{m, \, p, \, \Omega} \xrightarrow{\epsilon \to 0^+} 0. \end{equation*}

\section{Sobolev Spaces: $W_0^{m, \, p}(\Omega)$ and $H^{m, \, p}(\Omega)$}

In this final section, we introduce the spaces $W_0^{m, \, p}(\Omega)$ and $H^{m, \, p}(\Omega)$, and we also give a sketch of the proof of the well-known result "$H = W$".

\begin{definition}\index{Sobolev space!$W_0^{m, \, p}$} The $(m, \, p, \, 0)$-Sobolev space is the closure of the space of test functions with respect to the $(m, \, p)$-norm, that is,
\begin{equation*} W_0^{m, \, p}(\Omega) := \overline{ C_c^\infty(\Omega)}^{\| \cdot\|_{m, \, p, \, \Omega}}. \end{equation*}
\end{definition}

\begin{remark}The primary idea behind the notion of $W_0^{m, \, p}(\Omega)$ is to give a meaning to the Dirichlet boundary condition $u(x) = 0$ in the weak formulation of partial differential equations. \end{remark}

\begin{remark}The inclusion
\begin{equation*} W_0^{m, \, p}(\Omega) \subseteq W^{m, \, p}(\Omega) \end{equation*}
is always true since $C_c^\infty(\Omega)$ is a subspace of $W^{m, \, p}(\Omega)$, and so is its closure. On the other hand, if $\Omega= \R^n$, then it turns out that
\begin{equation*} W_0^{m, \, p}(\R^n) = W^{m, \, p}(\R^n), \end{equation*}
coherently with the intuitive meaning of "being zero at the boundary".\end{remark}

\begin{proof}It suffices to prove the inclusion
\begin{equation*} W_0^{m, \, p}(\R^n) \supseteq W^{m, \, p}(\R^n), \end{equation*}
that is, every function $f \in W^{m, \, p}(\R^n)$ is the limit of a sequence of functions $C_c^\infty(\R^n)$ with respect to the Sobolev norm $\| \cdot \|_{m, \, p, \, \Omega}$.

\paragraph{Step 1.} Let us consider a cut-off function such that
\begin{equation*} \eta \in C_c^\infty \left( B(0, \, 2) \right), \quad \eta \, \big|_{B(0, \, 1)} \equiv 1 \quad \text{and} \quad \eta(x) \in [0, \, 1],\end{equation*}
and let us consider the rescaling function
\begin{equation*}\eta_R(x) := \eta \left( \frac{x}{R} \right). \end{equation*}
By \hyperref[ex:product]{Lemma \ref{ex:product}} it turns out that the product function $\eta_R \cdot f$ belongs to $W^{k, \, p}(\Omega)$, and from the Leibniz formula we infer that
\begin{equation*}D^\alpha \left( \eta_R \cdot f \right) = \sum_{\beta \leq \alpha} \binom{\alpha}{\beta} \, D^\beta ( \eta_R) \, D^{\alpha - \beta} (f). \end{equation*}
The $\gamma$th derivative of $\eta_R$ goes to $0$ as $R^{- |\gamma|}$ for $R \to + \infty$, and therefore one can easily prove that
\begin{equation*} \sum_{\substack{\beta \leq \alpha \\[0.3em] \beta \neq 0}} \binom{\alpha}{\beta} \, D^\beta ( \eta_R) \, D^{\alpha - \beta} (f) \xrightarrow{R \to + \infty} 0, \end{equation*}
e.g., as a consequence of the dominated convergence theorem.

\paragraph{Step 2.} In particular, for any fixed $\epsilon > 0$ there exists $R > 0$ such that
\begin{equation*} \left\| \eta_R \cdot f - f \right\|_{m, \, p} \leq \epsilon. \end{equation*}
Let $\Phi \in C_c^\infty(\R^n)$ be a smooth mollifier, and let $\Phi_r$ be its rescaling. The convolution $\Phi_r \ast \left( \eta_R \cdot f \right)$ is smooth, and we find that
\begin{equation*} \Phi_{r} \ast \left( \eta_R \cdot f \right) \xrightarrow{r \to 0^+} \eta_R \cdot f \quad \text{with respect to the $W^{m, \, p}$ norm}. \end{equation*}
Set $r := 1/R$. We infer that
\begin{equation*} \Phi_{\frac{1}{R}} \ast \left( \eta_R \cdot f \right)  \xrightarrow{R \to + \infty} f \quad \text{with respect to the $W^{m, \, p}$ norm}, \end{equation*}
and this is exactly what we wanted to prove.
\end{proof}

\begin{definition}[Nikol'skii space] \index{Nikol'skii space} Let $\Omega \subseteq \R^n$ be an open set. The space $H^{m, \, p}(\Omega)$ is the completion of the space
\begin{equation*} \left\{ f \in C^m(\Omega) \: \left| \: \text{$D^\alpha  f \in L^p(\Omega)$ for all $|\alpha| \leq m$} \right. \right\} \end{equation*}
with respect to the Sobolev norm $\| \cdot \|_{m, \, p, \, \Omega}$.\end{definition}

\begin{theorem}[Meyers-Serrin]\index{Meyers-Serrin Theorem} Let $\Omega \subseteq \R^n$ be an open subset, and let $p \in [1, \, + \infty)$. Then
\begin{equation*} H^{m, \, p}(\Omega) = W^{m, \, p}(\Omega). \end{equation*}\end{theorem}

\begin{remark}The identity, in general, does not hold for the value $p = + \infty$. Indeed, in this particular case, the statement depends also on the regularity of the boundary of $\Omega$. \end{remark}

\begin{proof}The inclusion
\begin{equation*} H^{m, \, p}(\Omega) \subseteq W^{m, \, p}(\Omega) \end{equation*}
is trivial because
\begin{equation*} \left\{ f \in C^m(\Omega) \: \left| \: \text{$D^\alpha  f \in L^p(\Omega)$ for all $|\alpha| \leq m$} \right. \right\}  \subseteq W^{m, \, p}(\Omega), \end{equation*}
and therefore the completion with respect to its norm is also contained in $W^{m, \, p}(\Omega)$.

The opposite inclusion, on the other hand, requires a little bit more work. We divide the proof into four steps to ease the notation, and we leave to the reader to fill in the missing details.

\paragraph{Step 1.} Let us consider the family (as $k$ ranges in $\N$) of open sets
\begin{equation*}\mathcal{U}_k := \left\{ x \in \Omega \: \left| \: \text{$d(x, \, \Omega^c) > \frac{1}{k}$ and $|x| < k$} \right. \right\},\end{equation*}
and notice that it is increasing
\begin{equation*}\mathcal{U}_k \subset \subset \mathcal{U}_{k+1} \subset \subset \dots \subset \subset \Omega \implies \overline{\mathcal{U}_k} \subset \mathcal{U}_{k + 1},\end{equation*}
and such that every $\mathcal{U}_k$ has compact closure.

\paragraph{Step 2.} Let us consider a collection of open sets $\left\{V_k\right\}_{k \in \N}$ satisfying the following property:
\begin{equation*} \mathcal{U}_{k + 1} \supset V_k \supset \overline{U_k}. \end{equation*}
For every $k \geq 1$ there exists a cut-off function $\eta_k \in C_c^\infty \left(\mathcal{U}_k \right)$ such that
\begin{equation*} \eta_k \, \big|_{V_{k-1}} \equiv 1 \quad \text{and} \quad \eta_k \, \big|_{\mathcal{U}_k^c} \equiv 0. \end{equation*}
Let $\Omega_k := \mathcal{U}_k \setminus \overline{\mathcal{U}_{k-2}}$, and let
\begin{equation*} \varphi_k := \eta_k - \eta_{k - 2} \end{equation*}
be a partition of unity associated with the covering $\{\Omega_k\}_{k \geq 2}$. Indeed, it is easy to check that
\begin{equation*} \sum_{k \geq 2} \varphi_k(x) = 1 \quad \text{for all $x \in \Omega$}, \end{equation*}
and also that the sum is locally finite since
\begin{equation*} \Omega_k \cap \Omega_j \neq \emptyset \iff |k - j| = 1. \end{equation*}

\paragraph{Step 3.} Let $f \in W^{m, \, p}(\Omega)$. If we set $f_k(x) := f(x) \cdot \varphi_k(x) \in W_c^{m, \, p}(\Omega_k)$, then we can write
\begin{equation*} f(x) = \sum_{k \geq 2} f_k(x). \end{equation*}
Let $\epsilon > 0$ be a fixed real number. By definition, one can always find a collection of functions $g_k \in C_c^\infty(\Omega_k)$ such that
\begin{equation*} \| g_k - f_k \|_{m, \, p, \, \Omega_k} =  \| g_k - f_k \|_{m, \, p, \, \R^n} \leq \frac{\epsilon}{2^k}. \end{equation*}

\paragraph{Step 4.} Let
\begin{equation*} g(x) := \sum_{k \geq 2} g_k(x). \end{equation*}
The reader may prove by herself, as an exercise, that
\begin{equation*} g \in W^{m, \, p}(\Omega) \quad \text{and} \quad \| f - g \|_{m, \, p, \, \Omega} = \mathcal{O}(\epsilon), \end{equation*}
that is, the test functions are dense in $W^{m, \, p}(\Omega)$ with respect to the norm $\| \cdot \|_{m, \, p, \, \Omega}$. \end{proof}

\section{Exercises}

In this section, we denote by $I$ an open interval $(a, \, b)$ of the real line $\R$ (eventually unbounded), unless otherwise stated.

\begin{exercise}Let $I = (0, \, 1)$, and let $p \in (1, \, + \infty]$. \mbox{}
\begin{enumerate}[label=\textbf{(\arabic*)}]
\item The inclusion $W^{1, \, 1}(I) \subseteq C^0\left(\overline{I}\right)$ is not compact. 
\item The inclusion
\begin{equation*} W^{1, \, 1}(I) \hookrightarrow L^q(I) \end{equation*}
is compact for every $1 \leq q < \infty$.
\end{enumerate}\end{exercise}

\begin{proof} \mbox{}
\begin{enumerate}[label=\textbf{(\arabic*)}]
\item Let us consider the sequence $(u_n)_{n \geq 2} \subset W^{1, \, 1} \left( (0, \, 1) \right)$ defined by
\begin{equation*} u_n(x) = \begin{cases} 0 & \text{if $x \in [0, \, \frac{1}{2}]$}, \\[0.5em] n \cdot \left(x - \frac{1}{2}\right) & \text{if $x \in (\frac{1}{2}, \, \frac{1}{2}+\frac{1}{n}),$} \\[0.5em] 1 & \text{if $x \in [\frac{1}{2} + \frac{1}{n}, \, 1]$}. \end{cases} \end{equation*}
The weak derivatives sequence is clearly given by
\begin{equation*} D \, u_n(x) = \begin{cases} 0 & \text{if $x \in [0, \, \frac{1}{2}]$}, \\[0.5em] n & \text{if $x \in (\frac{1}{2}, \, \frac{1}{2}+\frac{1}{n}),$} \\[0.5em] 0 & \text{if $x \in [\frac{1}{2} + \frac{1}{n}, \, 1]$}, \end{cases} \end{equation*}
hence
\begin{equation*} \begin{aligned} \left\| u_n \right\|_{1, \, 1, \, I} & = n \cdot \int_{\frac{1}{2}}^{\frac{1}{2} + \frac{1}{n}} \left(x - \frac{1}{2} \right) \, \mathrm{d}x + \frac{1}{2} - \frac{1}{n} + n \cdot \int_{\frac{1}{2}}^{\frac{1}{2} + \frac{1}{n}} \mathrm{d}x \\[1em] & = \frac{1}{n} + \frac{1}{2} - \frac{1}{n} + 1 = \frac{3}{2}, \end{aligned} \end{equation*}
that is, the sequence is bounded in $W^{1, \, 1}$.

We now observe that sequence $u_n(x)$ converges pointwise to the bump function
\begin{equation*}u(x) = \begin{cases} 0 & \text{if $x \in [0, \, \frac{1}{2}]$}, \\[0.5em] 1 & \text{if $x \in [\frac{1}{2}, \, 1]$}. \end{cases}  \end{equation*}
On the other hand, for any $n \geq 2$ it turns out that
\begin{equation*} \left\| u_n - u \right\|_{\infty}= 1,\end{equation*}
hence no subsequence of $(u_n)_{n \geq 2}$ can converge to $u$ in the uniform norm.
\item Let $B$ be the unit ball in $W^{1, \, 1}(I)$. Let $P$ be the extension operator of \hyperref[th:ext]{Theorem \ref{th:ext}} and set $\mathcal{F} := P(B)$, so that $B = \mathcal{F} \, \big|_I$.

The set $\F$ is bounded in $W^{1, \, 1}(\R)$ by construction; hence it is bounded also in $L^q(\R)$ for all $1 \leq q < + \infty$. If we are able to prove that for every $f \in \F$ there is a modulus of $L^q$-continuity such that
\begin{equation*} \left\| \tau_h \, f - f \right\|_{L^q(\R)} \leq \omega(|h|), \end{equation*}
then we can apply the compactness result (\hyperref[frekol]{Theorem \ref{frekol}}) to infer that $B$ has compact closure in $L^q(\R)$. From \hyperref[char:ppp1]{Proposition \ref{char:ppp1}} it turns out that
\begin{equation*} \left\| \tau_h \, f - f \right\|_{L^1(\R)} = \mathcal{O}(|h|), \end{equation*}
and hence
\begin{equation*} \left\| \tau_h \, f - f \right\|_{L^q(\R)}^q \leq \left( 2 \, \|f\|_{L^\infty(\R)} \right)^{q - 1} \cdot \left\| \tau_h \, f - f \right\|_{L^1(\R)} \leq C \cdot |h|.\end{equation*}
If we take the $q$-th root, then we obtain the estimate
\begin{equation*} \left\| \tau_h \, f - f \right\|_{L^q(\R)} \leq \ C \cdot |h|^{\frac{1}{q}},\end{equation*}
and this concludes the proof since we can set $\omega(|h|) := C \cdot |h|^{\frac{1}{q}}$ for every $1 \leq q < + \infty$.
\end{enumerate}\end{proof}

\begin{exercise}The Sobolev space $W^{1, \, p}(I)$ is a function algebra, that is,
\begin{equation*}u, \, v \in W^{1, \, p}(I) \implies u \cdot v \in W^{1, \, p}(I), \end{equation*}
and the Leibniz rule also holds true for the weak derivative operator.\end{exercise}

\begin{exercise}[Convolution] Let $u \in W^{1, \, p}(\R)$, and let $v \in L^1(\R)$ be a convolution kernel. Prove that
\begin{equation*} u \ast v \in W^{1, \, p}(\R) \quad \text{and} \quad D \left(u \ast v \right) = u \ast v^\prime. \end{equation*}\end{exercise}

\begin{exercise}[Mollification] Let $\varphi \in C_c^\infty(\R)$ be a convolution kernel, and let us set
\begin{equation*} \varphi_\epsilon(x) := \frac{1}{\epsilon} \, \varphi\left(\frac{x}{\epsilon} \right). \end{equation*}
For any $p \in [1, \, + \infty)$ and for every $u \in W^{1, \, p}(\R)$, it turns out that
\begin{equation*} u_\epsilon := u \ast \varphi_\epsilon \: \xrightarrow{\epsilon \to 0^+} \: u \qquad \text{in $W^{1, \, p}(\R)$}, \end{equation*}
that is,
\begin{equation*} \begin{cases} u_\epsilon \to u & \text{in $L^p(\R)$} \\[1em] D u_\epsilon \to D  u & \text{in $L^p(\R)$}. \end{cases} \end{equation*}\end{exercise}

\begin{exercise}[Density] For any $p \in [1, \, +\infty)$, the inclusion
\begin{equation*}W^{1, \, p}(I) \cap C_c^\infty(I) \subset W^{1, \, p}(I)\end{equation*}
is dense.\end{exercise}

\begin{exercise}Let $g \in C^1(I)$ be a differentiable function such that $\| g^\prime \|_{\infty, \, I} < + \infty$. Prove that, for any $u \in W^{1, \, p}(I)$, it turns out that
\begin{equation*} g \circ u \in W^{1, \, p}(I) \quad \text{and} \quad \frac{\mathrm{d}}{\mathrm{d}x} \left[ g \left(u(x) \right) \right] = g^\prime(u(x)) D u(x). \end{equation*}\end{exercise}

\begin{exercise}Prove that, as a particular case of the previous exercise, it turns out that
\begin{equation*} u \in W^{1, \, p}(I) \implies |u| \in W^{1, \, p}(I). \end{equation*}\end{exercise}

\begin{exercise}[*] Let $I$ be a bounded interval. If $g : \R \longrightarrow \R$ is a Lipschitz function, then
\begin{equation*}u \in W^{1, \, p}(I) \implies \frac{\mathrm{d}}{\mathrm{d}x} \left[ g \left(u(x) \right) \right] = g^\prime(u(x)) Du(x) \quad \text{for almost every $x \in I$}. \end{equation*}
If $I$ is an unbounded interval, then the same formula holds true if we require $g$ to be essentially bounded, that is, if $\|g\|_{L^\infty(I)} < + \infty$.
\end{exercise}

\begin{proof}[\textbf{Solution}] The reader may refer to \cite[Theorem 2.1.11]{wdf} for a proof of this statement.\end{proof}

\begin{exercise}Prove that the Sobolev space $W^{1, \, p}(I)$ is a lattice. \end{exercise}

\begin{exercise}\label{ex:unsdksd}Prove that there exists a function $f \in W^{1, \, 2}(\R^2)$ such that $f$ is unbounded.\end{exercise}

\begin{proof}[\textbf{Solution}] Here we only present a road map of the solution. The reader may try to fill in the missing details as an easy computational exercise.

\paragraph{Step 1.} The function
\begin{equation*} g(x, \, y) = \log \left[ \log \left(1 + \frac{1}{\sqrt{x^2 + y^2}} \right) \right] \end{equation*}
is unbounded, but one can prove as an exercise that it belongs to $W^{1, \, 2}\left(\mathrm{Int}(B(0, \, 1)) \right)$.

\paragraph{Hint.} Take the derivative of $g$, find an estimate from above of its absolute value and then compute the integral using the polar coordinates.

\paragraph{Step 2.} Let $\Phi$ be a cut-off function such that $\phi \, \big|_{B(0, \, 1)} \equiv 1$ and $\mathrm{spt} \, \Phi \subset B(0, \, 2)$. The function
\begin{equation*} f(x, \, y) :=  \left( g \cdot \Phi \right)(x, \, y) \end{equation*}
is an unbounded function on $\R^2$ with compact support. On the other hand, based on what we have proved in the first step, it follows immediately that $f \in W^{1, \, 2}(\R^2)$.
 \end{proof}
\chapter{Functional Spaces}

In this chapter, we introduce some important functional spaces that will allow us to study the regularity of elliptic problems via integral estimates.

More precisely, in the first section, we recall the notion of $\alpha$-Hölder continuity, and we state some facts which will be useful in the next chapter.

Next, we introduce the so-called Morrey-Campanato spaces and prove that, under some assumptions, they are equivalent to $L^p$ spaces, but they are characterized by an integral norm, which is more useful for elliptic problems.

\section{Hölder Spaces}

Let $\Omega \subset \R^n$ be a bounded open set. A function $u$ is $\gamma$-Hölder continuous, with $\gamma \in (0, \, 1)$, on $\Omega$ if
\begin{equation*} [u]_{\gamma, \, \Omega} := \sup_{ \substack{x, \, y \in \Omega \\[0.2em] x \neq y} } \frac{|u(x) - u(y)|}{|x - y|^\gamma} < + \infty, \end{equation*}
and it is Lipschitz if the same holds with $\gamma = 1$. The space
\begin{equation*} C^{0, \, \gamma}(\Omega) := \left\{ u : \Omega \to \C \: \left| \: [u]_{\gamma, \, \Omega} < + \infty \right. \right\} \end{equation*}
is Banach endowed with the norm
\begin{equation*} \|u\|_{\gamma, \, \Omega} := \|u\|_{\infty, \, \Omega} + [u]_{\gamma, \, \Omega}. \end{equation*}
The inclusions
\begin{equation*}C^{0, \, \beta}(\Omega) \subseteq C^{0, \, \alpha}(\Omega) \subseteq C^0(\Omega) \end{equation*}
are easy to verify when $\Omega$ is bounded and $0 < \alpha \leq \beta \leq 1$, but the first one is false if $\Omega$ is unbounded since, e.g.,
\begin{equation*}\|x\|^{\beta} \in C^{0, \, \beta}(\Omega) \setminus C^{0, \, \alpha}(\Omega) \end{equation*}
for every $\alpha > \beta$. The function
\begin{equation*} v(t) := \begin{cases} \frac{1}{\log(t)} & t \neq 0 \\[0.5em] 0 & t = 0 \end{cases} \end{equation*}
is continuous in $\overline{\Omega}$, but it is not $\gamma$-Hölder for any $\gamma \in (0, \, 1]$. Moreover, the inclusion
\begin{equation*}C^{1}\left(\overline{\Omega} \right) \subseteq C^{0, \, 1}(\Omega) \end{equation*}
is false, even if $\Omega$ is bounded, since
\begin{equation*} w(x, \, y) := \begin{cases} \frac{x^2}{\sqrt{x^2 + y^2}} \cdot \arctan \frac{x + |x|}{2y} & y \neq 0 \\[0.5em] 0 & y = 0 \, \, x < 0 \end{cases} \end{equation*}
is differentiable, but not Lipschitz on $\Omega$.

\paragraph{$(k, \, \gamma)$-Hölder spaces.} The space
\begin{equation*} C^{k, \, \gamma}(\Omega) := \left\{ u \in C^k(\Omega) \: \left| \: \sum_{|\alpha| = k} \left[D^\alpha u \right]_{\gamma, \, \Omega} < + \infty \right. \right\} \end{equation*}
is Banach endowed with the norm
\begin{equation*} \|u\|_{k, \, \gamma, \, \Omega} := \sum_{|\alpha| < k} \left\|D^\alpha u \right\|_{\infty, \, \Omega} + \sum_{|\alpha| = k} \left[D^\alpha u \right]_{\gamma, \, \Omega}. \end{equation*}

\begin{theorem}\mbox{}
\begin{enumerate}[label=\textbf{(\arabic*)}]
\item The immersion
\begin{equation*}C^{k, \, \gamma}\left(\overline{\Omega} \right) \subseteq C^{k}\left(\overline{\Omega} \right) \end{equation*}
is continuous and compact for any $\gamma \in (0, \, 1]$.
\item The immersion
\begin{equation*}C^{k, \, \gamma}\left(\overline{\Omega} \right) \subseteq C^{k, \, \beta}\left(\overline{\Omega} \right) \end{equation*}
is continuous and compact for any $0 < \beta \leq \gamma \leq 1$.
\end{enumerate}\end{theorem}

\begin{theorem}[Ciesielski, \cite{siad}] There is an isomorphism
\begin{equation*} C^{k, \, \gamma}\left( \overline{\Omega} \right) \cong \ell_\infty. \end{equation*}
In particular, the space $C^{k, \, \gamma}\left( \overline{\Omega} \right)$ is not separable for any $k \in \N$ and $\gamma \in (0, \, 1]$. \end{theorem}

\section{Morrey Spaces}

Let $\Omega \subset \R^n$ be a bounded open set of diameter $\delta > 0$, and let us set
\begin{equation*} \Omega(x, \, \rho) := B(x, \, \rho) \cap \Omega. \end{equation*}

\begin{definition} The set $\Omega$ is of the type \textbf{A} if there exists a positive constant $A > 0$ such that for any $x \in \overline{\Omega}$ and any $\rho \in (0, \, \delta)$ it turns out that
\begin{equation*} \left| \Omega(x, \, \rho) \right| \geq A \cdot \rho^n. \end{equation*} \end{definition} 

\begin{example} The set defined by
\begin{equation*} \Omega := \left\{ (x, \, y) \in \R^2 \: \left| \: \text{$0 < x < 1$ and $0 < y < x^2$} \right. \right\} \end{equation*}
is not of type A (see \hyperref[Count1]{Figure \ref{Count1}}). \end{example}

\begin{figure}[h]
\centering
\includegraphics[width = 10cm, height = 8cm]{Images/EEEE1.png}
\caption{A set which is not of type A.}
\label{Count1}
\end{figure} 

\paragraph{N.B.} From now on, we shall assume $\Omega$ set of type A.

\paragraph{Morrey Spaces.} Let $u \in L^p(\Omega)$. The limit
\begin{equation*} \lim_{\rho \to 0^+} \int_{\Omega(x, \, \rho)} |u(x)|^p \, \mathrm{d}x \end{equation*}
is equal to zero since the integral is an absolutely continuous operator. For any $\lambda \geq 0$ and $p \in [1, \, + \infty)$, the space
\begin{equation*}L^{p, \, \lambda}(\Omega) := \left\{ u \in L^p(\Omega) \: \left| \: \left\| u \right\|_{p, \, \lambda, \, \Omega} < + \infty \right. \right\}, \end{equation*}
where
\begin{equation} \label{normamorrey}\left\| u \right\|_{p, \, \lambda, \, \Omega}^p := \sup_{\substack{x \in \Omega \\[0.2em] \rho \leq \delta}} \: \frac{1}{\rho^\lambda} \, \int_{\Omega(x, \, \rho)} |u(t)|^p \, \mathrm{d}t. \end{equation}

\begin{proposition}The Morrey space $L^{p, \, \lambda}(\Omega)$ is Banach endowed with the norm \eqref{normamorrey}. \end{proposition}

\begin{proof}Let $(u_n)_{n \in \N} \subset L^{p, \, \lambda}(\Omega)$ be a Cauchy sequence. Then the sequence is uniformly bounded, i.e., there exists $M > 0$ such that
\begin{equation} \label{morreyboun} \| u_n \|_{p, \, \lambda, \, \Omega} \leq M, \end{equation}
and we notice that
\begin{equation*} \| u_n - u_m \|_{0, \, p, \, \Omega} \leq \delta^\lambda \cdot \| u_n - u_m \|_{p, \, \lambda, \, \Omega}. \end{equation*}
In particular, the sequence $(u_n)_{n \in \N}$ is also Cauchy in $L^p$, and hence there exists $u \in L^p(\Omega)$ such that
\begin{equation*} u_n \xrightarrow{L^p(\Omega)} u, \qquad u_{n_k} \xrightarrow{\text{pointwise}} u \end{equation*}
and hence it suffices to prove that $u$ belongs to the Morrey space, and also that the convergence is in $L^{p, \, \lambda}(\Omega)$.

The pointwise convergence is enough to prove the first assertion since one can pass to the limit in \eqref{morreyboun} using the Lebesgue dominated convergence theorem, i.e.,
\begin{equation*}  \| u \|_{p, \, \lambda, \, \Omega} \leq M. \end{equation*}
By assumption for any $\epsilon > 0$ there exists $N_\epsilon$ such that for every $n, \, m > N_\epsilon$ it turns out that
\begin{equation*} \frac{1}{\rho^\lambda} \, \int_{\Omega(x, \, \rho)} \left| u_n(t) - u_m(t) \right|^p \, \mathrm{d}t \leq \| u_n - u_m \|_{p, \, \lambda, \, \Omega} < \epsilon \end{equation*}
for any $\rho > 0$ and for any $x \in \Omega$; on the other hand, if we take the limit as $n \to + \infty$, we infer that
\begin{equation*} \frac{1}{\rho^\lambda} \, \int_{\Omega(x, \, \rho)} \left| u(t) - u_m(t) \right|^p \, \mathrm{d}t < \epsilon \qquad \forall \, \rho > 0, \, \, \forall \, x \in \Omega, \end{equation*}
which implies the convergence in $L^{p, \, \lambda}(\Omega)$.
\end{proof}

\begin{theorem} \mbox{}
\begin{enumerate}[label=\textbf{\arabic*)}]
\item $L^{p, \, 0}(\Omega) \cong L^p(\Omega)$.
\item $L^{p, \, n}(\Omega) \cong L^\infty(\Omega)$.
\item If $\lambda > n$ strictly, then $L^{p, \, \lambda}(\Omega) = \{ 0\}$.
\item If $1 \leq p < q < + \infty$ and
\begin{equation*} \frac{\lambda - n}{p} \leq \frac{\mu - n}{q}, \end{equation*}
then $L^{q, \, \mu}(\Omega) \subseteq L^{p, \, \lambda}(\Omega)$.
\end{enumerate} \end{theorem}

\begin{proof} \mbox{}
\begin{enumerate}[label=\textbf{\arabic*)}]
\item This property follows straightforwardly from the definition.
\item To prove this point, we need to recall a known property of integration theory. If $u$ is a measurable function, we can consider the set
\begin{equation*} S(u, \, \sigma) := \left\{ x \in \Omega \: \left| \: |u(x)| > \sigma \right. \right\}\end{equation*}
as $\sigma$ ranges in $[0, \, + \infty)$. Clearly $S(u, \, \sigma)$ is also measurable and, if $u \in L^p(\Omega)$ for any $p \in [1, \, + \infty)$, then the function
\begin{equation*} \sigma \longmapsto \sigma^{p-1} \cdot | S(u, \, \sigma) | \end{equation*}
is summable on the interval $[0, \, + \infty)$ and it turns out that
\begin{equation} \label{recall:mt} \int_{\Omega} |u(x)|^p \, \mathrm{d}x = p \cdot \sigma^{p-1} \, \int_{0}^{+\infty} | S(u, \, \sigma) | \, \mathrm{d}\sigma. \end{equation}

The inclusion $L^{p, \, n}(\Omega) \supseteq L^\infty(\Omega)$ is rather obvious since
\begin{equation*} \begin{aligned} \frac{1}{\rho^n} \, \int_{\Omega(x, \, \rho)} |u(t)|^p \, \mathrm{d}t & \leq \sup_{t \in \Omega} |u(t)|^p \cdot \frac{| \Omega(x, \, p) |}{\rho^n} \leq \\[1em] & \leq \frac{\omega_n \cdot \rho^n}{\rho^n} \, \|u\|_{L^\infty(\Omega)} \leq \\[1em] & \leq c \cdot \|u \|_{L^\infty(\Omega)}. \end{aligned} \end{equation*}

To prove the opposite inclusion, we argue by contradiction assuming that $L^{p, \, n}(\Omega) \supset L^\infty(\Omega)$ is a strict inclusion. The estimate
\begin{equation*}|u(x)|^p = \lim_{\rho \to 0^+} \frac{1}{\rho^n} \, \int_{\Omega(x, \, \rho)} |u(t)|^p \, \mathrm{d}t \leq \sup_{0 < \rho \leq \delta} \frac{1}{\rho^n} |u(t)|^p \, \mathrm{d}t\end{equation*}
holds for every point $x \in \mathcal{L}(u, \, \Omega)$, i.e., for every Lebesgue point of $u$ in $\Omega$. Let $u$ be an element of $L^{p, \, n}(\Omega) \setminus L^\infty(\Omega)$, i.e.,
\begin{equation*} \|u\|_{\infty, \, \Omega} = + \infty. \end{equation*}
It is equivalent to saying that for any $t > 1$ the measure of the set $S(u, \, t)$ is positive (strictly), and hence that for any $x \in \mathcal{L}(u, \, \Omega) \cap S(u, \, t)$ it turns out that $|u(x)|^p > t$.

Therefore, for any $t > 1$ there exists a $x \in \Omega$ such that
\begin{equation*} t < \sup_{0 < \rho \leq \delta} \frac{1}{\rho^n} \int_{\Omega(x, \, \rho)} |u(t)|^p \, \mathrm{d}t \implies u \notin L^{p, \, n}(\Omega). \end{equation*}
\item This assertion follows trivially from \textbf{2)}. Indeed, for any $x \in \Omega$ we have
\begin{equation*} \sup_{x \in \Omega} \: \frac{1}{\rho^n} \int_{\Omega(x, \, \rho)} |u(t)|^p \, \mathrm{d}t \leq \| u \|_{p, \, \lambda, \, \Omega} \cdot \rho^{\lambda - n}, \end{equation*}
and hence
\begin{equation*} \sup_{\substack{x \in \Omega \\[0.1em] \rho \in (0, \, \rho_1] }} \: \frac{1}{\rho^n} \int_{\Omega(x, \, \rho)} |u(t)|^p \, \mathrm{d}t \leq \| u \|_{p, \, \lambda, \, \Omega} \cdot \rho_1^{\lambda - n}. \end{equation*}
By taking the limit as $\rho_1 \to 0^+$, we infer that the left-hand side converges to $|u(x)|^p$, while the right-hand side converges to $0$, i.e., $u(x) \equiv 0$.
\item For any $x \in \Omega$ and $\rho \in (0, \, \delta]$ it turns out that
\begin{equation*} \begin{aligned} \int_{\Omega(x, \, \rho)} |u(t)|^p \, \mathrm{d}t & = \int_{\Omega(x, \, \rho)} \left( |u(t)|^q \right)^{\frac{p}{q}} \cdot 1^{1 - \frac{p}{q}} \, \mathrm{d}t \leq \\[1em] & \leq \left[ \int_{\Omega(x, \, \rho)} |u(t)|^q \, \mathrm{d}t \right]^{\frac{p}{q}} \cdot \left| \Omega(x, \, \rho) \right|^{1 - \frac{p}{q}} \: {\color{red}\leq} \\[1em] & \: {\color{red}\leq} \: \left[ \int_{\Omega(x, \, \rho)} |u(t)|^q \, \mathrm{d}t \right]^{\frac{p}{q}} \cdot \left[ \omega_n \, \rho^n \right]^{1 - \frac{p}{q}} = \\[1em] & = c(n, \, p, \, q)  \cdot \rho^{ n \left( 1 - \frac{p}{q} \right) + \mu \, \frac{p}{q}} \cdot \left[\frac{1}{\rho^\mu} \, \int_{\Omega(x, \, \rho)} |u(t)|^q \, \mathrm{d}t \right]^{\frac{p}{q}},
\end{aligned} \end{equation*}
where the {\color{red}red} inequality follows from the inclusion $\Omega(x, \, \rho) \subset B(x, \, \rho)$. If we divide both sides by $\rho^\lambda$, it turns out that
\begin{equation*}\frac{1}{\rho^\lambda} \, \int_{\Omega(x, \, \rho)} |u(t)|^p \, \mathrm{d}t \leq c(n, \, p, \, q)  \cdot \rho^{ n \left( 1 - \frac{p}{q} \right) + \mu \, \frac{p}{q} - \lambda} \cdot \left[\frac{1}{\rho^\mu} \, \int_{\Omega(x, \, \rho)} |u(t)|^q \, \mathrm{d}t \right]^{\frac{p}{q}}, \end{equation*}
and the right-hand side is finite if and only if the exponent of $\rho$ is positive, that is,
\begin{equation*} n \left( 1 - \frac{p}{q} \right) + \mu \, \frac{p}{q} - \lambda \geq 0 \iff \frac{\mu - n}{q} \geq \frac{\lambda - n}{p}, \end{equation*}
which is exactly what we wanted to prove.
\end{enumerate} \end{proof}

\section{Campanato Spaces}

In this section, we shall denote by
\begin{equation*} u_{x, \, \rho} := \frac{1}{\left| \Omega(x, \, \rho) \right|} \, \int_{\Omega(x, \, \rho)} u(t) \, \mathrm{d}t \end{equation*}
the average of $u$ on the set $\Omega(x, \,\rho)$. The Campanato $(p, \, \lambda)$ space is defined by
\begin{equation*}\mathscr{L}^{p, \, \lambda}(\Omega) := \left\{ u \in L^p(\Omega) \: \left| \: [ u ]_{\mathscr{L}^{p, \, \lambda}(\Omega)} < + \infty \right. \right\}, \end{equation*}
where
\begin{equation} \label{seminormacampanato}\left[ u \right]_{\mathscr{L}^{p, \, \lambda}(\Omega)}^p := \sup_{\substack{x \in \Omega \\[0.2em] \rho \leq \delta}} \: \frac{1}{\rho^\lambda} \, \int_{\Omega(x, \, \rho)} |u(t) - u_{x, \, \rho}|^p \, \mathrm{d}t \end{equation}
is a seminorm on $\mathscr{L}^{p, \, \lambda}(\Omega)$.

\begin{remark}The Campanato space $\mathscr{L}^{p, \, \lambda}(\Omega)$ is Banach endowed with the norm
\begin{equation} \label{normacampanato}\left\| u \right\|_{\mathscr{L}^{p, \, \lambda}(\Omega)} = \| u \|_{L^p(\Omega)} + \left[ u \right]_{\mathscr{L}^{p, \, \lambda}(\Omega)}.\end{equation} \end{remark}

\begin{proposition}[Characterization] \label{campachar} A function $u : \Omega \to \C$ belongs to $\mathscr{L}^{p, \, \lambda}(\Omega)$ if and only if $u \in L^p(\Omega)$ and the seminorm
\begin{equation} \label{thirdcampanato} \vertiii{u}_{\mathscr{L}^{p, \, \lambda}(\Omega)}^p := \sup_{\substack{x \in \Omega \\[0.2em] \rho \leq \delta}} \: \frac{1}{\rho^\lambda} \, \inf_{c \in \R} \, \int_{\Omega(x, \, \rho)} |u(t) - c|^p \, \mathrm{d}t \end{equation}
is finite. \end{proposition}

\begin{proof}If $u \in \mathscr{L}^{p, \, \lambda}(\Omega)$, then it is a trivial consequence of the fact that
\begin{equation*} \vertiii{u}_{\mathscr{L}^{p, \, \lambda}(\Omega)}^p \leq \left[ u \right]_{\mathscr{L}^{p, \, \lambda}(\Omega)}^p <+ \infty. \end{equation*}
Vice versa, suppose that $u \in L^p(\Omega)$ and $\vertiii{u}_{\mathscr{L}^{p, \, \lambda}(\Omega)} < + \infty$. For every $x \in \Omega$ and $\rho \in (0, \, \delta]$ it turns out that\footnote{We shall freely use the inequality $(a + b)^p \leq 2^{p-1} \cdot (a^p + b^p)$ valid for every $p \geq 1$.}
\begin{equation*} \begin{aligned} \int_{\Omega(x, \, \rho)} |u(t) - u_{x, \, \rho}|^p \, \mathrm{d}t & \leq 2^{p-1} \, \left[  \int_{\Omega(x, \, \rho)} |u(t) - c|^p \, \mathrm{d}t +  \int_{\Omega(x, \, \rho)} |c - u_{x, \, \rho}|^p \, \mathrm{d}t \right] = \\[1em] & = 2^{p-1} \, \left[  \int_{\Omega(x, \, \rho)} |u(t) - c|^p \, \mathrm{d}t + \left| \Omega(x, \, \rho) \right|^{1 - p} \, \left| \int_{\Omega(x, \, \rho)} [u(t) - c]^p \, \mathrm{d}t \right|^p \right] \leq \\[1em] & \leq 2^{p} \int_{\Omega(x, \, \rho)} |u(y) - c|^p \, \mathrm{d}y\end{aligned} \end{equation*}
from which it follows that
\begin{equation*}\left[ u \right]_{\mathscr{L}^{p, \, \lambda}(\Omega)}^p \leq 2 \cdot \vertiii{u}_{\mathscr{L}^{p, \, \lambda}(\Omega)}^p. \end{equation*}
\end{proof}

\begin{corollary}The norms
\begin{equation*} \left\| u \right\|_{\mathscr{L}^{p, \, \lambda}(\Omega)} := \| u \|_{L^p(\Omega)} + \left[ u \right]_{\mathscr{L}^{p, \, \lambda}(\Omega)}\end{equation*}
and
\begin{equation*}\vertiiii{u}_{\mathscr{L}^{p, \, \lambda}(\Omega)} := \| u \|_{L^p(\Omega)} + \vertiii{u}_{\mathscr{L}^{p, \, \lambda}(\Omega)}\end{equation*}
are equivalent.
\end{corollary}

\begin{theorem}\label{campanatotheorem} \mbox{}
\begin{enumerate}[label=\textbf{\arabic*)}]
\item If $0 \leq \lambda < n$, then $\mathscr{L}^{p, \, 0}(\Omega) \cong L^{p, \, \lambda}(\Omega)$.
\item If $n < \lambda \leq n + p$, then $\mathscr{L}^{p, \, 0}(\Omega) \cong C^{0, \, \gamma}(\Omega)$ with
\begin{equation*} \gamma = \frac{\lambda - n}{p}. \end{equation*}
\item If $\lambda > n + p$ strictly, then $u \in \mathscr{L}^{p, \, 0}(\Omega)$ is locally constant.
\item If $1 \leq p < q < + \infty$ and
\begin{equation*} \frac{\lambda - n}{p} \leq \frac{\mu - n}{q}, \end{equation*}
then $\mathscr{L}^{q, \, \mu}(\Omega) \subseteq \mathscr{L}^{p, \, \lambda}(\Omega)$.
\end{enumerate} \end{theorem}

We will only deal with \textbf{1)} since it follows from an algebraic lemma which will be used consistently in the next chapters. The reader may check out the other properties in this book: \cite{mglm}.

\begin{lemma} \label{algebraiclemma} Let $\varphi$ and $\Phi$ be two nonnegative functions defined on $(0, \, d]$, and assume that $\Phi$ is nondecreasing. Assume that there exist positive constants $A, \, \alpha, \, \beta > 0$ such that $\alpha > \beta$ and, for any $t \in (0, \, 1)$ and $\sigma \in (0, \, d]$, 
\begin{equation}\label{eq:cond1} \varphi(t \sigma) \leq A \, t^\alpha \, \varphi(\sigma) + \sigma^\beta \, \Phi(\sigma). \end{equation}
Then, for any $\epsilon \in (0, \, \alpha - \beta]$, $t \in (0,\,1)$ and $\sigma \in (0, \, d]$, it turns out that
\begin{equation}\label{eq:result1} \varphi(t \sigma) \leq A \, t^{\alpha-\epsilon} \, \varphi(\sigma) + K(A) \, \left(t \sigma\right)^\beta \, \Phi(\sigma), \end{equation}
where
\begin{equation*} K(\xi) := \frac{(1 + \xi)^{\frac{2 \, \alpha}{\xi}}}{(1 + \xi)^{ \frac{\alpha - \beta}{\epsilon}} - \xi}. \end{equation*}\end{lemma}

\begin{proof}[Proof of Theorem \ref{campanatotheorem}] The inclusion $L^{p, \, \lambda}(\Omega) \subseteq \mathscr{L}^{p, \, \lambda}(\Omega)$ follows trivially by \hyperref[campachar]{Proposition \ref{campachar}}, since the infimum is taken over all $c \in \R$, included $c = 0$.

The opposite inclusion, on the other hand, is not trivial at all and it requires an application of the algebraic lemma stated above. For every $t \in (0, \, 1)$ and $\sigma \in (0, \, \delta]$ we have the following estimate:
\begin{equation*} \begin{aligned} \int_{\Omega(x, \, t \sigma)} |u(y)|^p \, \mathrm{d}y & \leq 2^{p-1} \, \left[  \int_{\Omega(x, \, \sigma)} |u(y) - u_{x, \, \sigma}|^p \, \mathrm{d}y +  \int_{\Omega(x, \, \sigma)} |u_{x, \, \sigma}|^p \, \mathrm{d}y \right] = \\[1em] & = 2^{p-1} \, \left[  \int_{\Omega(x, \, \sigma)} |u(y) - u_{x, \, \sigma}|^p \, \mathrm{d}y + \left| \Omega(x, \, \sigma) \right| \, \left| \frac{1}{|\Omega(x, \, \sigma)|^p} \, \int_{\Omega(x, \, \sigma)} u(y) \, \mathrm{d}t \right|^p \right] \: {\color{blue}\leq} \\[1em] & \: {\color{blue}\leq} \: 2^{p-1} \, \left[  \int_{\Omega(x, \, \sigma)} |u(y) - u_{x, \, \sigma}|^p \, \mathrm{d}y + \frac{\left| \Omega(x, \, \sigma) \right|}{\left| \Omega(x, \, \sigma) \right|} \, \int_{\Omega(x, \, \sigma)} |u(y)|^p \, \mathrm{d}t \right] \leq \\[1em] & \leq 2^{p-1} \, \left[  \int_{\Omega(x, \, \sigma)} |u(y) - u_{x, \, \sigma}|^p \, \mathrm{d}y + \frac{\omega_n \, (t \sigma)^n}{\left| \Omega(x, \, \sigma) \right|} \, \int_{\Omega(x, \, \sigma)} |u(y)|^p \, \mathrm{d}t \right] \leq \\[1em] & \leq 2^{p-1} \, \left[  \sigma^\lambda \, \left[ u \right]_{\mathscr{L}^{p, \, \lambda}(\Omega)}^p + \frac{\omega_n \, (t \sigma)^n}{\left| \Omega(x, \, \sigma) \right|} \, \int_{\Omega(x, \, \sigma)} |u(y)|^p \, \mathrm{d}t \right]  \: {\color{red}\leq}  \\[1em] & \: {\color{red}\leq}  \: C(p, \, n, \, \sigma) \cdot \left[\left[ u \right]_{\mathscr{L}^{p, \, \lambda}(\Omega)}^p + t^n \, \|u\|_{L^p(\Omega(x, \, \sigma))}^p \right] \: {\color{orange}\leq}  \\[1em] & \: {\color{orange}\leq} \: C^\prime(p, \, n, \, \sigma) \cdot \left[ (t \sigma)^{\lambda} \cdot \left[ u \right]_{\mathscr{L}^{p, \, \lambda}(\Omega)}^p + t^{n - \epsilon} \, \|u\|_{L^p(\Omega(x, \, \sigma))}^p \right] \: {\color{green}\leq}  \\[1em] & \: {\color{green}\leq} \: C^\prime(p, \, n, \, \sigma) \cdot \left(t \sigma\right)^{\lambda} \cdot \left[ \left[ u \right]_{\mathscr{L}^{p, \, \lambda}(\Omega)}^p + \frac{1}{\sigma^\lambda} \|u\|_{L^p(\Omega(x, \, \sigma))}^p \right], \end{aligned} \end{equation*}
and we conclude dividing both sides by $\left(t \sigma\right)^{\lambda}$. The marked inequalities need to be explained a little more in depth: \mbox{}
\begin{enumerate}
\item The {\color{blue}blue} inequality follows from a straightforward application of the Hölder inequality with $u$ and $1$.
\item The {\color{red}red} inequality follows from the fact that $\Omega$ is a set of the type A; more precisely, we have the estimates
\begin{equation*} \begin{aligned} &x \in \partial \, \Omega \implies \left| \Omega(x, \, \rho) \right| \geq A \cdot \sigma^n \implies \frac{1}{ \left| \Omega(x, \, \rho) \right|} \leq \frac{1}{A \cdot \sigma^n}, \\[1em] &x \in \Omega \implies \left| \Omega(x, \, \rho) \right| = \left|B(x, \, \rho)\right| \implies \left| \Omega(x, \, \rho) \right| = \omega_n \, \rho^n.  \end{aligned} \end{equation*}
\item The {\color{orange}orange} inequality follows from \hyperref[algebraiclemma]{Lemma \ref{algebraiclemma}} by setting
\begin{equation*} \begin{aligned} & \varphi(\sigma) = \int_{\Omega(x, \, \sigma)} |u(y)|^p \, \mathrm{d}y, \\[1em] & \Phi(\sigma) = \left[ u \right]_{\mathscr{L}^{p, \, \lambda}(\Omega)}^p \end{aligned} \end{equation*}
\item The {\color{green}green} inequality follows from the fact that we can chose $\epsilon = n - \lambda$.
\end{enumerate}
\end{proof}

\paragraph{Generalized Poincaré inequality.} In this brief paragraph, we want to state and prove a generalization of the Poincaré inequality which will be useful to show a regularity results in Morrey-Campanato spaces.

\begin{theorem}Let $\Omega \subset \R^n$ be an open bounded connected subset of $\R^n$ with Lipschitz boundary. There exists a positive constant $c(p, \, n, \, \Omega)$ such that for any $u \in H^{1, \, p}(\Omega)$, $1 \leq p < + \infty$, it turns out that
\begin{equation} \label{poincargen} \int_\Omega \left| u(x) - u_\Omega \right|^p \, \mathrm{d}x \leq c(p, \, n, \, \Omega) \cdot |u|_{1, \, p, \, \Omega}^p\end{equation} 
where $u_\Omega$ is the average on $\Omega$, i.e.,
\begin{equation*} u_\Omega = \dashint_\Omega u(x) \, \mathrm{d}x. \end{equation*}
\end{theorem}

\begin{proof}We may always assume, without loss of generality, that the average of $u$ on $\Omega$ is equal to zero.

We argue by contradiction. If \eqref{poincargen} does not hold, then there exists a sequence $(u_k)_{k \in \N} \subset H^{1, \, p}(\Omega)$ satisfying the following properties:
\begin{equation*} \begin{cases} \dashint_\Omega u_k(x) \, \mathrm{d}x = 0, \\[0.5em] \|u_k\|_{L^p(\Omega)} = 1, \\[0.5em] |u_k|_{1, \, p, \, \Omega} \leq \frac{1}{k}.\end{cases} \end{equation*}
Since $\{u_k\}_{k \in \N}$ is a bounded subset of $H^{1, \, p}(\Omega)$, by Rellich theorem there is a subsequence $(u_{k_h})_{h \in \N}$ converging to a function $u$ strongly in $L^p(\Omega$; in particular,
\begin{equation*} \| u \|_{L^p(\Omega)} = 1. \end{equation*}
On the other hand, $|u_k|_{1, \, p, \, \Omega} \to 0$ and hence from
\begin{equation*} \int_\Omega u_k(x) \, D^i \, \varphi(x) \, \mathrm{d}x \xrightarrow{m \to + \infty} \int_\Omega u(x) \, D^i \, \varphi(x) \, \mathrm{d}x, \qquad \forall \, \varphi \in C_c^\infty(\Omega) \end{equation*}
it follows that $D^i u = 0$ on $\Omega$ for each $i = 1, \, \dots, \, n$.

In particular, $u$ is locally constant on a connected set, i.e., $u$ is constant on $\Omega$ and hence it is equal to zero (since it is a function with average zero). This is the sought contradiction since $u$ has $L^p(\Omega)$ norm equal to one.\end{proof}

The next result gives an explicit expression for the dependence of the constant $c$ on $\Omega$ when $\Omega$ is a ball.

\begin{theorem}Let $x_0 \in \R^n$. There exists a positive constant $c(p, \, n)$ such that for any $u \in H^{1, \, p}(B(x_0, \, r))$, $1 \leq p < + \infty$, it turns out that
\begin{equation} \label{poincargen2} \int_{B(x_0, \, r)} \left| u(x) - u_{x_0, \, r} \right|^p \, \mathrm{d}x \leq c(p, \, n) \, r^p \cdot |u|_{1, \, p, \, B(x_0, \, r)}^p.\end{equation} 
\end{theorem}

\begin{proof}We may always assume, without loss of generality, that the point $x_0$ is the origin.

Let us consider the homothety $\alpha_r : x \mapsto r \cdot x$, and let us consider the function $v(y) := u \circ \alpha_r(y)$, which is of class $H^{1, \, p}(B(0, \, 1))$. The inequality \eqref{poincargen} holds for $\Omega = B(0, \, 1)$, hence there exists a constant $c^\prime(n, \, p) > 0$ (the constant does not depend on $\Omega$ since the unitary ball is entirely determined by the dimension $n$) such that
\begin{equation*} \int_{B(0, \, 1)} \left| u(x) - v_{B(0, \, 1)} \right|^p \, \mathrm{d}x \leq c^\prime(p, \, n) \cdot |u|_{1, \, p, \, B(0, \, 1)}^p.\end{equation*} 
A straightforward computation proves that
\begin{equation*}v_{B(0, \, 1)} = U_{B(0, \, r)} \qquad \text{and} \qquad \int_{B(0, \,1)} \left|D^i \, v(y) \right|^p \, \mathrm{d}y = r^{p - n} \, \int_{B(0, \, r)} \left|D^i \, u(y) \right|^p \, \mathrm{d}y, \end{equation*}
from which it follows that
\begin{equation*} \int_{B(0, \, 1)} \left| u(x) - v_{B(0, \, 1)} \right|^p \, \mathrm{d}x = \frac{1}{r^n} \, \int_{B(0, \, r)} \left| u(x) - v_{B(0, \, 1)} \right|^p \, \mathrm{d}x \leq c^\prime(p, \, n) \, r^n \cdot |u|_{1, \, p, \, B(0, \, 1)}^p.\end{equation*} 
\end{proof}

We are now ready to state and prove the last theorem which establishes a link between Morrey-Campanato spaces and Sobolev spaces (which is, somehow, a regularity result).

\begin{theorem}Let $\Omega \subset \R^n$ be an open bounded connected subset of $\R^n$ with Lipschitz boundary. If $u \in H^{1, \, p}(\Omega)$ and $D^i \, u \in L^{p, \, \lambda}(\Omega)$, $0 \leq \lambda < n$, for each $i = 1, \, \dots, \, n$, then $u \in \mathscr{L}^{p, \, \lambda + p}(\Omega)$ and
\begin{equation} \label{eq:laeifeoewqe} [u]_{\mathscr{L}^{p, \, \lambda + p}(\Omega)} \leq c(n, \, p, \, A) \cdot \sum_{i = 1}^n \| D^i u \|_{L^{p, \, \lambda}(\Omega)}. \end{equation}
In particular, if $\lambda + p < n$, then
\begin{equation} \label{eq:laeifeoewqe1} [u]_{\mathscr{L}^{p, \, \lambda + p}(\Omega)} \leq c(n, \, p, \, A, \, \lambda) \cdot \left[ \sum_{i = 1}^n \| D^i u \|_{L^{p, \, \lambda}(\Omega)} + \|u\|_{L^p(\Omega)} \right], \end{equation}
and, if $\lambda + p > n$ and $\gamma = 1 - (n-\lambda)/p$, then
\begin{equation} \label{eq:laeifeoewqe1} [u]_{0, \, \gamma, \, \Omega} \leq c(n, \, p, \, A) \cdot \sum_{i = 1}^n \| D^i u \|_{L^{p, \, \lambda}(\Omega)}. \end{equation}
\end{theorem}

\begin{proof}Fix $x_0 \in \Omega$ and $0 < \sigma \leq \delta$. The Poincaré inequality \eqref{poincargen2} it turns out that
\begin{equation*} \begin{aligned} \int_{\Omega(x_0, \, \sigma)} \left| u(x) - u_{x_0, \, \sigma} \right|^p \, \mathrm{d}x & \leq c(n, \, p, \, A) \, \sigma^p \cdot |u|_{1, \, p, \, \Omega(x_0, \, \sigma)}^p \leq \\[1em] & \leq c(n, \, p, \, A) \, \sigma^{p + \lambda} \cdot \sum_{i = 1}^n \left\|D^i u\right\|_{L^{p, \, \lambda}(\Omega)}. \end{aligned}\end{equation*} \end{proof}

\section{Bounded Mean Oscillation Spaces}

Let $Q_0$ be a $n$-dimensional cube in $\R^n$. A function $u \in L^1(Q_0)$ is a function of \textit{bounded mean oscillation}, and we shall denote it by $u \in \mathrm{BMO}(Q_0)$, if
\begin{equation} \label{bmosemi} [u]_{\mathrm{BMO}(Q_0)} = \sup_{Q \subset Q_0} \: \frac{1}{|Q|} \, \int_Q \left| u(x) - u_Q \right| \, \mathrm{d}x \end{equation}
is finite, where the supremum is taken over all the $n$-dimensional cubes $Q \subset Q_0$ with parallel edges.

\begin{remark}The space $\mathrm{BMO}(Q_0)$ coincides with the Campanato space $\mathscr{L}^{1, \, n}(Q_0)$ since one may always consider
\begin{equation*} \Omega(x, \, \rho) = \Omega \cap Q(x, \, \rho) \end{equation*}
instead of
\begin{equation*} \Omega(x, \, \rho) = \Omega \cap B(x, \, \rho), \end{equation*}
and obtain an equivalent Banach space.\end{remark}

Let $u \in L^1(Q_0)$ be given. For every $\sigma > 0$ and for every $n$-dimensional cube $Q \subset Q_0$, with parallel edges, we set
\begin{equation*} S(\sigma, \, Q) := \left\{ x \in Q \: \left| \: |u(x) - u_Q| > \sigma \right. \right\}. \end{equation*}

\begin{definition}[John-Nirenberg] A function $u : Q_0 \to \C$ belongs to $\mathcal{E}_0(Q_0)$ if there are positive constants $H, \, \beta > 0$ such that. for every $\sigma > 0$ and every $n$-dimensional cube $Q \subset Q_0$, it turns out that
\begin{equation} \label{qwoeoqw} \left|S(\sigma, \, Q)\right| \leq H \cdot \mathrm{e}^{- \beta \sigma} \cdot |Q|. \end{equation}\end{definition}

\begin{theorem}A function $u : Q_0 \to \C$ belongs to $\mathcal{E}_0(Q_0)$ if and only if $u$ belongs to the Campanato space $\mathscr{L}^{p, \, n}(Q_0)$, for any $p \geq 1$. \end{theorem}

\begin{theorem}The Campanato spaces $\mathscr{L}^{p, \, n}(Q_0)$, $p \geq 1$, are all equivalent between them. For any couple of real numbers $1 \leq p \leq q$ it turns out that
\begin{equation} \label{qwoeoqw1} \frac{\alpha}{H^{1/q} \cdot \left[ \Gamma(q + 1) \right]^{1/q}} \cdot [u]_{ \mathscr{L}^{q, \, n}(Q_0)} \leq [u]_{ \mathscr{L}^{p, \, n}(Q_0)} \leq [u]_{ \mathscr{L}^{q, \, n}(Q_0)}.\end{equation} \end{theorem}
\chapter{Regularity Theory in Morrey-Campanato Spaces}

\section{Caccioppoli Inequality}

\begin{lemma}Let $A(x) := \{a_{i, \, j}(x)\}_{i, \, j= 1, \, \dots, \, n}$ be a matrix, uniformly elliptic on the ball $B_r$. Assume that the coefficients $a_{i, \, j}(x)$ belongs to $L^\infty(B_r)$, and assume also that $u \in H^1(B_r)$ is a weak solution of the elliptic problem
\begin{equation} \label{ellpr1} \sum_{i, \, j=1}^n D^i \left( a_{i, \, j}(x) \, D^j u(x) \right) = \sum_{i = 1}^n D^i f_i(x), \end{equation}
where $f_i \in L^2(B_r)$ for each $i = 1, \, \dots, \, n$. Then there exists a constant $c(\nu) > 0$ such that, for every $\rho \in (0, \, r)$ and every $(n+1)$-tuple of real numbers $(s, \, s_1, \, \dots, \, s_n) \in \R^{n+1}$, it turns out that
\begin{equation}\label{caccioppoli} \sum_{i = 1}^n \| D^i u \|_{0, \, 2, \, B_\rho}^2 \leq c(\nu) \left[ \frac{1}{(r - \rho)^2} \, \|u - s\|_{0, \, 2, \, B_r}^2 + \sum_{j = 1}^n \| f_j - s \|_{0, \, 2, \, B_r}^2 \right]. \end{equation}
\end{lemma}

\begin{proof} The elliptic problem \eqref{ellpr1} can be equivalently rewritten in its integral form, i.e.,
\begin{equation} \label{ellpr2} \sum_{i, \, j=1}^n \int_{B_r} a_{i, \, j}(x) \, D^j \left[ u(x) - s \right] \, D^i \varphi(x) \, \mathrm{d}x = \sum_{i = 1}^n \int_{B_r} \left[f_i(x) - s_i \right] \ D^i \varphi(x), \qquad \forall \, \varphi \in H_0^1(B_r). \end{equation}
Let $\theta \in C_c^\infty(\R^n)$ be a cutoff function satisfying the following properties: \mbox{}
\begin{enumerate}[label=\textit{(\alph*)}]
\item The support of $\theta$ is compactly contained in $B_r$, and $\theta \equiv 1$ on $B_\rho$.
\item For every $x$ it turns out that $0 \leq \theta(x) \leq 1$.
\item The derivative is bounded, i.e.,
\begin{equation*}\| \nabla \, \theta \|_{\infty} \leq \frac{1}{r - \rho}.\end{equation*}
\end{enumerate}
We pick $\varphi(x) := \theta^2(x) \cdot \left[u(x) - s \right]$ in \eqref{ellpr2} as a test function, and we obtain the identity
\begin{equation*} \begin{aligned} \sum_{i, \, j=1}^n & \int_{B_r} a_{i, \, j}(x) \, D^j \left[ u(x) - s \right] \,  \theta(x) \, D^i \left( (u(x) - s) \, \theta(x) \right) \, \mathrm{d}x+ \dots \\[1em] & \dots + \sum_{i, \, j=1}^n \int_{B_r} a_{i, \, j}(x) \, D^j \left[ u(x) - s \right] \,  \theta(x) \, D^i \left( (u(x) - s) \, \theta(x) \right) \, \mathrm{d}x  = \\[1em] & = \sum_{i = 1}^n \int_{B_r} \left[f_i(x) - s_i \right] \, \theta(x) \, D^i\left( \theta(x) \, [u(x) - s] \right) \, \mathrm{d}x + \dots \\[1em]&  \dots + \sum_{i = 1}^n \int_{B_r} \left[f_i(x) - s_i \right] \, D^i \left(\theta(x) \right) \, \theta(x) \, [u(x) - s]  \, \mathrm{d}x  \end{aligned}\end{equation*}
from which it follows that
\begin{equation*} \begin{aligned} \sum_{i, \, j=1}^n & \int_{B_r} a_{i, \, j}(x) \, D^j \left( (u(x) - s) \, \theta(x) \right)\, D^i \left( (u(x) - s) \, \theta(x) \right) \, \mathrm{d}x = \\[1em] & = \sum_{i, \, j=1}^n \int_{B_r} a_{i, \, j}(x) \, [u(x) - s]^2 \, D^j \theta(x) \, D^i \theta(x) \, \mathrm{d}x + \dots \\[1em] & \dots + \sum_{i = 1}^n \int_{B_r} \left[f_i(x) - s_i \right] \, \theta(x) \, D^i\left( \theta(x) \, [u(x) - s] \right) \, \mathrm{d}x + \dots \\[1em]&  \dots + \sum_{i = 1}^n \int_{B_r} \left[f_i(x) - s_i \right] \, D^i \left(\theta(x) \right) \, \theta(x) \, [u(x) - s]  \, \mathrm{d}x. \end{aligned}\end{equation*}
If we take the modules of the identity above, then it turns out that
\begin{equation*} \begin{aligned} \nu \, \sum_{i =1}^n \int_{B_r} \left| D^i \left[ (u(x) - s) \, \theta(x) \right] \right|^2 \, \mathrm{d}x & \leq c(a_{i, \, j}) \, \frac{c^2}{(r - \rho)^2} \, \sum_{i, \, j=1}^n \int_{B_r} [u(x) - s]^2 \, \mathrm{d}x + \dots \\[1em] & \dots + \frac{1}{2 \epsilon} \, \sum_{i = 1}^n \int_{B_r} \left[f_i(x) - s_i \right]^2 \mathrm{d}x + \dots \\[1em] & \dots + \frac{\epsilon}{2} \sum_{i = 1}^n \int_{B_r} \left| D^i \left[ \theta(x) \, (u(x) - s)^2 \right] \right|^2 \, \mathrm{d}x + \dots \\[1em] & \dots + \frac{c}{(r - \rho)^2} \, \sum_{i, \, j=1}^n \int_{B_r} [u(x) - s]^2 \, \mathrm{d}x + \dots \\[1em] & \dots + \frac{1}{2}\, \sum_{i = 1}^n \int_{B_r} \left[f_i(x) - s_i \right]^2 \mathrm{d}x, \end{aligned}\end{equation*}
and we obtain the thesis by taking $\epsilon$ sufficiently small.
\end{proof}

\bibliography{Bibliografia}{} % BIBLIOGRAFIA
\bibliographystyle{plain}
\end{document}
