\chapter{Existence Results for Elliptic Problems}

In this chapter we are concerned with the existence of solutions  for elliptic problems, both in variational and in nonvariational form.

\section{Near Operators Theory}

\begin{definition}[Nearness] Let $\mathfrak{X}$ be any set, let $\mathcal{B}$ be a Banach space and let $A, \, B : \mathfrak{X} \to \mathcal{B}$ be operators. We say that $A$ is \textit{near} to $B$ if there exist a constant $\alpha > 0$ and $k \in (0, \, 1)$ such that
\begin{equation} \label{eq:nearness} \left\| B(x_1) - B(x_2) - \alpha \left[ A(x_1) - A(x_2) \right] \right\| \leq k \, \left\| B(x_1) - B(x_2) \right\|, \qquad \forall \, x_1, \, x_2 \in \mathfrak{X}. \end{equation}\end{definition}

\begin{remark} The condition is in general \textbf{not} symmetric: if $A$ is near $B$, then it's not necessarily true that $B$ is near $A$. On the other hand, if $\mathcal{B}$ is a Hilbert space, then the condition is symmetric. %DIM.
\end{remark}

\begin{lemma}Let $\mathfrak{X}$ be any set, let $\mathcal{B}$ be a Banach space, let $A, \, B : \mathfrak{X} \to \mathcal{B}$ be operators and assume that $A$ is near to $B$. Then the following inequalities are satisfied for any $x_1, \, x_2 \in \mathfrak{X}$:
\begin{equation} \label{ineq1} \left\| B(x_1) - B(x_2) \right\| \leq \frac{\alpha}{1 - k} \, \left\| A(x_1) - A(x_2) \right\|,\end{equation} 
\begin{equation} \label{ineq2} \left\| A(x_1) - A(x_2) \right\| \leq \frac{k+1}{\alpha} \, \left\| B(x_1) - B(x_2) \right\|. \end{equation}\end{lemma}

\begin{proof} We only prove the first inequality, and leave the second one to the reader. Since $A$ is near to $B$, we have that
\begin{equation*} \begin{aligned} \left\| B(x_1) - B(x_2) \right\| & \leq \left\| B(x_1) - B(x_2) - \alpha \left[ A(x_1) - A(x_2) \right] \right\| + \alpha \, \left\| A(x_1) - A(x_2) \right\| \leq \\ & \leq k \, \|B(x_1) - B(x_2)\|  + \alpha \, \left\| A(x_1) - A(x_2) \right\| , \end{aligned}\end{equation*} 
from which it follows easily that
\begin{equation*}\left(1 - k\right) \cdot \left\| B(x_1) - B(x_2) \right\| \leq \alpha \, \left\| A(x_1) - A(x_2) \right\|.\end{equation*} \end{proof}

\begin{lemma}\label{lemma:infjd}Let $\mathfrak{X}$ be any set, let $\mathcal{B}$ be a Banach space, let $A, \, B : \mathfrak{X} \to \mathcal{B}$ be operators and assume that $A$ is near to $B$. Then $A$ is injective if and only if $B$ is injective. \end{lemma}

\begin{proof}This is a immediate consequence of the previous Lemma, since \eqref{ineq1} and \eqref{ineq2} easily implies that $A(x_1) = A(x_2)$ if and only if $B(x_1) = B(x_2)$. \end{proof}

\begin{lemma}Let $B : \mathcal{X} \to \mathcal{B}$ be an injective operator. Then $(\mathcal{X}, \, d)$ is a metric space, where
\begin{equation*}d(x, \, y)  = \| B(x) - B(y) \|_{\mathcal{B}}, \qquad \forall \, x, \, y \in \mathcal{X}.\end{equation*} 
Moreover, if $B$ is also surjective, then the metric space $(\mathcal{X}, \, d)$ is complete. \end{lemma}

\begin{proof} We want to prove that $d$ is a distance on $\mathcal{X}$, that is, the characterizing properties hold true. By definition it is greater or equal to zero and, since $B$ is injective, it follows that
\begin{equation*}d(x, \, y)  = 0 \iff B(x) = B(y) \iff x = y. \end{equation*}
The symmetric property and the triangle inequality are straightforward consequences of the definition, therefore we will not give more details.

Assume that $B$ is bijective and let $(x_n)_{n \in \N} \subset \mathcal{X}$ be a Cauchy sequence. Then $\left(B(x_n) \right)_{n \in \N} \subset \mathcal{B}$ is a Cauchy sequence and, by completeness of $\mathcal{B}$, it converges to $y \in \mathcal{B}$. Since $B$ is surjective, there is $x \in \mathcal{X}$ such that $B(x) = y$; it follows that
\begin{equation*} d(x_n, \, x) = \| B(x_n) - y \|_{\mathcal{B}} \xrightarrow{n \to + \infty} 0, \end{equation*}
and this concludes the proof.  \end{proof}

\begin{theorem}\label{bijectle}Let $\mathfrak{X}$ be any set, let $\mathcal{B}$ be a Banach space, let $A, \, B : \mathfrak{X} \to \mathcal{B}$ be operators and assume that $A$ is near to $B$. If $B$ is a bijection, then also $A$ is a bijection. \end{theorem}

\begin{proof} It suffices to prove that $A$ is a \textit{surjective} operator. In fact, it is \textit{injective} as a consequence of \hyperref[lemma:infjd]{Lemma \ref{lemma:infjd}} which was proved earlier.

Let $f \in \mathcal{B}$ be any element. We want to prove that there exists a (unique) $u \in \mathcal{X}$ such that $A(u) = f$ or, equivalently, that
\begin{equation*} B(u) = B(u) - \alpha \, A(u) + \alpha \, f =:F(u). \end{equation*}

For any $u \in \mathcal{X}$, $F(u)$ belongs to $\mathcal{B}$, hence there is one and only one $v \in \mathcal{X}$ such that $B(v) = F(u)$, as $B$ is a bijective operator. 

Let us denote by $T : \mathcal{X} \to \mathcal{X}$ the application that sends $u$ in the unique solution $v$. If $v = T(u)$ and $w = T(z)$, it turns out that
\begin{equation*} \begin{aligned} d(v, \, w) & = \left\| B(v) - B(w) \right\|_{\mathcal{B}} = \left\| F(u) - F(z) \right\|_{\mathcal{B}} = \\ &  = \left\| B(v) - B(w) - \alpha \left[ A(v) - A(w) \right] \right\| \leq k \, \left\| B(v) - B(w) \right\|, \end{aligned} \end{equation*}
i.e., $T$ is a contraction (since $k$ is strictly less than one).

Finally $\mathcal{X}$ is a complete metric space, thus there is one and only one fixed point $u \in \mathcal{X}$ of $T$, that is, $A$ is surjective:
\begin{equation*} T(u) = u \implies B(u) = F(u) \implies A(u) = f. \end{equation*}\end{proof}

\paragraph{Equivalence Relation.} Let us endow the set $\mathcal{X}$ with the following equivalence relation:
\begin{equation*} u \sim v \iff B(u) = B(v). \end{equation*}
Let us denote by $[u]_\mathcal{X}$ the equivalence class of $u$ and let $X = \faktor{\mathcal{X}}{\sim}$. We can define two operators $A^\ast, \, B^\ast : X \to \mathcal{B}$ by setting
\begin{equation*} A^\ast( [u]_{\mathcal{X}} ) = A(u) \qquad \text{and} B^\ast( [u]_{\mathcal{X}} ) = B(u), \end{equation*}
then, if $A$ is near $B$, also $A^\ast$ is near $B^\ast$ (with the same constants). 

If we assume that $B$ is surjective, then $B^\ast$ is bijective (by definition). It follows from \hyperref[bijectle]{Theorem \ref{bijectle}} that $A^\ast$ is bijective, and this directly implies that $A$ is surjective.

\begin{theorem}Let $\mathfrak{X}$ be any set, let $\mathcal{B}$ be a Banach space, let $A, \, B : \mathfrak{X} \to \mathcal{B}$ be operators and assume that $A$ is near to $B$. If $B$ is a surjective operator, then also $A$ is a surjective operator. \end{theorem}

\begin{theorem}\label{th:conm} Let $\mathfrak{X}$ be any set, let $\mathcal{B}$ be a Banach space and let $\{A_t : \mathcal{X} \to \mathcal{B}\}_{t \in [0, \, 1]}$ be a family of operators such that: \mbox{}
\begin{enumerate}[label=\textbf{(\alph*)}]
\item There exists $t_0 \in [0, \, 1]$ such that $A_{t_0}$ is a bijection.
\item There exists $c > 0$ such that, for any $s, \, t \in [0, \, 1]$ and any $x, \, y \in \mathcal{X}$,
\begin{equation*} \left\| A_t(x) - A_t(y) - \left[ A_s(x) - A_s(y) \right] \right\|_{\mathcal{B}} \leq c \, |t - s| \, \|A_t(x) - A_t(y) \|_{\mathcal{B}}. \end{equation*}
\end{enumerate}
Then, for any $s \in [0, \, 1]$, it turns out that $A_s$ is a bijection.\end{theorem}

\begin{proof}Set $I := \left\{ t \in [0, \, 1] \: \left| \: \text{$A_t$ is a bijection} \right. \right\}$. By connectedness of $[0, \, 1]$, it suffices to prove that $I$ is nonempty, open and closed in the subspace topology.

The assumption \textbf{(a)} implies that $I$ is nonempty. Let $t \in I$ and let $\delta > 0$ be such that, for any $s \in [t - \delta, \, t + \delta] \cap [0, \, 1]$, it turns out that $c \, |t - s| < 1$. Then the inequality \textbf{(b)} implies that $A_s$ near $A_t$, and thus $A_s$ is bijective (i.e. $s \in I$) by \hyperref[bijectle]{Theorem \ref{bijectle}}.

Let $(s_n)_{n \in \N} \subset I$ be a converging sequence and let $s \in [0, \, 1]$ be its limit. There exists $N \in \N$ such that $c \, |s_n - s| < 1$, for any $n \geq N$, therefore $A_s$ is near $A_{s_N}$ and, again, it is bijective.\end{proof}

\paragraph{Fréchet Differential} Let $\mathcal{B}_1$ and $\mathcal{B}_2$ be Banach spaces and let $F : U \subset \mathcal{B}_1 \to \mathcal{B}_2$. The function $F$ is differentiable in the sense of Fréchet at $u_0 \in U$ if there is a linear continuous application $L : \mathcal{B}_1 \to \mathcal{B}_2$ such that
\begin{equation} \lim_{h \to 0} \frac{ \| F(u_0 + h) - F(u_0) - L \, h \|_{\mathcal{B}_2}}{\|h\|_{\mathcal{B}_1}} = 0, \end{equation}
and we denote it by $\mathrm{d}F_{u_0}$.

We say that $F$ is continuously differentiable at $u_0$ if it is differentiable in a neighborhood $W$ of $u_0$ and the mapping $W \ni u \mapsto \mathrm{d}F_u \in \mathcal{L}(\mathcal{B}_1, \, \mathcal{B}_2)$ is continuous at $u_0$.

\begin{theorem} Let $(Y, \, \| \cdot \|_Y)$ and $(Z, \, \| \cdot \|_Z)$ be two Banach spaces, let $y_0 \in Y$ be any point and let $F : U \to Z$ be a function defined on a neighborhood of $y_0$. Assume that \mbox{}
\begin{enumerate}[label=\textbf{(\alph*)}]
\item $F \in C^1(U)$;
\item the Fréchet differential $\mathrm{d}F_{y_0} : Y \to Z$ is invertible.
\end{enumerate}
Then there exist $k \in (0, \, 1)$ and a neighborhood $W \subset U$ such that for any $y_1, \, y_2 \in W$
\begin{equation*} \left\| \mathrm{d}F_{y_0}[y_1-y_2] - \left[F(y_1) - F(y_2)\right] \right\|_Z \leq \left\| \mathrm{d}F_{y_0}[y_1 - y_2] \right\|_Z. \end{equation*} \end{theorem}
 
\section{Historical Digression on Near Operators}

The idea of introducing the notion of \textit{nearness} between operators originates from the existence and uniqueness problem of nonvariational elliptic equations of the following kind:
\begin{equation} \label{problema} \begin{cases} u \in H^2(\Omega) \cap H_0^1(\Omega) \\ \\ \displaystyle\sum_{i, \, j = 1}^n a_{i, \, j}(x) \, D^{i, \, j} \, u(x) = f(x) & \text{for $x \in \Omega$}, \end{cases} \end{equation}
where $\Omega \subset \R^n$ is bounded and - for simplicity only - convex, $f \in L^2(\Omega)$, the coefficients $a_{i, \, j} \in L^\infty(\Omega)$ and the matrix $A = \left(a_{i, \, j} \right)_{i, \, j = 1, \, \dots, \, n}$ is uniformly elliptic and symmetric.

\vspace{1.8mm}
If $n > 2$ the problem \eqref{problema} is not well-defined under the solely assumption of uniformly ellipticity (see \hyperref[talenti]{Example \ref{talenti}}). Therefore, it is necessary to introduce more restrictive condition on the coefficients of the matrix $A$, e.g., regularity ($a_{i, \, j} \in C^0(\Omega)$) or algebraic conditions.

\begin{definition}[Corder Condition] Let $A = \left(a_{i, \, j} \right)_{i, \, j = 1, \, \dots, \, n}$ be a matrix such that $\|A(x)\| \neq 0$ for almost every $x \in \Omega$. We say that $A(x)$ satisfies the \textit{Corder condition} if there exists $\epsilon > 0$ such that
\begin{equation} \label{concord} \frac{ \left( \sum_{i = 1}^{n} a_{i, \, i}(x) \right)^2 }{\sum_{i, \, j = 1}^n a_{i, \, j}^2(x) } \geq n - 1 + \epsilon \qquad \text{for almost every $x \in \Omega$}. \end{equation} \end{definition}

\begin{definition}[Campanato Condition] Let $A = \left(a_{i, \, j} \right)_{i, \, j = 1, \, \dots, \, n}$ be a matrix. We say that $A(x)$ satisfies the \textit{$A_x$ condition} if there are real constants $\sigma, \, \gamma$ strictly positive, $\delta \geq 0$ and a function $a(x) \in L^\infty(\Omega)$ such that $\gamma + \delta < 1$, $a(x) \geq \sigma > 0$ and
\begin{equation} \label{condax} \left| \sum_{i = 1}^n \xi_{i, \, i} - a(x) \cdot \sum_{i, \, j = 1}^n a_{i, \, j}(x) \, \xi_{i, \, j} \right| \leq \gamma \cdot \left( \sum_{i, \, j = 1}^n \xi_{i, \, j}^2 \right)^{\frac{1}{2}} + \delta \cdot \left| \sum_{i = 1}^n \xi_{i, \, i} \right|, \end{equation}
for any matrix $\xi \in M_n(\R)$ and for almost every $x \in \Omega$. \end{definition}

\begin{remark}The Campanato condition implies the uniform ellipticity of $A$ (but, for $n > 2$, it is not equivalent) over $\Omega$. Indeed, let $\xi = \left( \eta_i \, \eta_j \right)_{i, \, j = 1, \, \dots, \, n}$ and substitute this in \eqref{condax}:
\begin{equation*} \left| \sum_{i = 1}^n \eta_i^2 - a(x) \cdot \sum_{i, \, j = 1}^n a_{i, \, j}(x) \, \eta_i \, \eta_j \right| \leq \gamma \cdot \left( \sum_{i, \, j = 1}^n \eta_i^2 \, \eta_j^2 \right)^{\frac{1}{2}} + \delta \cdot \left| \sum_{i = 1}^n \eta_i^2 \right|. \end{equation*}
It follows that
\begin{equation*}\left[ 1 - (\gamma + \delta) \right] \cdot \sum_{i = 1}^n \eta_i^2 \leq a(x) \cdot \sum_{i, \, j = 1}^n a_{i, \, j}(x) \, \eta_i \, \eta_j,\end{equation*}
and, if we let $\mu = \sup_{\Omega}a(x)$, it turns out that
\begin{equation*}\frac{\left[ 1 - (\gamma + \delta) \right]}{\mu} \cdot \sum_{i = 1}^n \eta_i^2 \leq \sum_{i, \, j = 1}^n a_{i, \, j}(x) \, \eta_i \, \eta_j ,\end{equation*}
which is the uniform ellipticity condition. \end{remark}

Observe that the Corder condition and the $A_x$ condition are actually \textbf{equivalent}. The proof is rather tedious, but simple, thus it is left to the reader.

\paragraph{Condition $A_x$.} In this brief paragraph, we want to motivate the Campanato condition and prove that the problem \eqref{problema} admits one and only one solution in that case.

\begin{theorem}\label{th:unicasol}Assume that the matrix $A = \left(a_{i, \, j} \right)_{i, \, j = 1, \, \dots, \, n}$ satisfies the Campanato condition. Then there exists one and only one solution of \eqref{problema}. \end{theorem}

\begin{proof}[Proof 1]Let us consider
\begin{equation} \Delta u(x) = \alpha \, f(x) + \Delta w(x) - \alpha \sumij \aij(x) \, D^{i, \, j} \, w(x), \label{problemapart} \end{equation}
and let us define the application $\mathcal{T} : H^2(\Omega) \cap H_0^1(\Omega) \to H^2(\Omega) \cap H_0^1(\Omega)$, which sends $w$ to the solution $u$ of the problem \eqref{problemapart}.

We now prove that $\mathcal{T}$ is a contraction of the space $H^2(\Omega) \cap H_0^1(\Omega)$, provided that the $A_x$ condition is satisfied. In order to ease the notation, we set
\begin{equation*} \Gamma_{i,  \, j}^{n} \, w(x) := \sumij \aij(x) \, D^{i, \, j} \, w(x). \end{equation*}
By \hyperref[rem:normeequ]{Remark \ref{rem:normeequ}} it follows that the $H^2(\Omega) \cap H_0^1(\Omega)$-norm is equivalent to
\begin{equation*} \| \Delta \, k(x) \|_{L^2(\Omega)} = \int_{\Omega} \left| \Delta \, k(x) \right|^2 \, \mathrm{d}x, \end{equation*}
therefore we can easily estimate the norm of the difference between two points
\begin{equation*} \begin{aligned} \| \mathcal{T}(w_1) - \mathcal{T}(w_2) \|_{H^{2, \, 2}(\Omega)}^2 & \leq \int_\Omega \left| \Delta u_1(x) - \Delta u_2(x) \right|^2 \, \mathrm{d}x = \\ & = \int_\Omega \left| \Delta w_1(x) - \alpha \, \Gamma_{i, \, j}^{n} \, w_1(x) - \left[ \Delta w_2(x) - \alpha \, \Gamma_{i, \, j}^{n} \, w_2(x) \right] \right|^2 \, \mathrm{d}x  = \\ & = \int_\Omega \left| \Delta \left(w_1(x) - w_2(x) \right) - \alpha \, \Gamma_{i, \, j}^n \, \left(w_1(x) - w_2(x) \right) \right|^2 \, \mathrm{d}x \stackrel{{\color{red}(\ast)}}{\leq} \\ & \stackrel{{\color{red}(\ast)}}{\leq} \int_\Omega \left\{ \gamma \cdot \left[ \Dij \left(w_1(x) - w_2(x) \right)^2 \right]^{\frac{1}{2}} + \delta \cdot \left| \Delta \left(w_1(x) - w_2(x) \right) \right| \right\}^2 \, \mathrm{d}x, \end{aligned} \end{equation*}
where the inequality ${\color{red}(\ast)}$ follows from a straightforward application of the Campanato condition. Notice now that for any $a, \, b \in \R$, it follows that
\begin{equation*} (\gamma \, a + \delta \, b)^2 \leq \gamma \, (\gamma + \delta) \, a^2 + \delta \, (\gamma + \delta) \, b^2, \end{equation*}
therefore
\begin{equation*} \begin{aligned}\dots &  \leq \int_\Omega \left[ \gamma \, (\gamma + \delta) \, D_{i, \, j} \, \left(w_1(x) - w_2(x) \right)^2 + \delta \, (\gamma + \delta) \, \left| \Delta \left(w_1(x) - w_2(x) \right) \right| \right]^2 \, \mathrm{d}x \stackrel{{\color{blue}(\ast)}}{\leq} \\ & \stackrel{{\color{blue}(\ast)}}{\leq} (\gamma + \delta)^2 \cdot \left\| \Delta \left(w_1(x) - w_2(x) \right) \right\|_{L^2(\Omega)}^2, \end{aligned} \end{equation*}
where ${\color{blue}(\ast)}$ is the \textbf{Miranda-Talenti} inequality (see \hyperref[MtLemma]{Lemma \ref{MtLemma}}).

We conclude that $\mathcal{T}$ is a contraction of the space $H^2(\Omega) \cap H^0(\Omega)$ into itself (again, by \hyperref[rem:normeequ]{Remark \ref{rem:normeequ}}), therefore it admits one and only one fixed point (i.e. the solution).
\end{proof}

\begin{lemma}[Miranda-Talenti]  \label{MtLemma} Let $\Omega$ be a convex subset of $\R^n$, and let $u \in H^2(\Omega) \cap H_0^1(\Omega)$ be a function. Then we have the following inequality:
\begin{equation*} \left\| u \right\|_{H^2(\Omega)} \leq \left\| \Delta \, u \right\|_{L^2(\Omega)}. \end{equation*}
\end{lemma}

\begin{proof} The inequality follows easily from the identity
\begin{equation*} \sum_{i, \, j = 1}^{n} (D_{i, \, j})^2 + \sum_{i, \, j= 1}^{n} \left[ D_{i, \, i}u \, D_{j, \, j} u - (D_{i, \, j})^2 \right] = \left(\Delta \, u \right)^2, \end{equation*}
which can be easily derived if one knows that the average curvature of $\partial \, \Omega$ is strictly negative (by convexity). \end{proof}

\begin{remark}\label{rem:normeequ} The Miranda-Talenti inequality implies that the $H^2(\Omega)$-norm and the $L^2(\Omega)$-norm of the laplacian are equivalents, but this is a more general fact holding also on non-convex subsets.

Indeed, it is obvious (by definition) that for any $u \in H^2(\Omega) \cap H_0^1(\Omega)$ there exists a constant $c > 0$ such that
\begin{equation*} \left\| \Delta \, u \right\|_{L^2(\Omega)} \leq c \cdot \left\| u \right\|_{H^2(\Omega)}. \end{equation*}
On the other hand, the divergence theorem and the Poincarè inequality immediately imply the opposite inequality, that is,
\begin{equation*} \begin{aligned} \int_{\Omega} \left| \nabla \, u(x) \right|^2 \, \mathrm{d}x & = \int_{\Omega} \nabla \, u(x) \cdot \nabla \, u(x) \, \mathrm{d}x \leq \\ & \leq \left( \int_{\Omega} \left| \Delta \, u (x) \right|^2 \, \mathrm{d}x \right)^{\frac{1}{2}} \cdot \left( \int_{\Omega} \left| u (x) \right|^2 \, \mathrm{d}x \right)^{\frac{1}{2}} \leq \\ & \leq c(\Omega) \cdot \left\| \Delta \, u \right\|_{L^2(\Omega)} \cdot \left\| u \right\|_{H^2(\Omega)}.\end{aligned} \end{equation*}
\end{remark}

We now give another proof of \hyperref[th:unicasol]{Theorem \ref{th:unicasol}} which relies on the near operator theory we have developed in the first section.

\begin{proof}[Proof 2] Let us set
\begin{equation*} B \, u(x) := \Delta \, u(x) \qquad \text{and} \qquad A \, u(x) := \sumij a_{i, \, j}(x) \, D^{i, \, j} \, u(x). \end{equation*}
The idea is to prove that the operator $A$ is a bijection from $H^2(\Omega) \cap H_0^1(\Omega)$ to $L^2(\Omega)$. We first notice that the following two facts hold true: \mbox{}
\begin{enumerate}[label=\textbf{(\arabic*)}]
\item There is an algebraic relation between $A$ and $B$.
\item The laplacian $\Delta$ is a bijection between $H^2(\Omega) \cap H_0^1(\Omega)$ to $L^2(\Omega)$.
\end{enumerate}
The proof of the first point is easy, and it is left to the reader. The second fact, on the other hand, will be proved later in the course.

The Campanato condition implies that $A$ is near $B$, therefore by \hyperref[bijectle]{Theorem \ref{bijectle}} we conclude that also $A$ is a bijection, which is exactly what we wanted to prove.\end{proof}

\begin{example}[Talenti] \label{talenti}We want to prove that, for $n > 2$, the problem \eqref{problema} is in general not well posed in $H^2(\Omega) \cap H_0^1(\Omega)$, provided that $f \in L^2(\Omega)$ and $a_{i, \, j} \in L^\infty(\Omega)$.

\vspace{2mm}
Let $\Omega = S(0, \, r) = \partial \, B(0, \, r)$ be the $n$-dimensional sphere, let $\lambda \in (0, \, 1)$ be a real number and consider the problem associated to the equation
\begin{equation*} \mathcal{A}(u) := \sumij a_{i, \, j}(x) \, D^{i, \, j} \, u(x) = 0. \end{equation*}
The coefficients of the matrix $A$ are given by
\begin{equation*} a_{i, \, j}(x) = \delta_{i, \, j} + b \, \frac{x_i \, x_j}{\|x\|^2}, \qquad \text{where} \qquad b = - 1 + \frac{n - 1}{1 - \lambda} \end{equation*}
for any $i, \, j = 1, \, \dots, \, n$.

We first prove that the matrix $A = \left(a_{i, \, j} \right)_{i, \, j = 1, \, \dots, \, n}$ is uniformly elliptic on $\Omega$. Indeed, it turns out that
\begin{equation*} \begin{aligned} \sumij \left( \delta_{i, \, j} + b \, \frac{x_i \, x_j}{\|x\|^2} \right) \, \xi_i \, \xi_j & = \sum_{i = 1}^n \xi_i^2 + \sumij b \, \frac{x_i \, x_j \, \xi_i \, \xi_j}{\|x\|^2} = \\ & = \|\xi\|^2 \, \left( 1 + b \, \sumij \frac{x_i \, x_j \, \xi_i \, \xi_j}{\|x\|^2 \, \|\xi\|^2} \right) \leq 1,\end{aligned} \end{equation*} 
since $b > - 1$ and by Cauchy-Schwartz inequality also
\begin{equation*}  \sumij \frac{x_i \, x_j \, \xi_i \, \xi_j}{\|x\|^2 \, \|\xi\|^2} = \frac{ (x, \, \xi)^4 }{\|x\|^2 \, \|\xi\|^2} \leq 1.\end{equation*}
The function $u(x) = \|x\|^\lambda$ is a solution of \eqref{problema} and its value at the boundary is exactly equal to $r^\lambda$. Similarly, the constant function $v(x) = r^\lambda$ also solves the problem \eqref{problema}, and the value at the boundary is the same.

We conclude that the problem associated to the operator $\mathcal{A}$ is \textbf{ill posed}. Moreover, it is not hard to check that the Cordes condition does not hold, coherently with the theory.
\end{example}

\paragraph{The $2$-dimensional case.} If $n = 2$, then the near operator theory allows us to prove the existence and the uniqueness of a solution of \eqref{problema}, provided that the coefficients are $L^\infty(\Omega)$ functions.

More precisely, the idea is to prove that the Campanato condition is, in fact, equivalent to the uniform ellipticity of the operator $\mathcal{A}$.

Suppose that $A(x)$ is uniformly elliptic and symmetric on $\Omega$. There are $\lambda_1(x), \, \lambda_2(x) \in \R$ eigenvalues such that $A(x)$ is similar to the diagonal matrix $\Gamma(x) := \mathrm{diag}\left(\lambda_1(x), \, \lambda_2(x) \right)$, and there is also a $\nu > 0$ such that $\lambda_1(x) \geq \nu$ and $\lambda_2(x) \geq \nu$ for almost every $x \in \Omega$.

It suffices to prove that there exists a function $a(x) \in L^\infty \left(\Omega \right)$ such that $a(x) \geq \omega > 0$ for almost every $x \in \Omega$ and, for any matrix $\xi := \{ \xi_{i, \, j} \}_{i, \, j = 1, \, 2}$, it turns out that
\begin{equation*} \begin{aligned} \left| \sum_{i = 1}^{2} \xi_{i, \, i} - a(x) \, \sum_{i, \, j = 1}^{2}  a_{i, \, j}(x) \, \xi_{i, \, j} \right| & = \left< \mathrm{I} - a(x) \, A(x), \, \xi \right> \leq \\ & \leq \left \| \mathrm{I} - a(x) \, A(x) \right\| \cdot \left\| \xi \right\| \leq \rho \, \|\xi\|, \end{aligned}\end{equation*}
for some $\rho \in (0, \, 1)$. Clearly
\begin{equation*} \left\| \mathrm{I} - a(x) \, A(x) \right\| \leq \rho \iff a^2(x) \, \left[ \lambda_1^2(x) + \lambda_2^2(x) \right] - 2 \, a(x) \, \left[ \lambda_1(x) + \lambda_2(x) \right] + 2 - \rho^2 \leq 0,\end{equation*}
and the right-hand side admits a real solution $a(x)$ if and only if the determinant is greater or equal than $0$, which  is possible if and only if
\begin{equation*} \frac{2 \, \lambda_1(x) \, \lambda_2(x)}{ \lambda_1^2(x) + \lambda_2^2(x) } \geq 1 - \rho^2. \end{equation*}
If we set $M = \max_{i = 1, \, 2} \, \sup_{x \in \Omega} \lambda_i(x)$, then we deduce that
\begin{equation*} \frac{2 \, \lambda_1(x) \, \lambda_2(x)}{ \lambda_1^2(x) + \lambda_2^2(x) } \geq \frac{\nu^2}{M^2}, \end{equation*}
therefore it is enough to choose $\rho$ and $a(x)$ in such a way that
\begin{equation*} \left[1 - \frac{\nu^2}{M^2} \right]^\frac{1}{2} \leq \rho < 1 \qquad \text{and} \qquad a(x) = \frac{ \lambda_1(x) + \lambda_2(x)}{ \lambda_1^2(x) + \lambda_2^2(x) }. \end{equation*}

\section{Regularity Conditions} 

In this section we want to prove that the problem \eqref{problema} is well-posed if we assume that the matrix $A(x) := \left\{ a_{i, \, j}(x) \right\}_{i, \, j = 1, \, \dots, \, n}$ is uniformly elliptic, and also that the coefficients $a_{i, \, j}(x)$ are $\alpha$-Hölder continuous, that is,
\begin{equation*} a_{i, \, j}(x) \in C^{0, \, \alpha}(\Omega). \end{equation*}

\begin{definition}[Sub(Super)-Solution] Let $u \in H^2(\Omega)$ be a function such that
\begin{equation} \label{subsup} \sum_{i, \, j= 1}^{n} a_{i, \, j}(x) \, D^{i, \, j} \, u(x) \geq 0 \qquad (\text{respectively, $\leq 0$}). \end{equation}
We say that $u$ is a \textit{sub-solution} (respectively, \textit{super-solution}) of the problem \eqref{problema} with no boundary condition.
\end{definition}

\begin{theorem}[Max. Principle] \label{th:mp} Let $u \in C^2 \left(\Omega \right) \cap C^0 \left( \overline{\Omega} \right)$ be a sub-solution of the problem \eqref{problema} with no boundary condition. If $A(x)$ is a uniformly elliptic matrix on $\Omega$, and the coefficients $a_{i, \, j}(x)$ are functions of class $C^0 \left( \overline{\Omega} \right)$, then
\begin{equation} \label{maxprc} \max_{x \in \overline{\Omega}} u(x) =  \max_{x \in \partial \, \Omega} u(x). \end{equation}
\end{theorem}

\begin{theorem}[Min. Principle] \label{th:minp} Let $u \in C^2 \left(\Omega \right) \cap C^0 \left( \overline{\Omega} \right)$ be a super-solution of the problem \eqref{problema} with no boundary condition. If $A(x)$ is a uniformly elliptic matrix on $\Omega$, and the coefficients $a_{i, \, j}(x)$ are functions of class $C^0 \left( \overline{\Omega} \right)$, then
\begin{equation} \label{minprc} \min_{x \in \overline{\Omega}} u(x) =  \min_{x \in \partial \, \Omega} u(x). \end{equation}
\end{theorem}

\begin{corollary}\label{ch:mp} Let $u$ be a solution of the problem
\begin{equation*} \sum_{i, \, j= 1}^{n} a_{i, \, j}(x) \, D^{i, \, j} \, v(x) = 0, \end{equation*}
and suppose that the same assumptions of \hyperref[th:mp]{Theorem \ref{th:mp}} are met. Then
\begin{equation*} \max_{x \in \overline{\Omega}} u(x) =  \max_{x \in \partial \, \Omega} u(x) \quad \text{and} \quad \min_{x \in \overline{\Omega}} u(x) =  \min_{x \in \partial \, \Omega} u(x). \end{equation*} \end{corollary}

\begin{corollary} \label{ch:uniqueness} If the same assumptions of \hyperref[th:mp]{Theorem \ref{th:mp}} are met, then the Dirichlet problem 
\begin{equation} \label{problema1} \begin{cases} \displaystyle\sum_{i, \, j = 1}^n a_{i, \, j}(x) \, D^{i, \, j} \, u(x) = f(x) & \text{for $x \in \Omega$} \\ \\ u(x) = g(x) & \text{for $x \in \partial \, \Omega$}. \end{cases} \end{equation}
admits at most one solution.\end{corollary}

\begin{proof} Suppose that $u_1(x)$ and $u_2(x)$ are both solution of class $C^2 \left(\Omega \right) \cap C^0 \left( \overline{\Omega} \right)$ of the problem \eqref{problema1}. If we set
\begin{equation*} W(x) := u_1(x) - u_2(x), \end{equation*}
then it's straightforward to prove that $W(x)$ is a solution of the Dirichlet problem
\begin{equation} \label{problema2} \begin{cases} \displaystyle\sum_{i, \, j = 1}^n a_{i, \, j}(x) \, D^{i, \, j} \, u(x) = 0 & \text{for $x \in \Omega$} \\ \\ u(x) = 0 & \text{for $x \in \partial \, \Omega$}. \end{cases} \end{equation}
Finally, \hyperref[ch:uniqueness]{Corollary \ref{ch:uniqueness}} proves that $W(x) \equiv 0$ is the unique solution of \eqref{problema2}, hence $u_1(x) \equiv u_2(x)$. \end{proof}

\begin{proof}[Proof of Maximum Principle] We divide the proof into two steps since the first one can be easily proved, and the second one follows from a simple approximation argument.

\paragraph{Step 1.} Suppose that for any $x \in \Omega$ it turns out that
\begin{equation} \label{st11} \sum_{i, \, j= 1}^n a_{i, \, j}(x) \, D^{i, \, j} \, u(x) > 0. \end{equation}
We argue by contradiction. Suppose that there exists a maximal point $x_0 \in \accentset{\circ}\Omega$ for $u$; then for any vector $v \in \R^n$, it turns out that
\begin{equation} \label{st12} \left( H_u(x_0) \, v, \, v \right)_{\R^n} \leq 0, \end{equation}
where $H_u(x_0)$ is the Hessian matrix of $u$ computed at the point $x_0$. By assumption, the matrix $A(x_0)$ is uniformly elliptic, hence there exists a real number $\nu > 0$ such that
\begin{equation} \label{st13} \left( A(x_0) \, v, \, v \right)_{\R^n} \geq \nu \, \|v\|_{\R^n}^2, \qquad \forall \, v \in \R^n. \end{equation}
It remains to prove that these relations yield to a contradiction. Indeed, the condition \eqref{st11} is equivalent to requiring that
\begin{equation*} \left( A(x), \, H(x) \right) > 0 \end{equation*}
at any point $x \in \R^n$ and, in particular, at $x = x_0$. On the other hand, both $A(x_0)$ and $H(x_0)$ are symmetric, thus there are $U$ and $V$ unitary matrices such that
\begin{equation*} U \, A(x_0) \, U^\ast =: \Lambda_A \quad \text{and} \quad V \, H(x_0) \, V^\ast =: \Lambda_H \end{equation*} 
are diagonal. Therefore, if we set $Q := U^\ast \, V$, then it is simple to prove that
\begin{equation*} \left( A(x), \, H(x) \right) = \left(\Lambda_A \, Q, \, Q \, \Lambda_H \right),\end{equation*}
and also that the latter scalar product is less or equal than zero, as a consequence of the fact that $A(x_0)$ is defined positive \eqref{st12} and $H(x_0)$ is semi-definite negative \eqref{st13}.

\paragraph{Step 2.} Suppose now that for any $x \in \Omega$ it turns out that
\begin{equation} \label{st21} \sum_{i, \, j= 1}^n a_{i, \, j}(x) \, D^{i, \, j} \, u(x) \geq 0. \end{equation}
We can define a small perturbation of the function $u(x)$, that is,
\begin{equation*} u_\epsilon(x) := u(x) + \epsilon \, \|x\|^2, \end{equation*}
so that $u_\epsilon(x)$ satisfies the condition \eqref{st11} for any $\epsilon > 0$. The first step proves that
\begin{equation*} \max_{x \in \overline{\Omega}} u_\epsilon(x) =  \max_{x \in \partial \, \Omega} u_\epsilon(x), \end{equation*}
hence the general case follows easily by taking the limit $\epsilon \to 0^+$.
\end{proof}

\begin{theorem}[Aleksandrov-Bakel'man-Pucci] Let
\begin{equation*}A(x, \, D) \, u := \sum_{i, \, j=1}^{n} a_{i, \, j}(x) \, D^{i, \, j} \, u(x) + \sum_{i = 1}^{n} b_i(x) \, D^i \, u(x) + c(x) \, u(x) \end{equation*}
be a differential operator such that the matrix $A(x)$ is uniformly elliptic on $\Omega$, and the coefficients $a_{i, \, j}(x)$, $b_i(x)$ and $c(x) \leq 0 $ are all of class $L^\infty(\Omega)$.

\vspace{1.8mm}
Let $u \in C^0 \left(\overline{\Omega}\right) \cap W_{\mathrm{loc}}^{2, \, n}(\Omega)$ be a function such that $A(x, \, D) \, u \geq f(x)$ for almost every $x \in \Omega$, and let
\begin{equation*} D(x) := \mathrm{det}\left(A(x)\right) \quad \text{and} \quad D^\ast(x) := \left[ D(x) \right]^{\frac{1}{n}}. \end{equation*}
Moreover, suppose that
\begin{equation*} \frac{f(x)}{D^\ast(x)} \in L^n(\Omega) \quad \text{and} \quad \frac{b(x)}{D^\ast(x)} \in L^n(\Omega). \end{equation*}
Then it turns out that
\begin{equation} \label{maxprch} \sup_{x \in \Omega} u(x) =  \sup_{x \in \partial \, \Omega} u^+(x) + c \, \left\| \frac{f}{D^\ast} \right\|_{L^n(\Omega)}, \end{equation}
where $u^+(x) := \max \{u(x), \, 0\}$ and the constant $c$ depends only on the dimension $n$ and on the following quantity: 
\begin{equation*} \left\| \frac{b(x)}{D^\ast(x)} \right\|_{L^n(\Omega)}. \end{equation*}\end{theorem}

The second step needed to prove that \eqref{problema} is well-posed, is the following a priori estimate.

\begin{theorem} \label{th:apriori} Let $u(x) \in H^{2, \, 2}(\Omega) \cap H_0^{1, \, 2}(\Omega)$ be a solution of the Dirichlet problem \eqref{problema}. Assume that $\partial \, \Omega$ is a manifold of class $C^3$, $f$ and $a_{i, \, j}$ belong to $C^{0, \, \alpha} \left(\overline{\Omega}\right)$. Then $D^{i, \, j} \, u \in C^{0, \, \alpha} \left(\overline{\Omega}\right)$, and there exists a positive constant $c > 0$ such that
\begin{equation} \label{eq:apriori} \sum_{i, \, j = 1}^n \left\| D^{i, \, j} \, u \right\|_{C^{0, \, \alpha} \left(\overline{\Omega}\right)}^2 \leq c \cdot \left( \|f\|_{C^{0, \, \alpha} \left(\overline{\Omega}\right)}^2 + \sum_{i, \, j = 1}^{n} \left\| D^{i, \, j} \, u \right\|_{\infty, \, \Omega}^2 \right). \end{equation}\end{theorem}

\begin{theorem} \label{th:aprioricho} Let $\Omega \subset \R^n$ be an open bounded subset, whose boundary $\partial \, \Omega$ is of class $C^3$. If $f \in C^{0, \, \alpha} \left(\overline{\Omega}\right)$, then there exists a positive constant
\begin{equation*} C := C \left(\Omega, \, \nu, \, \|a_{i, \, j}\|_{C^{0, \, \alpha} \left(\overline{\Omega}\right)} \right) > 0 \end{equation*}
such that, if $u \in H^{2, \, 2}(\Omega)$ is a solution of the Dirichlet problem \eqref{problema}, then
\begin{equation} \label{eq:aprioricho} \sum_{i, \, j = 1}^{n} \left\| D^{i, \, j} \, u \right\|_{0, \, \Omega}^2 \leq C \cdot \left( \|f\|_{C^{0, \, \alpha} \left(\overline{\Omega}\right)}^2 \right). \end{equation} \end{theorem}

\begin{proof}We argue by contradiction. If \eqref{eq:aprioricho} doesn't hold, then there exist: \mbox{}
\begin{enumerate}[label=\textbf{(\arabic*)}]
\item A uniformly bounded (by a constant $M > 0$) sequence of coefficients $\left(a_{i, \, j}^{(k)}(x) \right)_{k \in \N} \subset C^{0, \, \alpha} \left(\overline{\Omega}\right)$ such that the matrix
\begin{equation*} A^{(k)}(x) := \left\{ a_{i, \, j}^{(k)}(x) \right\}_{i, \, j = 1, \, \dots, \, n}\end{equation*}
is uniformly elliptic for any $k \in \N$, with the same constant $\nu>0$.
\item A sequence of functions $ \left( f_k(x) \right)_{k \in \N} \subset C^{0, \, \alpha} \left(\overline{\Omega}\right)$ such that
\begin{equation*} \| f_k \|_{C^{0, \, \alpha} \left(\overline{\Omega}\right)} \xrightarrow{k \to + \infty} 0. \end{equation*}
\item A sequence of solutions of the Dirichlet problems
\begin{equation} \label{problemaseq} \begin{cases} \displaystyle\sum_{i, \, j = 1}^n a_{i, \, j}(x) \, D^{i, \, j} \, u_k(x) = f_k(x) & \text{for $x \in \Omega$} \\ \\ u_k(x) = 0 & \text{for $x \in \partial \, \Omega$} \end{cases} \end{equation}
such that $\|u_k\|_{H^{2, \, 2}(\Omega)} = 1$ for all $k \in \N$.
\end{enumerate}
It follows from the Ascoli-Arzelà Theorem that there exists a subsequence $(k_n)_{n \in \N} \subset (k)_{k \in \N}$ such that $a_{i, \, j}^{k_n}(x)$ converges uniformly to $a_{i, \, j}(x)$ in $\overline{\Omega}$. From the a priori estimate \eqref{eq:apriori}, it turns out that for $k \in \N$ big enough
\begin{equation*} \sum_{i, \, j = 1}^n \left\| D^{i, \, j} \, u_k \right\|_{C^{0, \, \alpha} \left(\overline{\Omega}\right)}^2 \leq c \cdot \left( \|f_k\|_{C^{0, \, \alpha} \left(\overline{\Omega}\right)}^2 + \sum_{i, \, j = 1}^{n} \left\| D^{i, \, j} \, u_k \right\|_{\infty, \, \Omega}^2 \right) \leq c_1, \end{equation*}
since the norm of $f_k$ converges to zero, and the $H^{2, \, 2}(\Omega)$-norm of $u_k$ is constantly equal to one.

The sequence $(u_k)_{k \in \N}$ is equibounded in $C^{2, \, \alpha} \left(\overline{\Omega}\right)$, hence we may always extract a subsequence uniformly converging to $u$ (by Ascoli-Arzelà), and this implies the strong convergence in $H^{2, \, 2}(\Omega)$. Taking the limit for $k \to + \infty$ of the Dirichlet problem \eqref{problemaseq}, it turns out that $u$ is a solution of the problem \eqref{problema2}, hence $u \equiv 0$ on $\Omega$, that is, a contradiction since the $H^{2, \, 2}(\Omega)$-norm of $u$ is equal to $1$ (as a consequence of the uniform convergence).
\end{proof}

\begin{corollary} \label{apriori:cho} Under the assumptions of \hyperref[th:aprioricho]{Theorem \ref{th:aprioricho}}, the solution $u$ of the Dirichlet problem \eqref{problema} satisfies the following estimate:
\begin{equation}\label{eq:apriori:cho} \|u\|_{C^{2, \, \alpha} \left(\overline{\Omega} \right)} \leq c \, \|f\|_{C^{0, \, \alpha} \left( \overline{\Omega} \right)}. \end{equation} \end{corollary}

\begin{theorem}\label{th:aprioricho2} Let $\Omega \subset \R^n$ be an open bounded subset, whose boundary $\partial \, \Omega$ is of class $C^3$. Suppose that the coefficients $a_{i, \, j}(x)$ are of class $C^{0, \, \alpha}\left(\overline{\Omega}\right)$, and suppose that the matrix $A(x)$ is uniformly elliptic on $\Omega$.

Then, for any $f \in C^{0, \, \alpha} \left(\overline{\Omega}\right)$, the Dirichlet problem \eqref{problema} admits one and only one solution $u \in C^{2, \, \alpha} \left(\overline{\Omega}\right)$ satisfying the estimate \eqref{eq:apriori:cho}. \end{theorem}

\begin{proof}In order to prove this theorem, we use the continuity method introduced in the previous section. More precisely, let us consider the family of operators
\begin{equation*} A_t \, u := (1 - t) \, \nu \, \Delta \, u + t \, \sum_{i, \, j = 1}^n a_{i, \, j} \, D^{i, \, j} \, u, \qquad t \in [0, \, 1], \end{equation*}
and we notice that the coefficients $a_{i, \, j}^{(t)}(x) := (1-t) \, \nu \, \delta_{i, \, j} + t \, a_{i, \, j}(x)$ verifies the uniform ellipticity property, that is, the matrices
\begin{equation*} A^{(t)}(x) := \left\{ (1-t) \, \nu \, \delta_{i, \, j} + t \, a_{i, \, j}(x) \right\}_{i, \, j=1}^n\end{equation*}
are uniformly elliptic on $\Omega$. If we set $f_t := A_t \, u$, then it follows from \hyperref[apriori:cho]{Corollary \ref{apriori:cho}} that for any $u \in C^{2, \, \alpha} \left(\overline{\Omega} \right)$
\begin{equation*} \|u\|_{C^{2, \, \alpha} \left(\overline{\Omega} \right)}  \leq c \cdot \left\| A_t \, u \right\|_{C^{0, \, \alpha} \left(\overline{\Omega} \right)}\end{equation*}
since we can apply it to the Dirichlet problem
\begin{equation*} \begin{cases} \displaystyle\sum_{i, \, j = 1}^n (1-t) \, \nu \, \Delta \, u(x) + t \, \displaystyle\sum_{i, \, j = 1}^n a_{i, \, j}(x) \, D^{i, \, j} \, u(x) = f_t(x) & \text{for $x \in \Omega$} \\ \\ u(x) = 0 & \text{for $x \in \partial \, \Omega$}. \end{cases} \end{equation*}
It remains to check if the assumption of \hyperref[th:conm]{Theorem \ref{th:conm}} are met, where $\mfX = C^{2, \, \alpha} \left( \overline{\Omega} \right)$ and $\mathcal{B} = C^{0, \, \alpha} \left( \overline{\Omega} \right)$.

The first assumption holds true for $t = 0$ since the operator $\Delta$ is an isomorphism between $\mfX$ and $\mathcal{B}$; to check the second assumption, it suffices to observe that
\begin{equation*} \begin{aligned}\left\|A_t \, u - A_s \, u \right\|_{C^{0, \, \alpha} \left( \overline{\Omega} \right)} & = |t - s| \, \left\| \nu \, \Delta u - \sum_{i, \, j = 1}^n a_{i, \, j} \, D^{i, \, j} \, u \right\|_{C^{0, \, \alpha} \left( \overline{\Omega} \right)} \leq \\ & \leq c \, |t - s| \, \|u\|_{C^{2, \, \alpha} \left( \overline{\Omega} \right)}\,  {\color{red}\leq} \\ & {\color{red}\leq} \, c_1 \, c \, |t - s| \, \|A_t \, u\|_{C^{0, \, \alpha} \left( \overline{\Omega} \right)},\end{aligned} \end{equation*}
where the inequality ${\color{red}\leq}$ follows from \hyperref[apriori:cho]{Corollary \ref{apriori:cho}}.
\end{proof}

\section{Lax-Milgram Theorem}

\begin{theorem}[Lax-Milgram]\label{lax-milgram} Let $H$ be a Hilbert space, and let $a : H \times H \to \R$ be a function satisfying the following properties: \mbox{}
\begin{enumerate}[label=\textbf{(\arabic*)}]
\item $a(0, \, v) = 0$ for any $v \in H$.
\item $v \longmapsto a(u, \, v)$ is linear for any $u \in H$.
\item For any $u_1, \, u_2, \, v \in H$ it turns out that
\begin{equation*} \left| a(u_1, \, v) - a(u_2, \, v) \right| \leq M \, \|u_1 - u_2\|_H \, \|v\|_H. \end{equation*}
\item There exists $\nu > 0$ such that, for any $u_1, \, u_2 \in H$, the following inequality holds true:
\begin{equation*} a(u_1, \, u_1 - u_2) - a(u_2, \, u_1 - u_2) \geq \nu \, \|u_1 - u_2 \|_H^2. \end{equation*}
\end{enumerate}
Then for any $F \in H^\ast$ there exists one and only one $u \in H$ such that
\begin{equation*} a(u, \, v) = F(v), \qquad \forall \, v \in H. \end{equation*}
Moreover, the following estimate holds true:
\begin{equation*} c(\nu) \, \|u\|_H \leq \|F\|_{H^\ast}. \end{equation*}
\end{theorem}

\begin{proof} Let us consider the application
\begin{equation*} \mathcal{A} : H \to H^\ast, \qquad u \longmapsto \varphi_u \: : \: \varphi_u(v) := a(u, \, v). \end{equation*}
We want to prove that $\mathcal{A}$ is a bijection between $H$ and $H^\ast$, that is, for every $F \in H^\ast$ there is a unique $u \in H$ such that
\begin{equation*} \mathcal{A}(u)(v) = F(v) \qquad \forall \, v \in H \iff a(u,\, v) = F(v) \qquad \forall \, v \in H. \end{equation*}
By \hyperref[bijectle]{Theorem \ref{bijectle}} it suffices to prove that $A$ is near $\mathcal{J} : H \to H^\ast$ which is defined by
\begin{equation*} u \longmapsto \mathcal{J}_u \: : \: \mathcal{J}_u(v) := (u, \, v)_H, \end{equation*}
where $(\cdot, \, \cdot)_H$ is the scalar product. 

\paragraph{Step 1.} The operator $\mathcal{J}$ preserves the $H$-norm, that is, for any $u \in H$ it turns out that
\begin{equation*} \left\| \mathcal{J}(u) \right\|_{H^\ast} = \| u\|_H. \end{equation*}
The Riesz operator $\mathcal{R} : H^\ast \to H$ is defined by sending $F$ to the unique element $w_F \in H$ representing $F$, that is,
\begin{equation*} \mathcal{R}(F) = w_F \iff F(v) = \left(w, \, v \right)_H \qquad \forall \, v \in H. \end{equation*}
The Riesz operator $\mathcal{R}$ is an isometry (as it was proved in Riesz theorem), hence
\begin{equation*}\left(\mathcal{R} \left( \mathcal{A}(u) \right), \, v \right)_{H} = a(u, \, v) \qquad \text{and} \qquad \mathcal{R} = \mathcal{J}^{-1}, \end{equation*}
that is, $\mathcal{J}$ is an invertible operator.

\paragraph{Step 2.} The thesis of the theorem follows easily if we can prove that the nearness condition \eqref{eq:nearness} holds. By definition of the operators $\mathcal{A}$ and $\mathcal{J}$ it turns out that
\begin{equation*} \begin{aligned} \left\| \mathcal{J}(u_1) - \mathcal{J}(u_2) - \alpha \, \left[ \mathcal{A}(u_1) - \mathcal{A}(u_2) \right] \right\|_{H^\ast}^2 & = \left\| u_1 - u_2 - \alpha \left[ \mathcal{R} \left(\mathcal{A}(u_1) \right) - \mathcal{R} \left(\mathcal{A}(u_2) \right) \right] \right\|_{H}^2 = \\ & = \|u_1 - u_2\|_{H}^2 + \alpha^2 \, \left\|  \mathcal{R} \left(\mathcal{A}(u_1) \right) - \mathcal{R} \left(\mathcal{A}(u_2) \right)  \right\|_{H}^2 - \dots \\ & \dots - 2 \, \alpha \, \left(  \mathcal{R} \left(\mathcal{A}(u_1) \right) - \mathcal{R} \left(\mathcal{A}(u_2) \right), \, u_1 - u_2 \right)_H = \\ & = \|u_1 - u_2\|_{H}^2 + \alpha^2 \, \left\|  \mathcal{R} \left(\mathcal{A}(u_1) \right) - \mathcal{R} \left(\mathcal{A}(u_2) \right)  \right\|_{H}^2 - \dots \\ & \dots - 2 \, \alpha \, \left[ a(u_1, \, u_1 - u_2) - a(u_2, \, u_1 - u_2) \right] \, \, {\color{red} \leq}  \\ & {\color{red} \leq} \, \, \| u_1 - u_2 \|_H^2 + \alpha^2 \, M^2 \, \|u_1 - u_2\|_H^2 - 2 \, \alpha \, \nu \, \|u_1 - u_2\|_H^2 = \\ & = \left[1 + \alpha^2 \, M^2 - 2 \, \alpha \, \nu \right] \, \|u_1 - u_2\|_H^2 = \\ & = k \, \left\| \mathcal{J}(u_1) - \mathcal{J}(u_2) \right\|_{H^\ast}^2, \end{aligned}\end{equation*}
where the inequality ${\color{red} \leq}$ follows easily from properties \textbf{(2)} and \textbf{(3)} 
\end{proof}

\section{Garding Inequality}

Let us consider a differential elliptic operator in divergence form, that is,
\begin{equation*} A(x, \, D) \, u = \sum_{|\alpha| \leq m} \sum_{|\beta| \leq m} (-1)^{|\alpha|} \, D^\alpha \, \left(A_{\alpha, \, \beta}(x) \, D^\beta \, u \right). \end{equation*}
We denote by $A_0(x, \, D)$ the principal part of the operator, and we denote by $\left(\cdot, \, \cdot \right)_{\C^N}$ and $\| \cdot \|_{\C^N}$ respectively the scalar product and the norm in $\C^N$.

\begin{lemma} \label{lemma:gar1}  Let $A_0(D)$ be a differential operator in divergence form satisfying the weak Legendre-Hadamard condition \eqref{eq:l22}, and assume that the coefficients of $A$ are constants. Then for every $u \in H_0^m \left( \Omega; \; \R^N \right)$ it turns out that
\begin{equation} \label{eq:gar1} \int_{\Omega} \left[\sum_{|\alpha| = m} \sum_{|\beta| = m}  \left( A_{\alpha, \, \beta} D^\alpha \, u(x), \, D^\beta \, u(x) \right)  \right] \, \mathrm{d}x \geq c(\nu) \, \|u\|_{H^m \left(\Omega; \; \R^N \right)}. \end{equation}\end{lemma}

\begin{proof}We may always assume that $A_{\alpha, \, \beta} = A_{\alpha, \, \beta}^\ast$ since we can prove the thesis separately for the self-adjoint part and the anti self-adjoint part.

For any $u \in C_0^\infty(\R^n, \, \R^N)$ it turns out that
\begin{equation*} \begin{aligned}\int_{\R^n} \left[\sum_{|\alpha| = m} \sum_{|\beta| = m}  \left( A_{\alpha, \, \beta} D^\alpha \, u(x), \, D^\beta \, u(x) \right)  \right] \, \mathrm{d}x & =\sum_{|\alpha| = m} \sum_{|\beta| = m} \int_{\R^n}  \left( A_{\alpha, \, \beta} D^\alpha \, u(x), \, D^\beta \, u(x) \right)\, \mathrm{d}x  = \\ & = \int_{\R^n} \left[\sum_{|\alpha| = m} \sum_{|\beta| = m}  \left( A_{\alpha, \, \beta} \, \hat{u}(\xi), \, \overline{\hat{u}(\xi)} \right)_{\C^N}  \right] \, \xi^{\alpha + \beta} \, \mathrm{d}\xi \geq \\ & \geq \nu \, \int_{\R^n} \left( \|\mathfrak{Re}(\hat{u})(xi)\|^2 + \|\mathfrak{Im}(\hat{u})(xi)\|^2 \right) \, \|\xi\|^{2m} \, \mathrm{d}\xi = \\ & = \nu \, \int_{\R^n} \left(\hat{u}(\xi), \, \overline{\hat{u}(\xi)} \right)_{\C^N} \, \|\xi\|^{2m} \, \mathrm{d}\xi \geq \\ & \geq c(\nu) \, \int_{\R^n} \left[ \left( \hat{u}(\xi), \, \overline{\hat{u}(\xi)} \right)_{\C^N} \, \sum_{|\alpha| = m} \xi^{2 \alpha} \right] \, \mathrm{d}\xi = \\ & = c(\nu) \, \|u\|_{H^m(\R^n, \, R^N)}^2, \end{aligned} \end{equation*}
and this concludes the proof by density of the inclusion $C_0^\infty(\Omega, \, \R^N) \subset H_0^1(\Omega, \, \R^N)$.  \end{proof}

\begin{lemma} \label{lemma:gar2}  Let $A_0(D)$ be a differential operator in divergence form satisfying the weak Legendre-Hadamard condition \eqref{eq:l22}, and assume that the coefficients of $A$ are continuous on $\overline{\Omega}$. Then for every $u \in H_0^m \left( B(x_0, \, r); \; \R^N \right)$, with $x_0 \in \Omega$ and $r > 0$ small enough, it turns out that
\begin{equation} \label{eq:gar2} \int_{B(x_0, \, r)} \left[\sum_{|\alpha| = m} \sum_{|\beta| = m}  \left( A_{\alpha, \, \beta} D^\alpha \, u(x), \, D^\beta \, u(x) \right)  \right] \, \mathrm{d}x \geq \left[c(\nu) - \omega(r) \right] \, \|u\|_{H^m \left(\Omega; \; \R^N \right)}, \end{equation}
where $\omega(r)$ is the modulus of continuity defined by
\begin{equation*} \omega(r) := \sup \left\{ \left\|A_{\alpha, \, \beta}(x) - A_{\alpha, \, \beta}(y) \right\| \: \left| \: x, \, y \in \overline{\Omega}, \, \, \|x-y\| \leq r, \, \, |\alpha| = |\beta| = m \right. \right\}. \end{equation*}\end{lemma}

\begin{proof}Since
\begin{equation*} \begin{gathered}  \left| \int_{B(x_0, \, r)} \left[\sum_{|\alpha| = m} \sum_{|\beta| = m}  \left( \left[A_{\alpha, \, \beta}(x) - A_{\alpha, \, \beta}(x_0) \right] \, D^\alpha \, u(x), \, D^\beta \, u(x) \right)  \right]  \, \mathrm{d}x \right| \geq \dots \\ \dots \geq \omega(r) \, \int_{B(x_0, \, r)} \sum_{|\alpha = m} \|D^\alpha \, u(x) \|^2 \, \mathrm{d}x, \end{gathered} \end{equation*}
a simple application of \hyperref[lemma:gar1]{Lemma \ref{lemma:gar1}} gives us the estimate \eqref{eq:gar2}. \end{proof}

\begin{lemma} \label{lemma:gar3}  Let $\Omega \subset \R^n$ be a subset with a boundary locally Lipschitz. Let $A_0(D)$ be a differential operator in divergence form satisfying the weak Legendre-Hadamard condition \eqref{eq:l22}, and assume that the coefficients of $A$ are continuous on $\overline{\Omega}$. Then for every $u \in H_0^m \left( \Omega; \; \R^N \right)$ it turns out that
\begin{equation} \label{eq:gar3} \int_{\Omega} \left[\sum_{|\alpha| = m} \sum_{|\beta| = m}  \left( A_{\alpha, \, \beta} D^\alpha \, u(x), \, D^\beta \, u(x) \right)  \right] \, \mathrm{d}x \geq c(\nu) \, \left[ \|u\|_{H^m \left(\Omega; \; \R^N \right)} - \|u\|_{L^2 \left(\Omega; \; \R^N \right)} \right]. \end{equation}
\end{lemma}

\begin{proof} We divide the argument into many steps, in order to ease the notation.

\paragraph{Step 1.} ... \end{proof} %PRIMA O POI

\begin{lemma}\label{lemma:gar4} Let $A(x, \, D)$ be a differential operator in divergence form satisfying the weak Legendre-Hadamard condition \eqref{eq:l22}, and assume that: \mbox{}
\begin{enumerate}[label=\textbf{(\arabic*)}]
\item The coefficients of order $m$ are continuous on $\overline{\Omega}$.
\item The coefficients of order $<m$ are essentially bounded.
\end{enumerate}
Then for every $u \in H_0^m \left( \Omega; \; \R^N \right)$ it turns out that
\begin{equation} \label{eq:gar4} \int_{\Omega} \left[\sum_{|\alpha|\leq m} \sum_{|\beta| \leq m}  \left( A_{\alpha, \, \beta} D^\alpha \, u(x), \, D^\beta \, u(x) \right)  \right] \, \mathrm{d}x \geq c(\nu) \, \left[ \|u\|_{H^m \left(\Omega; \; \R^N \right)} - \|u\|_{L^2 \left(\Omega; \; \R^N \right)} \right]. \end{equation}
\end{lemma}

\begin{proof} The estimate \eqref{eq:gar4} is a straightforward consequence of the lemmas we have proved so far. More precisely, the terms of maximal order can be estimated via \hyperref[lemma:gar3]{Lemma \ref{lemma:gar3}}, while the other terms can be easily estimated with the interpolation inequality \eqref{eq:interp4}.  \end{proof}

\begin{theorem}[Lions] Let $X \subset Y \subset Z$ be Banach spaces such that the inclusion
\begin{equation*} X \hookrightarrow Y \end{equation*}
is continuous and compact, while the inclusion
\begin{equation*} Y \subset Z \end{equation*}
is continuous. For any $\epsilon > 0$ there exists a constant $c(\epsilon)>0$ such that for every $u \in X$ it turns out that
\begin{equation} \label{eq:interp}\|u\|_Y \leq \epsilon \, \|u\|_X + c(\epsilon) \, \|u\|_z. \end{equation} \end{theorem}

\begin{proof}We argue by contradiction. If \eqref{eq:interp} is not satisfied, then there exists $\epsilon > 0$ such that for any divergent sequence of real numbers $(c_n)_{n \in \N}$ there is a sequence $(u_n)_{n \in \N} \subset X$ such that
\begin{equation} \label{eq:interpfals}\|u_n\|_Y \geq \epsilon \, \|u_n\|_X + c_n \, \|u_n\|_z. \end{equation}
Let $v_n := u_n / \|u_n\|_X$. Then \eqref{eq:interpfals} can be written as follows
\begin{equation} \label{eq:interpfals2}\|v_n\|_Y \geq \epsilon + c_n \, \|v_n\|_z. \end{equation}
The $X$-norm of $v_n$ is equal to $1$ for each $n \in \N$, hence there exists a subsequence $\left(v_{n_k}\right)_{k \in \N} \subset Y$ (by compactness of the immersion) such that
\begin{equation*} v_{n_k} \xrightarrow{k \to + \infty} v_\infty \qquad \text{strongly in $Y$}. \end{equation*}
On the one hand, it follows from the continuity of the inclusion and \eqref{eq:interpfals2} that
\begin{equation*} \frac{c}{c_n} = \frac{c \, \|v_n\|_X}{c_n} \geq \frac{\|v_n\|_Y}{c_n} \geq \| v_n \|_Z, \end{equation*}
and by taking the limit as $n \to + \infty$ we infer that $\|v_n\|_{Z} \to 0$. This is absurd since by continuity of the inclusion $Y \subset Z$ it turns out that $v_\infty \equiv 0$, but \eqref{eq:interpfals2} implies also that $\|v_{n_k}\|_Y \geq \epsilon$ (in contradiction with the fact that $v_\infty$ is zero). \end{proof}

\begin{corollary}For any $\epsilon > 0$ and for any $u \in H_0^m \left( \Omega; \; \R^N \right)$ it turns out that
\begin{equation} \label{eq:interp4}\|u\|_{H^{m-1}\left( \Omega; \; \R^N \right)} \leq \epsilon \, \|u\|_{H^m\left( \Omega; \; \R^N \right)} + c(\epsilon) \, \|u\|_{L^2\left( \Omega; \; \R^N \right)}. \end{equation} \end{corollary}

\begin{proof}It is a straightforward consequence of the interpolation inequality \eqref{eq:interp} since the immersion
\begin{equation*} H^{m}\left( \Omega; \; \R^N \right) \hookrightarrow H^{m-1}\left( \Omega; \; \R^N \right) \end{equation*}
is compact by Rellich Theorem\footnote{Let $\Omega \subseteq \R^n$ be an open, bounded Lipschitz domain, and let $1 \leq p < n$. Set
\begin{equation*} p^\ast := \frac{np}{n - p}. \end{equation*}
Then the Sobolev space $W^{1, \, p} \left(\Omega; \; \R \right)$ is continuously embedded in the $L^p$-space $L^{p^\ast} \left(\Omega; \; \R \right)$ and is compactly embedded in $L^q\left(\Omega; \; \R \right)$ for every $1 \leq q < p^\ast$. In symbols,
\begin{equation*} W^{1, \, p}(\Omega) \hookrightarrow L^{p^\ast}(\Omega),\end{equation*}
and
\begin{equation*} W^{1, \, p}(\Omega) \subset \subset L^q(\Omega), \qquad \forall \, 1 \leq q < p^\ast. \end{equation*}}.\end{proof}

\section{Dirichlet problem for linear systems}

In this section, the primary goal is proving that the Dirichlet problem is well-posed in the linear systems setting.

\paragraph{Sobolev Spaces.} For any real number $s > 0$ the $-s$-Sobolev space is defined by setting
\begin{equation*} H^{-s}(\Omega) := \left( H_0^{s} (\Omega) \right)^\ast. \end{equation*}
The space of compactly supported infinitely derivable functions $\mathcal{D}(\Omega)$ is dense in $H_0^s(\Omega)$, hence
\begin{equation*} \mathcal{D}^\prime(\Omega) \supset H^{-s}(\Omega). \end{equation*}

\begin{theorem}\label{sob:char} Let $m > 0$ be an integer positive number. Every functional $f \in H^{-m}(\Omega)$ admits a non-unique representation as follows:
\begin{equation} \label{eq:repsob} f = \sum_{|\alpha| \leq m} D^\alpha \, f_\alpha, \qquad f_\alpha \in L^2(\Omega). \end{equation} \end{theorem}

\begin{proof}Let us consider, for $|\alpha| \leq m$, the linear application
\begin{equation*} \Psi_\alpha : H^m(\Omega) \ni u \longmapsto D^\alpha \, u \in L^2(\Omega). \end{equation*}
Clearly $\left(\Psi_\alpha \right)_{|\alpha| \leq m}$ establish an isomorphism between $H^m(\Omega)$ and a linear submanifold $V \subset \left[L^2(\Omega) \right]^h$, where $h$ is the number of all the derivatives $D^\alpha$ with $|\alpha| \leq m$.

Therefore to a linear continuous functional defined on $H^m(\Omega)$, corresponds a linear continuous functional defined on $V$. By Hahn-Banach theorem $L$ may be isometrically extended to a linear continuous functional $\tilde{L}$ defined over $\left[L^2(\Omega) \right]^h$. Moreover, it is easy to check that
\begin{equation*}\tilde{L} : \left[L^2(\Omega) \right]^h \to \R \implies \tilde{L} = \sum_{i = 1}^h L_i, \end{equation*}
where $L_i \in \left(L^2(\Omega) \right)^\ast$ for every index $i = 1, \, \dots, \, h$. By Riesz theorem, it turns out that
\begin{equation*} \tilde{L}(u) = \sum_{|\alpha| \leq m} \int_{\Omega} g_\alpha \, D^\alpha \, u, \qquad g_\alpha \in L^2(\Omega). \end{equation*}
In conclusion, since $L \in \left(H_0(\Omega) \right)^\ast$, the formula above is uniquely represented by the functional
\begin{equation*} L(\varphi) := \left< f, \, \varphi \right> \end{equation*}
defined on $\mathcal{D}(\Omega)$, and hence
\begin{equation*} f = \sum_{|\alpha| \leq m} D^\alpha \, f_\alpha, \qquad \text{where} \quad f_\alpha = (-1)^{|\alpha|} \, g_\alpha. \end{equation*} \end{proof}

\begin{theorem}[Global Existence, I] Let $A_0(D)$ be a differential operator in divergence form satisfying the weak Legendre-Hadamard condition \eqref{eq:l22}, and assume that the coefficients of $A$ are constants. Then for every $F \in H^{-m}(\Omega; \; \R^N)$ there exists one and only one solution in $H_0^m(\Omega; \; \R^N)$ of the system
\begin{equation*}A_0(D) \, u = F \end{equation*}
and the following estimate holds true:
\begin{equation} \label{eq:estisb} \|u\|_{H_0^m\left( \Omega; \; \R^N \right)} \leq c(\nu) \, \|F\|_{H^{-m}(\Omega; \; \R^N)}. \end{equation}\end{theorem}

\begin{remark}If we set
\begin{equation} \label{eq:sobn1} \|F\|_{-m, \, \Omega}^\ast := \inf \left\{ \sum_{j = 0}^m d_{\Omega}^{m-j} \, \left[ \int_\Omega \left( \sum_{|\alpha|=j} \|f_\alpha\|^2 \right) \, \mathrm{d}x \right]^{\frac{1}{2}} \right\}, \end{equation}
where the infimum is taken over the possible representations of $F$ of the form \eqref{eq:repsob}. One can easily show that \eqref{eq:sobn1} is a norm on $H^{-m}(\Omega)$, which is equivalent to the usual one:
\begin{equation} \label{eq:sobn2} \|F\|_{-m, \, \Omega} = \sup \left\{ \left| \left< F, \, \varphi \right> \right| \: \left| \: \varphi \in H_0^m(\Omega; \; \R^N), \, \, \|\varphi\|_{H_0^m(\Omega; \; \R^N)} = 1 \right. \right\}. \end{equation}\end{remark}

\begin{proof}Let us consider the bilinear form $a : H_0^m \left( \Omega; \; \R^N \right) \times H_0^m \left( \Omega; \; \R^N \right) \to \R$ defined by setting
\begin{equation*} a(u, \, v) := \int_{\Omega} \left[ \sum_{|\alpha| = |\beta| = m|} \left(A_{\alpha, \, \beta}(x) \, D^\alpha\, u(x), \, D^\beta \, v(x) \right) \right] \, \mathrm{d}x. \end{equation*}
The functional
\begin{equation*} L(v) := F(v), \qquad v \in H_0^m \left(\Omega; \; \R^N \right) \end{equation*}
is continuous, while $a$ is coercive on the product as a consequence of \hyperref[lemma:gar1]{Lemma \ref{lemma:gar1}}. By \hyperref[lax-milgram]{Lax-Milgram Theorem \ref{lax-milgram}} it turns out that there exists one and only one solution of the system
\begin{equation*} \int_{\Omega} \left[ \sum_{|\alpha| = |\beta| = m|} \left(A_{\alpha, \, \beta}(x) \, D^\alpha\, u(x), \, D^\beta \, v(x) \right) \right] \, \mathrm{d}x = F(v), \qquad \forall \, v \in H_0^m\left(\Omega; \; \R^N \right) \end{equation*}
along with the estimate \eqref{eq:estisb}. \end{proof}

\begin{theorem}[Global Existence, II] Let $A(D)$ be a differential operator in divergence form satisfying the weak Legendre-Hadamard condition \eqref{eq:l22}, and assume that: \mbox{}
\begin{enumerate}[label=\textbf{(\arabic*)}]
\item The coefficients of order $m$ are continuous on $\overline{\Omega}$.
\item The coefficients of order $<m$ are essentially bounded.
\end{enumerate}
Let $\gamma > 0$ be a big enough real number. Then for every $F \in H^{-m}(\Omega; \; \R^N)$ there exists one and only one solution in $H_0^m(\Omega; \; \R^N)$ of the system
\begin{equation*}A(x, \, D) \, u + \gamma \, u = F, \end{equation*}
and the following estimate holds true:
\begin{equation} \label{eq:estisb2} \|u\|_{H_0^m\left( \Omega; \; \R^N \right)} \leq c(\nu) \, \|F\|_{H^{-m}(\Omega; \; \R^N)}. \end{equation}
The same assertion holds true for any $\gamma > 0$, provided that the diameter $d_\Omega$ of the subset $\Omega$ is small enough.\end{theorem}

\begin{proof}Let us consider the bilinear form $a : H_0^m \left( \Omega; \; \R^N \right) \times H_0^m \left( \Omega; \; \R^N \right) \to \R$ defined by setting
\begin{equation*} a(u, \, v) := \int_{\Omega} \left[ \sum_{|\alpha| = |\beta| = m|} \left(A_{\alpha, \, \beta}(x) \, D^\alpha\, u(x), \, D^\beta \, v(x) \right) + \gamma \, u(x) \right] \, \mathrm{d}x. \end{equation*}
The functional
\begin{equation*} L(v) := F(v), \qquad v \in H_0^m \left(\Omega; \; \R^N \right) \end{equation*}
is continuous, hence the conclusion follows from \hyperref[lemma:gar4]{Lemma \ref{lemma:gar4}} for any $\gamma \geq c(\nu)$. Indeed, it suffices to observe that
\begin{equation*} \begin{aligned} a(u, \, v) & \geq c(\nu) \, \left[ \|u\|_{H^m(\Omega; \; \R^N)}^2 - \|u\|_{L^2(\Omega; \; \R^N)}^2 \right] + \gamma \, \|u\|_{L^2(\Omega; \; \R^N)}^{2} = \\ & = c(\nu) \|u\|_{H^m(\Omega; \; \R^N)}^2 + \left[\gamma - c(\nu) \right] \, \|u\|_{L^2(\Omega; \; \R^N)}^2 \geq \\ & \geq c(\nu) \, \|u\|_{H^m(\Omega; \; \R^N)}^2. \end{aligned} \end{equation*}
On the other hand, if $\gamma$ is any positive real number, it turns out that
\begin{equation*} \begin{aligned} a(u, \, v) & \geq c(\nu) \, \left[ \|u\|_{H^m(\Omega; \; \R^N)}^2 - \|u\|_{L^2(\Omega; \; \R^N)}^2 \right] + \gamma \, \|u\|_{L^2(\Omega; \; \R^N)}^{2} = \\ & = c(\nu) \|u\|_{H^m(\Omega; \; \R^N)}^2 + \left[\gamma - c(\nu) \right] \, \|u\|_{L^2(\Omega; \; \R^N)}^2 \geq \\ & \geq \left[ c(\nu) - d_\Omega \, \left(c(\nu) - \gamma \right) \right] \, \|u\|_{H^m \left( \Omega; \; \R^N \right)}^2 \, {\color{red} \geq} \\ & \, {\color{red} \geq} \,  c(\nu, \, \gamma, \, d_\Omega) \, \|u\|_{H^m \left( \Omega; \; \R^N \right)}^2, \end{aligned} \end{equation*}
where the {\color{red}red} inequality follows from the Poincaré inequality when $\gamma \leq c(\nu)$ and $d_\Omega$ is small enough.\end{proof}

\section{Global Existence via Special Operators}

Let us consider the Dirichlet problem
\begin{equation} \label{fpr} \begin{cases} u \in H_0^m\left(\Omega, \, \R^N \right) \\ \\ A(x, \, D) \, u(x) = F(x). \end{cases} \end{equation}
In the previous section, we have proved that for any $F \in H^{-m}\left(\Omega, \, \R^N \right)$ the problem \eqref{fpr} admits one and only one solution (i.e., it is well-posed) if the diameter of $\Omega$ is small enough. 

In this section, we show, via apriori estimates, that the same result holds true even when the diameter of $\Omega$ is small enough for the Poincaré inequality to hold. Let us set
\begin{equation*} \mathcal{P} \, u(x) = A(x, \, D) \, u(x). \end{equation*}

\begin{theorem}\label{th:dksdkksd}Let $\Omega \subset \R^n$ be an open bounded subset, and assume that $\partial \, \Omega$ is a boundary of class $C^m$. Suppose that $A_0(x, \, D)$ is an elliptic operator satisfying the Legendre-Hadamard condition \eqref{eq:l22}, with coefficients of class $C^1\left(\overline{\Omega} \right)$. Then the linear application
\begin{equation*} \mathcal{P} : H^{m+1}(\Omega, \, \R^N) \cap H_0^1(\Omega, \, \R^N) \to H^{1 - m} \left(\Omega, \, \R^N \right) \end{equation*}
has finite-dimensional kernel and closed rank. \end{theorem}

\begin{lemma}[Peetre] \label{lemma:peetre}Let $E, \, \Phi, \, G$ be three Banach spaces such that $E \subset \Phi$ is a compact immersion, and let $L$ be a continuous linear operator from $E$ to $G$. The following properties are equivalent: \mbox{}
\begin{enumerate}[label=\textbf{(\alph*)}]
\item The rank of $C$ is closed, and the kernel of $C$ is finite-dimensional.
\item There exists a positive constant $c > 0$ such that
\begin{equation} \label{eq:peetr} \|u\|_E \leq c \left( \|C \, u \|_G + \|u\|_{\Phi} \right), \qquad \forall \, u \in E. \end{equation}
\end{enumerate} \end{lemma}

\begin{proof}We divide the proof into two steps: second assertion implies the first assertion, and vice versa.

\paragraph{Step 1.} Let $E_0 := \mathrm{Ker}(C)$. The unitary ball in $E_0$ is compact in $\Phi$, thus, by \eqref{eq:peetr}, it is also compact in $E$. We conclude that $E_0$ is a finite-dimensional subspace of $E$\footnote{A Banach space such that every bounded subset is relatively sequentially compact, is necessarily a finite-dimensional space.}.

There exists $E_1 \subset E$ such that $E = E_0 \oplus E_1$, and the restriction of the operator $C$ on $E_1$ is injective by construction. We claim now that for any $u \in E_1$ it turns out that
\begin{equation} \label{eq:peetr1} \|u\|_E \leq c^\prime \, \|C \, u \|_G, \qquad \forall \, u \in E_1. \end{equation}
We argue by contradiction. If \eqref{eq:peetr1} does not hold true, then there exist a sequence $c_n^\prime \nearrow + \infty$ and a sequence $(u_n)_{n \in \N} \subset E_1$ such that
\begin{equation*} \|u_n\|_E > c_n^\prime \, \|C \, u_n \|_G. \end{equation*}
In particular, if we set $v_n := u_n / \|u_n\|_E$, the above inequality is equivalent to
\begin{equation} \label{eq:peetr2} \frac{1}{c_n^\prime} > \|C \, v_n \|_G. \end{equation}
The sequence $(v_n)_{n \in \N} \subset E$ is bounded, hence there exists a converging (in $\Phi$) subsequence $(v_{n_k})_{k \in \N}$ such that
\begin{equation*} v_{n_k} \xrightarrow{k \to + \infty}_{\Phi} v \in \Phi. \end{equation*}
It follows from \eqref{eq:peetr} and the above inequality that $(v_{n})_{n \in \N}$ is a Cauchy sequence in $E_1$, and thus it converges necessarily to the same $v$. Passing to the limit as $n \to + \infty$ in \eqref{eq:peetr2} immediately implies $v \equiv 0$, but this is absurd since by \eqref{eq:peetr} it turns out that
\begin{equation}1 \leq c \left( \|C \, v_n \|_G + \|v_n\|_{\Phi} \right), \qquad \forall \, n \in \N. \end{equation}
The claim \eqref{eq:peetr1} is now proved.

Let $(w_n)_{n \in \N} \subset \mathrm{Ran}(C)$ be a converging sequence, and let $w \in G$ be its limit. There exists a sequence $(u_n)_{n \in \N} \subset E$ such that $C \, u_n = w_n$ for every $n \in \N$; using the decomposition of $E$, it turns out that
\begin{equation*} u_n = v_n + z_n \implies C(u_n) = C(z_n) = w_n, \end{equation*}
and thus it follows from \eqref{eq:peetr1} that
\begin{equation*} \|z_n\|_E \leq c^\prime \, \|C \, z_n \|_G = c^\prime \, \|w_n\|_G.\end{equation*}
Consequently $(z_n)_{n \in \N} \subset E_1$ is a Cauchy sequence in $E$, hence it converges to $z \in E$. By the continuity of the operator $C$, we conclude that $C \, z = w$ (which is exactly what we wanted to prove).

\paragraph{Step 2.} Let $E = E_0 \oplus E_1$ as above. The restriction $C \, \big|_{E_1}$ is a closed map, thus by the closed graph theorem it follows that
\begin{equation}\label{eq:1111} \|v\|_E \leq c_1 \, \| C \, v \|_G, \qquad \forall \, v \in E_1. \end{equation}
On the other hand, for any $w \in E_0$ one can prove the inequality
\begin{equation} \label{eq:1112} \|w\|_E \leq c_2 \, \| w \|_\Phi. \end{equation}
We argue by contradiction. If \eqref{eq:1112} does not hold true, then there exist a sequence $(d_n) \subset \R$ increasingly converging to $+ \infty$, and a sequence $(w_n)_{n \in \N} \subset E_0$ such that
\begin{equation*} \|w_n\|_E \geq d_n \, \|w_n\|_{\Phi}. \end{equation*}
If we set $y_n := w_n / \|w_n\|_E$, then it turns out that
\begin{equation*} \frac{1}{d_n} \geq \|y_n\|_{\Phi}, \end{equation*}
hence $y_n \to 0$ in $\Phi$. This is absurd since the sequence $(y_n)_{n \in \N}$ belongs to a finite-dimensional subspace, hence it admits a converging subsequence to an  element $y$ of norm $1$. In conclusion, the inequality \eqref{eq:peetr} follows immediately from \eqref{eq:1111} and \eqref{eq:1112}.
\end{proof}

\begin{proof}[Proof of Theorem \ref{th:dksdkksd}] The thesis is an immediate corollary of \hyperref[lemma:peetre]{Peetre's Lemma \ref{lemma:peetre}}, where
\begin{equation*} \begin{aligned} & E = H^{m+1}(\Omega, \, \R^N) \cap H_0^1(\Omega, \, \R^N) \\ & \Phi = H_0^m \left(\Omega, \, \R^N \right) \\ & G = H^{1-m} \left(\Omega, \, \R^N \right) \\ & C \, u = \mathcal{P} \, u, \end{aligned} \end{equation*}
by noticing that \mbox{}
\begin{enumerate}[label=\textbf{(\arabic*)}]
\item the immersion $H^{m+1}\left(\Omega, \, \R^N \right)$ into $H^m\left(\Omega, \, \R^N \right)$ is compact by Rellich Theorem\footnote{Let $\Omega \subseteq \R^n$ be an open, bounded Lipschitz domain, and let $1 \leq p < n$. Set
\begin{equation*} p^\ast := \frac{np}{n - p}. \end{equation*}
Then the Sobolev space $W^{1, \, p} \left(\Omega; \; \R \right)$ is continuously embedded in the $L^p$-space $L^{p^\ast} \left(\Omega; \; \R \right)$ and is compactly embedded in $L^q\left(\Omega; \; \R \right)$ for every $1 \leq q < p^\ast$. In symbols,
\begin{equation*} W^{1, \, p}(\Omega) \hookrightarrow L^{p^\ast}(\Omega),\end{equation*}
and
\begin{equation*} W^{1, \, p}(\Omega) \subset \subset L^q(\Omega), \qquad \forall \, 1 \leq q < p^\ast. \end{equation*}};
\item the following estimate holds true:
\begin{equation*} \|u\|_{H^{m+1}(\Omega, \, \R^N)} \leq c \, \left( \|u\|_{H^m\left(\Omega, \, \R^N \right)} + \|F\|_{H^{1-m}\left(\Omega, \, \R^N \right)} \right). \end{equation*}
\end{enumerate}
\end{proof}