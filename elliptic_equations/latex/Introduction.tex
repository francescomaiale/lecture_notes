\chapter{Introduction To Elliptic Problems}

\begin{notation}In this course, we use the multi-index notation. If $\alpha = (\alpha_1, \, \dots, \, \alpha_n) \in \N^n$ is an $n$-tuple of natural numbers and $\xi \in \R^n$ is a real vector, then
\begin{equation*} \xi^\alpha := \xi_1^{\alpha_1} \, \dots \, \xi_n^{\alpha_n}. \end{equation*}
The length of a multi-index $\alpha$ is denoted by
\begin{equation*} |\alpha| := \alpha_1 + \dots + \alpha_n, \end{equation*}
and, as a consequence, the $\alpha$-derivative operator is given by
\begin{equation*} D^\alpha := D_{x_1}^{\alpha_1} \, \dots \, D_{x_n}^{\alpha_n}. \end{equation*}
\end{notation}

\section{Ellipticity Condition}

Let $a_\alpha : \R^n \to \C$ be collection of complex-valued functions, as $p$ ranges in a certain subset of $\N^n$, and let $\ell$ be a natural number. A linear differential operator of order $\ell$ is an operator of the form
\begin{equation} \label{eq:opdiff} A(x, \, D) \, u(x) := \sum_{\left| \alpha \right| \leq \ell} a_\alpha(x) \, D^\alpha \, u(x),\end{equation}
as $x$ ranges in $\Omega \subseteq \R^n$. The operator
\begin{equation}\label{eq:opdiffpri}A_0(x, \, D) \, u(x) := \sum_{\left| \alpha \right| = \ell} a_\alpha(x) \, D^\alpha \, u(x), \end{equation}
is the \textbf{principal part} of $A(x, \, D)$, while
\begin{equation} \label{eq:opdiffeq} \R^n \ni \xi \longmapsto A_0(x, \, \xi) := \sum_{\left| \alpha \right| = \ell} a_\alpha(x) \, \xi^\alpha \in \C \end{equation}
is the \textbf{characteristic polynomial} associated to $A_0(x, \, D)$.

\begin{definition}[Elliptic Operator] The linear differential operator $A(x, \, D)$ is an \textit{elliptic operator} at the point $x \in \R^n$ if its principal part is non-vanishing, that is,
\begin{equation*}A_0(x, \, \xi)\neq 0, \qquad \forall \, \xi \in \R^n \setminus \{0\}. \end{equation*}\end{definition}

\begin{theorem}Let $A(x, \, D)$ be an elliptic operator of order $\ell$ at the point $x \in X$. Then $\ell$ is even if either \mbox{}
\begin{enumerate}[label=\textbf{(\arabic*)}]
\item the coefficients $a_p(x)$ are real-valued functions; or
\item the dimension of the space is $n \geq 3$.
\end{enumerate} \end{theorem}

\begin{proof}Suppose that \textbf{(1)} holds true, and set $\xi = (\xi^\prime, \, \xi_n) \in \R^{n-1} \times \R$. Let us consider the polynomial
\begin{equation*} P(x, \, \xi^\prime, \, \xi_n) := A_0(x, \, \xi) = \sum_{|\alpha| = \ell} a_\alpha(x) \, \xi_1^{\alpha_1} \, \dots \, \xi_n^{\alpha_n}, \end{equation*}
and let us set
\begin{equation*} b_0(x) := P(x, \, 0, \, \dots, \, 0, \, 1) = a_{(0, \, \dots, \, 0, \, 1)}. \end{equation*}
The polynomial above can be decomposed as a sum of homogeneous terms in $\xi_n$ as follows
\begin{equation*} P(x, \, \xi^\prime, \, \xi_n) = b_0(x) \, \xi_n^\ell + b_1(x, \, \xi^\prime) \, \xi_{n}^{\ell - 1} + \dots + b_{\ell -1}(x, \, \xi^\prime) \, \xi_n + b_\ell(x, \, \xi^\prime), \end{equation*}
where the $b_i(x, \, \xi^\prime)$'s are polynomials of degree equal to $i$ with respect to the variable $\xi^\prime$, for any $i = 1, \, \dots, \, \ell$. 

Suppose that $b_0(x) > 0$ (the opposite case is symmetrical) and suppose also, by contradiction, that $\ell$ is an odd natural number. For any fixed $\xi^\prime \neq 0$, it turns out that
\begin{equation*} \lim_{\xi_n \to \pm \infty} P(x, \, \xi^\prime, \, \xi_n) = \pm \infty, \end{equation*}
that is there exists a point $\xi_n \in \R$ such that $P(x, \, \xi^\prime, \, \xi_n) = 0$, but this is absurd since we assumed $A(x, \, D)$ to be an elliptic operator.

\vspace{3mm}
\noindent Suppose that \textbf{(2)} holds true, and let $\xi_0^\prime \neq 0$ a fixed vector in $\R^{n-1}$. The polynomial
\begin{equation*} \R \ni \xi_n \longmapsto P(x, \, \xi_0^\prime, \, \xi_n) \end{equation*}
has non-real roots (because of the ellipticity condition), that is, roots whose imaginary part is nonzero $\mathfrak{Im}(\xi_n) \neq 0$.

Let $N^+(x, \, \xi_0^\prime)$ be the number of roots whose imaginary part is greater than $0$, and let $N^-(x, \, \xi_0^\prime)$ be the number of roots whose imaginary part is less than $0$.

Let $\Gamma$ be a path containing (see \hyperref[Path:Gamma]{Figure \ref{Path:Gamma}}) all the $N^+(x, \, \xi_0^\prime)$ roots with positive imaginary part, so that $P(x, \, \xi_0^\prime, \, \xi_n) \neq 0$ on $\Gamma$. By continuity of $P$ with respect to $\xi^\prime$, there exists a neighborhood $U^\prime$ of $\xi_0^\prime$ such that for any $\xi^\prime \in U^\prime$ it turns out that
\begin{equation*}\left| P(x, \, \xi_0^\prime, \, \xi_n) - P(x, \, \xi^\prime, \, \xi_n) \right| < \left| P(x, \, \xi_0^\prime, \, \xi_n) \right|, \qquad \forall \, \xi_n \in \Gamma.\end{equation*}
By Rouché theorem\footnote{\textbf{Rouché Theorem.} Let $K \subset G$ be a bounded region with continuous boundary $\partial \, K$. Two holomorphic functions $f, \, g \in \mathcal{H}(G)$ have the same number of roots (counting multiplicity) in $K$, if the strict inequality
\begin{equation*}\left|f(z) - g(z)\right| < \left|f(z)\right| + \left|g(z)\right|\end{equation*}
holds for every $z \in \partial \, K$.}, it follows that the polynomials $P(x, \, \xi_0^\prime, \, \xi_n)$ and $P(x, \, \xi^\prime, \, \xi_n)$ have the same number of roots inside $\Gamma$, i.e.,
\begin{equation*}N^+(x, \, \xi^\prime) = N^+(x, \, \xi_0^\prime), \qquad \forall \, \xi^\prime \in U^\prime.\end{equation*}
Since $N^+(x, \, \cdot)$ is a continuous integer-valued function, it is constant on connected components. It suffices to observe that $\{ \xi^\prime \in \R^{n-1} \} \setminus \{0\}$ is connected for any $n \geq 3$ to infer that
\begin{equation*}N^+(x, \, \xi^\prime) = N^+(x, \, - \xi^\prime) \quad \text{and} \quad N^-(x, \, \xi^\prime) = N^-(x, \, - \xi^\prime).\end{equation*}
The polynomial is homogeneous of degree $\ell$, hence
\begin{equation*}P(x, \, -\xi^\prime, \, -\xi_n) = (-1)^\ell \, P(x, \, \xi^\prime, \, \xi_n), \end{equation*}
and this implies that
\begin{equation*}N^+(x, \, \xi^\prime) = N^-(x, \, - \xi^\prime) \implies \ell = 2 \, N^+(x, \, \xi^\prime),\end{equation*}
i.e. $\ell$ is even.\end{proof}

\vspace{2.5mm}

\begin{figure}[h]
\centering
\includegraphics[width = 12cm, height = 6cm]{Images/EE1.png}
\label{Path:Gamma}
\caption{The path $\Gamma$ introduced in the proof.}
\end{figure} 

\vspace{2.5mm}

\begin{example}[Elliptic Operators] \mbox{}
\begin{enumerate}
\item[(1)]The \textbf{Cauchy-Riemann} operator is defined by
\begin{equation} \label{eq:c-r} A(D) := \frac{\partial}{\partial \, x_1} + \imath \, \frac{\partial}{\partial \, x_2}. \end{equation}
It's straightforward to see that this is an operator of order $\ell = 1$, whose coefficients are complex-valued constant functions. It is also elliptic, since
\begin{equation*}A_0(D) = \xi_1 + \imath \, \xi_2 = 0 \iff \left(\xi_1, \, \xi_2\right) = (0, \, 0). \end{equation*}
\item[(2)]The \textbf{Laplace} operator is defined by
\begin{equation} \label{eq:laplacian} \Delta := \sum_{i = 1}^{n} \frac{\partial^2}{\partial \, x_i^2} . \end{equation}
It's straightforward to see that this is an operator of even order $\ell = 2$, coherently with the fact that its coefficients are real-valued (constant) functions. It is elliptic since
\begin{equation*}A_0(D) = \| \xi \|^2 \iff \left(\xi_1, \,\dots, \, \xi_n \right) = (0, \, \dots, \, \, 0). \end{equation*}
\item[(3)]The \textbf{Bi-Laplace} operator is defined by
\begin{equation*} \Delta \, \Delta := \sum_{i = 1}^{n} \frac{\partial^2}{\partial \, x_i^2}  \left( \sum_{j = 1}^{n} \frac{\partial^2}{\partial \, x_j^2}  \right).\end{equation*}
It's straightforward to see that this is an operator of even order $\ell = 4$, coherently with the fact that its coefficients are real-valued (constant) functions. It is elliptic since
\begin{equation*}A_0(D) = \sum_{i=1}^{n} \xi_i^2 \, \| \xi \|^2 = \|\xi\|^4 = 0 \iff \left(\xi_1, \,\dots, \, \xi_n \right) = (0, \, \dots, \, \, 0). \end{equation*}
\end{enumerate}
\end{example}

\begin{definition}[Uniformly Elliptic Operator] An elliptic operator $A$ is called \textit{uniformly elliptic} on an open subset $\Omega \subset \R^n$ if there exists a positive constant $\nu > 0$ such that
\begin{equation*} \left|A_0(x, \, \xi) \right| \geq \nu \, \left| \xi \right|^\ell, \qquad \forall \, x \in \Omega, \, \, \forall \, \xi \in \R^n \end{equation*}
if the coefficients $a_\alpha(x)$ are complex-valued functions, and
\begin{equation*} A_0(x, \, \xi)  \geq \nu \, \left| \xi \right|^\ell, \qquad \forall \, x \in \Omega, \, \, \forall \, \xi \in \R^n \end{equation*}
if the coefficients $a_\alpha(x)$ are real-valued functions.\end{definition}

\begin{remark}If $A$ is an elliptic operator on a bounded subset $\Omega \subset \R^n$, and the coefficients $a_\alpha(x)$ are of class $C^0(\bar{\Omega})$, then $A$ is also uniformly elliptic. \end{remark}

In this course we shall be mainly concerned with two classes of elliptic operators, each of which has developed its own existence, regularity, etc... theory: \mbox{}
\begin{enumerate}[label={\color{red}\textbf{(\Roman*)}}]
\item The elliptic operator in \textbf{divergence} (or variational) form, i.e.
\begin{equation} \label{eq:var} A(x, \, D) \, u(x) = \sum_{|\beta| = m} \sum_{|\alpha| = m} D^\alpha \left( a_{\alpha, \, \beta}(x) \, D^\beta \, u(x) \right) . \end{equation}
\item The elliptic operator in \textbf{non-divergence} (or non-variational) form, i.e.
\begin{equation} \label{eq:nonvar} A(x, \, D) \, u(x) = \sum_{|\alpha| = 2 \, m} a_{\alpha}(x) \, D^\alpha \, u(x) . \end{equation}
\end{enumerate}

Clearly the two forms can be equivalent (up to terms of lower order) if the coefficients are sufficiently regular - for example of class $C^{m}\left(\bar{\Omega}\right)$.

However, in the general case, the two forms are not equivalent and the existence, uniqueness, regularity, etc... results we shall study latex on, are completely different - for example, elliptic problems in non-variational form rarely admits a solution.

The operators of the \textbf{second order} are particularly interesting because the coefficients identify a square matrix, thus the uniformly ellipticity condition (in the real case) can be rewritten in a more intuitive way (i.e. the matrix is positive-definite):
\begin{equation*} \sum_{i, \, j=1}^{n} a_{i, \, j}(x) \, \xi_i \, \xi_j  \geq \nu \, \left| \xi \right|^2, \qquad \forall \, x \in \Omega, \, \, \forall \, \xi \in \R^n. \end{equation*}
If an operator of the second order is in non-variational form \eqref{eq:nonvar}, then we may always assume - without loss of generality - that $(a_{i, \, j})_{i, \, j=1, \, \dots, \, n}$ is a symmetric matrix, provided that $u(x)$ is a regular function, e.g. of class $C^2$ or $H^2$.

In fact, if we set $a_{i, \, j} = a_{i, \, j}^+ + a_{i, \, j}^-$ (the sum of the symmetric and antisymmetric part), then
\begin{equation*} \begin{aligned} A(x, \, D) \, u(x) & = \sum_{i, \, j = 1}^{n} \left[a_{i, \, j}^+ + a_{i, \, j}^- \right] \, D^{i, \, j} \, u(x) = \\[1em] & = \sum_{i, \, j = 1}^{n} a_{i, \, j}^+ \, D^{i, \, j} \, u(x) +  \sum_{i, \, j = 1}^{n} a_{i, \, j}^- \, D^{i, \, j} \, u(x) = \\[1em] & =  \sum_{i, \, j = 1}^{n} a_{i, \, j}^+ \, D^{i, \, j} \, u(x) +  \sum_{i, \, j = 1}^{n} \frac{a_{i, \, j}(x) - a_{j, \, i}(x)}{2} \, D^{i, \, j} \, u(x) = \\[1em] & =  \sum_{i, \, j = 1}^{n} a_{i, \, j}^+ \, D^{i, \, j} \, u(x) +  \sum_{i, \, j = 1}^{n} \frac{a_{i, \, j}}{2} \, D^{i, \, j} \, u(x) - \sum_{i, \, j = 1}^{n} \frac{a_{j, \, i}}{2} \, D^{j, \, i} \, u(x) = \\[1em] & =  \sum_{i, \, j = 1}^{n} a_{i, \, j}^+ \, D^{i, \, j} \, u(x),\end{aligned}\end{equation*}
as a consequence of the Schwartz theorem\footnote{\textbf{Schwartz Theorem.} Let $f : \Omega \subset \R^n$ be a function of class $C^2$ defined on an open bounded subset $\Omega$. Then for any $i, \, j \in \{1, \, \dots, \, n\}$ it turns out that
\begin{equation*} \frac{\partial}{\partial \, x_i} \, \frac{\partial}{\partial \, x_j} \, f(x) = \frac{\partial}{\partial \, x_j} \, \frac{\partial}{\partial \, x_i} \, f(x). \end{equation*} }.

\begin{remark}If the coefficients of the matrix are real, then by symmetry its eigenvalues are necessarily real. On the other hand, the ellipticity condition implies that they are all positive.

Vice versa, any $n\times n$ matrix $B(x)$ whose eigenvalues at $x$ are real and positive defines an elliptic operator at the point $x$ in non-divergence form by setting
\begin{equation*}A(x, \, D) \, u(x) = \sum_{i, \, j = 1}^n B_{i, \, j}(x) \, D^{i, \, j} \, u(x) . \end{equation*}\end{remark}

\section{Elliptic Systems}

Let $\Omega \subseteq \R^n$ be any subset and assume that the coefficients are real-valued functions. We are particularly interested in elliptic system whose associated operator can be written as
\begin{equation} \label{eq:es} A(x, \, D) \, u(x) = \sum_{|\alpha| \leq m} \,  \sum_{ |\beta| \leq m} (-1)^{|\alpha|} \, D^\alpha \left[ A_{\alpha, \, \beta}(x) \, D^\beta \, u(x) \right], \end{equation}
where $u$ is a vector-valued function from $\Omega$ to $\R^N$ and $A_{\alpha, \, \beta} \in \mathcal{M}_{N \times N}\left( \R \right)$ is a real-valued square matrix for any admissible couple $(\alpha, \, \beta)$.

\begin{definition}[Legendre Condition] Let $A(x, \, D)$ be the operator \eqref{eq:es}. We say that $A$ satisfies the \textit{Legendre condition} (or, $A$ is \textit{strongly elliptic}) on $\Omega$ if there exists a positive constant $\nu > 0$ such that, for any system of vectors $\{ \xi^\alpha \}_{|\alpha| = m}$ of $\R^N$ and for any $x \in \Omega$,
\begin{equation} \label{eq:l1} \nu  \sum_{|\alpha| = m} \| \xi^\alpha \|_{\R^N}^{2} \leq \sum_{|\alpha| = m} \sum_{ |\beta| = m} \left( A_{\alpha, \, \beta}(x) \, \xi^\beta, \, \xi^\alpha \right)_{\R^N}, \end{equation}
where $(\cdot, \, \cdot)_{\R^N}$ denotes the standard scalar product in $\R^N$. \end{definition}

\begin{definition}[Legendre-Hadamard Condition] Let $A(x, \, D)$ be the operator \eqref{eq:es}. We say that $A$ satisfies the \textit{Legendre-Hadamard condition} (or, $A$ is \textit{elliptic}) if there exists a positive constant $\nu > 0$ such that, for any $\eta \in \R^N$, for any $\lambda \in \R^n$ and for any $x \in \Omega$,
\begin{equation} \label{eq:l2} \nu \, \left( \|\lambda\|_{\R^n} \right)^{2m} \, \| \eta \|_{\R^N}^2 \leq \sum_{|\alpha| = m} \sum_{ |\beta| = m} \left( A_{\alpha, \, \beta}(x) \, \eta, \, \eta \right)_{\R^N} \, \lambda^{\alpha} \, \lambda^{\beta}. \end{equation} \end{definition}

Clearly, any $A$ satisfying the Legendre condition also satisfies the Legendre-Hadamard condition. In fact, if we set $\xi_i^{\alpha} = \lambda^\alpha \, \eta_i$ for $i = 1, \, \dots, \, N$, then \eqref{eq:l1} implies \eqref{eq:l2}.

On the other hand, the opposite is generally \textbf{not} true. A simple counterexample ($N = n = 3$) is given by the operator of linear elasticity defined as
\begin{equation*} A(D) \, u(x) = a \, \Delta u(x) + (a + 2 \, \ell) \, \mathrm{grad} \left( \mathrm{div} \, u(x) \right), \end{equation*}
where $a, \, \ell > 0$ are (strictly) positive real constants. It's straightforward (but tedious) to prove that $A$ satisfies the condition \eqref{eq:l2}, but it doesn't satisfy the strong condition \eqref{eq:l1}.

\begin{remark} If an operator $A$ is elliptic, then the diagonal operators $A_{i, \, i}$ are also elliptic. In fact, if we set $\eta = e_i$, then the ellipticity condition \eqref{eq:l2} implies that
\begin{equation*} \nu \, \|\lambda\|_{\R^n}^{2m} \leq  \sum_{|\alpha| = m} \sum_{ |\beta| = m} A_{\alpha, \, \beta}^{i, \, i} (x) \, \lambda^{\alpha} \, \lambda^{\beta}. \end{equation*}
A similar argument proves that if $A$ is a strongly elliptic operator, then the diagonal operators are also strongly elliptic. In fact, if we set $\xi = e_i$ then \eqref{eq:l1} implies that
\begin{equation*} \nu \leq \sum_{|\alpha| = m} \sum_{ |\beta| = m}  A_{\alpha, \, \beta}^{i, \, i}(x) . \end{equation*} %?
\end{remark}

\begin{example}[Strong Ellipticity] Let $N = n = 2$. The operator
\begin{equation*} A = \begin{pmatrix} \Delta & 0 \\ 0 & \Delta \end{pmatrix} \end{equation*}
is strongly elliptic and, coherently with the argument above, its diagonal elements are also strongly elliptic. On the other hand, the operator
\begin{equation*} A = \begin{pmatrix} \Delta & \epsilon \, D_1^2 \\ \epsilon \, D_2^2 & \Delta \end{pmatrix} \end{equation*}
is strongly elliptic, but $\epsilon \, D_i^2$ are not even elliptic for any $|\epsilon| < 2$.\end{example}

We conclude this section by introducing the conditions \eqref{eq:l1} and \eqref{eq:l2} in the particular case of second-order operators.

\begin{definition}[Second-Order Legendre Condition] Let $A(x, \, D)$ be the operator \eqref{eq:es} of second-order. We say that $A$ satisfies the \textit{Legendre condition} (or $A$ is \textit{strongly elliptic}) if there exists a positive constant $\nu > 0$ such that, for any $\tau \in \R^{nN}$ and for any $x \in \Omega$, the following inequality holds true:
\begin{equation} \label{eq:l21} \nu  \sum_{i = 1}^{n} \| \tau^i \|_{\R^N}^{2} \leq \sum_{i, \, j = 1}^n \, \sum_{h, \,k = 1}^n A_{i, \, j}^{h, \, k}(x) \, \tau_i^h \, \tau_j^k. \end{equation} \end{definition}

\begin{definition}[Legendre-Hadamard Condition] Let $A(x, \, D)$ be the operator \eqref{eq:es} of second-order. We say that $A$ satisfies the \textit{Legendre-Hadamard condition} (or $A$ is \textit{elliptic}) if there exists a positive constant $\nu > 0$ such that, for any $\eta \in \R^N$, for any $\xi \in \R^n$ and for any $x \in \Omega$, the following inequality holds true:
\begin{equation} \label{eq:l22} \nu \, \| \xi \|_{\R^n}^{2} \, \| \eta \|_{\R^N}^2 \leq \sum_{i, \, j = 1}^n \, \sum_{h, \,k = 1}^n A_{i, \, j}^{h, \, k}(x) \, \xi_i \, \xi_j \, \eta_h \, \eta_k. \end{equation} \end{definition}

\section{Definitions of Solution}

The elliptic problems in non variational form (class {\color{red}(II)}) are, generally, in the following form:
\begin{equation*} A(x, \, D) \, u(x)= \sum_{|\alpha| \leq 2m} a_p(x) \, D^p \, u(x) = f(x). \end{equation*}
Therefore, if $f$ is a continuous function, we would like to find solution(s) of class $C^{2m}(\Omega)$, but unfortunately it is generally not possible.

In fact, $C^k(\Omega)$ is not a \textit{good} space to look into when studying the regularity (or even the existence) of solution(s) of an elliptic problem. The general method consists of two simple steps: \mbox{}
\begin{enumerate}[label=\textbf{(\alph*)}]
\item Prove the existence of solution(s) in bigger spaces, e.g. the Sobolev spaces $H^{2m, \, p}(\Omega)$ or the Campanato-Morrey spaces.
\item Use regularity theory and prove that the solution(s) in the weaker space are, actually, a lot more regular. As we shall see later on, we can generally prove that solution(s) are of class $C^{2m, \, \alpha}(\Omega)$ (i.e. functions of class $C^2$, whose second derivative is $\alpha$-Hölder continuous), provided that $f$ is also $\alpha$-Hölder.
\end{enumerate}

\begin{definition}[Classical Solution] Let $\Omega \subseteq \R^n$, let $f \in C^0\left(\bar{\Omega}; \; \R^N \right)$ and let $A_\alpha \in C^0\left(\bar{\Omega}; \; \R^N \times \R^N \right)$ for any $|\alpha| \leq 2m$. Then $u : \bar{\Omega} \to \R^N$ is a \textit{classical solution} of the elliptic problem
\begin{equation*}A(x, \, D) \, u(x) = f(x), \qquad x \in \Omega \end{equation*}
if $u \in C^{2m}\left(\Omega; \; \R^N \right) \cap C^0\left(\bar{\Omega}; \; \R^N \right)$ and $u$ satisfies the equation for any $x \in \bar{\Omega}$. \end{definition}

\begin{example}Let $\Omega = \left\{(x, \, y) \in \R^2 \: \left| \: x^2 + y^2 < r < 1 \right. \right\} \setminus \{(0, \, 0)\}$, let
\begin{equation*} u(x, \, y) = (x^2 - y^2) \cdot \sqrt{ - \log \, \sqrt{x^2 + y^2} }, \end{equation*}
and let
\begin{equation*} f(x, \, y) = \begin{cases} \frac{y^2 - x^2}{2x^2 + 2y^2} \cdot \left[ \frac{4}{\sqrt{ - \log \, \sqrt{x^2 + y^2} }} + \frac{1}{2 \, \sqrt{ - \log^6 \, \sqrt{x^2 + y^2} }} \right] & \text{if $(x, \, y) \neq (0, \, 0)$} \\ 0 & \text{if $(x, \, y) = (0, \, 0)$}. \end{cases} \end{equation*}
It's straightforward, but extremely tedious, to prove that $f \in C^0\left(\bar{\Omega} \right)$ and $u \in C^0\left(\bar{\Omega} \right) \cap C^\infty \left( \bar{\Omega} \setminus \{(0, \, 0)\} \right)$.

Surprisingly, it turns out that $u$ is not a classical solution of the elliptic problem $\Delta u(x, \, y) = f(x, \, y)$, since
\begin{equation*} \lim_{(x, \, y) \to (0, \, 0)} \frac{ \partial^2 \, u}{\partial \, x^2} (x, \, y) = + \infty.\end{equation*}\end{example}

\begin{definition}[Strong Solution] Let $\Omega \subseteq \R^n$, let $f \in L^p\left(\Omega; \; \R^N \right)$ for $p > 1$ and let $A_\alpha \in L^\infty \left(\Omega; \; \R^N \times \R^N \right)$ for any $|\alpha| \leq 2m$. Then $u : \bar{\Omega} \to \R^N$ is a \textit{strong solution} of the problem
\begin{equation*}A(x, \, D) \, u(x) = f(x), \qquad x \in \Omega \end{equation*}
if $u \in H^{2m, \, p}\left(\Omega; \; \R^N \right)$, and it solves the elliptic problem for almost every $x \in \bar{\Omega}$.\end{definition}

\begin{definition}[Weak Solution] Let $\Omega \subseteq \R^n$, let $f_\beta \in L^p\left(\Omega; \; \R^N\right)$ for $p > 1$ and let $A_\alpha \in L^\infty\left(\Omega; \; \R^N \times \R^N \right)$ for any $|\alpha|, \, |\beta| \leq m$. Then $u : \bar{\Omega} \to \R^N$ is a \textit{weak solution} of the problem
\begin{equation*} \sum_{|\alpha| \leq m} \sum_{|\beta| \leq m}  (-1)^{|\beta|} \, D^\beta \left( A_{\alpha, \, \beta}(x) \, D^\alpha \, u(x) \right) = \sum_{|\beta| \leq m} (-1)^{|\beta|} \, D^\beta \, f_\beta(x) \end{equation*}
if $u \in H^{m, \, p}\left(\Omega; \; \R^N \right)$ and, for any $\varphi \in H^{m, \, p^\prime}\left(\Omega; \; \R^N \right)$,
\begin{equation*} \int_{\Omega} \left[ \sum_{|\alpha|, \, |\beta| \leq m} \left( A_{\alpha, \, \beta}(x) \, D^\alpha \, u(x), \, D^\beta \varphi(x) \right)_{\R^N} \right]\, \mathrm{d}x = \int_{\Omega} \left[ \sum_{|\beta| \leq m} \left( f_\beta(x), \, D^\beta \, \varphi(x) \right)_{\R^N} \right] \, \mathrm{d}x.\end{equation*}\end{definition}

\section{Boundary Problems}

\paragraph{Posedness} In this section we introduce the notion of \textit{boundary} elliptic problem and the notion of \textit{well-posed} boundary elliptic problem.

Let $\Omega \subseteq \R^n$ be an open and bounded subset of $\R^n$, let $A(x, \, D)$ be an elliptic operator and let $B(x, \, D)$ be a linear boundary operator. The system
\begin{equation*} \label{eq:sys} \begin{cases} A(x, \, D) \, u(x) = f(x) & \text{a.e. $x \in \Omega$} \\ B(x, \,D) \, u(x) = g(x) & \text{$x \in \partial \, \Omega$} \end{cases} \end{equation*}
is called a \textit{boundary} elliptic problem.

\begin{definition}[Well-Posed] The problem \eqref{eq:sys} is \textit{well-posed} (in sense of Hadamard) if there exists \textbf{one and only one} solution that depends \textbf{continuously} on the initial data.\end{definition}

\begin{example}Let $\Omega = \left\{ (t, \, x) \:  \left| \: t > 0, \, \, x \in \R \right. \right\}$ and let
\begin{equation*} \Delta \, u = \frac{\partial^2 \, u}{\partial \, t^2} + \frac{\partial^2 \, u}{\partial \, x^2}. \end{equation*}
The problem
\begin{equation*} \begin{cases} \Delta u(t, \, x) = 0 & \text{$(t, \, x) \in \Omega$} \\ u(0, \, x) = \Phi(x) & \text{$x \in \R$} \\ u_t(0, \, x) = \Psi(x) & \text{$x \in \R$}\end{cases} \end{equation*}
is not well posed for any choice of $\Phi$ and $\Psi$. In fact, if we let
\begin{equation*} \begin{cases} \Phi_n(x) = \mathrm{e}^{-\sqrt{n}} \, \sin (nx) \\ \Psi_n(x) = \mathrm{e}^{-\sqrt{n}} \, n \, \sin (nx),\end{cases} \end{equation*}
there there's one and only one classical solution for any $n \in \N$, given by
\begin{equation*} u_n(x) = \mathrm{e}^{-\sqrt{n} + nt} \, \sin (nx).\end{equation*}
On the other hand, for any $\epsilon > 0$, there exists $n_\epsilon \in \N$ such that
\begin{equation*} \sup_{x \in \R} |\Phi_n(x)| \leq \epsilon \quad \text{and} \quad \sup_{x \in \R} |\Psi_n(x)| \leq \epsilon,\end{equation*}
for any $n \geq n_\epsilon$. The reason is simple: the limit as $n \to + \infty$ of both $\Phi_n(x)$ and $\Phi_n(x)$ is $0$, uniformly with respect to $x$ (therefore we can take the supremum and the limit is still zero).

The solution doesn't depend continuously on the initial data. In fact, for any $t_0 > 0$ fixed, it turns out that
\begin{equation*} \lim_{n \to + \infty} \sup_{x \in \R} |u_n(t_0, \, x)| \sim \lim_{n \to + \infty} \mathrm{e}^{n - \sqrt{n}} = + \infty. \end{equation*}
\end{example}

More precisely, let $U$ and $W$ be topological vector spaces. Let $u \in U$ be a solution of a given problem and let $f \in W$ be the given data; the problem is \textbf{well posed} if: \mbox{}
\begin{enumerate}[label=\textbf{\arabic*})]
\item For any $f$ there exists a solution $u$ of the problem.
\item The solution is unique.
\item The solution depends continuously on $f$.
\end{enumerate}

If these condition are not satisfied, the problem is said to be \textbf{ill-posed}. Notice that even problems with a physical meaning may be ill-posed (the reader should think about a string equilibrium problem, for example).

\paragraph{Boundary Operators} In this paragraph we are interested in introducing general conditions on the boundary operator, so that the problem becomes well-posed.

Let $\Omega$ be a subset of $\R^n$ with a sufficiently regular boundary, let $A$ be an operator of order $2m$ and let $B_j(x, \, D)$, for $j = 0, \, \dots, \, k- 1$ be boundary operators defined by
\begin{equation} B_j(x, \, D) \, \varphi(x) = \sum_{ |\alpha| \leq m_j } b_{j, \, \alpha}(x) \, D^\alpha \, \varphi(x), \qquad x \in \partial \, \Omega. \end{equation}
More precisely $B_j(x, \, D)$ represent the following operator
\begin{equation} \varphi \longmapsto \sum_{ |\alpha| \leq m_j } b_{j, \, \alpha}(x) \, \gamma_0 \left(D^\alpha \, \varphi (x)\right),\end{equation}
where $\varphi$ is a function defined on the close $\bar{\Omega}$ and $\gamma_0(\cdot)$ is the \textbf{trace} map in the classical sense or, if needed, in the Sobolev space sense. Let us consider the problem
\begin{equation*} \label{eq:sys} \begin{cases} A(x, \, D) \, u(x) = f(x) & \text{a.e. $x \in \Omega$} \\ B(x, \,D) \, u(x) = g(x) & \text{$x \in \partial \, \Omega$}. \end{cases} \end{equation*}

\begin{definition}[Normal] A system of operators $\left\{B_j(x, \, D)\right\}_{j = 0, \, \dots, \, k-1}$ is \textit{normal} if the following properties are satisfied: \mbox{}
\begin{enumerate}[label=\textbf{\arabic*})]
\item The polynomial $\sum_{|\alpha| = m_j} b_{j, \, \alpha} \, \xi^\alpha \neq 0$ for any $\xi \neq 0$ and it is orthogonal to $\partial \, \Omega$.
\item The orders are different, that is, $m_i \neq m_j$ for any $i \neq j$.
\end{enumerate}
\end{definition}

Moreover, if we assume that $k = m$, then we can give another important definition.

\begin{definition}[Covering] A system of operators $\left\{B_j(x, \, D)\right\}_{j = 0, \, \dots, \, m-1}$ is a \textit{covering} of the operator $A$ on $\partial \, \Omega$ if for any $x \in \partial \, \Omega$, for any $\xi \in \R^n \setminus \{0\}$ tangent to $\partial \, \Omega$ at $x$ and any $\xi^\prime \in \R^n \setminus \{0\}$ orthogonal to $\partial \, \Omega$ at $x$, the polynomials of the complex variable $\tau$
\begin{equation*}\sum_{|\alpha| = m_j} b_{j, \, \alpha} \, \left( \xi + \tau \, \xi^\prime \right)^\alpha, \qquad j = 0, \, \dots, \, m - 1\end{equation*}
are linearly independent modulo the polynomial
\begin{equation*}\prod_{i = 1}^{m} \left( \tau - \tau_i^+(x, \, \xi, \, \xi^\prime) \right),\end{equation*}
where $\tau_i^+(x, \, \xi, \, \xi^\prime)$ are the roots with positive imaginary part of the polynomial $A_0(x, \, \xi + \tau \, \xi^\prime)$.
\end{definition}

In particular, some of the most important assumption on \eqref{eq:sys} are the following ones: \mbox{}
\begin{enumerate}[label=\textbf{\arabic*})]
\item The operator $A$ is uniformly elliptic in $\bar{\Omega}$, with coefficients in a suitable functional space.
\item The boundary operators $B_j$ are $m$, with coefficients in a suitable functional space.
\item The system $\left\{B_j(x, \, D)\right\}_{j = 0, \, \dots, \, m-1}$ is \textit{normal} on $\partial \, \Omega$.
\item The system $\left\{B_j(x, \, D)\right\}_{j = 0, \, \dots, \, m-1}$ \textit{covers} the operator $A$ on $\partial \, \Omega$.
\item The order of $B_j$ is less or equal to $2 \, m - 1$, for any $j$.
\end{enumerate}

\begin{example} The most common example of boundary operators satisfying the conditions above is given by the system of the Dirichlet conditions, that is,
\begin{equation*} B_j = \gamma_j = \frac{\partial^j}{\partial \, \nu^j}, \qquad j = 0, \, \dots, \, m -1 \end{equation*}
where $\nu$ is the normal to $\partial \, \Omega$ pointing toward the interior part of $\Omega$. The problem
\begin{equation*} \begin{cases}A(x, \, D)u = f & \text{$x \in \Omega$} \\ \gamma_0 \, u = \varphi_0 & \text{$x \in \partial \, \Omega$} \\ \dots \\ \dots \\ \gamma_{m-1}\, u = \varphi_{m-1} & \text{$x \in \partial \, \Omega$}, \end{cases} \end{equation*}
is called \textbf{Dirichlet problem}.\end{example}

\begin{example} Another common example of boundary operators satisfying the conditions above is given by the system of the Navier conditions, that is,
\begin{equation*} B_j(x, \, D) \, u = \Delta_{j-1} \, u, \qquad j = 1, \, \dots, \, m \end{equation*}
so that $m_j = 2(j - 1)$. The associated problem is
\begin{equation*} \begin{cases}A(x, \, D)u = f & \text{$x \in \Omega$} \\ u = \varphi_0 & \text{$x \in \partial \, \Omega$} \\ \dots \\ \dots \\ \Delta_{m-1}\, u = \varphi_{m-1} & \text{$x \in \partial \, \Omega$}. \end{cases} \end{equation*}
is called \textbf{Navier problem}.\end{example}

\begin{remark}Let $\Omega$ be a subset of $\R^n$ with a sufficiently regular boundary. A function $u \in H_0^m(\Omega)$ if and only if $u \in H^m(\Omega)$ and $\gamma_0 = \dots = \gamma_{m-1} = 0$ on $\partial \, \Omega$. Therefore the Dirichlet problem can be equivalently written as
\begin{equation*} \begin{cases}A(x, \, D)u = f & \text{$x \in \Omega$} \\ u \in H_0^m(\Omega). \end{cases} \end{equation*} \end{remark}

\section{The Laplacian Operator and The Calculus of Variations}

Let $\Omega \subset \R^n$ be an open subset and let $\varphi : \partial \, \Omega \to \R$ be a given function. The \textit{energy functional} associated to the laplacian operator is defined by setting
\begin{equation} \label{eq:laspdl} E(u) = \frac{1}{2} \int_\Omega \left| \nabla \, u(x) \right|^2 \, \mathrm{d}x - \int_\Omega f(x) \, u(x) \, \mathrm{d}x = \mathcal{E}_e(u) - \mathcal{E}_p(u), \end{equation}
where $\mathcal{E}_p$ is the internal elastic energy, while $\mathcal{E}_p$ is the exogenous potential energy. We also require that $u$ is an element of the \textit{eligible class}, given by
\begin{equation*} \mathcal{A} = \left\{ u : \Omega \to \R \: \left| \: \text{$u(x) = \varphi(x)$ for any $x \in \partial \, \Omega$ and $E(u) < + \infty$} \right. \right\}. \end{equation*}

\begin{theorem}Let $\mathcal{A} \neq \emptyset$. If there exists $\underline{u} \in \mathcal{A}$ such that $\underline{u}$ is a minimizer of the energy $E$, that is, for any $u \in \mathcal{A}$
\begin{equation*} E(\underline{u}) \leq E(u), \end{equation*}
then $\underline{u}$ is a solution to the Dirichlet problem
\begin{equation} \label{dir} \begin{cases} - \Delta \, \underline{u}(x) = f(x) & \text{$x \in \Omega$} \\ \underline{u}(x) = \varphi(x) & \text{$x \in \partial \, \Omega$}. \end{cases} \end{equation} \end{theorem}

Before we get to the proof, some questions arise naturally (which we solve during the proof): \mbox{}
\begin{enumerate}[label=\textbf{\arabic*})]
\item What assumptions do we need on $\varphi$ and $\Omega$ to assure that $\mathcal{A} \neq \emptyset$?
\item There exists at least a minimum point of $E$ in $\mathcal{A}$?
\end{enumerate}

\begin{proof} Let $\mathcal{A}_0$ be the vector space associated to the solution space, that is, 
\begin{equation*} \mathcal{A}_0 = \left\{ u : \Omega \to \R \: \left| \: \text{$u(x) = 0$ for any $x \in \partial \, \Omega$ and $E(u) < + \infty$} \right. \right\}. \end{equation*}
Let $\underline{u} \in \mathcal{A}$ and $v \in \mathcal{A}_0$ and let
\begin{equation*} F(t) := \frac{1}{2} \int_\Omega \left| \nabla \, (\underline{u} + t \, v)(x) \right|^2 \, \mathrm{d}x - \int_\Omega f(x) \, \left(\underline{u}(x) + t \, v(x) \right) \, \mathrm{d}x, \end{equation*}
then, by assumption, $t = 0$ is a local minimum of $F$. If we compute the derivative of $F$ in $t = 0$ and set it equal to $0$, it turns out that
\begin{equation*}\int_{\Omega} \left[ \nabla \, \underline{u}(x) \cdot \nabla \, v(x) - f(x) \, v(x) \right] \, \mathrm{d}x = 0, \qquad \forall \, v \in \mathcal{A}_0.\end{equation*}
The point $\underline{u}$ is thus a \textbf{weak} solution of the Dirichlet problem \eqref{dir}. To prove that it is a strong solution we need to use the Gauss-Green formula, and this can only be done if we assume $u$ and $f$ to be sufficiently regular (e.g. $f \in L^2(\Omega)$ and $u \in H^2(\Omega)$ would be fine).

\noindent Suppose that we can apply the Gauss-Green formula, then we integrate by parts and obtain the relation
\begin{equation*}\int_{\Omega} \left[ - \Delta \, \underline{u}(x) - f(x)\right] \, v(x) \, \mathrm{d}x = 0, \qquad \forall \, v \in \mathcal{A}_0.\end{equation*}
Finally, we apply the Fundamental Lemma of Calculus of Variations \footnote{Let $u \in L^1(\Omega)$. Suppose that, for any $v \in L^1(\Omega)$ which is compactly supported, it turns out that \begin{equation*} \int_\Omega u(x) \, v(x) \, \mathrm{d}x = 0. \end{equation*}

Then $u(x) = 0$ for almost every $x \in \Omega$.} and we conclude that $\underline{u}$ is a strong solution of the Dirichlet problem \eqref{dir}.\end{proof}

A similar argument allows us to study another important physic model: the \textbf{Kirchoff-Love} model for a thin plate. We don't give the details of this problem, bur the reader may find them in \textbf{Tarsia's notes}.