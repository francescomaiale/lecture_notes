\chapter{Functional Spaces}

In this chapter, we introduce some important functional spaces that will allow us to study the regularity of elliptic problems via integral estimates.

More precisely, in the first section, we recall the notion of $\alpha$-Hölder continuity, and we state some facts which will be useful in the next chapter.

Next, we introduce the so-called Morrey-Campanato spaces and prove that, under some assumptions, they are equivalent to $L^p$ spaces, but they are characterized by an integral norm, which is more useful for elliptic problems.

\section{Hölder Spaces}

Let $\Omega \subset \R^n$ be a bounded open set. A function $u$ is $\gamma$-Hölder continuous, with $\gamma \in (0, \, 1)$, on $\Omega$ if
\begin{equation*} [u]_{\gamma, \, \Omega} := \sup_{ \substack{x, \, y \in \Omega \\[0.2em] x \neq y} } \frac{|u(x) - u(y)|}{|x - y|^\gamma} < + \infty, \end{equation*}
and it is Lipschitz if the same holds with $\gamma = 1$. The space
\begin{equation*} C^{0, \, \gamma}(\Omega) := \left\{ u : \Omega \to \C \: \left| \: [u]_{\gamma, \, \Omega} < + \infty \right. \right\} \end{equation*}
is Banach endowed with the norm
\begin{equation*} \|u\|_{\gamma, \, \Omega} := \|u\|_{\infty, \, \Omega} + [u]_{\gamma, \, \Omega}. \end{equation*}
The inclusions
\begin{equation*}C^{0, \, \beta}(\Omega) \subseteq C^{0, \, \alpha}(\Omega) \subseteq C^0(\Omega) \end{equation*}
are easy to verify when $\Omega$ is bounded and $0 < \alpha \leq \beta \leq 1$, but the first one is false if $\Omega$ is unbounded since, e.g.,
\begin{equation*}\|x\|^{\beta} \in C^{0, \, \beta}(\Omega) \setminus C^{0, \, \alpha}(\Omega) \end{equation*}
for every $\alpha > \beta$. The function
\begin{equation*} v(t) := \begin{cases} \frac{1}{\log(t)} & t \neq 0 \\[0.5em] 0 & t = 0 \end{cases} \end{equation*}
is continuous in $\overline{\Omega}$, but it is not $\gamma$-Hölder for any $\gamma \in (0, \, 1]$. Moreover, the inclusion
\begin{equation*}C^{1}\left(\overline{\Omega} \right) \subseteq C^{0, \, 1}(\Omega) \end{equation*}
is false, even if $\Omega$ is bounded, since
\begin{equation*} w(x, \, y) := \begin{cases} \frac{x^2}{\sqrt{x^2 + y^2}} \cdot \arctan \frac{x + |x|}{2y} & y \neq 0 \\[0.5em] 0 & y = 0 \, \, x < 0 \end{cases} \end{equation*}
is differentiable, but not Lipschitz on $\Omega$.

\paragraph{$(k, \, \gamma)$-Hölder spaces.} The space
\begin{equation*} C^{k, \, \gamma}(\Omega) := \left\{ u \in C^k(\Omega) \: \left| \: \sum_{|\alpha| = k} \left[D^\alpha u \right]_{\gamma, \, \Omega} < + \infty \right. \right\} \end{equation*}
is Banach endowed with the norm
\begin{equation*} \|u\|_{k, \, \gamma, \, \Omega} := \sum_{|\alpha| < k} \left\|D^\alpha u \right\|_{\infty, \, \Omega} + \sum_{|\alpha| = k} \left[D^\alpha u \right]_{\gamma, \, \Omega}. \end{equation*}

\begin{theorem}\mbox{}
\begin{enumerate}[label=\textbf{(\arabic*)}]
\item The immersion
\begin{equation*}C^{k, \, \gamma}\left(\overline{\Omega} \right) \subseteq C^{k}\left(\overline{\Omega} \right) \end{equation*}
is continuous and compact for any $\gamma \in (0, \, 1]$.
\item The immersion
\begin{equation*}C^{k, \, \gamma}\left(\overline{\Omega} \right) \subseteq C^{k, \, \beta}\left(\overline{\Omega} \right) \end{equation*}
is continuous and compact for any $0 < \beta \leq \gamma \leq 1$.
\end{enumerate}\end{theorem}

\begin{theorem}[Ciesielski, \cite{siad}] There is an isomorphism
\begin{equation*} C^{k, \, \gamma}\left( \overline{\Omega} \right) \cong \ell_\infty. \end{equation*}
In particular, the space $C^{k, \, \gamma}\left( \overline{\Omega} \right)$ is not separable for any $k \in \N$ and $\gamma \in (0, \, 1]$. \end{theorem}

\section{Morrey Spaces}

Let $\Omega \subset \R^n$ be a bounded open set of diameter $\delta > 0$, and let us set
\begin{equation*} \Omega(x, \, \rho) := B(x, \, \rho) \cap \Omega. \end{equation*}

\begin{definition} The set $\Omega$ is of the type \textbf{A} if there exists a positive constant $A > 0$ such that for any $x \in \overline{\Omega}$ and any $\rho \in (0, \, \delta)$ it turns out that
\begin{equation*} \left| \Omega(x, \, \rho) \right| \geq A \cdot \rho^n. \end{equation*} \end{definition} 

\begin{example} The set defined by
\begin{equation*} \Omega := \left\{ (x, \, y) \in \R^2 \: \left| \: \text{$0 < x < 1$ and $0 < y < x^2$} \right. \right\} \end{equation*}
is not of type A (see \hyperref[Count1]{Figure \ref{Count1}}). \end{example}

\begin{figure}[h]
\centering
\includegraphics[width = 10cm, height = 8cm]{Images/EEEE1.png}
\caption{A set which is not of type A.}
\label{Count1}
\end{figure} 

\paragraph{N.B.} From now on, we shall assume $\Omega$ set of type A.

\paragraph{Morrey Spaces.} Let $u \in L^p(\Omega)$. The limit
\begin{equation*} \lim_{\rho \to 0^+} \int_{\Omega(x, \, \rho)} |u(x)|^p \, \mathrm{d}x \end{equation*}
is equal to zero since the integral is an absolutely continuous operator. For any $\lambda \geq 0$ and $p \in [1, \, + \infty)$, the space
\begin{equation*}L^{p, \, \lambda}(\Omega) := \left\{ u \in L^p(\Omega) \: \left| \: \left\| u \right\|_{p, \, \lambda, \, \Omega} < + \infty \right. \right\}, \end{equation*}
where
\begin{equation} \label{normamorrey}\left\| u \right\|_{p, \, \lambda, \, \Omega}^p := \sup_{\substack{x \in \Omega \\[0.2em] \rho \leq \delta}} \: \frac{1}{\rho^\lambda} \, \int_{\Omega(x, \, \rho)} |u(t)|^p \, \mathrm{d}t. \end{equation}

\begin{proposition}The Morrey space $L^{p, \, \lambda}(\Omega)$ is Banach endowed with the norm \eqref{normamorrey}. \end{proposition}

\begin{proof}Let $(u_n)_{n \in \N} \subset L^{p, \, \lambda}(\Omega)$ be a Cauchy sequence. Then the sequence is uniformly bounded, i.e., there exists $M > 0$ such that
\begin{equation} \label{morreyboun} \| u_n \|_{p, \, \lambda, \, \Omega} \leq M, \end{equation}
and we notice that
\begin{equation*} \| u_n - u_m \|_{0, \, p, \, \Omega} \leq \delta^\lambda \cdot \| u_n - u_m \|_{p, \, \lambda, \, \Omega}. \end{equation*}
In particular, the sequence $(u_n)_{n \in \N}$ is also Cauchy in $L^p$, and hence there exists $u \in L^p(\Omega)$ such that
\begin{equation*} u_n \xrightarrow{L^p(\Omega)} u, \qquad u_{n_k} \xrightarrow{\text{pointwise}} u \end{equation*}
and hence it suffices to prove that $u$ belongs to the Morrey space, and also that the convergence is in $L^{p, \, \lambda}(\Omega)$.

The pointwise convergence is enough to prove the first assertion since one can pass to the limit in \eqref{morreyboun} using the Lebesgue dominated convergence theorem, i.e.,
\begin{equation*}  \| u \|_{p, \, \lambda, \, \Omega} \leq M. \end{equation*}
By assumption for any $\epsilon > 0$ there exists $N_\epsilon$ such that for every $n, \, m > N_\epsilon$ it turns out that
\begin{equation*} \frac{1}{\rho^\lambda} \, \int_{\Omega(x, \, \rho)} \left| u_n(t) - u_m(t) \right|^p \, \mathrm{d}t \leq \| u_n - u_m \|_{p, \, \lambda, \, \Omega} < \epsilon \end{equation*}
for any $\rho > 0$ and for any $x \in \Omega$; on the other hand, if we take the limit as $n \to + \infty$, we infer that
\begin{equation*} \frac{1}{\rho^\lambda} \, \int_{\Omega(x, \, \rho)} \left| u(t) - u_m(t) \right|^p \, \mathrm{d}t < \epsilon \qquad \forall \, \rho > 0, \, \, \forall \, x \in \Omega, \end{equation*}
which implies the convergence in $L^{p, \, \lambda}(\Omega)$.
\end{proof}

\begin{theorem} \mbox{}
\begin{enumerate}[label=\textbf{\arabic*)}]
\item $L^{p, \, 0}(\Omega) \cong L^p(\Omega)$.
\item $L^{p, \, n}(\Omega) \cong L^\infty(\Omega)$.
\item If $\lambda > n$ strictly, then $L^{p, \, \lambda}(\Omega) = \{ 0\}$.
\item If $1 \leq p < q < + \infty$ and
\begin{equation*} \frac{\lambda - n}{p} \leq \frac{\mu - n}{q}, \end{equation*}
then $L^{q, \, \mu}(\Omega) \subseteq L^{p, \, \lambda}(\Omega)$.
\end{enumerate} \end{theorem}

\begin{proof} \mbox{}
\begin{enumerate}[label=\textbf{\arabic*)}]
\item This property follows straightforwardly from the definition.
\item To prove this point, we need to recall a known property of integration theory. If $u$ is a measurable function, we can consider the set
\begin{equation*} S(u, \, \sigma) := \left\{ x \in \Omega \: \left| \: |u(x)| > \sigma \right. \right\}\end{equation*}
as $\sigma$ ranges in $[0, \, + \infty)$. Clearly $S(u, \, \sigma)$ is also measurable and, if $u \in L^p(\Omega)$ for any $p \in [1, \, + \infty)$, then the function
\begin{equation*} \sigma \longmapsto \sigma^{p-1} \cdot | S(u, \, \sigma) | \end{equation*}
is summable on the interval $[0, \, + \infty)$ and it turns out that
\begin{equation} \label{recall:mt} \int_{\Omega} |u(x)|^p \, \mathrm{d}x = p \cdot \sigma^{p-1} \, \int_{0}^{+\infty} | S(u, \, \sigma) | \, \mathrm{d}\sigma. \end{equation}

The inclusion $L^{p, \, n}(\Omega) \supseteq L^\infty(\Omega)$ is rather obvious since
\begin{equation*} \begin{aligned} \frac{1}{\rho^n} \, \int_{\Omega(x, \, \rho)} |u(t)|^p \, \mathrm{d}t & \leq \sup_{t \in \Omega} |u(t)|^p \cdot \frac{| \Omega(x, \, p) |}{\rho^n} \leq \\[1em] & \leq \frac{\omega_n \cdot \rho^n}{\rho^n} \, \|u\|_{L^\infty(\Omega)} \leq \\[1em] & \leq c \cdot \|u \|_{L^\infty(\Omega)}. \end{aligned} \end{equation*}

To prove the opposite inclusion, we argue by contradiction assuming that $L^{p, \, n}(\Omega) \supset L^\infty(\Omega)$ is a strict inclusion. The estimate
\begin{equation*}|u(x)|^p = \lim_{\rho \to 0^+} \frac{1}{\rho^n} \, \int_{\Omega(x, \, \rho)} |u(t)|^p \, \mathrm{d}t \leq \sup_{0 < \rho \leq \delta} \frac{1}{\rho^n} |u(t)|^p \, \mathrm{d}t\end{equation*}
holds for every point $x \in \mathcal{L}(u, \, \Omega)$, i.e., for every Lebesgue point of $u$ in $\Omega$. Let $u$ be an element of $L^{p, \, n}(\Omega) \setminus L^\infty(\Omega)$, i.e.,
\begin{equation*} \|u\|_{\infty, \, \Omega} = + \infty. \end{equation*}
It is equivalent to saying that for any $t > 1$ the measure of the set $S(u, \, t)$ is positive (strictly), and hence that for any $x \in \mathcal{L}(u, \, \Omega) \cap S(u, \, t)$ it turns out that $|u(x)|^p > t$.

Therefore, for any $t > 1$ there exists a $x \in \Omega$ such that
\begin{equation*} t < \sup_{0 < \rho \leq \delta} \frac{1}{\rho^n} \int_{\Omega(x, \, \rho)} |u(t)|^p \, \mathrm{d}t \implies u \notin L^{p, \, n}(\Omega). \end{equation*}
\item This assertion follows trivially from \textbf{2)}. Indeed, for any $x \in \Omega$ we have
\begin{equation*} \sup_{x \in \Omega} \: \frac{1}{\rho^n} \int_{\Omega(x, \, \rho)} |u(t)|^p \, \mathrm{d}t \leq \| u \|_{p, \, \lambda, \, \Omega} \cdot \rho^{\lambda - n}, \end{equation*}
and hence
\begin{equation*} \sup_{\substack{x \in \Omega \\[0.1em] \rho \in (0, \, \rho_1] }} \: \frac{1}{\rho^n} \int_{\Omega(x, \, \rho)} |u(t)|^p \, \mathrm{d}t \leq \| u \|_{p, \, \lambda, \, \Omega} \cdot \rho_1^{\lambda - n}. \end{equation*}
By taking the limit as $\rho_1 \to 0^+$, we infer that the left-hand side converges to $|u(x)|^p$, while the right-hand side converges to $0$, i.e., $u(x) \equiv 0$.
\item For any $x \in \Omega$ and $\rho \in (0, \, \delta]$ it turns out that
\begin{equation*} \begin{aligned} \int_{\Omega(x, \, \rho)} |u(t)|^p \, \mathrm{d}t & = \int_{\Omega(x, \, \rho)} \left( |u(t)|^q \right)^{\frac{p}{q}} \cdot 1^{1 - \frac{p}{q}} \, \mathrm{d}t \leq \\[1em] & \leq \left[ \int_{\Omega(x, \, \rho)} |u(t)|^q \, \mathrm{d}t \right]^{\frac{p}{q}} \cdot \left| \Omega(x, \, \rho) \right|^{1 - \frac{p}{q}} \: {\color{red}\leq} \\[1em] & \: {\color{red}\leq} \: \left[ \int_{\Omega(x, \, \rho)} |u(t)|^q \, \mathrm{d}t \right]^{\frac{p}{q}} \cdot \left[ \omega_n \, \rho^n \right]^{1 - \frac{p}{q}} = \\[1em] & = c(n, \, p, \, q)  \cdot \rho^{ n \left( 1 - \frac{p}{q} \right) + \mu \, \frac{p}{q}} \cdot \left[\frac{1}{\rho^\mu} \, \int_{\Omega(x, \, \rho)} |u(t)|^q \, \mathrm{d}t \right]^{\frac{p}{q}},
\end{aligned} \end{equation*}
where the {\color{red}red} inequality follows from the inclusion $\Omega(x, \, \rho) \subset B(x, \, \rho)$. If we divide both sides by $\rho^\lambda$, it turns out that
\begin{equation*}\frac{1}{\rho^\lambda} \, \int_{\Omega(x, \, \rho)} |u(t)|^p \, \mathrm{d}t \leq c(n, \, p, \, q)  \cdot \rho^{ n \left( 1 - \frac{p}{q} \right) + \mu \, \frac{p}{q} - \lambda} \cdot \left[\frac{1}{\rho^\mu} \, \int_{\Omega(x, \, \rho)} |u(t)|^q \, \mathrm{d}t \right]^{\frac{p}{q}}, \end{equation*}
and the right-hand side is finite if and only if the exponent of $\rho$ is positive, that is,
\begin{equation*} n \left( 1 - \frac{p}{q} \right) + \mu \, \frac{p}{q} - \lambda \geq 0 \iff \frac{\mu - n}{q} \geq \frac{\lambda - n}{p}, \end{equation*}
which is exactly what we wanted to prove.
\end{enumerate} \end{proof}

\section{Campanato Spaces}

In this section, we shall denote by
\begin{equation*} u_{x, \, \rho} := \frac{1}{\left| \Omega(x, \, \rho) \right|} \, \int_{\Omega(x, \, \rho)} u(t) \, \mathrm{d}t \end{equation*}
the average of $u$ on the set $\Omega(x, \,\rho)$. The Campanato $(p, \, \lambda)$ space is defined by
\begin{equation*}\mathscr{L}^{p, \, \lambda}(\Omega) := \left\{ u \in L^p(\Omega) \: \left| \: [ u ]_{\mathscr{L}^{p, \, \lambda}(\Omega)} < + \infty \right. \right\}, \end{equation*}
where
\begin{equation} \label{seminormacampanato}\left[ u \right]_{\mathscr{L}^{p, \, \lambda}(\Omega)}^p := \sup_{\substack{x \in \Omega \\[0.2em] \rho \leq \delta}} \: \frac{1}{\rho^\lambda} \, \int_{\Omega(x, \, \rho)} |u(t) - u_{x, \, \rho}|^p \, \mathrm{d}t \end{equation}
is a seminorm on $\mathscr{L}^{p, \, \lambda}(\Omega)$.

\begin{remark}The Campanato space $\mathscr{L}^{p, \, \lambda}(\Omega)$ is Banach endowed with the norm
\begin{equation} \label{normacampanato}\left\| u \right\|_{\mathscr{L}^{p, \, \lambda}(\Omega)} = \| u \|_{L^p(\Omega)} + \left[ u \right]_{\mathscr{L}^{p, \, \lambda}(\Omega)}.\end{equation} \end{remark}

\begin{proposition}[Characterization] \label{campachar} A function $u : \Omega \to \C$ belongs to $\mathscr{L}^{p, \, \lambda}(\Omega)$ if and only if $u \in L^p(\Omega)$ and the seminorm
\begin{equation} \label{thirdcampanato} \vertiii{u}_{\mathscr{L}^{p, \, \lambda}(\Omega)}^p := \sup_{\substack{x \in \Omega \\[0.2em] \rho \leq \delta}} \: \frac{1}{\rho^\lambda} \, \inf_{c \in \R} \, \int_{\Omega(x, \, \rho)} |u(t) - c|^p \, \mathrm{d}t \end{equation}
is finite. \end{proposition}

\begin{proof}If $u \in \mathscr{L}^{p, \, \lambda}(\Omega)$, then it is a trivial consequence of the fact that
\begin{equation*} \vertiii{u}_{\mathscr{L}^{p, \, \lambda}(\Omega)}^p \leq \left[ u \right]_{\mathscr{L}^{p, \, \lambda}(\Omega)}^p <+ \infty. \end{equation*}
Vice versa, suppose that $u \in L^p(\Omega)$ and $\vertiii{u}_{\mathscr{L}^{p, \, \lambda}(\Omega)} < + \infty$. For every $x \in \Omega$ and $\rho \in (0, \, \delta]$ it turns out that\footnote{We shall freely use the inequality $(a + b)^p \leq 2^{p-1} \cdot (a^p + b^p)$ valid for every $p \geq 1$.}
\begin{equation*} \begin{aligned} \int_{\Omega(x, \, \rho)} |u(t) - u_{x, \, \rho}|^p \, \mathrm{d}t & \leq 2^{p-1} \, \left[  \int_{\Omega(x, \, \rho)} |u(t) - c|^p \, \mathrm{d}t +  \int_{\Omega(x, \, \rho)} |c - u_{x, \, \rho}|^p \, \mathrm{d}t \right] = \\[1em] & = 2^{p-1} \, \left[  \int_{\Omega(x, \, \rho)} |u(t) - c|^p \, \mathrm{d}t + \left| \Omega(x, \, \rho) \right|^{1 - p} \, \left| \int_{\Omega(x, \, \rho)} [u(t) - c]^p \, \mathrm{d}t \right|^p \right] \leq \\[1em] & \leq 2^{p} \int_{\Omega(x, \, \rho)} |u(y) - c|^p \, \mathrm{d}y\end{aligned} \end{equation*}
from which it follows that
\begin{equation*}\left[ u \right]_{\mathscr{L}^{p, \, \lambda}(\Omega)}^p \leq 2 \cdot \vertiii{u}_{\mathscr{L}^{p, \, \lambda}(\Omega)}^p. \end{equation*}
\end{proof}

\begin{corollary}The norms
\begin{equation*} \left\| u \right\|_{\mathscr{L}^{p, \, \lambda}(\Omega)} := \| u \|_{L^p(\Omega)} + \left[ u \right]_{\mathscr{L}^{p, \, \lambda}(\Omega)}\end{equation*}
and
\begin{equation*}\vertiiii{u}_{\mathscr{L}^{p, \, \lambda}(\Omega)} := \| u \|_{L^p(\Omega)} + \vertiii{u}_{\mathscr{L}^{p, \, \lambda}(\Omega)}\end{equation*}
are equivalent.
\end{corollary}

\begin{theorem}\label{campanatotheorem} \mbox{}
\begin{enumerate}[label=\textbf{\arabic*)}]
\item If $0 \leq \lambda < n$, then $\mathscr{L}^{p, \, 0}(\Omega) \cong L^{p, \, \lambda}(\Omega)$.
\item If $n < \lambda \leq n + p$, then $\mathscr{L}^{p, \, 0}(\Omega) \cong C^{0, \, \gamma}(\Omega)$ with
\begin{equation*} \gamma = \frac{\lambda - n}{p}. \end{equation*}
\item If $\lambda > n + p$ strictly, then $u \in \mathscr{L}^{p, \, 0}(\Omega)$ is locally constant.
\item If $1 \leq p < q < + \infty$ and
\begin{equation*} \frac{\lambda - n}{p} \leq \frac{\mu - n}{q}, \end{equation*}
then $\mathscr{L}^{q, \, \mu}(\Omega) \subseteq \mathscr{L}^{p, \, \lambda}(\Omega)$.
\end{enumerate} \end{theorem}

We will only deal with \textbf{1)} since it follows from an algebraic lemma which will be used consistently in the next chapters. The reader may check out the other properties in this book: \cite{mglm}.

\begin{lemma} \label{algebraiclemma} Let $\varphi$ and $\Phi$ be two nonnegative functions defined on $(0, \, d]$, and assume that $\Phi$ is nondecreasing. Assume that there exist positive constants $A, \, \alpha, \, \beta > 0$ such that $\alpha > \beta$ and, for any $t \in (0, \, 1)$ and $\sigma \in (0, \, d]$, 
\begin{equation}\label{eq:cond1} \varphi(t \sigma) \leq A \, t^\alpha \, \varphi(\sigma) + \sigma^\beta \, \Phi(\sigma). \end{equation}
Then, for any $\epsilon \in (0, \, \alpha - \beta]$, $t \in (0,\,1)$ and $\sigma \in (0, \, d]$, it turns out that
\begin{equation}\label{eq:result1} \varphi(t \sigma) \leq A \, t^{\alpha-\epsilon} \, \varphi(\sigma) + K(A) \, \left(t \sigma\right)^\beta \, \Phi(\sigma), \end{equation}
where
\begin{equation*} K(\xi) := \frac{(1 + \xi)^{\frac{2 \, \alpha}{\xi}}}{(1 + \xi)^{ \frac{\alpha - \beta}{\epsilon}} - \xi}. \end{equation*}\end{lemma}

\begin{proof}[Proof of Theorem \ref{campanatotheorem}] The inclusion $L^{p, \, \lambda}(\Omega) \subseteq \mathscr{L}^{p, \, \lambda}(\Omega)$ follows trivially by \hyperref[campachar]{Proposition \ref{campachar}}, since the infimum is taken over all $c \in \R$, included $c = 0$.

The opposite inclusion, on the other hand, is not trivial at all and it requires an application of the algebraic lemma stated above. For every $t \in (0, \, 1)$ and $\sigma \in (0, \, \delta]$ we have the following estimate:
\begin{equation*} \begin{aligned} \int_{\Omega(x, \, t \sigma)} |u(y)|^p \, \mathrm{d}y & \leq 2^{p-1} \, \left[  \int_{\Omega(x, \, \sigma)} |u(y) - u_{x, \, \sigma}|^p \, \mathrm{d}y +  \int_{\Omega(x, \, \sigma)} |u_{x, \, \sigma}|^p \, \mathrm{d}y \right] = \\[1em] & = 2^{p-1} \, \left[  \int_{\Omega(x, \, \sigma)} |u(y) - u_{x, \, \sigma}|^p \, \mathrm{d}y + \left| \Omega(x, \, \sigma) \right| \, \left| \frac{1}{|\Omega(x, \, \sigma)|^p} \, \int_{\Omega(x, \, \sigma)} u(y) \, \mathrm{d}t \right|^p \right] \: {\color{blue}\leq} \\[1em] & \: {\color{blue}\leq} \: 2^{p-1} \, \left[  \int_{\Omega(x, \, \sigma)} |u(y) - u_{x, \, \sigma}|^p \, \mathrm{d}y + \frac{\left| \Omega(x, \, \sigma) \right|}{\left| \Omega(x, \, \sigma) \right|} \, \int_{\Omega(x, \, \sigma)} |u(y)|^p \, \mathrm{d}t \right] \leq \\[1em] & \leq 2^{p-1} \, \left[  \int_{\Omega(x, \, \sigma)} |u(y) - u_{x, \, \sigma}|^p \, \mathrm{d}y + \frac{\omega_n \, (t \sigma)^n}{\left| \Omega(x, \, \sigma) \right|} \, \int_{\Omega(x, \, \sigma)} |u(y)|^p \, \mathrm{d}t \right] \leq \\[1em] & \leq 2^{p-1} \, \left[  \sigma^\lambda \, \left[ u \right]_{\mathscr{L}^{p, \, \lambda}(\Omega)}^p + \frac{\omega_n \, (t \sigma)^n}{\left| \Omega(x, \, \sigma) \right|} \, \int_{\Omega(x, \, \sigma)} |u(y)|^p \, \mathrm{d}t \right]  \: {\color{red}\leq}  \\[1em] & \: {\color{red}\leq}  \: C(p, \, n, \, \sigma) \cdot \left[\left[ u \right]_{\mathscr{L}^{p, \, \lambda}(\Omega)}^p + t^n \, \|u\|_{L^p(\Omega(x, \, \sigma))}^p \right] \: {\color{orange}\leq}  \\[1em] & \: {\color{orange}\leq} \: C^\prime(p, \, n, \, \sigma) \cdot \left[ (t \sigma)^{\lambda} \cdot \left[ u \right]_{\mathscr{L}^{p, \, \lambda}(\Omega)}^p + t^{n - \epsilon} \, \|u\|_{L^p(\Omega(x, \, \sigma))}^p \right] \: {\color{green}\leq}  \\[1em] & \: {\color{green}\leq} \: C^\prime(p, \, n, \, \sigma) \cdot \left(t \sigma\right)^{\lambda} \cdot \left[ \left[ u \right]_{\mathscr{L}^{p, \, \lambda}(\Omega)}^p + \frac{1}{\sigma^\lambda} \|u\|_{L^p(\Omega(x, \, \sigma))}^p \right], \end{aligned} \end{equation*}
and we conclude dividing both sides by $\left(t \sigma\right)^{\lambda}$. The marked inequalities need to be explained a little more in depth: \mbox{}
\begin{enumerate}
\item The {\color{blue}blue} inequality follows from a straightforward application of the Hölder inequality with $u$ and $1$.
\item The {\color{red}red} inequality follows from the fact that $\Omega$ is a set of the type A; more precisely, we have the estimates
\begin{equation*} \begin{aligned} &x \in \partial \, \Omega \implies \left| \Omega(x, \, \rho) \right| \geq A \cdot \sigma^n \implies \frac{1}{ \left| \Omega(x, \, \rho) \right|} \leq \frac{1}{A \cdot \sigma^n}, \\[1em] &x \in \Omega \implies \left| \Omega(x, \, \rho) \right| = \left|B(x, \, \rho)\right| \implies \left| \Omega(x, \, \rho) \right| = \omega_n \, \rho^n.  \end{aligned} \end{equation*}
\item The {\color{orange}orange} inequality follows from \hyperref[algebraiclemma]{Lemma \ref{algebraiclemma}} by setting
\begin{equation*} \begin{aligned} & \varphi(\sigma) = \int_{\Omega(x, \, \sigma)} |u(y)|^p \, \mathrm{d}y, \\[1em] & \Phi(\sigma) = \left[ u \right]_{\mathscr{L}^{p, \, \lambda}(\Omega)}^p \end{aligned} \end{equation*}
\item The {\color{green}green} inequality follows from the fact that we can chose $\epsilon = n - \lambda$.
\end{enumerate}
\end{proof}

\paragraph{Generalized Poincaré inequality.} In this brief paragraph, we want to state and prove a generalization of the Poincaré inequality which will be useful to show a regularity results in Morrey-Campanato spaces.

\begin{theorem}Let $\Omega \subset \R^n$ be an open bounded connected subset of $\R^n$ with Lipschitz boundary. There exists a positive constant $c(p, \, n, \, \Omega)$ such that for any $u \in H^{1, \, p}(\Omega)$, $1 \leq p < + \infty$, it turns out that
\begin{equation} \label{poincargen} \int_\Omega \left| u(x) - u_\Omega \right|^p \, \mathrm{d}x \leq c(p, \, n, \, \Omega) \cdot |u|_{1, \, p, \, \Omega}^p\end{equation} 
where $u_\Omega$ is the average on $\Omega$, i.e.,
\begin{equation*} u_\Omega = \dashint_\Omega u(x) \, \mathrm{d}x. \end{equation*}
\end{theorem}

\begin{proof}We may always assume, without loss of generality, that the average of $u$ on $\Omega$ is equal to zero.

We argue by contradiction. If \eqref{poincargen} does not hold, then there exists a sequence $(u_k)_{k \in \N} \subset H^{1, \, p}(\Omega)$ satisfying the following properties:
\begin{equation*} \begin{cases} \dashint_\Omega u_k(x) \, \mathrm{d}x = 0, \\[0.5em] \|u_k\|_{L^p(\Omega)} = 1, \\[0.5em] |u_k|_{1, \, p, \, \Omega} \leq \frac{1}{k}.\end{cases} \end{equation*}
Since $\{u_k\}_{k \in \N}$ is a bounded subset of $H^{1, \, p}(\Omega)$, by Rellich theorem there is a subsequence $(u_{k_h})_{h \in \N}$ converging to a function $u$ strongly in $L^p(\Omega$; in particular,
\begin{equation*} \| u \|_{L^p(\Omega)} = 1. \end{equation*}
On the other hand, $|u_k|_{1, \, p, \, \Omega} \to 0$ and hence from
\begin{equation*} \int_\Omega u_k(x) \, D^i \, \varphi(x) \, \mathrm{d}x \xrightarrow{m \to + \infty} \int_\Omega u(x) \, D^i \, \varphi(x) \, \mathrm{d}x, \qquad \forall \, \varphi \in C_c^\infty(\Omega) \end{equation*}
it follows that $D^i u = 0$ on $\Omega$ for each $i = 1, \, \dots, \, n$.

In particular, $u$ is locally constant on a connected set, i.e., $u$ is constant on $\Omega$ and hence it is equal to zero (since it is a function with average zero). This is the sought contradiction since $u$ has $L^p(\Omega)$ norm equal to one.\end{proof}

The next result gives an explicit expression for the dependence of the constant $c$ on $\Omega$ when $\Omega$ is a ball.

\begin{theorem}Let $x_0 \in \R^n$. There exists a positive constant $c(p, \, n)$ such that for any $u \in H^{1, \, p}(B(x_0, \, r))$, $1 \leq p < + \infty$, it turns out that
\begin{equation} \label{poincargen2} \int_{B(x_0, \, r)} \left| u(x) - u_{x_0, \, r} \right|^p \, \mathrm{d}x \leq c(p, \, n) \, r^p \cdot |u|_{1, \, p, \, B(x_0, \, r)}^p.\end{equation} 
\end{theorem}

\begin{proof}We may always assume, without loss of generality, that the point $x_0$ is the origin.

Let us consider the homothety $\alpha_r : x \mapsto r \cdot x$, and let us consider the function $v(y) := u \circ \alpha_r(y)$, which is of class $H^{1, \, p}(B(0, \, 1))$. The inequality \eqref{poincargen} holds for $\Omega = B(0, \, 1)$, hence there exists a constant $c^\prime(n, \, p) > 0$ (the constant does not depend on $\Omega$ since the unitary ball is entirely determined by the dimension $n$) such that
\begin{equation*} \int_{B(0, \, 1)} \left| u(x) - v_{B(0, \, 1)} \right|^p \, \mathrm{d}x \leq c^\prime(p, \, n) \cdot |u|_{1, \, p, \, B(0, \, 1)}^p.\end{equation*} 
A straightforward computation proves that
\begin{equation*}v_{B(0, \, 1)} = U_{B(0, \, r)} \qquad \text{and} \qquad \int_{B(0, \,1)} \left|D^i \, v(y) \right|^p \, \mathrm{d}y = r^{p - n} \, \int_{B(0, \, r)} \left|D^i \, u(y) \right|^p \, \mathrm{d}y, \end{equation*}
from which it follows that
\begin{equation*} \int_{B(0, \, 1)} \left| u(x) - v_{B(0, \, 1)} \right|^p \, \mathrm{d}x = \frac{1}{r^n} \, \int_{B(0, \, r)} \left| u(x) - v_{B(0, \, 1)} \right|^p \, \mathrm{d}x \leq c^\prime(p, \, n) \, r^n \cdot |u|_{1, \, p, \, B(0, \, 1)}^p.\end{equation*} 
\end{proof}

We are now ready to state and prove the last theorem which establishes a link between Morrey-Campanato spaces and Sobolev spaces (which is, somehow, a regularity result).

\begin{theorem}Let $\Omega \subset \R^n$ be an open bounded connected subset of $\R^n$ with Lipschitz boundary. If $u \in H^{1, \, p}(\Omega)$ and $D^i \, u \in L^{p, \, \lambda}(\Omega)$, $0 \leq \lambda < n$, for each $i = 1, \, \dots, \, n$, then $u \in \mathscr{L}^{p, \, \lambda + p}(\Omega)$ and
\begin{equation} \label{eq:laeifeoewqe} [u]_{\mathscr{L}^{p, \, \lambda + p}(\Omega)} \leq c(n, \, p, \, A) \cdot \sum_{i = 1}^n \| D^i u \|_{L^{p, \, \lambda}(\Omega)}. \end{equation}
In particular, if $\lambda + p < n$, then
\begin{equation} \label{eq:laeifeoewqe1} [u]_{\mathscr{L}^{p, \, \lambda + p}(\Omega)} \leq c(n, \, p, \, A, \, \lambda) \cdot \left[ \sum_{i = 1}^n \| D^i u \|_{L^{p, \, \lambda}(\Omega)} + \|u\|_{L^p(\Omega)} \right], \end{equation}
and, if $\lambda + p > n$ and $\gamma = 1 - (n-\lambda)/p$, then
\begin{equation} \label{eq:laeifeoewqe1} [u]_{0, \, \gamma, \, \Omega} \leq c(n, \, p, \, A) \cdot \sum_{i = 1}^n \| D^i u \|_{L^{p, \, \lambda}(\Omega)}. \end{equation}
\end{theorem}

\begin{proof}Fix $x_0 \in \Omega$ and $0 < \sigma \leq \delta$. The Poincaré inequality \eqref{poincargen2} it turns out that
\begin{equation*} \begin{aligned} \int_{\Omega(x_0, \, \sigma)} \left| u(x) - u_{x_0, \, \sigma} \right|^p \, \mathrm{d}x & \leq c(n, \, p, \, A) \, \sigma^p \cdot |u|_{1, \, p, \, \Omega(x_0, \, \sigma)}^p \leq \\[1em] & \leq c(n, \, p, \, A) \, \sigma^{p + \lambda} \cdot \sum_{i = 1}^n \left\|D^i u\right\|_{L^{p, \, \lambda}(\Omega)}. \end{aligned}\end{equation*} \end{proof}

\section{Bounded Mean Oscillation Spaces}

Let $Q_0$ be a $n$-dimensional cube in $\R^n$. A function $u \in L^1(Q_0)$ is a function of \textit{bounded mean oscillation}, and we shall denote it by $u \in \mathrm{BMO}(Q_0)$, if
\begin{equation} \label{bmosemi} [u]_{\mathrm{BMO}(Q_0)} = \sup_{Q \subset Q_0} \: \frac{1}{|Q|} \, \int_Q \left| u(x) - u_Q \right| \, \mathrm{d}x \end{equation}
is finite, where the supremum is taken over all the $n$-dimensional cubes $Q \subset Q_0$ with parallel edges.

\begin{remark}The space $\mathrm{BMO}(Q_0)$ coincides with the Campanato space $\mathscr{L}^{1, \, n}(Q_0)$ since one may always consider
\begin{equation*} \Omega(x, \, \rho) = \Omega \cap Q(x, \, \rho) \end{equation*}
instead of
\begin{equation*} \Omega(x, \, \rho) = \Omega \cap B(x, \, \rho), \end{equation*}
and obtain an equivalent Banach space.\end{remark}

Let $u \in L^1(Q_0)$ be given. For every $\sigma > 0$ and for every $n$-dimensional cube $Q \subset Q_0$, with parallel edges, we set
\begin{equation*} S(\sigma, \, Q) := \left\{ x \in Q \: \left| \: |u(x) - u_Q| > \sigma \right. \right\}. \end{equation*}

\begin{definition}[John-Nirenberg] A function $u : Q_0 \to \C$ belongs to $\mathcal{E}_0(Q_0)$ if there are positive constants $H, \, \beta > 0$ such that. for every $\sigma > 0$ and every $n$-dimensional cube $Q \subset Q_0$, it turns out that
\begin{equation} \label{qwoeoqw} \left|S(\sigma, \, Q)\right| \leq H \cdot \mathrm{e}^{- \beta \sigma} \cdot |Q|. \end{equation}\end{definition}

\begin{theorem}A function $u : Q_0 \to \C$ belongs to $\mathcal{E}_0(Q_0)$ if and only if $u$ belongs to the Campanato space $\mathscr{L}^{p, \, n}(Q_0)$, for any $p \geq 1$. \end{theorem}

\begin{theorem}The Campanato spaces $\mathscr{L}^{p, \, n}(Q_0)$, $p \geq 1$, are all equivalent between them. For any couple of real numbers $1 \leq p \leq q$ it turns out that
\begin{equation} \label{qwoeoqw1} \frac{\alpha}{H^{1/q} \cdot \left[ \Gamma(q + 1) \right]^{1/q}} \cdot [u]_{ \mathscr{L}^{q, \, n}(Q_0)} \leq [u]_{ \mathscr{L}^{p, \, n}(Q_0)} \leq [u]_{ \mathscr{L}^{q, \, n}(Q_0)}.\end{equation} \end{theorem}