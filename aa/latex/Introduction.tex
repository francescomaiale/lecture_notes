\chapter{Introduction} \thispagestyle{empty}

In this first chapter, we want to motivate the necessity of studying the so-called \textit{distribution theory}, which is the natural generalisation of the concept of \textit{function}. Indeed, even a simple physical problem that consists of finding the electric potential of a point-charge cannot be solved, at least formally, if we do not introduce a broader class of objects.

\paragraph{Electric Potential.} Suppose that there is an electric charge $\rho$ at the origin of the $3$-dimensional space $\R^3$. The Maxwell equations describing this problem are simply given by
\begin{equation*} \begin{cases} \partial_x E_x + \partial_y E_y + \partial_z E_z = \rho \\[0.8em] \nabla u = \vec{E},\end{cases} \end{equation*}  
where $\vec{E}$ is the electric vector field generated by $\rho$, and $u$ denotes the electric potential. Clearly, this is equivalent\footnote{Use the well-known vector identity $\mathrm{div} \left[ \nabla (\cdot) \right] = \Delta (\cdot)$.} to the Laplace equation
\begin{equation} \label{eq.1} \Delta u = \rho, \end{equation} 
and, more precisely, to the fact that the laplacian of the potential $u$ equals $0$ whenever $(x, \, y, \, z) \neq \mathbf{0}$ and $\rho$ at the point $(x, \, y, \, z) = \mathbf{0}$.

Assume that $u$ is radially symmetric, that is, there exists a function $v$ defined on $[0, \, \infty)$ such that $u(x) = v(r)$ with $r = |x|$. Plugging the formula for the laplacian in polar coordinates into \eqref{eq.1} yields to a much easier differential equation, which is
\begin{equation*} v_{rr} + \frac{2}{r} v_r = 0\end{equation*} 
for all positive $r > 0$, where $v_r$ denotes the derivative of $v$ with respect to $r$. Therefore, up to a constant, we find that
\begin{equation*}v(r) = \frac{1}{r} \implies u(\mathbf{x}) = \frac{1}{|\mathbf{x}|} \quad \text{for all $\mathbf{x} = (x, \, y, \, z) \in \R^3 \setminus \{ \mathbf{0}\}$.}\end{equation*} 
We will now compute the "distributional" laplacian of the function $|x|^{-1}$ to check whether or not $u$ is a solution to the Laplace equation. Set
\begin{equation*}\left(\Delta u, \, \varphi \right) := \left(u, \, \Delta  \varphi\right) = \int_{\mathbb{R}^3} u(\mathbf{x})  \Delta \varphi(\mathbf{x}) \, \mathrm{d}\mathbf{x} \end{equation*}
for some $\varphi \in C_{c}^{\infty}\left(\mathbb{R}^3\right)$ smooth functions with compact support, and let $R > 0$ be such that $\mathrm{spt} \, \varphi$ is compactly contained in the ball of radius $R$. The function $u(x)$ is not well-defined at the origin, and thus to compute the value of the integral we simply remove a small ball of radius $\epsilon > 0$ and take the limit for $\epsilon \to 0^+$, that is, we have
\begin{equation*}\left(\Delta u, \, \varphi \right) = \lim_{\epsilon \to 0^+} \int_{B_R \setminus B_\epsilon} u(\mathbf{x}) \Delta \varphi(\mathbf{x}) \, \mathrm{d}\mathbf{x}. \end{equation*}
The divergence formula implies that
\begin{equation*} \begin{cases} u(\mathbf{x}) \Delta \varphi(\mathbf{x}) = u(\mathbf{x}) \mathrm{div}(\nabla \varphi)(\mathbf{x}) =  \mathrm{div}(u \nabla \varphi)(\mathbf{x}) - \nabla u \cdot \nabla \varphi(\mathbf{x}), \\[0.8em]  \varphi(\mathbf{x}) \Delta u(\mathbf{x}) = \varphi (\mathbf{x}) \mathrm{div}(\nabla u)(\mathbf{x}) = \mathrm{div}( \varphi \nabla u)(\mathbf{x}) - \nabla u \cdot \nabla \varphi(\mathbf{x}). \end{cases} \end{equation*}
Subtract these two identities. Since $\Delta u$ equals zero for all $\mathbf{x} \in B_R \setminus B_\epsilon$, we infer that
\begin{equation}\label{eq.fac.1} u (\mathbf{x}) \Delta \varphi (\mathbf{x}) = \mathrm{div}(u \nabla \varphi)(\mathbf{x}) - \mathrm{div}( \varphi  \nabla u)(\mathbf{x}). \end{equation}
Fix $\epsilon > 0$. The identity \eqref{eq.fac.1}, together with the diverge theorem, immediately shows that
\begin{equation*} \begin{aligned} \int_{B_R \setminus B_\epsilon} u(\mathbf{x})  \Delta \varphi(\mathbf{x}) \, \mathrm{d}\mathbf{x} & = \int_{\partial (B_R \setminus B_\epsilon)} \left[u(\mathbf{x}) \varphi_r(\mathbf{x}) - \varphi(\mathbf{x}) u_r(\mathbf{x}) \right]\, \mathrm{d}\sigma =
\\[1em] & = \int_{\partial B_\epsilon} \left[ \varphi(\mathbf{x}) u_r(\mathbf{x}) - u(\mathbf{x}) \varphi_r(\mathbf{x}) \right]\, \mathrm{d}\sigma, \end{aligned}\end{equation*}
since both $\varphi$ and $\varphi_r$ are identically equal to zero on the boundary of $B_R$ for the support is strictly contained in the open ball of radius $R$. Finally, we roughly estimate the remaining terms as
\begin{equation*}\int_{\partial B_\epsilon} \left[\varphi(\mathbf{x}) u_r(\mathbf{x}) - u(\mathbf{x}) \varphi_r(\mathbf{x}) \right]\, \mathrm{d}\sigma = \epsilon^{-1} \int_{\partial B_\epsilon}\varphi_r \, \mathrm{d}\sigma +\epsilon^{-2} \int_{\partial B_\epsilon} \varphi \mathrm{d}\sigma \simeq \mathrm{A}(B_\epsilon) \frac{1}{\epsilon} + \mathrm{A}(B_\epsilon) \frac{1}{\epsilon^2},\end{equation*}
where $\mathrm{A}(B_\epsilon)$ denotes the surface of the ball of radius $\epsilon > 0$. Therefore, the second term only survives at the limit, and the distributional laplacian is given by
\begin{equation*}(\Delta u, \, \varphi) = \lim_{\epsilon \to 0^+} \int_{B_R \setminus B_\epsilon} u(\mathbf{x}) \Delta \varphi(\mathbf{x}) \, \mathrm{d}x = 4 \pi. \end{equation*}
If we denote by $\delta(\cdot)$ the delta of Dirac at the origin, then the solution is a "function" $u(x)$ satisfying the following equation:
\begin{equation*} \Delta u(\mathbf{x}) = 4 \pi \delta(\mathbf{x}). \end{equation*}