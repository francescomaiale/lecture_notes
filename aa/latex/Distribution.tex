\chapter{Distribution Theory} \thispagestyle{empty}

The primary goal of this chapter is to give an overview of the distribution theory, focusing as much as possible on the properties of the convolution (e.g., regularity, associativity, etc.)

\section{Definitions and Main Properties}

Let $\Omega \subset \R^n$ be an open set. The space of \textit{distributions}\index{distribution}, or generalized functions, is the dual space of $\mathcal{D}(\Omega)$, that is, the space of all linear and continuous forms
\begin{equation*}f : \mathcal{D}(\Omega) \longrightarrow \C. \end{equation*}
Recall that the topology on $\mathcal{D}(\Omega)$ is induced by the family of seminorms $\mathcal{P} := \{ p_{K, \, \alpha} \}$, where $K$ ranges among all compact subsets of $\Omega$ and $\alpha \in \N^n$, may also be generated by the \textit{enlarged family}
\begin{equation*} \tilde{\mathcal{P}} := \{ \| \cdot \|_N \}_{N \in \N}, \end{equation*}
defined by setting
\begin{equation}\label{eq:normma} \| \varphi \|_N := \max_{|\alpha| \leq N} \max_{x \in \Omega} \left| D^\alpha  \varphi(x) \right|. \end{equation}
The topological results obtained in the previous chapters will now come into use; specifically, to characterise the continuity of a linear functional.

\begin{theorem} \label{conts}Let $f : \mathcal{D}(\Omega) \longrightarrow \C$ be a linear functional. Then $f$ is continuous if and only if for all compact subsets $K \subset \Omega$ we can find $c(K) := c > 0$ and $N := N(K) \in \N$ such that
\begin{equation*} \left| f(\varphi) \right| \lesssim_K \| \varphi \|_N \quad \text{for all $\varphi \in \mathcal{D}_K$.} \end{equation*} \end{theorem}

\begin{proof} We know (see \hyperref[theorem:contid]{Theorem \ref{theorem:contid}}) that a linear functional is continuous on $\mathcal{D}(\Omega)$ if and only if it is continuous on $\mathcal{D}_K$ for all compact subset $K \subset \Omega$. In addition, the increasing family $\{ \| \cdot \|_N \}_{N \in \N}$ induces a local basis of the origin in $\mathcal{D}_K$ given by the balls
\begin{equation*}B_{N, \, K}^r := r \cdot B_N := \left\{ \varphi \in \mathcal{D}_K \: \left| \: \| \varphi \|_N < r \right. \right\}. \end{equation*}
Thus $f$ is continuous on $\mathcal{D}_K$ if and only if there exist $\epsilon > 0$ and $N \in \N$ such that
\begin{equation*} f \left( \epsilon \cdot B_N \right) \subset B_\C. \end{equation*}
For a generic $\varphi \in \mathcal{D}_K$, we find that
\begin{equation*} \frac{ \epsilon \varphi }{ 2  \| \varphi \|_N } \in \epsilon \cdot B_N \implies \left| f(\varphi) \right| \leq \frac{2}{\epsilon} \|\varphi\|_N  \end{equation*}
and this completes the proof.\end{proof}

\begin{definition}[Order] \index{distribution!order} Let $f \in \mathcal{D}^\prime(\Omega)$ be a distribution. The \textit{order} of $f$ is the minimal natural number $N \in \N$ such that, for all compact subsets $K \subset \Omega$, there exists a constant $c(K) := c > 0$ such that $f$ satisfies
\begin{equation*} \left| f(\varphi) \right| \lesssim_K \| \varphi \|_N \quad \text{for all $\varphi \in \mathcal{D}_K$.} \end{equation*}
Furthermore, if such a natural number does not exists, we say that $f$ is a distribution of order $\infty$.  \end{definition}

\begin{example}[Dirac Delta] \index{Dirac delta} Let $p \in \R$. The \textit{Dirac delta} at the point $p$ is defined by
\begin{equation*} \delta_p: \mathcal{D}(\R) \ni \varphi \longmapsto \varphi(0) \in \C. \end{equation*}
The map obviously defined a distribution (as it is both linear and continuous) of order zero. Indeed, if we let $K \subset \Omega$ be a generic compact subset, then we have
\begin{equation*} \left| \delta_p(\varphi) \right| \leq \| \varphi \|_0 \end{equation*}
for all $\varphi \in \mathcal{D}(\R)$, which means that $\delta_p$ is continuous as a consequence of \hyperref[conts]{Theorem \ref{conts}}.\end{example}

\begin{example} The linear functional defined by setting
\begin{equation*}f(\varphi) := \sum_{j = 0}^{+ \infty} \varphi^\prime(j) \quad \text{for $\varphi \in \mathcal{D}(\R)$} \end{equation*}
is a distribution of order $1$. To prove this assertion, we start by considering the exhaustion by compact sets of $\R$ given by $K_M := \left[-M, \, M\right]$. Indeed, if we fix $M \in \N$, it turns out that
\begin{equation*}\left|f(\varphi) \right| =\left| \sum_{j = 0}^{M} \varphi^\prime(j) \right| \leq M \| \varphi \|_1 \quad \text{for all $\varphi \in \mathcal{D}_{K_M}$}. \end{equation*}
It remains to prove that the order is not $0$. Consider the function $\varphi_k \in \mathcal{D}_{K_1}$ such that $\varphi_k^\prime(0) = 1$ and $\| \varphi_k \|_0 \leq \frac{1}{k}$ for all $K \in \N$ - see \hyperref[fig:ex1]{Figure \ref{fig:ex1}} -. The reader should check that
\begin{equation*} \frac{ \left| f(\varphi_k) \right|}{\|\varphi_k\|_0} = k \xrightarrow{k \to + \infty} + \infty. \end{equation*} \end{example}

\begin{figure}[htp!]
\centering
\includegraphics[width = 9cm, height = 5cm]{Images/Ansup11.png}
\caption{Sketch of the Function}
\label{fig:ex1}
\end{figure} 

\begin{example} The linear functional defined by setting
\begin{equation*}f(\varphi) := \sum_{j = 0}^{+ \infty} \varphi^{(j)}(j)  \quad \text{for $\varphi \in \mathcal{D}(\R)$} \end{equation*}
is a distribution of order $\infty$. To prove this assertion, we start by considering the exhaustion by compact sets of $\R$ given by $K_M := \left[-M, \, M\right]$. Indeed, if we fix $M \in \N$, it turns out that
\begin{equation*}\left|f(\varphi) \right| =\left| \sum_{j = 0}^{M} \varphi^{(j)}(j) \right| \leq M \| \varphi \|_M \quad \text{for all $\varphi \in \mathcal{D}_{K_M}$}. \end{equation*}
It remains to prove that the order is not finite. Fix $M \in \N$, and consider the function $\varphi_{k, \, M} : K_M \longrightarrow \R$ defined in such a way that $\varphi_k^{(M)}(0) = 1$ and $\| \varphi_k \|_{M-1} \leq \frac{1}{k}$ for all $K \in \N$. The reader should check that
\begin{equation*} \frac{ \left| f(\varphi_k^{(M)}) \right|}{\|\varphi_k^{(M)}\|_{M - 1}} \xrightarrow{k \to + \infty} + \infty \quad \text{for all $M \in \N$}. \end{equation*} \end{example} 

\begin{example}[Locally Summable] Let $\Omega \in \subset \R^n$ be an open set. If $f \in L_{\mathrm{loc}}^1(\Omega)$, then we can define a distribution by setting
\begin{equation*}\Lambda_f(\varphi) := \int_{\Omega} f(x) \varphi(x) \, \mathrm{d}x \quad \text{for all $\varphi \in \mathcal{D}(\Omega)$.} \end{equation*}
The linearity is obvious (the integral itself is linear), while the continuity is an easy consequence of the following estimate:
\begin{equation*} \left| \Lambda_f(\varphi) \right| \leq \|f\|_{L^1(K)}  \| \varphi \|_0. \end{equation*}\end{example}

\begin{example}[Measure] Let $\Omega \in \subset \R^n$ be an open set. If $\mu$ is any Borel measure (or a locally finite positive measure) defined on $\Omega$, then we can define a distribution by setting
\begin{equation*}\Lambda_\mu(\varphi) := \int_{\Omega}\varphi(x) \, \mathrm{d} \mu(x) \quad\text{for all $\varphi \in \mathcal{D}(\Omega)$.} \end{equation*} \end{example}

\begin{definition}[Derivative] \index{distribution!derivative} Let $f \in \mathcal{D}^\prime(\Omega)$ be a distribution. The \textit{derivative} of $f$ is given by
\begin{equation} \label{distr:deriv}\partial_{x_k} f(\varphi) := - f \left( \partial_{x_k} \varphi \right) \quad \text{for $k= 1, \, \dots, \, n$}. \end{equation} \end{definition}

\begin{remark}The notion of derivative is well-defined. \end{remark}

\begin{proof} We need to show that $f^\prime$ is a distribution. Indeed, the linearity is obvious and the continuity is a consequence of \hyperref[conts]{Theorem \ref{conts}} since
\begin{equation*} \left| \partial_{x_k} f (\varphi) \right| \lesssim \left\| \partial_{x_k} \varphi \right\|_N \simeq \| \varphi \|_{N + 1}. \end{equation*} \end{proof}

\begin{example}[Heaviside] \index{Heaviside distribution} Let $H : \R \longrightarrow \R$ be the function defined by
\begin{equation*} H(x) = \begin{cases} 1 & \text{if $x \geq 0$,} \\[0.5em] 0 & \text{if $x < 0$}. \end{cases} \end{equation*}
This function is not differentiable, but it admits a distributional derivative. The functional associated to $H$ is given by $\Lambda_H$, and its derivative is
\begin{equation*} \partial_{x_k} \Lambda_{H}(\varphi) = - \int_\R H(x) \varphi^\prime(x) \, \mathrm{d}x = \varphi(0) \end{equation*}
for all $\varphi \in \mathcal{D}(\R)$. It follows that the distributional derivative of the Heaviside function is nothing but the Dirac delta at $0$.\end{example}

Let us consider the family of valuations $\mathcal{F}_v := \{ \lambda_\varphi \} \subset \mathcal{D}^{\prime \prime}(\Omega)$, and endow the space $\mathcal{D}^\prime(\Omega)$ with the weak-$\ast$ topology induced by this choice, that is,
\begin{equation*} \left( \mathcal{D}^\prime(\Omega), \, \sigma(\mathcal{D}^\prime(\Omega), \, \mathcal{F}_v) \right). \end{equation*}

\begin{remark} The valuation $\lambda_\varphi$, defined by
\begin{equation*} \lambda_\varphi(f) := f(\varphi) \end{equation*}
is continuous if and only if $\lambda_\varphi$ is continuous at the point $0$ if and only if the absolute value $|\lambda_\varphi|$ is continuous at the point $0$. \end{remark}

\begin{remark} The family of seminorms $\left\{ \left| \lambda_\varphi \right| \: : \: \varphi \in \mathcal{D}(\Omega)  \right\}$ is separated. \end{remark}

In particular, the weak-$\ast$ topology $\sigma(\mathcal{D}^\prime(\Omega), \, \mathcal{F}_v)$ is locally convex. A sequence of distributions $(f_n)_{n \in \N} \subset \mathcal{D}^\prime(\Omega)$ converges to a distribution $f$, and we denote it by $f_n \stackrel{\ast}{\rightharpoonup} f$, if and only if
\begin{equation*} \left| \lambda_\varphi \right| (f_n - f) \longrightarrow 0 \quad \text{for all $\varphi \in \mathcal{D}(\Omega)$}. \end{equation*}\index{distribution!weak-$\ast$ convergence}

\begin{theorem} Let $(f_n)_{n \in \N} \subset \mathcal{D}^\prime(\Omega)$ be a converging sequence, and let $f$ be its limit. For all multi-indices $\alpha \in \N^n$ we have that
\begin{equation*} D^\alpha f_n \weak D^\alpha  f. \end{equation*} \end{theorem}

\begin{proof}It follows immediately from the definitions. Indeed, notice that
\begin{equation*} D^\alpha  f_n(\varphi) = (-1)^{|\alpha|} f_n \left( D^\alpha \varphi \right) \xrightarrow{n \to + \infty} (-1)^{|\alpha|} f \left( D^\alpha  \varphi \right) = D^\alpha  f(\varphi) \quad \text{for all $\varphi \in \mathcal{D}(\Omega)$.} \end{equation*} \end{proof}

\begin{notation} Let $X$ and $Y$ be topological vector spaces, and let $\Gamma$ be a family (eventually uncountable) of linear and continuous applications from $X$ to $Y$. For every set $S \subset X$, we denote by $\Gamma(S)$ the union of the images, that is
\begin{equation*} \Gamma(S) := \bigcup_{T \in \Gamma} T(S), \end{equation*}
and, for every $R \subset Y$, we denote by $\Gamma^{-1}(R)$ the intersection of the preimages, that is
\begin{equation*} \Gamma^{-1}(R) := \bigcap_{T \in \Gamma} T^{-1}(R). \end{equation*}
In particular, the inclusion $\Gamma(S) \subset R$ is a compact way to express that
\begin{equation*} T(S) \subset R \quad \text{for every $T \in \Gamma$}, \end{equation*}
and, similarly, it is also equivalent to the inclusion
\begin{equation*} S \subset T^{-1}(R) \quad \text{for every $T \in \Gamma$}. \end{equation*}\end{notation}

\begin{definition}[Equicontinuous] \index{equicontinuous} Let $X$ and $Y$ be topological vector spaces. A family $\Gamma \subset \mathcal{L}(X, \, Y)$ of linear and continuous applications is \textit{equicontinuous} if and only if
\begin{equation*} \forall \, V \in \mathcal{U}_0(Y), \, \exists U \in \mathcal{U}_0(X)\: : \: \Gamma(U) \subset V. \end{equation*} \end{definition}

\begin{remark} If $X$ and $Y$ are metric spaces, this notion is completely equivalent to the equicontinuity in the sense of $\epsilon$-$\delta$.\end{remark}

\begin{remark}If $X$ and $Y$ are normed spaces, a family $\Gamma$ is equicontinuous if and only if $\Gamma$ is equibounded in $\mathcal{L}(X, \, Y)$ with respect to the operator norm.\end{remark}

\begin{definition}[Meager Set] \index{meager set} Let $(X, \, \tau)$ be a topological space, and let $S \subset X$. We say that $S$ is a \textit{meager} (or \textit{first-category}) set if and only if there exists a countable cover made up of nowhere dense subsets of $X$, that is,
\begin{equation*} S = \bigcup_{n \in \N} X_n, \qquad \mathrm{Int} \, \overline{X_n} = \varnothing. \end{equation*}
Furthermore, we say that $S$ is a \textit{second-category} set if it is not a first-category set. \end{definition}

\begin{theorem}[Baire] \index{Baire Theorem} \label{bairetheorem}Let $X$ be either a complete metric space or a locally compact topological space. Then each open nonempty subset of $X$ is a second-category set. \end{theorem}

\begin{theorem}[Banach-Steinhaus]\index{Banach-Steinhaus theorem} \label{stein}Let $X$ and $Y$ be topological vector spaces and let $\Gamma \subset \mathcal{L}(X, \, Y)$ be a collection (eventually uncountable) of linear and continuous applications. If
\begin{equation*} E := \left\{ x \in X \: \left| \: \text{$\Gamma(\{x\})$ is bounded} \right. \right\} \end{equation*}
is a second-category set (i.e., $\Gamma$ is pointwise bounded in a second-category set), then $\Gamma$ is a equicontinuous family. \end{theorem}

\begin{proof} Fix $U \in \mathcal{U}_0(Y)$ neighbourhood of the origin in $Y$. Recall that we can always find a neighbourhood $V \in \mathcal{U}_0(Y)$ closed, balanced and satisfying the the inclusion $V + V \subset U$.

\paragraph{Step 1.} By assumption, for all $x \in E$ there exists a positive natural number $m(x) \in \N$ such that $\Gamma(x) \subset m(x) \cdot V$ (or, equivalently, $x \in m(x) \cdot \Gamma^{-1}(V)$). In particular, we have that
\begin{equation*} E \subseteq \bigcup_{n \in \N} n \cdot \Gamma^{-1}(V) \quad \text{and} \quad \text{$\Gamma^{-1}(V)$ closed}. \end{equation*}

\paragraph{Step 2.} Since $E$ is a second-category set, there exists $m \in \N$ such that $m \cdot \Gamma^{-1}(V)$ has nonempty internal part (and, hence, the same applies to $n \cdot \Gamma^{-1}(V)$ for every $n \in \N$). In particular,
\begin{equation*}\mathfrak{V} := \Gamma^{-1}(V) - \Gamma^{-1}(V) \in \mathcal{U}_0(X), \end{equation*}
and by linearity of $\Gamma$ we also have that
\begin{equation*} \Gamma(\mathfrak{V}) = \Gamma(\Gamma^{-1}(V)) - \Gamma(\Gamma^{-1}(V)) = V - V \subseteq U. \end{equation*}
Therefore, the set $\mathfrak{V}$ is a neighbourhood of the origin in $X$ whose image is contained in $U$, and this is exactly what we wanted to prove.\end{proof}

\begin{theorem} Let $(f_n)_{n \in \N} \subset \mathcal{D}^\prime(\Omega)$ be a sequence of distributions such that the limit
\begin{equation*} f(\varphi) := \lim_{n \to + \infty} f_n(\varphi) \end{equation*}
exists and is finite for all $\varphi \in \mathcal{D}(\Omega)$. Then $f$ is also a distribution and $f_n \weak f$. \end{theorem}

\begin{proof} The limit defined by $f$ is linear. Thus, it suffices to show that $f$ is continuous or, equivalently, that $f$ is continuous on $\mathcal{D}_K$ for all compact subset $K \subset \Omega$. Now consider the family
\begin{equation*} \mathcal{F} = \{ f_n : \mathcal{D}_K \longrightarrow \C \}_{n \in \N}\end{equation*}
and apply the \hyperref[stein]{Banach-Steinhaus Theorem}. It follows that $\mathcal{F}$ is equicontinuous, and thus for all $\epsilon > 0$ there exists a neighbourhood $U$ of the origin in $\mathcal{D}_K$ such that $f_n(U) \subset B_\epsilon(0)$. Then
\begin{equation*} f(U) \subset \overline{B_\epsilon(0)} \implies \text{$f$ continuous}. \end{equation*}
\end{proof}

\begin{definition}[Product Distribution] \index{distribution!product} Let $f \in \mathcal{D}^\prime(\Omega)$, and let $g \in C^\infty(\Omega)$. The product of $f$ and $g$ is defined by setting
\begin{equation} \label{distr:prod} g \cdot f(\varphi) := f( g \cdot \varphi) \quad \text{for all $\varphi \in \mathcal{D}(\Omega)$}. \end{equation} \end{definition}

\begin{remark}The notion of product is well-defined. \end{remark}

\begin{proof}It is enough to prove that the product is a distribution. The linearity is obvious, and the continuity is a consequence of the Leibniz rule. Indeed, it turns out that
\begin{equation*} \left| g \cdot f(\varphi) \right| \lesssim_{g, \, \dots, \, g^{(N)}, \, K} \| \varphi \|_{N(K)} \quad \text{for all $\varphi \in \mathcal{D}_K$},\end{equation*}
Notice that the Leibniz rule holds true and it is particularly simple in the case of the product between a distribution and a regular function:
\begin{equation*}\begin{aligned} D^\alpha  (g \cdot f)(\varphi) & = \sum_{\beta \leq \alpha} \binom{\alpha}{\beta} D^\beta g  D^{\alpha - \beta}  f (\varphi) = \\[1em] & = \sum_{\beta \leq \alpha} (-1)^{|\alpha| - |\beta|} \binom{\alpha}{\beta} f \left( D^{\alpha - \beta} \varphi D^\beta g \right). \end{aligned} \end{equation*} \end{proof}

\begin{theorem} Let $X$ be a complete metric space, and let $Y$ and $Z$ be topological vector spaces. If
\begin{equation*} B : X \times Y \longrightarrow Z \end{equation*}
is a bilinear application separately sequentially continuous, then $B$ is jointly sequentially continuous, that is, sequentially continuous with respect to the couple. \end{theorem}

\begin{proof} We first prove a particular case, and then we generalise it with a simple algebraic trick.

\paragraph{Step 1.} Let $(x_n)_{n \in \N} \subset X$ be a sequence converging to $0$, and let $(y_n)_{n \in \N} \subset Y$ be a converging sequence. The linear mapping
\begin{equation*} B(\cdot, \, y_n) : X \longrightarrow Z \end{equation*}
is clearly continuous for every $n \in \N$, and it is pointwise bounded. Indeed, for $u \in X$ fixed, the subset
\begin{equation*} \left\{ B(u, \, y_n) \right\}_{n \in \N} \subset Z \end{equation*}
is bounded in $Z$, since converging sequences are bounded. The \hyperref[bairetheorem]{Baire Theorem \ref{bairetheorem}} holds in the complete metric space $X$, and hence the \hyperref[stein]{Banach-Steinhaus Theorem \ref{stein}} implies that for every $U \in \mathcal{U}_0(Z)$ there exists $V \in \mathcal{U}_0(X)$ such that
\begin{equation*} B \left( V, \, y_n \right) \subset U \quad \text{for every $n \in \N$}. \end{equation*}
The sequence $(x_n)_{n \in \N}$ converges to $0$; thus, $x_n \in V$ definitively, and this means that $B(x_n, \, y_n)$ belongs to $U$ for $n$ sufficiently large, that is, $B(x_n, \, y_n) \to 0$.

\paragraph{Step 2.} If $x_n \to x \in X$, then the thesis follows from the previous step if one notices that the difference $(x_n - x)_{n \in \N}$ converges to $0$. Then
\begin{equation*} B(x_n, \, y_n) = B(x_n - x, \, y_n) + B(x, \, y_n) \xrightarrow{n \to + \infty} 0 + B(x, \, y) = B(x, \, y).\end{equation*}
\end{proof}

\begin{corollary}Let $(f_n)_{n \in \N} \subset \mathcal{D}^\prime(\Omega)$ be a sequence of distributions converging to some $f$, and let $(g_n)_{n \in \N} \subset C^\infty(\Omega)$ be a sequence to a smooth function $g$. Then
\begin{equation*} g_n \cdot f_n \weak g \cdot f. \end{equation*}
\end{corollary}

\section{Localization and Support of a Distribution}

Let $\omega \subset \Omega$ be two open subsets of $\R^n$. There is a natural continuous inclusion $\mathcal{D}(\omega) \subset \mathcal{D}(\Omega)$, which induces the opposite inclusion of the dual spaces
\begin{equation*} \mathcal{D}^\prime(\Omega) \hookrightarrow \mathcal{D}^\prime(\omega). \end{equation*}
The inclusion is continuous, and thus a distribution $f \in \mathcal{D}^\prime(\Omega)$ is zero in $\mathcal{D}^\prime(\omega)$ if and only if
\begin{equation*} f(\varphi) = 0 \quad \text{for all $\varphi \in \mathcal{D}(\omega)$.} \end{equation*}
In this section, we are mainly interested in the "opposite" implication, that is, given a collection of compatibles distributions on a open cover of $\Omega$, find a global distribution $f \in \mathcal{D}^\prime(\Omega)$ such that
\begin{equation*}f \, \big|_{\Omega_\alpha} \equiv f_\alpha. \end{equation*}

The fundamental tool here to define globally something that has already been defined locally is the well-known \textit{partition of the unity}\index{partition of the unity}. In particular, we associate to a cover of the space a set of functions which "glue together the local properties."

\begin{theorem}[Partition of Unity] \label{partunit} Let $\mathcal{F}$ be a collection of open subsets of $\R^n$ whose union is equal to $\Omega$, that is,
\begin{equation*}\Omega = \bigcup_{\alpha} \Omega_\alpha. \end{equation*}
Then there exist a collection of compactly supported functions $\{ \psi_n \}_{n \in \N}$ such that the following properties hold: \mbox{}
\begin{enumerate}[label=\textbf{(\arabic*)}]
\item The function $\psi_n$ is nonnegative for all $n \in \N$.
\item For all $n \in \N$ there exists $\alpha$ such that $\Omega_\alpha \in \mathcal{F}$ satisfies
\begin{equation*} \mathrm{spt} \, \psi_n \subset \Omega_\alpha. \end{equation*}
\item For all $x \in \Omega$ we have
\begin{equation*} \sum_{n \in \N} \psi_n(x) = 1. \end{equation*}
\item For all compact subset $K \subset \Omega$ we can find an open neighbourhood $W \supset K$ such that, for all $x \in W$, we have $\psi_n(x) \neq 0$ for finitely many $n \in \N$.
\end{enumerate}
\end{theorem}

\begin{proof} Let us consider the family of balls
\begin{equation*} \mathcal{B} := \{B_{x, \, r} \: : \: B_{x, \, r} \subset \Omega_\alpha, \, x \in \Q^n, \, r \in \Q^+\}. \end{equation*}
This family is obviously countable, and its union contains the whole $\Omega$, that is,
\begin{equation*} \bigcup_{n \in \N} B_n \supset \Omega.\end{equation*}
Consider now the scaled family of balls
\begin{equation*} \mathcal{V} := \left\{ \frac{1}{2} B_{x, \, r} \: : \: B_{x, \, r} \in \mathcal{B} \right\} \end{equation*}
and notice that $\mathcal{V}$ is also a cover of $\Omega$.

\paragraph{Step 1.} Let $\{\varphi_n\}_{n \in \N}$ be a countable collection of functions in $\mathcal{D}(\Omega)$ defined in such a way that
\begin{equation*} \varphi_n(x) = \begin{cases} 1 & \text{if $x \in V_n$}, \\[0.5em] 0 & \text{if $x \notin B_n$}, \\[0.5em] \in (0, \, 1) & \text{if $x \in B_n \setminus V_n$}. \end{cases} \end{equation*} 
We can define the desired partition of the unity by induction as follows. Let $\psi_0 := \varphi_0$, and let
\begin{equation*} \psi_{n + 1} := (1 - \varphi_0) \dots (1 - \varphi_n) \cdot \varphi_{n+1} \quad \text{for all $n \geq 1$}.\end{equation*}
A straightforward computation shows that $\psi_n$ is positive for all $n \in \N$ and, by construction, the support of $\psi_n$ is contained in the support of $\varphi_n$, which is contained in some $B_n$. Furthermore,
\begin{equation*} \sum_{n = 0}^{N} \psi_n = 1 - \prod_{n = 0}^{N} (1 - \varphi_n), \end{equation*}
and thus, since for all $x \in \Omega$ there exists a natural number $m \in \N$ such that $x \in V_m$, we have 
\begin{equation*} \sum_{n = 0}^{m} \psi_n = 1 - \prod_{n = 0}^{m} (1 - \varphi_n) = 1. \end{equation*}
In particular, for a given $x \in \Omega$, only finitely many $\psi_n(x)$ are nonzero.

\paragraph{Step 2.} Let $K \subset \Omega$ be a compact subset. There exists a natural number $m \in \N$ such that $K \subset \cup_{i=1}^{m} V{n_i}$. It follows that
\begin{equation*} \sum_{n = 0}^{m} \psi_n = 1 - \prod_{n = 0}^{m} (1 - \varphi_n) = 1, \end{equation*}
which is exactly what we wanted to prove.
\end{proof}

\begin{theorem} \label{theormefdf}Let $\mathcal{F}$ be an open cover of $\Omega$. Suppose that for all $\omega \in \mathcal{F}$ there is a distribution $f_\omega \in \mathcal{D}^\prime(\omega)$, and assume also that for all $\omega, \, \omega^\prime \in \mathcal{F}$ the corresponding distributions satisfy
\begin{equation*} f_\omega(\varphi) = f_{\omega^\prime}(\varphi) \quad \text{for all $\varphi \in \mathcal{D}( \omega \cap \omega^\prime)$.} \end{equation*}
Then there exists a unique $f \in \mathcal{D}^\prime(\Omega)$ such that
\begin{equation*} f(\varphi) = f_\omega(\varphi) \quad \text{for all $\varphi \in \mathcal{D}(\omega)$}. \end{equation*}
\end{theorem}

\begin{proof} Let $\{ \psi_n \}_{n \in \N}$ be the partition of unity constructed in \hyperref[partunit]{Theorem \ref{partunit}}, and let $\omega_n$ be the element of $\mathcal{F}$ such that the support of $\psi_n$ is contained in $\omega_n$.

\paragraph{Step 1.} Let $\varphi \in \mathcal{D}(\Omega)$ be given. Then
\begin{equation*} \varphi = \sum_{n \in \N} \psi_n \cdot \varphi, \end{equation*}
and only a finite number of addendum is different from zero since we can always find a compact subset $K \subset \Omega$ such that $\mathrm{spt} \, \varphi \subset K$. Define the functional
\begin{equation*} f(\varphi) := \sum_{n \in \N} f_{\omega_n} \left( \psi_n \cdot \varphi \right). \end{equation*}
It is easy to see that $f$ is linear on $\mathcal{D}(\Omega)$; hence we only need to prove that it is also continuous.

\paragraph{Step 2.} Let $(\varphi_k)_{k \in \N} \subset \mathcal{D}(\Omega)$ be a sequence converging to zero. We know that there exists a compact subset $K \subset \Omega$ that contains the support of $\varphi_k$ for all $k \in \N$. It follows that
\begin{equation*} f(\varphi_k) := \sum_{n=0}^{m} f_{\omega_n} \left( \psi_n \cdot \varphi_j \right) \quad \text{for all $j \in \N$}\end{equation*}
for some fixed $m \in \N$. The sequence $(\psi_n \cdot \varphi_j)_{j \in \N}$ converges to $0$ in $\mathcal{D}(\omega_n)$ for all $n \in \N$ fixed; therefore 
\begin{equation*} f_{\omega_n} ( \psi_n \cdot \varphi_j ) \to 0 \quad \text{for all $n \in \N$}. \end{equation*}
It follows that $f(\varphi_k) \to 0$, which means that $f$ is continuous (and thus a distribution.)

\paragraph{Step 3.} To prove that $f$ coincides with $f_\omega$ on each open subset $\omega \in \mathcal{F}$, notice that, given a smooth function $\varphi \in \mathcal{D}(\omega)$, the product $\psi_n \cdot \varphi$ belongs to $\mathcal{D}(\omega_n \cap \omega)$. It follows from the assumptions
\begin{equation*}f_{\omega_n}(\psi_n \cdot \varphi) = f_\omega ( \psi_n \cdot \varphi), \end{equation*}
which, in turn, implies the thesis:
\begin{equation*}f(\varphi) = \sum_{n \in \N} f_\omega \left( \psi_n \cdot \varphi \right) = f_\omega \left( \sum_{n \in \N} \psi_n \cdot \varphi \right) = f_\omega(\varphi). \end{equation*}
\end{proof}

\begin{corollary} Let $f, \, g \in \mathcal{D}^\prime(\Omega)$, and let $\mathcal{F}$ be an open cover of $\Omega$. Suppose that
\begin{equation*} f \, \big|_{\mathcal{D}^\prime(\omega)} = g \, \big|_{\mathcal{D}^\prime(\omega)} \end{equation*} 
for all $\omega \in \mathcal{F}$. Then $f = g$ as distributions in $\mathcal{D}^\prime(\Omega$. \end{corollary}

\begin{definition}[Support]\index{distribution!support} Let $f \in \mathcal{D}^\prime(\Omega)$ be a distribution. The \textit{support} of $f$ is defined by
\begin{equation} \label{spt} \mathrm{spt} \, f := \Omega \setminus \bigcup \left\{ \omega \subset \Omega \: \left| \: \text{$\omega$ is an open subset such that $f \equiv 0$ on $\mathcal{D}^\prime(\omega)$} \right. \right\}. \end{equation} \end{definition}

\begin{remark}The notion is well-defined since $f$ is actually zero as an element of $\mathcal{D}^\prime \left( \Omega \setminus \mathrm{spt} \, f \right)$. \end{remark}

\begin{theorem} \label{theorem:support}Let $f \in \mathcal{D}^\prime(\Omega)$, and let $S := \mathrm{spt} \, f$. Then the following properties hold: \mbox{}
\begin{enumerate}[label=\textbf{(\alph*)}]
\item If $\varphi \in \mathcal{D}(\Omega)$ has support disjoint from $S$, then $f(\varphi) = 0$.
\item If $S = \varnothing$, then $f = 0$.
\item If $g \in C^\infty(\Omega)$ is a smooth function such that $g \equiv 1$ on some open subset $U$ containing $S$, then $g \cdot f = f$.
\item If $S$ is compact, then there exists $N \in \N$ such that
\begin{equation*} \left| f(\varphi) \right| \lesssim \| \varphi \|_N. \end{equation*}
In particular, the order of $f$ is finite and it can be extended in a unique to a linear and continuous functional on $\mathcal{E} = C^\infty(\Omega)$, that is, $f \in \mathcal{E}^\prime(\Omega)$.
\end{enumerate}
\end{theorem}

\begin{proof} \mbox{}
\begin{enumerate}[label=\textbf{(\alph*)}]
\item If the support of $\varphi$ does not intersect $S$, then $\varphi \in \mathcal{D} \left( \Omega \setminus S \right)$. But $f$ is equal to $0$ outside of its support, and thus $f(\varphi) = 0$.
\item Obvious.
\item  Let $U \supset S$. Then $\mathcal{F} := \{ U, \, \Omega \setminus S \}$ is an open cover of $\Omega$, and it is easy to check that
\begin{equation*}f \, \big|_{\mathcal{D}^\prime(\Omega \setminus S)} = 0 = g \cdot f \, \big|_{\mathcal{D}^\prime(\Omega \setminus S)} \quad \text{and} \quad g \cdot f \, \big|_{\mathcal{D}^\prime(U)} = 1 \cdot f = f \, \big|_{\mathcal{D}^\prime(U)}. \end{equation*}
\item First, we prove that there exists a function $g \in \mathcal{D}(\Omega)$ such that $g \, \big|_U \equiv 1$ for some open set $U$ containing the support $S$. Let $\{B_n\}_{n = 1, \, \dots, \, N}$ be a finite collection of balls such that their closure is contained in $\Omega$ and 
\begin{equation*} \bigcup_{n = 1}^N\frac{1}{2} \cdot B_n \supset S. \end{equation*}
Then
\begin{equation*} \left\{ \Omega \setminus S, \, B_1, \, \dots, \, B_N \right\} \end{equation*}
is an open cover of $\Omega$, and therefore we can find a partition of unity $\{ \psi_S, \, \psi_1, \, \dots, \, \psi_N \}$ such that $\psi_n \, \big|_{ \frac{1}{2} \cdot B_n } \equiv 1$ for all $n = 1, \,\dots, \, n$. The reader should check that
\begin{equation*} g(x) = \sum_{n = 1}^{N} \psi_n(x) \end{equation*}
does the work. We now know that $g \cdot f = f$ and, for any $\varphi \in \mathcal{D}(\Omega)$, it turns out that $g \cdot \varphi \in \mathcal{D}_K$ - where $K := \mathrm{spt} \, g$ -. But on $\mathcal{D}_K$ we can estimate the absolute value of the functional as follows:
\begin{equation*} \left| f( g \cdot \varphi ) \right| \lesssim \| g \cdot \varphi \|_{N}, \end{equation*}
where $c > 0$ and $N \in \N$. It follows from the Leibniz rule that there exists a constant $c(g) > 0$ such that
\begin{equation*} \left| f( \varphi ) \right| \lesssim_g \| \varphi \|_{N}. \end{equation*}
To prove that $f$ can be uniquely extended to a linear and continuous functional on $\mathcal{E}$, we set
\begin{equation*} f(\varphi) := g \cdot f(\varphi) = f(g \cdot \varphi) \quad \text{for all $\varphi \in \mathcal{E}$}. \end{equation*}
The support of $g \cdot f$ is the intersection of $\mathrm{spt} \, g$ and $S$, which is also compact.
\end{enumerate} \end{proof}

\begin{lemma} \label{lemma:algebrico} Let $X$ be a vector space, and let $\xi, \, \xi_1, \, \dots, \, \xi_n \in X^\prime$. Then
\begin{equation*} \mathrm{ker} \, \xi \supseteq \bigcap_{i = 1}^n \mathrm{ker} \, \xi_i \iff \xi(x) = \sum_{i = 1}^n \lambda_i \xi_i(x). \end{equation*}\end{lemma}

\begin{proof} Let us consider the operator $f : X \longrightarrow \mathbb{K}^n$ defined by
\begin{equation*} f(x) = \left( \xi_1(x), \, \dots, \, \xi_n(x) \right). \end{equation*}
Suppose that $\mathrm{ker} \, f$ is a subset (eventually proper) of the kernel of $\xi$. We know (algebraic geometry) that we can always find a linear mapping $h : \mathbb{K}^n \longrightarrow \mathbb{K}$ such that the diagram
\begin{equation*}\begin{tikzcd}
X \arrow{r}{f} \arrow{dr}{\xi} & \mathbb{K}^n \arrow{d}{h} \\[1em]
 & \mathbb{K}
\end{tikzcd}  \end{equation*}
commutes. Therefore, if $f(y_1, \, \dots, \, y_n) = \sum_{i=1}^{n} \lambda_i y_i$, then
\begin{equation*} \xi(x) = \sum_{i = 1}^{n} \lambda_i \cdot \xi_i(x). \end{equation*}
\end{proof}

\section{Derivative Form of Distributions}

In this section, the primary goal is to prove that every distribution is, in some sense, the $\alpha$th derivative of a specific continuous function.

\begin{theorem} Let $f \in \mathcal{D}^\prime(\Omega)$, and let $p \in \Omega$. Assume that the support of $f$ is contained in $\{p\}$. Then there exist a natural number $N \in \N$ and constants $c_\alpha > 0$ such that
\begin{equation*} f = \sum_{\left| \alpha \right| \leq N} c_\alpha \cdot D^\alpha \delta_p. \end{equation*}\end{theorem}

\begin{proof} We may assume, without loss of generality, that $p$ is the origin.

\paragraph{Step 1.} By \hyperref[theorem:support]{Theorem \ref{theorem:support}} the order of $f$ is finite, and the existence of the natural number $N \in \N$ is obvious. Furthermore, it follows from \hyperref[lemma:algebrico]{Lemma \ref{lemma:algebrico}} that it is enough to show that $f(\varphi) = 0$ for every $\varphi$ such that
\begin{equation*} D^\alpha \delta_0(\varphi) = 0 \quad \text{for all $\alpha$ such that $\left|\alpha\right|\leq N$.} \end{equation*}
We can work in a compact set containing the origin, that is, $K = \overline{B_r(0)}$. In particular, there exists $c > 0$ such that
\begin{equation*} \left| f(\varphi) \right| \lesssim \|\varphi\|_N \quad \text{for all $\varphi \in \mathcal{D}_{K}$}. \end{equation*}
Let $g \in \mathcal{D}(\R^n)$ be a function such that
\begin{equation*} g(x) = \begin{cases} 1 & \text{if $x \in B_1(0)$}, \\[0.5em] 0 & \text{if $x \notin B_2(0)$},
\\[0.5em] \in (0, \,1) & \text{otherwise}, \end{cases} \end{equation*}
and consider the scalings as $\rho > 0$
\begin{equation*} g_\rho(x) := g \left( \frac{x}{\rho} \right). \end{equation*}
By \hyperref[theorem:support]{Theorem \ref{theorem:support}} it follows that there exists $\rho > 0$ such that $f = g_\rho \cdot f$ and
\begin{equation*} \rho < \frac{r}{2} \implies \mathrm{spt} \, g_\rho \subset K. \end{equation*}
The estimate above implies that
\begin{equation*} \left| f(\varphi) \right| \lesssim \| g_\rho \cdot \varphi\|_N \end{equation*}
for all $\varphi \in \mathcal{D}_{\bar{B}_r}$, which means that we only need to estimate the derivative of the product between $\varphi$ and $g_\rho$.

\paragraph{Step 2.} For every $\epsilon > 0$ we can find $\ell > 0$ such that the $N$-th derivative of $\varphi$ is smaller than $\epsilon$, that is,
\begin{equation*} \left| D^\alpha \varphi(x) \right| \leq \epsilon \quad \text{for all $|x| \leq \ell$ and $|\alpha| = N$}. \end{equation*}
A straightforward computation, together with the induction principle, shows that
\begin{equation*} \left| D^\alpha \varphi(x) \right| \lesssim \epsilon \left|x\right|^{N - \left|\alpha\right|} \quad \text{for all $|x| \leq \ell$}. \end{equation*}
Note that the constant does not depend on $\varphi$ or on the value of $\epsilon$. Similarly, we may estimate the derivatives of the function $g_\rho$ as follows:
\begin{equation*} \left| D^\alpha g_\rho(x) \right| \lesssim \frac{1}{\rho^{|\alpha|}} \quad \text{for all $\rho> 0$}. \end{equation*}
If we now choose $\rho$ to be exactly equal to $\ell$, we obtain the chain of inequalities
\begin{equation*} \begin{aligned} \left| D^\alpha \left( g_\rho \cdot \varphi \right)(x) \right| & \leq \sum_{\beta \leq \alpha} \binom{\alpha}{\beta} \left| D^\beta g_\rho(x) \right|  \left| D^{\alpha - \beta} \varphi(x) \right| \lesssim
\\[1em] & \lesssim  \sum_{\beta \leq \alpha} \binom{\alpha}{\beta} \frac{1}{\ell^{|\beta|}}  \epsilon \cdot |x|^{N - |\alpha| - |\beta|} \lesssim
\\[1em] & \lesssim \epsilon \sum_{\beta \leq \alpha} \ell^{N - |\alpha|} \lesssim \epsilon, \end{aligned} \end{equation*}
and by the arbitrariness of $\epsilon > 0$ we conclude that $f(\varphi) = 0$.
\end{proof}

\begin{theorem} \label{derivatikds}Let $\Omega \subset \R^n$ be an open set. Let $f \in \mathcal{D}^\prime(\Omega)$ be a distribution, and let $K$ be a compact subset of $\Omega$. Then there exists a continuous function $g \in C^0(\Omega; \; \C)$ such that
\begin{equation*} f(\varphi) = (-1)^{|\alpha|} \int_{\Omega} g(x)  D^\alpha \varphi(x) \, \mathrm{d}x \quad \text{for all $\varphi \in \mathcal{D}_K$}. \end{equation*} \end{theorem}

\begin{proof}We may assume, without loss of generality, that $K \subset Q := [0, \, 1]^n$. Furthermore, given $\varphi \in \mathcal{D}_K$, we consider the null extension outside of $K$, still denoted by $\varphi$, in such a way that $\varphi \in \mathcal{D}_Q$. 


\paragraph{Step 1.} We use the Poincaré inequality to estimate $\varphi$ by a higher-order derivative. Indeed, we known (e.g., Lagrange theorem) that for all $x \in Q$ there is a point $y \in [0, \, 1]$ such that
\begin{equation*} \varphi(x) = \varphi(x_1, \, \dots, \, x_n) - \varphi(0, \, x_2, \, \dots, \, x_n) = \partial_{x_1} \, \varphi(y, \, \dots, \, x_n). \end{equation*}
It follows that
\begin{equation*} \| \varphi \|_0 \lesssim \left\| \partial_{x_1} \varphi \right\|_{0}, \end{equation*}
and therefore (by induction) for all $\mathbf{N} := (N, \, \dots, \, N) \in \N^n$ we have that
\begin{equation*} \| \varphi \|_N \lesssim \| D^{\mathbf{N}} \varphi \|_0. \end{equation*}
We now observe that a function $\varphi \in \mathcal{D}_Q$, for $x \in Q$, satisfies
\begin{equation*} \varphi(x) = \int_{0}^{x_1} \partial_{x_1} \varphi(y, \, \dots, \, x_n) \, \mathrm{d}y, \end{equation*}
and thus, if we repeat the same process for all the directions, we obtain
\begin{equation} \label{eq:utile} \varphi(x) = \int_{Q(x)} D^{\mathbf{1}} \varphi (y) \, \mathrm{d}y, \end{equation}
where $Q(x) := [0, \, x_1] \times \cdot \times [0, \, x_n]$. In particular, the uniform norm of $\varphi$ may be estimated with the $L^1$-norm of the derivatives, that is,
\begin{equation*} \left\| \varphi \right\|_0 \leq \| D^{\mathbf{1}} \varphi \|_{L^1(Q)}. \end{equation*}

\paragraph{Step 2.} We know that, for any fixed $K \subset \Omega$, there exist $c := c(K) > 0$ and $N := N(K) \in \N$ such that
\begin{equation*} \left| f(\varphi) \right| \lesssim_K \| \varphi \|_N \quad \text{for all $\varphi \in \mathcal{D}_K$}. \end{equation*}
Now the previous estimates show that $f$ can also be estimated using the $L^1$-norm of the $(\mathbf{N} + \mathbf{1})$-order derivative, that is,
\begin{equation*} \left| f(\varphi) \right| \lesssim \| \varphi^{\mathbf{N}} \|_0 \lesssim \| D^{\mathbf{N} + \mathbf{1}}  \varphi \|_{L^1(Q)}. \end{equation*}
The integral formula \eqref{eq:utile} implies the injectivity of the differential operator $D^{\mathbf{1}}$, and thus also the iterate $D^{\mathbf{N} + \mathbf{1}}$ is injective as well. If we set
\begin{equation*} V  := D^{\mathbf{N} + \mathbf{1}}  \left( \mathcal{D}_Q \right) = \left\{ D^{\mathbf{N} + \mathbf{1}}  \varphi \: \left| \: \varphi \in \mathcal{D}_Q \right. \right\}, \end{equation*}
then it is possible to find a linear functional $L : V \longrightarrow \C$ such that the following diagram is commutative:
\begin{equation*}\begin{tikzcd}
\mathcal{D}_Q \arrow{r}{D^{\mathbf{N} + \mathbf{1}}} \arrow{dr}{f} & V \arrow{d}{L} \\[1em]
 & \C
\end{tikzcd}  \end{equation*}
that is, a functional such that
\begin{equation*} L \left( D^{\mathbf{N} + \mathbf{1}} \varphi \right) = f(\varphi). \end{equation*}
We observe that $L$ is continuous with respect to the $L^1$-norm since
\begin{equation*} \left| L(\psi) \right| \lesssim \int_Q \left| \psi(x) \right| \, \mathrm{d}x.\end{equation*}

\paragraph{Step 3.} The Hahn-Banach theorem extends $L$ to a functional $\tilde{L}$ linear, continuous and defined on the whole space $L^1(Q)$. Furthermore, the Riesz theorem allows us to find a representation of $\widetilde{L}$ with a function $h \in L^\infty(Q)$, which is the dual of $L^1$, that is,
\begin{equation*} \widetilde{L}(\psi) := \int_{Q} h(x) \psi(x) \, \mathrm{d}x. \end{equation*}
Therefore, for all $\varphi \in \mathcal{D}_K$, we have that
\begin{equation*} f(\varphi) = \int_{Q} h(x) D^{\mathbf{N} + \mathbf{1}} \varphi(x) \, \mathrm{d}x.\end{equation*}
To obtain a continuous function, we need to make another integration. Set
\begin{equation*} g(x) := \int_{Q(x)} h(y) \, \mathrm{d}y. \end{equation*}
The integral is absolutely continuous, and thus $f$ is a Lipschitz-regular on the whole $Q$. It remains to prove that the distributional derivative of $f$ is equal to $g$. For all $\varphi \in \mathcal{D}_Q$ we have
\begin{equation*} \begin{aligned} \int_Q g(x)  D^{\mathbf{1}} \varphi(x) \, \mathrm{d}x & = \int_Q \left( \int_{0}^{x_1} \dots \int_{0}^{x_n} h(y) \, \mathrm{d}y_n \dots \mathrm{d}y_1 \right) D^{\mathbf{1}}  \varphi(x) \, \mathrm{d}x =
\\[1em] & = \int_Q h(y) \left( \int_{y_1}^{1} \dots \int_{y_n}^{1}D^{\mathbf{1}} \varphi(x)  \, \mathrm{d}x_n \dots \mathrm{d}x_1 \right) \mathrm{d}y =
\\[1em] & = (-1)^n \int_Q h(y)  \varphi(y) \, \mathrm{d}y. \end{aligned} \end{equation*}
Therefore, we conclude by observing that
\begin{equation*}f(\varphi) = \int_{Q} h(x) D^{\mathbf{N} + \mathbf{1}} \varphi(x) \, \mathrm{d}x = (-1)^n \int_Q g(x) D^{\mathbf{N} + \mathbf{2}} \varphi(x) \, \mathrm{d}x.\end{equation*}
\end{proof}

\begin{theorem}\label{theorem:ccsod}Let $f \in \mathcal{D}^\prime(\Omega)$ be a distribution with support contained in a compact set $K \subset \Omega$, and let $V$ be an open set such that $K \subset V \subset \Omega$. Then there are $f_1, \, \dots, \, f_m : \Omega \longrightarrow \C$ continuous functions, whose support is contained in $K$, and multi-indices $\alpha_1, \, \dots, \, \alpha_m \in \N^n$ such that
\begin{equation*} f = \sum_{k = 1}^{m} D^{\alpha_k} \Lambda_{f_k}. \end{equation*}\end{theorem}

\begin{proof} Let $W$ be an open set containing $K$, whose closure is compact in $\Omega$. Let $\psi \in \mathcal{D}(\Omega)$ be a smooth function such that
\begin{equation*} \psi(x) = \begin{cases} 1 & \text{if $x \in K$}, \\[0.5em] 0 & \text{if $x \notin W$}, \\[0.5em] \in (0, \, 1) & \text{otherwise.} \end{cases} \end{equation*}
By \hyperref[theorem:support]{Theorem \ref{theorem:support}} it follows that $f \cdot \psi = f$, and thus
\begin{equation*} f(\varphi) = f(\varphi \cdot \psi) \quad \text{for all $\varphi \in \mathcal{D}^\prime(\Omega)$.} \end{equation*}
We now apply \hyperref[derivatikds]{Theorem \ref{derivatikds}} to the compact set $\overline{W}$; it turns out that there exist $g \in C^0(\Omega)$ and $\alpha \in \N^n$ such that
\begin{equation*} f(\varphi \cdot \psi) = (-1)^{|\alpha|} \int_\Omega g(x) D^\alpha (\psi \cdot \varphi)(x) \, \mathrm{d}x \quad \text{for all $\varphi \in \ccs$}. \end{equation*}
Set
\begin{equation*}g_\beta(x) := (-1)^{|\alpha| + |\beta|} g(x)  D^{\alpha - \beta} \psi(x).\end{equation*}
Then, using the Leibniz rule, we obtain
\begin{equation*} \begin{aligned} f(\varphi) & = \sum_{\beta \leq \alpha} (-1)^{|\alpha|} \binom{\alpha}{\beta} \int_\Omega g(x)  D^{\alpha - \beta} \psi(x)  D^\beta \varphi(x) \, \mathrm{d}x =
\\[1em] & = \sum_{\beta \leq \alpha} (-1)^{|\beta|} \int_\Omega g_\beta(x) D^\beta \varphi(x) \, \mathrm{d}x = 
\\[1em] & = \sum_{b \leq \alpha} D^\beta  \Lambda_{g_\beta}(\varphi). \end{aligned} \end{equation*}
\end{proof}
 
\begin{theorem}Let $f \in \mathcal{D}^\prime(\Omega)$. For all $\alpha \in \N^n$ we can find $g_\alpha \in C^0(\Omega; \; \C)$ such that
\begin{equation*} f = \sum_\alpha D^{\alpha} \, \Lambda_{g_\alpha}. \end{equation*}
Furthermore, for all compact subset $K \subset \Omega$, only finitely many $g_\alpha$ are different from zero.\end{theorem}

\begin{proof} Let $(V_j)_{j \in \N}$ be a countable family of open sets with compact closure such that
\begin{equation*} V_j \nearrow \Omega. \end{equation*}
Let $\omega_j := V_j \setminus \overline{V_{j-2}}$. It is easy to prove that $\{ \omega_j \}_{j \in \N}$ is an open covering of $\Omega$, and hence it admits a partition of the unity
\begin{equation*}\left\{\psi_j \in \ccs \: \left| \: \mathrm{spt}(\psi_j) \subset \omega_j \right. \right\}_{j \in \N}. \end{equation*}
The function $\varphi \in \ccs$ has compact support, and thus there are only finitely many $\psi_j$ different from zero on the support of $\varphi$. Therefore,
\begin{equation*} f( \varphi ) = f \left( \sum_{j \in \N} \psi_j \cdot \varphi \right) = \sum_{j \in \N} f \left( \psi_j \cdot \varphi \right) =\sum_{j \in \N} \psi_j \cdot f(\varphi). \end{equation*}
We know that $\psi_j \cdot f$ is a distribution with compact support contained in the closure $\bar{V}_j$. Hence it follows from \hyperref[theorem:ccsod]{Theorem \ref{theorem:ccsod}} that
\begin{equation*} \psi_j \cdot f = \sum_{k = 1}^{m_j} D^{\alpha_{k, \, j}} \Lambda_{g_{k, \, j}}, \end{equation*}
which, in turn, implies that
\begin{equation*}f = \sum_{j \in \N} \sum_{k = 1}^{m_j} D^{\alpha_{k, \, j}} \Lambda_{g_{k, \, j}}. \end{equation*}  \end{proof}

\begin{theorem}Let $f \in \mathcal{D}^\prime(\Omega)$ be a positive distribution, whose support is contained in a compact subset $K \subset \Omega$. Then there exists a positive finite measure $\mu$ such that
\begin{equation*} f(\varphi) = \int \varphi(x) \, \mathrm{d}\mu(x). \end{equation*}
Moreover, if the support is not in a compact subset, then $\mu$ is $\sigma$-finite and positive.\end{theorem}

\section{Convolution}

Let $\varphi, \, \psi \in \mathcal{D}(\R^n) := \mathcal{D}$. Recall that the convolution\index{convolution} of $\varphi$ and $\psi$ is defined by
\begin{equation} \label{congfs} (\varphi \ast \psi) (x) := \int_{\R^n} \varphi(y) \psi(x - y) \, \mathrm{d}y. \end{equation}
The goal of this section is to find a well-defined notion of convolution between a distribution $f \in \mathcal{D}^\prime$ and a function $\varphi \in \mathcal{D}$. Set
\begin{equation} \label{convoluzione} \left(f \ast \varphi \right)(x) := f \left( \tau_x \widecheck{\varphi} \, \right). \end{equation}
Here $\tau_x$ denotes the translation operator\index{translation} $g(y) \longmapsto g(y - x)$ and $\widecheck{\varphi}(y)$ is equal to the function at the symmetric point\index{symmetric} $\varphi(-y)$.

\begin{definition} \index{distribution!translation}\index{distribution!symmetric} Let $f \in \D^\prime$ be a distribution. We define the translation and the symmetric of $f$ respectively as
\begin{equation*}\tau_x f(\varphi) := f \left( \tau_{-x} \varphi \right) \quad \text{and} \quad \widecheck{f}(\varphi) := f \left( \, \widecheck{\varphi} \, \right). \end{equation*} \end{definition}

\begin{theorem}[Regularity] Let $f \in \D^\prime$, $\varphi \in \D$ and let $x$ be a point of $\R^n$. \mbox{}
\begin{enumerate}[label=\textbf{(\alph*)}]
\item The translation operator $\tau_x$ acting on the product can be absorbed on any of the factors, i.e.,
\begin{equation*} \tau_x (f \ast \varphi) = (\tau_x  f) \ast \varphi = f \ast (\tau_x  \varphi). \end{equation*}
\item The convolution $f \ast \varphi$ is a function of class $C^\infty(\R^n)$. Furthermore, we have
\begin{equation*} D^\alpha( f \ast \varphi) = (D^\alpha f) \ast \varphi = f \ast (D^\alpha \varphi) \end{equation*}
for all multi-indices $\alpha \in \N^n$.
\end{enumerate}\end{theorem}

\begin{proof}\mbox{}
\begin{enumerate}[label=\textbf{(\alph*)}]
\item The thesis follows immediately by noticing that for all $y \in \R^n$ we have
\begin{equation*} \begin{aligned}&\tau_x (f \ast \varphi)(y) = \tau_x f \left( \tau_y \check{\varphi} \right) = f \left( \tau_{y - x} \check{\varphi} \right), \\[1em]
&   (\tau_x f) \ast \varphi(y) = \tau_x f \left( \tau_y \check{\varphi} \right) = f \left( \tau_{y - x} \check{\varphi} \right), \\[1em]
& f \ast (\tau_x \varphi)(y) = f \left( \tau_y \check{\tau_x  \varphi} \right) = f \left( \tau_{y - x} \check{\varphi} \right).  \end{aligned} \end{equation*}
\item First, we prove that $f \ast \varphi$ is a function of class $C^\infty$. Indeed, the incremental ratio, which defines the $j$th directional derivative of the convolution, is given by
\begin{equation*} \begin{aligned} \frac{ (f \ast \varphi)(x + h e_j) - (f \ast \varphi)(x)}{h} & = \frac{\left(f \ast (\tau_{- h  e_j} \varphi ) \right)(x) - (f \ast \varphi)(x)}{h} = \\[1em] & = \tau_x \widecheck{f}  \left( \frac{\tau_{-h  e_j}  \varphi - \varphi}{h} \right) (x). \end{aligned} \end{equation*}
The function $\varphi$ belongs to $\D$, and hence the incremental ratio converges to the $j$th derivative:
\begin{equation*} \frac{\tau_{-h e_j} \varphi - \varphi}{h} (x) = \frac{\varphi(x + h e_j) - \varphi(x)}{h} \xrightarrow{h \to 0^+} \partial_{x_j} \varphi(x). \end{equation*}
Furthermore, the convergence is uniform with respect to the variable $x$ since $\varphi$ is differentiable and the derivative is bounded. Similarly, we have that
\begin{equation*} \frac{\tau_{-h e_j} \varphi - \varphi}{h} \xrightarrow{\D} \partial_{x_j} \varphi\end{equation*}
with respect to the notion of convergence in $\D$. The distribution $\tau_x \check{f}$ is continuous, and therefore the limit must exist. It follows that $f \ast \varphi$ is differentiable in the direction $e_j$, $j = 1, \, \dots, \, n$, and the desired equality holds:
\begin{equation*} \partial_{x_j} (f \ast \varphi) = \tau_x  \check{f} \left( \partial_{x_j} \varphi \right) = f \ast (\partial_{x_j} \varphi). \end{equation*}
The same holds true for any direction and derivative of $\varphi$; thus $f \ast \varphi \in C^\infty(\R^n)$ and
\begin{equation*} D^\alpha( f \ast \varphi) = \left(f \ast (D^\alpha \varphi) \right). \end{equation*}
In conclusion, we notice that
\begin{equation*} \begin{aligned} \left( \partial_{x_j} f \right) \ast \varphi (x) & = \partial_{x_j} f  \left(\tau_x \check{\varphi} \, \right) =
\\[1em] & = - f \left( \partial_{x_j} \tau_x \check{\varphi} \right) =
\\[1em] & = - f \left( \tau_x  \partial_{x_j} \check{\varphi} \right) =
\\[1em] & = \left( f \ast (\partial_{x_j} \varphi) \right) (x). \end{aligned} \end{equation*}
We can easily generalize this chain of equalities to a derivative of order $\alpha \in \N^n$, which is exactly what we needed to conclude.
\end{enumerate}\end{proof}

\begin{theorem}[Associativity] \index{distribution!associative convolution} \label{associative1}Let $f \in \D^\prime$, and let $\varphi, \, \psi \in \D$. The convolution is associative, i.e.,
\begin{equation*} (f \ast \varphi) \ast \psi = f \ast (\varphi \ast \psi). \end{equation*}\end{theorem}

\begin{proof}The proof is rather involved. To ease the notations we will divide it into four steps.

\paragraph{Step 1.} Let $z \in \R^n$ be a point. Then
\begin{equation*} \begin{aligned} \left( (f \ast \varphi) \ast \psi \right)(z) & = \int_{\R^n} (f \ast \varphi)(x) \psi(z-x) \, \mathrm{d}x =
\\[1em] & = \int_{\R^n} f (\tau_x  \check{\varphi} \, \psi(z-x) \, \mathrm{d}x =
\\[1em] & =  \int_{\R^n} f \left(\tau_x \check{\varphi} \psi(z-x) \right) \, \mathrm{d}x =
\\[1em] & =  \int_{\R^n} f \left(y \mapsto \psi(z - x) \varphi(x - y) \right) \, \mathrm{d}x  =
\\[1em] & = \int_{\R^n} f \left( \Phi(x) \right) \, \mathrm{d}x, \end{aligned}\end{equation*}
where $\Phi : \R^n \longrightarrow \D$ is defined by
\begin{equation*} \Phi : x \longmapsto \Phi_{x} \leadsto \Phi_x(y) := \psi(z - x) \varphi(x - y). \end{equation*}
On the other hand, we have
\begin{equation*} \begin{aligned} \left( f \ast(\varphi \ast \psi) \right)(z) & = f \left( \tau_z (\varphi \, \check{\ast} \, \psi) \right) =
\\[1em] & = f \left( y \mapsto (\varphi \ast \psi)(z - y) \right) =
\\[1em] & =   f \left( y \mapsto \int \varphi(x)  \psi(z - y - x) \, \mathrm{d}x \right) =
\\[1em] & =  f \left( y \mapsto \int \varphi(x - y) \psi(z - x) \, \mathrm{d}x \right) =
\\[1em] & = f \left( y \mapsto \int_{\R^n} \Phi(x)(y) \, \mathrm{d}x \right). \end{aligned}\end{equation*}

\paragraph{Step 2.} First, we notice that $\Phi(x) := \Phi_x$ is the zero function whenever $z - x \notin \mathrm{spt}\, \psi$, that is, when
\begin{equation*}x \notin z - \mathrm{spt} \, \psi. \end{equation*}
Consequently, we have the inclusion
\begin{equation*} \mathrm{spt}\, \Phi \subset J := z - \mathrm{spt} \, \psi. \end{equation*}
Furthermore, a point $x \in J$ satisfies $\Phi_x(y) = 0$ if and only if $x - y \notin \mathrm{spt} \, \varphi$, which is equivalent to requiring that
\begin{equation*} y \notin x - \mathrm{spt}\, \varphi. \end{equation*}
We finally infer that
\begin{equation*} \mathrm{spt}\, \Phi_x \subset x - \mathrm{spt}(\varphi) \subset K := J - \mathrm{spt}(\varphi). \end{equation*}

\paragraph{Step 3.} The function $\Phi_x$ belongs to $\D_K$ and the following composition is well-defined:
\begin{equation*} J \xrightarrow{\Phi} \D_K \xrightarrow{f} \C. \end{equation*}
We now want to compute the integral $\int_{\R^n} f \left( \Phi(x) \right) \, \mathrm{d}x$. The function $f \circ \Phi : J \longrightarrow \C$ can be uniformly approximated by a sequence of functions of the following kind:
\begin{equation*} \sum_{i = 1}^{N} f \circ \Phi(x_i^k) \cdot \chi_{Q_i^k}. \end{equation*}
Here we require $Q_i^k$ to be the collection of cubes centered at $x_i^k$ and with edges of length $\frac{1}{k}$ such that the disjoint union covers the set $J$, that is,
\begin{equation*} J \subseteq \bigcup_{i = 1}^N Q_I^k. \end{equation*}
It follows that
\begin{equation*}\int_{\R^n} f \left( \Phi(x) \right) \, \mathrm{d}x = \lim_{k \to + \infty} \sum_{i = 1}^{N} f \circ \Phi(x_i^k)  \left| {Q_i^k} \right| = \lim_{k \to + \infty} f(S_k), \end{equation*}
where
\begin{equation*}S_k := \sum_{i = 1}^{N} \Phi(x_i^k)  \left| {Q_i^k} \right| = \frac{1}{k^n}  \sum_{i = 1}^{N} \Phi(x_i^k) \in \D_K. \end{equation*}

\paragraph{Step 4.} We now claim that $S_k \to \int_{\R^n} \Phi(x) \, \mathrm{d}x$ uniformly on $K$ for $k \to + \infty$. Indeed, we have
\begin{equation*} \begin{aligned} \left| S_k(y) - \int_{\R^n} \Phi(x) (y) \, \mathrm{d}x \right| & \leq \sum_{i  = 1}^{N} \int_{Q_i^k} \left| \Phi(x_i^k)(y) - \Phi(x)(y) \right| \, \mathrm{d}x \leq
\\[1em] & \leq \sum_{i = 1}^{N} \int_{Q_i} \left| \psi(z - x_i^k) \varphi(x_i^k - y) - \psi(z - x) \varphi(x - y) \right| \, \mathrm{d}x \leq
\\[1em] & \lesssim \frac{\| \varphi \|_{L^1(Q)}  \| \psi \|_{L^1(Q)}}{k^n}. \end{aligned} \end{equation*}
It follows immediately that $\int_{\R^n} \Phi(x) \, \mathrm{d}x$ is a continuous function for it is the uniform limit of continuous functions. In a similar fashion, we observe that
\begin{equation*} D^\alpha S_k = \frac{1}{k^n} \sum_{i = 1}^{N} D^\alpha \Phi(x_i^k), \end{equation*}
which, in turn, implies that 
\begin{equation*} \|D^\alpha S_k \|_0 \leq |J| \cdot \| \Phi \|_{|\alpha|}. \end{equation*}
Consequently, the derivatives of the function $S_k$ are equibounded. \caution{Note that the equiboundedness of the second-order derivatives is enough to infer the equicontinuity of the first-order derivatives (and so on..) }Thus, up to subsequences, we find from the Ascoli-Arzelà theorem that
\begin{equation*} D^\alpha S_k \xrightarrow{\text{uniformly on $K$}} D^\alpha \int_{\R^n} \Phi(x) \, \mathrm{d}x \end{equation*}
for all multi-indices $\alpha \in \N^n$. This is equivalent to the notion of convergence in $\D_K$, and, by continuity of $f$, we find that
\begin{equation*} f(S_k) \to f \left( \int_{\R^n} \Phi(x) \, \mathrm{d}x \right) = f \left( y \mapsto \int_{\R^n} \Phi(x)(y) \, \mathrm{d}x \right),\end{equation*}
which is exactly what we wanted to prove.\end{proof}
 
The next goal is a proof of the fact that the associative property of the convolution is true under more general assumptions, (e.g., at least two out of the three elements have compact supports.)

\begin{remark} Recall that there is a bijective correspondence between compactly supported distributions and distributions defined on $C^\infty$ function, i.e., 
\begin{equation*} \text{$f \in \mathcal{D}^\prime$ such that $\mathrm{spt} \, f$ is compact} \longleftrightarrow \text{$\tilde{f} \in \mathcal{E}^\prime$}. \end{equation*} \end{remark}

\begin{lemma}\label{lemma:spt} Let $f \in \D^\prime$, and let $\varphi \in \D$. The support of the convolution is contained in the sum of the supports, that is,
\begin{equation*} \mathrm{spt} \, f \ast \varphi \subseteq  \mathrm{spt} \, f +  \mathrm{spt}\, \varphi. \end{equation*}
\end{lemma}

\begin{proof} A point $x$ belongs to the support of the convolution if
\begin{equation*} (f \ast \varphi) (x) \neq 0.\end{equation*}
Employing the definition of convolution, this is equivalent to requiring that $x$ satisfies
\begin{equation*} f \left(\tau_x \, \check{\varphi} \right) \neq 0, \end{equation*}
which, in turn, is equivalent to the fact that the intersection of the supports is not disjoint, that is,
\begin{equation*} \mathrm{spt} \, f \cap \left( x + \mathrm{spt} \, \check{\varphi} \right) \neq \varnothing. \end{equation*}
The symmetric $\check{\varphi}$ has support exactly equal to $- \mathrm{spt} \, \varphi$, and therefore
\begin{equation*} \mathrm{spt} \, f \cap \left( x - \mathrm{spt} \, \varphi \right) \neq \varnothing \implies x \in \mathrm{spt} \, f +  \mathrm{spt} \, \varphi, \end{equation*}
and this concludes the proof.
\end{proof}

\begin{theorem}\index{distribution!associative convolution} Let $f \in \mathcal{E}^\prime$ be a compactly supported distribution, and let $\varphi \in \D$ and $\psi \in \mathcal{E}$. Then the convolution is associative, that is,
\begin{equation}\label{th:122222} \left(f \ast \psi\right) \ast \varphi = f \ast \left(\psi \ast \varphi\right) = \left(f \ast \varphi \right) \ast \psi. \end{equation}
\end{theorem}

\begin{proof}As an immediate consequence of \hyperref[lemma:spt]{Lemma \ref{lemma:spt}} we find that the supports of the three convolutions considered in the statement are all contained in the sum
\begin{equation*} \mathrm{spt} \, f+  \mathrm{spt} \, \varphi + \mathrm{spt}\, \psi. \end{equation*}
The idea is to approximate $\psi$ with a compactly supported function $\psi_C \in \D$ which is sufficiently near to $\psi$ in a specific topology. Note that, if $\psi_C \in \D$, then \hyperref[associative1]{Theorem \ref{associative1}} implies the identity \eqref{th:122222}, i.e.,
\begin{equation*} \left(f \ast \psi_C\right) \ast \varphi = f \ast \left(\psi_C \ast \varphi\right) = \left(f \ast \varphi \right) \ast \psi_C. \end{equation*}
Therefore, it is more than enough to prove that there is a suitable choice of $\psi_C$ such that
\begin{equation*}\begin{gathered} \left(f \ast \psi_C\right) \ast \varphi = \left(f \ast \psi\right) \ast \varphi, \qquad f \ast \left(\psi_C \ast \varphi\right) = f \ast \left(\psi \ast \varphi\right) \\[1em] \left(f \ast \varphi \right) \ast \psi_C = \left(f \ast \varphi \right) \ast \psi. \end{gathered} \end{equation*}
We now show that $\psi_C$ can be chosen in such a way that the first identity holds true. The other two identities are similar, and thus they are left to the reader as an exercise. We want to prove that
\begin{equation*} \left[f \ast \left( \psi_C - \psi \right) \right] \ast \varphi (z) = 0 \end{equation*}
for all $z \in \R^n$. Note that this is equivalent to requiring that
\begin{equation*} z \notin \mathrm{spt} \, f +  \mathrm{spt}\, \varphi + \mathrm{spt}(\psi_C - \psi) \iff \mathrm{spt}(\psi_C - \psi) \cap \left( z - \mathrm{spt} \, f - \mathrm{spt}\, \varphi \right) = \varnothing. \end{equation*}
The supports of $f$ and $\varphi$ are compact; thus the translation of the sum,
\begin{equation*} z - \mathrm{spt} \, f - \mathrm{spt}\, \varphi, \end{equation*}
is a compact set, which we will denote by $K_z$. Consequently, we can define $\psi_C$ by setting
\begin{equation*}\psi_C(x) := \begin{cases} \psi(x) & \text{for all $x \in B(0, \, R),$} \\[0.8em] \eta(x) & \text{otherwise}, \end{cases} \end{equation*}
where $B(0 \, R)$ is a closed ball containing $K_z$ and $\eta(x)$ is a cut-off (smooth) function. In particular, it is easy to see that the following inclusions hold:
\begin{equation*} \mathrm{spt}(\psi_C - \psi) \subset B_R^c \quad \text{and} \quad B_R^c \cap \left( z - \mathrm{spt} \, f - \mathrm{spt}\, \varphi \right) = \varnothing. \end{equation*}
Finally, the arbitrariness of the point $z$ allows us to conclude that the equality at $z$ is actually an equality between functions.
\end{proof}

\begin{theorem} \label{th:skeoweo} Let $f, \, g \in \D^\prime$ be two distributions, and let $\varphi \in \mathcal{E}$ be a smooth function. Then the convolution is associative, that is,
\begin{equation*} f \ast \left( g \ast \varphi \right) = g \ast \left ( f \ast \varphi \right), \end{equation*}
provided that one of the following conditions holds true: \mbox{}
\begin{enumerate}[label=\textbf{(\alph*)}]
\item Both $f$ and $g$ have compact support ($f, \,g \in \mathcal{E}^\prime$).
\item The distribution $f$ has compact support ($f \in \mathcal{E}^\prime$) and the function $\varphi$ belongs to $\mathcal{D}$.
\end{enumerate}
\end{theorem}

The proof of this theorem relies on the following technical result, which characterises the equality of two distributions in terms of the convolution and, surprisingly, also in terms of the pointwise values.

\begin{lemma}\label{lemma:chars}Let $f, \, g \in \D^\prime$ be distributions. The following properties are equivalent:
\begin{enumerate}[label=\textbf{(\arabic*)}]
\item The distributions are equal, that is, $f = g$.
\item For all $\varphi \in \D$ it turns out that $f \ast \varphi = g \ast \varphi$ as functions.
\item For all $\varphi \in \D$ it turns out that $f \ast \varphi(0) = g \ast \varphi(0)$.
\end{enumerate}
\end{lemma}

\begin{proof}The unique nontrivial implication is the last one. Fix $\varphi \in D$. Then
\begin{equation*}\begin{aligned} f(\varphi) & = f \left( \tau_0 \, \check{\check{\varphi}} \right) = \left( f \ast \check{\varphi} \right)(0) = \\[1em] & = \left( g \ast \check{\varphi} \right)(0) = g \left( \tau_0 \, \check{\check{\varphi}} \right)  = g(\varphi).  \end{aligned} \end{equation*} \end{proof}

\begin{proof}[Proof of the Theorem \ref{th:skeoweo}] We only sketch the proof under the assumption $\mathbf{(b)}$; the other one is left to the reader as an exercise.

\paragraph{Proof (b).} Let $\psi \in \D$ be a test function. By \hyperref[lemma:chars]{Lemma \ref{lemma:chars}} it is enough to check that
\begin{equation*} \left[ f \ast \left(g \ast \varphi\right) \right] \ast \psi = \left[ g \ast \left(f \ast \varphi\right) \right] \ast \psi. \end{equation*}
We may apply the associative results we have proved so far to both members and obtain the sought equality. Indeed, it turns out that
\begin{equation*} \left[ f \ast \left(g \ast \varphi\right) \right] \ast \psi = f \ast \left[ \left(g \ast \varphi\right) \ast \psi \right] = f \ast \left[ \psi \ast \left(g \ast \varphi\right) \right] = \left( f \ast \psi \right) \ast \left( g \ast \varphi \right),\end{equation*}
and also that
\begin{equation*} \left[ g \ast \left(f \ast \varphi\right) \right] \ast \psi =  g \ast \left[ \left(f \ast \varphi\right) \ast \psi \right] = g \ast \left[ \left(f \ast \psi\right) \ast \varphi \right]  =  \left( f \ast \psi \right) \ast \left( g \ast \varphi \right).\end{equation*}
\end{proof}

\section{Convolution Between Distributions}

The goal of this section is to generalise the notion of convolution to two distributions. The leading idea is that the resulting object should also be a distribution, and therefore we need to characterise its value on a generic test function.

\begin{theorem}\label{def:op} Let $f \in \mathcal{D}^\prime$ be a distribution. The operator defined by
\begin{equation*} f_\ast : \D \to \mathcal{E}, \qquad \varphi \longmapsto f \ast \varphi \end{equation*}
is linear, continuous and translation-invariant. Furthermore, if $L \in \mathcal{L}(\mathcal{D}, \, \mathcal{E})$ is a linear continuous operator which commutes with the translations, then
\begin{equation*} \text{there exists a unique $f \in \D^\prime$ such that $L = f_\ast$.} \end{equation*} \end{theorem}

\begin{proof} The operator $f_\ast$ is clearly linear and translation-invariant; hence we only need to show that it is continuous with respect to the $\D$-topology.

\paragraph{Step 1.} The characterisation of the continuity with respect to the $\D$-topology (see \hyperref[chartsdjsdl]{Theorem \ref{chartsdjsdl}}) asserts that $f_\ast$ is continuous if and only if the restriction $f \, \big|_{\D_K}$ is continuous for all compact subsets $K \subset \R^n$. This is equivalent (e.g., closed graph theorem) to showing that
\begin{equation*} \begin{cases} \varphi_k \xrightarrow{\D_K} \varphi \\[1em] f_\ast \left( \varphi_k \right) \xrightarrow{k \to + \infty} \psi \end{cases} \stackrel{?}{\implies}\: f_\ast (\varphi) = \psi. \end{equation*}
Now notice that the second assumption is equivalent to
\begin{equation*} \psi(x) = \lim_{k \to + \infty} f \ast \varphi_k(x) = \lim_{k \to + \infty} f \left( \tau_x \check{\varphi}_k \right). \end{equation*}
The continuity of $f$, together with the fact that $\tau_x \varphi_k \to \tau_x \varphi$ and $\check{\varphi}_k \to \check{\varphi}$, is enough to infer that the limit above is exactly equal to $f_\ast(\varphi)(x)$.

\paragraph{Step 2.} Let $L \in \mathcal{L}(\mathcal{D}, \, \mathcal{E})$ be a translation-invariant operator. If the distribution $f$ exists, then it is uniquely determined by the formula
\begin{equation} \label{defs} f(\varphi) = L(\check{\varphi})(0), \end{equation}
as a consequence of the well-known identity
\begin{equation*}f(\varphi) = f \ast \check{\varphi}(0). \end{equation*}
Suppose now that $f$ is defined by \eqref{defs}. The convolution operator is clearly linear and continuous, and therefore we only need to check that $f_\ast = L$. Since $L$ is translation-invariant, we find that
\begin{equation*} \begin{aligned} f_\ast(\varphi)(x) & = f \ast \varphi(x) =
\\[1em] & = f \left(\tau_x \check{\varphi} \right) =
\\[1em] & = L \left(\tau_{-x} \varphi \right) =
\\[1em] & = \tau_{-x}  L \varphi(0) = L \varphi(x), \end{aligned} \end{equation*}
which is exactly what we wanted to prove.
\end{proof}

\begin{definition}[Convolution]\index{distribution!convolution (of distributions)} Let $f \in \mathcal{E}^\prime$ and let $g \in \D^\prime$ be two distributions. The convolution of $f$ and $g$ is the unique distribution satisfying
\begin{equation*} \left(f \ast g \right)(\varphi) := f \left( \check{g} \ast \varphi \right) = f \circ \check{g}_\ast(\varphi). \end{equation*} \end{definition}

\begin{lemma}\label{lemma:spt2} Let $f \in \mathcal{E}^\prime$ and let $g \in \D^\prime$. The support of the convolution is contained in the sum of the supports, that is,
\begin{equation*} \mathrm{spt} \, f \ast g \subseteq  \mathrm{spt} \, f +  \mathrm{spt} \, g. \end{equation*}
\end{lemma}

\begin{proof} A point $\varphi$ belongs to the support of the convolution if
\begin{equation*} (f \ast g) (\varphi) \neq 0. \end{equation*}
Employing the definition of convolution, this is equivalent to requiring that $\varphi$ satisfies
\begin{equation*}f \left( \check{g}_\ast(\varphi) \right) \neq 0, \end{equation*}
and this is clearly equivalent to requiring that the intersection of the supports is not disjoint, that is,
\begin{equation*} \mathrm{spt} \, f \cap \mathrm{spt} \, \check{g}_\ast(\varphi)\, \neq \varnothing. \end{equation*}
The symmetric $\check{g}$ has support exactly equal to $- \mathrm{spt} \, g$, and therefore
\begin{equation*} \mathrm{spt} \, f \cap \left[ \mathrm{spt}\, \varphi - \mathrm{spt}\, g \right] \neq \varnothing \implies  \left( \mathrm{spt}\, f + \mathrm{spt} \, g \right) \cap \mathrm{spt} \,\varphi \neq \varnothing, \end{equation*}
and this concludes the proof.
\end{proof}

\begin{theorem} Let $f, \, g \in \mathcal{E}^\prime$, and let $h \in \D^\prime$. Then the convolution is associative, that is,
\begin{equation*} \left( f \ast g \right) \ast h = f \ast \left( g \ast h \right). \end{equation*}
\end{theorem}

\begin{proof}By \hyperref[lemma:chars]{Lemma \ref{lemma:chars}} it is enough to check that
\begin{equation*} \left( f \ast g \right) \ast h \ast \varphi = f \ast \left( g \ast h \right) \ast \varphi\end{equation*}
for all $\varphi \in \D$. By definition we have that
\begin{equation*} \left( f \ast g \right) \ast h \ast \varphi = (f \ast g) \ast (h \ast \varphi) = f \ast \left( g \ast (h \ast \varphi) \right), \end{equation*}
and, similarly, that
\begin{equation*} f \ast \left( g \ast h \right) \ast \varphi = f \ast \left( (g \ast h) \ast \varphi \right) = f \ast \left(g \ast (h \ast \varphi) \right). \end{equation*}
These expressions are identical, and thus the criterion given by the result mentioned above concludes the proof.
\end{proof}

\begin{proposition} \label{prop:423}Let $\varphi \in \D$, and let $f \in \D^\prime$. Then
\begin{equation*} \delta_0 \ast \varphi = \varphi \qquad \text{and} \qquad \delta_0 \ast f = f. \end{equation*}\end{proposition}

\begin{proof}The first identity is obvious:
\begin{equation*}\delta_0 \ast \varphi(x) = \delta_0 \left(\tau_x \check{\varphi} \right) = \tau_x \check{\varphi}(0) = \varphi(x). \end{equation*}
The second identity, on the other hand, follows immediately from the first one applying \hyperref[lemma:chars]{Lemma \ref{lemma:chars}}. Indeed, for any $\psi \in \D$ we have
\begin{equation*} \delta_0 \ast f \ast \psi = f \ast (\delta_0 \ast \psi) = f \ast \psi. \end{equation*}\end{proof}

\begin{proposition} Let $f \in \D^\prime$, and let $\alpha \in \N^n$ be any multi-index. Then
\begin{equation*} D^\alpha f = \left(D^\alpha \delta_0 \right) \ast f. \end{equation*}\end{proposition}

\begin{proof}Fix $\psi \in \D$. A straightforward application of \hyperref[prop:423]{Proposition \ref{prop:423}} shows that
\begin{equation*} \left(D^\alpha f \right) \ast \psi = \left( D^\alpha f \right) \ast \left(\delta_0 \ast \psi \right) = f \ast D^\alpha \delta_0 \ast \psi = D^\alpha \delta_0 \ast f \ast \psi, \end{equation*}
and this is enough to conclude using \hyperref[lemma:chars]{Lemma \ref{lemma:chars}} and the arbitrariness of $\psi$.\end{proof}

\begin{proposition} Let $f \in \D^\prime$, and let $g \in \mathcal{E}^\prime$. Then
\begin{equation*} D^\alpha \left(f \ast g \right) = \left(D^\alpha f \right) \ast g = f \ast \left(D^\alpha g \right). \end{equation*}\end{proposition}

\begin{proof}First, we notice that, if $g = \delta_0$, then
\begin{equation*} \left(D^\alpha f \right) \ast \varphi = f \ast \left(D^\alpha \varphi \right) = f \ast D^\alpha \left(\delta_0 \ast \varphi \right) = f \ast D^\alpha \delta_0 \ast \varphi, \end{equation*}
from which we infer that
\begin{equation*} D^\alpha  f = \left(D^\alpha \delta_0 \right) \ast f. \end{equation*}
It follows that, in the general case, we have
\begin{equation*} D^\alpha \left(f \ast g \right) = \left( D^\alpha \delta_0 \right) \ast f \ast g = \left(D^\alpha f \right) \ast g,\end{equation*}
which is exactly what we wanted to prove.\end{proof}

\section{Regularization by Mollification}

The goal of this brief section is to use the \textit{mollification}\index{mollification}\index{mollifier} method to regularize a function. Let $\rho \in \D$ be a function with compact support
\begin{equation*} \mathrm{spt} \, \rho \subset B(0, \, 1), \end{equation*}
and with unitary mass, that is,
\begin{equation*}\int \rho(x) \, \mathrm{d}x = 1. \end{equation*}
We can define, for all $\epsilon > 0$, the \textit{mollifier} as
\begin{equation*} \rho_\epsilon(x) := \frac{1}{\epsilon^n} \rho \left( \frac{x}{\epsilon} \right). \end{equation*}
Clearly, the function $\rho_\epsilon$ has supported contained in the ball $B(0, \, \epsilon)$ and unitary mass.

\begin{theorem}[Regularization] Let $\varphi \in \D$ and let $f \in \D^\prime$. Then
\begin{equation*} \varphi \ast \rho_\epsilon \xrightarrow{\epsilon \to 0^+} \varphi \qquad \text{and} \qquad f \ast \rho_\epsilon \xrightarrow{\epsilon \to 0^+} f. \end{equation*}\end{theorem}

\begin{proof} We first prove that the convolution product between $\varphi$ and the mollifiers $\rho_\epsilon$ converges uniformly to $\varphi$ as $\epsilon \to 0^+$. Fix $x \in \R^n$. Then
\begin{equation*} \begin{aligned} \left| \varphi \ast \rho_\epsilon(x) - \varphi(x) \right| & = \left| \int_{\R^n} \left(\varphi(x) - \varphi(y)\right) \rho_\epsilon(x - y) \, \mathrm{d}y \right| \leq
\\[1em] & \leq \int_{\R^n} \left| \varphi(x) - \varphi(y) \right| \rho_\epsilon (x - y) \, \mathrm{d}y =
\\[1em] & = \int_{B_\epsilon}  \left| \varphi(x) - \varphi(y) \right| \rho_\epsilon (x - y) \, \mathrm{d}y \leq \epsilon \cdot \| \varphi \|_1, \end{aligned} \end{equation*}
and therefore $\varphi \ast \rho_\epsilon$ converges to $\varphi$ uniformly as $\epsilon \to 0^+$. In a similar fashion, we can prove the same estimate for
\begin{equation*} D^\alpha \left( \varphi \ast \rho_\epsilon \right) = D^\alpha \, \varphi \ast \rho_\epsilon, \end{equation*}
which, in turn, converges uniformly to $D^\alpha \varphi$ for all multi-indices $\alpha \in \N^n$. By \hyperref[chartsdjsdl]{Theorem \ref{chartsdjsdl}} the uniform convergence of all derivatives is equivalent to the convergence in $\D$, which means that
\begin{equation*} \varphi \ast \rho_\epsilon \xrightarrow{\D} \varphi. \end{equation*}
The second assertion now follows from the characterization \hyperref[lemma:chars]{Lemma \ref{lemma:chars}}. Fix $\varphi \in \D$. Then
\begin{equation*} f \ast \rho_\epsilon (\varphi) = f \ast \rho_\epsilon \ast \check{\varphi}(0), \end{equation*}
and $\check{\rho}_\epsilon \ast \varphi \to \varphi$ in $\D$ as a consequence of the first assertion. The continuity of $f$ is now enough to infer the thesis.\end{proof}